%%############################################################################%%
%%                                                                            %%
%% Datei:  ASBA-Vorwort.tex                                                   %%
%% Inhalt: Kapitel "Vorwort" und "Vereinbarungen"                             %%
%%                                                                            %%
%% Copyright (C) 2017  Winfried Teschers                                      %%
%%                                                                            %%
%% This program is free software: you can redistribute it and/or modify       %%
%% it under the terms of the GNU Affero General Public License as published   %%
%% by the Free Software Foundation, either version 3 of the License, or       %%
%% (at your option) any later version.                                        %%
%%                                                                            %%
%% This program is distributed in the hope that it will be useful,            %%
%% but WITHOUT ANY WARRANTY; without even the implied warranty of             %%
%% MERCHANTABILITY or FITNESS FOR A PARTICULAR PURPOSE.  See the              %%
%% GNU Affero General Public License for more details.                        %%
%%                                                                            %%
%% You should have received a copy of the GNU Affero General Public License   %%
%% along with this program.  If not, see <http://www.gnu.org/licenses/>.      %%
%%                                                                            %%
%% Dr. Winfried Teschers                                                      %%
%% Anton-Günther-Straße 26c                                                   %%
%% 91083 Baiersdorf                                                           %%
%% Germany                                                                    %%
%%                                                                            %%
%% e-mail: winfried.teschers@t-online.de                                      %%
%%                                                                            %%
%%############################################################################%%

% !TeX root = ASBA.tex
% !TeX encoding = UTF-8
% !TeX spellcheck = de_DE

%\chapter                     {Vorwort}% #######################################
\phantomsection% sichert korrekten Link im Inhaltsverzeichnis
\label                    {cha:Vorwort}
~\vskip 1.6cm
\likeChapterFt                {Vorwort}
\vskip 0.8cm
\beginchapter[]               {Vorwort}
\addcontentsline{toc}{chapter}{Vorwort}% Eintrag ins Inhaltsverzeichnis

Schon während meiner aktiven Zeit habe ich davon geträumt, ein Programm zu erstellen, mit dem man mathematische Sätze und Beweise speichern und überprüfen kann.
Es sollte auch statistische Auswertungen beherrschen und \textua\ Fragen beantworten können wie \textzB\
"`Welche Axiome sind zum Beweis eines bestimmten Satzes erforderlich?"' oder
"`Wie viele Beweisschritte erfordert ein bestimmter Beweis?"'.
Ein Beweis mit weniger Axiomen und weniger Beweisschritten wäre dann vorzuziehen.

Einige Jahre nach meiner Pensionierung habe ich Ende 2016 endlich damit angefangen, das Projekt ASBA zu starten.
Im Internet habe ich das Projekt "`Hilbert II"' \cite{bib:HilbertII} gefunden, dass eine ähnliche Zielsetzung hat.
Ich habe dann mit dem Projektleiter Michael Meyling Kontakt aufgenommen und war zuversichtlich, Synergien nutzen zu können.
Leider hat sich dann herausgestellt, dass mein Ansatz viel umfangreicher und somit mit "`Hilbert II"' wohl nicht kompatibel ist.
Daher betreibe ich ASBA als ein Ein-Mann-Projekt und dies wird bis zur Fertigstellung der ersten Version dieses Dokuments wohl so bleiben müssen.
Vielleicht ergibt sich dann ja eine Zusammenarbeit mit anderen Enthusiasten.

Da \hier\ viele mathematische Formeln vorkommen und ASBA auch \LaTeX-Code generieren soll, ist es in \LaTeX\ verfasst.
Dieses für mich neue Textsystem war eine große, spannende Herausforderung und ist einer der Gründe für die lange Dauer der Erstellung dieses Dokuments.
Hinzu kommt, dass ich keinen Termindruck habe und endlich mal 100\% versuchen kann -- in meinem Job wurde ich daran aus verständlichen Gründen gehindert.

ASBA soll eine Basis für die Überprüfung und Archivierung mathematischer Sätze und Beweise sein.
Daher halte ich es für unerlässlich, alle verwendeten Begriffe und Bezeichnungen (\textdh\ Benennungen und Symbole) eindeutig genug zu definieren (100\%!).
Natürlich will ich mich dabei an die übliche Nomenklatur halten.
Aber was ist üblich?
Steht \MtsSubset\ für "`Teilmenge"' oder "`echteTeilmenge"'?
Ist $0$ ein Element aus \MtsIN\ oder nicht?
Daher habe ich versucht, alle wichtigen, verwendeten Bezeichnungen der Mathematik, mit dem Schwerpunkt Logik, aber auch der formalen Metasprache streng zu definieren, normalerweise im Text, teilweise aber nur in einer Fußnote, auf jeden Fall aber im Glossar.
Dort sind auch manche Bezeichnungen aufgeführt, die im Text nicht definiert wurden.

\bigskip

Baiersdorf, den 07. Dezember 2018

Winfried Teschers

\Endchapter

\newpage

%\chapter                     {Vereinbarungen}% ################################
\phantomsection% sichert korrekten Link im Inhaltsverzeichnis
\label                    {cha:Vereinbarungen}
~\vskip 1.6cm
\likeChapterFt                {Vereinbarungen}
\vskip 0.8cm
\beginchapter[]               {Vereinbarungen}
\addcontentsline{toc}{chapter}{Vereinbarungen}% Eintrag ins Inhaltsverzeichnis

\Hier\ werden verschiedene Textauszeichnungen mit folgenden Bedeutungen verwendet:
\begin{itemize}

	\item In mathematischen Formeln:
	\begin{itemize}
		\item $\Varft      {Variable\ allgemein}$; normalerweise ein Buchstabe.
		\item $\varft          {Variablensymbol}$; normalerweise ein Kleinbuchstabe.
		\item $\Conft                {Konstante}$; normalerweise ein Wort.
		\item $\Idxft        {Konstanter\ Index}$; normalerweise ein Buchstabe.
		\item $\Setft  {VORGEGEBENE\ \ BEREICHE}$; normalerweise ein Großbuchstabe.%
			\footnote{Kleinbuchstaben gibt es in dieser Schriftart nicht.}
		\item $\Elmft          {Element\ daraus}$; normalerweise ein Großbuchstabe.
		\item $\sOpft        {Bereichsoperation}$; normalerweise ein Wort.
		\item $\Drvft{Bereich\ von\ Ableitungen}$; normalerweise ein Großbuchstabe.
		\item $\drvft          {Element\ daraus}$; normalerweise ein Buchstabe.
		\item $\Preft               {Pr\"adikat}$; normalerweise ein Wort.
	\end{itemize}

	\item In Zitaten aus \Wikipedia:
	\begin{itemize}
		\item \likeWikiFt  {Wie im Original.}
		\item \wikiBoldFt  {Wie im Original.}
		\item \wikiItalicFt{Wie im Original.}
		\item \wikiLinkFt  {Wie im Original, aber ohne Link.}
	\end{itemize}

	\item In sonstigem Text (ohne Überschriften):
	\begin{itemize}
		\item \likeLinkFt{Interner Link.}; auch in Überschriften.
			Die Farbe kann mit anderen Textauszeichnungen kombiniert werden.
		\item \likeBibFt{Nummer als Link ins Literaturverzeichnis.}
		\item \CharFt   {Zeichen [in Zeichenketten].}
		\item \DefFt    {Definition.}
		\item \OptFt    {Optionale  Teile von Sprechweisen.}
		\item \ManFt    {Notwendige Teile von Sprechweisen.}
		\item \GloFt    {Erstmalige Selbstreferenz (ohne Link).}
		\item \gloFt               {Selbstreferenz (ohne Link).}
		\item \likePreFt{Prädikat.}
		\iftestFlg
			\item
			\begin{offen}
				Teile, deren Bearbeitung zurückgestellt ist.
			\end{offen}
		\else\fi
	\end{itemize}

\end{itemize}

Fußnoten dienen nur zu weiteren Erläuterungen sowie Verweisen in dieses Dokument und die Literatur.
Daher können sie auch etwas "`lascher"' formuliert sein.
Für das Verständnis des Textes sollten sie nicht nötig sein, es reichen Grundkenntnisse der Mathematik.

\Endchapter
