%%############################################################################%%
%%                                                                            %%
%% Datei:  ASBA-Vereinbarungen.tex                                            %%
%% Inhalt: Kapitel "Vereinbarungen"                                           %%
%%                                                                            %%
%% Copyright (C) 2017  Winfried Teschers                                      %%
%%                                                                            %%
%% This program is free software: you can redistribute it and/or modify       %%
%% it under the terms of the GNU Affero General Public License as published   %%
%% by the Free Software Foundation, either version 3 of the License, or       %%
%% (at your option) any later version.                                        %%
%%                                                                            %%
%% This program is distributed in the hope that it will be useful,            %%
%% but WITHOUT ANY WARRANTY; without even the implied warranty of             %%
%% MERCHANTABILITY or FITNESS FOR A PARTICULAR PURPOSE.  See the              %%
%% GNU Affero General Public License for more details.                        %%
%%                                                                            %%
%% You should have received a copy of the GNU Affero General Public License   %%
%% along with this program.  If not, see <http://www.gnu.org/licenses/>.      %%
%%                                                                            %%
%% Dr. Winfried Teschers                                                      %%
%% Anton-Günther-Straße 26c                                                   %%
%% 91083 Baiersdorf                                                           %%
%% Germany                                                                    %%
%%                                                                            %%
%% e-mail: winfried.teschers@t-online.de                                      %%
%%                                                                            %%
%%############################################################################%%

% !TeX root = ASBA.tex
% !TeX encoding = UTF-8
% !TeX spellcheck = de_DE

\chapter     {Vereinbarungen}% #################################################
\beginchapter{Vereinbarungen}
\label   {cha:Vereinbarungen}

\footnoteForNotDefinedItem

\section     {Begriffe}% =======================================================
\beginsection{Begriffe}
\label   {sec:Begriffe}

Speziell \vrefincha{cha:Grundlagen} wollen wir mit möglichst exakt definierten \defGlo{\Begriffen} --- \hier\ normalerweise \Objekte\ --- und den zugehörigen einheitlichen \defGlo{\Bezeichnungen} (\textdh\ \defGlo{\Benennungen} und \Symbolketten) arbeiten.

\begin{table}[h]
	\begin{center}
		\begin{tabular}{|l|c|c|}
			\hline
			Die Sache an sich:  & \multicolumn{2}{c|}{\Begriff}     \\
			\hline
			Darstellung:        & \multicolumn{2}{c|}{\Bezeichnung} \\
			Darstellungsmittel: & \Benennung       &  \Symbolkette  \\
			\hline
		\end{tabular}
		\caption{\Bezeichnung\ von \Begriffen.}
		\label{tab:Bezeichnung}% Erst nach '\caption'!
	\end{center}
\end{table}

Wenn die \Bezeichnung\ mit einem internen Link versehen ist, gibt es eine Definition im Symbolverzeichnis (ab \Pageref{dic:Symbolverzeichnis}) oder Glossar (ab \Pageref{dic:Glossar})%
\footnote{%
	Möglicherweise steht dort statt einer Definition auch nur eine Referenz zur Definition im laufenden Text.
}.
Diese Bedeutung ist dann mit der \Bezeichnung\ gemeint.
Wenn die \Benennung\ mit der Fußnote "`\footnotemark[0]"' versehen ist, steht die vollständige Definition nur im Glossar und nicht im laufenden Text.
Im Glossar und Symbolverzeichnis sind vertiefende Beschreibungen unabhängig davon immer möglich.
Eine Übersicht über die verwendeten Farben und Textauszeichnungen findet man \vrefinsec{sec:Textauszeichnungen}.

Im Zusammenhang mit \Bezeichnungen\ definieren wir:

\begin{description}
	\item[\defTxt               {\Kette}]     \glsentrydesc               {Kette}

	\item[\defTxt   {\lateinischerBuchstabe}] \glsentrydesc   {lateinischerBuchstabe}
	\item[\defTxt      {\deutscherBuchstabe}] \glsentrydesc      {deutscherBuchstabe}
	\item[\defTxt   {\griechischerBuchstabe}] \glsentrydesc   {griechischerBuchstabe}
	\item[\defTxt           {\Textbuchstabe}] \glsentrydesc           {Textbuchstabe}

	\item[\defTxt   {\lateinischesWort}]      \glsentrydesc   {lateinischesWort}
	\item[\defTxt      {\deutschesWort}]      \glsentrydesc      {deutschesWort}
	\item[\defTxt   {\griechischesWort}]      \glsentrydesc   {griechischesWort}
	\item[\defTxt           {\Textwort}]      \glsentrydesc           {Textwort}

	\item[\defTxt{\typographischesSymbol}]    \glsentrydesc{typographischesSymbol}
	\item[\defTxt{\typographischesZeichen}]   \glsentrydesc{typographischesZeichen}

	\item[\defTxt {\atomaresSymbol}]          \glsentrydesc {atomaresSymbol}
	\item[\defTxt         {\Symbol}]          \glsentrydesc         {Symbol}
	\item[\defTxt        {\Zeichen}]          \glsentrydesc        {Zeichen}

	\item[\defTxt         {\Symbolkette}]     \glsentrydesc         {Symbolkette}
	\item[\defTxt        {\Zeichenkette}]     \glsentrydesc        {Zeichenkette}
\end{description}

\TypographischeZeichen\ sind normalerweise zusammenhängend.
Es gibt aber auch Ausnahmen, wie \textzB\ die Buchstaben i und j, das Gleichheitszeichen und die deutschen Umlaute.
Letztere kann man auch \defTxt{\zusammengesetzt} aus einem Buchstaben und dem Doppelpunkt darüber betrachten.
Wenn nichts anderes gesagt wird, betrachten wir diese Zeichen jedoch als Einheit.

Zur Verdeutlichung des logischen Zusammenhangs obiger Definitionen dient \vreftab{tab:Bezeichnungen}:

\begin{table}[h]
	\begin{center}
		\begin{threeparttable}
			\setlength\tabcolsep{3pt}
			\setlength\extrarowheight{3pt}
			\begin{tabularx}{14,17cm}{|p{2,0cm}|p{1,3cm}|p{2,0cm}|p{2,1cm}|p{2,8cm}|p{2,6cm}|}
				\hline
				\multirow{5}{2,0cm}{\atomar,\newline\unzerlegbar}
				&\multicolumn{5}{c|}{\Zeichen}                                                                          \\
				\cdashline{2-6}
				& \multicolumn{4}{c|}{\typographischesZeichen} & \multirow{4}{2,6cm}{Leerzeichen, Zwischenraum}         \\
				\cdashline{2-5}
				& \multicolumn{3}{c|}{\Textbuchstabe\Tnote{1}} & \multirow{3}{2,8cm}{\typographischesSymbol\Tnote{2}} & \\
				\cdashline{2-4}
				& \multicolumn{2}{c|}{\deutscherBuchstabe}     & \multirow{2}{2,1cm}{\griechischerBuchstabe}        & & \\
				\cdashline{2-3}
				& Umlaut oder ß &     \lateinischerBuchstabe   &                                                    & & \\
				\hline
				\multirow{5}{2,0cm}{daraus gebildete \Ketten\ (bis auf \Zeichenketten\ nichtleer)}
				&               & \lateinischesWort\Tnote{3}   & \multirow{2}{2,1cm}{\griechischesWort\Tnote{3}}    & & \\
				\cdashline{2-3}
				& \multicolumn{2}{c|}{\deutschesWort\Tnote{3}} &                                                    & & \\
				\cdashline{2-4}
				& \multicolumn{3}{c|}{\Textwort\Tnote{3}}                                                           & & \\
				& \multicolumn{4}{c|}{\atomaresSymbol\Tnote{4}}                                                                        & \\
				\cdashline{2-5}
				& \multicolumn{4}{c|}{\Symbol}                                                                        & \\
				\cdashline{2-6}
				& \multicolumn{5}{c|}{\Symbolkette}                                                                     \\
				\cdashline{2-6}
				& \multicolumn{5}{c|}{\Zeichenkette\Tnote{5}}                                                           \\
				\hline
			\end{tabularx}
			\begin{tablenotes}
				\footnotesize
				\item[1] Verschiedene Schriftarten heißt auch verschiedene \Textbuchstaben.
				\item[2] Für \typographischeSymbole\ gibt es nicht immer verschiedenen Schriftarten.
				\item[3] Jeweils das ganze \Textwort\ in derselben Schriftart, aus demselben Alphabet und mit denselben Textauszeichnungen.
				\item[4] Entweder ein \Textwort\ oder ein \typographischesSymbol.
				\item[5] \Zeichenketten\ dürfen auch leer sein. In dem Fall besitzen sie kein \Kettenglied.
			\end{tablenotes}
			\caption{Bezeichnungen für Zeichen und Worte.}
			\label{tab:Bezeichnungen}% Erst nach '\caption'!
		\end{threeparttable}
	\end{center}
\end{table}

\section     {Textauszeichnungen}% =============================================
\beginsection{Textauszeichnungen}
\label   {sec:Textauszeichnungen}

\Hier\ werden verschiedene Textauszeichnungen verwendet.
Die Farben \LinkFarbe\ und \BibFarbe\ bedeuten, dass --- außer bei Zitaten aus \Wikipedia%
\footnote{In \Wikipedia\ ist dort ein Link vorhanden.}
--- im PDF-Dokument der farbige Teil mit einem Link versehen ist.
Farben können mit anderen Textauszeichnungen kombiniert werden, sind davon aber unabhängig.
Die Bedeutung der Farben und Textauszeichnungen wird im Folgenden beschrieben:
\begin{itemize}
	\item Farben:
	\begin{itemize}
		\item \likeLinkFt{Interner Link} (\LinkFarbe); auch in Überschriften und mathematischen \Formeln, aber nicht in Zitaten aus \Wikipedia.
		\item\ [\likeBibFt{Nummer}] (\BibFarbe); ein Link ins Literaturverzeichnis.
		\iftestFlg
		\item
			\begin{offen}
				Teile, deren Bearbeitung zurückgestellt ist.
			\end{offen}
		(\offenFarbe)
		\else\fi
	\end{itemize}
	\item Seitennummern im Index, Symbolverzeichnis und Glossar:
	\begin{itemize}
		\item \likeHyperDef{Seitennummer} (\HyperDefTyp), auf der die  Definition des \Begriffs\ steht.
		\item \likeHyperTxt{Seitennummer} (\HyperTxtTyp), auf der eine Verwendung des \Begriffs\ steht.
	\end{itemize}
	Die Seitennummern sind immer mit einem Link auf die entsprechende Seite versehen.
\end{itemize}
Für die folgenden Textauszeichnungen spielen Farben keine Rolle mehr.
\begin{itemize}
	\item In Zitaten aus \Wikipedia:
	\begin{itemize}
		\item \likeWikiFt{Zitate aus \Wikipedia} (\WikiTyp) werden, soweit möglich, mit den originalen Textauszeichnungen versehen --- \textzB\ \likeWikiFt{\wikiBoldFt{\wikiBoldTyp}} und \likeWikiFt{\wikiItalicFt{\wikiItalicTyp}}, allerdings ohne die originalen Links.
		Farbige Teile deuten nur darauf hin, das im Original dort ein Links ist.
	\end{itemize}
	\item In Definitionen
	\begin{itemize}
		\item \DefFt{\Benennung} (\DefTyp) des zu definierenden \Begriffs.\footnote{%
			Die \Benennung\ ist häufig noch mit der Farbe für einen Link versehen.
			Für \Symbole\ gibt es leider keine Textauszeichnung außer Farben.
		}
		\item \ManFt{Notwendige Teile} (\ManTyp) von Sprechweisen.
		\item \OptFt{Optionale  Teile} (\OptTyp)von Sprechweisen.
		\item \GloFt{Erstmalige Selbstreferenz} (\GloTyp) (ohne Link).
		\item \gloFt           {Selbstreferenz} (\gloTyp) (ohne Link).
	\end{itemize}
	\item In mathematischen Formeln:
	\begin{itemize}
		\item $\Varft       {\Variable\ allgemein}$ (\VarTyp); ein \Textbuchstabe, normalerweise klein.
		\item $\varft            {Variablensymbol}$ (\varTyp); ein \Textbuchstabe, normalerweise klein.
		\item $\Conft                 {\Konstante}$ (\ConTyp); ein \Textwort.
		\item $\Setft     {VORGEGEBENER\ \BEREICH}$ (\SetTyp); ein \Textbuchstabe, immer groß.%
			\footnote{Kleinbuchstaben gibt es in dieser Schriftart nicht.}
		\item $\Elmft           {\Element\ daraus}$ (\ElmTyp); ein \Textbuchstabe, normalerweise groß.
		\item $\BOpft         {\Bereichsoperation}$ (\BOpTyp); ein \Textwort.
		\item $\Drvft{\Bereich\ von\ \Ableitungen}$ (\DrvTyp); ein \Textbuchstabe, normalerweise groß.
		\item $\drvft           {\Element\ daraus}$ (\drvTyp); ein \Textbuchstabe.
		\item $\Preft                 {\Praedikat}$ (\PreTyp); ein \Textwort.
	\end{itemize}
	\item In sonstigem Text (ohne Überschriften):
	\begin{itemize}
		\item    \CharFt{Zeichen} (\CharTyp); einzeln oder in \Zeichenketten.
		\item \likePreFt{\Praedikat} (\PreTyp) außerhalb von mathematischen \Formeln.
	\end{itemize}
\end{itemize}

%TODO ### hier weitermachen
\section     {Quotierung}% =====================================================
\beginsection{Quotierung}
\label   {sec:Quotierung}

Zur Verdeutlichung der soeben definierten Quotierungen ein Beispiel:\footnote{%
	Was \atomare\ und was \zerlegbare\ \Symbole\ sind, muss jeweils definiert werden, \textbzw\ ergibt sich aus dem Zusammenhang.
}

\begin{tabular}{llll}
	&        $\sin$  & \Objekt
	& die Sinusfunktion
	\\
	& \chrqt{$\sin$} & \Bezeichnung
	& für das \Objekt
	\\
	& \seqqt{$\sin$} & \Symbolkette\ (\Formel)
	& aus dem \zusammengesetzten, \atomaren\ \Symbol\ \chrqt{$\sin$}
	\\
	& \seqqt {$sin$} & \Symbolkette\ (\Formel)
	& aus den einfachen \Symbolen\ \chrqt{$s$}, \chrqt{$i$} und \chrqt{$n$}
	\\
	& \strqt  {sin}  & \Zeichenkette
	& aus den einfachen \Symbolen\ \chrqt{\CharFt{s}}, \chrqt{\CharFt{i}} und \chrqt{\CharFt{n}}
\end{tabular}

\section     {Sonstiges}% ======================================================
\beginsection{Sonstiges}
\label   {sec:Sonstiges}

Fußnoten dienen nur zu weiteren Erläuterungen sowie Verweisen in dieses Dokument und die Literatur.
Daher können sie auch etwas "`lascher"' formuliert sein.
Für das Verständnis des Textes sollten sie nicht nötig sein, es reichen Grundkenntnisse der Mathematik.

Wenn im Text "`wir"' verwendet wird, geht es um Definitionen, die so nicht allgemein verwendet werden.\footnote{%
	"`Wir"' und nicht "`ich"', da der Leser eingeschlossen werden soll und in Zukunft möglicherweise auch andere Autoren an diesem Dokument beteiligt sein werden.
}
Die Verwendung von "`wir"' ist allerdings nicht immer konsistent und soll nur als Hinweis dienen.

\iftestFlg
\section     {Klassifizierung von Glossar-Einträgen}% ======================
\beginsection{Klassifizierung}
\label   {sec:Klassifizierung}

Mögliche Klassifizierungen von Symbolen und Glossareinträgen:
\begin{itemize}
%%%	\item \todoKlassifizieren Nicht klassifiziert.
	\item \todoBeschreiben    Eintrag muss noch beschrieben werden.
	\item \todoErgaenzen      Eintrag muss noch ergänzt werden.
	\item \todoPruefen        Eintrag muss noch geprüft werden.
	\item \todoGeprueft       Eintrag ist geprüft, Link oder Definition im Text noch nicht.
	\item \todoOk             Eintrag ist geprüft und ok.
\end{itemize}
\else\fi

\Endchapter
