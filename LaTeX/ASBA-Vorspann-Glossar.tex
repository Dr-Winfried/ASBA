%%############################################################################%%
%%                                                                            %%
%% Datei:  ASBA-Vorspann-Glossary.tex                                         %%
%% Inhalt: Vorspann Glossareinträge für ASBA                                  %%
%%                                                                            %%
%% Copyright (C) 2017  Winfried Teschers                                      %%
%%                                                                            %%
%% This program is free software: you can redistribute it and/or modify       %%
%% it under the terms of the GNU Affero General Public License as published   %%
%% by the Free Software Foundation, either version 3 of the License, or       %%
%% (at your option) any later version.                                        %%
%%                                                                            %%
%% This program is distributed in the hope that it will be useful,            %%
%% but WITHOUT ANY WARRANTY; without even the implied warranty of             %%
%% MERCHANTABILITY or FITNESS FOR A PARTICULAR PURPOSE.  See the              %%
%% GNU Affero General Public License for more details.                        %%
%%                                                                            %%
%% You should have received a copy of the GNU Affero General Public License   %%
%% along with this program.  If not, see <http://www.gnu.org/licenses/>.      %%
%%                                                                            %%
%% Dr. Winfried Teschers                                                      %%
%% Anton-Günther-Straße 26c                                                   %%
%% 91083 Baiersdorf                                                           %%
%% Germany                                                                    %%
%%                                                                            %%
%% e-mail: winfried.teschers@t-online.de                                      %%
%%                                                                            %%
%%############################################################################%%

% !TeX root = ASBA.tex
% !TeX encoding = UTF-8
% !TeX spellcheck = de_DE

% Elemente, die keine Glossareinträge sind und dafür nicht gebraucht werden,
% werden in "ASBA-Vorspann.tex" und "ASBA-Mathematik-Vorspann.tex" definiert.

\renewcommand*{\acronymfont}[1]{\textbf{#1}}% vermeidet fehlenden Glossar-Font

%TODO Indices und Glossar in die richtige Reihenfolge bringen; Manche Symbole sind mehrfach vorhanden
%TODO Im Index und Glossar prüfen: Haben alle Einträge einen Verweise auf die Definition?

% Kommandos zum Eintragen im IndeX, Symbolverzeichnis und Glossar

%TODO optionalen Parameter anwenden - z.B. bei der Definition
% Texte;#1=Font-Kommando;#2=Golossary key
\newcommand*{\idx}[2][]{\sindex[idx]{#2|#1}}% 2=Index-Eintrag

\newcommand*{\glsIdx}  [2][]{\gls       {#2}\idx[#1]{\glsentryname{#2}}}
\newcommand*{\glsIdxG} [2][]{\glsuseri  {#2}\idx[#1]{\glsentryname{#2}}}
\newcommand*{\glsIdxD} [2][]{\glsuserii {#2}\idx[#1]{\glsentryname{#2}}}
\newcommand*{\glsIdxA} [2][]{\glsuseriii{#2}\idx[#1]{\glsentryname{#2}}}
\newcommand*{\glsIdxPl}[2][]{\glspl     {#2}\idx[#1]{\glsentryname{#2}}}
\newcommand*{\glsIdxPG}[2][]{\glsuseriv {#2}\idx[#1]{\glsentryname{#2}}}
\newcommand*{\glsIdxPD}[2][]{\glsuserv  {#2}\idx[#1]{\glsentryname{#2}}}
\newcommand*{\glsIdxPA}[2][]{\glsuservi {#2}\idx[#1]{\glsentryname{#2}}}

\newcommand*{\GlsIdx}  [2][]{\Gls       {#2}\idx[#1]{\Glsentryname{#2}}}
\newcommand*{\GlsIdxG} [2][]{\Glsuseri  {#2}\idx[#1]{\Glsentryname{#2}}}
\newcommand*{\GlsIdxD} [2][]{\Glsuserii {#2}\idx[#1]{\Glsentryname{#2}}}
\newcommand*{\GlsIdxA} [2][]{\Glsuseriii{#2}\idx[#1]{\Glsentryname{#2}}}
\newcommand*{\GlsIdxPl}[2][]{\Glspl     {#2}\idx[#1]{\Glsentryname{#2}}}
\newcommand*{\GlsIdxPG}[2][]{\Glsuseriv {#2}\idx[#1]{\Glsentryname{#2}}}
\newcommand*{\GlsIdxPD}[2][]{\Glsuserv  {#2}\idx[#1]{\Glsentryname{#2}}}
\newcommand*{\GlsIdxPA}[2][]{\Glsuservi {#2}\idx[#1]{\Glsentryname{#2}}}

% Symbole
\newcommand*{\sym}[2][]{\sindex[sym]{\ensuremath{#2}|#1}}% 2=Symbol-Eintrag
\newcommand*{\glsSym}[2][]{\glssymbol {#2}\sym[#1]{\glsentrysymbol{#2}}}
\newcommand*{\glsTag}[2][]{\glssymbol*{#2}\sym[#1]{\glsentrysymbol{#2}}}

%TODO Definitions-Verweise ins Glossar müssen noch definiert werden
\newcommand*{\glos}[1]{\textsc{#1}}% Für Verweise ins Glossar

% Glossar-Einträge #############################################################

% Symbole für Mengen -----------------------------------------------------------
% \symXX - Ausgabe als Symbol und Aufnahme in Symbolliste und Glossar

\newcommand*{\AussageLetter} {A}% [a]ussagenlogisch
\newcommand*{\polnischLetter}{p}% ...in [p]olnischer Notation

\newcommand*        {\symIN}[1][]{\glsSym[#1]{IN}}
\newglossaryentry       {IN}{
	name  ={\ensuremath{\IN}},
	symbol={\ensuremath{\IN}},
	sort  ={N},
	description={
		Die Menge der natürlichen Zahlen ohne 0.
		\\-- Zur Definition \vrefseesub{sub-Bezeichnungen}
	}
}
\newcommand*        {\symINo}[1][]{\glsSym[#1]{INo}}
\newglossaryentry       {INo}{
	name  ={\ensuremath{\INo}},
	symbol={\ensuremath{\INo}},
	sort  ={N 0},
	description={
		Die Menge der natürlichen Zahlen einschließlich 0.
		\\-- Zur Definition \vrefseesub{sub-Bezeichnungen}.
	}
}
\newcommand*           {\alABCLetter}{A}% [A]lphabet der al Sprache
\newcommand*        {\symalABC}[1][]{\glsSym[#1]{alABC}}
\newglossaryentry       {alABC}{
	name  ={\ensuremath{\alABC}},
	symbol={\ensuremath{\alABC}},
	sort  ={A},%        \alABCLetter
	description={
		Das Alphabet der aussagenlogischen \gls{Sprache}.
		\\-- Zur Definition \vrefseesubsub{subsub-Formeln}.
	}
}
\newcommand*        {\symalABCx}[1][]{\glsSym[#1]{alABCx}}
\newglossaryentry       {alABCx}{
	name  ={\ensuremath{\alABC_x}},
	symbol={\ensuremath{\alABC_x}},
	sort  ={A x},%      \alABCLetter x
	description={
		Eine Teilmenge des Alphabets $\alABC$ der aussagenlogischen \gls{Sprache}.
		\\-- Zur Definition \vrefseesubsub{subsub-Formeln}.
	}
}
\newcommand*           {\alBinLetter}{O}% binäre [O]perationssymbole
\newcommand*        {\symalBin}[1][]{\glsSym[#1]{alBin}}
\newglossaryentry       {alBin}{
	name  ={\ensuremath{\alBin}},
	symbol={\ensuremath{\alBin}},
	sort  ={O},%        \alBinLetter
	description={
		Die Menge der binären \glspl{Junktor}.
		\\-- Zur Definition \vrefseesubsub{subsub-Bausteine}.
	}
}
\newcommand*           {\alConLetter}{K}% [K]onstantensymbole
\newcommand*        {\symalCon}[1][]{\glsSym[#1]{alCon}}
\newglossaryentry       {alCon}{
	name  ={\ensuremath{\alCon}},
	symbol={\ensuremath{\alCon}},
	sort  ={K},%        \alConLetter
	description={
		Die Menge der aussagenlogischen Konstanten.
		\\-- Zur Definition \vrefseesubsub{subsub-Bausteine}.
	}
}
\newcommand*           {\alForLetter}{L}% Sprache, [l]anguage; siehe auch \formulaSetLetter
\newcommand*        {\symalFor}[1][]{\glsSym[#1]{alFor}}
\newglossaryentry       {alFor}{
	name  ={\ensuremath{\alFor}},
	symbol={\ensuremath{\alFor}},
	sort  ={L A},%      \alForLetter \AussageLetter
	description={
		Die Menge der aussagenlogischen \glspl{Formel} mit Klammerung.
	}
}
\newcommand*        {\symalForp}[1][]{\glsSym[#1]{alForp}}
\newglossaryentry       {alForp}{
	name  ={\ensuremath{\alForp}},
	symbol={\ensuremath{\alForp}},
	sort  ={L Ap},%     \alForLetter A \polnischLetter
	description={
		Die Menge der aussagenlogischen \glspl{Formel} in polnischer Notation.
	}
}
\newcommand*        {\symalForx}[1][]{\glsSym[#1]{alForx}}
\newglossaryentry       {alForx}{
	name  ={\ensuremath{\alFor_x}},
	symbol={\ensuremath{\alFor_x}},
	sort  ={L A x},%    \alForLetter A x
	description={
		Eine Teilmenge der Menge $\symalFor$ der aussagenlogischen \glspl{Formel} mit Klammerung.
	}
}
\newcommand*        {\symalForpx}[1][]{\glsSym[#1]{alForpx}}
\newglossaryentry       {alForpx}{
	name  ={\ensuremath{\alForp_x}},
	symbol={\ensuremath{\alForp_x}},
	sort  ={L Ap x},%  \alForLetter A\polnischLetter x
	description={
		Eine Teilmenge der Menge $\symalForp$ der aussagenlogischen \glspl{Formel} in polnischer Notation.
	}
}
\newcommand*           {\alJunLetter}{J}% [J]unktoren
\newcommand*        {\symalJun}[1][]{\glsSym[#1]{alJun}}
\newglossaryentry       {alJun}{
	name  ={\ensuremath{\alJun}},
	symbol={\ensuremath{\alJun}},
	sort  ={J},%        \alJunLetter
	description={
		Die Menge der \glspl{Junktorsymbol}.
		\\-- Zur Definition \vrefseesubsub{subsub-Bausteine}.
	}
}
\newcommand*        {\symalJunx}[1][]{\glsSym[#1]{alJunx}}
\newglossaryentry       {alJunx}{
	name  ={\ensuremath{\alJun_x}},
	symbol={\ensuremath{\alJun_x}},
	sort  ={J x},%      \alJunLetter x
	description={
		Eine Teilmenge der Menge $\alJun$ der \glspl{Junktorsymbol}.
		\\-- Zur Definition \vrefseesubsub{subsub-Bausteine}.
	}
}
\newcommand*           {\alUnaLetter}{U}% [u]näre Operationssymbole
\newcommand*        {\symalUna}[1][]{\glsSym[#1]{alUna}}
\newglossaryentry       {alUna}{
	name  ={\ensuremath{\alUna}},
	symbol={\ensuremath{\alUna}},
	sort  ={U},%        \alUnaLetter
	description={
		Die Menge der unären \glspl{Junktor}.
		\\-- Zur Definition \vrefseesubsub{subsub-Bausteine}.
	}
}
\newcommand*           {\alvarLetter}{q}% Name einer Variablen
\newcommand*           {\alVarLetter}{Q}% Variablensymbole
\newcommand*        {\symalVar}[1][]{\glsSym[#1]{alVar}}
\newglossaryentry       {alVar}{
	name  ={\ensuremath{\alVar}},
	symbol={\ensuremath{\alVar}},
	sort  ={Q},%        \alVarLetter
	description={
		Die Menge der aussagenlogischen Variablen $\alvar_i$ für $i \in \INo$.
		\\-- Zur Definition \vrefseesubsub{subsub-Bausteine}.
	}
}
\newcommand*           {\formulaSetLetter}{L}% Sprache, [l]anguage; siehe auch \alForLetter
\newcommand*        {\symformulaSet}[1][]{\glsSym[#1]{formulaSet}}
\newglossaryentry       {formulaSet}{
	name  ={\ensuremath{\formulaSet}},
	symbol={\ensuremath{\formulaSet}},
	sort  ={L},%        \formulaSetLetter
	description={
		\gls{Formelmenge}.
	}
}
\newcommand*  {\symMengeMo}[1][]{\glsSym[#1]{MengeMo}}
\newglossaryentry {MengeMo}{
	name  ={\ensuremath{M^0}},
	symbol={\ensuremath{M^0}},
	sort  ={M 0},
	description={
		$\{()\}$ , wobei $()$ das 0-Tupel ist.
		\\-- Zur Definition \vrefseesub{sub-Bezeichnungen}.
	}
}
\newcommand*  {\symMengeMn}[1][]{\glsSym[#1]{MengeMn}}
\newglossaryentry {MengeMn}{
	name  ={\ensuremath{M^n}},
	symbol={\ensuremath{M^n}},
	sort  ={M n},
	description={
		Das kartesische Produkt $M \times \dots \times M$ aus $n$ Mengen $M$ mit $n \in \INo$.
		\\-- Zur Definition \vrefseesub{sub-Bezeichnungen}.
	}
}
\newcommand*           {\tupelSetLetter}{S}% Menge der Tupel; [S]equenz
\newcommand*        {\symtupelSet}[1][]{\glsSym[#1]{tupelSet}}
\newglossaryentry       {tupelSet}{
	name  ={\ensuremath{\tupelSet}},
	symbol={\ensuremath{\tupelSet}},
	sort  ={S},%        \tupelSetLetter
	see   ={[siehe auch]{Tupelmenge}},
	description={
		$(M)$ ist die Menge aller \Tupel\ aus $M$.
	}
}

% Symbole für Beispieloperationen und -relationen ------------------------------
% \symXX - Ausgabe als Symbol und Aufnahme in Symbolliste und Glossar

\newcommand*        {\symopubsp}[1][]{\glsSym[#1]{opubsp}}
\newglossaryentry       {opubsp}{
	name  ={\ensuremath{\opubsp}},
	symbol={\ensuremath{\opubsp}},
	sort  ={= 0 1 1},
	description={
		Beispielsymbol für eine unäre \gls{Operation}.
		\\-- Zur Definition \vrefseesub{sub-Beispielsymbole}.
	}
}
\newcommand*        {\symopbsp}[1][]{\glsSym[#1]{opbsp}}
\newglossaryentry       {opbsp}{
	name  ={\ensuremath{\opbsp}},
	symbol={\ensuremath{\opbsp}},
	sort  ={= 0 1 2},
	description={
		Beispielsymbol für eine binäre \gls{Operation}.
		\\-- Zur Definition \vrefseesub{sub-Beispielsymbole}.
	}
}
\newcommand*        {\symrelbsp}[1][]{\glsSym[#1]{relbsp}}
\newglossaryentry       {relbsp}{
	name  ={\ensuremath{\relbsp}},
	symbol={\ensuremath{\relbsp}},
	sort  ={= 0 1 3},
	description={
		Beispielsymbol für eine binäre \gls{Relation} mit \gls{Umkehrrelation} \gls{relbackbsp}.
		\\-- Zur Definition \vrefseesub{sub-Beispielsymbole}.
	}
}
\newcommand*        {\symreleqbsp}[1][]{\glsSym[#1]{releqbsp}}
\newglossaryentry       {releqbsp}{
	name  ={\ensuremath{\releqbsp}},
	symbol={\ensuremath{\releqbsp}},
	sort  ={= 0 1 4},
	description={
		Beispielsymbol für eine binäre \gls{Relation} mit \gls{Gleichheit} und \gls{Umkehrrelation} \gls{relbackeqbsp}.
		\\-- Zur Definition \vrefseesub{sub-Beispielsymbole}.
	}
}
\newcommand*        {\symrelbackbsp}[1][]{\glsSym[#1]{relbackbsp}}
\newglossaryentry       {relbackbsp}{
	name  ={\ensuremath{\relbackbsp}},
	symbol={\ensuremath{\relbackbsp}},
	sort  ={= 0 1 5},
	description={
		Beispielsymbol für eine binäre \gls{Relation} mit \gls{Umkehrrelation} \gls{relbsp}.
		\\-- Zur Definition \vrefseesub{sub-Beispielsymbole}.
	}
}
\newcommand*        {\symrelbackeqbsp}[1][]{\glsSym[#1]{relbackeqbsp}}
\newglossaryentry       {relbackeqbsp}{
	name  ={\ensuremath{\relbackeqbsp}},
	symbol={\ensuremath{\relbackeqbsp}},
	sort  ={= 0 1 6},
	description={
		Beispielsymbol für eine binäre \gls{Relation} mit \gls{Gleichheit} und \gls{Umkehrrelation} \gls{releqbsp}.
		\\-- Zur Definition \vrefseesub{sub-Beispielsymbole}.
	}
}
\newcommand*        {\symrelnbsp}[1][]{\glsSym[#1]{relnbsp}}
\newglossaryentry       {relnbsp}{
	name  ={\ensuremath{\relnbsp}},
	symbol={\ensuremath{\relnbsp}},
	sort  ={= 0 1 7},
	description={
		Verneinung von $\relbsp$.
		\\-- Zur Definition \vrefseesub{sub-Beispielsymbole}.
	}
}
\newcommand*        {\symrelnebsp}[1][]{\glsSym[#1]{relnebsp}}
\newglossaryentry       {relnebsp}{
	name  ={\ensuremath{\relnebsp}},
	symbol={\ensuremath{\relnebsp}},
	sort  ={= 0 1 8},
	description={
		Verneinung von $\releqbsp$.
		\\-- Zur Definition \vrefseesub{sub-Beispielsymbole}.
	}
}

% Meta-Symbole -----------------------------------------------------------------
% \symXX - Ausgabe als Symbol und Aufnahme in Symbolliste und Glossar

\newcommand*        {\symmetanot}[1][]{\glsSym[#1]{metanot}}
\newglossaryentry       {metanot}{
	name  ={\ensuremath{\metanot}},
	symbol={\ensuremath{\metanot}},
	sort  ={= 1 1 1},
	description={
		Eine unäre \gls{Metaoperation}:~ \textdots\ \emph{gilt nicht}
		\\-- Zur Definition \vrefseesub{sub-AussagenUndMetaoperationen}.
	}
}
\newcommand*        {\symmetaand}[1][]{\glsSym[#1]{metaand}}
\newglossaryentry       {metaand}{
	name  ={\ensuremath{\metaand}},
	symbol={\ensuremath{\metaand}},
	sort  ={= 1 1 2},
	description={
		Eine \gls{Metaoperation}:~ \textdots\ \emph{und} \textdots
		\\-- Zur Definition \vrefseesub{sub-AussagenUndMetaoperationen}.
	}
}
\newcommand*        {\symmetaor}[1][]{\glsSym[#1]{metaor}}
\newglossaryentry       {metaor}{
	name  ={\ensuremath{\metaor}},
	symbol={\ensuremath{\metaor}},
	sort  ={= 1 1 3},
	description={
		Eine \gls{Metaoperation}:~ \textdots\ \emph{oder} \textdots
		\\-- Zur Definition \vrefseesub{sub-AussagenUndMetaoperationen}.
	}
}
\newcommand*        {\symmetaimp}[1][]{\glsSym[#1]{metaimp}}
\newglossaryentry       {metaimp}{
	name  ={\ensuremath{\metaimp}},
	symbol={\ensuremath{\metaimp}},
	sort  ={= 1 2 1},
	description={
		Eine \gls{Metarelation}:~ \textdots\ \emph{dann auch} \textdots, die \gls{Umkehrrelation} zu \gls{metarep}.
		\\-- Zur Definition \vrefseesub{sub-AussagenUndMetaoperationen}.
	}
}
\newcommand*        {\symmetarep}[1][]{\glsSym[#1]{metarep}}
\newglossaryentry       {metarep}{
	name  ={\ensuremath{\metarep}},
	symbol={\ensuremath{\metarep}},
	sort  ={= 1 2 2},
	description={
		Eine \gls{Metarelation}:~ \textdots\ \emph{sofern} \textdots , die \gls{Umkehrrelation} zu \gls{metaimp}.
		\\-- Zur Definition \vrefseesub{sub-AussagenUndMetaoperationen}.
	}
}
\newcommand*        {\symmetaequiv}[1][]{\glsSym[#1]{metaequiv}}
\newglossaryentry       {metaequiv}{
	name  ={\ensuremath{\metaequiv}},
	symbol={\ensuremath{\metaequiv}},
	sort  ={= 1 2 3},
	description={
		Eine \gls{Metarelation}:~ \textdots\ \emph{genau dann wenn} \textdots
		\\-- Zur Definition \vrefseesub{sub-AussagenUndMetaoperationen}.
	}
}
\newcommand*        {\symeq}[1][]{\glsSym[#1]{eq}}
\newglossaryentry       {eq}{
	name  ={\ensuremath{\eq}},
	symbol={\ensuremath{\eq}},
	sort  ={= 1 3 1},
	description={
		Eine \gls{Metarelation}:~ \textdots\ \emph{ist gleich} (dasselbe wie; identisch zu) \textdots
		\\-- Siehe \gls{Gleichheit}.
		\\-- Zur Definition \vrefseesubsub{subsub-Vergleiche} und \vrefseesub{sub-ausJunktorDef}.
	}
}
\newcommand*        {\symne}[1][]{\glsSym[#1]{ne}}
\newglossaryentry       {ne}{
	name  ={\ensuremath{\ne}},
	symbol={\ensuremath{\ne}},
	sort  ={= 1 3 2},
	description={
		Eine \gls{Metarelation}:~ \textdots\ \emph{ist ungleich} (nicht dasselbe wie, nicht identisch zu) \textdots
	}
}
\newcommand*        {\symequiv}[1][]{\glsSym[#1]{equiv}}
\newglossaryentry       {equiv}{
	name  ={\ensuremath{\equiv}},
	symbol={\ensuremath{\equiv}},
	sort  ={= 1 3 3},
	description={
		Eine \gls{Metarelation}:~ \textdots\ \emph{äquivalent} (so wie; ähnlich) \textdots
		\\-- Siehe \gls{Aequivalenz}.
		\\-- Zur Definition \vrefseesubsub{subsub-Vergleiche} und \vrefseesub{sub-ausJunktorDef}.
	}
}
\newcommand*        {\symnequiv}[1][]{\glsSym[#1]{nequiv}}
\newglossaryentry       {nequiv}{
	name  ={\ensuremath{\nequiv}},
	symbol={\ensuremath{\nequiv}},
	sort  ={= 1 3 4},
	description={
		Eine \gls{Metarelation}:~ \textdots\ \emph{nicht äquivalent} (nicht so wie; nicht ähnlich) \textdots
	}
}
\newcommand*        {\symmetadefeq}[1][]{\glsSym[#1]{metadefeq}}
\newglossaryentry       {metadefeq}{
	name  ={\ensuremath{\metadefeq}},
	symbol={\ensuremath{\metadefeq}},
	sort  ={= 1 4 1},
	description={
		\gls{Metadefinition}:~ \textdots\ \emph{definitionsgemäß genau dann wenn} \textdots
	}
}
\newcommand*        {\symdefeq}[1][]{\glsSym[#1]{defeq}}
\newglossaryentry       {defeq}{
	name  ={\ensuremath{\defeq}},
	symbol={\ensuremath{\defeq}},
	sort  ={= 1 4 2},
	description={
		\gls{Definition}:~ \textdots\ \emph{definitionsgemäß gleich} (dasselbe wie; identisch zu) \textdots
	}
}

\newcommand*        {\symsrand}[1][]{\glsSym[#1]{srand}}
\newglossaryentry       {srand}{
	name  ={\ensuremath{\srand}},
	symbol={\ensuremath{\srand}},
	sort  ={= 1 5 1},
	description={
		Eine \gls{Metaoperation}:~ \textdots\ \emph{und} \textdots\
		\\-- Wird nur bei den \glspl{Schlussregel} verwendet.
	}
}
\newcommand*        {\symderive}[1][]{\glsSym[#1]{derive}}
\newglossaryentry       {derive}{
	name  ={\ensuremath{\derive}},
	symbol={\ensuremath{\derive}},
	sort  ={= 1 5 2},
	description={
		\gls{Ableitungsrelation}:~ \textdots\ \emph{\gls{ableitbar}} (\gls{beweisbar}) \textdots
	}
}
\newcommand*        {\symderiveR}[1][]{\glsSym[#1]{deriveR}}
\newglossaryentry       {deriveR}{
	name  ={\ensuremath{\derive_R}},
	symbol={\ensuremath{\derive_R}},
	sort  ={= 1 5 2R},
	description={
		Eine Darstellung der \gls{Relation} $R$ aus $\Rel(\Pot(\formulaSet))$ als \gls{Ableitungsrelation}.
	}
}
\newcommand*        {\symsubst}[1][]{\glsSym[#1]{subst}}
\newglossaryentry       {subst}{
	name  ={\ensuremath{\subst}},
	symbol={\ensuremath{\subst}},
	sort  ={= 1 5 3},
	description={
		\gls{Substitution}:~ \textdots\ \emph{substituiert durch} \textdots\
		\\-- Zur Definition \vrefseesub{sub-Identitaetsregeln}.
	}
}
\newcommand*        {\symswap}[1][]{\glsSym[#1]{swap}}
\newglossaryentry       {swap}{
	name  ={\ensuremath{\swap}},
	symbol={\ensuremath{\swap}},
	sort  ={= 1 5 4},
	description={
		\gls{Vertauschung}:~ \textdots\ \emph{vertauscht mit} \textdots\
		\\-- Zur Definition \vrefseesub{sub-Identitaetsregeln}.
	}
}

% aussagenlogische Operationen, dargestellt mit Symbolen -----------------------
% \symXX - Ausgabe als Symbol und Aufnahme in Symbolliste und Glossar

\newcommand*        {\symlfalse}[1][]{\glsSym[#1]{lfalse}}
\newglossaryentry       {lfalse}{
	name  ={\ensuremath{\lfalse}},
	symbol={\ensuremath{\lfalse}},
	sort  ={= 2 0 1},
	description={
		Ein 0-stelliger \gls{Junktor}, \textdh\ eine aussagenlogische Konstante (\gls{Wahrheitswert}): \emph{$\falsch$}
		\\-- Zur Definition \vrefseetab{tab-Symbole}.
	}
}
\newcommand*        {\symltrue}[1][]{\glsSym[#1]{ltrue}}
\newglossaryentry       {ltrue}{
	name  ={\ensuremath{\ltrue}},
	symbol={\ensuremath{\ltrue}},
	sort  ={= 2 0 2},
	description={
		Ein 0-stelliger \gls{Junktor}, \textdh\ eine aussagenlogische Konstante (\gls{Wahrheitswert}): \emph{$\wahr$}
		\\-- Zur Definition \vrefseetab{tab-Symbole}.
	}
}
\newcommand*        {\symlnot}[1][]{\glsSym[#1]{lnot}}
\newglossaryentry       {lnot}{
	name  ={\ensuremath{\lnot}},
	symbol={\ensuremath{\lnot}},
	sort  ={= 2 1 1},
	description={
		Ein unärer \gls{Junktor}:~ \emph{nicht} \textdots\
		\\-- Zur Definition \vrefseetab{tab-Symbole}.
	}
}
\newcommand*        {\symland}[1][]{\glsSym[#1]{land}}
\newglossaryentry       {land}{
	name  ={\ensuremath{\land}},
	symbol={\ensuremath{\land}},
	sort  ={= 2 1 2},
	description={
		Ein binärer \gls{Junktor}:~ \textdots\ \emph{und} \textdots\
		\\-- Zur Definition \vrefseetab{tab-Symbole}.
	}
}
\newcommand*        {\symlor}[1][]{\glsSym[#1]{lor}}
\newglossaryentry       {lor}{
	name  ={\ensuremath{\lor}},
	symbol={\ensuremath{\lor}},
	sort  ={= 2 1 3},
	description={
		Ein binärer \gls{Junktor}:~ \textdots\ \emph{oder} \textdots\
	}
	\\-- Zur Definition \vrefseetab{tab-Symbole}.
}
\newcommand*        {\symlimp}[1][]{\glsSym[#1]{limp}}
\newglossaryentry       {limp}{
	name  ={\ensuremath{\limp}},
	symbol={\ensuremath{\limp}},
	sort  ={= 2 2 1},
	description={
		Ein binärer \gls{Junktor}:~ \emph{aus} \textdots\ \emph{folgt} \textdots\
		\\-- Zur Definition \vrefseetab{tab-Symbole}.
	}
}
\newcommand*        {\symlrep}[1][]{\glsSym[#1]{lrep}}
\newglossaryentry       {lrep}{
	name  ={\ensuremath{\lrep}},
	symbol={\ensuremath{\lrep}},
	sort  ={= 2 2 2},
	description={
		Ein binärer \gls{Junktor}:~ \textdots\ \emph{folgt aus} \textdots\
		\\-- Zur Definition \vrefseetab{tab-Symbole}.
	}
}
\newcommand*        {\symlequiv}[1][]{\glsSym[#1]{lequiv}}
\newglossaryentry       {lequiv}{
	name  ={\ensuremath{\lequiv}},
	symbol={\ensuremath{\lequiv}},
	sort  ={= 2 2 3},
	description={
		Ein binärer \gls{Junktor}:~ \textdots\ \emph{genau dann wenn} \textdots\
		\\-- Zur Definition \vrefseetab{tab-Symbole}.
	}
}
\newcommand*        {\symlxor}[1][]{\glsSym[#1]{lxor}}
\newglossaryentry       {lxor}{
	name  ={\ensuremath{\lxor}},
	symbol={\ensuremath{\lxor}},
	sort  ={= 2 3 1},
	description={
		Ein binärer \gls{Junktor}:~ \emph{entweder} \textdots\ \emph{oder} \textdots\
		\\-- Zur Definition \vrefseetab{tab-Symbole}.
	}
}
\newcommand*        {\symlnand}[1][]{\glsSym[#1]{lnand}}
\newglossaryentry       {lnand}{
	name  ={\ensuremath{\lnand}},
	symbol={\ensuremath{\lnand}},
	sort  ={= 2 3 2},
	description={
		Ein binärer \gls{Junktor}:~ \emph{nicht zugleich} \textdots\ \emph{und} \textdots\
		\\-- Zur Definition \vrefseetab{tab-Symbole}.
	}
}
\newcommand*        {\symlnor}[1][]{\glsSym[#1]{lnor}}
\newglossaryentry       {lnor}{
	name  ={\ensuremath{\lnor}},
	symbol={\ensuremath{\lnor}},
	sort  ={= 2 3 3},
	description={
		Ein binärer \gls{Junktor}:~ \emph{weder} \textdots\ \emph{noch} \textdots\
	}
	\\-- Zur Definition \vrefseetab{tab-Symbole}.
}

% Mengen-Operatoren ------------------------------------------------------------
% \symXX - Ausgabe als Symbol und Aufnahme in Symbolliste und Glossar

\newcommand*        {\symsubset}[1][]{\glsSym[#1]{subset}}
\newglossaryentry       {subset}{
	name  ={\ensuremath{\subset}},
	symbol={\ensuremath{\subset}},
	sort  ={= 3 1 1},
	description={
		Teilmengenbeziehung:~ \textdots\ \emph{ist echte Teilmenge von} \textdots\
		; Insbesondere kann keine \Gleichheit\ bestehen.
		In der Literatur wird $\subset$ oft im Sinne von $\subseteq$ verwendet.
		\\-- Zur Definition \vrefseesub{sub-Bezeichnungen}.
	}
}
\newcommand*        {\symsubseteq}[1][]{\glsSym[#1]{subseteq}}
\newglossaryentry       {subseteq}{
	name  ={\ensuremath{\subseteq}},
	symbol={\ensuremath{\subseteq}},
	sort  ={= 3 1 2},
	description={
		Teilmengenbeziehung:~ \textdots\ \emph{ist Teilmenge von} \textdots\
		; Insbesondere kann \Gleichheit\ bestehen.
		\\-- Zur Definition \vrefseesub{sub-Bezeichnungen}.
	}
}
\newcommand*        {\symnsubset}[1][]{\glsSym[#1]{nsubset}}
\newglossaryentry       {nsubset}{
	name  ={\ensuremath{\nsubset}},
	symbol={\ensuremath{\nsubset}},
	sort  ={= 3 1 3},
	description={
		Teilmengenbeziehung:~ \textdots\ \emph{ist keine echte Teilmenge von} \textdots\
	}
}
\newcommand*        {\symsupset}[1][]{\glsSym[#1]{supset}}
\newglossaryentry       {supset}{
	name  ={\ensuremath{\supset}},
	symbol={\ensuremath{\supset}},
	sort  ={= 3 2 1},
	description={
		Teilmengenbeziehung:~ \textdots\ \emph{ist echte Obermenge von} \textdots\
		; Insbesondere kann keine \Gleichheit\ bestehen.
		In der Literatur wird $\supset$ oft im Sinne von $\supseteq$ verwendet.
	}
}
\newcommand*        {\symsupseteq}[1][]{\glsSym[#1]{supseteq}}
\newglossaryentry       {supseteq}{
	name  ={\ensuremath{\supseteq}},
	symbol={\ensuremath{\supseteq}},
	sort  ={= 3 2 2},
	description={
		Teilmengenbeziehung:~ \textdots\ \emph{ist Obermenge von} \textdots\
		; Insbesondere kann \Gleichheit\ bestehen.
	}
}
\newcommand*        {\symnsupset}[1][]{\glsSym[#1]{nsupset}}
\newglossaryentry       {nsupset}{
	name  ={\ensuremath{\nsupset}},
	symbol={\ensuremath{\nsupset}},
	sort  ={= 3 2 3},
	description={
		Teilmengenbeziehung:~ \textdots\ \emph{ist keine echte Obermenge von} \textdots\
	}
}

% Schlussregeln ----------------------------------------------------------------
% \XX    - Ausgabe sowohl im Text- als auch Mathematik-Modus
% \symXX - Ausgabe als Symbol und Eintrag in Symbolliste und Glossar
% \tagXX - wie \symXX, aber ohne Verweis ins Glossar
% Verweise:
%   \ref    {def-XX} -->  \XX
%   \eqref  {def-XX} --> (\XX)
%   \vreffor{def-XX} --> (\XX) auf Seite n

\newcommand*    {\AR}{\ensuremath{\text{AR}}}
\newcommand* {\symAR}[1][]{\glsSym [#1]{AR}}
\newcommand* {\tagAR}[1][]{\glsTag [#1]{AR}}
\newglossaryentry{AR}{
	name      ={(\AR)},
	symbol     ={\AR},
	sort        ={AR},
	description={
		\Anfangsregel\ - Eine \gls{Schlussregel}.
	}
}
\newcommand*    {\FS}{\ensuremath{\text{FS}}}
\newcommand* {\symFS}[1][]{\glsSym [#1]{FS}}
\newcommand* {\tagFS}[1][]{\glsTag [#1]{FS}}
\newglossaryentry{FS}{
	name      ={(\FS)},
	symbol     ={\FS},
	sort        ={FS},
	description={
		\formalerSatz\ - Eine \gls{Schlussregel}.
	}
}
\newcommand*    {\MR}{\ensuremath{\text{MR}}}
\newcommand* {\symMR}[1][]{\glsSym [#1]{MR}}
\newcommand* {\tagMR}[1][]{\glsTag [#1]{MR}}
\newglossaryentry{MR}{
	name      ={(\MR)},
	symbol     ={\MR},
	sort        ={MR},
	description={
		\Monotonieregel\ - Eine \gls{Schlussregel}.
	}
}
\newcommand*    {\SR}{\ensuremath{\text{SR}}}
\newcommand* {\symSR}[1][]{\glsSym [#1]{SR}}
\newcommand* {\tagSR}[1][]{\glsTag [#1]{SR}}
\newglossaryentry{SR}{
	name      ={(\SR)},
	symbol     ={\SR},
	sort        ={SR},
	description={
		\Schnittregel\ - Eine \gls{Schlussregel}.
	}
}
\newcommand*    {\TR}{\ensuremath{\text{TR}}}
\newcommand* {\symTR}[1][]{\glsSym [#1]{TR}}
\newcommand* {\tagTR}[1][]{\glsTag [#1]{TR}}
\newglossaryentry{TR}{
	name      ={(\TR)},
	symbol     ={\TR},
	sort        ={TR},
	description={
		\Abtrennungsregel\ - Eine \gls{Schlussregel}.
	}
}
\newcommand*    {\andB}{\ensuremath{\land\text{B}}}
\newcommand* {\symandB}[1][]{\glsSym   [#1]{andB}}
\newcommand* {\tagandB}[1][]{\glsTag   [#1]{andB}}
\newglossaryentry{andB}{
	name      ={(\andB)},
	symbol     ={\andB},
	sort       ={= 9 1 B},
	description={
		Eine \gls{Schlussregel} - Beseitigung von \chrqt{$\land$}.
	}
}
\newcommand*    {\andE}{\ensuremath{\land\text{E}}}
\newcommand* {\symandE}[1][]{\glsSym   [#1]{andE}}
\newcommand* {\tagandE}[1][]{\glsTag   [#1]{andE}}
\newglossaryentry{andE}{
	name      ={(\andE)},
	symbol     ={\andE},
	sort       ={= 9 1 E},
	description={
		Eine \gls{Schlussregel} - Einführung von \chrqt{$\land$}.
	}
}
%%%\newcommand*    {\orB}{\ensuremath{\lor\text{B}}}
%%%\newcommand* {\symorB}[1][]{\glsSym   [#1]{orB}}
%%%\newcommand* {\tagorB}[1][]{\glsTag   [#1]{orB}}
%%%\newglossaryentry{orB}{
%%%	name      ={(\orB)},
%%%	symbol     ={\orB},
%%%	sort      ={= 9 2 B},
%%%	description={
%%%		Eine \gls{Schlussregel} - Beseitigung von \chrqt{$\lor$}.
%%%	}
%%%}
%%%\newcommand*    {\orE}{\ensuremath{\lor\text{E}}}
%%%\newcommand* {\symorE}[1][]{\glsSym   [#1]{orE}}
%%%\newcommand* {\tagorE}[1][]{\glsTag   [#1]{orE}}
%%%\newglossaryentry{orE}{
%%%	name      ={(\orE)},
%%%	symbol     ={\orE},
%%%	sort      ={= 9 2 E},
%%%	description={
%%%		Eine \gls{Schlussregel} - Einführung von \chrqt{$\lor$}.
%%%	}
%%%}
\newcommand*    {\impB}{\ensuremath{\limp\text{B}}}
\newcommand* {\symimpB}[1][]{\glsSym   [#1]{impB}}
\newcommand* {\tagimpB}[1][]{\glsTag   [#1]{impB}}
\newglossaryentry{impB}{
	name      ={(\impB)},
	symbol     ={\impB},
	sort       ={= 9 3 B},
	description={
		Eine \gls{Schlussregel} - Beseitigung von \chrqt{$\limp$}.
	}
}
\newcommand*    {\impE}{\ensuremath{\limp\text{E}}}
\newcommand* {\symimpE}[1][]{\glsSym   [#1]{impE}}
\newcommand* {\tagimpE}[1][]{\glsTag   [#1]{impE}}
\newglossaryentry{impE}{
	name      ={(\impE)},
	symbol     ={\impE},
	sort       ={= 9 3 E},
	description={
		Eine \gls{Schlussregel} - Einführung von \chrqt{$\limp$}.
	}
}
\newcommand*    {\nota}{\ensuremath{\lnot\text{1}}}
\newcommand* {\symnota}[1][]{\glsSym   [#1]{nota}}
\newcommand* {\tagnota}[1][]{\glsTag   [#1]{nota}}
\newglossaryentry{nota}{
	name      ={(\nota)},
	symbol     ={\nota},
	sort       ={= 9 4 1},
	description={
		Eine \gls{Schlussregel} - Einführung/Beseitigung von \chrqt{$\lnot$} Teil 1.
	}
}
\newcommand*    {\notb}{\ensuremath{\lnot\text{2}}}
\newcommand* {\symnotb}[1][]{\glsSym   [#1]{notb}}
\newcommand* {\tagnotb}[1][]{\glsTag   [#1]{notb}}
\newglossaryentry{notb}{
	name      ={(\notb)},
	symbol     ={\notb},
	sort       ={= 9 4 2},
	description={
		Eine \gls{Schlussregel} - Einführung/Beseitigung von \chrqt{$\lnot$} Teil 2.
	}
}
\newcommand*    {\notc}{\ensuremath{\lnot\text{3}}}
\newcommand* {\symnotc}[1][]{\glsSym   [#1]{notc}}
\newcommand* {\tagnotc}[1][]{\glsTag   [#1]{notc}}
\newglossaryentry{notc}{
	name      ={(\notc)},
	symbol     ={\notc},
	sort       ={= 9 4 3},
	description={
		Eine \gls{Schlussregel} - Beweistechnik \enquote{Indirekter \gls{Beweis}}.
	}
}
\newcommand*    {\notd}{\ensuremath{\lnot\text{4}}}
\newcommand* {\symnotd}[1][]{\glsSym   [#1]{notd}}
\newcommand* {\tagnotd}[1][]{\glsTag   [#1]{notd}}
\newglossaryentry{notd}{
	name      ={(\notd)},
	symbol     ={\notd},
	sort       ={= 9 4 4},
	description={
		Eine \gls{Schlussregel} - Reductio ad absurdum (Indirekter \gls{Beweis}).
	}
}
\newcommand*    {\eqB}{\ensuremath{\eq\text{B}}}
\newcommand* {\symeqB}[1][]{\glsSym  [#1]{eqB}}
\newcommand* {\tageqB}[1][]{\glsTag  [#1]{eqB}}
\newglossaryentry{eqB}{
	name      ={(\eqB)},
	symbol     ={\eqB},
	sort      ={= 9 5 B},
	description={
		Eine \gls{Schlussregel} - Beseitigung von \chrqt{$\eq$}.
	}
}
\newcommand*    {\eqE}{\ensuremath{\eq\text{E}}}
\newcommand* {\symeqE}[1][]{\glsSym  [#1]{eqE}}
\newcommand* {\tageqE}[1][]{\glsTag  [#1]{eqE}}
\newglossaryentry{eqE}{
	name      ={(\eqE)},
	symbol     ={\eqE},
	sort      ={= 9 5 E},
	description={
		Eine \gls{Schlussregel} - Einführung von \chrqt{$\eq$}.
	}
}

% Operationen mit Namen (Buchstaben) -------------------------------------------
% \symXX - Ausgabe als Symbol und Aufnahme in Symbolliste und Glossar

\newcommand*{\finiteLetter}{e}% [e]ndlich

\newcommand*{\DbSymbol}{dom}% Definitionsbereich ([dom]ain) einer Funktion
\newcommand*        {\symDb}[1][]{\glsSym[#1]{Db}}
\newglossaryentry       {Db}{
	name  ={\ensuremath{\Db}},
	symbol={\ensuremath{\Db}},
	sort  ={dom},%      \DbSymbol
	description={
		$\Db(f)$ für $f : A \rightarrow B$ ist die Menge $A$
		\\-- Symbol: $\Db$
	}
}
\newcommand*{\lenSymbol}{len}% Länge ([len]gth) eines Tupels, einer Folge
\newcommand*        {\symlen}[1][]{\glsSym[#1]{len}}
\newglossaryentry       {len}{
	name  ={\ensuremath{\len}},
	symbol={\ensuremath{\len}},
	sort  ={len},%      \lenSymbol
	description={
		$\len(\vec{a})$ ist die Länge, \textdh\ die Anzahl der Komponenten eines Tupels \textbzw\ einer Folge.%TODO Verweis fehlt; Komponente?
		\\-- Symbol: $\len$
	}
}
\newcommand*{\graphSymbol}{graph}% [Graph] von Funktionen und Relationen
\newcommand*        {\symgraph}[1][]{\glsIdx[#1]{graph}}
\newglossaryentry       {graph}{
	name  ={\ensuremath{\graph}},
	plural={\ensuremath{\graph}},
	sort  ={graph},%    \graphSymbol
	description ={
		$(R)$ ist der \gls{Graph} der Funktion \textbzw\ Relation $R$.
		\\-- Zur genaueren Definition \vrefseesub{sub-weitereBezeichnungen}.
	}
}
\newcommand*           {\PotLetter}{P}% [P]otenzmenge
\newcommand*        {\symPot}[1][]{\glsSym[#1]{Pot}}
\newglossaryentry       {Pot}{
	name  ={\ensuremath{\Pot}},
	symbol={\ensuremath{\Pot}},
	sort  ={P},%        \PotLetter
	description={
		\gls{Potenzmenge}.
	}
}
\newcommand*        {\symPotf}[1][]{\glsSym[#1]{Potf}}
\newglossaryentry       {Potf}{
	name  ={\ensuremath{\Potf}},
	symbol={\ensuremath{\Potf}},
	sort  ={P e},%      \PotLetter \finiteLetter
	description={
		Menge der endlichen Teilmengen.
	}
}
%%%\newcommand*{\QbSymbol}{src}% Quellbereich ([s]ou[rc]e) einer partiellen Fkt.
%%%\newcommand*        {\symQb}[1][]{\glsSym[#1]{Qb}}
%%%\newglossaryentry       {Qb}{
%%%	name  ={\ensuremath{\Qb}},
%%%	symbol={\ensuremath{\Qb}},
%%%	sort  ={src},%      \QbSymbol
%%%	description={
%%%		$\Qb(f)$ für $f : A \rightarrow B$ ist die Menge $\{a \in A | f(a) \text{ existiert}}$.
%%%		\\-- Symbol: $\Qb$
%%%	}
%%%}
\newcommand*           {\RelLetter}{R}% Menge der [R]elationen
\newcommand*        {\symRel}[1][]{\glsSym[#1]{Rel}}
\newglossaryentry       {Rel}{
	name  ={\ensuremath{\Rel}},
	symbol={\ensuremath{\Rel}},
	sort  ={R},%        \RelLetter
	description={
		Menge der binären Relationen.
	}
}
\newcommand*        {\symRelf}[1][]{\glsSym[#1]{Relf}}
\newglossaryentry       {Relf}{
	name  ={\ensuremath{\Relf}},
	symbol={\ensuremath{\Relf}},
	sort  ={R e},%      \RelLetter\finiteLetter
	description={
		Menge der endlichen binären Relationen.
	}
}
\newcommand*{\SetSymbol}{Set}% Menge der Komponenten eines Tupels / einer Folge
\newcommand*        {\symSet}[1][]{\glsSym[#1]{Set}}
\newglossaryentry       {Set}{
	name  ={\ensuremath{\Set}},
	symbol={\ensuremath{\Set}},
	sort  ={Set},%      \SetSymbol
	description={
		$\Set(\vec{a})$ ist die Menge der Komponenten eines \Tupel s \textbzw\ einer Folge.%TODO Verweis ins Glossar
		\\-- Symbol: $\Set$
	}
}
\newcommand*    {\stelfuncSymbol}{stel_f}% [Stel]ligkeit für [F]unktionen
\newcommand* {\symstelfunc}[1][]{\glsSym[#1]{stelfunc}}
\newglossaryentry{stelfunc}{
	name        ={\ensuremath{\stelfunc}},
	symbol      ={\ensuremath{\stelfunc}},
	sort        ={stel f},%   \stelfuncSymbol
	description ={
		\gls{Stelligkeit} einer \gls{Funktion}.
		\\-- Symbol: $\stelfunc$
		\\-- Zur genaueren Definition \vrefseesub{sub-weitereBezeichnungen}.
	}
}
\newcommand*    {\stelrelSymbol} {stel_r}% [Stel]ligkeit für [R]elationen
\newcommand* {\symstelrel}[1][]{\glsSym[#1]{stelrel}}
\newglossaryentry{stelrel}{
	name        ={\ensuremath{\stelrel}},
	symbol      ={\ensuremath{\stelrel}},
	sort        ={stel r},%   \stelrelSymbol
	description ={
		\gls{Stelligkeit} einer \gls{Relation}.
		\\-- Symbol: $\stelrel$
		\\-- Zur genaueren Definition \vrefseesub{sub-weitereBezeichnungen}.
	}
}
\newcommand*           {\traegerSymbol}{car}%  ([car]rier) Trägermenge einer Relation
\newcommand*        {\symtraeger}[1][]{\glsSym[#1]{len}}
\newglossaryentry       {traeger}{
	name  ={\ensuremath{\traeger}},
	symbol={\ensuremath{\traeger}},
	sort  ={car},%      \traegerSymbol
	description={
		$\traeger_i(R)$ für $R \subseteq A_1 \times \cdots \times A_n$ ist die \gls{Traegermenge} $A_i$ für $i$ von $1$ bis $n$.
		\\-- Symbol: $\traeger_i$
	}
}
\newcommand*{\ZbSymbol}{tar}% Zielbereich ([tar]get) einer Funktion
\newcommand*        {\symZb}[1][]{\glsSym[#1]{Zb}}
\newglossaryentry       {Zb}{
	name  ={\ensuremath{\Zb}},
	symbol={\ensuremath{\Zb}},
	sort  ={tar},%      \ZbSymbol
	description={
		$\Zb(f)$ für $f : A \rightarrow B$ ist die Menge $B$
		\\-- Symbol: $\Zb$
	}
}
%%%\newcommand*{\WbSymbol}{ran}% Wertebereich ([ran]ge) einer Funktion
%%%\newcommand*        {\symWb}[1][]{\glsSym[#1]{Wb}}
%%%\newglossaryentry       {Wb}{
%%%	name  ={\ensuremath{\Wb}},
%%%	symbol={\ensuremath{\Wb}},
%%%	sort  ={ran},%      \WbSymbol
%%%	description={
%%%		$\Wb(f)$ für $f : A \rightarrow B$ ist die Menge $\{f(a) | a \in A}$.
%%%		\\-- Symbol: $\Wb$
%%%	}
%%%}

% Fachbegriffe #################################################################
% Hilfsmakros:       Glossary-   Index-Eintrag  Textausgabe
%   \glsIdx  {key}   name        name           text
%%  \glsIdxG {key}   name        name           user1
%   \glsIdxD {key}   name        name           user2
%%  \glsIdxA {key}   name        name           user3
%   \glsIdxPl{key}   name        name           plural
%   \GlsIdxPl{key}   name        name           Plural
%%  \glsIdxPG{key}   name        name           user4
%   \glsIdxPD{key}   name        name           user5
%%  \glsIdxPA{key}   name        name           user6

%A === A === A === A === A === A === A === A === A === A === A === A === A === A

\newcommand*{\ASBA}[1][]{\glsIdx  [#1]{ASBA}}
\newacronym{ASBA}{ASBA}{
	Programmsystem, das \textbf{A}xiome, \textbf{S}ätze, \textbf{B}eweise und \textbf{A}uswertungen behandeln kann.
}
\newcommand*    {\ableitbar} [1][]{\glsIdx  [#1]{ableitbar}}
\newcommand*    {\ableitbare}[1][]{\glsIdxPl[#1]{ableitbar}}
\newglossaryentry{ableitbar}{
	name        ={ableitbar},
	plural      ={ableitbare},
	description ={
		Wenn sich eine \gls{Formel} $\beta$ aus einer anderen \gls{Formel} $\alpha$ mittels \glos{zulässiger Transformationen} ableiten lässt, heißt $\beta$ \gls{ableitbar} aus $\alpha$.
		Sprechweise: \seqqt{$ \alpha \text{ ableitbar } \beta $}.
		Eine oder beide \glspl{Formel} $\alpha$ \textbzw\ $\beta$ dürfen dabei durch \glspl{Formelmenge} ersetzt werden.
		\\-- Siehe \gls{Ableitungsrelation} und $\derive$.
		\\-- Synonym: \gls{beweisbar}.
	}
}
\newcommand*    {\Ableitung}  [1][]{\glsIdx  [#1]{Ableitung}}
\newcommand*    {\Ableitungen}[1][]{\glsIdxPl[#1]{Ableitung}}
\newglossaryentry{Ableitung}{
	name        ={Ableitung},
	plural      ={Ableitungen},
	description ={
		Eine \gls{Aussage} $A \derive B$ \textbzw\ allgemeiner $A \derive_R B$.
		Dies entspricht einem Element $(A,B)$ einer \gls{Ableitungsrelation} $\derive$ \textbzw\ $\derive_R$.
		Die semantische Aussage ist, das die \glspl{Formel} aus $B$ aus den \glspl{Formel} aus $A$ abgeleitet werden können.
	}
}
%%%\newcommand*    {\Ableitungsmenge} [1][]{\glsIdx  [#1]{Ableitungsmenge}}
%%%\newcommand*    {\Ableitungsmengen}[1][]{\glsIdxPl[#1]{Ableitungsmenge}}
%%%\newglossaryentry{Ableitungsmenge}{
%%%	name        ={Ableitungsmenge},
%%%	plural      ={Ableitungsmengen},
%%%	description ={
%%%		Eine Menge aus \glspl{Ableitung}, letztlich nichts anderes als eine \gls{Ableitungsrelation}.
%%%	}
%%%}
\newcommand*    {\Ableitungsrelation}  [1][]{\glsIdx  [#1]{Ableitungsrelation}}
\newcommand*    {\Ableitungsrelationen}[1][]{\glsIdxPl[#1]{Ableitungsrelation}}
\newglossaryentry{Ableitungsrelation}{
	name        ={Ableitungsrelation},
	plural      ={Ableitungsrelationen},
	see         ={[siehe auch]{Ableitung}},
	description ={
		Eine binäre \gls{Relation} $\derive$ aus $\deriveSet$.
		Für $R \in \deriveSet$ auch mit $\derive_R$ bezeichnet.
	}
}
\newcommand*    {\Abtrennungsregel}[1][]{\glsIdx  [#1]{Abtrennungsregel}}
\newglossaryentry{Abtrennungsregel}{
	name        ={Abtrennungsregel},
	description ={
		Eine \gls{Schlussregel} -- siehe~\gls{TR}.
	}
}
\newcommand*    {\Aequivalenz}  [1][]{\glsIdx  [#1]{Aequivalenz}}
\newcommand*    {\Aequivalenzen}[1][]{\glsIdxPl[#1]{Aequivalenz}}
\newglossaryentry{Aequivalenz}{
	name        ={Äquivalenz},
	plural      ={Äquivalenzen},
	description ={
		Eine \gls{Gleichheitsrelation}:
		Zwei Objekte $A$ und $B$ sind \emph{äquivalent} (ähnlich), $A \equiv B$, wenn sie in den \glos{interessierenden Eigenschaften} für $\equiv$ übereinstimmen.
		\\-- Zur Definition \vrefseesubsub{subsub-Vergleiche}.
	}
}
\newcommand*    {\Aequivalenzrelation}  [1][]{\glsIdx  [#1]{Aequivalenzrelation}}
\newcommand*    {\Aequivalenzrelationen}[1][]{\glsIdxPl[#1]{Aequivalenzrelation}}
\newglossaryentry{Aequivalenzrelation}{
	name        ={Äquivalenzrelation},
	plural      ={Äquivalenzrelationen},
	description ={
		Eine binäre \gls{Relation} $\sim$ auf einer Menge $M$ mit folgenden Eigenschaften:
		\begin{description}
			\item [reflexiv] ($a \sim a$)
			\item [transitiv] ($((a \sim b) \metaand (b \sim c)) \metaimp (a \sim c)$)
			\item[symmetrisch] ($(a \sim b) \metaimp (b \sim a)$)
		\end{description}
		jeweils für alle Elemente $a$, $b$ und $c$ aus $M$.
		\\-- \vrefSeesubsub{subsub-Vergleiche}.
	}
}
\newcommand*    {\allgemeingueltig}  [1][]{\glsIdx  [#1]{allgemeingueltig}}
\newcommand*    {\allgemeingueltige} [1][]{\glsIdxPl[#1]{allgemeingueltig}}
\newcommand*    {\allgemeingueltigen}[1][]{\glsIdxPl[#1]{allgemeingueltig}n}
\newglossaryentry{allgemeingueltig}{
	name        ={allgemeingültig},
	plural      ={allgemeingültige},
	description ={
		Eine \gls{Schlussregel} heißt \defn{allgemeingültig}, wenn sie aus den \glspl{Basisregel} und schon bekannten \glos{allgemeingültigen} \glspl{Schlussregel} abgeleitet werden kann.
		\\-- Zur Definition \vrefseesub{sub-Schlussregeln}.
	}
}
\newcommand*    {\Anfangsregel}[1][]{\glsIdx  [#1]{Anfangsregel}}
\newglossaryentry{Anfangsregel}{
	name        ={Anfangsregel},
	description ={
		Die \gls{Schlussregel} \gls{AR} um anfangen zu können.
	}
}
\newcommand*    {\atomar} [1][]{\glsIdx  [#1]{atomar}}
\newcommand*    {\atomare}[1][]{\glsIdxPl[#1]{atomar}}
\newglossaryentry{atomar}{
	name        ={atomar},
	plural      ={atomare},
	description ={
		Synonym zu \gls{unzerlegbar}, siehe dort; vergleiche auch \gls{zerlegbar}.
		Das Attribut betrifft \glspl{Aussage} und \glspl{Formel}.
	}
}
\newcommand*    {\Ausgabeschema}  [1][]{\glsIdx  [#1]{Ausgabeschema}}
\newcommand*    {\Ausgabeschemata}[1][]{\glsIdxPl[#1]{Ausgabeschema}}
\newglossaryentry{Ausgabeschema}{
	name        ={Ausgabeschema},
	plural      ={Ausgabeschemata},
	description ={
		Ein Schema, mit dem bestimmte mathematische \glspl{Objekt} ausgegeben werden sollen.
	}
}
\newcommand*    {\Aussage} [1][]{\glsIdx  [#1]{Aussage}}
\newcommand*    {\Aussagen}[1][]{\glsIdxPl[#1]{Aussage}}
\newglossaryentry{Aussage}{
	name        ={Aussage},
	plural      ={Aussagen},
	description ={
		Eine \gls{Aussage} in natürlicher Sprache oder als \gls{Formel}, die einen \gls{Wahrheitswert} liefert.
		\\-- Zur Definition \vrefseesub{sub-AussagenUndMetaoperationen}.
	}
}
\newcommand*    {\Aussagenlogik}[1][]{\glsIdx  [#1]{Aussagenlogik}}
\newglossaryentry{Aussagenlogik}{
	name        ={Aussagenlogik},
	description ={
		-- Zur Definition \vrefseesec{sec-Aussagenlogik}.
	}
}
\newcommand*    {\axiomLetter}{X}%           A[x]iom
\newcommand*    {\Axiom}  [1][]{\glsIdx  [#1]{Axiom}}
\newcommand*    {\Axiome} [1][]{\glsIdxPl[#1]{Axiom}}
\newcommand*    {\Axiomen}[1][]{\glsIdxPl[#1]{Axiom}n}
\newglossaryentry{Axiom}{
	name        ={Axiom},
	plural      ={Axiome},
	description ={
		Eine \gls{Formel}, die unbewiesen als wahr angesehen wird.
		\\-- Standardsymbole:
		$\axiom$    = ein Axiom,
		$\axiomSet$ = eine Menge aus Axiomen
		\\-- Zur Definition \vrefseesub{sub-Schlussregeln} und \vref{sub-ausAxiome}.
	}
}
\newcommand*    {\Axiomensystem} [1][]{\glsIdx  [#1]{Axiomensystem}}
\newcommand*    {\Axiomensysteme}[1][]{\glsIdxPl[#1]{Axiomensystem}}
\newglossaryentry{Axiomensystem}{
	name        ={Axiomensystem},
	plural      ={Axiomensysteme},
	description ={
		Eine Menge aus \glspl{Axiom}.
		\\-- Zur Definition \vrefseesub{sub-Schlussregeln} und \vref{sub-ausAxiome}.
	}
}

%B === B === B === B === B === B === B === B === B === B === B === B === B === B

\newcommand*    {\Basisregel} [1][]{\glsIdx  [#1]{Basisregel}}
\newcommand*    {\Basisregeln}[1][]{\glsIdxPl[#1]{Basisregel}}
\newglossaryentry{Basisregel}{
	name        ={Basisregel},
	plural      ={Basisregeln},
	description ={
		Eine \gls{Schlussregel}, die nicht mehr auf andere zurückgeführt wird.
		Obwohl das auch auf die \glspl{Identitaetsregel} zutrifft, werden diese hier aber nicht dazu gezählt.
		\\-- Zur Definition \vrefseesub{sub-Basisregeln}.
	}
}
\newcommand*    {\beschraenkt}  [1][]{\glsIdx  [#1]{beschraenkt}}
\newcommand*    {\beschraenkte} [1][]{\glsIdxPl[#1]{beschraenkt}}
\newcommand*    {\beschraenkten}[1][]{\glsIdxPl[#1]{beschraenkt}n}
\newglossaryentry{beschraenkt}{
	name        ={beschränkt},
	plural      ={beschränkte},
	description ={
		Eine \gls{Schlussregel} heißt \gls{beschraenkt}, wenn sie nur endlich viele Voraussetzungen und Folgerungen hat.
	}
}
\newcommand*    {\Beweis}  [1][]{\glsIdx  [#1]{Beweis}}
\newcommand*    {\Beweise} [1][]{\glsIdxPl[#1]{Beweis}}
\newcommand*    {\Beweises}[1][]{\glsIdx  [#1]{Beweis}es}
\newcommand*    {\Beweisen}[1][]{\glsIdxPl[#1]{Beweis}n}
\newglossaryentry{Beweis}{
	name        ={Beweis},
	plural      ={Beweise},
	description ={
		Eine zulässige Ableitung von \glspl{Folgerung} aus gegebenen \glspl{Voraussetzung}.
		\\-- \vrefSeesec{sec-BeweiseASBA}.
	}
}
\newcommand*    {\beweisbar} [1][]{\glsIdx  [#1]{beweisbar}}
\newcommand*    {\beweisbare}[1][]{\glsIdxPl[#1]{beweisbar}}
\newglossaryentry{beweisbar}{
	name        ={beweisbar},
	plural      ={beweisbare},
	description ={
		Synonym zu \gls{ableitbar}.
	}
}
\newcommand*{\proofstepLetter}   {b}%                [B]eweisschritt
\newcommand*{\proofstepSetLetter}{B}% Tupel/Folge aus[B]eweisschritten,
\newcommand*    {\Beweisschritt}  [1][]{\glsIdx  [#1]{Beweisschritt}}
\newcommand*    {\Beweisschritte} [1][]{\glsIdxPl[#1]{Beweisschritt}}
\newcommand*    {\Beweisschritten}[1][]{\glsIdxPl[#1]{Beweisschritt}n}
\newglossaryentry{Beweisschritt}{
	name        ={Beweisschritt},
	plural      ={Beweisschritte},
	symbol      ={\proofstepLetter},
	description ={
		Eine Vorschrift, wie aus vorgegebenen \glspl{Aussage} (den \glspl{Voraussetzung}) weitere (die \glspl{Folgerung}) folgen.
		\\-- Standardsymbole:
		$\proofstep$    =    ein            Beweisschritt,
		$\proofstepTup$ =    eine Folge aus Beweisschritten,
		$\proofstepSet$ =    eine Menge aus Beweisschritten
		\\-- Zur Definition \vrefseesub{sub-Beweisschritte}.
	}
}
\newcommand*    {\Beweisschrittfolge} [1][]{\glsIdx  [#1]{Beweisschrittfolge}}
\newcommand*    {\Beweisschrittfolgen}[1][]{\glsIdxPl[#1]{Beweisschrittfolge}}
\newglossaryentry{Beweisschrittfolge}{
	name        ={Beweisschrittfolge},
	plural      ={Beweisschrittfolgen},
	description ={
		Eine Folge aus \glos{Beweisschritten}.
		\\-- Zur Definition \vrefseesub{sub-Beweisschritte}.
	}
}
\newcommand*    {\Beweisschrittmenge} [1][]{\glsIdx  [#1]{Beweisschrittmenge}}
\newcommand*    {\Beweisschrittmengen}[1][]{\glsIdxPl[#1]{Beweisschrittmenge}}
\newglossaryentry{Beweisschrittmenge}{
	name        ={Beweisschrittmenge},
	plural      ={Beweisschrittmengen},
	description ={
		Eine Menge aus \glos{Beweisschritten}, insbesondere die Menge der Glieder einer \gls{Beweisschrittfolge}.
		\\-- Zur Definition \vrefseesub{sub-Beweisschritte}.
	}
}
%TODO entry Signatur definieren
\newcommand*    {\BoolescheSignatur} [1][]{\glsIdx  [#1]{BoolescheSignatur}}
\newcommand*    {\BooleschenSignatur}[1][]{\glsIdxPl[#1]{BoolescheSignatur}}
\newglossaryentry{BoolescheSignatur}{
	name        ={Signatur, Boolesche},
	text        ={Boolesche Signatur},
	plural      ={Boolesche Signatur},
	description ={
		Die \glos{logische Signatur} $\{\lnot, \land, \lor\}$.
	}
}

%D === D === D === D === D === D === D === D === D === D === D === D === D === D

\newcommand*    {\Definition}  [1][]{\glsIdx  [#1]{Definition}}
\newcommand*    {\Definitionen}[1][]{\glsIdxPl[#1]{Definition}}
\newglossaryentry{Definition}{
	name        ={Definition},
	plural      ={Definitionen},
	description ={
		Eine Definition mit Hilfe des Symbols \chrqt{$\defeq$}.
		\seqqt{$A \defeq B$} steht für \enquote{$A$ \emph{ist definitionsgemäß gleich} $B$} für \glspl{Objekt} $A$ und $B$.
		Gewissermaßen ist $A$ nur eine andere Schreibweise für $B$.
		\\-- Man vergleiche auch den Begriff \enquote{\gls{Metadefinition}} und das zugehörige \gls{Symbol} \chrqt{$\metadefeq$}.
		\\-- Zur Definition \vrefseesub{subsub-Definitionen}.
	}
}
\newcommand*    {\Definitionsbereich} [1][]{\glsIdx  [#1]{Definitionsbereich}}
\newcommand*    {\Definitionsbereiche}[1][]{\glsIdxPl[#1]{Definitionsbereich}}
\newglossaryentry{Definitionsbereich}{
	name        ={Definitionsbereich},
	plural      ={Definitionsbereiche},
	description ={
		einer \gls{Funktion}.
		\\-- Symbol: %\DbSymbol%
		\\-- Zur genaueren Definition \vrefseesub{sub-weitereBezeichnungen}.
	}
}

%E === E === E === E === E === E === E === E === E === E === E === E === E === E

\newcommand*{\outcomeLetter}{O}%                 Ergebnis, [o]utcome
\newcommand*    {\Ergebnis}  [1][]{\glsIdx  [#1]{Ergebnis}}
\newcommand*    {\Ergebnisse}[1][]{\glsIdxPl[#1]{Ergebnis}}
\newglossaryentry{Ergebnis}{
	name        ={Ergebnis},
	plural      ={Ergebnisse},
	description ={
		Ein \gls{Ergebnis} eines \glos{Beweises}.
		\\-- Standardsymbole:
		$\outcome$    = ein Ergebnis
		$\outcomeSet$ = eine Menge aus Ergebnissen
		$\outcomeRel$ = eine Relation (als Menge aufgefasst) aus Ergebnissen
		\\-- Zur Definition \vrefseesub{sub-Beweise}.
	}
}
\newcommand*    {\Ergebnismenge} [1][]{\glsIdx  [#1]{Ergebnismenge}}
\newcommand*    {\Ergebnismengen}[1][]{\glsIdxPl[#1]{Ergebnismenge}}
\newglossaryentry{Ergebnismenge}{
	name        ={Ergebnismenge},
	plural      ={Ergebnismengen},
	description ={
		Die Menge der \glspl{Ergebniss} eines \glos{Beweises}.
		\\-- Standardsymbol:
		$\outcomeSet$
		\\-- Zur Definition \vrefseesub{sub-Beweise}.
	}
}

%F === F === F === F === F === F === F === F === F === F === F === F === F === F

\newcommand*    {\Fachbegriff}  [1][]{\glsIdx  [#1]{Fachbegriff}}
\newcommand*    {\Fachbegriffe} [1][]{\glsIdxPl[#1]{Fachbegriff}}
\newcommand*    {\Fachbegriffen}[1][]{\glsIdxPl[#1]{Fachbegriff}n}
\newglossaryentry{Fachbegriff}{
	name        ={Fachbegriff},
	plural      ={Fachbegriffe},
	description ={
		Ein Name für einen mathematischen Begriff.
	}
}
\newcommand*    {\Fachgebiet}  [1][]{\glsIdx  [#1]{Fachgebiet}}
\newcommand*    {\Fachgebiete} [1][]{\glsIdxPl[#1]{Fachgebiet}}
\newcommand*    {\Fachgebieten}[1][]{\glsIdxPl[#1]{Fachgebiet}n}
\newglossaryentry{Fachgebiet}{
	name        ={Fachgebiet},
	plural      ={Fachgebiete},
	description ={
		Ein Teil der Mathematik mit einer zugehörigen Basis aus \glos{Axiomen}, \glos{Sätzen}, \glos{Fachbegriffen} und Darstellungsweisen.
	}
}
\newcommand*{\conclusionLetter}   {f}%           [F]olgerung
\newcommand*{\conclusionSetLetter}{F}%           [F]olgerungen
\newcommand*    {\Folgerung}  [1][]{\glsIdx  [#1]{Folgerung}}
\newcommand*    {\Folgerungen}[1][]{\glsIdxPl[#1]{Folgerung}}
\newglossaryentry{Folgerung}{
	name        ={Folgerung},
	plural      ={Folgerungen},
	description ={
		Die \glspl{Folgerung} einer \gls{Schlussregel} $\frac{\prerequisiteSet}{\conclusionSet}$ sind die Elemente von $\conclusionSet$.
		\\-- Standardsymbole:
		$\conclusion$    = eine Folgerung
		$\conclusionSet$ = eine Menge aus Folgerungen
		$\conclusionRel$ = eine Relation (als Menge aufgefasst) aus Folgerungen
		\\-- Zur Definition \vrefseesub{sub-Schlussregeln}.
	}
}
%%%\newcommand*    {\Folgerungsmenge} [1][]{\glsIdx  [#1]{Folgerungsmenge}}
%%%\newcommand*    {\Folgerungsmengen}[1][]{\glsIdxPl[#1]{Folgerungsmenge}}
%%%\newglossaryentry{Folgerungsmenge}{
%%%	name        ={Folgerungsmenge},
%%%	plural      ={Folgerungsmengen},
%%%	description ={
%%%		Die Menge der \glspl{Folgerung} einer \gls{Schlussregel} \textbzw\ eines \glos{Beweises}.
%%%		\\-- Standardsymbol:
%%%		$\conclusionSet$
%%%		\\-- Zur Definition \vrefseesub{:Schlussregeln}.
%%%	}
%%%}
\newcommand*    {\formalerSatz} [1][]{\glsIdx  [#1]{formalerSatz}}
\newcommand*    {\formalenSatz} [1][]{\glsIdxPl[#1]{formalerSatz}}
\newglossaryentry{formalerSatz}{
	name        ={Satz, formal},
	text        ={formaler Satz},
	plural      ={formalen Satz},% Akkusativ
	description ={
		Formale Darstellung eines mathematischen \glos{Satzes}.
		\\-- Siehe~\gls{FS}; zur Definition \vrefseesub{sub-Schlussregeln}.
	}
}
\newcommand*    {\Formel} [1][]{\glsIdx  [#1]{Formel}}
\newcommand*    {\Formeln}[1][]{\glsIdxPl[#1]{Formel}}
\newglossaryentry{Formel}{
	name        ={Formel},
	plural      ={Formeln},
	description ={
		Unter einer \gls{Formel} verstehen wir stets eine mathematische \gls{Formel}.
		Diese kann aus einem einzigen \gls{Symbol} bestehen (\glos{atomare Formel}), andererseits aber auch mehrdimensional sein, lässt sich dann aber mittels geeigneter \glspl{Definition} immer eindeutig als eine \gls{Zeichenfolge} schreiben.
		\glspl{Satz}, \glspl{Beweis} und \glspl{Schlussregel} betrachten wir \emph{nicht} als \glspl{Formel}.
		\\-- Zur Definition \vrefseesub{sub-Bezeichnungen}
		\\-- Zur Definition \vrefseesubsub{subsub-Formeln}.
	}
}
\newcommand*    {\Formelmenge} [1][]{\glsIdx  [#1]{Formelmenge}}
\newcommand*    {\Formelmengen}[1][]{\glsIdxPl[#1]{Formelmenge}}
\newglossaryentry{Formelmenge}{
	name        ={Formelmenge},
	plural      ={Formelmengen},
	description ={
		Eine Menge aus \glspl{Formel}, oft mit \glssymbol{formulaSet} bezeichnet.
		Man nennt \glssymbol{formulaSet} auch eine \gls{Sprache} und ihre Elemente \glspl{Wort}, insbesondere dann, wenn es eindeutige Regeln zur Konstruktion von \glssymbol{formulaSet} gibt.
		Wir bevorzugen \enquote{\gls{Formel}} und \enquote{\gls{Formelmenge}}.
	}
}
\newcommand*    {\Funktion}  [1][]{\glsIdx  [#1]{Funktion}}
\newcommand*    {\Funktionen}[1][]{\glsIdxPl[#1]{Funktion}}
\newglossaryentry{Funktion}{
	name        ={Funktion},
	plural      ={Funktionen},
	description ={
		Eine \defn{$n$-stellige Funktion} $f$ von einer Menge $A = A_1 \times \dots \times A_n$, dem \gls{Definitionsbereich}, in eine Menge $B$, den \gls{Zielbereich}, ist eine ($n$+1)-stellige \gls{Relation} $(G,A_1,\dots,A_n,B)$ derart, dass es für jedes $\vec{a} = (a_1,\dots,a_n)$ mit $a_i \in A_i$ genau ein $b \in B$ gibt mit $(a_1,\dots,a_n,b) \in f$.
		Dieses $b$ wird auch mit \seqqt{$f(a_1,\dots,a_n)$} , \seqqt{$f a_1 \dots a_n$} , \seqqt{$f(\vec{a})$} oder \seqqt{$f\vec{a}$} bezeichnet.
		\\Schreibweise: \seqqt{$f : A \rightarrow B$} \textbzw\ \seqqt{$f : A_1 \times \dots \times A_n \rightarrow B$}
		\\-- Zur Definition \vrefseesec{sub-weitereBezeichnungen}.
	}
}
\newcommand*    {\Funktionswert} [1][]{\glsIdx  [#1]{Funktionswert}}
\newcommand*    {\Funktionswerte}[1][]{\glsIdxPl[#1]{Funktionswert}}
\newglossaryentry{Funktionswert}{
	name        ={Funktionswert},
	plural      ={Funktionswerte},
	description ={
		einer \gls{Funktion}.
		\\-- Zur genaueren Definition \vrefseesub{sub-weitereBezeichnungen}.
	}
}

%G === G === G === G === G === G === G === G === G === G === G === G === G === G

\newcommand*    {\Gleichheit}[1][]{\glsIdx  [#1]{Gleichheit}}
\newglossaryentry{Gleichheit}{
	name        ={Gleichheit},
	description ={
		Eine \gls{Gleichheitsrelation}:
		Zwei Objekte $A$ und $B$ sind \emph{gleich} (dasselbe; identisch), $A \eq B$, wenn sie in den \glos{interessierenden Eigenschaften} für $\eq$ übereinstimmen.
		\\-- Zur Definition \vrefseesubsub{subsub-Vergleiche}
	}
}
\newcommand*    {\Gleichheitsrelation}  [1][]{\glsIdx  [#1]{Gleichheitsrelation}}
\newcommand*    {\Gleichheitsrelationen}[1][]{\glsIdxPl[#1]{Gleichheitsrelation}}
\newglossaryentry{Gleichheitsrelation}{
	name        ={Gleichheitsrelation},
	plural      ={Gleichheitsrelationen},
	description ={
		Eine mit \gls{Gleichheit} verwandte \gls{Relation}: $\eq$, $\ne$, $\equiv$ und $\nequiv$.
	}
}
\newcommand*    {\Graph}  [1][]{\glsIdx  [#1]{Graph}}
\newcommand*    {\Graphen}[1][]{\glsIdxPl[#1]{Graph}}
\newglossaryentry{Graph}{
	name        ={Graph},
	plural      ={Graphen},
	description ={
		einer \gls{Funktion} oder \gls{Relation}.
		\\-- Symbol: $\symgraph$
		\\-- Zur genaueren Definition \vrefseesub{sub-weitereBezeichnungen}.
	}
}

%I === I === I === I === I === I === I === I === I === I === I === I === I === I

\newcommand*    {\Identitaetsregel} [1][]{\glsIdx  [#1]{Identitaetsregel}}
\newcommand*    {\Identitaetsregeln}[1][]{\glsIdxPl[#1]{Identitaetsregel}}
\newglossaryentry{Identitaetsregel}{
	name        ={Identitätsregel},
	plural      ={Identitätsregeln},
	description ={
		Eigentlich eine \gls{Basisregel} zur Identität.
		Da die \glspl{Identitaetsregel} nur zur Rechtfertigung der \gls{Substitution} verwendet werden, werden sie hier nicht zu den \glspl{Basisregel} gezählt.
		\\-- Zur Definition \vrefseesub{sub-Identitaetsregeln}.
	}
}
\newcommand*    {\interessierendeEigenschaft}   [1][]{\glsIdx  [#1]{interessierendeEigenschaft}}
\newcommand*    {\interessierendenEigenschaft}  [1][]{\glsIdxD [#1]{interessierendeEigenschaft}}
\newcommand*    {\interessierendenEigenschaften}[1][]{\glsIdxPl[#1]{interessierendeEigenschaft}}
\newglossaryentry{interessierendeEigenschaft}{
	name        ={Eigenschaft, interessierende},
	text        ={interessierende Eigenschaft},
	user2       ={interessierenden Eigenschaft},%   Dativ
	user5       ={interessierenden Eigenschaften},% Dativ Plural
	description ={
		Solche Eigenschaften von \glos{Objekten}, die im aktuellen Zusammenhang von Interesse sind, \textzB\ einen bestimmten Wert zu haben, Element einer bestimmten Menge zu sein, ein bestimmtes \gls{Objekt} zu bezeichnen, usw.
	}
}

%J === J === J === J === J === J === J === J === J === J === J === J === J === J

\newcommand*    {\Junktor}  [1][]{\glsIdx  [#1]{Junktor}}
\newcommand*    {\Junktoren}[1][]{\glsIdxPl[#1]{Junktor}}
\newglossaryentry{Junktor}{
	name        ={Junktor},
	plural      ={Junktoren},
	description ={
		Eine aussagenlogische \gls{Operation}.
		Da die Werte einer aussagenlogischen \gls{Operation} \glspl{Wahrheitswert} sind, kann man einen \gls{Junktor} auch als \gls{Relation} verstehen.
		\\-- Zur Definition \vrefseesub{sub-weitereBezeichnungen}
		\\-- Zur Definition \vrefseesub{sub-ausJunktorDef}.
	}
}
\newcommand*    {\Junktorsymbol} [1][]{\glsIdx  [#1]{Junktorsymbol}}
\newcommand*    {\Junktorsymbole}[1][]{\glsIdxPl[#1]{Junktorsymbol}}
\newglossaryentry{Junktorsymbol}{
	name        ={Junktorsymbol},
	plural      ={Junktorsymbole},
	description ={
		Ein \gls{Symbol} für einen \gls{Junktor}.%
		\footnote{%
			Entsprechend \emph{Funktionssymbol}, \emph{Operatorsymbol}, \emph{Relationssymbol}, usw.
		}
	}
}

%K === K === K === K === K === K === K === K === K === K === K === K === K === K

\newcommand*    {\Kontraposition}[1][]{\glsIdx  [#1]{Kontraposition}}
\newglossaryentry{Kontraposition}{
	name        ={Kontraposition},
	description ={
		Die allgemeingültige \gls{Aussage}: $ (\alpha \limp \beta) \limp (\lnot\beta \limp \lnot\alpha) $.
	}
}
\newcommand*    {\Kontravalenz}[1][]{\glsIdx  [#1]{Kontravalenz}}
\newglossaryentry{Kontravalenz}{
	name        ={Kontravalenz},
	description ={
		Eine \gls{Gleichheitsrelation}:
		Zwei Objekte $A$ und $B$ sind \emph{nicht äquivalent} (nicht ähnlich), $A \nequiv B$, wenn sie in mindestens einer \glos{interessierenden Eigenschaft} für $\equiv$ nicht übereinstimmen.
		\\-- Zur Definition \vrefseesubsub{subsub-Vergleiche}.
	}
}

%L === L === L === L === L === L === L === L === L === L === L === L === L === L

\newcommand*    {\logischeSignatur}  [1][]{\glsIdx  [#1]{logischeSignatur}}
\newcommand*    {\logischenSignatur} [1][]{\glsIdxD [#1]{logischeSignatur}}
\newcommand*    {\logischeSignaturen}[1][]{\glsIdxPl[#1]{logischeSignatur}}
\newglossaryentry{logischeSignatur}{
	name        ={Signatur, logische},
	text        ={logische Signatur},
	user2       ={logischen Signatur},% Dativ
	plural      ={logische Signaturen},
	description ={
		Eine Teilmenge von $\alJun$, ausreichend um damit alle anderen Elemente aus $\alJun$ zu definieren.
	}
}

%M === M === M === M === M === M === M === M === M === M === M === M === M === M

\newcommand*    {\Mengenlehre}[1][]{\glsIdx  [#1]{Mengenlehre}}
\newglossaryentry{Mengenlehre}{
	name={Mengenlehre},
	description ={
		-- Zur Definition \vrefseesec{sec-Mengenlehre}.
	}
}
\newcommand*    {\Metadefinition}  [1][]{\glsIdx  [#1]{Metadefinition}}
\newcommand*    {\Metadefinitionen}[1][]{\glsIdxPl[#1]{Metadefinition}}
\newglossaryentry{Metadefinition}{
	name        ={Metadefinition},
	plural      ={Metadefinitionen},
	description ={
		Eine \gls{Definition} in \gls{Metasprache} mit Hilfe des \emph{Metadefinitionssymbols} \chrqt{$\metadefeq$}.
		\seqqt{$A \metadefeq B$} steht für \enquote{$A$ \emph{ist definitionsgemäß äquivalent zu} $B$} für \glspl{Aussage} $A$ und $B$.
		Gewissermaßen ist $A$ nur eine andere Schreibweise für $B$.
		\\-- Man vergleiche auch den Begriff \enquote{\gls{Definition}} und das zugehörige \gls{Symbol} \chrqt{$\defeq$}.
		\\-- Zur Definition \vrefseesubsub{subsub-Definitionen}.
	}
}
\newcommand*    {\Metaoperation}  [1][]{\glsIdx  [#1]{Metaoperation}}
\newcommand*    {\Metaoperationen}[1][]{\glsIdxPl[#1]{Metaoperation}}
\newglossaryentry{Metaoperation}{
	name        ={Metaoperation},
	plural      ={Metaoperationen},
	description ={
		Eine \gls{Operation} der \gls{Metasprache}: $\metaand$, $\metaor$ oder $\srand$.
		\\-- Zur Definition \vrefseesub{sub-AussagenUndMetaoperationen}.
	}
}
\newcommand*    {\Metarelation}  [1][]{\glsIdx  [#1]{Metarelation}}
\newcommand*    {\Metarelationen}[1][]{\glsIdxPl[#1]{Metarelation}}
\newglossaryentry{Metarelation}{
	name        ={Metarelation},
	plural      ={Metarelationen},
	description ={
		Eine \gls{Relation} der \gls{Metasprache}: $\metaimp$, $\metarep$ oder $\metaequiv$.
		\\-- Zur Definition \vrefseesub{sub-AussagenUndMetaoperationen}.
	}
}
\newcommand*    {\Metasprache} [1][]{\glsIdx  [#1]{Metasprache}}
\newcommand*    {\Metasprachen}[1][]{\glsIdxPl[#1]{Metasprache}}
\newglossaryentry{Metasprache}{
	name        ={Metasprache},
	plural      ={Metasprachen},
	description ={
		Eine Sprache, in der \glspl{Aussage} über Elemente einer anderen Sprache getroffen werden können.
		In diesem Dokument ist dies immer die normale Sprache.
		\\-- \vrefSeesec{sec-Metasprache}.
	}
}
\newcommand*    {\Monotonieregel}[1][]{\glsIdx  [#1]{Monotonieregel}}
\newglossaryentry{Monotonieregel}{
	name        ={Monotonieregel},
	description ={
		Eine \gls{Schlussregel}. -- siehe~\gls{MR}.
	}
}

%O === O === O === O === O === O === O === O === O === O === O === O === O === O

\newcommand*    {\Objekt} [1][]{\glsIdx  [#1]{Objekt}}
\newcommand*    {\Objekte}[1][]{\glsIdxPl[#1]{Objekt}}
\newcommand*    {\Objekts}[1][]{\glsIdx  [#1]{Objekt}s}
\newglossaryentry{Objekt}{
	name        ={Objekt},
	plural      ={Objekte},
	description ={
		\glspl{Symbol}, \glspl{Formel} und \glspl{Aussage} sowie Mengen, \glspl{Zeichenfolge}, Zahlen; ganz allgemein reale oder gedachte Dinge an sich.
		\\-- Zur Definition \vrefseesub{sub-Bezeichnungen}.
	}
}
\newcommand*    {\Operation}  [1][]{\glsIdx  [#1]{Operation}}
\newcommand*    {\Operationen}[1][]{\glsIdxPl[#1]{Operation}}
\newglossaryentry{Operation}{
	name        ={Operation},
	plural      ={Operationen},
	description ={
		Eine -- meistens binäre, \textdh\ zweiwertige -- Funktion $M^n \rightarrow M$.
		Für eine binäre \gls{Operation} $\opbsp : M \times M \rightarrow M$ schreibt man meistens $x \opbsp y$ statt $\opbsp(x,y)$.
		\\-- Zur Definition \vrefseesub{sub-weitereBezeichnungen}
		\\-- \vrefSeesub{sub-Beispielsymbole} und \vref{sub-Operationen}.
	}
}
\newcommand*    {\Operationssymbol} [1][]{\glsIdx  [#1]{Operationssymbol}}
\newcommand*    {\Operationssymbole}[1][]{\glsIdxPl[#1]{Operationssymbol}}
\newglossaryentry{Operationssymbol}{
	name        ={Operationssymbol},
	plural      ={Operationssymbole},
	description ={
		Ein \gls{Symbol} für eine \gls{Operation}.
	}
}

%P === P === P === P === P === P === P === P === P === P === P === P === P === P

\newcommand*    {\PolnischeNotation}  [1][]{\glsIdx  [#1]{PolnischeNotation}}
\newcommand*    {\PolnischenNotation} [1][]{\glsIdxD [#1]{PolnischeNotation}}
\newcommand*    {\PolnischeNotationen}[1][]{\glsIdxPl[#1]{PolnischeNotation}}
\newglossaryentry{PolnischeNotation}{
	name        ={Notation, Polnische},
	text        ={Polnische Notation},
	user2       ={Polnischen Notation},% Dativ
	plural      ={Polnische Notationen},
	description ={
		Bei der \glos{Polnischen Notation} stehen die Operanden \textbzw\ Argumente von \glspl{Relation} und \glspl{Funktion} stets rechts von den Relations- und Funktionssymbolen.
		Dadurch kann auf Gliederungszeichen wie Klammern und Kommata verzichtet werden.
		Noch einfacher für Computer ist die \defn{umgekehrte} \glos{Polnische Notation}, bei der die Operanden und Argumente links von den Symbolen stehen.
	}
}
\newcommand*    {\Potenzmenge} [1][]{\glsIdx  [#1]{Potenzmenge}}
\newcommand*    {\Potenzmengen}[1][]{\glsIdxPl[#1]{Potenzmenge}}
\newglossaryentry{Potenzmenge}{
	name        ={Potenzmenge},
	plural      ={Potenzmengen},
	description ={
		Die \gls{Potenzmenge} $\Pot(M)$ einer Menge $M$ ist die Menge ihrer Teilmengen.
		\\-- Zur Definition \vrefseesub{sub-Bezeichnungen}.
	}
}
\newcommand*    {\Praedikat} [1][]{\glsIdx  [#1]{Praedikat}}
\newcommand*    {\Praedikate}[1][]{\glsIdxPl[#1]{Praedikat}}
\newglossaryentry{Praedikat}{
	name        ={Prädikat},
	plural      ={Prädikate},
	description ={
		Ein Element der \gls{Praedikatenlogik}.
		\\-- Zur Definition \vrefseesec{sec-Praedikatenlogik}.
		\\\textZB\ kann man eine Gruppe als ein zweistelliges \gls{Praedikat} $\mathrm{Gruppe}(G,+)$ definieren, in dem $G$ eine Menge und $+$ eine \gls{Operation}, \textdh\ eine binäre (zweistellige) Funktion $ +: G \times G \rightarrow G $ ist, so dass die Gruppenaxiome erfüllt sind.
	}
}
\newcommand*    {\Praedikatenlogik}[1][]{\glsIdx  [#1]{Praedikatenlogik}}
\newglossaryentry{Praedikatenlogik}{
	name={Prädikatenlogik},
	description ={
		-- Zur Definition \vrefseesec{sec-Praedikatenlogik}.
	}
}

%R === R === R === R === R === R === R === R === R === R === R === R === R === R

\newcommand*    {\Relation}  [1][]{\glsIdx  [#1]{Relation}}
\newcommand*    {\Relationen}[1][]{\glsIdxPl[#1]{Relation}}
\newglossaryentry{Relation}{
	name        ={Relation},
	plural      ={Relationen},
	description ={
		Eine \defn{$n$-stellige} \gls{Relation} $R$ ist ein (1+$n$)-\gls{Tupel} $(G,A_1,\dots,A_n$) mit $G \subseteq A_1 \times \dots \times A_n)$.
		\\-- Zur genaueren Definition \vrefseesub{sub-weitereBezeichnungen}
 		\\-- \vrefSeesub{sub-Beispielsymbole} und \vref{sub-Gleichheit}.
	}
}

%S === S === S === S === S === S === S === S === S === S === S === S === S === S

\newcommand*    {\Satz}   [1][]{\glsIdx  [#1]{Satz}}
\newcommand*    {\Saetze} [1][]{\glsIdxPl[#1]{Satz}}
\newcommand*    {\Satzes} [1][]{\glsIdx  [#1]{Satz}e}
\newcommand*    {\Saetzen}[1][]{\glsIdxPl[#1]{Satz}n}
\newglossaryentry{Satz}{
	name        ={Satz},
	plural      ={Sätze},
	description ={
		Eine mathematische \gls{Aussage}, dass bestimmte \glspl{Folgerung} aus gegebenen \glspl{Voraussetzung} abgeleitet werden können.
	}
}
\newcommand*{\conclusionruleLetter}{C}%             Schlussregel, [c]onclusion
\newcommand*    {\Schlussregel} [1][]{\glsIdx  [#1]{Schlussregel}}
\newcommand*    {\Schlussregeln}[1][]{\glsIdxPl[#1]{Schlussregel}}
\newglossaryentry{Schlussregel}{
	name        ={Schlussregel},
	plural      ={Schlussregeln},
	see         ={allgemeingueltig},
	description ={
		Eine \gls{Schlussregel} $\frac{\prerequisiteSet}{\conclusionSet}$ entspricht der \gls{Aussage}:
		\begin{quote}
			Wenn alle \glspl{Voraussetzung} $\prerequisite$ aus $\prerequisiteSet$ zutreffen, dann auch alle \glspl{Folgerung} $\conclusion$ aus $\conclusionSet$.
		\end{quote}
		Wenn diese \gls{Aussage} zutrifft, kann die Schlussregel zur \glos{zulässigen Transformation} von \glspl{Formel} dienen.
		\\-- Standardsymbole:
		$\conclusionrule$    = eine Schlussregel
		$\conclusionruleSet$ = eine Menge aus Schlussregeln
		\\-- Zur Definition \vrefseesub{sub-Schlussregeln}.
	}
}
\newcommand*    {\Schlussregelmenge} [1][]{\glsIdx  [#1]{Schlussregelmenge}}
\newcommand*    {\Schlussregelmengen}[1][]{\glsIdxPl[#1]{Schlussregelmenge}}
\newglossaryentry{Schlussregelmenge}{
	name        ={Schlussregelmenge},
	plural      ={Schlussregelmengen},
	description ={
		Eine Menge aus \glspl{Schlussregel}, meistens mit $\conclusionruleSet$ bezeichnet.
		\\-- Zur Definition \vrefseesub{:Schlussregeln}.
	}
}
\newcommand*    {\Schnittregel}[1][]{\glsIdx  [#1]{Schnittregel}}
\newglossaryentry{Schnittregel}{
	name        ={Schnittregel},
	plural      ={Schnittregeln},
	description ={
		Eine \glos{allgemeingültige Schlussregel}.
		\\-- Siehe~\gls{SR}.
	}
}
\newcommand*    {\Sprache} [1][]{\glsIdx  [#1]{Sprache}}
\newcommand*    {\Sprachen}[1][]{\glsIdxPl[#1]{Sprache}}
\newglossaryentry{Sprache}{
	name        ={Sprache},
	plural      ={Sprachen},
	description ={
		-- Siehe \gls{Formelmenge}.
	}
}
\newcommand*    {\Stelligkeit}  [1][]{\glsIdx  [#1]{Stelligkeit}}
\newcommand*    {\Stelligkeiten}[1][]{\glsIdxPl[#1]{Stelligkeit}}
\newglossaryentry{Stelligkeit}{
	name        ={Stelligkeit},
	plural      ={Stelligkeiten},
	description ={
		einer \gls{Funktion} oder \gls{Relation}.
		\\-- Symbole:
		$\stelfunc$ = Stelligkeit einer Funktion,
		$\stelrel$  = Stelligkeit einer Relation,
		\\-- Zur genaueren Definition \vrefseesub{sub-weitereBezeichnungen}.
	}
}
\newcommand*    {\substitutionLetter}{E}%            Substitution, [E]rsetzung
\newcommand*    {\Substitution}  [1][]{\glsIdx  [#1]{Substitution}}
\newcommand*    {\Substitutionen}[1][]{\glsIdxPl[#1]{Substitution}}
\newglossaryentry{Substitution}{
	name        ={Substitution},
	plural      ={Substitutionen},
	description ={
		Eine \gls{Funktion} zur \gls{Transformation} einer \gls{Formel} mittels \gls{Substitution} in eine gleichwertige.
		Die \gls{Substitution} heißt \gls{zulaessig}, wenn sie vorgegebene Regeln erfüllt.
		\\-- Zur Definition \vrefseesub{sub-Beweise}.
	}
}
\newcommand*    {\Substitutionsmenge} [1][]{\glsIdx  [#1]{Substitutionsmenge}}
\newcommand*    {\Substitutionsmengen}[1][]{\glsIdxPl[#1]{Substitutionsmenge}}
\newglossaryentry{Substitutionsmenge}{
	name        ={Substitutionsmenge},
	plural      ={Substitutionsmengen},
	description ={
		Eine Menge aus \glspl{Substitution}, meistens mit $\substitutionSet$ bezeichnet.
	}
}
\newcommand*    {\Symbol}  [1][]{\glsIdx  [#1]{Symbol}}
\newcommand*    {\Symbole} [1][]{\glsIdxPl[#1]{Symbol}}
\newcommand*    {\Symbols} [1][]{\glsIdx  [#1]{Symbol}s}
\newcommand*    {\Symbolen}[1][]{\glsIdxPl[#1]{Symbol}n}
\newglossaryentry{Symbol}{
	name        ={Symbol},
	plural      ={Symbole},
	description ={
		Ein \defn{einfaches} \gls{Symbol} ist ein druckbares typographisches Zeichen.
		Ein \defn{zusammengesetztes} \gls{Symbol} besteht aus mehreren einfachen \glspl{Symbol}.
		In beiden Fällen wird ein \gls{Symbol} als \gls{unzerlegbar} angesehen.
		\\-- Zur Definition \vrefseesec{sec-Notationen}.
	}
}

%T === T === T === T === T === T === T === T === T === T === T === T === T === T

\newcommand*    {\Traegermenge} [1][]{\glsIdx  [#1]{Traegermenge}}
\newcommand*    {\Traegermengen}[1][]{\glsIdxPl[#1]{Traegermenge}}
\newglossaryentry{Traegermenge}{
	name        ={Trägermenge},
	plural      ={Trägermengen},
	description ={
		einer \gls{Relation}.
		\\-- Symbol: $\traeger$
		\\-- Zur genaueren Definition \vrefseesub{sub-weitereBezeichnungen}.
	}
}
\newcommand*        {\transformationLetter}{T}%           [T]ransformation
\newcommand*        {\Transformation}  [1][]{\glsIdx  [#1]{Transformation}}
\newcommand*        {\Transformationen}[1][]{\glsIdxPl[#1]{Transformation}}
\newglossaryentry{Transformation}{
	name            ={Transformation},
	plural          ={Transformationen},
	description     ={
		Eine Umformung oder Erzeugung einer \gls{Formel} aus einer vorgegebenen Menge aus \glspl{Formel}, \textdh\ die Anwendung einer \gls{Schlussregel}.
		\glspar
		Eine \gls{Transformation} heißt \defn{zulässig}, wenn sie Element einer vorgegebenen Menge aus \glspl{Transformation} oder eine daraus zulässigerweise abgeleitete \gls{Transformation} ist.
		\glspar
		Standardsymbole:
		$\transformation$    = eine Transformation,
		$\transformationTup$ = eine Folge aus Transformationen
	}
}
\newcommand*    {\Transformationsfolge} [1][]{\glsIdx  [#1]{Transformationsfolge}}
\newcommand*    {\Transformationsfolgen}[1][]{\glsIdxPl[#1]{Transformationsfolge}}
\newglossaryentry{Transformationsfolge}{
	name        ={Transformationsfolge},
	plural      ={Transformationsfolgen},
	description ={
		Eine Folge aus \glspl{Transformation}.
		\\-- Standardsymbol: $\transformationTup$
		\\-- Zur Definition \vrefseesub{sub-Beweisschritte}.
	}
}
\newcommand*    {\Tupel} [1][]{\glsIdx  [#1]{Tupel}}
\newglossaryentry{Tupel}{
	name        ={Tupel},
	plural      ={Tupel},
	description ={
		Ein $n$-\gls{Tupel}\alternativ{Vektor} $\vec{a}$ ist eine endliche Folge\alternativ{Sequenz} $(a_1, \dots, a_n)$ \defn{aus} seinen \defn{Komponenten} $a_i$.
		Sind alle Komponenten Elemente einer Menge $M$, so heißt $\vec{a}$ ein $n$-\gls{Tupel} \defn{auf} $M$.
		\\-- Zur Definition \vrefseesub{sub-weitereBezeichnungen}.
	}
}
\newcommand*    {\Tupelmenge} [1][]{\glsIdx  [#1]{Tupelmenge}}
\newcommand*    {\Tupelmengen}[1][]{\glsIdxPl[#1]{Tupelmenge}}
\newglossaryentry{Tupelmenge}{
	name        ={Tupelmenge},
	plural      ={Tupelmengen},
	description ={
		Die \gls{Tupelmenge} $\tupelSet(M)$ einer Menge $M$ ist die Menge aller $n$-Tupel aus $M^n$ für alle $n \in \INo$.
		\\-- Zur Definition \vrefseesub{sub-Bezeichnungen}.
	}
}

%U === U === U === U === U === U === U === U === U === U === U === U === U === U

\newcommand*    {\Umkehrrelation}  [1][]{\glsIdx  [#1]{Umkehrrelation}}
\newcommand*    {\Umkehrrelationen}[1][]{\glsIdxPl[#1]{Umkehrrelation}}
\newglossaryentry{Umkehrrelation}{
	name        ={Umkehrrelation},
	plural      ={Umkehrrelationen},
	description ={
		Die \gls{Umkehrrelation} zu einer binären \gls{Relation} $(G,A,B)$ ist die \gls{Relation} $(H,B,A)$ mit $H = \{(b,a)|(a,b) \in G\}$.
		Üblicherweise wird das zugehörige Relationssymbol gespiegelt.
	}
}
\newcommand*    {\Ungleichheit}[1][]{\glsIdx  [#1]{Ungleichheit}}
\newglossaryentry{Ungleichheit}{
	name        ={Ungleichheit},
	description ={
		Eine \gls{Gleichheitsrelation}:
		Zwei Objekte $A$ und $B$ sind \emph{nicht gleich} (nicht dasselbe; nicht identisch), $A \ne B$, wenn sie in mindestens einer \glos{interessierenden Eigenschaft} für $\eq$ nicht übereinstimmen.
		\\-- Zur Definition \vrefseesubsub{subsub-Vergleiche}.
	}
}
\newcommand*    {\unzerlegbar} [1][]{\glsIdx  [#1]{unzerlegbar}}
\newcommand*    {\unzerlegbare}[1][]{\glsIdxPl[#1]{unzerlegbar}}
\newglossaryentry{unzerlegbar}{
	name        ={unzerlegbar},
	plural      ={unzerlegbare},
	description ={
		Eine \gls{Aussage}, die keine \gls{Metaoperation}, \textbzw\ eine \gls{Formel}, die keine \gls{Operation} und keine \gls{Relation} enthält, heißt \defn{unzerlegbar}.
		\\-- Synonym: \gls{atomar}; vergleiche auch \gls{zerlegbar}.
	}
}

%V === V === V === V === V === V === V === V === V === V === V === V === V === V

\newcommand*    {\vergleichbar} [1][]{\glsIdx  [#1]{vergleichbar}}
\newcommand*    {\vergleichbare}[1][]{\glsIdxPl[#1]{vergleichbar}}
\newglossaryentry{vergleichbar}{
	name        ={vergleichbar},
	plural      ={vergleichbare},
	description ={
		Zwei \glspl{Objekt} $A$ und $B$ sind \gls{vergleichbar}, wenn beide von derselben Art sind, \textdh\ wenn beide \textzB\ jeweils Mengen, \glspl{Zeichenfolge}, Zahlen, \textusw\ sind.
		Dabei muss bei \glspl{Formel} zwischen der \gls{Formel} an sich und ihrem \emph{Wert} oder \emph{Ergebnis} unterschieden werden.
		\\-- Zur Definition \vrefseesub{subsub-Vergleichbar}.
	}
}
\newcommand*    {\Vertauschung}  [1][]{\glsIdx  [#1]{Vertauschung}}
\newcommand*    {\Vertauschungen}[1][]{\glsIdxPl[#1]{Vertauschung}}
\newglossaryentry{Vertauschung}{
	name        ={Vertauschung},
	plural      ={Vertauschungen},
	description ={
		Die \emph{Vertauschung} von zwei unabhängigen Teil-\glspl{Formel} ($\alpha$ und $\beta$) in einer anderen \gls{Formel} ($\gamma$)
		\\-- Formal: $\gamma(\alpha\swap\beta)$.
		Die \emph{Vertauschung} ist eine spezielle Form der \gls{Substitution}.
		\\-- Zur Definition siehe~\eqref{def-Vertauschung} \vrefinsub{sub-Identitaetsregeln}.
	}
}
\newcommand*{\prerequisiteLetter}   {v}%             [V]oraussetzung
\newcommand*{\prerequisiteSetLetter}{V}%             [V]oraussetzungen
\newcommand*    {\Voraussetzung}  [1][]{\glsIdx  [#1]{Voraussetzung}}
\newcommand*    {\Voraussetzungen}[1][]{\glsIdxPl[#1]{Voraussetzung}}
\newglossaryentry{Voraussetzung}{
	name        ={Voraussetzung},
	plural      ={Voraussetzungen},
	description ={
		Die \glspl{Voraussetzung} einer \gls{Schlussregel} $\frac{\prerequisiteSet}{\conclusionSet}$ sind die Elemente aus $\prerequisiteSet$.
		\\-- Standardsymbole:
		$\prerequisite$    = eine Voraussetzung,
		$\prerequisiteSet$ = eine Menge aus Voraussetzungen,
		$\prerequisiteRel$ = eine Relation (als Menge aufgefasst) aus Voraussetzungen
		\\-- Zur Definition \vrefseesub{sub-Schlussregeln}.
	}
}
\newcommand*    {\Voraussetzungsmenge} [1][]{\glsIdx  [#1]{Voraussetzungsmenge}}
\newcommand*    {\Voraussetzungsmengen}[1][]{\glsIdxPl[#1]{Voraussetzungsmenge}}
\newglossaryentry{Voraussetzungsmenge}{
	name        ={Voraussetzungsmenge},
	plural      ={Voraussetzungsmengen},
	description ={
		Die Menge der \glspl{Voraussetzung} einer \gls{Schlussregel} \textbzw\ eines \glos{Beweises}.
		\\-- Standardsymbol:
		$\prerequisiteSet$
		\\-- Zur Definition \vrefseesub{:Schlussregeln}.
	}
}

%W === W === W === W === W === W === W === W === W === W === W === W === W === W

\newcommand*    {\Wahrheitswert}  [1][]{\glsIdx  [#1]{Wahrheitswert}}
\newcommand*    {\Wahrheitswerte} [1][]{\glsIdxPl[#1]{Wahrheitswert}}
\newcommand*    {\Wahrheitswerten}[1][]{\glsIdxPl[#1]{Wahrheitswert}n}
\newglossaryentry{Wahrheitswert}{
	name        ={Wahrheitswert},
	plural      ={Wahrheitswerte},
	description ={
		Die Werte \chrqt{$\ltrue$} und \chrqt{$\lfalse$}, oft auch mit \chrqt{$\wahr$} \textbzw\ \chrqt{$\falsch$}, \chrqt{$\mathrm{true}$} \textbzw\ \chrqt{$\mathrm{false}$} oder einfach \chrqt{$1$} \textbzw\ \chrqt{$0$} bezeichnet.
	}
}
\newcommand*    {\Wort}   [1][]{\glsIdx  [#1]{Wort}}
\newcommand*    {\Worte}  [1][]{\glsIdxPl[#1]{Wort}}
\newcommand*    {\Woerter}[1][]{\glsIdxPl[#1]{Wort}}
\newglossaryentry{Wort}{
	name        ={Wort},
	plural      ={Wörter},
	description ={
		Ein Element einer \gls{Sprache}.
		In dem Fall Synonym zu \gls{Formel}.
		\\-- Siehe \gls{Formelmenge}.
	}
}

%Z === Z === Z === Z === Z === Z === Z === Z === Z === Z === Z === Z === Z === Z

\newcommand*    {\Zeichenfolge} [1][]{\glsIdx  [#1]{Zeichenfolge}}
\newcommand*    {\Zeichenfolgen}[1][]{\glsIdxPl[#1]{Zeichenfolge}}
\newglossaryentry{Zeichenfolge}{
	name        ={Zeichenfolge},
	plural      ={Zeichenfolgen},
	description ={
		Eine Folge aus \glspl{Symbol}, wobei Leerstellen und sonstiger Zwischenraum nicht zählen und nur zur besseren Darstellung dienen.
		Dabei sind als spezielle \glspl{Symbol} auch \glspl{Zeichenkette} erlaubt, solange die Zerlegung eindeutig bleibt.
		\textZB\ kann \chrqt{sin} als ein einzelnes \gls{Symbol} -- für die Sinusfunktion -- aufgefasst werden, aber auch als Folge aus den Buchstaben \chrqt{s}, \chrqt{i} und \chrqt{n}.
		\glspl{Formel} werden immer als \glspl{Zeichenfolge} aufgefasst.
		\\-- Siehe auch \gls{Zeichenkette}.
		\\-- Zur Definition \vrefseesub{subsub-Definitionen}.
	}
}
\newcommand*    {\Zeichenkette} [1][]{\glsIdx  [#1]{Zeichenkette}}
\newcommand*    {\Zeichenketten}[1][]{\glsIdxPl[#1]{Zeichenkette}}
\newglossaryentry{Zeichenkette}{
	name        ={Zeichenkette},
	plural      ={Zeichenketten},
	description ={
		Eine Folge aus (typographischen) Zeichen, auch Leerstellen und sonstigem Zwischenraum.
		\\-- Siehe auch \gls{Zeichenfolge}.
		\\-- Zur Definition \vrefseesub{subsub-Definitionen}.
	}
}
\newcommand*    {\zerlegbar} [1][]{\glsIdx  [#1]{zerlegbar}}
\newcommand*    {\zerlegbare}[1][]{\glsIdxPl[#1]{zerlegbar}}
\newcommand*    {\Zerlegbare}[1][]{\GlsIdxPl[#1]{zerlegbar}}
\newglossaryentry{zerlegbar}{
	name        ={zerlegbar},
	plural      ={zerlegbare},
	description ={
		Eine \gls{Aussage}, die eine \gls{Metaoperation}, \textbzw\ eine \gls{Formel}, die eine \gls{Operation} oder eine \gls{Relation} enthält, heißen \gls{zerlegbar}.
		\\-- Vergleiche auch \gls{unzerlegbar}.
	}
}
\newcommand*    {\Ziel} [1][]{\glsIdx  [#1]{Ziel}}
\newcommand*    {\Ziele}[1][]{\glsIdxPl[#1]{Ziel}}
\newglossaryentry{Ziel}{
	name        ={Ziel},
	plural      ={Ziele},
	description ={
		In diesem Dokument sind \glspl{Ziel} die Anforderungen an \gls{ASBA}.
	}
}
\newcommand*    {\Zielbereich} [1][]{\glsIdx  [#1]{Zielbereich}}
\newcommand*    {\Zielbereiche}[1][]{\glsIdxPl[#1]{Zielbereich}}
\newglossaryentry{Zielbereich}{
	name        ={Zielbereich},
	plural      ={Zielbereiche},
	description ={
		einer \gls{Funktion}.
		\\-- Symbol: $\Zb$
		\\-- Zur genaueren Definition \vrefseesub{sub-weitereBezeichnungen}.
	}
}
\newcommand*    {\zulaessig}  [1][]{\glsIdx  [#1]{zulaessig}}
\newcommand*    {\zulaessige} [1][]{\glsIdxPl[#1]{zulaessig}}
\newcommand*    {\zulaessigen}[1][]{\glsIdx  [#1]{zulaessig}en}
\newcommand*    {\zulaessiger}[1][]{\glsIdxPl[#1]{zulaessig}r}
\newglossaryentry{zulaessig}{
	name        ={zulässig},
	plural      ={zulässige},
	description ={
		Eine Eigenschaft von \gls{Formel}, \gls{Transformation} und \gls{Substitution}.
	}
}
