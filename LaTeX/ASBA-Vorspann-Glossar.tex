%%############################################################################%%
%%                                                                            %%
%% Datei:  ASBA-Vorspann-Glossary.tex                                         %%
%% Inhalt: Vorspann Glossareinträge für ASBA                                  %%
%%                                                                            %%
%% Copyright (C) 2017  Winfried Teschers                                      %%
%%                                                                            %%
%% This program is free software: you can redistribute it and/or modify       %%
%% it under the terms of the GNU Affero General Public License as published   %%
%% by the Free Software Foundation, either version 3 of the License, or       %%
%% (at your option) any later version.                                        %%
%%                                                                            %%
%% This program is distributed in the hope that it will be useful,            %%
%% but WITHOUT ANY WARRANTY; without even the implied warranty of             %%
%% MERCHANTABILITY or FITNESS FOR A PARTICULAR PURPOSE.  See the              %%
%% GNU Affero General Public License for more details.                        %%
%%                                                                            %%
%% You should have received a copy of the GNU Affero General Public License   %%
%% along with this program.  If not, see <http://www.gnu.org/licenses/>.      %%
%%                                                                            %%
%% Dr. Winfried Teschers                                                      %%
%% Anton-Günther-Straße 26c                                                   %%
%% 91083 Baiersdorf                                                           %%
%% Germany                                                                    %%
%%                                                                            %%
%% e-mail: winfried.teschers@t-online.de                                      %%
%%                                                                            %%
%%############################################################################%%

% !TeX root = ASBA.tex
% !TeX encoding = UTF-8
% !TeX spellcheck = de_DE

% Elemente, die keine Glossareinträge sind, werden in "ASBA-Vorspann.tex" und "ASBA-Mathematik-Vorspann.tex" definiert.

% Symbole für Mengen -----------------------------------------------------------
% \symXX - Ausgabe als Symbol und Aufnahme in Symbolliste und Glossar

\newcommand*{\polnischLetter}{p}% ...in [p]olnischer Notation

\newcommand*        {\symIN}[1][]{\glsSym[#1]{IN}{\IN}}
\newglossaryentry       {IN}{
	name  ={\ensuremath{\IN}},
	symbol={\ensuremath{\IN}},
	sort  ={N},
	description={
		Die Menge der natürlichen Zahlen ohne 0.
		\\-- Zur Definition \vrefseesub{sub:Bezeichnungen}
	}
}
\newcommand*        {\symINo}[1][]{\glssymbol{INo}\sym[#1]{\INo}}
\newglossaryentry       {INo}{
	name  ={\ensuremath{\INo}},
	symbol={\ensuremath{\INo}},
	sort  ={N0},
	description={
		Die Menge der natürlichen Zahlen einschließlich 0.
		\\-- Zur Definition \vrefseesub{sub:Bezeichnungen}.
	}
}
\newcommand*           {\alABCLetter}{A}% [A]lphabet der al Sprache
\newcommand*        {\symalABC}[1][]{\glsSym[#1]{alABC}{\alABC}}
\newglossaryentry       {alABC}{
	name  ={\ensuremath{\alABC}},
	symbol={\ensuremath{\alABC}},
	sort  ={A},%        \alABCLetter
	description={
		Das Alphabet der aussagenlogischen \glos{Sprache}.
		\\-- Zur Definition \vrefseesubsub{subsub:Formeln}.
	}
}
\newcommand*        {\symalABCx}[1][]{\glsSym[#1]{alABCx}{\alABC_x}}
\newglossaryentry       {alABCx}{
	name  ={\ensuremath{\alABC_x}},
	symbol={\ensuremath{\alABC_x}},
	sort  ={Ax},%       \alABCLetter x
	description={
		Eine Teilmenge des Alphabets $\alABC$ der aussagenlogischen \glos{Sprache}.
		\\-- Zur Definition \vrefseesubsub{subsub:Formeln}.
	}
}
\newcommand*           {\alBinLetter}{O}% binäre [O]perationssymbole
\newcommand*        {\symalBin}[1][]{\glsSym[#1]{alBin}{\alBin}}
\newglossaryentry       {alBin}{
	name  ={\ensuremath{\alBin}},
	symbol={\ensuremath{\alBin}},
	sort  ={O},%        \alBinLetter
	description={
		Die Menge der binären \glos{Junktoren}.
		\\-- Zur Definition \vrefseesubsub{subsub:Bausteine}.
	}
}
\newcommand*           {\alConLetter}{K}% [K]onstantensymbole
\newcommand*        {\symalCon}[1][]{\glsSym[#1]{alCon}{\alCon}}
\newglossaryentry       {alCon}{
	name  ={\ensuremath{\alCon}},
	symbol={\ensuremath{\alCon}},
	sort  ={K},%        \alConLetter
	description={
		Die Menge der aussagenlogischen Konstanten.
		\\-- Zur Definition \vrefseesubsub{subsub:Bausteine}.
	}
}
\newcommand*           {\alForLetter}{L}
\newcommand*        {\symalFor}[1][]{\glsSym[#1]{alFor}{\alFor}}
\newglossaryentry       {alFor}{
	name  ={\ensuremath{\alFor}},
	symbol={\ensuremath{\alFor}},
	sort  ={L},%        \alForLetter
	description={
		Die Menge der aussagenlogischen \glos{Formeln} mit Klammerung.
	}
}
\newcommand*        {\symalForp}[1][]{\glsSym[#1]{alForp}{\alForp}}
\newglossaryentry       {alForp}{
	name  ={\ensuremath{\alForp}},
	symbol={\ensuremath{\alForp}},
	sort  ={Lp},%       \alForLetter\polnischLetter
	description={
		Die Menge der aussagenlogischen \glos{Formeln} in polnischer Notation.
	}
}
\newcommand*        {\symalForx}[1][]{\glsSym[#1]{alForx}{\alFor_x}}
\newglossaryentry       {alForx}{
	name  ={\ensuremath{\alFor_x}},
	symbol={\ensuremath{\alFor_x}},
	sort  ={Lx},%       \alForLetter x
	description={
		Eine Teilmenge der Menge $\alFor$ der aussagenlogischen \glos{Formeln} mit Klammerung.
	}
}
\newcommand*        {\symalForpx}[1][]{\glsSym[#1]{alForpx}{\alForp_x}}
\newglossaryentry       {alForpx}{
	name  ={\ensuremath{\alForp_x}},
	symbol={\ensuremath{\alForp_x}},
	sort  ={Lpx},%      \alForLetter\polnischLetter x
	description={
		Eine Teilmenge der Menge $\alForp$ der aussagenlogischen \glos{Formeln} in polnischer Notation.
	}
}
\newcommand*           {\alJunLetter}{J}% [J]unktoren
\newcommand*        {\symalJun}[1][]{\glsSym[#1]{alJun}{\alJun}}
\newglossaryentry       {alJun}{
	name  ={\ensuremath{\alJun}},
	symbol={\ensuremath{\alJun}},
	sort  ={J},%        \alJunLetter
	description={
		Die Menge der \glos{Junktorsymbole}.
		\\-- Zur Definition \vrefseesubsub{subsub:Bausteine}.
	}
}
\newcommand*        {\symalJunx}[1][]{\glsSym[#1]{alJunx}{\alJun_x}}
\newglossaryentry       {alJunx}{
	name  ={\ensuremath{\alJun_x}},
	symbol={\ensuremath{\alJun_x}},
	sort  ={Jx},%       \alJunLetter x
	description={
		Eine Teilmenge der Menge $\alJun$ der \glos{Junktorsymbole}.
		\\-- Zur Definition \vrefseesubsub{subsub:Bausteine}.
	}
}
\newcommand*           {\alUnaLetter}{U}% [u]näre Operationssymbole
\newcommand*        {\symalUna}[1][]{\glsSym[#1]{alUna}{\alUna}}
\newglossaryentry       {alUna}{
	name  ={\ensuremath{\alUna}},
	symbol={\ensuremath{\alUna}},
	sort  ={U},%        \alUnaLetter
	description={
		Die Menge der unären \glos{Junktoren}.
		\\-- Zur Definition \vrefseesubsub{subsub:Bausteine}.
	}
}
\newcommand*           {\alvarLetter}{q}% Name einer Variablen
\newcommand*           {\alVarLetter}{Q}% Variablensymbole
\newcommand*        {\symalVar}[1][]{\glsSym[#1]{alVar}{\alVar}}
\newglossaryentry       {alVar}{
	name  ={\ensuremath{\alVar}},
	symbol={\ensuremath{\alVar}},
	sort  ={Q},%        \alVarLetter
	description={
		Die Menge der aussagenlogischen Variablen $\alvar_i$ für $i \in \INo$.
		\\-- Zur Definition \vrefseesubsub{subsub:Bausteine}.
	}
}
\newcommand*           {\formulaSetLetter}{L}% Sprache, [l]anguage
\newcommand*        {\symformulaSet}[1][]{\glsSym[#1]{formulaSet}{\formulaSet}}
\newglossaryentry       {formulaSet}{
	name  ={\ensuremath{\formulaSet}},
	symbol={\ensuremath{\formulaSet}},
	sort  ={L},%        \formulaSetLetter
	description={
		\glos{Formelmenge}.
	}
}
\newcommand* {\symMengeMo}[1][]{\glsSym[#1]{MengeMo}{M^0}}
\newglossaryentry{MengeMo}{
	name  ={\ensuremath{M^0}},
	symbol={\ensuremath{M^0}},
	sort  ={M0},
	description={
		$\{()\}$ , wobei $()$ das 0-Tupel ist.
		\\-- Zur Definition \vrefseesub{sub:Bezeichnungen}.
	}
}
\newcommand* {\symMengeMn}[1][]{\glsSym[#1]{MengeMn}{M^n}}
\newglossaryentry{MengeMn}{
	name  ={\ensuremath{M^n}},
	symbol={\ensuremath{M^n}},
	sort  ={Mn},
	description={
		Das kartesische Produkt $M \times \dots \times M$ aus $n$ Mengen $M$ mit $n \in \INo$.
		\\-- Zur Definition \vrefseesub{sub:Bezeichnungen}.
	}
}
\newcommand*           {\tupelSetLetter}{T}% Menge der [T]upel
\newcommand*        {\symtupelSet}[1][]{\glsSym[#1]{tupelSet}{\tupelSet}}
\newglossaryentry       {tupelSet}{
	name  ={\ensuremath{\tupelSet}},
	symbol={\ensuremath{\tupelSet}},
	sort  ={T},%        \tupelSetLetter
	description={
		\glos{Tupelmenge}.
	}
}

% Symbole für Beispieloperationen und -relationen ------------------------------
% \symXX - Ausgabe als Symbol und Aufnahme in Symbolliste und Glossar

%%%\newcommand*        {\symlrelbsp}[1][]{\glsSym[#1]{lrelbsp}{\lrelbsp}}
%%%\newglossaryentry       {lrelbsp}{
%%%	name  ={\ensuremath{\lrelbsp}},
%%%	symbol={\ensuremath{\lrelbsp}},
%%%	description={
%%%		Beispielsymbol für eine binäre \glos{Relation} mit \glos{Umkehrrelation} $\rrelbsp$.
%%%		\\-- Zur Definition \vrefseesub{sub:Beispielsymbole}.
%%%	}
%%%}
%%%\newcommand*        {\symlreleqbsp}[1][]{\glsSym[#1]{lreleqbsp}{\lreleqbsp}}
%%%\newglossaryentry       {lreleqbsp}{
%%%	name  ={\ensuremath{\lreleqbsp}},
%%%	symbol={\ensuremath{\lreleqbsp}},
%%%	description={
%%%		Beispielsymbol für eine binäre \glos{Relation} mit \glos{Gleichheit} und \glos{Umkehrrelation} $\rreleqbsp$.
%%%		\\-- Zur Definition \vrefseesub{sub:Beispielsymbole}.
%%%	}
%%%}
%%%\newcommand*        {\symlreleqbsp}[1][]{\glsSym[#1]{lreleqbsp}{\lreleqbsp}}
%%%\newglossaryentry       {lrelnbsp}{
%%%	name  ={\ensuremath{\lrelnbsp}},
%%%	symbol={\ensuremath{\lrelnbsp}},
%%%	description={
%%%		Beispielsymbol für eine binäre negierte \glos{Relation} mit \glos{Umkehrrelation} $\rrelnbsp$.
%%%		\\-- Zur Definition \vrefseesub{sub:Beispielsymbole}.
%%%	}
%%%}
%%%\newcommand*        {\symlrelnebsp}[1][]{\glsSym[#1]{lrelnebsp}{\lrelnebsp}}
%%%\newglossaryentry       {lrelnebsp}{
%%%	name  ={\ensuremath{\lrelnebsp}},
%%%	symbol={\ensuremath{\lrelnebsp}},
%%%	description={
%%%		Beispielsymbol für eine binäre negierte \glos{Relation} mit \glos{Ungleichheit} und \glos{Umkehrrelation} $\rrelnebsp$.
%%%		\\-- Zur Definition \vrefseesub{sub:Beispielsymbole}.
%%%	}
%%%}
\newcommand*        {\symopbsp}[1][]{\glsSym[#1]{opbsp}{\opbsp}}
\newglossaryentry       {opbsp}{
	name  ={\ensuremath{\opbsp}},
	symbol={\ensuremath{\opbsp}},
	description={
		Beispielsymbol für eine binäre \glos{Operation}.
		\\-- Zur Definition \vrefseesub{sub:Beispielsymbole}.
	}
}
\newcommand*        {\symopubsp}[1][]{\glsSym[#1]{opubsp}{\opubsp}}
\newglossaryentry       {opubsp}{
	name  ={\ensuremath{\opubsp}},
	symbol={\ensuremath{\opubsp}},
	description={
		Beispielsymbol für eine unäre \glos{Operation}.
		\\-- Zur Definition \vrefseesub{sub:Beispielsymbole}.
	}
}
\newcommand*        {\symrelbsp}[1][]{\glsSym[#1]{relbsp}{\relbsp}}
\newglossaryentry       {relbsp}{
	name  ={\ensuremath{\relbsp}},
	symbol={\ensuremath{\relbsp}},
	description={
		Beispielsymbol für eine binäre \glos{Relation} mit \glos{Umkehrrelation} $\relbackbsp$
		\\-- Zur Definition \vrefseesub{sub:Beispielsymbole}.
	}
}
\newcommand*        {\symreleqbsp}[1][]{\glsSym[#1]{releqbsp}{\releqbsp}}
\newglossaryentry       {releqbsp}{
	name  ={\ensuremath{\releqbsp}},
	symbol={\ensuremath{\releqbsp}},
	description={
		Beispielsymbol für eine binäre \glos{Relation} mit \glos{Gleichheit} und \glos{Umkehrrelation} $\relbackeqbsp$
		\\-- Zur Definition \vrefseesub{sub:Beispielsymbole}.
	}
}
\newcommand*        {\symrelnbsp}[1][]{\glsSym[#1]{relnbsp}{\relnbsp}}
\newglossaryentry       {relnbsp}{
	name  ={\ensuremath{\relnbsp}},
	symbol={\ensuremath{\relnbsp}},
	description={
		Verneinung von $\relbsp$.
		\\-- Zur Definition \vrefseesub{sub:Beispielsymbole}.
	}
}
\newcommand*        {\symrelnebsp}[1][]{\glsSym[#1]{relnebsp}{\relnebsp}}
\newglossaryentry       {relnebsp}{
	name  ={\ensuremath{\relnebsp}},
	symbol={\ensuremath{\relnebsp}},
	description={
		Verneinung von $\releqbsp$.
		\\-- Zur Definition \vrefseesub{sub:Beispielsymbole}.
	}
}
\newcommand*        {\symrelbackbsp}[1][]{\glsSym[#1]{relbackbsp}{\relbackbsp}}
\newglossaryentry       {relbackbsp}{
	name  ={\ensuremath{\relbackbsp}},
	symbol={\ensuremath{\relbackbsp}},
	description={
		Beispielsymbol für eine binäre \glos{Relation} %%% mit \glos{Umkehrrelation} $\relbackbackbsp$
		\\-- Zur Definition \vrefseesub{sub:Beispielsymbole}.
	}
}
\newcommand*        {\symrelbackeqbsp}[1][]{\glsSym[#1]{relbackeqbsp}{\relbackeqbsp}}
\newglossaryentry       {relbackeqbsp}{
	name  ={\ensuremath{\relbackeqbsp}},
	symbol={\ensuremath{\relbackeqbsp}},
	description={
		Beispielsymbol für eine binäre \glos{Relation} mit \glos{Gleichheit} %%%  und \glos{Umkehrrelation} $\releqbsp$
		\\-- Zur Definition \vrefseesub{sub:Beispielsymbole}.
	}
}
%%%\newcommand*        {\symrrelbsp}[1][]{\glsSym[#1]{rrelbsp}{\rrelbsp}}
%%%\newglossaryentry       {rrelbsp}{
%%%	name  ={\ensuremath{\rrelbsp}},
%%%	symbol={\ensuremath{\rrelbsp}},
%%%	description={
%%%		Beispielsymbol für eine binäre \glos{Relation} mit \glos{Umkehrrelation} $\lrelbsp$.
%%%		\\-- Zur Definition \vrefseesub{sub:Beispielsymbole}.
%%%	}
%%%}
%%%\newcommand*        {\symrreleqbsp}[1][]{\glsSym[#1]{rreleqbsp}{\rreleqbsp}}
%%%\newglossaryentry       {rreleqbsp}{
%%%	name  ={\ensuremath{\rreleqbsp}},
%%%	symbol={\ensuremath{\rreleqbsp}},
%%%	description={
%%%		Beispielsymbol für eine binäre \glos{Relation} mit \glos{Gleichheit} und \glos{Umkehrrelation} $\lreleqbsp$.
%%%		\\-- Zur Definition \vrefseesub{sub:Beispielsymbole}.
%%%	}
%%%}
%%%\newcommand*        {\symrrelnbsp}[1][]{\glsSym[#1]{rrelnbsp}{\rrelnbsp}}
%%%\newglossaryentry       {rrelnbsp}{
%%%	name  ={\ensuremath{\rrelnbsp}},
%%%	symbol={\ensuremath{\rrelnbsp}},
%%%	description={
%%%		Beispielsymbol für eine binäre negierte \glos{Relation} mit \glos{Umkehrrelation} $\lrelnbsp$.
%%%		\\-- Zur Definition \vrefseesub{sub:Beispielsymbole}.
%%%	}
%%%}
%%%\newcommand*        {\symrrelnebsp}[1][]{\glsSym[#1]{rrelnebsp}{\rrelnebsp}}
%%%\newglossaryentry       {rrelnebsp}{
%%%	name  ={\ensuremath{\rrelnebsp}},
%%%	symbol={\ensuremath{\rrelnebsp}},
%%%	description={
%%%		Beispielsymbol für eine binäre negierte \glos{Relation} mit \glos{Ungleichheit} und \glos{Umkehrrelation} $\lrelnebsp$.
%%%		\\-- Zur Definition \vrefseesub{sub:Beispielsymbole}.
%%%	}
%%%}

% Meta-Symbole -----------------------------------------------------------------
% \symXX - Ausgabe als Symbol und Aufnahme in Symbolliste und Glossar

\newcommand*        {\symdefeq}[1][]{\glsSym[#1]{defeq}{\defeq}}
\newglossaryentry       {defeq}{
	name  ={\ensuremath{\defeq}},
	symbol={\ensuremath{\defeq}},
	description={
		\glos{Definition}:~ \textdots\ \emph{definitionsgemäß gleich} \textdots
	}
}
\newcommand*        {\symderive}[1][]{\glsSym[#1]{derive}{\derive}}
\newglossaryentry       {derive}{
	name  ={\ensuremath{\derive}},
	symbol={\ensuremath{\derive}},
	description={
		\glos{Ableitungsrelation}:~ \textdots\ \emph{\glos{ableitbar}} (\glos{beweisbar}) \textdots
	}
}
\newcommand*        {\symderiveR}[1][]{\glsSym[#1]{deriveR}{\derive_R}}
\newglossaryentry       {deriveR}{
	name  ={\ensuremath{\derive_R}},
	symbol={\ensuremath{\derive_R}},
	description={
		Eine Darstellung der \glos{Relation} $R$ aus $\Rel(\Pot(\formulaSet))$ als \glos{Ableitungsrelation}.
	}
}
\newcommand*        {\symeq}[1][]{\glsSym[#1]{eq}{\eq}}
\newglossaryentry       {eq}{
	name  ={\ensuremath{\eq}},
	symbol={\ensuremath{\eq}},
	description={
		Eine \glos{Metarelation}:~ \textdots\ \emph{gleich} (ist dasselbe wie; ist identisch zu) \textdots
		\\-- Siehe \glos{Gleichheit}.
		\\-- Zur Definition \vrefseesubsub{subsub:Vergleiche} und \vrefseesub{sub:ausJunktorDef}.
	}
}
\newcommand*        {\symequiv}[1][]{\glsSym[#1]{equiv}{\equiv}}
\newglossaryentry       {equiv}{
	name  ={\ensuremath{\equiv}},
	symbol={\ensuremath{\equiv}},
	description={
		Eine \glos{Metarelation}:~ \textdots\ \emph{äquivalent zu} (ist das gleiche wie; ist so wie) \textdots
		\\-- Siehe \glos{Äquivalenz}.
		\\-- Zur Definition \vrefseesubsub{subsub:Vergleiche} und \vrefseesub{sub:ausJunktorDef}.
	}
}
\newcommand*        {\symmetaand}[1][]{\glsSym[#1]{metaand}{\metaand}}
\newglossaryentry       {metaand}{
	name  ={\ensuremath{\metaand}},
	symbol={\ensuremath{\metaand}},
	description={
		Eine \glos{Metaoperation}:~ \textdots\ \emph{und} \textdots
		\\-- Zur Definition \vrefseesub{sub:AussagenUndMetaoperationen}.
	}
}
\newcommand*        {\symmetadefeq}[1][]{\glsSym[#1]{metadefeq}{\metadefeq}}
\newglossaryentry       {metadefeq}{
	name  ={\ensuremath{\metadefeq}},
	symbol={\ensuremath{\metadefeq}},
	description={
		Eine \glos{Metadefinition}:~ \textdots\ \emph{definitionsgemäß genau dann wenn} \textdots
	}
}
\newcommand*        {\symmetaequiv}[1][]{\glsSym[#1]{metaequiv}{\metaequiv}}
\newglossaryentry       {metaequiv}{
	name  ={\ensuremath{\metaequiv}},
	symbol={\ensuremath{\metaequiv}},
	description={
		Eine \glos{Metarelation}:~ \textdots\ \emph{genau dann wenn} \textdots
		\\-- Zur Definition \vrefseesub{sub:AussagenUndMetaoperationen}.
	}
}
\newcommand*        {\symmetaimp}[1][]{\glsSym[#1]{metaimp}{\metaimp}}
\newglossaryentry       {metaimp}{
	name  ={\ensuremath{\metaimp}},
	symbol={\ensuremath{\metaimp}},
	description={
		Eine \glos{Metarelation}:~ \textdots\ \emph{dann auch} \textdots , die Umkehrrelation zu $\metarep$
		\\-- Zur Definition \vrefseesub{sub:AussagenUndMetaoperationen}.
	}
}
\newcommand*        {\symmetanot}[1][]{\glsSym[#1]{metanot}{\metanot}}
\newglossaryentry       {metanot}{
	name  ={\ensuremath{\metanot}},
	symbol={\ensuremath{\metanot}},
	description={
		Eine unäre \glos{Metaoperation}:~ \textdots emph{gilt nicht}
		\\-- Zur Definition \vrefseesub{sub:AussagenUndMetaoperationen}.
	}
}
\newcommand*        {\symmetaor}[1][]{\glsSym[#1]{metaor}{\metaor}}
\newglossaryentry       {metaor}{
	name  ={\ensuremath{\metaor}},
	symbol={\ensuremath{\metaor}},
	description={
		Eine \glos{Metaoperation}:~ \textdots\ \emph{oder} \textdots
		\\-- Zur Definition \vrefseesub{sub:AussagenUndMetaoperationen}.
	}
}
\newcommand*        {\symmetarep}[1][]{\glsSym[#1]{metarep}{\metarep}}
\newglossaryentry       {metarep}{
	name  ={\ensuremath{\metarep}},
	symbol={\ensuremath{\metarep}},
	description={
		Eine \glos{Metarelation}:~ \textdots\ \emph{sofern} \textdots , die Umkehrrelation zu $\metaimp$
		\\-- Zur Definition \vrefseesub{sub:AussagenUndMetaoperationen}.
	}
}
\newcommand*        {\symne}[1][]{\glsSym[#1]{ne}{\ne}}
\newglossaryentry       {ne}{
	name  ={\ensuremath{\ne}},
	symbol={\ensuremath{\ne}},
	description={
		Eine \glos{(Meta-)Operation}:~ \textdots\ \emph{ungleich} (nicht dasselbe wie; nicht identisch zu) \textdots
	}
}
\newcommand*        {\symnequiv}[1][]{\glsSym[#1]{nequiv}{\nequiv}}
\newglossaryentry       {nequiv}{
	name  ={\ensuremath{\nequiv}},
	symbol={\ensuremath{\nequiv}},
	description={
		Eine \glos{Metarelation}:~ \textdots\ \emph{nicht äquivalent} (ist nicht das gleiche wie; ist nicht so wie) \textdots
	}
}
\newcommand*        {\symsubst}[1][]{\glsSym[#1]{subst}{\subst}}
\newglossaryentry       {subst}{
	name  ={\ensuremath{\subst}},
	symbol={\ensuremath{\subst}},
	description={
		\glos{Substitution}:~ \textdots\ \emph{substituiert durch} \textdots\
		\\-- Zur Definition \vrefseesub{sub:Identitätsregeln}.
	}
}
\newcommand*        {\symswap}[1][]{\glsSym[#1]{swap}{\swap}}
\newglossaryentry       {swap}{
	name  ={\ensuremath{\swap}},
	symbol={\ensuremath{\swap}},
	description={
		\glos{Vertauschung}:~ \textdots\ \emph{vertauscht mit} \textdots\
		\\-- Zur Definition \vrefseesub{sub:Identitätsregeln}.
	}
}
\newcommand*        {\symsrand}[1][]{\glsSym[#1]{srand}{\srand}}
\newglossaryentry       {srand}{
	name  ={\ensuremath{\srand}},
	symbol={\ensuremath{\srand}},
	description={
		Eine \glos{Metaoperation}:~ \textdots\ \emph{und} \textdots\
		\\-- Wird nur bei den \glos{Schlussregeln} verwendet.
	}
}

% Operationen ------------------------------------------------------------------
% \symXX - Ausgabe als Symbol und Aufnahme in Symbolliste und Glossar

\newcommand*{\DbSymbol}{dom}% Definitionsbereich ([dom]ain) einer Funktion
\newcommand*        {\symDb}[1][]{\glsSym[#1]{Db}{\Db}}
\newglossaryentry       {Db}{
	name  ={\ensuremath{\Db}},
	symbol={\ensuremath{\Db}},
	sort  ={dom},%      \DbSymbol
	description={
		$\Db(f)$ für $f : A \rightarrow B$ ist die Menge $A$
		\\-- Symbol: $\Db$
	}
}
\newcommand*        {\symland}[1][]{\glsSym[#1]{land}{\land}}
\newglossaryentry       {land}{
	name  ={\ensuremath{\land}},
	symbol={\ensuremath{\land}},
	description={
		Ein binärer \glos{Junktor}:~ \textdots\ \emph{und} \textdots\
		\\-- Zur Definition \vrefseetab{tab:Symbole}.
	}
}
\newcommand*        {\symlequiv}[1][]{\glsSym[#1]{lequiv}{\lequiv}}
\newglossaryentry       {lequiv}{
	name  ={\ensuremath{\lequiv}},
	symbol={\ensuremath{\lequiv}},
	description={
		Ein binärer \glos{Junktor}:~ \textdots\ \emph{genau dann wenn} \textdots\
		\\-- Zur Definition \vrefseetab{tab:Symbole}.
	}
}
\newcommand*{\lenSymbol}{len}% Länge ([len]gth) eines/r Tupels, Vektors, Folge, Reihe
\newcommand*        {\symlen}[1][]{\glsSym[#1]{len}{\len}}
\newglossaryentry       {len}{
	name  ={\ensuremath{\len}},
	symbol={\ensuremath{\len}},
	sort  ={len},%      \lenSymbol
	description={
		$\len(\vec{a})$ ist die Länge, \textdh\ die Anzahl der Elemente eines Vektors.
		\\-- Symbol: $\len$
	}
}
\newcommand*        {\symlfalse}[1][]{\glsSym[#1]{lfalse}{\lfalse}}
\newglossaryentry       {lfalse}{
	name  ={\ensuremath{\lfalse}},
	symbol={\ensuremath{\lfalse}},
	description={
		Ein 0-stelliger \glos{Junktor}, \textdh\ eine aussagenlogische Konstante (\glos{Wahrheitswert}): \emph{$\falsch$}
		\\-- Zur Definition \vrefseetab{tab:Symbole}.
	}
}
\newcommand*{\graphSymbol}{graph}% [Graph] von Funktionen und Relationen
\newcommand*        {\symgraph}[1][]{\glsIdx[#1]{graph}{\graph}}
\newglossaryentry       {graph}{
	name  ={\ensuremath{\graph}},
	plural={\ensuremath{\graph}},
	sort  ={graph},%    \graphSymbol
	description ={
		$\graph(X)$ ist der \glos{Graph} der Funktion \textbzw\ Relation $X$.
		\\-- Zur genaueren Definition \vrefseesub{sub:weitereBezeichnungen}.
	}
}
\newcommand*        {\symlimp}[1][]{\glsSym[#1]{limp}{\limp}}
\newglossaryentry       {limp}{
	name  ={\ensuremath{\limp}},
	symbol={\ensuremath{\limp}},
	description={
		Ein binärer \glos{Junktor}:~ \emph{Aus} \textdots\ \emph{folgt} \textdots\
		\\-- Zur Definition \vrefseetab{tab:Symbole}.
	}
}
\newcommand*        {\symlnand}[1][]{\glsSym[#1]{lnand}{\lnand}}
\newglossaryentry       {lnand}{
	name  ={\ensuremath{\lnand}},
	symbol={\ensuremath{\lnand}},
	description={
		Ein binärer \glos{Junktor}:~ \emph{Nicht zugleich}\textdots\ \emph{und} \textdots\
		\\-- Zur Definition \vrefseetab{tab:Symbole}.
	}
}
\newcommand*        {\symlnor}[1][]{\glsSym[#1]{lnor}{\lnor}}
\newglossaryentry       {lnor}{
	name  ={\ensuremath{\lnor}},
	symbol={\ensuremath{\lnor}},
	description={
		Ein binärer \glos{Junktor}:~ \emph{Weder} \textdots\ \emph{noch} \textdots\
	}
		\\-- Zur Definition \vrefseetab{tab:Symbole}.
}
\newcommand*        {\symlnot}[1][]{\glsSym[#1]{lnot}{\lnot}}
\newglossaryentry       {lnot}{
	name  ={\ensuremath{\lnot}},
	symbol={\ensuremath{\lnot}},
	description={
		Ein unärer \glos{Junktor}:~ \emph{Nicht} \textdots\
		\\-- Zur Definition \vrefseetab{tab:Symbole}.
	}
}
\newcommand*        {\symlor}[1][]{\glsSym[#1]{lor}{\lor}}
\newglossaryentry       {lor}{
	name  ={\ensuremath{\lor}},
	symbol={\ensuremath{\lor}},
	description={
		Ein binärer \glos{Junktor}:~ \textdots\ \emph{oder} \textdots\
	}
	\\-- Zur Definition \vrefseetab{tab:Symbole}.
}
\newcommand*        {\symlrep}[1][]{\glsSym[#1]{lrep}{\lrep}}
\newglossaryentry       {lrep}{
	name  ={\ensuremath{\lrep}},
	symbol={\ensuremath{\lrep}},
	description={
		Ein binärer \glos{Junktor}:~ \textdots\ \emph{folgt aus} \textdots\
		\\-- Zur Definition \vrefseetab{tab:Symbole}.
	}
}
\newcommand*        {\symltrue}[1][]{\glsSym[#1]{ltrue}{\ltrue}}
\newglossaryentry       {ltrue}{
	name  ={\ensuremath{\ltrue}},
	symbol={\ensuremath{\ltrue}},
	description={
		Ein 0-stelliger \glos{Junktor}, \textdh\ eine aussagenlogische Konstante (\glos{Wahrheitswert}): \emph{$\wahr$}
		\\-- Zur Definition \vrefseetab{tab:Symbole}.
	}
}
\newcommand*        {\symlxor}[1][]{\glsSym[#1]{lxor}{\lxor}}
\newglossaryentry       {lxor}{
	name  ={\ensuremath{\lxor}},
	symbol={\ensuremath{\lxor}},
	description={
		Ein binärer \glos{Junktor}:~ \emph{entweder} \textdots\ \emph{oder} \textdots\
		\\-- Zur Definition \vrefseetab{tab:Symbole}.
	}
}

% sonstige mathematische Symbole -----------------------------------------------
% \symXX - Ausgabe als Symbol und Aufnahme in Symbolliste und Glossar

\newcommand*{\finiteLetter}{e}% [e]ndlich

\newcommand*        {\symnsubset}[1][]{\glsSym[#1]{nsubset}{\nsubset}}
\newglossaryentry       {nsubset}{
	name  ={\ensuremath{\nsubset}},
	symbol={\ensuremath{\nsubset}},
	description={
		Teilmengenbeziehung:~ \textdots\ \emph{ist keine echte Teilmenge von} \textdots\
	}
}
\newcommand*        {\symnsupset}[1][]{\glsSym[#1]{nsupset}{\nsupset}}
\newglossaryentry       {nsupset}{
	name  ={\ensuremath{\nsupset}},
	symbol={\ensuremath{\nsupset}},
	description={
		Teilmengenbeziehung:~ \textdots\ \emph{ist keine echte Obermenge von} \textdots\
	}
}
\newcommand*           {\PotLetter}{P}% [P]otenzmenge
\newcommand*        {\symPot}[1][]{\glsSym[#1]{Pot}{\Pot}}
\newglossaryentry       {Pot}{
	name  ={\ensuremath{\Pot}},
	symbol={\ensuremath{\Pot}},
	sort  ={P},%        \PotLetter
	description={
		\glos{Potenzmenge}.
	}
}
\newcommand*        {\symPotf}[1][]{\glsSym[#1]{Potf}{\Potf}}
\newglossaryentry       {Potf}{
	name  ={\ensuremath{\Potf}},
	symbol={\ensuremath{\Potf}},
	sort  ={Pe},%       \PotLetter\finiteLetter
	description={
		Menge der endlichen Teilmengen.
	}
}
%%%\newcommand*{\QbSymbol}{src}% Quellbereich ([s]ou[rc]e) einer partiellen Fkt.
%%%\newcommand*        {\symQb}[1][]{\glsSym[#1]{Qb}{\Qb}}
%%%\newglossaryentry       {Qb}{
%%%	name  ={\ensuremath{\Qb}},
%%%	symbol={\ensuremath{\Qb}},
%%%	sort  ={src},%      \QbSymbol
%%%	description={
%%%		$\Qb(f)$ für $f : A \rightarrow B$ ist die Menge $\{a \in A | f(a) \text{ existiert}}$.
%%%		\\-- Symbol: $\Qb$
%%%	}
%%%}
\newcommand*           {\RelLetter}{R}% Menge der [R]elationen
\newcommand*        {\symRel}[1][]{\glsSym[#1]{Rel}{\Rel}}
\newglossaryentry       {Rel}{
	name  ={\ensuremath{\Rel}},
	symbol={\ensuremath{\Rel}},
	sort  ={R},%        \RelLetter
	description={
		Menge der binären Relationen.
	}
}
\newcommand*        {\symRelf}[1][]{\glsSym[#1]{Relf}{\Relf}}
\newglossaryentry       {Relf}{
	name  ={\ensuremath{\Relf}},
	symbol={\ensuremath{\Relf}},
	sort  ={Re},%       \RelLetter\finiteLetter
	description={
		Menge der endlichen binären Relationen.
	}
}
\newcommand*{\SetSymbol}{Set}% Komponentenmenge eines Tupels, Vektors, ...
\newcommand*        {\symSet}[1][]{\glsSym[#1]{Set}{\Set}}
\newglossaryentry       {Set}{
	name  ={\ensuremath{\Set}},
	symbol={\ensuremath{\Set}},
	sort  ={Set},%      \SetSymbol
	description={
		$\Set(\vec{a})$ ist die Menge der Elemente eines Vektors.
		\\-- Symbol: $\Set$
	}
}
\newcommand*    {\stelfuncSymbol}{stel_f}% [Stel]ligkeit für [F]unktionen
\newglossaryentry{stelfunc}{
	name        ={\ensuremath{\stelfunc}},
	symbol      ={\ensuremath{\stelfunc}},
	sort        ={stelf},%    \stelfuncSymbol
	description ={
		\glos{Stelligkeit} einer \glos{Funktion}.
		\\-- Symbol: $\stelfunc$
		\\-- Zur genaueren Definition \vrefseesub{sub:weitereBezeichnungen}.
	}
}
\newcommand*    {\stelrelSymbol} {stel_r}% [Stel]ligkeit für [R]elationen
\newglossaryentry{stelrel}{
	name        ={\ensuremath{\stelrel}},
	symbol      ={\ensuremath{\stelrel}},
	sort        ={stelr},%    \stelrelSymbol
	description ={
		\glos{Stelligkeit} einer \glos{Relation}.
		\\-- Symbol: $\stelrel$
		\\-- Zur genaueren Definition \vrefseesub{sub:weitereBezeichnungen}.
	}
}
\newcommand*        {\symsubset}[1][]{\glsSym[#1]{subset}{\subset}}
\newglossaryentry       {subset}{
	name  ={\ensuremath{\subset}},
	symbol={\ensuremath{\subset}},
	description={
		Teilmengenbeziehung:~ \textdots\ \emph{ist echte Teilmenge von} \textdots\
		; Insbesondere kann keine \Gleichheit\ bestehen.
		In der Literatur wird $\subset$ oft im Sinne von $\subseteq$ verwendet.
		\\-- Zur Definition \vrefseesub{sub:Bezeichnungen}.
	}
}
\newcommand*        {\symsubseteq}[1][]{\glsSym[#1]{subseteq}{\subseteq}}
\newglossaryentry       {subseteq}{
	name  ={\ensuremath{\subseteq}},
	symbol={\ensuremath{\subseteq}},
	description={
		Teilmengenbeziehung:~ \textdots\ \emph{ist Teilmenge von} \textdots\
		; Insbesondere kann \Gleichheit\ bestehen.
		\\-- Zur Definition \vrefseesub{sub:Bezeichnungen}.
	}
}
\newcommand*        {\symsupset}[1][]{\glsSym[#1]{supset}{\supset}}
\newglossaryentry       {supset}{
	name  ={\ensuremath{\supset}},
	symbol={\ensuremath{\supset}},
	description={
		Teilmengenbeziehung:~ \textdots\ \emph{ist echte Obermenge von} \textdots\
		; Insbesondere kann keine \Gleichheit\ bestehen.
		In der Literatur wird $\supset$ oft im Sinne von $\supseteq$ verwendet.
	}
}
\newcommand*        {\symsupseteq}[1][]{\glsSym[#1]{supseteq}{\supseteq}}
\newglossaryentry       {supseteq}{
	name  ={\ensuremath{\supseteq}},
	symbol={\ensuremath{\supseteq}},
	description={
		Teilmengenbeziehung:~ \textdots\ \emph{ist Obermenge von} \textdots\
		; Insbesondere kann \Gleichheit\ bestehen.
	}
}
\newcommand*           {\traegerSymbol}{car}%  ([car]rier) Trägermenge einer Relation
\newcommand*        {\symtraeger}[1][]{\glsSym[#1]{len}{\len}}
\newglossaryentry       {traeger}{
	name  ={\ensuremath{\traeger}},
	symbol={\ensuremath{\traeger}},
	sort  ={car},%      \traegerSymbol
	description={
		$\traeger_i(R)$ für $R \subseteq A_1 \times \cdots \times A_n$ ist die \glos{Trägermenge} $A_i$ für $i$ von $1$ bis $n$.
		\\-- Symbol: $\traeger_i$
	}
}
\newcommand*{\ZbSymbol}{tar}% Zielbereich ([tar]get) einer Funktion
\newcommand*        {\symZb}[1][]{\glsSym[#1]{Zb}{\Zb}}
\newglossaryentry       {Zb}{
	name  ={\ensuremath{\Zb}},
	symbol={\ensuremath{\Zb}},
	sort  ={tar},%      \ZbSymbol
	description={
		$\Zb(f)$ für $f : A \rightarrow B$ ist die Menge $B$
		\\-- Symbol: $\Zb$
	}
}
%%%\newcommand*{\WbSymbol}{ran}% Wertebereich ([ran]ge) einer Funktion
%%%\newcommand*        {\symWb}[1][]{\glsSym[#1]{Wb}{\Wb}}
%%%\newglossaryentry       {Wb}{
%%%	name  ={\ensuremath{\Wb}},
%%%	symbol={\ensuremath{\Wb}},
%%%	sort  ={ran},%      \WbSymbol
%%%	description={
%%%		$\Wb(f)$ für $f : A \rightarrow B$ ist die Menge $\{f(a) | a \in A}$.
%%%		\\-- Symbol: $\Wb$
%%%	}
%%%}

% Schlussregeln ----------------------------------------------------------------
% \XX    - Ausgabe sowohl im Text- als auch Mathematik-Modus
% \symXX - Ausgabe als Symbol und Eintrag in Symbolliste und Glossar
% Verweise:
%   \ref    {def:XX} -->  \XX
%   \eqref  {def:XX} --> (\XX)
%   \vreffor{def:XX} --> (\XX) auf Seite n

\newcommand*    {\AR}{\ensuremath{\text{AR}}}
\newglossaryentry{AR}{
	name      ={(\AR)},
	sort        ={AR},
	description={
		\Anfangsregel\ - Eine \glos{Schlussregel}.
	}
}
\newcommand*    {\FS}{\ensuremath{\text{FS}}}
\newglossaryentry{FS}{
	name      ={(\FS)},
	sort        ={FS},
	description={
		\formalerSatz\ - Eine \glos{Schlussregel}.
	}
}
\newcommand*    {\MR}{\ensuremath{\text{MR}}}
\newglossaryentry{MR}{
	name      ={(\MR)},
	sort        ={MR},
	description={
		\Monotonieregel\ - Eine \glos{Schlussregel}.
	}
}
\newcommand*    {\SR}{\ensuremath{\text{SR}}}
\newglossaryentry{SR}{
	name      ={(\SR)},
	sort        ={SR},% ###
	description={
		\Schnittregel\ - Eine \glos{Schlussregel}.
	}
}
\newcommand*    {\TR}{\ensuremath{\text{TR}}}
\newglossaryentry{TR}{
	name      ={(\TR)},
	sort        ={TR},
	description={
		\Abtrennungsregel\ - Eine \glos{Schlussregel}.
	}
}
\newcommand*    {\eqB}{\ensuremath{\eq\text{B}}}
\newglossaryentry{eqB}{
	name      ={(\eqB)},
	sort          ={B=},
	description={
		Eine \glos{Schlussregel} - Beseitigung von \chrqt{$\eq$}.
	}
}
\newcommand*    {\eqE}{\ensuremath{\eq\text{E}}}
\newglossaryentry{eqE}{
	name      ={(\eqE)},
	sort          ={E=},
	description={
		Eine \glos{Schlussregel} - Einführung von \chrqt{$\eq$}.
	}
}
\newcommand*    {\andB}{\ensuremath{\land\text{B}}}
\newglossaryentry{andB}{
	name      ={(\andB)},
	sort           ={B},
	description={
		Eine \glos{Schlussregel} - Beseitigung von \chrqt{$\land$}.
	}
}
\newcommand*    {\andE}{\ensuremath{\land\text{E}}}
\newglossaryentry{andE}{
	name      ={(\andE)},
	sort           ={E},
	description={
		Eine \glos{Schlussregel} - Einführung von \chrqt{$\land$}.
	}
}
\newcommand*    {\impB}{\ensuremath{\limp\text{B}}}
\newglossaryentry{impB}{
	name      ={(\impB)},
	sort           ={B},
	description={
		Eine \glos{Schlussregel} - Beseitigung von \chrqt{$\limp$}.
	}
}
\newcommand*    {\impE}{\ensuremath{\limp\text{E}}}
\newglossaryentry{impE}{
	name      ={(\impE)},
	sort           ={E},
	description={
		Eine \glos{Schlussregel} - Einführung von \chrqt{$\limp$}.
	}
}
\newcommand*    {\nota}{\ensuremath{\lnot\text{1}}}
\newglossaryentry{nota}{
	name      ={(\nota)},
	sort           ={1},
	description={
		Eine \glos{Schlussregel} - Einführung/Beseitigung von \chrqt{$\lnot$} Teil 1.
	}
}
\newcommand*    {\notb}{\ensuremath{\lnot\text{2}}}
\newglossaryentry{notb}{
	name      ={(\notb)},
	sort           ={2},
	description={
		Eine \glos{Schlussregel} - Einführung/Beseitigung von \chrqt{$\lnot$} Teil 2.
	}
}
\newcommand*    {\notc}{\ensuremath{\lnot\text{3}}}
\newglossaryentry{notc}{
	name      ={(\notc)},
	sort           ={3},
	description={
		Eine \glos{Schlussregel} - Beweistechnik \enquote{Indirekter \glos{Beweis}}.
	}
}
\newcommand*    {\notd}{\ensuremath{\lnot\text{4}}}
\newglossaryentry{notd}{
	name      ={(\notd)},
	sort           ={4},
	description={
		Eine \glos{Schlussregel} - Reductio ad absurdum (Indirekter \glos{Beweis}).
	}
}
%%%\newcommand*    {\orB}{\ensuremath{\lor\text{B}}}
%%%\newglossaryentry{orB}{
%%%	name      ={(\orB)},
%%%	sort          ={B},
%%%	description={
%%%		Eine \glos{Schlussregel} - Beseitigung von \chrqt{$\lor$}.
%%%	}
%%%}
%%%\newcommand*    {\orE}{\ensuremath{\lor\text{E}}}
%%%\newglossaryentry{orE}{
%%%	name      ={(\orE)},
%%%	sort          ={E},
%%%	description={
%%%		Eine \glos{Schlussregel} - Einführung von \chrqt{$\lor$}.
%%%	}
%%%}

% Fachbegriffe #################################################################
% Hilfsmakros: gls=Glossary Kennung;      name=Glossary name;
%         singular=Glossary text(name); plural=Glossary plural
%                              Glossary-Eintrag   Index-Eintrag  Textausgabe
%   \glsIdx  {gls}             singular           name           singular
%   \glsIdxPl{gls}             singular           name           plural
%   \GlsIdxPl{gls}             singular           name           Plural
%   \glsIdy  {gls}{idx}        singular           idx            singular
%   \glsIdyPl{gls}{idx}        singular           idx            plural

%A === A === A === A === A === A === A === A === A === A === A === A === A === A

\newcommand*{\ASBA}[1][]{\glsIdx  [#1]{ASBA}}
\newacronym{ASBA}{ASBA}{
	Programmsystem, das \textbf{A}xiome, \textbf{S}ätze, \textbf{B}eweise und \textbf{A}uswertungen behandeln kann.
}
\newcommand*    {\ableitbar} [1][]{\glsIdx  [#1]{ableitbar}}
\newcommand*    {\ableitbare}[1][]{\glsIdxPl[#1]{ableitbar}}
\newglossaryentry{ableitbar}{
	name        ={ableitbar},
	plural      ={ableitbare},
	description ={
		Wenn sich eine \glos{Formel} $\beta$ aus einer anderen \glos{Formel} $\alpha$ mittels \glos{zulässiger Transformationen} ableiten lässt, heißt $\beta$ \glos{ableitbar} aus $\alpha$.
		Sprechweise: \seqqt{$ \alpha \text{ ableitbar } \beta $}.
		Eine oder beide \glos{Formeln} $\alpha$ \textbzw\ $\beta$ dürfen dabei durch \glos{Formelmengen} ersetzt werden.
		\\-- Siehe \glos{Ableitungsrelation} und $\derive$.
		\\-- Synonym: \glos{beweisbar}.
	}
}
\newcommand*    {\Ableitung}  [1][]{\glsIdx  [#1]{Ableitung}}
\newcommand*    {\Ableitungen}[1][]{\glsIdxPl[#1]{Ableitung}}
\newglossaryentry{Ableitung}{
	name        ={Ableitung},
	plural      ={Ableitungen},
	description ={
		Eine \glos{Aussage} $A \derive B$ \textbzw\ allgemeiner $A \derive_R B$.
		Dies entspricht einem Element $(A,B)$ einer \glos{Ableitungsrelation} $\derive$ \textbzw\ $\derive_R$.
		Die semantische Aussage ist, das die \glos{Formeln} von $B$ aus den \glos{Formeln} von $A$ abgeleitet werden können.
	}
}
%%%\newcommand*    {\Ableitungsmenge} [1][]{\glsIdx  [#1]{Ableitungsmenge}}
%%%\newcommand*    {\Ableitungsmengen}[1][]{\glsIdxPl[#1]{Ableitungsmenge}}
%%%\newglossaryentry{Ableitungsmenge}{
%%%	name        ={Ableitungsmenge},
%%%	plural      ={Ableitungsmengen},
%%%	description ={
%%%		Eine Menge von \glos{Ableitungen}, letztlich nichts anderes als eine \glos{ABleitungsrelation}.
%%%	}
%%%}
\newcommand*    {\Ableitungsrelation}  [1][]{\glsIdx  [#1]{Ableitungsrelation}}
\newcommand*    {\Ableitungsrelationen}[1][]{\glsIdxPl[#1]{Ableitungsrelation}}
\newglossaryentry{Ableitungsrelation}{
	name        ={Ableitungsrelation},
	plural      ={Ableitungsrelationen},
	description ={
		Eine binäre \glos{Relation} $\derive$ \textbzw\ allgemeiner $\derive_R$ aus $\formulaSetSet$.
		Siehe auch \glos{Ableitung}
	}
}
\newcommand*    {\Abtrennungsregel}[1][]{\glsIdx  [#1]{Abtrennungsregel}}
\newglossaryentry{Abtrennungsregel}{
	name        ={Abtrennungsregel},
	description ={
		Eine \glos{Schlussregel} -- siehe~\glos{TR}.
	}
}
\newcommand*    {\Aequivalenz}  [1][]{\glsIdy  [#1]{Aequivalenz}{Äquivalenz}}
\newcommand*    {\Aequivalenzen}[1][]{\glsIdyPl[#1]{Aequivalenz}{Äquivalenz}}
\newglossaryentry{Aequivalenz}{
	name        ={Äquivalenz},
	plural      ={Äquivalenzen},
	sort        ={Aquivalenz},
	description ={
		Eine \glos{Gleichheitsrelation}:
		Zwei Objekte $A$ und $B$ sind \emph{gleich} (äquivalent), $A \equiv B$, wenn sie in den \glos{interessierenden Eigenschaften} für $\equiv$ übereinstimmen.
		\\-- Zur Definition \vrefseesubsub{subsub:Vergleiche}.
	}
}
\newcommand*    {\Aequivalenzrelation}  [1][]{\glsIdy  [#1]{Aequivalenzrelation}{Äquivalenzrelation}}
\newcommand*    {\Aequivalenzrelationen}[1][]{\glsIdyPl[#1]{Aequivalenzrelation}{Äquivalenzrelation}}
\newglossaryentry{Aequivalenzrelation}{
	name        ={Äquivalenzrelation},
	plural      ={Äquivalenzrelationen},
	sort        ={Aquivalenzrelation},
	description ={
		Eine binäre \glos{Relation} $\sim$ auf einer Menge $M$ mit folgenden Eigenschaften:
		\begin{description}
			\item [reflexiv] ($a \sim a$)
			\item [transitiv] ($((a \sim b) \metaand (b \sim c)) \metaimp (a \sim c)$)
			\item[symmetrisch] ($(a \sim b) \metaimp (b \sim a)$)
		\end{description}
		jeweils für alle Elemente $a$, $b$ und $c$ aus $M$.
		\\-- \vrefSeesubsub{subsub:Vergleiche}.
	}
}
\newcommand*    {\allgemeingueltig}  [1][]{\glsIdy  [#1]{allgemeingueltig}{allgemeingültig}}
\newcommand*    {\allgemeingueltige} [1][]{\glsIdyPl[#1]{allgemeingueltig}{allgemeingültig}}
\newcommand*    {\allgemeingueltigen}[1][]{\glsIdyPl[#1]{allgemeingueltig}{allgemeingültig}n}
\newglossaryentry{allgemeingueltig}{
	name        ={allgemeingültig},
	plural      ={allgemeingültige},
	description ={
		Eine \glos{Schlussregel} heißt \defn{allgemeingültig}, wenn sie aus den \glos{Basisregeln} und schon bekannten \glos{allgemeingültigen} \glos{Schlussregeln} abgeleitet werden kann.
		\\-- Zur Definition \vrefseesub{sub:Schlussregeln}.
	}
}
\newcommand*    {\Anfangsregel}[1][]{\glsIdx  [#1]{Anfangsregel}}
\newglossaryentry{Anfangsregel}{
	name        ={Anfangsregel},
	description ={
		Die \glos{Schlussregel} \glos{\AR} um anfangen zu können.
	}
}
\newcommand*    {\atomar} [1][]{\glsIdx  [#1]{atomar}}
\newcommand*    {\atomare}[1][]{\glsIdxPl[#1]{atomar}}
\newglossaryentry{atomar}{
	name        ={atomar},
	plural      ={atomare},
	description ={
		Synonym zu \glos{unzerlegbar}, siehe dort; vergleiche auch \glos{zerlegbar}.
		Das Attribut betrifft \glos{Aussagen} und \glos{Formeln}.
	}
}
\newcommand*    {\Ausgabeschema}  [1][]{\glsIdx  [#1]{Ausgabeschema}}
\newcommand*    {\Ausgabeschemata}[1][]{\glsIdxPl[#1]{Ausgabeschema}}
\newglossaryentry{Ausgabeschema}{
	name        ={Ausgabeschema},
	plural      ={Ausgabeschemata},
	description ={
		Ein Schema, mit dem bestimmte mathematische \glos{Objekte} ausgegeben werden sollen.
	}
}
\newcommand*    {\Aussage} [1][]{\glsIdx  [#1]{Aussage}}
\newcommand*    {\Aussagen}[1][]{\glsIdxPl[#1]{Aussage}}
\newglossaryentry{Aussage}{
	name        ={Aussage},
	plural      ={Aussagen},
	description ={
		Eine \glos{Aussage} in natürlicher Sprache oder als \glos{Formel}, die einen \glos{Wahrheitswert} liefert.
		\\-- Zur Definition \vrefseesub{sub:AussagenUndMetaoperationen}.
	}
}
\newcommand*    {\Aussagenlogik}[1][]{\glsIdx  [#1]{Aussagenlogik}}
\newglossaryentry{Aussagenlogik}{
	name        ={Aussagenlogik},
	description ={
		-- Zur Definition \vrefseesec{sec:Aussagenlogik}.
	}
}
\newcommand*    {\axiomLetter}{X}%           A[x]iom
\newcommand*    {\Axiom}  [1][]{\glsIdx  [#1]{Axiom}}
\newcommand*    {\Axiome} [1][]{\glsIdxPl[#1]{Axiom}}
\newcommand*    {\Axiomen}[1][]{\glsIdxPl[#1]{Axiom}n}
\newglossaryentry{Axiom}{
	name        ={Axiom},
	plural      ={Axiome},
	description ={
		Eine \glos{Formel}, die unbewiesen als wahr angesehen wird.
		\\-- Standardsymbole:
		$\axiom$    = ein Axiom,
		$\axiomSet$ = eine Menge von Axiomen
		\\-- Zur Definition \vrefseesub{sub:Schlussregeln} und \vref{sub:ausAxiome}.
	}
}
\newcommand*    {\Axiomensystem} [1][]{\glsIdx  [#1]{Axiomensystem}}
\newcommand*    {\Axiomensysteme}[1][]{\glsIdxPl[#1]{Axiomensystem}}
\newglossaryentry{Axiomensystem}{
	name        ={Axiomensystem},
	plural      ={Axiomensysteme},
	description ={
		Eine Menge von \glos{Axiomen}.
		\\-- Zur Definition \vrefseesub{sub:Schlussregeln} und \vref{sub:ausAxiome}.
	}
}

%B === B === B === B === B === B === B === B === B === B === B === B === B === B

\newcommand*    {\Basisregel} [1][]{\glsIdx  [#1]{Basisregel}}
\newcommand*    {\Basisregeln}[1][]{\glsIdxPl[#1]{Basisregel}}
\newglossaryentry{Basisregel}{
	name        ={Basisregel},
	plural      ={Basisregeln},
	description ={
		Eine \glos{Schlussregel}, die nicht mehr auf andere zurückgeführt wird.
		Obwohl das auch auf die \glos{Identitätsregeln} zutrifft, werden diese hier aber nicht dazu gezählt.
		\\-- Zur Definition \vrefseesub{sub:Basisregeln}.
	}
}
\newcommand*    {\beschraenkt}  [1][]{\glsIdy  [#1]{beschraenkt}{beschränkt}}
\newcommand*    {\beschraenkte} [1][]{\glsIdyPl[#1]{beschraenkt}{beschränkt}}
\newcommand*    {\beschraenkten}[1][]{\glsIdyPl[#1]{beschraenkt}{beschränkt}n}
\newglossaryentry{beschraenkt}{
	name        ={beschränkt},
	plural      ={beschränkte},
	description ={
		Eine \glos{Schlussregel} heißt \glos{beschränkt}, wenn sie nur endlich viele Voraussetzungen und Folgerungen hat.
	}
}
\newcommand*    {\Beweis}  [1][]{\glsIdx  [#1]{Beweis}}
\newcommand*    {\Beweise} [1][]{\glsIdxPl[#1]{Beweis}}
\newcommand*    {\Beweises}[1][]{\glsIdx  [#1]{Beweis}es}
\newcommand*    {\Beweisen}[1][]{\glsIdxPl[#1]{Beweis}n}
\newglossaryentry{Beweis}{
	name        ={Beweis},
	plural      ={Beweise},
	description ={
		Eine zulässige Ableitung von \glos{Folgerungen} aus gegebenen \glos{Voraussetzungen}.
		\\-- \vrefSeesec{sec:BeweiseASBA}.
	}
}
\newcommand*    {\beweisbar} [1][]{\glsIdx  [#1]{beweisbar}}
\newcommand*    {\beweisbare}[1][]{\glsIdxPl[#1]{beweisbar}}
\newglossaryentry{beweisbar}{
	name        ={beweisbar},
	plural      ={beweisbare},
	description ={
		Synonym zu \glos{ableitbar}.
	}
}
\newcommand*{\proofstepLetter}   {B}%                [B]eweisschritt
\newcommand*{\proofstepTupLetter}{S}% Tupel/Folge von Beweisschritten, [s]equenz
\newcommand*    {\Beweisschritt}  [1][]{\glsIdx  [#1]{Beweisschritt}}
\newcommand*    {\Beweisschritte} [1][]{\glsIdxPl[#1]{Beweisschritt}}
\newcommand*    {\Beweisschritten}[1][]{\glsIdxPl[#1]{Beweisschritt}n}
\newglossaryentry{Beweisschritt}{
	name        ={Beweisschritt},
	plural      ={Beweisschritte},
	symbol      ={\proofstepLetter},
	description ={
		Eine Vorschrift, wie aus vorgegebenen \glos{Aussagen} (den \glos{Voraussetzungen}) weitere (die \glos{Folgerungen}) folgen.
		\\-- Standardsymbole:
		$\proofstep$        = ein Beweisschritt,
		$\proofstepTup$ = eine Folge von Beweisschritten,
		$\proofstepSet$     = eine Menge von Beweisschritten
		\\-- Zur Definition \vrefseesub{sub:Beweisschritte}.
	}
}
\newcommand*    {\Beweisschrittfolge} [1][]{\glsIdx  [#1]{Beweisschrittfolge}}
\newcommand*    {\Beweisschrittfolgen}[1][]{\glsIdxPl[#1]{Beweisschrittfolge}}
\newglossaryentry{Beweisschrittfolge}{
	name        ={Beweisschrittfolge},
	plural      ={Beweisschrittfolgen},
	description ={
		Eine Folge von \glos{Beweisschritten}.
		\\-- Zur Definition \vrefseesub{sub:Beweisschritte}.
	}
}
\newcommand*    {\Beweisschrittmenge} [1][]{\glsIdx  [#1]{Beweisschrittmenge}}
\newcommand*    {\Beweisschrittmengen}[1][]{\glsIdxPl[#1]{Beweisschrittmenge}}
\newglossaryentry{Beweisschrittmenge}{
	name        ={Beweisschrittmenge},
	plural      ={Beweisschrittmengen},
	description ={
		Eine Menge von \glos{Beweisschritten}, insbesondere die Menge der Glieder einer \glos{Beweisschrittfolge}.
		\\-- Zur Definition \vrefseesub{sub:Beweisschritte}.
	}
}
\newcommand*    {\BoolscheSignatur} [1][]{\glsIdy  [#1]{Boolsche-Signatur}{Signatur, Boolsche}}
\newcommand*    {\BoolschenSignatur}[1][]{\glsIdyPl[#1]{Boolsche-Signatur}{Signatur, Boolsche}}
\newglossaryentry{Boolsche-Signatur}{
	name        ={Signatur, Boolsche},
	text        ={Boolsche Signatur},
	plural      ={Boolsche Signatur},
	description ={
		Die \glos{logische Signatur} $\{\lnot, \land, \lor\}$.
	}
}

%D === D === D === D === D === D === D === D === D === D === D === D === D === D

\newcommand*    {\Definition}  [1][]{\glsIdx  [#1]{Definition}}
\newcommand*    {\Definitionen}[1][]{\glsIdxPl[#1]{Definition}}
\newglossaryentry{Definition}{
	name        ={Definition},
	plural      ={Definitionen},
	description ={
		Eine Definition mit Hilfe des Symbols \chrqt{$\defeq$}.
		\seqqt{$A \defeq B$} steht für \enquote{$A$ \emph{ist definitionsgemäß gleich} $B$} für \glos{Objekte} $A$ und $B$.
		Gewissermaßen ist $A$ nur eine andere Schreibweise für $B$.
		\\-- Man vergleiche auch den Begriff \enquote{\glos{Metadefinition}} und das zugehörige \glos{Symbol} \chrqt{$\metadefeq$}.
		\\-- Zur Definition \vrefseesub{subsub:Definitionen}.
	}
}
\newcommand*    {\Definitionsbereich} [1][]{\glsIdx  [#1]{Definitionsbereich}}
\newcommand*    {\Definitionsbereiche}[1][]{\glsIdxPl[#1]{Definitionsbereich}}
\newglossaryentry{Definitionsbereich}{
	name        ={Definitionsbereich},
	plural      ={Definitionsbereiche},
	description ={
		einer \glos{Funktion}.
		\\-- Symbol: %\DbSymbol%
		\\-- Zur genaueren Definition \vrefseesub{sub:weitereBezeichnungen}.
	}
}

%E === E === E === E === E === E === E === E === E === E === E === E === E === E

\newcommand*{\outcomeLetter}{O}%                 Ergebnis, [o]utcome
\newcommand*    {\Ergebnis}  [1][]{\glsIdx  [#1]{Ergebnis}}
\newcommand*    {\Ergebnisse}[1][]{\glsIdxPl[#1]{Ergebnis}}
\newglossaryentry{Ergebnis}{
	name        ={Ergebnis},
	plural      ={Ergebnisse},
	description ={
		Ein \glos{Ergebnis} eines \glos{Beweises}.
		\\-- Standardsymbole:
		$\outcome$    = ein Ergebnis
		$\outcomeSet$ = eine Menge von Ergebnissen
		$\outcomeRel$ = eine Relation (als Menge aufgefasst) aus Ergebnissen
		\\-- Zur Definition \vrefseesub{sub:Beweise}.
	}
}
\newcommand*    {\Ergebnismenge} [1][]{\glsIdx  [#1]{Ergebnismenge}}
\newcommand*    {\Ergebnismengen}[1][]{\glsIdxPl[#1]{Ergebnismenge}}
\newglossaryentry{Ergebnismenge}{
	name        ={Ergebnismenge},
	plural      ={Ergebnismengen},
	description ={
		Die Menge der \glos{Ergebnisse} eines \glos{Beweises}.
		\\-- Standardsymbol:
		$\outcomeSet$
		\\-- Zur Definition \vrefseesub{sub:Beweise}.
	}
}

%F === F === F === F === F === F === F === F === F === F === F === F === F === F

\newcommand*    {\Fachbegriff}  [1][]{\glsIdx  [#1]{Fachbegriff}}
\newcommand*    {\Fachbegriffe} [1][]{\glsIdxPl[#1]{Fachbegriff}}
\newcommand*    {\Fachbegriffen}[1][]{\glsIdxPl[#1]{Fachbegriff}n}
\newglossaryentry{Fachbegriff}{
	name        ={Fachbegriff},
	plural      ={Fachbegriffe},
	description ={
		Ein Name für einen mathematischen Begriff.
	}
}
\newcommand*    {\Fachgebiet}  [1][]{\glsIdx  [#1]{Fachgebiet}}
\newcommand*    {\Fachgebiete} [1][]{\glsIdxPl[#1]{Fachgebiet}}
\newcommand*    {\Fachgebieten}[1][]{\glsIdxPl[#1]{Fachgebiet}n}
\newglossaryentry{Fachgebiet}{
	name        ={Fachgebiet},
	plural      ={Fachgebiete},
	description ={
		Ein Teil der Mathematik mit einer zugehörigen Basis aus \glos{Axiomen}, \glos{Sätzen}, \glos{Fachbegriffen} und Darstellungsweisen.
	}
}
\newcommand*{\conclusionLetter}   {f}%           [F]olgerung
\newcommand*{\conclusionSetLetter}{F}%           [F]olgerungen
\newcommand*    {\Folgerung}  [1][]{\glsIdx  [#1]{Folgerung}}
\newcommand*    {\Folgerungen}[1][]{\glsIdxPl[#1]{Folgerung}}
\newglossaryentry{Folgerung}{
	name        ={Folgerung},
	plural      ={Folgerungen},
	description ={
		Die \glos{Folgerungen} einer \glos{Schlussregel} $\frac{\prerequisiteSet}{\conclusionSet}$ sind die Elemente von $\conclusionSet$.
		\\-- Standardsymbole:
		$\conclusion$    = eine Folgerung
		$\conclusionSet$ = eine Menge von Folgerungen
		$\conclusionRel$ = eine Relation (als Menge aufgefasst) aus Folgerungen
		\\-- Zur Definition \vrefseesub{sub:Schlussregeln}.
	}
}
%%%\newcommand*    {\Folgerungsmenge} [1][]{\glsIdx  [#1]{Folgerungsmenge}}
%%%\newcommand*    {\Folgerungsmengen}[1][]{\glsIdxPl[#1]{Folgerungsmenge}}
%%%\newglossaryentry{Folgerungsmenge}{
%%%	name        ={Folgerungsmenge},
%%%	plural      ={Folgerungsmengen},
%%%	description ={
%%%		Die Menge der \glos{Folgerungen} einer \glos{Schlussregel} \textbzw\ eines \glos{Beweises}.
%%%		\\-- Standardsymbol:
%%%		$\conclusionSet$
%%%		\\-- Zur Definition \vrefseesub{:Schlussregeln}.
%%%	}
%%%}
\newcommand*    {\formalerSatz} [1][]{\glsIdy  [#1]{formaler-Satz}{Satz, formal}}
\newcommand*    {\formalenSatz} [1][]{\glsIdyPl[#1]{formaler-Satz}{Satz, formal}}
\newglossaryentry{formaler-Satz}{
	name        ={Satz, formal},
	text        ={formaler Satz},
	plural      ={formalen Satz},% Akkusativ
	description ={
		Formale Darstellung eines mathematischen \glos{Satzes}.
		\\-- Siehe~\glos{FS}; zur Definition \vrefseesub{sub:Schlussregeln}.
	}
}
\newcommand*    {\Formel} [1][]{\glsIdx  [#1]{Formel}}
\newcommand*    {\Formeln}[1][]{\glsIdxPl[#1]{Formel}}
\newglossaryentry{Formel}{
	name        ={Formel},
	plural      ={Formeln},
	description ={
		Unter einer \glos{Formel} verstehen wir stets eine mathematische \glos{Formel}.
		Diese kann aus einem einzigen \glos{Symbol} bestehen (\glos{atomare Formel}), andererseits aber auch mehrdimensional sein, lässt sich dann aber mittels geeigneter \glos{Definitionen} immer eindeutig als eine \glos{Zeichenfolge} schreiben.
		\glos{Sätze}, \glos{Beweise} und \glos{Schlussregeln} betrachten wir \emph{nicht} als \glos{Formeln}.
		\\-- Zur Definition \vrefseesub{sub:Bezeichnungen}
		\\-- Zur Definition \vrefseesubsub{subsub:Formeln}.
	}
}
\newcommand*    {\Formelmenge} [1][]{\glsIdx  [#1]{Formelmenge}}
\newcommand*    {\Formelmengen}[1][]{\glsIdxPl[#1]{Formelmenge}}
\newglossaryentry{Formelmenge}{
	name        ={Formelmenge},
	plural      ={Formelmengen},
	description ={
		Eine Menge von \glos{Formeln}, oft mit $\formulaSet$ bezeichnet.
		Man nennt $\formulaSet$ auch eine \glos{Sprache} und ihre Elemente \glos{Worte}, insbesondere dann, wenn es eindeutige Regeln zur Konstruktion von $\formulaSet$ gibt.
		Wir bevorzugen \enquote{\glos{Formel}} und \enquote{\glos{Formelmenge}}.
	}
}
\newcommand*    {\Funktion}  [1][]{\glsIdx  [#1]{Funktion}}
\newcommand*    {\Funktionen}[1][]{\glsIdxPl[#1]{Funktion}}
\newglossaryentry{Funktion}{
	name        ={Funktion},
	plural      ={Funktionen},
	description ={
		Eine \glos{$n$-stellige Funktion} $f$ von einer Menge $A = A_1 \times \dots \times A_n$, dem \glos{Definitionsbereich}, in eine Menge $B$, den \glos{Zielbereich}, ist eine ($n$+1)-stellige \glos{Relation} $(G,A_1,\dots,A_n,B)$ derart, dass es für jedes $\vec{a} = (a_1,\dots,a_n)$ mit $a_i \in A_i$ genau ein $b \in B$ gibt mit $(a_1,\dots,a_n,b) \in f$.
		Dieses $b$ wird auch mit \seqqt{$f(a_1,\dots,a_n)$} , \seqqt{$f a_1 \dots a_n$} , \seqqt{$f(\vec{a})$} oder \seqqt{$f\vec{a}$} bezeichnet.
		\\Schreibweise: \seqqt{$f : A \rightarrow B$} \textbzw\ \seqqt{$f : A_1 \times \dots \times A_n \rightarrow B$}
		\\-- Zur Definition \vrefseesec{sub:weitereBezeichnungen}.
	}
}
\newcommand*    {\Funktionswert} [1][]{\glsIdx  [#1]{Funktionswert}}
\newcommand*    {\Funktionswerte}[1][]{\glsIdxPl[#1]{Funktionswert}}
\newglossaryentry{Funktionswert}{
	name        ={Funktionswert},
	plural      ={Funktionswerte},
	description ={
		einer \glos{Funktion}.
		\\-- Zur genaueren Definition \vrefseesub{sub:weitereBezeichnungen}.
	}
}

%G === G === G === G === G === G === G === G === G === G === G === G === G === G

\newcommand*    {\Gleichheit}[1][]{\glsIdx  [#1]{Gleichheit}}
\newglossaryentry{Gleichheit}{
	name        ={Gleichheit},
	description ={
		Eine \glos{Gleichheitsrelation}:
		Zwei Objekte $A$ und $B$ sind \emph{identisch}, $A \eq B$, wenn sie in den \glos{interessierenden Eigenschaften} für $\eq$ übereinstimmen.
		\\-- Zur Definition \vrefseesubsub{subsub:Vergleiche}
	}
}
\newcommand*    {\Gleichheitsrelation}  [1][]{\glsIdx  [#1]{Gleichheitsrelation}}
\newcommand*    {\Gleichheitsrelationen}[1][]{\glsIdxPl[#1]{Gleichheitsrelation}}
\newglossaryentry{Gleichheitsrelation}{
	name        ={Gleichheitsrelation},
	plural      ={Gleichheitsrelationen},
	description ={
		Eine mit \glos{Gleichheit} verwandte \glos{Relation}: $\eq$, $\ne$, $\equiv$ und $\nequiv$.
	}
}
\newcommand*    {\Graph}  [1][]{\glsIdx  [#1]{Graph}}
\newcommand*    {\Graphen}[1][]{\glsIdxPl[#1]{Graph}}
\newglossaryentry{Graph}{
	name        ={Graph},
	plural      ={Graphen},
	description ={
		einer \glos{Funktion} oder \glos{Relation}.
		\\-- Symbol: $\symgraph$
		\\-- Zur genaueren Definition \vrefseesub{sub:weitereBezeichnungen}.
	}
}

%I === I === I === I === I === I === I === I === I === I === I === I === I === I

\newcommand*    {\Identitaetsregel} [1][]{\glsIdy  [#1]{Identitaetsregel}{Identitätsregel}}
\newcommand*    {\Identitaetsregeln}[1][]{\glsIdyPl[#1]{Identitaetsregel}{Identitätsregel}}
\newglossaryentry{Identitaetsregel}{
	name        ={Identitätsregel},
	plural      ={Identitätsregeln},
	description ={
		Eigentlich eine \glos{Basisregel} zur Identität.
		Da die \glos{Identitätsregeln} nur zur Rechtfertigung der \glos{Substitution} verwendet werden, werden sie hier nicht zu den \glos{Basisregeln} gezählt.
		\\-- Zur Definition \vrefseesub{sub:Identitätsregeln}.
	}
}
\newcommand*    {\interessierendeEigenschaft}   [1][]{\glsIdy  [#1]{interessierende-Eigenschaft}{Eigenschaft, interessierende}}
\newcommand*    {\interessierendenEigenschaft}  [1][]{\glsIdyPl[#1]{interessierende-Eigenschaft}{Eigenschaft, interessierende}}
\newcommand*    {\interessierendenEigenschaften}[1][]{\glsIdyPl[#1]{interessierende-Eigenschaft}{Eigenschaft, interessierende}en}
\newglossaryentry{interessierende-Eigenschaft}{
	name        ={Eigenschaft, interessierende},
	text        ={interessierende Eigenschaft},
	plural      ={interessierenden Eigenschaft},%Akkusativ
	description ={
		Solche Eigenschaften von \glos{Objekten}, die im aktuellen Zusammenhang von Interesse sind, \textzB\ einen bestimmten Wert zu haben, Element einer bestimmten Menge zu sein, ein bestimmtes \glos{Objekt} zu bezeichnen, usw.
	}
}

%J === J === J === J === J === J === J === J === J === J === J === J === J === J

\newcommand*    {\Junktor}  [1][]{\glsIdx  [#1]{Junktor}}
\newcommand*    {\Junktoren}[1][]{\glsIdxPl[#1]{Junktor}}
\newglossaryentry{Junktor}{
	name        ={Junktor},
	plural      ={Junktoren},
	description ={
		Eine aussagenlogische \glos{Operation}.
		Da die Werte einer aussagenlogischen \glos{Operation} \glos{Wahrheitswerte} sind, kann man einen \glos{Junktor} auch als \glos{Relation} verstehen.
		\\-- Zur Definition \vrefseesub{sub:weitereBezeichnungen}
		\\-- Zur Definition \vrefseesub{sub:ausJunktorDef}.
	}
}
\newcommand*    {\Junktorsymbol} [1][]{\glsIdx  [#1]{Junktorsymbol}}
\newcommand*    {\Junktorsymbole}[1][]{\glsIdxPl[#1]{Junktorsymbol}}
\newglossaryentry{Junktorsymbol}{
	name        ={Junktorsymbol},
	plural      ={Junktorsymbole},
	description ={
		Ein \glos{Symbol} für einen \glos{Junktor}.%
		\footnote{%
			Entsprechend \emph{Funktionssymbol}, \emph{Operatorsymbol}, \emph{Relationssymbol}, usw.
		}
	}
}

%K === K === K === K === K === K === K === K === K === K === K === K === K === K

\newcommand*    {\Kontraposition}[1][]{\glsIdx  [#1]{Kontraposition}}
\newglossaryentry{Kontraposition}{
	name        ={Kontraposition},
	description ={
		Die allgemeingültige \glos{Aussage}: $ (\alpha \limp \beta) \limp (\lnot\beta \limp \lnot\alpha) $.
	}
}
\newcommand*    {\Kontravalenz}[1][]{\glsIdx  [#1]{Kontravalenz}}
\newglossaryentry{Kontravalenz}{
	name        ={Kontravalenz},
	description ={
		Eine \glos{Gleichheitsrelation}:
		Zwei Objekte $A$ und $B$ sind \emph{nicht gleich} (nicht äquivalent), $A \nequiv B$, wenn sie in mindestens einer \glos{interessierenden Eigenschaft} für $\equiv$ nicht übereinstimmen.
		\\-- Zur Definition \vrefseesubsub{subsub:Vergleiche}.
	}
}

%L === L === L === L === L === L === L === L === L === L === L === L === L === L

\newcommand*    {\logischeSignatur}  [1][]{\glsIdy  [#1]{logische-Signatur}{Signatur, logische}}
\newcommand*    {\logischeSignaturen}[1][]{\glsIdy  [#1]{logische-Signatur}{Signatur, logische}en}
\newcommand*    {\logischenSignatur} [1][]{\glsIdyPl[#1]{logische-Signatur}{Signatur, logische}}
\newglossaryentry{logische-Signatur}{
	name        ={Signatur, logische},
	text        ={logische Signatur},
	plural      ={logischen Signatur},% Akkusativ
	description ={
		Eine Teilmenge von $\alJun$, ausreichend um damit alle anderen Elemente von $\alJun$ zu definieren.
	}
}

%M === M === M === M === M === M === M === M === M === M === M === M === M === M

\newcommand*    {\Mengenlehre}[1][]{\glsIdx  [#1]{Mengenlehre}}
\newglossaryentry{Mengenlehre}{
	name={Mengenlehre},
	description ={
		-- Zur Definition \vrefseesec{sec:Mengenlehre}.
	}
}
\newcommand*    {\Metadefinition}  [1][]{\glsIdx  [#1]{Metadefinition}}
\newcommand*    {\Metadefinitionen}[1][]{\glsIdxPl[#1]{Metadefinition}}
\newglossaryentry{Metadefinition}{
	name        ={Metadefinition},
	plural      ={Metadefinitionen},
	description ={
		Eine \glos{Definition} in der \glos{Metasprache} mit Hilfe des \emph{Metadefinitionssymbols} \chrqt{$\metadefeq$}.
		\seqqt{$A \metadefeq B$} steht für \enquote{$A$ \emph{ist definitionsgemäß gleich} $B$} für \glos{Aussagen} $A$ und $B$.
		Gewissermaßen ist $A$ nur eine andere Schreibweise für $B$.
		\\-- Man vergleiche auch den Begriff \enquote{\glos{Definition}} und das zugehörige \glos{Symbol} \chrqt{$\defeq$}.
		\\-- Zur Definition \vrefseesubsub{subsub:Definitionen}.
	}
}
\newcommand*    {\Metaoperation}  [1][]{\glsIdx  [#1]{Metaoperation}}
\newcommand*    {\Metaoperationen}[1][]{\glsIdxPl[#1]{Metaoperation}}
\newglossaryentry{Metaoperation}{
	name        ={Metaoperation},
	plural      ={Metaoperationen},
	description ={
		Eine \glos{Operation} der \glos{Metasprache}: $\metaand$, $\metaor$ oder $\srand$.
		\\-- Zur Definition \vrefseesub{sub:AussagenUndMetaoperationen}.
	}
}
\newcommand*    {\Metarelation}  [1][]{\glsIdx  [#1]{Metarelation}}
\newcommand*    {\Metarelationen}[1][]{\glsIdxPl[#1]{Metarelation}}
\newglossaryentry{Metarelation}{
	name        ={Metarelation},
	plural      ={Metarelationen},
	description ={
		Eine \glos{Relation} der \glos{Metasprache}: $\metaimp$, $\metarep$ oder $\metaequiv$.
		\\-- Zur Definition \vrefseesub{sub:AussagenUndMetaoperationen}.
	}
}
\newcommand*    {\Metasprache} [1][]{\glsIdx  [#1]{Metasprache}}
\newcommand*    {\Metasprachen}[1][]{\glsIdxPl[#1]{Metasprache}}
\newglossaryentry{Metasprache}{
	name        ={Metasprache},
	plural      ={Metasprachen},
	description ={
		Eine Sprache, in der \glos{Aussagen} über Elemente einer anderen Sprache getroffen werden können.
		In diesem Dokument ist dies immer die normale Sprache.
		\\-- \vrefSeesec{sec:Metasprache}.
	}
}
\newcommand*    {\Monotonieregel}[1][]{\glsIdx  [#1]{Monotonieregel}}
\newglossaryentry{Monotonieregel}{
	name        ={Monotonieregel},
	description ={
		Eine \glos{Schlussregel}. -- siehe~\glos{MR}.
	}
}

%O === O === O === O === O === O === O === O === O === O === O === O === O === O

\newcommand*    {\Objekt} [1][]{\glsIdx  [#1]{Objekt}}
\newcommand*    {\Objekte}[1][]{\glsIdxPl[#1]{Objekt}}
\newcommand*    {\Objekts}[1][]{\glsIdx  [#1]{Objekt}s}
\newglossaryentry{Objekt}{
	name        ={Objekt},
	plural      ={Objekte},
	description ={
		\glos{Symbole}, \glos{Formeln} und \glos{Aussagen} sowie Mengen, \glos{Zeichenfolgen}, Zahlen; ganz allgemein reale oder gedachte Dinge an sich.
		\\-- Zur Definition \vrefseesub{sub:Bezeichnungen}.
	}
}
\newcommand*    {\Operation}  [1][]{\glsIdx  [#1]{Operation}}
\newcommand*    {\Operationen}[1][]{\glsIdxPl[#1]{Operation}}
\newglossaryentry{Operation}{
	name        ={Operation},
	plural      ={Operationen},
	description ={
		Eine -- meistens binäre, \textdh\ zweiwertige -- Funktion $M^n \rightarrow M$.
		Für eine binäre \glos{Operation} $\opbsp : M \times M \rightarrow M$ schreibt man meistens $x \opbsp y$ statt $\opbsp(x,y)$.
		\\-- Zur Definition \vrefseesub{sub:weitereBezeichnungen}
		\\-- \vrefSeesub{sub:Beispielsymbole} und \vref{sub:Operationen}.
	}
}
\newcommand*    {\Operationssymbol} [1][]{\glsIdx  [#1]{Operationssymbol}}
\newcommand*    {\Operationssymbole}[1][]{\glsIdxPl[#1]{Operationssymbol}}
\newglossaryentry{Operationssymbol}{
	name        ={Operationssymbol},
	plural      ={Operationssymbole},
	description ={
		Ein \glos{Symbol} für eine \glos{Operation}.
	}
}

%P === P === P === P === P === P === P === P === P === P === P === P === P === P

\newcommand*    {\PolnischeNotation}  [1][]{\glsIdy  [#1]{Polnische-Notation}{Notation, Polnische}}
\newcommand*    {\PolnischeNotationen}[1][]{\glsIdy  [#1]{Polnische-Notation}{Notation, Polnische}en}
\newcommand*    {\PolnischenNotation} [1][]{\glsIdyPl[#1]{Polnische-Notation}{Notation, Polnische}}
\newglossaryentry{Polnische-Notation}{
	name        ={Notation, Polnische},
	text        ={Polnische Notation},
	plural      ={Polnischen Notation},% Akkusativ
	description ={
		Bei der \glos{Polnischen Notation} stehen die Operanden \textbzw\ Argumente von \glos{Relationen} und \glos{Funktionen} stets rechts von den Relations- und Funktionssymbolen.
		Dadurch kann auf Gliederungszeichen wie Klammern und Kommata verzichtet werden.
		Noch einfacher für Computer ist die \defn{umgekehrte} \glos{Polnische Notation}, bei der die Operanden und Argumente links von den Symbolen stehen.
	}
}
\newcommand*    {\Potenzmenge} [1][]{\glsIdx  [#1]{Potenzmenge}}
\newcommand*    {\Potenzmengen}[1][]{\glsIdxPl[#1]{Potenzmenge}}
\newglossaryentry{Potenzmenge}{
	name        ={Potenzmenge},
	plural      ={Potenzmengen},
	description ={
		Die \glos{Potenzmenge} $\Pot(M)$ einer Menge $M$ ist die Menge ihrer Teilmengen.
		\\-- Zur Definition \vrefseesub{sub:Bezeichnungen}.
	}
}
\newcommand*    {\Praedikat} [1][]{\glsIdy  [#1]{Praedikat}{Prädikat}}
\newcommand*    {\Praedikate}[1][]{\glsIdyPl[#1]{Praedikat}{Prädikat}}
\newglossaryentry{Praedikat}{
	name        ={Prädikat},
	plural      ={Prädikate},
	description ={
		Ein Element der \glos{Prädikatenlogik}.
		\\-- Zur Definition \vrefseesec{sec:Prädikatenlogik}.
		\\\textZB\ kann man eine Gruppe als ein zweistelliges \glos{Prädikat} $\mathrm{Gruppe}(G,+)$ definieren, in dem $G$ eine Menge und $+$ eine \glos{Operation}, \textdh\ eine binäre (zweistellige) Funktion $ +: G \times G \rightarrow G $ ist, so dass die Gruppenaxiome erfüllt sind.
	}
}
\newcommand*    {\Praedikatenlogik}[1][]{\glsIdy  [#1]{Praedikatenlogik}{Prädikatenlogik}}
\newglossaryentry{Praedikatenlogik}{
	name={Prädikatenlogik},
	description ={
		-- Zur Definition \vrefseesec{sec:Prädikatenlogik}.
	}
}

%R === R === R === R === R === R === R === R === R === R === R === R === R === R

\newcommand*    {\Relation}  [1][]{\glsIdx  [#1]{Relation}}
\newcommand*    {\Relationen}[1][]{\glsIdxPl[#1]{Relation}}
\newglossaryentry{Relation}{
	name        ={Relation},
	plural      ={Relationen},
	description ={
		Eine \glos{$n$-stellige Relation} $R$ ist ein (1+$n$)-Tupel $(G,A_1,\dots,A_n$) mit $G \subseteq A_1 \times \dots \times A_n)$.
		\\-- Zur genaueren Definition \vrefseesub{sub:weitereBezeichnungen}
 		\\-- \vrefSeesub{sub:Beispielsymbole} und \vref{sub:Gleichheit}.
	}
}

%S === S === S === S === S === S === S === S === S === S === S === S === S === S

\newcommand*    {\Satz}   [1][]{\glsIdx  [#1]{Satz}}
\newcommand*    {\Saetze} [1][]{\glsIdxPl[#1]{Satz}}
\newcommand*    {\Satzes} [1][]{\glsIdx  [#1]{Satz}e}
\newcommand*    {\Saetzen}[1][]{\glsIdxPl[#1]{Satz}n}
\newglossaryentry{Satz}{
	name        ={Satz},
	plural      ={Sätze},
	description ={
		Eine mathematische \glos{Aussage}, dass bestimmte \glos{Folgerungen} aus gegebenen \glos{Voraussetzungen} abgeleitet werden können.
	}
}
\newcommand*{\conclusionruleLetter}{C}%             Schlussregel, [c]onclusion
\newcommand*    {\Schlussregel} [1][]{\glsIdx  [#1]{Schlussregel}}
\newcommand*    {\Schlussregeln}[1][]{\glsIdxPl[#1]{Schlussregel}}
\newglossaryentry{Schlussregel}{
	name        ={Schlussregel},
	plural      ={Schlussregeln},
	see         ={allgemeingueltig},
	description ={
		Eine \glos{Schlussregel} $\frac{\prerequisiteSet}{\conclusionSet}$ entspricht der \glos{Aussage}:
		\begin{quote}
			Wenn alle \glos{Voraussetzungen} $\prerequisite$ aus $\prerequisiteSet$ zutreffen, dann auch alle \glos{Folgerungen} $\conclusion$ aus $\conclusionSet$.
		\end{quote}
		Wenn diese \glos{Aussage} zutrifft, kann die Schlussregel zur \glos{zulässigen Transformation} von \glos{Formeln} dienen.
		\\-- Standardsymbole:
		$\conclusionrule$    = eine Schlussregel
		$\conclusionruleSet$ = eine Menge von Schlussregeln
		\\-- Zur Definition \vrefseesub{sub:Schlussregeln}.
	}
}
\newcommand*    {\Schlussregelmenge} [1][]{\glsIdx  [#1]{Schlussregelmenge}}
\newcommand*    {\Schlussregelmengen}[1][]{\glsIdxPl[#1]{Schlussregelmenge}}
\newglossaryentry{Schlussregelmenge}{
	name        ={Schlussregelmenge},
	plural      ={Schlussregelmengen},
	description ={
		Eine Menge von \glos{Schlussregeln}, meistens mit $\conclusionruleSet$ bezeichnet.
		\\-- Zur Definition \vrefseesub{:Schlussregeln}.
	}
}
\newcommand*    {\Schnittregel}[1][]{\glsIdx  [#1]{Schnittregel}}
\newglossaryentry{Schnittregel}{
	name        ={Schnittregel},
	plural      ={Schnittregeln},
	description ={
		Eine \glos{allgemeingültige Schlussregel}.
		\\-- Siehe~\glos{SR}.
	}
}
\newcommand*    {\Sprache} [1][]{\glsIdx  [#1]{Sprache}}
\newcommand*    {\Sprachen}[1][]{\glsIdxPl[#1]{Sprache}}
\newglossaryentry{Sprache}{
	name        ={Sprache},
	plural      ={Sprachen},
	description ={
		-- Siehe \glos{Formelmenge}.
	}
}
\newcommand*    {\Stelligkeit}  [1][]{\glsIdx  [#1]{Stelligkeit}}
\newcommand*    {\Stelligkeiten}[1][]{\glsIdxPl[#1]{Stelligkeit}}
\newglossaryentry{Stelligkeit}{
	name        ={Stelligkeit},
	plural      ={Stelligkeiten},
	description ={
		einer \glos{Funktion} oder \glos{Relation}.
		\\-- Symbole:
		$\stelfunc$ = Stelligkeit einer Funktion,
		$\stelrel$  = Stelligkeit einer Relation,
		\\-- Zur genaueren Definition \vrefseesub{sub:weitereBezeichnungen}.
	}
}
\newcommand*    {\substitutionLetter}{E}%            Substitution, [E]rsetzung
\newcommand*    {\Substitution}  [1][]{\glsIdx  [#1]{Substitution}}
\newcommand*    {\Substitutionen}[1][]{\glsIdxPl[#1]{Substitution}}
\newglossaryentry{Substitution}{
	name        ={Substitution},
	plural      ={Substitutionen},
	description ={
		Eine \glos{Funktion} zur \glos{Transformation} einer \glos{Formel} mittels \glos{Substitution} in eine gleichwertige.
		Die \glos{Substitution} heißt \defn{zulässig}, wenn sie vorgegebene Regeln erfüllt.
		\\-- Zur Definition \vrefseesub{sub:Beweise}.
	}
}
\newcommand*    {\Substitutionsmenge} [1][]{\glsIdx  [#1]{Substitutionsmenge}}
\newcommand*    {\Substitutionsmengen}[1][]{\glsIdxPl[#1]{Substitutionsmenge}}
\newglossaryentry{Substitutionsmenge}{
	name        ={Substitutionsmenge},
	plural      ={Substitutionsmengen},
	description ={
		Eine Menge von \glos{Substitutionen}, meistens mit $\substitutionSet$ bezeichnet.
	}
}
\newcommand*    {\Symbol}  [1][]{\glsIdx  [#1]{Symbol}}
\newcommand*    {\Symbole} [1][]{\glsIdxPl[#1]{Symbol}}
\newcommand*    {\Symbols} [1][]{\glsIdx  [#1]{Symbol}s}
\newcommand*    {\Symbolen}[1][]{\glsIdxPl[#1]{Symbol}n}
\newglossaryentry{Symbol}{
	name        ={Symbol},
	plural      ={Symbole},
	description ={
		Ein \defn{einfaches} \glos{Symbol} ist ein druckbares typographisches Zeichen.
		Ein \defn{zusammengesetztes} \glos{Symbol} besteht aus mehreren einfachen \glos{Symbolen}.
		In beiden Fällen wird ein \glos{Symbol} als \defn{\glos{unzerlegbar}} angesehen.
		\\-- Zur Definition \vrefseesec{sec:Notationen}.
	}
}

%T === T === T === T === T === T === T === T === T === T === T === T === T === T

\newcommand*    {\Traegermenge} [1][]{\glsIdy  [#1]{Traegermenge}{Trägermenge}}
\newcommand*    {\Traegermengen}[1][]{\glsIdyPl[#1]{Traegermenge}{Trägermenge}}
\newglossaryentry{Traegermenge}{
	name        ={Trägermenge},
	plural      ={Trägermengen},
	description ={
		einer \glos{Relation}.
		\\-- Symbol: $\traeger$
		\\-- Zur genaueren Definition \vrefseesub{sub:weitereBezeichnungen}.
	}
}
\newcommand*        {\transformationLetter}{T}%           [T]ransformation
\newcommand*        {\Transformation}  [1][]{\glsIdx  [#1]{Transformation}}
\newcommand*        {\Transformationen}[1][]{\glsIdxPl[#1]{Transformation}}
\longnewglossaryentry{Transformation}{
	name            ={Transformation},
	plural          ={Transformationen},
}
{
	Eine Umformung oder Erzeugung einer \glos{Formel} aus einer vorgegebenen Menge von \glos{Formeln}, \textdh\ die Anwendung einer \glos{Schlussregel}.

	Eine \glos{Transformation} heißt \defn{zulässig}, wenn sie Element einer vorgegebenen Menge von \glos{Transformationen} oder eine daraus zulässigerweise abgeleitete \glos{Transformation} ist.

	Standardsymbole:
	$\transformation$    = eine Transformation,
	$\transformationTup$ = eine Folge von Transformationen
}
\newcommand*    {\Transformationsfolge} [1][]{\glsIdx  [#1]{Transformationsfolge}}
\newcommand*    {\Transformationsfolgen}[1][]{\glsIdxPl[#1]{Transformationsfolge}}
\newglossaryentry{Transformationsfolge}{
	name        ={Transformationsfolge},
	plural      ={Transformationsfolgen},
	description ={
		Eine Folge von \glos{Transformationen}.
		\\-- Standardsymbol: $\transformationTup$
		\\-- Zur Definition \vrefseesub{sub:Beweisschritte}.
	}
}
\newcommand*    {\Tupel} [1][]{\glsIdx  [#1]{Tupel}}
\newglossaryentry{Tupel}{
	name        ={Tupel},
	plural      ={Tupel},
	description ={
		Ein $n$-\glos{Tupel} $\vec{a}$ ist eine endliche Folge $\vec{a} = (a_1, \dots, a_n)$ seiner Komponenten $a_i$.
		\\-- Zur Definition \vrefseesub{sub:weitereBezeichnungen}.
	}
}
\newcommand*    {\Tupelmenge} [1][]{\glsIdx  [#1]{Tupelmenge}}
\newcommand*    {\Tupelmengen}[1][]{\glsIdxPl[#1]{Tupelmenge}}
\newglossaryentry{Tupelmenge}{
	name        ={Tupelmenge},
	plural      ={Tupelmengen},
	description ={
		Die \glos{Tupelmenge} $\tupelSet(M)$ einer Menge $M$ ist die Menge aller $n$-Tupel aus $M^n$ für alle $n \in \INo$.
		\\-- Zur Definition \vrefseesub{sub:Bezeichnungen}.
	}
}

%U === U === U === U === U === U === U === U === U === U === U === U === U === U

\newcommand*    {\Umkehrrelation}  [1][]{\glsIdx  [#1]{Umkehrrelation}}
\newcommand*    {\Umkehrrelationen}[1][]{\glsIdxPl[#1]{Umkehrrelation}}
\newglossaryentry{Umkehrrelation}{
	name        ={Umkehrrelation},
	plural      ={Umkehrrelationen},
	description ={
		Die \glos{Umkehrrelation} zu einer binären \glos{Relation} $(G,A,B)$ ist die \glos{Relation} $(\{(b,a)|(a,b) \in G\},B,A)$.
		Üblicherweise wird das zugehörige Relationssymbol gespiegelt.
	}
}
\newcommand*    {\Ungleichheit}[1][]{\glsIdx  [#1]{Ungleichheit}}
\newglossaryentry{Ungleichheit}{
	name        ={Ungleichheit},
	description ={
		Eine \glos{Gleichheitsrelation}:
		Zwei Objekte $A$ und $B$ sind \emph{nicht identisch}, $A \ne B$, wenn sie in mindestens einer \glos{interessierenden Eigenschaft} für $\eq$ nicht übereinstimmen.
		\\-- Zur Definition \vrefseesubsub{subsub:Vergleiche}.
	}
}
\newcommand*    {\unzerlegbar} [1][]{\glsIdx  [#1]{unzerlegbar}}
\newcommand*    {\unzerlegbare}[1][]{\glsIdxPl[#1]{unzerlegbar}}
\newglossaryentry{unzerlegbar}{
	name        ={unzerlegbar},
	plural      ={unzerlegbare},
	description ={
		Eine \glos{Aussage}, die keine \glos{Metaoperation}, \textbzw\ eine \glos{Formel}, die keine \glos{Operation} und keine \glos{Relation} enthält, heißt \glos{unzerlegbar}.
		\\-- Synonym: \glos{atomar}; vergleiche auch \glos{zerlegbar}.
	}
}

%V === V === V === V === V === V === V === V === V === V === V === V === V === V

\newcommand*    {\vergleichbar} [1][]{\glsIdx  [#1]{vergleichbar}}
\newcommand*    {\vergleichbare}[1][]{\glsIdxPl[#1]{vergleichbar}}
\newglossaryentry{vergleichbar}{
	name        ={vergleichbar},
	plural      ={vergleichbare},
	description ={
		Zwei \glos{Objekte} $A$ und $B$ sind \glos{vergleichbar}, wenn beide von derselben Art sind, \textdh\ wenn beide \textzB\ jeweils Mengen, \glos{Zeichenfolgen}, Zahlen, \textusw\ sind.
		Dabei muss bei \glos{Formeln} zwischen der \glos{Formel} an sich und dem Ergebnis der \glos{Formel} unterschieden werden.
		\\-- Zur Definition \vrefseesub{subsub:Vergleichbar}.
	}
}
\newcommand*    {\Vertauschung}  [1][]{\glsIdx  [#1]{Vertauschung}}
\newcommand*    {\Vertauschungen}[1][]{\glsIdxPl[#1]{Vertauschung}}
\newglossaryentry{Vertauschung}{
	name        ={Vertauschung},
	plural      ={Vertauschungen},
	description ={
		Die \glos{Vertauschung} von zwei unabhängigen Teil-\glos{Formeln} ($\alpha$ und $\beta$) in einer anderen \glos{Formel} ($\gamma$)
		\\-- Formal: $\gamma(\alpha\swap\beta)$.
		Die \glos{Vertauschung} ist eine spezielle Form der \glos{Substitution}.
		\\-- Zur Definition siehe~\eqref{def:Vertauschung} \vrefinsub{sub:Identitätsregeln}.
	}
}
\newcommand*{\prerequisiteLetter}   {v}%             [V]oraussetzung
\newcommand*{\prerequisiteSetLetter}{V}%             [V]oraussetzungen
\newcommand*    {\Voraussetzung}  [1][]{\glsIdx  [#1]{Voraussetzung}}
\newcommand*    {\Voraussetzungen}[1][]{\glsIdxPl[#1]{Voraussetzung}}
\newglossaryentry{Voraussetzung}{
	name        ={Voraussetzung},
	plural      ={Voraussetzungen},
	description ={
		Die \glos{Voraussetzungen} einer \glos{Schlussregel} $\frac{\prerequisiteSet}{\conclusionSet}$ sind die Elemente von $\prerequisiteSet$.
		\\-- Standardsymbole:
		$\prerequisite$    = eine Voraussetzung,
		$\prerequisiteSet$ = eine Menge von Voraussetzungen,
		$\prerequisiteRel$ = eine Relation (als Menge aufgefasst) aus Voraussetzungen
		\\-- Zur Definition \vrefseesub{sub:Schlussregeln}.
	}
}
\newcommand*    {\Voraussetzungsmenge} [1][]{\glsIdx  [#1]{Voraussetzungsmenge}}
\newcommand*    {\Voraussetzungsmengen}[1][]{\glsIdxPl[#1]{Voraussetzungsmenge}}
\newglossaryentry{Voraussetzungsmenge}{
	name        ={Voraussetzungsmenge},
	plural      ={Voraussetzungsmengen},
	description ={
		Die Menge der \glos{Voraussetzungen} einer \glos{Schlussregel} \textbzw\ eines \glos{Beweises}.
		\\-- Standardsymbol:
		$\prerequisiteSet$
		\\-- Zur Definition \vrefseesub{:Schlussregeln}.
	}
}

%W === W === W === W === W === W === W === W === W === W === W === W === W === W

\newcommand*    {\Wahrheitswert}  [1][]{\glsIdx  [#1]{Wahrheitswert}}
\newcommand*    {\Wahrheitswerte} [1][]{\glsIdxPl[#1]{Wahrheitswert}}
\newcommand*    {\Wahrheitswerten}[1][]{\glsIdxPl[#1]{Wahrheitswert}n}
\newglossaryentry{Wahrheitswert}{
	name        ={Wahrheitswert},
	plural      ={Wahrheitswerte},
	description ={
		Die Werte \chrqt{$\ltrue$} und \chrqt{$\lfalse$}, oft auch mit \chrqt{$\wahr$} \textbzw\ \chrqt{$\falsch$}, \chrqt{$\mathrm{true}$} \textbzw\ \chrqt{$\mathrm{false}$} oder einfach \chrqt{$1$} \textbzw\ \chrqt{$0$} bezeichnet.
	}
}
\newcommand*    {\Wort}   [1][]{\glsIdx  [#1]{Wort}}
\newcommand*    {\Worte}  [1][]{\glsIdxPl[#1]{Wort}}
\newcommand*    {\Woerter}[1][]{\glsIdxPl[#1]{Wort}}
\newglossaryentry{Wort}{
	name        ={Wort},
	plural      ={Wörter},
	description ={
		Ein Element einer \glos{Sprache}.
		In dem Fall Synonym zu \glos{Formel}.
		\\-- Siehe \glos{Formelmenge}.
	}
}

%Z === Z === Z === Z === Z === Z === Z === Z === Z === Z === Z === Z === Z === Z

\newcommand*    {\Zeichenfolge} [1][]{\glsIdx  [#1]{Zeichenfolge}}
\newcommand*    {\Zeichenfolgen}[1][]{\glsIdxPl[#1]{Zeichenfolge}}
\newglossaryentry{Zeichenfolge}{
	name        ={Zeichenfolge},
	plural      ={Zeichenfolgen},
	description ={
		Folgen von \glos{Symbolen}, wobei Leerstellen und sonstiger Zwischenraum nicht zählen und nur zur besseren Darstellung dienen.
		Dabei sind als spezielle \glos{Symbole} auch \glos{Zeichenketten} erlaubt, solange die Zerlegung eindeutig bleibt.
		\textZB\ kann \chrqt{sin} als ein einzelnes \glos{Symbol} -- für die Sinusfunktion -- aufgefasst werden, aber auch als Folge der Buchstaben \chrqt{s}, \chrqt{i} und \chrqt{n}.
		\glos{Formeln} werden immer als \glos{Zeichenfolgen} aufgefasst.
		\\-- Siehe auch \glos{Zeichenkette}.
		\\-- Zur Definition \vrefseesub{subsub:Definitionen}.
	}
}
\newcommand*    {\Zeichenkette} [1][]{\glsIdx  [#1]{Zeichenkette}}
\newcommand*    {\Zeichenketten}[1][]{\glsIdxPl[#1]{Zeichenkette}}
\newglossaryentry{Zeichenkette}{
	name        ={Zeichenkette},
	plural      ={Zeichenketten},
	description ={
		Folgen von \glos{Symbolen}, auch Leerstellen und sonstigem Zwischenraum.
		\\-- Siehe auch \glos{Zeichenfolge}.
		\\-- Zur Definition \vrefseesub{subsub:Definitionen}.
	}
}
\newcommand*    {\zerlegbar} [1][]{\glsIdx  [#1]{zerlegbar}}
\newcommand*    {\zerlegbare}[1][]{\glsIdxPl[#1]{zerlegbar}}
\newcommand*    {\Zerlegbare}[1][]{\GlsIdxPl[#1]{zerlegbar}}
\newglossaryentry{zerlegbar}{
	name        ={zerlegbar},
	plural      ={zerlegbare},
	description ={
		Eine \glos{Aussage}, die eine \glos{Metaoperation}, \textbzw\ eine \glos{Formel}, die eine \glos{Operation} oder eine \glos{Relation} enthält, heißen \glos{zerlegbar}.
		\\-- Vergleiche auch \glos{unzerlegbar}.
	}
}
\newcommand*    {\Ziel} [1][]{\glsIdx  [#1]{Ziel}}
\newcommand*    {\Ziele}[1][]{\glsIdxPl[#1]{Ziel}}
\newglossaryentry{Ziel}{
	name        ={Ziel},
	plural      ={Ziele},
	description ={
		In diesem Dokument sind \glos{Ziele} die Anforderungen an \glos{ASBA}.
	}
}
\newcommand*    {\Zielbereich} [1][]{\glsIdx  [#1]{Zielbereich}}
\newcommand*    {\Zielbereiche}[1][]{\glsIdxPl[#1]{Zielbereich}}
\newglossaryentry{Zielbereich}{
	name        ={Zielbereich},
	plural      ={Zielbereiche},
	description ={
		einer \glos{Funktion}.
		\\-- Symbol: $\Zb$
		\\-- Zur genaueren Definition \vrefseesub{sub:weitereBezeichnungen}.
	}
}
\newcommand*    {\zulaessig}  [1][]{\glsIdy [#1]{zulaessig}{zulässig}}
\newcommand*    {\zulaessige} [1][]{\glsIdyPl[#1]{zulaessig}{zulässig}}
\newcommand*    {\zulaessigen}[1][]{\glsIdy  [#1]{zulaessig}{zulässig}en}
\newcommand*    {\zulaessiger}[1][]{\glsIdyPl[#1]{zulaessig}{zulässig}r}
\newglossaryentry{zulaessig}{
	name        ={zulässig},
	plural      ={zulässige},
	description ={
		Ein Attribut einer \glos{Transformation} \textbzw\ \glos{\Substitution}.
	}
}
