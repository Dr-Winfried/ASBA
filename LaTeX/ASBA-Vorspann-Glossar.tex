%%############################################################################%%
%%                                                                            %%
%% Datei:  ASBA-Vorspann-Glossar.tex                                          %%
%% Inhalt: Vorspann Glossareinträge für ASBA                                  %%
%%                                                                            %%
%% Copyright (C) 2017  Winfried Teschers                                      %%
%%                                                                            %%
%% This program is free software: you can redistribute it and/or modify       %%
%% it under the terms of the GNU Affero General Public License as published   %%
%% by the Free Software Foundation, either version 3 of the License, or       %%
%% (at your option) any later version.                                        %%
%%                                                                            %%
%% This program is distributed in the hope that it will be useful,            %%
%% but WITHOUT ANY WARRANTY; without even the implied warranty of             %%
%% MERCHANTABILITY or FITNESS FOR A PARTICULAR PURPOSE.  See the              %%
%% GNU Affero General Public License for more details.                        %%
%%                                                                            %%
%% You should have received a copy of the GNU Affero General Public License   %%
%% along with this program.  If not, see <http://www.gnu.org/licenses/>.      %%
%%                                                                            %%
%% Dr. Winfried Teschers                                                      %%
%% Anton-Günther-Straße 26c                                                   %%
%% 91083 Baiersdorf                                                           %%
%% Germany                                                                    %%
%%                                                                            %%
%% e-mail: winfried.teschers@t-online.de                                      %%
%%                                                                            %%
%%############################################################################%%
% !TeX root = ASBA.tex
% !TeX encoding = UTF-8
% !TeX spellcheck = de_DE

\newglossary[nlg]{symbols}{not}{ntn}{Symbolverzeichnis}

% Elemente, die keine Glossareinträge sind und dafür nicht gebraucht werden,
% werden in "ASBA-Vorspann.tex" und "ASBA-Vorspann-Logik.tex" definiert.

\newcommand*{\SymbolAmRand}[1]{\marginpar{\raggedright{\glsentrysymbol{#1}}}}

\newcommand*{\formaleDefinition}[1]{\hspace{0.5em}\ensuremath{#1}}% Formel
\newcommand*{\informelleDefinition}[1]{\hspace{0.5em}[#1]}%         Text

\newcommand*{\baueBeschreibung}[2][]{% {glo-Label}% Konstruktion einer Beschreibung
	\def\ArgumentEins{#1}
	\def\ASBAundefined{}
	\ifx\ArgumentEins\ASBAundefined\else #1: \fi%  wenn Text, dann Ausgabe
	\glsentrysymbol{#2}%
	\glsentryuservi{#2}%
	\SymbolAmRand  {#2}%
}

% Mögliche Einträge nach \[long]\]newglossary am Start von description:
\newcommand*{\todoBeschreiben}{[Beschreibung fehlt noch ]}% Eintrag muss noch beschrieben werden
\newcommand*{\todoErgaenzen}  {[Beschreibung ergänzen   ]}% Eintrag muss noch ergänzt werden
\newcommand*{\todoPruefen}    {[?]  }% Eintrag muss noch geprüft werden
\newcommand*{\todoGeprueft}   {[!]  }% Eintrag ist geprüft, Link oder Definition im Text noch nicht
\newcommand*{\todoOk}         {[ok] }% Eintrag ist geprüft und ok

% Wikipedia-Einträge; normalerweise nur im Glossar
\newif\ifmitWikiFlg% Schalter für mit "Wikipedia-Eintrag"; AUS
        \mitWikiFlgtrue% ... jetzt EIN
\newcommand{\wikicite}[3][\Wikipedia]{% Kommandoname ohne *!
	\ifmitWikiFlg
		#1 \cite{#2} schreibt dazu:
		\begin{quote}\wikiFamily #3 \end{quote}
	\else
		(Ein Auszug aus #1 \cite{#2} steht im Glossar.)\newline
	\fi
}
\newcommand{\wikiciteChapter}[4][\Wikipedia]{% Kommandoname ohne *!
	\ifmitWikiFlg
		#1 \cite{#2} Kapitel #3 schreibt dazu:
		\begin{quote}\wikiFamily #4 \end{quote}
	\else
		(Ein Auszug aus #1 \cite{#2} Kapitel #3 steht im Glossar.)\par
	\fi
}
% Glossar-Eintrag beinhaltet Teile, die nur für Text bzw. nur im Glossar gelten
\newif\ifimGlossarFlg% Schalter für "im Glossar"; AUS
        \imGlossarFlgtrue% ... jetzt EIN
\newcommand  {\nurImGlossar}[1]{\ifimGlossarFlg #1\else\fi}% Kommandoname ohne *!
\newcommand{\nichtImGlossar}[1]{\ifimGlossarFlg\else #1\fi}% Kommandoname ohne *!
\newcommand*{\glsBeschreibungMitWiki}[1]{%
	\imGlossarFlgfalse
	\glsentrydesc{#1}
	\imGlossarFlgtrue
}
\newcommand*{\glsBeschreibungOhneWiki}[1]{%
	\mitWikiFlgfalse
	\imGlossarFlgfalse
	\glsentrydesc{#1}
	\imGlossarFlgtrue
	\mitWikiFlgtrue
}
\newcommand*{\glsBeschreibung}[1]{\glsBeschreibungOhneWiki{#1}}

% Fonts für die Liste der Seitenangaben
\newcommand*      {\HyperDef}[1]{\textsf{\hyperbf{#1}}}   % für Seitenangaben bei Definitionen
\newcommand*      {\HyperTxt}[1]        {\hyperrm{#1}}    % für Seitenangaben im Glossar
\newcommand*  {\likeHyperDef}[1]{\DefFt {\likeLinkFt{#1}}}% simuliert Verweise bei Definitionen
\newcommand*  {\likeHyperTxt}[1]        {\likeLinkFt{#1}} % simuliert Verweise ins Glossar
\GlsAddXdyAttribute{HyperDef}                             % damit xindy damit umgehen kann
\GlsAddXdyAttribute{HyperTxt}                             % damit xindy damit umgehen kann

% Seitenangaben mit 'f' und 'ff' (für xindy)
\GlsSetXdyMinRangeLength{1}
\glsSetSuffixF{f}
\glsSetSuffixFF{ff}

\makeglossaries
\setacronymstyle{long-sc-short}
\renewcommand*{\glsnumberformat}[1]{\HyperTxt{#1}}% Standardformat für Seitenliste

% Makros für neue Bezeichnungen ================================================
% [<Anhang>]{<Makro>}{<gls-Makro>}{<glo-Label>}
\newcommand*{\newVerweis}  [4][]{\newcommand*{#2}[1][]{#3[##1]{#4}[#1]}}
\newcommand*{\dummyVerweis}[4][]{\newcommand*{#2}[1][]{\likeLinkFt{#4#1}}}% wenn <gls-Makro> noch fehlt

% neue Benennungen -------------------------------------------------------------
% Ausgabe und Aufnahme und Link ins Glossar mit Hervorhebung der Seitennummer
% {<Makro>} - An <Makro> muss [format=<Font-Makro>] angehängt werden können
\newcommand*{\defTxt}[1]{\DefFt{#1[format=HyperDef]}}
\newcommand*{\defGlo}[1]{\DefFt{#1[format=HyperDef]}\footnotemark[0]}
\newcommand*{\footnoteForNotDefinedItem}{\footnotetext[0]{Die vollständige Definition steht nur im Glossar.}}% Fußnote für im Text nicht definierte Begriffe

% neue Symbole -----------------------------------------------------------------
% Ausgabe und Aufnahme ins Symbolverzeichnis, ohne Link, mit Hervorhebung der Seitennummer
% {<Glossary Key>}
\newcommand*{\glsTag}[1]{\tag{\glsuseri{#1}}\glsadd[format=HyperDef]{#1}}	
% Ausgabe und Aufnahme und Link ins Symbolverzeichnis mit Hervorhebung der Seitennummer
% {<Makro>} - An <Makro> muss [<Font-Makro>] angehängt werden können
\newcommand*{\defSym}   [1]          {#1[format=HyperDef]}
\newcommand*{\defSymUna}[1]  {\defSym{#1}\;}%  unär: folgender  Abstand
\newcommand*{\defSymBin}[1]{\;\defSym{#1}\;}% binär: umgebender Abstand

% Automatischer Index ==========================================================
\newcommand*{\addIdx}[2][]{% [gloKeys]{Index-Eintrag} - Index hinzufügen
	\newterm[name={#2},#1]{ind-#2}
	\glsadd               {ind-#2}
}
\newcommand*{\idx}[2][]{% [Text]{Index-Eintrag} - Ausgabe und Index hinzufügen
	\addIdx{#2}
	\def\ArgumentEins{#1}
	\def\ASBAundefined{}
	\ifx\ArgumentEins\ASBAundefined #2\else #1\fi%  wenn Text leer, Index-Eintrag ausgeben
}

% Synonyme =====================================================================

% [<Benennung>]{Makro für die <Benennungen>}{<Benennungen-Label>}{<Makro für das Synonym>}
\newcommand*{\newsynonym}[4][]{
	\def\ASBAundefined{}
	\def\ArgumentEins{#1}
	\ifx\ArgumentEins\ASBAundefined
		\newcommand*     {#2}[1][]{\glstext[##1]{#3}}
		\newglossaryentry{#3}{
			name        ={#3 \addIdx[
				name    ={#3}]                  {#3}},
			text        ={#3},
			description ={Synonym zu #4.}
		}
	\else
		\newcommand*     {#2}[1][]{\glstext[##1]{#3}}
		\newglossaryentry{#3}{
			name        ={#1 \addIdx[
				name    ={#1}]                  {#3}},
			text        ={#1},
			description ={Synonym zu #4.}
		}
	\fi
}

% Glossar-Einträge #############################################################

% ### Symbolverzeichnis: Symbole ###############################################

% ==============================================================================
% \* - Ausgabe als Symbol und Eintrag (und Link) ins Symbolverzeichnis
% Fachbegriffe =================================================================

\iftestFlg% Definition von Dummy Smbolverzeichniseinträgen
	\newVerweis     {\dummy}{\glstext}{dummy}
	\newglossaryentry{dummy}{
		text        ={\ensuremath{\#}},%         Ausgabe     im Text
		name        ={\ensuremath{\#} \addIdx[%  Ausgabe     im Smbolverzeichnis
			name    ={\ensuremath{dummy}},%      Ausgabe     im Index
			sort    ={dummy}]         {dummy}},% Reihenfolge im Index
		sort        ={=},%                       Reihenfolge im Smbolverzeichnis
		type        ={symbols},%                 gehört     zum Smbolverzeichnis
		symbol      ={},%                        formale    Definition
		user6       ={},%                        informelle Definition
		see         ={},%             Verweise ins Symbolverzeichnis und Glossar
		description ={\todoBeschreiben%
		}
	}
\else\fi

% ==============================================================================
% \Bsp* - Ausgabe als Symbol und Eintrag und Link ins Symbolverzeichnis
\newglossaryentry{Glo-Beispielsymbole}{
	name     ={\gloFt{Beispielsymbole} für \gloFt{Operationen} und \gloFt{Relationen}},% =====
	sort     ={= 0 0 0},
	type     ={symbols},
	see      ={Beispielsymbol,Operation,Relation},
	description={Im Folgenden seien $A$ und $B$ passende \Objekte.}
}

\newVerweis               {\BspOpU}{\glstext}{BspOpU}
\newglossaryentry          {BspOpU}{
	name  ={\ensuremath{\RawBspOpU}},
	sort  ={= 0 1 1},
	type  ={symbols},
	symbol={\formaleDefinition{\RawBspOpU A}},
	description={\baueBeschreibung[\Beispielsymbol\ für eine \unaere\ \Operation]{BspOpU}.}
}

\newVerweis               {\BspOpB}{\glstext}{BspOpB}
\newglossaryentry          {BspOpB}{
	name  ={\ensuremath{\RawBspOpB}},
	sort  ={= 0 1 2},
	type  ={symbols},
	symbol={\formaleDefinition{A \RawBspOpB B}},
	description={\baueBeschreibung[\Beispielsymbol\ für eine \binaere\ \Operation]{BspOpB}.}
}

\newVerweis               {\BspRel}{\glstext}{BspRel}
\newglossaryentry          {BspRel}{
	name  ={\ensuremath{\RawBspRel}},
	sort  ={= 0 2 1},
	type  ={symbols},
	symbol={\formaleDefinition{A \RawBspRel B}},
	description={\baueBeschreibung[\Beispielsymbol\ für eine \binaere\ \Relation]{BspRel}.}
}

\newVerweis               {\BspRelEq}{\glstext}{BspRelEq}
\newglossaryentry          {BspRelEq}{
	name  ={\ensuremath{\RawBspRelEq}},
	sort  ={= 0 2 2},
	type  ={symbols},
	symbol={\formaleDefinition{A \RawBspRelEq B}},
	description={\baueBeschreibung[\Beispielsymbol\ für eine \binaere\ \Relation]{BspRelEq}.}
}

\newVerweis               {\BspRelBck}{\glstext}{BspRelBck}
\newglossaryentry          {BspRelBck}{
	name  ={\ensuremath{\RawBspRelBck}},
	sort  ={= 0 2 3},
	type  ={symbols},
	symbol={\formaleDefinition{(A \RawBspRelBck B) \RawMtsDefEquiv (B \RawBspRel A)}},
	see   ={MtsDefEquiv},
	description={
		\baueBeschreibung[\Beispielsymbol\ für eine \binaere\ \Relation]{BspRelBck}.
		\\Die \Umkehrrelation\ von \BspRel.
	}
}

\newVerweis               {\BspRelBckEq}{\glstext}{BspRelBckEq}
\newglossaryentry          {BspRelBckEq}{
	name  ={\ensuremath{\RawBspRelBckEq}},
	sort  ={= 0 2 4},
	type  ={symbols},
	symbol={\formaleDefinition{(A \RawBspRelBckEq B) \RawMtsDefEquiv (B \RawBspRelEq A)}},
	see   ={MtsDefEquiv},
	description={
		\baueBeschreibung[\Beispielsymbol\ für eine \binaere\ \Relation]{BspRelBckEq}.
		\\Die \Umkehrrelation\ von \BspRelEq.
	}
}

\newVerweis               {\BspRelN}{\glstext}{BspRelN}
\newglossaryentry          {BspRelN}{
	name  ={\ensuremath{\RawBspRelN}},
	sort  ={= 0 3 1},
	type  ={symbols},
	symbol={\formaleDefinition{(A \RawBspRelN B) \RawMtsDefEquiv \RawMtsNot (A \RawBspRel B)}},
	see   ={MtsNot,MtsDefEquiv},
	description={
		\baueBeschreibung[\Beispielsymbol\ für eine \binaere\ \Relation]{BspRelN}.
		\\Die \Negation\ von \BspRel.
	}
}

\newVerweis               {\BspRelEqN}{\glstext}{BspRelEqN}
\newglossaryentry          {BspRelEqN}{
	name  ={\ensuremath{\RawBspRelEqN}},
	sort  ={= 0 3 2},
	type  ={symbols},
	symbol={\formaleDefinition{(A \RawBspRelEqN B) \RawMtsDefEquiv \RawMtsNot (A \RawBspRelEq B)}},
	see   ={MtsNot,MtsDefEquiv},
	description={
		\baueBeschreibung[\Beispielsymbol\ für eine \binaere\ \Relation]{BspRelEqN}.
		\\Die \Negation\ von \BspRelEq.
	}
}

\newVerweis               {\BspRelBckN}{\glstext}{BspRelBckN}
\newglossaryentry          {BspRelBckN}{
	name  ={\ensuremath{\RawBspRelBckN}},
	sort  ={= 0 3 3},
	type  ={symbols},
	symbol={\formaleDefinition{(A \RawBspRelBckN B) \RawMtsDefEquiv \RawMtsNot (B \RawBspRel A)}},
	see   ={MtsNot,MtsDefEquiv},
	description={
		\baueBeschreibung[\Beispielsymbol\ für eine \binaere\ \Relation]{BspRelBckN}.
		\\Die \Negation\ der \Umkehrrelation\ und gleichzeitig die \Umkehrrelation\ der \Negation\ von \BspRel.
	}
}

\newVerweis               {\BspRelBckEqN}{\glstext}{BspRelBckEqN}
\newglossaryentry          {BspRelBckEqN}{
	name  ={\ensuremath{\RawBspRelBckEqN}},
	sort  ={= 0 3 4},
	type  ={symbols},
	symbol={\formaleDefinition{(A \RawBspRelBckEqN B) \RawMtsDefEquiv \RawMtsNot (B \RawBspRelEq A)}},
	see   ={MtsNot,MtsDefEquiv},
	description={
		\baueBeschreibung[\Beispielsymbol\ für eine \binaere\ \Relation]{BspRelBckEqN}.
		\\Die \Negation\ der \Umkehrrelation\ und gleichzeitig die \Umkehrrelation\ der \Negation\ von \BspRelEq.
	}
}

% ==============================================================================
% \Mts* - Ausgabe als Symbol und Eintrag und Link ins Symbolverzeichnis
% In 'symbol' steht die formale Definition, in 'user6' die Benennung, ggf.
% in einer Formel: Wörter, die weggelassen werden können, kursiv
\newglossaryentry{Glo-MetaAussagen}{
	name     ={\gloFt{Metaoperationen} und \gloFt{-relationen} mit \gloFt{Aussagen}},% =====
	sort     ={= 1 0 0},
	type     ={symbols},
	see      ={Metaoperation,Metarelation},
	description={Im Folgenden seien $A$ und $B$ beliebige \metasprachlicheAussagen.}
}

\newVerweis               {\MtsNot}{\glstext}{MtsNot}
\newglossaryentry          {MtsNot}{
	name  ={\ensuremath{\RawMtsNot}},
	sort  ={= 1 1 1},
	type  ={symbols},
	symbol={\formaleDefinition{\RawMtsNot A}},
	user6 ={\informelleDefinition{\OptFt{es gilt} \ManFt{nicht} $A$}},
	see   ={OjkNot},
	description={\baueBeschreibung[Eine \unaere\ \Metaoperation]{MtsNot}.}
}

\newVerweis               {\MtsAnd}{\glstext}{MtsAnd}
\newglossaryentry          {MtsAnd}{
	name  ={\ensuremath{\RawMtsAnd}},
	sort  ={= 1 1 2},
	type  ={symbols},
	symbol={\formaleDefinition{(A \RawMtsAnd B)}},
	user6 ={\informelleDefinition{\OptFt{es gilt} $A$ \ManFt{und} $B$}},
	see   ={MtsUnd,OjkAnd},
	description={\baueBeschreibung[Eine \binaere\ \Metaoperation]{MtsAnd}.}
}

\newVerweis               {\MtsOr}{\glstext}{MtsOr}
\newglossaryentry          {MtsOr}{
	name  ={\ensuremath{\RawMtsOr}},
	sort  ={= 1 1 3},
	type  ={symbols},
	symbol={\formaleDefinition{(A \RawMtsOr B)}},
	user6 ={\informelleDefinition{\OptFt{es gilt} $A$ \ManFt{oder} $B$}},
	see   ={OjkOr},
	description={\baueBeschreibung[Eine \binaere\ \Metaoperation]{MtsOr}.}
}

\newVerweis               {\MtsUnd}{\glstext}{MtsUnd}
\newglossaryentry          {MtsUnd}{
	name  ={\ensuremath{\RawMtsUnd}},
	sort  ={= 1 2 1},
	type  ={symbols},
	symbol={\formaleDefinition{(A \RawMtsUnd B) \RawMtsDefEquiv (A \RawMtsAnd B)}},
	user6 ={\informelleDefinition{\OptFt{es gilt }$A$ \ManFt{und} $B$}},
	see   ={MtsAnd,MtsDefEquiv,OjkAnd},
	description={
		\baueBeschreibung[Eine \binaere\ \Metaoperation]{MtsUnd}.
		\\Nur in \Schlussregeln!
	}
}

\newVerweis               {\MtsImp}{\glstext}{MtsImp}
\newglossaryentry          {MtsImp}{
	name  ={\ensuremath{\RawMtsImp}},
	sort  ={= 1 3 1},
	type  ={symbols},
	symbol={\formaleDefinition{(A \RawMtsImp B)}},
	user6 ={\informelleDefinition{\OptFt{wenn} $A$ \OptFt{gilt,} \ManFt{dann} \OptFt{gilt auch} $B$}},
	see   ={OjkImp},
	description={\baueBeschreibung[Eine \binaere\ \Metarelation]{MtsImp}.}
}

\newVerweis               {\MtsRep}{\glstext}{MtsRep}
\newglossaryentry          {MtsRep}{
	name  ={\ensuremath{\RawMtsRep}},
	sort  ={= 1 3 2},
	type  ={symbols},
	symbol={\formaleDefinition{(A \RawMtsRep B) \RawMtsDefEquiv (B \RawMtsImp A)}},
	user6 ={\informelleDefinition{$A$ \OptFt{gilt dann,} \ManFt{wenn} $B$ \OptFt{gilt}}},
	see   ={MtsDefEquiv,OjkRep},
	description={
		\baueBeschreibung[Eine \binaere\ \Metarelation]{MtsRep}.
		\\Die \Umkehrrelation\ von \MtsImp.
	}
}

\newVerweis               {\MtsEquiv}{\glstext}{MtsEquiv}
\newglossaryentry          {MtsEquiv}{
	name  ={\ensuremath{\RawMtsEquiv}},
	sort  ={= 1 3 3},
	type  ={symbols},
	symbol={\formaleDefinition{(A \RawMtsEquiv B) \RawMtsDefEquiv ((A \RawMtsImp B) \RawMtsAnd (B \RawMtsImp A))}},
	user6 ={\informelleDefinition{$A$ \OptFt{gilt genau} \ManFt{dann wenn} $B$ \OptFt{gilt}}},
	see   ={MtsAnd,MtsImp,MtsDefEquiv,OjkEquiv},
	description={\baueBeschreibung[Eine \binaere\ \Metarelation]{MtsEquiv}.}
}

\newVerweis               {\MtsDefEquiv}{\glstext}{MtsDefEquiv}
\newglossaryentry          {MtsDefEquiv}{
	name  ={\ensuremath{\RawMtsDefEquiv}},
	sort  ={= 1 4 1},
	see   ={MtsDefEq,Objektdefinition},
	type  ={symbols},
	symbol={\formaleDefinition{(A \RawMtsDefEquiv B)}},
	user6 ={\informelleDefinition{$A$ \OptFt{gilt} \ManFt{definitionsgemäß} \OptFt{genau} \ManFt{dann wenn} $B$ \OptFt{gilt}}},
	description={\todoOk%
		\baueBeschreibung[Die \DefFt{\Aussagedefinition} -- eine \binaere\ \Metarelation]{MtsDefEquiv}.
	}
}

\newglossaryentry{Glo-MetaObjekte}{
	name     ={\gloFt{Metaoperationen} und \gloFt{-relationen} mit \gloFt{Objekten}},% =====
	sort     ={= 1 5 0},
	type     ={symbols},
	see      ={Metaoperation,Metarelation},
	description={Im Folgenden seien $A$ und $B$ beliebige \metasprachlicheObjekte.}
}

\newVerweis               {\MtsEq}{\glstext}{MtsEq}
\newglossaryentry          {MtsEq}{
	name  ={\ensuremath{\RawMtsEq}},
	sort  ={= 1 6 1},
	see   ={OjkEq},
	type  ={symbols},
	symbol={\formaleDefinition{(A \RawMtsEq B)}},
	user6 ={\informelleDefinition{$A$ \OptFt{ist} \ManFt{gleich} $B$}},
	description={
		\baueBeschreibung[\DefFt{Metasprachliche \Gleichheit} -- eine \binaere\ \Metarelation]{MtsEq}\alternativMtsEq.
	}
}
\newcommand*{\alternativMtsEq}{\alternativii{dasselbe wie}{identisch zu}}

\newVerweis               {\MtsEqN}{\glstext}{MtsEqN}
\newglossaryentry          {MtsEqN}{
	name  ={\ensuremath{\RawMtsEqN}},
	sort  ={= 1 6 2},
	see   ={MtsNot,MtsDefEquiv,MtsEq,OjkEqN},
	type  ={symbols},
	symbol={\formaleDefinition{(A \RawMtsEqN B) \RawMtsDefEquiv \RawMtsNot (A \RawMtsEq B)}},
	user6 ={\informelleDefinition{$A$ \OptFt{ist} \ManFt{ungleich} $B$}},
	description={
		\baueBeschreibung[Metasprachliche \DefFt{\Ungleichheit} -- eine \binaere\ \Metarelation]{MtsEqN}\alternativMtsEqN.
		\\Die \Negation\ von \MtsEq.
	}
}
\newcommand*{\alternativMtsEqN}{\alternativiii{nicht gleich}{nicht dasselbe wie}{nicht identisch zu}}

%%%\newVerweis               {\MtsAequiv}{\glstext}{MtsAequiv}
%%%\newglossaryentry          {MtsAequiv}{
%%%	name  ={\ensuremath{\RawMtsAequiv}},
%%%	sort  ={= 1 7 1},
%%%	type  ={symbols},
%%%	symbol={\formaleDefinition{(A \RawMtsAequiv B)}},
%%%	user6 ={\informelleDefinition{$A$ \OptFt{ist} \ManFt{äquivalent} \OptFt{zu} $B$}},
%%%	description={
%%%		\baueBeschreibung[Die \DefFt{\Aequivalenz} -- eine \binaere\ \Metarelation]{MtsAequiv}\alternativMtsAequiv.
%%%	}
%%%}
%%%\newcommand*{\alternativMtsAequiv}{\alternativii{so wie}{ähnlich}}
%%%
%%%\newVerweis               {\MtsAequivN}{\glstext}{MtsAequivN}
%%%\newglossaryentry          {MtsAequivN}{
%%%	name  ={\ensuremath{\RawMtsAequivN}},
%%%	sort  ={= 1 7 2},
%%%	see   ={MtsNot,MtsDefEquiv,Aequivalenz},
%%%	type  ={symbols},
%%%	symbol={\formaleDefinition{(A \RawMtsAequivN B) \RawMtsDefEquiv \RawMtsNot (A \RawMtsAequiv B)}},
%%%	user6 ={\informelleDefinition{$A$ \OptFt{ist} \ManFt{nicht äquivalent} \OptFt{zu} $B$}},
%%%	description={
%%%		\baueBeschreibung[Eine \binaere\ \Metarelation]{MtsAequivN}\alternativMtsAequivN.
%%%		\\Die \Negation\ von \MtsAequiv.
%%%	}
%%%}
%%%\newcommand*{\alternativMtsAequivN}{\alternativii{nicht so wie}{nicht ähnlich}}

\newVerweis               {\MtsDefEq}{\glstext}{MtsDefEq}
\newglossaryentry          {MtsDefEq}{
	name  ={\ensuremath{\RawMtsDefEq}},
	sort  ={= 1 7 3},
	see   ={MtsDefEquiv,Aussagedefinition}
	type  ={symbols},
	symbol={\formaleDefinition{(A \RawMtsDefEq B)}},
	user6 ={\informelleDefinition{$A$ \OptFt{ist} \ManFt{definitionsgemäß gleich} $B$}},
	description={\todoOk%
		\baueBeschreibung[Die \DefFt{\Objektdefinition} -- eine \binaere\ \Metarelation]{MtsDefEq}\alternativMtsDefEq.
	}
}
\newcommand*{\alternativMtsDefEq}{\alternativii{dasselbe wie}{identisch zu}}

\newglossaryentry{Glo-MetaSonstige}{
	name       ={Sonstige \gloFt{Metaoperationen} und \gloFt{-relationen}},% =====
	sort       ={= 1 8 0},
	type       ={symbols},
	description={Im Folgenden seien $A$ und $B$ \metasprachlicheAussagen\ oder \Bereichen\ davon und $\alpha$ und $\beta$ ???.}% Todo Was sind \alpha und \beta?
}

\newVerweis               {\MtsDerive}{\glstext}{MtsDerive}
\newglossaryentry          {MtsDerive}{
	name  ={\ensuremath{\RawMtsDerive}},
	sort  ={= 1 8 1},
	type  ={symbols},
	symbol={\formaleDefinition{(A \RawMtsDerive B)}},
	user6 ={\informelleDefinition{$A$ \OptFt{ist} \ManFt{ableitbar} \OptFt{aus} $B$}},
	see   ={ableitbar},
	description={
		\baueBeschreibung[Die \DefFt{\Ableitungsrelation} -- eine \binaere\ \Metarelation]{MtsDerive}\alternativMtsDerive.
	}
}
\newcommand*{\alternativMtsDerive}{\synonym{\beweisbar}}

\newVerweis               {\MtsDeriveR}{\glstext}{MtsDeriveR}
\newglossaryentry          {MtsDeriveR}{
	name  ={\ensuremath{\RawMtsDerive_R}},
	sort  ={= 1 8 2},
	type  ={symbols},
	symbol={\formaleDefinition{(A \RawMtsDerive_R B) \RawMtsDefEquiv ((A,B) \RawMtsIn R_{\RawMtsIdxGraph})}},
	user6 ={\informelleDefinition{$A$ \OptFt{ist} \ManFt{$R$-ableitbar} \OptFt{aus} $B$}},
	see   ={MtsDerive,MtsDefEquiv,MtsIn,MtsIdxGraph,ableitbar},
	description={
		\baueBeschreibung[Die \DefFt{$R$-\Ableitungsrelation} -- eine \binaere\ \Metarelation]{MtsDeriveR}\alternativMtsDeriveR.
		\\Die \Darstellung\ einer \Relation\ $R \MtsIn \MtsRelAllDerive$ als \Ableitungsrelation.
	}
}
\newcommand*{\alternativMtsDeriveR}{\synonym{$R$-\gloFt{beweisbar}}}

\newVerweis               {\MtsSubst}{\glstext}{MtsSubst}
\newglossaryentry          {MtsSubst}{% ToDo Was sind die Operanden?
	name  ={\ensuremath{\RawMtsSubst}},
	sort  ={= 1 8 3},
	type  ={symbols},
	symbol={\formaleDefinition{(\alpha \RawMtsSubst \beta)}},
	user6 ={\informelleDefinition{$\alpha$ \OptFt{wird} \ManFt{ersetzt durch} $\beta$}},
	description={\baueBeschreibung[Die \DefFt{\Ersetzung}]{MtsSubst}\alternativMtsSubst.}
}
\newcommand*{\alternativMtsSubst}{\alternativi{substituiert durch}}

\newVerweis               {\MtsSwap}{\glstext}{MtsSwap}
\newglossaryentry          {MtsSwap}{% ToDo Was sind die Operanden?
	name  ={\ensuremath{\RawMtsSwap}},
	sort  ={= 1 8 4},
	type  ={symbols},
	symbol={\formaleDefinition{(\alpha \RawMtsSwap \beta)}},
	user6 ={\informelleDefinition{$\alpha$ \OptFt{wird} \ManFt{vertauscht mit} $\beta$}},
	description={\baueBeschreibung[Die \DefFt{\Vertauschung}]{MtsSwap}.}
}

% ==============================================================================
% \Mts* - Ausgabe als Symbol und Eintrag und Link ins Symbolverzeichnis
\newglossaryentry{Glo-Elementrelationen}{
	name     ={\gloFt{Elementrelationen}},% ====================================
	sort     ={= 3 0 0},
	type     ={symbols},
	description={\todoOk Im Folgenden sei $x$ ein beliebiges \Element\ und $M$ eine beliebige \Menge.}
}

\newVerweis               {\MtsIn}{\glstext}{MtsIn}
\newglossaryentry          {MtsIn}{
	name  ={\ensuremath{\RawMtsIn}},
	sort  ={= 3 1 1},
	type  ={symbols},
	symbol={\formaleDefinition{(x \RawMtsIn M)}},
	user6 ={\informelleDefinition{$x$ \OptFt{ist ein Element} \ManFt{aus} $M$}},
	see   ={Element,Komponente},
	description={\todoOk%
		\baueBeschreibung[Eine \DefFt{\Elementrelation}]{MtsIn}\alternativMtsIn.
		\\Die grundlegende \Relation\ der \Mengenlehre.
	}
}
\newcommand*{\alternativMtsIn}{\alternativi[;
	\enquote{$a$ von $M$} könnte \textzB\ auch \enquote{\Komponente\ $a$ von der \Folge\ $M$} meinen.
	Daher bevorzugen wir für \Elemente\ \enquote{aus} und für \Komponenten\ \enquote{von}.
]{von}}

\newVerweis               {\MtsNi}{\glstext}{MtsNi}
\newglossaryentry          {MtsNi}{
	name  ={\ensuremath{\RawMtsNi}},
	sort  ={= 3 1 2},
	type  ={symbols},
	symbol={\formaleDefinition{(M \RawMtsNi x) \RawMtsDefEquiv (x \RawMtsIn M)}},
	user6 ={\informelleDefinition{$M$ \ManFt{enthält} $x$ \OptFt{als Element}}},
	see   ={MtsDefEquiv,Element},
	description={\todoOk%
		\baueBeschreibung[Eine \DefFt{\Elementrelation}]{MtsNi}.
		\\Die \Umkehrrelation\ von \MtsIn.
	}
}

\newVerweis               {\MtsInN}{\glstext}{MtsInN}
\newglossaryentry          {MtsInN}{
	name  ={\ensuremath{\RawMtsInN}},
	sort  ={= 3 2 1},
	type  ={symbols},
	user6 ={\informelleDefinition{$x$ \OptFt{ist} \ManFt{nicht} \OptFt{ein Element} \ManFt{aus} $M$}},
	symbol={\formaleDefinition{(x \RawMtsInN M) \RawMtsDefEquiv \RawMtsNot (x \RawMtsIn M)}},
	see   ={MtsNot,MtsDefEquiv},
	description={\todoOk%
		\baueBeschreibung[Eine \DefFt{\Elementrelation}]{MtsInN}\alternativMtsInN.
		\\Die \Negation\ von \MtsIn.
	}
}
\newcommand*{\alternativMtsInN}{\alternativi{kein \Element\ aus}}

\newVerweis               {\MtsNiN}{\glstext}{MtsNiN}
\newglossaryentry          {MtsNiN}{
	name  ={\ensuremath{\RawMtsNiN}},
	sort  ={= 3 2 2},
	type  ={symbols},
	symbol={\formaleDefinition{(M \RawMtsNiN x) \RawMtsDefEquiv \RawMtsNot (x \RawMtsIn M)}},
	user6 ={\informelleDefinition{$M$ \ManFt{enthält} $x$ \ManFt{nicht} \OptFt{als Element}}},
	see   ={MtsNot,MtsDefEquiv},
	description={\todoOk%
		\baueBeschreibung[Eine \DefFt{\Elementrelation}]{MtsNiN}.
		\\Die \Negation\ der \Umkehrrelation\ und gleichzeitig die \Umkehrrelation\ der \Negation\ von \MtsIn.
	}
}

% ==============================================================================
% \Mts* - Ausgabe als Symbol und Eintrag und Link ins Symbolverzeichnis
\newglossaryentry{Glo-Bereichsoperationen}{
	name     ={\gloFt{Bereichsrelationen} und \gloFt{-operationen}},% ============
	sort     ={= 4 0 0},
	type     ={symbols},
	see      ={Menge,Metaoperation,Metarelation},
	description={
		\footnote{In diesem Dokument \Metarelationen\ und \Moperationen.}
		Im Folgenden seien $M$ und $N$ beliebige \Mengen.
	}
}

\newVerweis               {\MtsSubset}{\glstext}{MtsSubset}
\newglossaryentry          {MtsSubset}{
	name  ={\ensuremath{\RawMtsSubset}},
	sort  ={= 4 1 1},
	type  ={symbols},
	symbol={\formaleDefinition{(M \RawMtsSubset N) \RawMtsDefEquiv ((M \RawMtsSubsetEq N) \RawMtsAnd (M \RawMtsEqN N))}},
	user6 ={\informelleDefinition{$M$ \OptFt{ist eine} \ManFt{echte Teilmenge} \OptFt{von} $N$}},
	see   ={MtsAnd,MtsDefEquiv,MtsEqN,echteTeilmenge},
	description={\todoOk%
		\baueBeschreibung[Eine \Bereichsrelation]{MtsSubset}.
		\\Ursprünglich wurde \MtsSubset\ im Sinne von \MtsSubsetEq\ verwendet.
	}
}

\newVerweis               {\MtsSubsetEq}{\glstext}{MtsSubsetEq}
\newglossaryentry          {MtsSubsetEq}{
	name  ={\ensuremath{\RawMtsSubsetEq}},
	sort  ={= 4 1 2},
	type  ={symbols},
	symbol={\formaleDefinition{(M \RawMtsSubsetEq N) \RawMtsDefEquiv \RawMtsForall x:((x \RawMtsIn M) \RawMtsImp (x \RawMtsIn N))}},
	user6 ={\informelleDefinition{$M$ \OptFt{ist eine} \ManFt{Teilmenge} \OptFt{von} $N$}},
	see   ={MtsImp,MtsDefEquiv,MtsIn,MtsForall,Teilmenge},
	description={\todoOk%
		\baueBeschreibung[Eine \Bereichsrelation]{MtsSubsetEq}.
	}
}

\newVerweis               {\MtsSupset}{\glstext}{MtsSupset}
\newglossaryentry          {MtsSupset}{
	name  ={\ensuremath{\RawMtsSupset}},
	sort  ={= 4 1 3},
	type  ={symbols},
	symbol={\formaleDefinition{(M \RawMtsSupset N) \RawMtsDefEquiv (N \RawMtsSubset M)}},
	user6 ={\informelleDefinition{$M$ \OptFt{ist eine} \ManFt{echte Obermenge von} $N$}},
	see   ={MtsDefEquiv,echteObermenge},
	description={\todoOk%
		\baueBeschreibung[Eine \Bereichsrelation]{MtsSupset}.
		\\Die \Umkehrrelation\ von \MtsSubset.
		Ursprünglich wurde \MtsSupset\ im Sinne von \MtsSupsetEq\ verwendet.
	}
}

\newVerweis               {\MtsSupsetEq}{\glstext}{MtsSupsetEq}
\newglossaryentry          {MtsSupsetEq}{
	name  ={\ensuremath{\RawMtsSupsetEq}},
	sort  ={= 4 1 4},
	type  ={symbols},
	symbol={\formaleDefinition{(M \RawMtsSupsetEq N) \RawMtsDefEquiv (N \RawMtsSubsetEq M)}},
	user6 ={\informelleDefinition{$M$ \OptFt{ist eine} \ManFt{Obermenge von} $N$}},
	see   ={MtsDefEquiv,Obermenge},
	description={\todoOk%
		\baueBeschreibung[Eine \Bereichsrelation]{MtsSupsetEq}.
		\\Die \Umkehrrelation\ von \MtsSubsetEq.
	}
}

\newVerweis               {\MtsSubsetN}{\glstext}{MtsSubsetN}
\newglossaryentry          {MtsSubsetN}{
	name  ={\ensuremath{\RawMtsSubsetN}},
	sort  ={= 4 2 1},
	type  ={symbols},
	symbol={\formaleDefinition{(M \RawMtsSubsetN N) \RawMtsDefEquiv \RawMtsNot (M \RawMtsSubset N)}},
	user6 ={\informelleDefinition{$M$ \OptFt{ist} \ManFt{keine echte Teilmenge} \OptFt{von} $N$}},
	see   ={MtsNot,MtsDefEquiv,echteTeilmenge},
	description={\todoOk%
		\baueBeschreibung[Eine \Bereichsrelation]{MtsSubsetN}.
		\\Die \Negation\ von \MtsSubset.
	}
}

\newVerweis               {\MtsSubsetEqN}{\glstext}{MtsSubsetEqN}
\newglossaryentry          {MtsSubsetEqN}{
	name  ={\ensuremath{\RawMtsSubsetEqN}},
	sort  ={= 4 2 2},
	type  ={symbols},
	symbol={\formaleDefinition{(M \RawMtsSubsetEqN N) \RawMtsDefEquiv \RawMtsNot (M \RawMtsSubsetEq N)}},
	user6 ={\informelleDefinition{$M$ \OptFt{ist} \ManFt{keine Teilmenge} \OptFt{von} $N$}},
	see   ={MtsNot,MtsDefEquiv,Teilmenge},
	description={\todoOk%
		\baueBeschreibung[Eine \Bereichsrelation]{MtsSubsetEqN}.
		\\Die \Negation\ von \MtsSubsetEq.
	}
}

\newVerweis               {\MtsSupsetN}{\glstext}{MtsSupsetN}
\newglossaryentry          {MtsSupsetN}{
	name  ={\ensuremath{\RawMtsSupsetN}},
	sort  ={= 4 2 3},
	type  ={symbols},
	symbol={\formaleDefinition{(M \RawMtsSupsetN N) \RawMtsDefEquiv \RawMtsNot (N \RawMtsSubset M)}},
	user6 ={\informelleDefinition{$M$ \OptFt{ist} \ManFt{keine echte Obermenge von} $N$}},
	see   ={MtsNot,MtsDefEquiv},
	description={\todoOk%
		\baueBeschreibung[Eine \Bereichsrelation]{MtsSupsetN}.
		\\Die \Negation\ der \Umkehrrelation\ und gleichzeitig die \Umkehrrelation\ der \Negation\ von \MtsSubset.
	}
}

\newVerweis               {\MtsSupsetEqN}{\glstext}{MtsSupsetEqN}
\newglossaryentry          {MtsSupsetEqN}{
	name  ={\ensuremath{\RawMtsSupsetEqN}},
	sort  ={= 4 2 4},
	type  ={symbols},
	symbol={\formaleDefinition{(M \RawMtsSupsetEqN N) \RawMtsDefEquiv \RawMtsNot (N \RawMtsSubsetEq M)}},
	user6 ={\informelleDefinition{$M$ \OptFt{ist} \ManFt{keine Obermenge von} $N$}},
	see   ={MtsNot,MtsDefEquiv},
	description={\todoOk%
		\baueBeschreibung[Eine \Bereichsrelation]{MtsSupsetEqN}.
		\\Die \Negation\ der \Umkehrrelation\ und gleichzeitig die \Umkehrrelation\ der \Negation\ von \MtsSubsetEq.
	}
}

\newVerweis               {\MtsCap}{\glstext}{MtsCap}
\newglossaryentry          {MtsCap}{
	name  ={\ensuremath{\RawMtsCap}},
	sort  ={= 4 3 1},
	type  ={symbols},
	symbol={\formaleDefinition{M \RawMtsCap N \RawMtsDefEq \RawMengeDef{x}{(x \RawMtsIn M) \RawMtsAnd (x \RawMtsIn N)}}},
	user6 ={\informelleDefinition{\OptFt{Der} \ManFt{Durchschnitt von} $M$ \ManFt{und} $N$}},
	see   ={MtsAnd,MtsDefEq,MtsIn,Durchschnitt},
	description={\baueBeschreibung[Eine \Bereichsoperation]{MtsCap}.}
}

\newVerweis               {\MtsCup}{\glstext}{MtsCup}
\newglossaryentry          {MtsCup}{
	name  ={\ensuremath{\RawMtsCup}},
	sort  ={= 4 3 2},
	type  ={symbols},
	symbol={\formaleDefinition{M \RawMtsCup N \RawMtsDefEq \RawMengeDef{x}{(x \RawMtsIn M) \RawMtsOr (x \RawMtsIn N)}}},
	user6 ={\informelleDefinition{\OptFt{Die} \ManFt{Vereinigung von} $M$ \ManFt{und} $N$}},
	see   ={MtsOr,MtsDefEq,MtsIn,Menge,Bereichsoperation,Vereinigung},
	description={\baueBeschreibung[Eine \Bereichsoperation]{MtsCup}.}
}

\newVerweis               {\MtsSetminus}{\glstext}{MtsSetminus}
\newglossaryentry          {MtsSetminus}{
	name  ={\ensuremath{\RawMtsSetminus}},
	sort  ={= 4 3 3},
	type  ={symbols},
	symbol={\formaleDefinition{M \RawMtsSetminus N \RawMtsDefEq \RawMengeDef{x}{(x \RawMtsIn M) \RawMtsAnd (x \RawMtsInN N)}}},
	user6 ={\informelleDefinition{\OptFt{Die} \ManFt{Differenz von} $M$ \ManFt{und} $N$}},
	see   ={MtsAnd,MtsDefEq,MtsIn,MtsInN,Differenz,Menge,Bereichsoperation},
	description={\baueBeschreibung[Eine \Bereichsoperation]{MtsSetminus}.}
}

\newVerweis               {\MtsTimes}{\glstext}{MtsTimes}
\newglossaryentry          {MtsTimes}{
	name  ={\ensuremath{\RawMtsTimes}},
	sort  ={= 4 3 4},
	type  ={symbols},
	symbol={\formaleDefinition{M \RawMtsTimes N \RawMtsDefEq \RawMengeDef{(x,y)}{(x \RawMtsIn M) \RawMtsAnd (y \RawMtsIn N)}}},
	user6 ={\informelleDefinition{\OptFt{Das} \ManFt{kartesische Produkt von} $M$ \ManFt{und} $N$}},
	see   ={MtsAnd,MtsDefEq,MtsIn,kartesischesProdukt,Menge,Bereichsoperation,Mengenprodukt},
	description={\baueBeschreibung[Eine \Bereichsoperation]{MtsTimes}\synonymMtsTimes.}
}
\newcommand*{\synonymMtsTimes}{\synonym{\gloFt{Mengenprodukt}}}

% ==============================================================================
% \MtsSeq* - Ausgabe als Symbol und Eintrag und Link ins Symbolverzeichnis
\newglossaryentry{Glo-Komponentenrelationen}{
	name     ={\gloFt{Komponentenrelationen}},% ================================
	sort     ={= 5 0 0},
	type     ={symbols},
	see      ={Komponentenrelation},
	description={
		Im Folgenden sei $x$ eine beliebige \Komponente\ und $F$ eine beliebige \Folge.
	}
}

\newVerweis               {\MtsSeqIn}{\glstext}{MtsSeqIn}
\newglossaryentry          {MtsSeqIn}{
	name  ={\ensuremath{\RawMtsSeqIn}},
	sort  ={= 5 1 1},
	type  ={symbols},
	symbol={\formaleDefinition{(x \RawMtsSeqIn F)}},
	user6 ={\informelleDefinition{$x$ \OptFt{ist eine} \ManFt{Komponente von} $F$}},
	see   ={Element,Komponente},
	description={\baueBeschreibung[Eine \Komponentenrelation]{MtsSeqIn}\alternativMtsSeqIn.}
}
\newcommand*{\alternativMtsSeqIn}{\alternativi[;
	\enquote{$x$ aus $F$} könnte \textzB\ auch \enquote{\Element\ $x$ aus der \Menge\ $F$} meinen.
	Daher bevorzugen wir für \Komponenten\ \enquote{von} und für \Elemente\ \enquote{aus}.
]{aus}}

\newVerweis               {\MtsSeqNi}{\glstext}{MtsSeqNi}
\newglossaryentry          {MtsSeqNi}{
	name  ={\ensuremath{\RawMtsSeqNi}},
	sort  ={= 5 1 2},
	type  ={symbols},
	symbol={\formaleDefinition{(F \RawMtsSeqNi x) \RawMtsDefEq (x \RawMtsSeqIn F)}},
	user6 ={\informelleDefinition{$F$ \ManFt{enthält} $x$ \OptFt{als Komponente}}},
	see   ={MtsDefEq,Komponente},
	description={
		\baueBeschreibung[Eine \Komponentenrelation]{MtsSeqNi}.
		\\Die \Umkehrrelation\ von \MtsSeqIn.
	}
}

\newVerweis               {\MtsSeqInN}{\glstext}{MtsSeqInN}
\newglossaryentry          {MtsSeqInN}{
	name  ={\ensuremath{\RawMtsSeqInN}},
	sort  ={= 5 2 1},
	type  ={symbols},
	symbol={\formaleDefinition{(x \RawMtsSeqInN F) \RawMtsDefEq \RawMtsNot (x \RawMtsSeqIn F)}},
	user6 ={\informelleDefinition{$x$ \OptFt{ist} \ManFt{keine Komponente aus} $F$}},
	see   ={MtsNot,MtsDefEq,Komponente},
	description={
		\baueBeschreibung[Eine \Komponentenrelation]{MtsSeqInN}.
		\\Die \Negation\ von \MtsSeqIn.
	}
}

\newVerweis               {\MtsSeqNiN}{\glstext}{MtsSeqNiN}
\newglossaryentry          {MtsSeqNiN}{
	name  ={\ensuremath{\RawMtsSeqNiN}},
	sort  ={= 5 2 2},
	type  ={symbols},
	symbol={\formaleDefinition{(F \RawMtsSeqNiN x) \RawMtsDefEq \RawMtsNot (x \RawMtsSeqIn F)}},
	user6 ={\informelleDefinition{$F$ \ManFt{enthält} $x$ \ManFt{nicht} \OptFt{als Komponente}}},
	see   ={MtsNot,MtsDefEq,Komponente},
	description={
		\baueBeschreibung[Eine \Komponentenrelation]{MtsSeqNiN}.
		\\Die \Negation\ der \Umkehrrelation\ und gleichzeitig die \Umkehrrelation\ der \Negation\ von \MtsSeqIn.
	}
}

% ==============================================================================
% \Mts* - Ausgabe als Symbol und Eintrag und Link ins Symbolverzeichnis
\newglossaryentry{Glo-Folgenrelationen}{
	name     ={\gloFt{Folgenoperationen} und \gloFt{-relationen}},% ============
	sort     ={= 6 0 0},
	see      ={Folgenoperation,Folgenrelation,Tupel},
	type     ={symbols},
	description={
		Im Folgenden seien $\vec{a}$ eine endliche und $\vec{b}$, $\vec{c}$ und $\vec{d}$ beliebige \Folgen.
	}
}

\newVerweis               {\MtsCat}{\glstext}{MtsCat}
\newglossaryentry          {MtsCat}{
	name  ={\ensuremath{\RawMtsCat}},
	sort  ={= 6 0 1},
	type  ={symbols},
	symbol={\formaleDefinition{\{a_1, \dots, a_n\} \RawMtsCat \{c_1, c_2, \dots\} \RawMtsDefEq \{a_1, \dots, a_n, c_1, c_2, \dots\}}},
	user6 ={\informelleDefinition{$\vec{a}$ \ManFt{verkettet} \OptFt{mit} $\vec{c}$}},
	see   ={MtsDefEq,Folge,Verkettung},
	description={\baueBeschreibung[Eine \Folgenoperation]{MtsCat}.}
}

\newVerweis               {\MtsSubseq}{\glstext}{MtsSubseq}
\newglossaryentry          {MtsSubseq}{
	name  ={\ensuremath{\RawMtsSubseq}},
	sort  ={= 6 1 1},
	type  ={symbols},
	symbol={\formaleDefinition{(\vec{c} \RawMtsSubseq \vec{d}) \RawMtsDefEquiv ((\vec{c} \RawMtsSubseqEq \vec{d}) \RawMtsAnd (\vec{c} \RawMtsEqN \vec{d}))}},
	user6 ={\informelleDefinition{$\vec{c}$ \OptFt{ist eine} \ManFt{echte Teilfolge} \OptFt{von} $\vec{d}$}},
	see   ={MtsAnd,MtsDefEquiv,MtsEqN,MtsSubseqEq,echteTeilfolge},
	description={\baueBeschreibung[Eine \Folgenoperation]{MtsSubseq}.}
}

\newVerweis               {\MtsSubseqEq}{\glstext}{MtsSubseqEq}
\newglossaryentry          {MtsSubseqEq}{
	name  ={\ensuremath{\RawMtsSubseqEq}},
	sort  ={= 6 1 2},
	type  ={symbols},
	symbol={\formaleDefinition{
			(\vec{c} \RawMtsSubseqEq \vec{d}) \RawMtsDefEquiv (
				(\RawMtsExists \vec{a} :         (\vec{a} \RawMtsCat \vec{c})                   \RawMtsEq \vec{d}) \RawMtsOr
				(\RawMtsExists \vec{a},\vec{b} : (\vec{a} \RawMtsCat \vec{c} \RawMtsCat \vec{b}) \RawMtsEq \vec{d})
			)
		}
	},
	user6 ={\informelleDefinition{$\vec{c}$ \OptFt{ist eine} \ManFt{Teilfolge von} $\vec{d}$}},
	see   ={MtsDefEquiv,MtsEq,MtsExists,Folge,Teilfolge},
	description={
		\baueBeschreibung[Eine \Folgenrelation]{MtsSubseqEq}%
		\footnote{Im letzteren Fall muss $\vec{c}$ eine endliche \Folge\ sein.}.
	}
}

\newVerweis               {\MtsSupseq}{\glstext}{MtsSupseq}
\newglossaryentry          {MtsSupseq}{
	name  ={\ensuremath{\RawMtsSupseq}},
	sort  ={= 6 1 3},
	type  ={symbols},
	symbol={\formaleDefinition{(\vec{c} \RawMtsSupseq \vec{d}) \RawMtsDefEquiv (\vec{d} \RawMtsSubseq \vec{c})}},
	user6 ={\informelleDefinition{$\vec{c}$ \OptFt{ist eine} \ManFt{echte Oberfolge von} $\vec{d}$}},
	see   ={MtsDefEquiv,echteOberfolge},
	description={
		\baueBeschreibung[Eine \Folgenrelation]{MtsSupseq}.
		\\Die \Umkehrrelation\ von \MtsSubseq.
	}
}

\newVerweis               {\MtsSupseqEq}{\glstext}{MtsSupseqEq}
\newglossaryentry          {MtsSupseqEq}{
	name  ={\ensuremath{\RawMtsSupseqEq}},
	sort  ={= 6 1 4},
	type  ={symbols},
	symbol={\formaleDefinition{(\vec{c} \RawMtsSupseqEq \vec{d}) \RawMtsDefEquiv (\vec{d} \RawMtsSubseqEq \vec{c})}},
	user6 ={\informelleDefinition{$\vec{c}$ \OptFt{ist eine} \ManFt{Oberfolge von} $\vec{d}$}},
	see   ={MtsDefEquiv,Oberfolge},
	description={
		\baueBeschreibung[Eine \Folgenrelation]{MtsSupseqEq}.
		\\Die \Umkehrrelation\ von \MtsSubseqEq.
	}
}

\newVerweis               {\MtsSubseqN}{\glstext}{MtsSubseqN}
\newglossaryentry          {MtsSubseqN}{
	name  ={\ensuremath{\RawMtsSubseqN}},
	sort  ={= 6 2 1},
	type  ={symbols},
	symbol={\formaleDefinition{(\vec{c} \RawMtsSubseqN \vec{d}) \RawMtsDefEquiv \RawMtsNot (\vec{c} \RawMtsSubseq \vec{d})}},
	user6 ={\informelleDefinition{$\vec{c}$ \OptFt{ist} \ManFt{keine echte Teilfolge von} $\vec{d}$}},
	see   ={MtsNot,MtsDefEquiv,echteTeilfolge},
	description={
		\baueBeschreibung[Eine \Folgenrelation]{MtsSubseqN}.
		\\Die \Negation\ von \MtsSubseq.
	}
}

\newVerweis               {\MtsSubseqEqN}{\glstext}{MtsSubseqEqN}
\newglossaryentry          {MtsSubseqEqN}{
	name  ={\ensuremath{\RawMtsSubseqEqN}},
	sort  ={= 6 2 2},
	type  ={symbols},
	symbol={\formaleDefinition{(\vec{c} \RawMtsSubseqEqN \vec{d}) \RawMtsDefEquiv \RawMtsNot (\vec{c} \RawMtsSubseqEq \vec{d})}},
	user6 ={\informelleDefinition{$\vec{c}$ \OptFt{ist} \ManFt{keine Teilfolge von} $\vec{d}$}},
	see   ={MtsNot,MtsDefEquiv,Teilfolge},
	description={
		\baueBeschreibung[Eine \Folgenrelation]{MtsSubseqEqN}.
		\\Die \Negation\ von \MtsSubseqEq.
	}
}

\newVerweis               {\MtsSupseqN}{\glstext}{MtsSupseqN}
\newglossaryentry          {MtsSupseqN}{
	name  ={\ensuremath{\RawMtsSupseqN}},
	sort  ={= 6 2 3},
	type  ={symbols},
	symbol={\formaleDefinition{(\vec{c} \RawMtsSupseqN \vec{d}) \RawMtsDefEquiv \RawMtsNot (\vec{d} \RawMtsSubseq \vec{c})}},
	user6 ={\informelleDefinition{$\vec{c}$ \OptFt{ist} \ManFt{keine echte Oberfolge von} $\vec{d}$}},
	see   ={MtsNot,MtsDefEquiv,echteOberfolge},
	description={
		\baueBeschreibung[Eine \Folgenrelation]{MtsSupseqN}.
		\\Die \Negation\ der \Umkehrrelation\ und gleichzeitig die \Umkehrrelation\ der \Negation\ von \MtsSubseq.
	}
}

\newVerweis               {\MtsSupseqEqN}{\glstext}{MtsSupseqEqN}
\newglossaryentry          {MtsSupseqEqN}{
	name  ={\ensuremath{\RawMtsSupseqEqN}},
	sort  ={= 6 2 4},
	type  ={symbols},
	symbol={\formaleDefinition{(\vec{c} \RawMtsSupseqEqN \vec{d}) \RawMtsDefEquiv \RawMtsNot (\vec{d} \RawMtsSubseqEq \vec{c})}},
	user6 ={\informelleDefinition{$\vec{c}$ \OptFt{ist} \ManFt{keine Oberfolge von} $\vec{d}$}},
	see   ={MtsNot,MtsDefEquiv,Oberfolge},
	description={
		\baueBeschreibung[Eine \Folgenrelation]{MtsSupseqEqN}.
		\\Die \Negation\ der \Umkehrrelation\ und gleichzeitig die \Umkehrrelation\ der \Negation\ von \MtsSubseqEq.
	}
}

% ==============================================================================
% \Ojk* - Ausgabe als Symbol und Eintrag und Link ins Symbolverzeichnis
\newglossaryentry{Glo-Objektsymbole}{
	name       ={\gloFt{Junktoren}},% ==========================================
	sort       ={= 7 0 0},
	type       ={symbols},
	see   ={Junktor},
	description={
		\footnote{In diesem Dokument \aussagenlogischeKonstante, \aRelationen\ und \aOperationen, \textdh\ \Objektkonstante, \Orelationen\ und \Ooperationen.}
		Im Folgenden seien $A$ und $B$ beliebige \logischeAussagen.
	}
}

\newVerweis               {\OjkFalse}{\glstext}{OjkFalse}
\newglossaryentry          {OjkFalse}{
	name  ={\ensuremath{\RawOjkFalse}},
	sort  ={= 7 0 1},
	type  ={symbols},
	symbol={\formaleDefinition{Wert(\RawOjkFalse) \RawMtsDefEq \RawTxtFalse}},
	user6 ={\informelleDefinition{\ManFt{falsch}}},
	see   ={MtsFalse,TxtFalse,Wahrheitswert},
	description={\todoOk%
		\baueBeschreibung[Ein $0$-\stelliger\ \Junktor, \textdh\ eine \aussagenlogischeKonstante]{OjkFalse}.
	}
}

\newVerweis               {\OjkTrue}{\glstext}{OjkTrue}
\newglossaryentry          {OjkTrue}{
	name  ={\ensuremath{\RawOjkTrue}},
	sort  ={= 7 0 2},
	type  ={symbols},
	symbol={\formaleDefinition{Wert(\RawOjkTrue) \RawMtsDefEq \RawTxtTrue}},
	user6 ={\informelleDefinition{\ManFt{wahr}}},
	see   ={MtsTrue,TxtTrue,Wahrheitswert},
	description={\todoOk%
		\baueBeschreibung[Ein $0$-\stelliger\ \Junktor, \textdh\ eine \aussagenlogischeKonstante]{OjkTrue}.
	}
}

\newVerweis               {\OjkNot}{\glstext}{OjkNot}
\newglossaryentry          {OjkNot}{
	name  ={\ensuremath{\RawOjkNot}},
	sort  ={= 7 1 1},
	type  ={symbols},
	symbol={\formaleDefinition{\RawOjkNot A}},
	user6 ={\informelleDefinition{\DefFt{nicht} $A$}},
	see   ={MtsNot},
	description={\baueBeschreibung[Ein \unaererJunktor]{OjkNot}.}
}

\newVerweis               {\OjkAnd}{\glstext}{OjkAnd}
\newglossaryentry          {OjkAnd}{
	name  ={\ensuremath{\RawOjkAnd}},
	sort  ={= 7 1 2},
	type  ={symbols},
	symbol={\formaleDefinition{A \RawOjkAnd B}},
	user6 ={\informelleDefinition{$A$ \DefFt{und} $B$}},
	see   ={MtsAnd,OjkNand},
	description={\baueBeschreibung[Ein \binaererJunktor]{OjkAnd}\alternativOjkAnd.}
}
\newcommand*{\alternativOjkAnd}{\alternativi{sowohl \textdots\ als auch \textdots}}

\newVerweis               {\OjkOr}{\glstext}{OjkOr}
\newglossaryentry          {OjkOr}{
	name  ={\ensuremath{\RawOjkOr}},
	sort  ={= 7 1 3},
	type  ={symbols},
	symbol={\formaleDefinition{A \RawOjkOr B}},
	user6 ={\informelleDefinition{$A$ \DefFt{oder} $B$}},
	see   ={MtsOr,OjkNor,OjkXor},
	description={\baueBeschreibung[Ein \binaererJunktor]{OjkOr}.}
}

\newVerweis               {\OjkImp}{\glstext}{OjkImp}
\newglossaryentry          {OjkImp}{
	name  ={\ensuremath{\RawOjkImp}},
	sort  ={= 7 2 1},
	type  ={symbols},
	symbol={\formaleDefinition{(A \RawOjkImp B) \RawMtsDefEquiv (\RawOjkNot A \RawOjkOr B})},
	user6 ={\informelleDefinition{wenn $A$ \DefFt{dann} $B$}},
	see   ={MtsImp},
	description={\baueBeschreibung[Ein \binaererJunktor]{OjkImp}.}
}

\newVerweis               {\OjkRep}{\glstext}{OjkRep}
\newglossaryentry          {OjkRep}{
	name  ={\ensuremath{\RawOjkRep}},
	sort  ={= 7 2 2},
	type  ={symbols},
	symbol={\formaleDefinition{(A \RawOjkRep B) \RawMtsDefEquiv (B \RawOjkImp A)}},
	user6 ={\informelleDefinition{$A$ \DefFt{wenn} $B$}},
	see   ={MtsRep},
	description={\baueBeschreibung[Ein \binaererJunktor]{OjkRep}.}
}

\newVerweis               {\OjkEquiv}{\glstext}{OjkEquiv}
\newglossaryentry          {OjkEquiv}{
	name  ={\ensuremath{\RawOjkEquiv}},
	sort  ={= 7 2 3},
	type  ={symbols},
	symbol={\formaleDefinition{(A \RawOjkEquiv B) \RawMtsDefEquiv ((A \RawOjkImp B) \RawOjkAnd (B \RawOjkImp A))}},
	user6 ={\informelleDefinition{$A$ genau \DefFt{dann wenn} $B$}},
	see   ={MtsEquiv},
	description={\baueBeschreibung[Ein \binaererJunktor]{OjkEquiv}.}
}

\newVerweis               {\OjkNand}{\glstext}{OjkNand}
\newglossaryentry          {OjkNand}{
	name  ={\ensuremath{\RawOjkNand}},
	sort  ={= 7 3 1},
	type  ={symbols},
	symbol={\formaleDefinition{(A \RawOjkNand B) \RawMtsDefEquiv \RawOjkNot (A \RawOjkAnd B)}},
	user6 ={\informelleDefinition{\DefFt{nicht} ($A$ \DefFt{und} $B$)}},
	see   ={OjkAnd},
	description={\baueBeschreibung[Ein \binaererJunktor]{OjkNand}.}
}

\newVerweis               {\OjkNor}{\glstext}{OjkNor}
\newglossaryentry          {OjkNor}{
	name  ={\ensuremath{\RawOjkNor}},
	sort  ={= 7 3 2},
	type  ={symbols},
	symbol={\formaleDefinition{(A \RawOjkNor B) \RawMtsDefEquiv \RawOjkNot (A \RawOjkOr B)}},
	user6 ={\informelleDefinition{\DefFt{nicht} ($A$ \DefFt{oder} $B$}},
	see   ={OjkOr,OjkXor},
	description={\baueBeschreibung[Ein \binaererJunktor]{OjkNor}.\alternativOjkNor}
}
\newcommand*{\alternativOjkNor}{\alternativi{weder \textdots\ noch \textdots}}

\newVerweis               {\OjkXor}{\glstext}{OjkXor}
\newglossaryentry          {OjkXor}{
	name  ={\ensuremath{\RawOjkXor}},
	sort  ={= 7 3 3},
	type  ={symbols},
	symbol={\formaleDefinition{(A \RawOjkXor B) \RawMtsDefEquiv ((A \RawOjkAnd \RawOjkNot B) \RawOjkOr (B \RawOjkAnd \RawOjkNot A))}},
	user6 ={\informelleDefinition{\DefFt{entweder} $A$ \DefFt{oder} $B$}},
	see   ={OjkOr,OjkNor},
	description={\baueBeschreibung[Ein \binaererJunktor]{OjkXor}.}
}

\newVerweis               {\OjkEq}{\glstext}{OjkEq}
\newglossaryentry          {OjkEq}{
	name  ={\ensuremath{\RawOjkEq}},
	sort  ={= 7 4 1},
	type  ={symbols},
	symbol={\formaleDefinition{A \RawOjkEq B}},
	user6 ={\informelleDefinition{$A$ ist \DefFt{gleich} $B$}},
	see   ={MtsEq},
	description={\baueBeschreibung[\DefFt{Logische\Gleichheit}]{OjkEq}.}
}

\newVerweis               {\OjkEqN}{\glstext}{OjkEqN}
\newglossaryentry          {OjkEqN}{
	name  ={\ensuremath{\RawOjkEqN}},
	sort  ={= 7 4 2},
	type  ={symbols},
	symbol={\formaleDefinition{A \RawOjkEqN B}},
	user6 ={\informelleDefinition{$A$ ist \DefFt{ungleich} $B$}},
	see   ={MtsEqN},
	description={\baueBeschreibung[\DefFt{Logische \Ungleichheit}]{OjkEqN}.}
}

% ==============================================================================
% \Mts* - Ausgabe als Symbol und Eintrag und Link ins Symbolverzeichnis
% \Ojk* - Ausgabe als Symbol und Eintrag und Link ins Symbolverzeichnis
\newglossaryentry{Glo-Quantoren}{
	name     ={\gloFt{Quantoren}},% ============================================
	sort     ={= 8 0 0},
	type     ={symbols},
	see      ={Aussage,Formel,Quantor,Variable,logischeVariable,metasprachlicheVariable},
	description={
		Im Folgenden seien $a$, $b$ und $x$ \metasprachlicheV\ \textbzw\ \logischeVariablen\ und $A(x)$ eine \Aussage\ \textbzw\ \Formel\ mit der \freienVariablen\ $x$.
	}% Todo = freie Variable
}

\newVerweis               {\MtsForall}{\glstext}{MtsForall}
\newglossaryentry          {MtsForall}{
	name  ={\ensuremath{\RawMtsForall}},
	sort  ={= 8 1 1},
	type  ={symbols},
	symbol={\formaleDefinition{\RawMtsForall x A(x)}},
	user6 ={\informelleDefinition{\DefFt{für alle} $x$ \DefFt{gilt} $A(x)$}},
	see   ={OjkForall},
	description={\baueBeschreibung[Ein \metasprachlicherQuantor]{MtsForall}.}
}

\newVerweis               {\MtsExists}{\glstext}{MtsExists}
\newglossaryentry          {MtsExists}{
	name  ={\ensuremath{\RawMtsExists}},
	sort  ={= 8 1 2},
	type  ={symbols},
	symbol={\formaleDefinition{\RawMtsExists x A(x)}},
	user6 ={\informelleDefinition{\DefFt{es gibt ein} $x$ \DefFt{so dass} $A(x)$}},
	see   ={OjkExists},
	description={\baueBeschreibung[Ein \metasprachlicherQuantor]{MtsExists}.}
}

\newVerweis               {\MtsExione}{\glstext}{MtsExione}
\newglossaryentry          {MtsExione}{
	name  ={\ensuremath{\RawMtsExione}},
	sort  ={= 8 1 3},
	type  ={symbols},
	symbol={\formaleDefinition{(\RawMtsExione x A(x)) \RawMtsDefEquiv ((\RawMtsExists x A(x)) \RawMtsAnd ((A(a) \RawMtsAnd A(b)) \RawMtsImp (a = b)))}},
	user6 ={\informelleDefinition{\DefFt{es gibt genau ein} $x$ \DefFt{so dass} $A(x)$}},
	see   ={OjkExione},
	description={\baueBeschreibung[Ein \metasprachlicherQuantor]{MtsExione}.}
}

\newVerweis               {\OjkForall}{\glstext}{OjkForall}
\newglossaryentry          {OjkForall}{
	name  ={\ensuremath{\RawOjkForall}},
	sort  ={= 8 2 1},
	type  ={symbols},
	symbol={\formaleDefinition{\RawOjkForall x A(x)}},
	user6 ={\informelleDefinition{\DefFt{für alle} $x$ \DefFt{gilt} $A(x)$}},
	see   ={MtsForall},
	description={\baueBeschreibung[Ein \DefFt{\logischerQuantor}]{OjkForall}.}
}

\newVerweis               {\OjkExists}{\glstext}{OjkExists}
\newglossaryentry          {OjkExists}{
	name  ={\ensuremath{\RawOjkExists}},
	sort  ={= 8 2 2},
	type  ={symbols},
	symbol={\formaleDefinition{\RawOjkExists x A(x)}},
	user6 ={\informelleDefinition{\DefFt{es gibt ein} $x$ \DefFt{so dass} $A(x)$}},
	see   ={MtsExists},
	description={\baueBeschreibung[Ein \DefFt{\logischerQuantor}]{OjkExists}.}
}

\newVerweis               {\OjkExione}{\glstext}{OjkExione}
\newglossaryentry          {OjkExione}{
	name  ={\ensuremath{\RawOjkExione}},
	sort  ={= 8 2 3},
	type  ={symbols},
	symbol={\formaleDefinition{(\RawOjkExione x A(x)) \RawMtsDefEquiv ((\RawOjkExists x A(x)) \RawOjkAnd ((A(a) \RawOjkAnd A(b)) \RawOjkImp (a = b)))}},
	user6 ={\informelleDefinition{\DefFt{es gibt genau ein} $x$ \DefFt{so dass} $A(x)$}},
	see   ={MtsExione},
	description={\baueBeschreibung[Ein \DefFt{\logischerQuantor}]{OjkExione}.}
}

% ==============================================================================
% \sym* - Ausgabe als geklammertes Symbol und Eintrag ins Symbolverzeichnis
% \gls* - wie \sym*            und zusätzlich Link ins Symbolverzeichnis
% \tag* - Tag in einer Formel setzen      und Eintrag ins Symbolverzeichnis
% Verweise als geklammertes Symbol auf die Formel mit dem Tag:
%   \ref    {def:*} -->  \*
%   \eqref  {def:*} --> (\*)
%   \vreffor{def:*} --> (\*) auf Seite <n>
\newglossaryentry{Glo-Schlussregeln}{
	name     ={\gloFt{Schlussregeln}},% ========================================
	sort     ={= 9},
	type     ={symbols},
	description={}
}

% TODO ### symbol := formale Definition (für Schlussregeln)

\newcommand*    {\AR}{\ensuremath{\text{AR}}}
\newVerweis  {\glsAR}{\glstext }       {AR}
\newVerweis  {\symAR}{\glsuseri}       {AR}
\newcommand* {\tagAR}{\glsTag          {AR}}
\newglossaryentry{AR}{
	name      ={(\AR)},
	user1      ={\AR},
	sort    ={= 9 AR},
	type       ={symbols},
	description={Eine \Schlussregel: Die \Anfangsregel.}
}

\newcommand*    {\FS}{\ensuremath{\text{FS}}}
\newVerweis  {\glsFS}{\glstext }       {FS}
\newVerweis  {\symFS}{\glsuseri}       {FS}
\newcommand* {\tagFS}{\glsTag          {FS}}
\newglossaryentry{FS}{
	name      ={(\FS)},
	user1      ={\FS},
	sort    ={= 9 FS},
	type       ={symbols},
	description={Eine \Schlussregel: Ein \formalerSatz.}
}

\newcommand*    {\MR}{\ensuremath{\text{MR}}}
\newVerweis  {\glsMR}{\glstext }       {MR}
\newVerweis  {\symMR}{\glsuseri}       {MR}
\newcommand* {\tagMR}{\glsTag          {MR}}
\newglossaryentry{MR}{
	name      ={(\MR)},
	user1      ={\MR},
	sort    ={= 9 MR},
	type       ={symbols},
	description={Eine \Schlussregel: Die \Monotonieregel.}
}

\newcommand*    {\SR}{\ensuremath{\text{SR}}}
\newVerweis  {\glsSR}{\glstext }       {SR}
\newVerweis  {\symSR}{\glsuseri}       {SR}
\newcommand* {\tagSR}{\glsTag          {SR}}
\newglossaryentry{SR}{
	name      ={(\SR)},
	user1      ={\SR},
	sort    ={= 9 SR},
	type       ={symbols},
	description={Eine \Schlussregel: Die \Schnittregel.}
}

\newcommand*    {\TR}{\ensuremath{\text{TR}}}
\newVerweis  {\glsTR}{\glstext }       {TR}
\newVerweis  {\symTR}{\glsuseri}       {TR}
\newcommand* {\tagTR}{\glsTag          {TR}}
\newglossaryentry{TR}{
	name      ={(\TR)},
	user1      ={\TR},
	sort    ={= 9 TR},
	type       ={symbols},
	description={Eine \Schlussregel: Die \Abtrennungsregel.}
}

\newcommand*    {\andB}{\ensuremath{\RawOjkAnd\text{B}}}
\newVerweis  {\glsandB}{\glstext }{andB}
\newVerweis  {\symandB}{\glsuseri}{andB}
\newcommand* {\tagandB}{\glsTag   {andB}}
\newglossaryentry{andB}{
	name      ={(\andB)},
	user1      ={\andB},
	sort       ={= 9 1 1},
	type       ={symbols},
	see        ={OjkAnd},
	description={Eine \Schlussregel: Beseitigung von \OjkAnd.}
}

\newcommand*    {\andE}{\ensuremath{\RawOjkAnd\text{E}}}
\newVerweis  {\glsandE}{\glstext }        {andE}
\newVerweis  {\symandE}{\glsuseri}        {andE}
\newcommand* {\tagandE}{\glsTag           {andE}}
\newglossaryentry{andE}{
	name      ={(\andE)},
	user1      ={\andE},
	sort       ={= 9 1 2},
	type       ={symbols},
	description={Eine \Schlussregel: Einführung von \OjkAnd.}
}

\newcommand*    {\orB}{\ensuremath{\RawOjkOr\text{B}}}
\newVerweis  {\glsorB}{\glstext }        {orB}
\newVerweis  {\symorB}{\glsuseri}        {orB}
\newcommand* {\tagorB}{\glsTag           {orB}}
\newglossaryentry{orB}{
	name      ={(\orB)},
	user1      ={\orB},
	sort       ={= 9 2 1},
	type       ={symbols},
	description={Eine \Schlussregel: Beseitigung von \OjkOr.}
}

\newcommand*    {\orE}{\ensuremath{\RawOjkOr\text{E}}}
\newVerweis  {\glsorE}{\glstext }        {orE}
\newVerweis  {\symorE}{\glsuseri}        {orE}
\newcommand* {\tagorE}{\glsTag           {orE}}
\newglossaryentry{orE}{
	name      ={(\orE)},
	user1      ={\orE},
	sort       ={= 9 2 2},
	type       ={symbols},
	description={Eine \Schlussregel: Einführung von \OjkOr.}
}

\newcommand*    {\impB}{\ensuremath{\RawOjkImp\text{B}}}
\newVerweis  {\glsimpB}{\glstext }        {impB}
\newVerweis  {\symimpB}{\glsuseri}        {impB}
\newcommand* {\tagimpB}{\glsTag           {impB}}
\newglossaryentry{impB}{
	name      ={(\impB)},
	user1      ={\impB},
	sort       ={= 9 3 1},
	type       ={symbols},
	description={Eine \Schlussregel: Beseitigung von \OjkImp.}
}

\newcommand*    {\impE}{\ensuremath{\RawOjkImp\text{E}}}
\newVerweis  {\glsimpE}{\glstext }        {impE}
\newVerweis  {\symimpE}{\glsuseri}        {impE}
\newcommand* {\tagimpE}{\glsTag           {impE}}
\newglossaryentry{impE}{
	name      ={(\impE)},
	user1      ={\impE},
	sort       ={= 9 3 2},
	type       ={symbols},
	description={Eine \Schlussregel: Einführung von \OjkImp.}
}

\newcommand*    {\nota}{\ensuremath{\RawOjkNot\text{1}}}
\newVerweis  {\glsnota}{\glstext }        {nota}
\newVerweis  {\symnota}{\glsuseri}        {nota}
\newcommand* {\tagnota}{\glsTag           {nota}}
\newglossaryentry{nota}{
	name      ={(\nota)},
	user1      ={\nota},
	sort       ={= 9 4 1},
	type       ={symbols},
	description={Eine \Schlussregel: Einführung/Beseitigung von \OjkNot, Teil 1.}
}

\newcommand*    {\notb}{\ensuremath{\RawOjkNot\text{2}}}
\newVerweis  {\glsnotb}{\glstext }        {notb}
\newVerweis  {\symnotb}{\glsuseri}        {notb}
\newcommand* {\tagnotb}{\glsTag           {notb}}
\newglossaryentry{notb}{
	name      ={(\notb)},
	user1      ={\notb},
	sort       ={= 9 4 2},
	type       ={symbols},
	description={Eine \Schlussregel: Einführung/Beseitigung von \OjkNot, Teil 2.}
}

\newcommand*    {\notc}{\ensuremath{\RawOjkNot\text{3}}}
\newVerweis  {\glsnotc}{\glstext }        {notc}
\newVerweis  {\symnotc}{\glsuseri}        {notc}
\newcommand* {\tagnotc}{\glsTag           {notc}}
\newglossaryentry{notc}{
	name      ={(\notc)},
	user1      ={\notc},
	sort       ={= 9 4 3},
	type       ={symbols},
	see        ={Beweis},% Todo = Indirekter Beweis
	description={Eine \Schlussregel: Beweistechnik „\DefFt{Indirekter Beweis}“.}
}

\newcommand*    {\notd}{\ensuremath{\RawOjkNot\text{4}}}
\newVerweis  {\glsnotd}{\glstext }        {notd}
\newVerweis  {\symnotd}{\glsuseri}        {notd}
\newcommand* {\tagnotd}{\glsTag           {notd}}
\newglossaryentry{notd}{
	name      ={(\notd)},
	user1      ={\notd},
	sort       ={= 9 4 4},
	type       ={symbols},
	see        ={Beweis},% Todo = Reductio ad absurdum, Indirekter Beweis
	description={Eine \Schlussregel]: \DefFt{Reductio ad absurdum} (\DefFt{Indirekter Beweis}).}
}

\newcommand*    {\eqB}{\ensuremath{\RawOjkEq\text{B}}}
\newVerweis  {\glseqB}{\glstext }        {eqB}
\newVerweis  {\symeqB}{\glsuseri}        {eqB}
\newcommand* {\tageqB}{\glsTag           {eqB}}
\newglossaryentry{eqB}{
	name      ={(\eqB)},
	user1      ={\eqB},
	sort       ={= 9 5 1},
	type       ={symbols},
	description={Eine \Schlussregel: Beseitigung von \OjkEq.}
}

\newcommand*    {\eqE}{\ensuremath{\RawOjkEq\text{E}}}
\newVerweis  {\glseqE}{\glstext }        {eqE}
\newVerweis  {\symeqE}{\glsuseri}        {eqE}
\newcommand* {\tageqE}{\glsTag           {eqE}}
\newglossaryentry{eqE}{
	name      ={(\eqE)},
	user1      ={\eqE},
	sort       ={= 9 5 2},
	type       ={symbols},
	description={Eine \Schlussregel: Einführung von \OjkEq.}
}

% TODO ### Symbol am Seitenrand ausgeben: mit \baueBeschreibung{label} oder \SymbolAmRand{label}

% TODO ### symbol={\formaleDefinition{MathMode}},

% TODO ### user6 ={\informelleDefinition{TextMode}},

% ### Symbolverzeichnis und Index ##############################################
% Anmerkung:
%   Eigentlich gehören die weiteren aufgeführten Symbole alle zur Metasprache.
%   Solche, die zur Bildung von aussagen- und prädikatenlogischen Formeln
%   dienen, sind trotzdem mit 'Ojk' statt 'Mts' markiert.

% ==============================================================================
\newglossaryentry{Glo-TextSymbole}{
	name    ={Text-\gloFt{Symbole}},% ==========================================
	sort    ={A},
	type    ={symbols},
	description={
		Die folgenden \Symbole\ sind alphabetisch geordnet und auch im Index aufgeführt.
		$\square$ dient nur zur Verdeutlichung, an welche Stelle die Indizes gehören.
	}
}

% ==============================================================================
% Zur Unterscheidung der einbuchstabigen Symbole im Key 'sort':
% ... _   = Index
% ... Ber = Bereich
% ... Men = Menge
% ... MeT = Teilmenge
% ... Rel = Relation
% ... Tup = Folge/Tupel
% ... xEl = Element (allgemein)

% ==============================================================================
% \StrMtsIdx* - Ausgabe als Text-Symbol und Eintrag und Link ins Symbolverzeichnis
% \Mts* - Ausgabe als Text-Symbol und Eintrag und Link ins Symbolverzeichnis
% Operationen mit Namen (Buchstaben) ===========================================

\newcommand*             {\StrMtsValue}            {val}% Definitionsbereich [domain]
\newVerweis                 {\MtsValue}{\glstext}{MtsValue}
\newglossaryentry            {MtsValue}{
	text    ={\ensuremath{\RawMtsValue}},
	name    ={\ensuremath{\RawMtsValue} \addIdx[
		name={\ensuremath{\RawMtsValue}},
		sort={val}]                              {MtsValue}},
	sort    ={dom},%      \StrMtsValue
	type    ={symbols},
	symbol  ={\formaleDefinition{\RawMtsValue}},
	see     ={Formel,Variable,Wert},
	description={
		Der \Wert\ einer \Formel, nachdem die \Variablen\ mit \Werten\ belegt wurden.
	}
}

\newcommand*                      {\LtrMtsIdxEndlich}{e}% nur endliche Elemente
\newVerweis                          {\MtsIdxEndlich} {\glstext}{MtsIdxEndlich}
\newglossaryentry                     {MtsIdxEndlich}{
	text    ={\ensuremath         {\RawMtsIdxEndlich}},
	name    ={\ensuremath{\square_{\RawMtsIdxEndlich}} \addIdx[
		name={\ensuremath{\square_{\RawMtsIdxEndlich}}},
		sort={e _}]                                             {MtsIdxEndlich}},
	sort    ={e _},%         \LtrIdxEndlich Index
	type    ={symbols},
	symbol  ={},
	user6   ={},
	see     ={Menge},
	description={
		Eine \Operation\ mittels eines Index:
		\[
			X_{\RawMtsIdxEndlich} \RawMtsDefEq
			\begin{cases}
				\RawMengeDef{M  \RawMtsIn X}{|M|                   \quad \RawMtsIn \RawMtsINo}
				& \text{, für eine \Menge\ $X$ von \Mengen}     \\
				\RawMengeDef{R\;\RawMtsIn X}{|R_{\RawMtsIdxGraph}| \quad \RawMtsIn \RawMtsINo}
				& \text{, für eine \Menge\ $X$ von \Relationen} \\
				\RawMengeDef{F\;\RawMtsIn X}{\RawMtsLen(F)               \RawMtsIn \RawMtsINo}
				& \text{, für eine \Menge\ $X$ von \Folgen}
			\end{cases}
		\]
	}
}

\newcommand*                      {\LtrMtsIdxGraph}{g}%              Graph von
\newVerweis                          {\MtsIdxGraph} {\glstext}{MtsIdxGraph}
\newglossaryentry                     {MtsIdxGraph}{
	text    ={\ensuremath         {\RawMtsIdxGraph}},
	name    ={\ensuremath{\square_{\RawMtsIdxGraph}} \addIdx[
		name={\ensuremath{\square_{\RawMtsIdxGraph}}},
		sort={g _}]                                           {MtsIdxGraph}},
	sort    ={g _},%               \LtrMtsIdxGraph Index
	type    ={symbols},
	symbol  ={},
	user6   ={},
	see     ={},
	description={
		Eine \Operation\ mittels eines Index:
		$X_{\RawMtsIdxGraph} \RawMtsDefEq \RawMtsGraph(X)$ für \Funktionen\ und \Relationen\ $X$.
	}
}

\newcommand*                      {\LtrMtsIdxPolnisch}{p}% in Polnischer Notation
\newVerweis                          {\MtsIdxPolnisch} {\glstext}{MtsIdxPolnisch}
\newglossaryentry                     {MtsIdxPolnisch}{
	text    ={\ensuremath         {\RawMtsIdxPolnisch}},
	name    ={\ensuremath{\square^{\RawMtsIdxPolnisch}} \addIdx[
		name={\ensuremath{\square^{\RawMtsIdxPolnisch}}},
		sort={p _}]                                              {MtsIdxPolnisch}},
	sort    ={p _},%               \LtrMtsIdxPolnisch Index
	type    ={symbols},
	symbol  ={},
	user6   ={}, 
	see     ={Menge},
	description={
		Eine \Operation\ mittels eines Index: Für eine \Menge\ $L$ von \Formeln\ und eine \Formel\ $\alpha$ ist\\
		$L^{\RawMtsIdxPolnisch} \RawMtsDefEq \RawMengeDef{\alpha^{\RawMtsIdxPolnisch}}{\alpha \RawMtsIn L}$.
		mit $\alpha^{\RawMtsIdxPolnisch} \RawMtsDefEq ( \alpha$ umgewandelt in \PolnischeNotation ).
	}
}

\newcommand*             {\StrMtsDb}               {dom}% Definitionsbereich [domain]
\newVerweis                 {\MtsDb}{\glstext}{MtsDb}
\newglossaryentry            {MtsDb}{
	text    ={\ensuremath{\RawMtsDb}},
	name    ={\ensuremath{\RawMtsDb} \addIdx[
		name={\ensuremath{\RawMtsDb}},
		sort={dom}]                           {MtsDb}},
	sort    ={dom},%      \StrMtsDb
	type    ={symbols},
	symbol  ={},
	user6   ={},
	see     ={},
	description={
		Für eine \Funktion\ \FunktionDef{f}{A}{B} ist $\MtsDb(f) \MtsDefEq A$, der \Definitionsbereich\ von $f$.
	}
}

\newcommand*             {\LtrMtsFol}              {F}%  Folgenmenge
\newVerweis                 {\MtsFol}{\glstext}{MtsFol}
\newglossaryentry            {MtsFol}{
	text    ={\ensuremath{\RawMtsFol}},
	name    ={\ensuremath{\RawMtsFol} \addIdx[
		name={\ensuremath{\RawMtsFol}},
		sort={F Tup}]                          {MtsFol}},
	sort    ={F Tup},%    \LtrMtsFol Folge
	type    ={symbols},
	symbol  ={},
	user6   ={},
	see     ={MtsFolf,Menge},
	description={
		$\RawMtsFol(M) \RawMtsDefEq \RawMengeDef{F}{F \text{ ist \Folge\ über } M}$.
	}
}

\newVerweis                 {\MtsFolf}{\glstext}{MtsFolf}
\newglossaryentry            {MtsFolf}{
	text    ={\ensuremath{\RawMtsFolf}},
	name    ={\ensuremath{\RawMtsFolf} \addIdx[
		name={\ensuremath{\RawMtsFolf}},
		sort={F Tup e}]                         {MtsFolf}},
	sort    ={F Tup e},%  \LtrMtsFol Folge \LtrIdxEndlich
	type    ={symbols},
	symbol  ={},
	user6   ={},
	see     ={MtsFol,Folgenmenge,Menge},
	description={
		$\RawMtsFol(M) \RawMtsDefEq \RawMengeDef{F \RawMtsIn \RawMtsFol(M)}{\RawMtsLen(F) \RawMtsIn \RawMtsINo}$.
	}
}

\newcommand*             {\StrMtsGraph}            {graph}% Graph; Funktionen/Relationen
\newVerweis                 {\MtsGraph}{\glstext}{MtsGraph}
\newglossaryentry            {MtsGraph}{
	text    ={\ensuremath{\RawMtsGraph}},
	name    ={\ensuremath{\RawMtsGraph} \addIdx[
		name={\ensuremath{\RawMtsGraph}},
		sort={graph}]                            {MtsGraph}},
	sort    ={graph},%    \StrMtsGraph
	type    ={symbols},
	symbol  ={},
	user6   ={},
	see     ={Graph,Menge},
	description={
		Für eine \Relation\ $R = (G, A_1, \dots, A_n)$ ist $\RawMtsGraph(R) \RawMtsDefEq G$.\\
		Für eine \Funktion\ \FunktionDef{f}{A}{B} ist $\RawMtsGraph(f) \RawMtsDefEq \RawMengeDef{(a,f(a))}{a \RawMtsIn A}$.
	}
}

\newcommand*             {\StrMtsLen}              {len}% Länge [length] (Tupel)
\newVerweis                 {\MtsLen}{\glstext}{MtsLen}
\newglossaryentry            {MtsLen}{
	text    ={\ensuremath{\RawMtsLen}},
	name    ={\ensuremath{\RawMtsLen} \addIdx[
		name={\ensuremath{\RawMtsLen}},
		sort={len}]                            {MtsLen}},
	sort    ={len},%      \StrMtsLen
	type    ={symbols},
	symbol  ={},
	user6   ={},
	see     ={},
	description={
		$\RawMtsLen(\vec{a}) \RawMtsDefEq$ Anzahl der \Komponenten\ einer endlichen \Folge\, \textdh\ eines \Tupels\ $\vec{a}$
	}
}

\newcommand*             {\LtrMtsPot}              {P}%  Potenzmenge
\newVerweis                 {\MtsPot}{\glstext}{MtsPot}
\newglossaryentry            {MtsPot}{
	text    ={\ensuremath{\RawMtsPot}},
	name    ={\ensuremath{\RawMtsPot} \addIdx[
		name={\ensuremath{\RawMtsPot}},
		sort={P Men}]                          {MtsPot}},
	sort    ={P Men},%    \LtrMtsPot Menge
	type    ={symbols},
	symbol  ={},
	user6   ={},
	see     ={MtsPotf,Menge},
	description={
		$\RawMtsPot(M) \RawMtsDefEq \RawMengeDef{N}{N \RawMtsSubsetEq M}$, die \Potenzmenge\ einer \Menge\ $M$.
	}
}

\newVerweis                 {\MtsPotf}{\glstext}{MtsPotf}
\newglossaryentry            {MtsPotf}{
	text    ={\ensuremath{\RawMtsPotf}},
	name    ={\ensuremath{\RawMtsPotf} \addIdx[
		name={\ensuremath{\RawMtsPotf}},
		sort={P Men e}]                         {MtsPotf}},
	sort    ={P Men e},%  \LtrMtsPot \LtrIdxEndlich Menge
	type    ={symbols},
	symbol  ={},
	user6   ={},
	see     ={Menge},
	description={
		$\RawMtsPot(M) \RawMtsDefEq \RawMengeDef{N \RawMtsIn \RawMtsPot(M)}{|N| \RawMtsIn \RawMtsINo}$.
	}
}

\newcommand*             {\StrMtsQb}               {src}% Quellbereich [source]
\newVerweis                 {\MtsQb}{\glstext}{MtsQb}
\newglossaryentry            {MtsQb}{
	text    ={\ensuremath{\RawMtsQb}},
	name    ={\ensuremath{\RawMtsQb} \addIdx[
		name={\ensuremath{\RawMtsQb}},
		sort={src}]                           {MtsQb}},
	sort    ={src},%      \StrMtsQb
	type    ={symbols},
	symbol  ={},
	user6   ={},
	see     ={Menge},
	description={
		Für eine \Funktion\ \FunktionDef{f}{A}{B} ist $\RawMtsQb(f) \RawMtsDefEq \RawMengeDef{a \in A}{f(a) \text{ existiert}}$ der \Quellbereich\ von $f$.
	}
}

\newcommand*             {\LtrMtsRel}              {R}% Menge der Relationen
\newVerweis                 {\MtsRel}{\glstext}{MtsRel}
\newglossaryentry            {MtsRel}{
	text    ={\ensuremath{\RawMtsRel}},
	name    ={\ensuremath{\RawMtsRel} \addIdx[
		name={\ensuremath{\RawMtsRel}},
		sort={R Men}]                          {MtsRel}},
	sort    ={R Men},%    \LtrMtsRel Menge
	type    ={symbols},
	symbol  ={},
	user6   ={},
	see     ={MtsRelf,Relation},
	description={
		Für eine \Menge\ $M$ ist RAWMtsRel\ RAWMtsDefEq\ die \Menge\ der \binaeren\ \Relationen\ in $M$.
	}
}

\newVerweis                 {\MtsRelf}{\glstext}{MtsRelf}%
\newglossaryentry            {MtsRelf}{
	text    ={\ensuremath{\RawMtsRelf}},
	name    ={\ensuremath{\RawMtsRelf} \addIdx[
		name={\ensuremath{\RawMtsRelf}},
		sort={R Men e}]                         {MtsRelf}},
	sort    ={R Men e},%  \LtrMtsRel Menge \LtrIdxEndlich
	type    ={symbols},
	symbol  ={},
	user6   ={},
	see     ={Menge},
	description={
		Für eine \Menge\ $M$ ist $\RawMtsRelf(M) \RawMtsDefEq \RawMengeDef{R \RawMtsIn \RawMtsRel(M)}{|R_{\RawMtsIdxEndlich}| \RawMtsIn \RawMtsINo}$ die \Menge\ der endlichen, \binaeren\ \Relationen\ in $M$.
	}
}

\newcommand*             {\StrMtsSet}              {set}% Komponentenmenge (Tupel/Folge)
\newVerweis                 {\MtsSet}{\glstext}{MtsSet}
\newglossaryentry            {MtsSet}{
	text    ={\ensuremath{\RawMtsSet}},
	name    ={\ensuremath{\RawMtsSet} \addIdx[
		name={\ensuremath{\RawMtsSet}},
		sort={Set}]                            {MtsSet}},
	sort    ={Set},%      \StrMtsSet
	type    ={symbols},
	symbol  ={},
	user6   ={},
	see     ={Folge,Komponentenmenge,Menge,Tupel},
	description={
		$\RawMtsSet(\vec{a}) \RawMtsDefEq \RawMengeDef{a}{a \RawMtsSeqIn \vec{a}}$.
	}
}

\newcommand*             {\StrMtsStel}             {stel}% [Stel]ligkeit Funktionen/Relationen
\newVerweis                 {\MtsStelF}{\glstext}{MtsStelF}
\newglossaryentry            {MtsStelF}{
	text    ={\ensuremath{\RawMtsStelF}},
	name    ={\ensuremath{\RawMtsStelF} \addIdx[
		name={\ensuremath{\RawMtsStelF}},
		sort={stel f}]                           {MtsStelF}},
	sort    ={stel f},%   \StrMtsStel f
	type    ={symbols},
	symbol  ={},
	user6   ={},
	see     ={Funktion,Stelligkeit},
	description={
		$\RawMtsStelF(f) \RawMtsDefEq n$ für $\FunktionDef{f}{A_1 \RawMtsTimes \dots \RawMtsTimes A_n}{B}$.
	}
}

\newVerweis                 {\MtsStelR}{\glstext}{MtsStelR}
\newglossaryentry            {MtsStelR}{
	text    ={\ensuremath{\RawMtsStelR}},
	name    ={\ensuremath{\RawMtsStelR} \addIdx[
		name={\ensuremath{\RawMtsStelR}},
		sort={stel r}]                           {MtsStelR}},
	sort    ={stel r},%   \StrMtsStel r
	type    ={symbols},
	symbol  ={},
	user6   ={},
	see     ={Relation,Stelligkeit},
	description={
		$\RawMtsStelR(R) \RawMtsDefEq n$ für $R \RawMtsSubsetEq A_1 \RawMtsTimes \dots \RawMtsTimes A_n$.
	}
}

\newcommand*             {\StrMtsTraeger}          {car}% Trägermenge [carrier] (Relation)
\newVerweis                 {\MtsTraeger}{\glstext}{MtsTraeger}
\newglossaryentry            {MtsTraeger}{
	text    ={\ensuremath{\RawMtsTraeger}},
	name    ={\ensuremath{\RawMtsTraeger} \addIdx[
		name={\ensuremath{\RawMtsTraeger}},
		sort={car}]                                {MtsTraeger}},
	sort    ={car},%      \StrMtsTraeger
	type    ={symbols},
	symbol  ={},
	user6   ={},
	see     ={Traegermenge},
	description={
		Für eine \Relation%
		\footnote{%
			\Funktionen\ sind spezielle \Relationen.
			Für eine \Funktion\ $\FunktionDef{f}{A_1 \RawMtsTimes \dots \RawMtsTimes A_n}{B}$ gilt demnach:
			\\$\RawMtsTraeger(f) \RawMtsDefEq A_1 \RawMtsTimes \dots \RawMtsTimes A_n \RawMtsTimes B$;
			\quad $\RawMtsTraeger_i(f) \RawMtsDefEq A_i$ für $1 \le i \le n$;
			\quad $\RawMtsTraeger_{n+1}(f) \RawMtsDefEq B$
		}
		$R = (G, A_1, \dots, A_n)$ ist
		$\RawMtsTraeger(R) \RawMtsDefEq A_1 \RawMtsTimes \dots \RawMtsTimes A_n$ und
		\ifmarginparFlg\newline\else\fi
		$\RawMtsTraeger_i(R) \RawMtsDefEq A_i$ für $1 \le i \le n$.
	}
}

\newcommand*             {\LtrMtsUniversum}        {U}%   Diskursuniversum [Universe of Discourse]
\newVerweis                 {\MtsUniversum}{\glstext}{MtsUniversum}
\newglossaryentry            {MtsUniversum}{
	text    ={\ensuremath{\RawMtsUniversum}},
	name    ={\ensuremath{\RawMtsUniversum} \addIdx[
		name={\ensuremath{\RawMtsUniversum}},
		sort={U Ber}]                                {MtsUniversum}},
	sort    ={U Ber},%    \LtrMtsUniversum Bereich
	type    ={symbols},
	symbol  ={},
	user6   ={},
	see     ={},
	description={
		Das \Diskursuniversum.
	}
}

\newcommand*             {\StrMtsWb}               {ran}% Wertebereich [range]
\newVerweis                 {\MtsWb}{\glstext}{MtsWb}
\newglossaryentry            {MtsWb}{
	text    ={\ensuremath{\RawMtsWb}},
	name    ={\ensuremath{\RawMtsWb} \addIdx[
		name={\ensuremath{\RawMtsWb}},
		sort={ran}]                           {MtsWb}},
	sort    ={ran},%      \StrMtsWb
	type    ={symbols},
	symbol  ={},
	user6   ={},
	see     ={Menge},
	description={
		Für eine \Funktion\ \FunktionDef{f}{A}{B} ist $\RawMtsWb(f) \RawMtsDefEq \RawMengeDef{f(a)}{a \in A}$ der \Wertebereich\ von $f$.
	}
}

\newcommand*             {\StrMtsZb}               {tar}% Zielbereich [target]
\newVerweis                 {\MtsZb}{\glstext}{MtsZb}
\newglossaryentry            {MtsZb}{
	text    ={\ensuremath{\RawMtsZb}},
	name    ={\ensuremath{\RawMtsZb} \addIdx[
		name={\ensuremath{\RawMtsZb}},
		sort={tar}]                           {MtsZb}},
	sort    ={tar},%      \StrMtsZb
	type    ={symbols},
	symbol  ={},
	user6   ={},
	see     ={},
	description={
		Für eine \Funktion\ \FunktionDef{f}{A}{B} ist $\RawMtsZb(f) \RawMtsDefEq B$ der \Zielbereich\ von $f$.
	}
}

% ==============================================================================
% \Mts* - Ausgabe als Text-Symbol und Eintrag und Link ins Symbolverzeichnis
% Mengen und Elemente ==========================================================

\newcommand*             {\LtrMtsAussagen}         {A}% Aussagen
\newVerweis                 {\MtsAussagen}{\glstext}{MtsAussagen}
\newglossaryentry            {MtsAussagen}{
	text    ={\ensuremath{\RawMtsAussagen}},
	name    ={\ensuremath{\RawMtsAussagen} \addIdx[
		name={\ensuremath{\RawMtsAussagen}},
		sort={A Ber}]                               {MtsAussagen}},
	sort    ={A Ber},%    \LtrMtsAussagen   Bereich
	type    ={symbols},
	symbol  ={},
	user6   ={},
	see     ={},
	description={\todoOk%
		Der \Bereich\ der \Aussagen\ in \Objektsprache.
	}
}

\newcommand*             {\LtrMtsAxiom}            {X}%        A[x]iom
\newVerweis                 {\MtsAxiom}{\glstext}{MtsAxiom}
\newglossaryentry            {MtsAxiom}{
	text    ={\ensuremath{\RawMtsAxiom}},
	name    ={\ensuremath{\RawMtsAxiom} \addIdx[
		name={\ensuremath{\RawMtsAxiom}},
		sort={X xEl}]                            {MtsAxiom}},
	sort    ={X xEl},%    \LtrMtsAxiom Element
	type    ={symbols},
	symbol  ={},
	user6   ={},
	see     ={},
	description={
		Ein \Axiom.
	}
}

\newVerweis                 {\MtsAxiomSet}{\glstext}{MtsAxiomSet}
\newglossaryentry            {MtsAxiomSet}{
	text    ={\ensuremath{\RawMtsAxiomSet}},
	name    ={\ensuremath{\RawMtsAxiomSet} \addIdx[
		name={\ensuremath{\RawMtsAxiomSet}},
		sort={X Men}]                               {MtsAxiomSet}},
	sort    ={X Men},%    \LtrMtsAxiom Menge
	type    ={symbols},
	symbol  ={},
	user6   ={},
	see     ={},
	description={
		Eine \Menge\ von \Axiomen.
	}
}

\newcommand*             {\LtrMtsBeweisschritt}    {b}%              Beweisschritt
\newVerweis                 {\MtsBeweisschritt}{\glstext}{MtsBeweisschritt}
\newglossaryentry            {MtsBeweisschritt}{
	text    ={\ensuremath{\RawMtsBeweisschritt}},
	name    ={\ensuremath{\RawMtsBeweisschritt} \addIdx[
		name={\ensuremath{\RawMtsBeweisschritt}},
		sort={b xEl}]                                    {MtsBeweisschritt}},
	sort    ={b xEl},%    \LtrMtsBeweisschritt Element
	type    ={symbols},
	symbol  ={},
	user6   ={},
	see     ={},
	description={
		Ein \Beweisschritt.
	}
}

\newVerweis                 {\MtsBeweisschrittTup}{\glstext}{MtsBeweisschrittTup}
\newglossaryentry            {MtsBeweisschrittTup}{
	text    ={\ensuremath{\RawMtsBeweisschrittTup}},
	name    ={\ensuremath{\RawMtsBeweisschrittTup} \addIdx[
		name={\ensuremath{\RawMtsBeweisschrittTup}},
		sort={b Tup}]                                       {MtsBeweisschrittTup}},
	sort    ={b Tup},%    \LtrMtsBeweisschritt Tupel
	type    ={symbols},
	symbol  ={},
	user6   ={},
	see     ={},
	description={
		Ein \Tupel\ von \Beweisschritten.
	}
}

\newcommand*             {\LtrMtsBeweisschrittSet} {B}% Menge der       Beweisschritte
\newVerweis                 {\MtsBeweisschrittSet}{\glstext}{MtsBeweisschrittSet}
\newglossaryentry            {MtsBeweisschrittSet}{
	text    ={\ensuremath{\RawMtsBeweisschrittSet}},
	name    ={\ensuremath{\RawMtsBeweisschrittSet} \addIdx[
		name={\ensuremath{\RawMtsBeweisschrittSet}},
		sort={B Men}]                                       {MtsBeweisschrittSet}},
	sort    ={B Men},%    \LtrMtsBeweisschrittSet Menge
	type    ={symbols},
	symbol  ={},
	user6   ={},
	see     ={},
	description={
		Eine \Menge\ von \Beweisschritten.
	}
}

\newcommand*             {\LtrMtsErgebnis}         {e}% result; Ergebnis
\newVerweis                 {\MtsErgebnis}{\glstext}{MtsErgebnis}
\newglossaryentry            {MtsErgebnis}{
	text    ={\ensuremath{\RawMtsErgebnis}},
	name    ={\ensuremath{\RawMtsErgebnis} \addIdx[
		name={\ensuremath{\RawMtsErgebnis}},
		sort={r xEl}]                               {MtsErgebnis}},
	sort    ={r xEl},%    \LtrMtsErgebnis Element
	type    ={symbols},
	symbol  ={},
	user6   ={},
	see     ={},
	description={
		Ein \Ergebnis.
	}
}

\newcommand*             {\LtrMtsErgebnisSet}      {E}% resultset; Ergebnismeng
\newVerweis                 {\MtsErgebnisSet}{\glstext}{MtsErgebnisSet}
\newglossaryentry            {MtsErgebnisSet}{
	text    ={\ensuremath{\RawMtsErgebnisSet}},
	name    ={\ensuremath{\RawMtsErgebnisSet} \addIdx[
		name={\ensuremath{\RawMtsErgebnisSet}},
		sort={R Men}]                                  {MtsErgebnisSet}},
	sort    ={R Men},%    \LtrMtsErgebnisSet   Menge
	type    ={symbols},
	symbol  ={},
	user6   ={},
	see     ={},
	description={
		Eine \Menge\ von \Ergebnissen.
	}
}

\newVerweis                 {\MtsErgebnisRel}{\glstext}{MtsErgebnisRel}
\newglossaryentry            {MtsErgebnisRel}{
	text    ={\ensuremath{\RawMtsErgebnisRel}},
	name    ={\ensuremath{\RawMtsErgebnisRel} \addIdx[
		name={\ensuremath{\RawMtsErgebnisRel}},
		sort={R Rel}]                                  {MtsErgebnisRel}},
	sort    ={R Rel},%    \LtrMtsErgebnisSet Relation
	type    ={symbols},
	symbol  ={},
	user6   ={},
	see     ={},
	description={
		Eine \Relation\ (aufgefasst als \Menge) von \Ergebnissen.
	}
}

\newcommand*             {\LtrMtsErsetzung}        {E}% Substitution; Ersetzung
\newVerweis                 {\MtsErsetzung}{\glstext}{MtsErsetzung}
\newglossaryentry            {MtsErsetzung}{
	text    ={\ensuremath{\RawMtsErsetzung}},
	name    ={\ensuremath{\RawMtsErsetzung} \addIdx[
		name={\ensuremath{\RawMtsErsetzung}},
		sort={E xEl}]                                {MtsErsetzung}},
	sort    ={E xEl},%    \LtrMtsErsetzung Element
	type    ={symbols},
	symbol  ={},
	user6   ={},
	see     ={MtsErsetzungSet},
	description={
		Eine \Ersetzung.
	}
}

\newVerweis                 {\MtsErsetzungSet}{\glstext}{MtsErsetzungSet}
\newglossaryentry            {MtsErsetzungSet}{
	text    ={\ensuremath{\RawMtsErsetzungSet}},
	name    ={\ensuremath{\RawMtsErsetzungSet} \addIdx[
		name={\ensuremath{\RawMtsErsetzungSet}},
		sort={E Men}]                                   {MtsErsetzungSet}},
	sort    ={E Men},%    \LtrMtsErsetzung Menge
	type    ={symbols},
	symbol  ={},
	user6   ={},
	see     ={MtsErsetzung},
	description={
		Eine \Menge\ von \Ersetzungen.
	}
}

\newcommand*             {\LtrMtsKonklusion}       {k}% eine     Konklusion
\newVerweis                 {\MtsKonklusion}{\glstext}{MtsKonklusion}
\newglossaryentry            {MtsKonklusion}{
	text    ={\ensuremath{\RawMtsKonklusion}},
	name    ={\ensuremath{\RawMtsKonklusion} \addIdx[
		name={\ensuremath{\RawMtsKonklusion}},
		sort={k xEl}]                                {MtsKonklusion}},
	sort    ={k xEl},%    \LtrMtsKonklusion Element
	type    ={symbols},
	symbol  ={},
	user6   ={},
	see     ={},
	description={
		Eine \Konklusion.
	}
}

\newcommand*             {\LtrMtsKonklusionSet}    {K}% Menge von   Konklusionen
\newVerweis                 {\MtsKonklusionSet}{\glstext}{MtsKonklusionSet}
\newglossaryentry            {MtsKonklusionSet}{
	text    ={\ensuremath{\RawMtsKonklusionSet}},
	name    ={\ensuremath{\RawMtsKonklusionSet} \addIdx[
		name={\ensuremath{\RawMtsKonklusionSet}},
		sort={K Men}]                                   {MtsKonklusionSet}},
	sort    ={K Men},%    \LtrMtsKonklusionSet Menge
	type    ={symbols},
	symbol  ={},
	user6   ={},
	see     ={},
	description={
		Eine \Menge\ von \Konklusionen.
	}
}

\newVerweis                 {\MtsKonklusionRel}{\glstext}{MtsKonklusionRel}
\newglossaryentry            {MtsKonklusionRel}{
	text    ={\ensuremath{\RawMtsKonklusionRel}},
	name    ={\ensuremath{\RawMtsKonklusionRel} \addIdx[
		name={\ensuremath{\RawMtsKonklusionRel}},
		sort={K Rel}]                                    {MtsKonklusionRel}},
	sort    ={K Rel},%    \LtrMtsKonklusionSet Relation
	type    ={symbols},
	symbol  ={},
	user6   ={},
	see     ={},
	description={
		Eine \Relation\ (aufgefasst als \Menge) von \Konklusionen.
	}
}

\newVerweis                 {\MtsEmptyset}{\glstext}{MtsEmptyset}
\newglossaryentry            {MtsEmptyset}{
	text    ={\ensuremath{\RawMtsEmptyset}},
	name    ={\ensuremath{\RawMtsEmptyset} \addIdx[
		name={\ensuremath{\RawMtsEmptyset}},
		sort={O Men}]                               {MtsEmptyset}},
	sort    ={O Men},% ... Menge
	type    ={symbols},
	symbol  ={},
	user6   ={},
	see     ={},
	description={
		Die \leereMenge, \textdh\ die einzige \Menge\ ohne \Elemente; auch mit $\{\}$ bezeichnet.
	}
}

\newcommand*             {\LtrMtsIN}               {N}% Natürliche Zahlen
\newVerweis                 {\MtsIN}{\glstext}{MtsIN}
\newglossaryentry            {MtsIN}{
	text    ={\ensuremath{\RawMtsIN}},
	name    ={\ensuremath{\RawMtsIN} \addIdx[
		name={\ensuremath{\RawMtsIN}},
		sort={N Men}]                         {MtsIN}},
	sort    ={N Men},%    \LtrMtsIN Menge
	type    ={symbols},
	symbol  ={},
	user6   ={},
	see     ={},
	description={
		Die \Menge\ der \natuerlichenZahlen\ ohne 0.
	}
}

\newVerweis                 {\MtsINo}{\glstext}{MtsINo}
\newglossaryentry            {MtsINo}{
	text    ={\ensuremath{\RawMtsINo}},
	name    ={\ensuremath{\RawMtsINo} \addIdx[
		name={\ensuremath{\RawMtsINo}},
		sort={N Men 0}]                        {MtsINo}},
	sort    ={N Men 0},%  \LtrMtsIN Menge 0
	type    ={symbols},
	symbol  ={},
	user6   ={},
	see     ={},
	description={
		Die \Menge\ der \natuerlichenZahlen\ (mit 0).
	}
}

\newVerweis                 {\MtsMn}{\glstext}{MtsMn}
\newglossaryentry            {MtsMn}{
	text    ={\ensuremath{\RawMtsMn}},
	name    ={\ensuremath{\RawMtsMn} \addIdx[
		name={\ensuremath{\RawMtsMn}},
		sort={M Men n}]                       {MtsMn}},
	sort    ={M Men n},% ... Menge
	type    ={symbols},
	symbol  ={},
	user6   ={},
	see     ={Tupel},
	description={
		Das \kartesischeProdukt\ $M \RawMtsTimes \dots \RawMtsTimes M$ aus $n$ \Mengen\ $M$ mit $n \RawMtsIn \RawMtsINo$.
	}
}

\newVerweis                 {\MtsMo}{\glstext}{MtsMo}
\newglossaryentry            {MtsMo}{
	text    ={\ensuremath{\RawMtsMo}},
	name    ={\ensuremath{\RawMtsMo} \addIdx[
		name={\ensuremath{\RawMtsMo}},
		sort={M Men 0}]                       {MtsMo}},
	sort    ={M Men 0},
	type    ={symbols},
	symbol  ={},
	user6   ={},
	see     ={},
	description={
		$\{()\}$, wobei $()$ das $0$-\Tupel\ ist.
	}
}

\newcommand*             {\LtrMtsObjekte}         {O}% Objekte
\newVerweis                 {\MtsObjekte}{\glstext}{MtsObjekte}
\newglossaryentry            {MtsObjekte}{
	text    ={\ensuremath{\RawMtsObjekte}},
	name    ={\ensuremath{\RawMtsObjekte} \addIdx[
		name={\ensuremath{\RawMtsObjekte}},
		sort={A Ber}]                               {MtsObjekte}},
	sort    ={A Ber},%    \LtrMtsObjekte   Bereich
	type    ={symbols},
	symbol  ={},
	user6   ={},
	see     ={},
	description={\todoOk%
		Der \Bereich\ der \Objekte\ in \Objektsprache, die \Parameter\ von \Aussagen\ aus \MtsAussagen\ ersetzen dürfen.
	}
}

\newcommand*             {\LtrMtsPraemisse}        {p}% Eine Voraussetzung; Prämisse
\newVerweis                 {\MtsPraemisse}{\glstext}{MtsPraemisse}
\newglossaryentry            {MtsPraemisse}{
	text    ={\ensuremath{\RawMtsPraemisse}},
	name    ={\ensuremath{\RawMtsPraemisse} \addIdx[
		name={\ensuremath{\RawMtsPraemisse}},
		sort={p xEl}]                                {MtsPraemisse}},
	sort    ={p xEl},%    \LtrMtsPraemisse   Element
	type    ={symbols},
	symbol  ={},
	user6   ={},
	see     ={},
	description={
		Eine \Praemisse.
	}
}

\newcommand*             {\LtrMtsPraemisseSet}     {P}% Menge der Voraussetzungen; Prämissen
\newVerweis                 {\MtsPraemisseSet}{\glstext}{MtsPraemisseSet}
\newglossaryentry            {MtsPraemisseSet}{
	text    ={\ensuremath{\RawMtsPraemisseSet}},
	name    ={\ensuremath{\RawMtsPraemisseSet} \addIdx[
		name={\ensuremath{\RawMtsPraemisseSet}},
		sort={P Men}]                                   {MtsPraemisseSet}},
	sort    ={P Men},%    \LtrMtsPraemisseSet Menge
	type    ={symbols},
	symbol  ={},
	user6   ={},
	see     ={},
	description={
		Eine \Menge\ von \Praemissen.
	}
}

\newVerweis                 {\MtsPraemisseRel}{\glstext}{MtsPraemisseRel}
\newglossaryentry            {MtsPraemisseRel}{
	text    ={\ensuremath{\RawMtsPraemisseRel}},
	name    ={\ensuremath{\RawMtsPraemisseRel} \addIdx[
		name={\ensuremath{\RawMtsPraemisseRel}},
		sort={P Rel}]                                   {MtsPraemisseRel}},
	sort    ={P Rel},%    \LtrMtsPraemisseSet Relation
	type    ={symbols},
	symbol  ={},
	user6   ={},
	see     ={},
	description={
		Eine \Relation\ (aufgefasst als \Menge) von \Praemissen.
	}
}

\newcommand*             {\LtrMtsSchlussregel}     {C}% conclusionrule; Schlussregel
\newVerweis                 {\MtsSchlussregel}{\glstext}{MtsSchlussregel}
\newglossaryentry            {MtsSchlussregel}{
	text    ={\ensuremath{\RawMtsSchlussregel}},
	name    ={\ensuremath{\RawMtsSchlussregel} \addIdx[
		name={\ensuremath{\RawMtsSchlussregel}},
		sort={C xEl}]                                   {MtsSchlussregel}},
	sort    ={C xEl},%    \LtrMtsSchlussregel Element
	type    ={symbols},
	symbol  ={},
	user6   ={},
	see     ={},
	description={
		Eine \Schlussregel.
	}
}

\newVerweis                 {\MtsSchlussregelSet}{\glstext}{MtsSchlussregelSet}
\newglossaryentry            {MtsSchlussregelSet}{
	text    ={\ensuremath{\RawMtsSchlussregelSet}},
	name    ={\ensuremath{\RawMtsSchlussregelSet} \addIdx[
		name={\ensuremath{\RawMtsSchlussregelSet}},
		sort={C Men}]                                      {MtsSchlussregelSet}},
	sort    ={C Men},%    \LtrMtsSchlussregel Menge
	type    ={symbols},
	symbol  ={},
	user6   ={},
	see     ={},
	description={
		Eine \Menge\ von \Schlussregeln.
	}
}

\newcommand*             {\LtrMtsSprache}          {L}%      Sprache; language; \LtrOjkFor
\newVerweis                 {\MtsSprache}{\glstext}{MtsSprache}
\newglossaryentry            {MtsSprache}{
	text    ={\ensuremath{\RawMtsSprache}},
	name    ={\ensuremath{\RawMtsSprache} \addIdx[
		name={\ensuremath{\RawMtsSprache}},
		sort={L Men}]                              {MtsSprache}},
	sort    ={L Men},%    \LtrMtsSprache Menge
	type    ={symbols},
	symbol  ={},
	user6   ={},
	see     ={Formelmenge},
	description={
		Eine \Sprache.
	}
}

\newcommand*             {\LtrMtsTup}              {T}% sequenz; Menge der Tupel
\newVerweis                 {\MtsTup}{\glstext}{MtsTup}
\newglossaryentry            {MtsTup}{
	text    ={\ensuremath{\RawMtsTup}},
	name    ={\ensuremath{\RawMtsTup} \addIdx[
		name={\ensuremath{\RawMtsTup}},
		sort={T Men}]                              {MtsTup}},
	sort    ={T Men},%    \LtrMtsTup Menge
	type    ={symbols},
	symbol  ={},
	user6   ={},
	see     ={Tupelmenge},
	description={
		Eine \Bereichsoperation: $\RawMtsTup(M)$ ist die \Menge\ aller \Tupel\ von $M$.
	}
}

\newcommand*             {\LtrMtsTransformation}   {T}% Transformation, Transformation,
\newVerweis                 {\MtsTransformation}{\glstext}{MtsTransformation}
\newglossaryentry            {MtsTransformation}{
	text    ={\ensuremath{\RawMtsTransformation}},
	name    ={\ensuremath{\RawMtsTransformation} \addIdx[
		name={\ensuremath{\RawMtsTransformation}},
		sort={T xEl}]                                     {MtsTransformation}},
	sort    ={T xEl},%    \LtrMtsTransformation Element
	type    ={symbols},
	symbol  ={},
	user6   ={},
	see     ={},
	description={
		Eine \Transformation.
	}
}

\newVerweis                 {\MtsTransformationTup}{\glstext}{MtsTransformationTup}
\newglossaryentry            {MtsTransformationTup}{
	text    ={\ensuremath{\RawMtsTransformationTup}},
	name    ={\ensuremath{\RawMtsTransformationTup} \addIdx[
		name={\ensuremath{\RawMtsTransformationTup}},
		sort={T Tup}]                                        {MtsTransformationTup}},
	sort    ={T Tup},%    \LtrMtsTransformation Tupel
	type    ={symbols},
	symbol  ={},
	user6   ={},
	see     ={},
	description={
		Ein \Tupel\ von \Transformationen.
	}
}

% ==============================================================================
% \Ojk* - Ausgabe als Text-Symbol und Eintrag und Link ins Symbolverzeichnis
% Symbole für die Konstruktiuon von logischen Formeln ==========================

\newcommand*             {\LtrOjkABC}              {A}
\newVerweis                 {\OjkABC}{\glstext}{OjkABC}
\newglossaryentry            {OjkABC}{
	text    ={\ensuremath{\RawOjkABC}},
	name    ={\ensuremath{\RawOjkABC} \addIdx[
		name={\ensuremath{\RawOjkABC}},
		sort={A Men}]                          {OjkABC}},
	sort    ={A Men},%    \LtrOjkABC Menge
	type    ={symbols},
	symbol  ={},
	user6   ={},
	see     ={},
	description={
		Das \Alphabet\ der \aussagenlogischenSprache.
	}
}

\newVerweis                 {\OjkABCx}{\glstext}{OjkABCx}
\newglossaryentry            {OjkABCx}{
	text    ={\ensuremath{\RawOjkABC_x}},
	name    ={\ensuremath{\RawOjkABC_x} \addIdx[
		name={\ensuremath{\RawOjkABC_x}},
		sort={A MeT x}]                         {OjkABCx}},
	sort    ={A MeT x},%  \LtrOjkABC Teilmenge x
	type    ={symbols},
	symbol  ={},
	user6   ={},
	see     ={},
	description={
		Eine \Teilmenge\ des \Alphabets\ \OjkABC\ der \aussagenlogischenSprache.
	}
}

\newcommand*             {\LtrOjkFor}              {L}% Sprache; language; \LtrMtsSprache
\newVerweis                 {\OjkFor}{\glstext}{OjkFor}
\newglossaryentry            {OjkFor}{
	text    ={\ensuremath{\RawOjkFor}},
	name    ={\ensuremath{\RawOjkFor} \addIdx[
		name={\ensuremath{\RawOjkFor}},
		sort={L Men A}]                        {OjkFor}},
	sort    ={L Men A},%  \LtrOjkFor Menge \LtrMtsIdxLogisch
	type    ={symbols},
	symbol  ={},
	user6   ={},
	see     ={},
	description={
		Eine \Formelmenge: Die \Menge\ der \aussagenlogischenFormeln\ mit \Klammerung.
	}
}

\newVerweis                 {\OjkForp}{\glstext}{OjkForp}
\newglossaryentry            {OjkForp}{
	text    ={\ensuremath{\RawOjkForp}},
	name    ={\ensuremath{\RawOjkForp} \addIdx[
		name={\ensuremath{\RawOjkForp}},
		sort={L Men Ap}]                        {OjkForp}},
	sort    ={L Men Ap},% \LtrOjkFor Menge \LtrMtsIdxLogisch\LtrMtsIdxPolnisch
	type    ={symbols},
	symbol  ={},
	user6   ={},
	see     ={},
	description={
		Eine \Formelmenge: Die \Menge\ der \aussagenlogischenFormeln\ in \PolnischerNotation.
	}
}

\newVerweis                 {\OjkForx}{\glstext}{OjkForx}
\newglossaryentry            {OjkForx}{
	text    ={\ensuremath{\RawOjkFor_x}},
	name    ={\ensuremath{\RawOjkFor_x} \addIdx[
		name={\ensuremath{\RawOjkFor_x}},
		sort={L Men A x}]                       {OjkForx}},
	sort    ={L Men A x},%\LtrOjkFor \LtrMtsIdxLogisch x
	type    ={symbols},
	symbol  ={},
	user6   ={},
	see     ={},
	description={
		Eine \Formelmenge: Eine \Teilmenge\ der \Menge\ \OjkFor\ der \aussagenlogischenFormeln\ mit \Klammerung.
	}
}

\newVerweis                 {\OjkForpx}{\glstext}{OjkForpx}
\newglossaryentry            {OjkForpx}{
	text    ={\ensuremath{\RawOjkForp_x}},
	name    ={\ensuremath{\RawOjkForp_x} \addIdx[
		name={\ensuremath{\RawOjkForp_x}},
		sort={L Men Ap x}]                       {OjkForpx}},
	sort    ={L Men Ap x},%\LtrOjkFor Menge \LtrMtsIdxLogisch\LtrMtsIdxPolnisch x
	type    ={symbols},
	symbol  ={},
	user6   ={},
	see     ={},
	description={
		Eine \Formelmenge: Eine \Teilmenge\ der \Menge\ \OjkForp\ der \aussagenlogischenFormel\ in \PolnischerNotation.
	}
}

\newcommand*             {\LtrOjkJun}              {J}% Junktoren
\newVerweis                 {\OjkJun}{\glstext}{OjkJun}
\newglossaryentry            {OjkJun}{
	text    ={\ensuremath{\RawOjkJun}},
	name    ={\ensuremath{\RawOjkJun} \addIdx[
		name={\ensuremath{\RawOjkJun}},
		sort={J Men}]                          {OjkJun}},
	sort    ={J Men},%    \LtrOjkJun Menge
	type    ={symbols},
	symbol  ={},
	user6   ={},
	see     ={Junktor},
	description={
		Die \Menge\ der \Junktorsymbole.
	}
}

\newVerweis                 {\OjkJunx}{\glstext}{OjkJunx}
\newglossaryentry            {OjkJunx}{
	text    ={\ensuremath{\RawOjkJun_x}},
	name    ={\ensuremath{\RawOjkJun_x} \addIdx[
		name={\ensuremath{\RawOjkJun_x}},
		sort={J Men x}]                         {OjkJunx}},
	sort    ={J Men x},%  \LtrOjkJun Menge x
	type    ={symbols},
	symbol  ={},
	user6   ={},
	see     ={},
	description={
		Eine \Teilmenge\ der \Menge\ \OjkJun\ der \Junktorsymbole.
	}
}

\newVerweis                 {\OjkBin}{\glstext}{OjkBin}
\newglossaryentry            {OjkBin}{
	text    ={\ensuremath{\RawOjkBin}},
	name    ={\ensuremath{\RawOjkBin} \addIdx[
		name={\ensuremath{\RawOjkBin}},
		sort={J Men b}]                        {OjkBin}},
	sort    ={J Men b},%  \LtrOjkJun Menge \StrMtsIdxBin
	type    ={symbols},
	symbol  ={},
	user6   ={},
	see     ={},
	description={
		Die \Menge\ der \binaerenJunktoren.
	}
}

\newVerweis                 {\OjkCon}{\glstext}{OjkCon}
\newglossaryentry            {OjkCon}{
	text    ={\ensuremath{\RawOjkCon}},
	name    ={\ensuremath{\RawOjkCon} \addIdx[
		name={\ensuremath{\RawOjkCon}},
		sort={J Men c}]                        {OjkCon}},
	sort    ={J Men c},%  \LtrOjkJun Menge \StrMtsIdxCon
	type    ={symbols},
	symbol  ={},
	user6   ={},
	see     ={},
	description={
		Die \Menge\ der \aussagenlogischenKonstanten.
	}
}

\newVerweis                 {\OjkUna}{\glstext}{OjkUna}
\newglossaryentry            {OjkUna}{
	text    ={\ensuremath{\RawOjkUna}},
	name    ={\ensuremath{\RawOjkUna} \addIdx[
		name={\ensuremath{\RawOjkUna}},
		sort={J Men u}]                        {OjkUna}},
	sort    ={J Men u},%  \LtrOjkJun Menge \StrMtsIdxUna
	type    ={symbols},
	symbol  ={},
	user6   ={},
	see     ={},
	description={
		Die \Menge\ der \unaerenJunktoren.
	}
}

\newcommand*             {\LtrOjkvar}              {q}% Name aussagenlogische Variable
\newVerweis                 {\Ojkvar}{\glstext}{Ojkvar}
\newglossaryentry            {Ojkvar}{
	text    ={\ensuremath{\RawOjkvar}},
	name    ={\ensuremath{\RawOjkvar} \addIdx[
		name={\ensuremath{\RawOjkvar}},
		sort={q xEl}]                          {Ojkvar}},
	sort    ={q xEl},%    \LtrOjkvar Menge
	type    ={symbols},
	symbol  ={},
	user6   ={},
	see     ={Aussagenlogik},
	description={
		Eine \aussagenlogischeVariable.
	}
}

\newcommand*             {\LtrOjkVar}              {Q}% Menge aussagenlogische Variable
\newVerweis                 {\OjkVar}{\glstext}{OjkVar}
\newglossaryentry            {OjkVar}{
	text    ={\ensuremath{\RawOjkVar}},
	name    ={\ensuremath{\RawOjkVar} \addIdx[
		name={\ensuremath{\RawOjkVar}},
		sort={Q Men}]                          {OjkVar}},
	sort    ={Q Men},%    \LtrOjkVar Menge
	type    ={symbols},
	symbol  ={},
	user6   ={},
	see     ={Aussagenlogik,Menge},
	description={
		$\OjkVar \RawMtsDefEq \RawMengeDef{\Ojkvar_i}{i \in \RawMtsINo}$,
		die \Menge\ der \aussagenlogischenVariablen.
	}
}

% ==============================================================================
% \Mts* - Ausgabe als Text-Symbol und Eintrag und Link ins Symbolverzeichnis
% Wahrheitswerte ===============================================================

\newcommand*             {\StrMtsFalse}            {false}
\newVerweis                 {\MtsFalse}{\glstext}{MtsFalse}
\newglossaryentry            {MtsFalse}{
	text    ={\ensuremath{\RawMtsFalse}},
	name    ={\ensuremath{\RawMtsFalse} \addIdx[
		name={\ensuremath{\RawMtsFalse}},
		sort={false}]                            {MtsFalse}},
	sort    ={false},%    \StrMtsFalse
	type    ={symbols},
	symbol  ={},
	user6   ={},
	see     ={MtsTrue,OjkFalse},
	description={\todoOk%
		Der \metasprachlicheWahrheitswert\ \TxtFalse\ als \Symbol.
	}
}

\newcommand*             {\StrMtsTrue}             {true}
\newVerweis                 {\MtsTrue}{\glstext}{MtsTrue}
\newglossaryentry            {MtsTrue}{
	text    ={\ensuremath{\RawMtsTrue}},
	name    ={\ensuremath{\RawMtsTrue} \addIdx[
		name={\ensuremath{\RawMtsTrue}},
		sort={true}]                            {MtsTrue}},
	sort    ={true},%     \StrMtsTrue
	type    ={symbols},
	symbol  ={},
	user6   ={},
	see     ={MtsFalse,OjkTrue},
	description={\todoOk%
		Der \metasprachlicheWahrheitswert\ \TxtTrue\ als \Symbol.
	}
}

% ==============================================================================
% \Txt* - Ausgabe als formatierter Text und Eintrag und Link ins Glossar
% Wahrheitswerte ===============================================================

\newcommand*             {\StrTxtFalse}            {falsch}
\newVerweis                 {\TxtFalse}{\glstext}{TxtFalse}
\newglossaryentry            {TxtFalse}{
	name       =         {\RawTxtFalse \addIdx[
		name   =         {\RawTxtFalse},
		sort   ={falsch}]    {TxtFalse}},
	sort       ={falsch},%\StrTxtFalse
	text       =         {\RawTxtFalse},
	see        ={TxtTrue,MtsFalse,OjkFalse},
	description={\todoOk%
		Ein \metasprachlicherWahrheitswert\ in Textform.
	}
}

\newcommand*               {\StrTxtTrue}           {wahr}
\newVerweis                   {\TxtTrue}{\glstext}{TxtTrue}
\newglossaryentry              {TxtTrue}{
	name       =           {\RawTxtTrue \addIdx[
		name   =           {\RawTxtTrue},
		sort   ={wahr}]        {TxtTrue}},
	sort       ={wahr},%  \StrTxtTrue
	text       =           {\RawTxtTrue},
	see        ={TxtFalse,MtsTrue,OjkTrue},
	description={\todoOk%
		Ein \metasprachlicherWahrheitswert\ in Textform.
	}
}
