%%############################################################################%%
%%                                                                            %%
%% Datei:  ASBA-Vorspann-Glossary.tex                                         %%
%% Inhalt: Vorspann Glossareinträge für ASBA                                  %%
%%                                                                            %%
%% Copyright (C) 2017  Winfried Teschers                                      %%
%%                                                                            %%
%% This program is free software: you can redistribute it and/or modify       %%
%% it under the terms of the GNU Affero General Public License as published   %%
%% by the Free Software Foundation, either version 3 of the License, or       %%
%% (at your option) any later version.                                        %%
%%                                                                            %%
%% This program is distributed in the hope that it will be useful,            %%
%% but WITHOUT ANY WARRANTY; without even the implied warranty of             %%
%% MERCHANTABILITY or FITNESS FOR A PARTICULAR PURPOSE.  See the              %%
%% GNU Affero General Public License for more details.                        %%
%%                                                                            %%
%% You should have received a copy of the GNU Affero General Public License   %%
%% along with this program.  If not, see <http://www.gnu.org/licenses/>.      %%
%%                                                                            %%
%% Dr. Winfried Teschers                                                      %%
%% Anton-Günther-Straße 26c                                                   %%
%% 91083 Baiersdorf                                                           %%
%% Germany                                                                    %%
%%                                                                            %%
%% e-mail: winfried.teschers@t-online.de                                      %%
%%                                                                            %%
%%############################################################################%%

% !TeX root = ASBA.tex
% !TeX encoding = UTF-8
% !TeX spellcheck = de_DE

% Elemente, die keine Glossareinträge sind, werden in "ASBA-Vorspann.tex" und "ASBA-Mathematik-Vorspann.tex" definiert.

\newcommand*{\glos}[1]{\emph{\textsc{#1}}}% Schriftart für Verweise ins Glossar

% Symbole für Mengen -----------------------------------------------------------

\newcommand*      {\symIN}[1][]{\glsSym[#1]{IN}{\IN}}
\newglossaryentry     {IN}{
	name={\ensuremath{\IN}},
	description     ={
		Die Menge der natürlichen Zahlen ohne 0.
		\\-- Zur Definition \vrefseesub{sub:Bezeichnungen}
	}
}
\newcommand*      {\symINo}[1][]{\glsSym[#1]{INo}{\INo}}
\newglossaryentry     {INo}{
	name={\ensuremath{\INo}},
	description     ={
		Die Menge der natürlichen Zahlen einschließlich 0.
		\\-- Zur Definition \vrefseesub{sub:Bezeichnungen}.
	}
}
%TODO === Verweise auf Definition prüfen und hinzufügen
\newcommand*      {\symalABC}[1][]{\glsSym[#1]{alABC}{\alABC}}
\newglossaryentry     {alABC}{
	name={\ensuremath{\alABC}},
	description     ={
		Das Alphabet der aussagenlogischen \glos{Sprache}.
		\\-- Zur Definition \vrefseesubsub{subsub:Formeln}.
	}
}
\newcommand*      {\symalABCx}[1][]{\glsSym[#1]{alABCx}{\alABC_x}}
\newglossaryentry     {alABCx}{
	name={\ensuremath{\alABC_x}},
	description     ={
		Eine Teilmenge des Alphabets $\alABC$ der aussagenlogischen \glos{Sprache}.
		\\-- Zur Definition \vrefseesubsub{subsub:Formeln}.
	}
}
\newcommand*      {\symalBin}[1][]{\glsSym[#1]{alBin}{\alBin}}
\newglossaryentry     {alBin}{
	name={\ensuremath{\alBin}},
	description     ={
		Die Menge der binären \glos{Junktoren}.
		\\-- Zur Definition \vrefseesubsub{subsub:Bausteine}.
	}
}
\newcommand*      {\symalCon}[1][]{\glsSym[#1]{alCon}{\alCon}}
\newglossaryentry     {alCon}{
	name={\ensuremath{\alCon}},
	description     ={
		Die Menge der aussagenlogischen Konstanten.
		\\-- Zur Definition \vrefseesubsub{subsub:Bausteine}.
	}
}
\newcommand*      {\symalFor}[1][]{\glsSym[#1]{alFor}{\alFor}}
\newglossaryentry     {alFor}{
	name={\ensuremath{\alFor}},
	description     ={
		Die Menge der aussagenlogischen \glos{Formeln} mit Klammerung.
	}
}
\newcommand*      {\symalForp}[1][]{\glsSym[#1]{alForp}{\alForp}}
\newglossaryentry     {alForp}{
	name={\ensuremath{\alForp}},
	description     ={
		Die Menge der aussagenlogischen \glos{Formeln} in polnischer Notation.
	}
}
\newcommand*      {\symalForx}[1][]{\glsSym[#1]{alForx}{\alFor_x}}
\newglossaryentry     {alForx}{
	name={\ensuremath{\alFor_x}},
	plural          ={\alFor_x},%%% im Mathematikmodus - überflüssig
	description     ={
		Eine Teilmenge der Menge $\alFor$ der aussagenlogischen \glos{Formeln} mit Klammerung.
	}
}
\newcommand*      {\symalForpx}[1][]{\glsSym[#1]{alForpx}{\alForp_x}}
\newglossaryentry     {alForpx}{
	name={\ensuremath{\alForp_x}},
	plural          ={\alForp_x},%%% im Mathematikmodus - überflüssig
	description     ={
		Eine Teilmenge der Menge $\alForp$ der aussagenlogischen \glos{Formeln} in polnischer Notation.
	}
}
\newcommand*      {\symalJun}[1][]{\glsSym[#1]{alJun}{\alJun}}
\newglossaryentry     {alJun}{
	name={\ensuremath{\alJun}},
	plural          ={\alJun},%%% im Mathematikmodus - überflüssig
	description     ={
		Die Menge der \glos{Junktorsymbole}.
		\\-- Zur Definition \vrefseesubsub{subsub:Bausteine}.
	}
}
\newcommand*      {\symalJunx}[1][]{\glsSym[#1]{alJunx}{\alJun_x}}
\newglossaryentry     {alJunx}{
	name={\ensuremath{\alJun_x}},
	plural          ={\alJun_x},%%% im Mathematikmodus - überflüssig
	description     ={
		Eine Teilmenge der Menge $\alJun$ der \glos{Junktorsymbole}.
		\\-- Zur Definition \vrefseesubsub{subsub:Bausteine}.
	}
}
\newcommand*      {\symalUna}[1][]{\glsSym[#1]{alUna}{\alUna}}
\newglossaryentry     {alUna}{
	name={\ensuremath{\alUna}},
	plural          ={\alUna},%%% im Mathematikmodus - überflüssig
	description     ={
		Die Menge der unären \glos{Junktoren}.
		\\-- Zur Definition \vrefseesubsub{subsub:Bausteine}.
	}
}
\newcommand*      {\symalVar}[1][]{\glsSym[#1]{alVar}{\alVar}}
\newglossaryentry     {alVar}{
	name={\ensuremath{\alVar}},
	plural          ={\alVar},%%% im Mathematikmodus - überflüssig
	description     ={
		Die Menge der aussagenlogischen Variablen.
		\\-- Zur Definition \vrefseesubsub{subsub:Bausteine}.
	}
}
\newcommand*      {\symMengeMo}[1][]{\glsSym[#1]{MengeMo}{M^0}}
\newglossaryentry     {MengeMo}{
	name={\ensuremath{M^0}},
	plural          ={M^0},%%% im Mathematikmodus - überflüssig
	description     ={
		$\{()\}$ , wobei $()$ das 0-Tupel ist.
		\\-- Zur Definition \vrefseesub{sub:Bezeichnungen}.
	}
}
\newcommand*      {\symMengeMn}[1][]{\glsSym[#1]{MengeMn}{M^n}}
\newglossaryentry     {MengeMn}{
	name={\ensuremath{M^n}},
	plural          ={M^n},%%% im Mathematikmodus - überflüssig
	description     ={
		Das karthesische Produkt $M \times M \times \dots \times M$ aus $n$ Mengen $M$ mit $n \in \INo$.
		\\-- Zur Definition \vrefseesub{sub:Bezeichnungen}.
	}
}

% Symbole für Beispieloperationen und -relationen ------------------------------

\newcommand*      {\symlrelbsp}[1][]{\glsSym[#1]{lrelbsp}{\lrelbsp}}
%%%\newglossaryentry     {lrelbsp}{
%%%	name={\ensuremath{\lrelbsp}},
%%%	plural          ={\lrelbsp},%%% im Mathematikmodus - überflüssig
%%%	description     ={
%%%		Beispielsymbol für eine binäre \glos{Relation} mit \emph{Umkehrrelation} $\rrelbsp$.
%%%		\\-- Zur Definition \vrefseesub{sub:Beispielsymbole}.
%%%	}
%%%}
%%%\newcommand*      {\symlreleqbsp}[1][]{\glsSym[#1]{lreleqbsp}{\lreleqbsp}}
%%%\newglossaryentry     {lreleqbsp}{
%%%	name={\ensuremath{\lreleqbsp}},
%%%	plural          ={\lreleqbsp},%%% im Mathematikmodus - überflüssig
%%%	description     ={
%%%		Beispielsymbol für eine binäre \glos{Relation} mit \glos{Gleichheit} und \emph{Umkehrrelation} $\rreleqbsp$.
%%%		\\-- Zur Definition \vrefseesub{sub:Beispielsymbole}.
%%%	}
%%%}
%%%\newcommand*      {\symlreleqbsp}[1][]{\glsSym[#1]{lreleqbsp}{\lreleqbsp}}
%%%\newglossaryentry     {lrelnbsp}{
%%%	name={\ensuremath{\lrelnbsp}},
%%%	plural          ={\lrelnbsp},%%% im Mathematikmodus - überflüssig
%%%	description     ={
%%%		Beispielsymbol für eine binäre negierte \glos{Relation} mit \emph{Umkehrrelation} $\rrelnbsp$.
%%%		\\-- Zur Definition \vrefseesub{sub:Beispielsymbole}.
%%%	}
%%%}
%%%\newcommand*      {\symlrelnebsp}[1][]{\glsSym[#1]{lrelnebsp}{\lrelnebsp}}
%%%\newglossaryentry     {lrelnebsp}{
%%%	name={\ensuremath{\lrelnebsp}},
%%%	plural          ={\lrelnebsp},%%% im Mathematikmodus - überflüssig
%%%	description     ={
%%%		Beispielsymbol für eine binäre negierte \glos{Relation} mit \emph{Ungleichheit} und \emph{Umkehrrelation} $\rrelnebsp$.
%%%		\\-- Zur Definition \vrefseesub{sub:Beispielsymbole}.
%%%	}
%%%}
\newcommand*      {\symopbsp}[1][]{\glsSym[#1]{opbsp}{\opbsp}}
\newglossaryentry     {opbsp}{
	name={\ensuremath{\opbsp}},
	plural          ={\opbsp},%%% im Mathematikmodus - überflüssig
	description     ={
		Beispielsymbol für eine binäre \glos{Operation}.
		\\-- Zur Definition \vrefseesub{sub:Beispielsymbole}.
	}
}
\newcommand*      {\symopubsp}[1][]{\glsSym[#1]{opubsp}{\opubsp}}
\newglossaryentry     {opubsp}{
	name={\ensuremath{\opubsp}},
	plural          ={\opubsp},%%% im Mathematikmodus - überflüssig
	description     ={
		Beispielsymbol für eine unäre \glos{Operation}.
		\\-- Zur Definition \vrefseesub{sub:Beispielsymbole}.
	}
}
\newcommand*      {\symrelbsp}[1][]{\glsSym[#1]{relbsp}{\relbsp}}
\newglossaryentry     {relbsp}{
	name={\ensuremath{\relbsp}},
	plural          ={\relbsp},%%% im Mathematikmodus - überflüssig
	description     ={
		Beispielsymbol für eine binäre \glos{Relation} %%% mit \emph{Umkehrrelation} $\relbackbsp$
		\\-- Zur Definition \vrefseesub{sub:Beispielsymbole}.
	}
}
\newcommand*      {\symreleqbsp}[1][]{\glsSym[#1]{releqbsp}{\releqbsp}}
\newglossaryentry     {releqbsp}{
	name={\ensuremath{\releqbsp}},
	plural          ={\releqbsp},%%% im Mathematikmodus - überflüssig
	description     ={
		Beispielsymbol für eine binäre \glos{Relation} mit \glos{Gleichheit} %%%  und \emph{Umkehrrelation} $\relbackeqbsp$
		\\-- Zur Definition \vrefseesub{sub:Beispielsymbole}.
	}
}
\newcommand*      {\symrelnbsp}[1][]{\glsSym[#1]{relnbsp}{\relnbsp}}
\newglossaryentry     {relnbsp}{
	name={\ensuremath{\relnbsp}},
	plural          ={\relnbsp},%%% im Mathematikmodus - überflüssig
	description     ={
		Verneinung von $\relbsp$.
		\\-- Zur Definition \vrefseesub{sub:Beispielsymbole}.
	}
}
\newcommand*      {\symrelnebsp}[1][]{\glsSym[#1]{relnebsp}{\relnebsp}}
\newglossaryentry     {relnebsp}{
	name={\ensuremath{\relnebsp}},
	plural          ={\relnebsp},%%% im Mathematikmodus - überflüssig
	description     ={
		Verneinung von $\releqbsp$.
		\\-- Zur Definition \vrefseesub{sub:Beispielsymbole}.
	}
}
\newcommand*      {\symrelbackbsp}[1][]{\glsSym[#1]{relbackbsp}{\relbackbsp}}
\newglossaryentry     {relbackbsp}{
	name={\ensuremath{\relbackbsp}},
	plural          ={\relbackbsp},%%% im Mathematikmodus - überflüssig
	description     ={
		Beispielsymbol für eine binäre \glos{Relation} %%% mit \emph{Umkehrrelation} $\relbackbackbsp$
		\\-- Zur Definition \vrefseesub{sub:Beispielsymbole}.
	}
}
\newcommand*      {\symrelbackeqbsp}[1][]{\glsSym[#1]{relbackeqbsp}{\relbackeqbsp}}
\newglossaryentry     {relbackeqbsp}{
	name={\ensuremath{\relbackeqbsp}},
	plural          ={\relbackeqbsp},%%% im Mathematikmodus - überflüssig
	description     ={
		Beispielsymbol für eine binäre \glos{Relation} mit \glos{Gleichheit} %%%  und \emph{Umkehrrelation} $\releqbsp$
		\\-- Zur Definition \vrefseesub{sub:Beispielsymbole}.
	}
}
%%%\newcommand*      {\symrrelbsp}[1][]{\glsSym[#1]{rrelbsp}{\rrelbsp}}
%%%\newglossaryentry     {rrelbsp}{
%%%	name={\ensuremath{\rrelbsp}},
%%%	plural          ={\rrelbsp},%%% im Mathematikmodus - überflüssig
%%%	description     ={
%%%		Beispielsymbol für eine binäre \glos{Relation} mit \emph{Umkehrrelation} $\lrelbsp$.
%%%		\\-- Zur Definition \vrefseesub{sub:Beispielsymbole}.
%%%	}
%%%}
%%%\newcommand*      {\symrreleqbsp}[1][]{\glsSym[#1]{rreleqbsp}{\rreleqbsp}}
%%%\newglossaryentry     {rreleqbsp}{
%%%	name={\ensuremath{\rreleqbsp}},
%%%	plural          ={\rreleqbsp},%%% im Mathematikmodus - überflüssig
%%%	description     ={
%%%		Beispielsymbol für eine binäre \glos{Relation} mit \glos{Gleichheit} und \emph{Umkehrrelation} $\lreleqbsp$.
%%%		\\-- Zur Definition \vrefseesub{sub:Beispielsymbole}.
%%%	}
%%%}
%%%\newcommand*      {\symrrelnbsp}[1][]{\glsSym[#1]{rrelnbsp}{\rrelnbsp}}
%%%\newglossaryentry     {rrelnbsp}{
%%%	name={\ensuremath{\rrelnbsp}},
%%%	plural          ={\rrelnbsp},%%% im Mathematikmodus - überflüssig
%%%	description     ={
%%%		Beispielsymbol für eine binäre negierte \glos{Relation} mit \emph{Umkehrrelation} $\lrelnbsp$.
%%%		\\-- Zur Definition \vrefseesub{sub:Beispielsymbole}.
%%%	}
%%%}
%%%\newcommand*      {\symrrelnebsp}[1][]{\glsSym[#1]{rrelnebsp}{\rrelnebsp}}
%%%\newglossaryentry     {rrelnebsp}{
%%%	name={\ensuremath{\rrelnebsp}},
%%%	plural          ={\rrelnebsp},%%% im Mathematikmodus - überflüssig
%%%	description     ={
%%%		Beispielsymbol für eine binäre negierte \glos{Relation} mit \glos{Ungleichheit} und \emph{Umkehrrelation} $\lrelnebsp$.
%%%		\\-- Zur Definition \vrefseesub{sub:Beispielsymbole}.
%%%	}
%%%}

% Meta-Symbole -----------------------------------------------------------------

\newcommand*      {\symdefeq}[1][]{\glsSym[#1]{defeq}{\defeq}}
%TODO Makro \defeq hier nicht bekannt?
\newglossaryentry     {defeq}{
	name={\ensuremath{:=}},
	plural          ={:=},%%% im Mathematikmodus - überflüssig
	description     ={
		\glos{Definition}:~ \textdots\ \emph{definitionsgemäß gleich} \textdots
	}
}
\newcommand*      {\symderiveR}[1][]{\glsSym[#1]{derive}{\derive_R}}
\newcommand*      {\symderive}[1][]{\glsSym[#1]{derive}{\derive}}
\newglossaryentry     {derive}{
	name={\ensuremath{\derive}},
	plural          ={\derive},%%% im Mathematikmodus - überflüssig
	description     ={
		\glos{Ableitungsrelation}:~ \textdots\ \emph{ableitbar} (beweisbar) \textdots
		\\-- Siehe \glos{ableitbar}.
	}
}
\newcommand*      {\symeq}[1][]{\glsSym[#1]{eq}{\eq}}
\newglossaryentry     {eq}{
	name={\ensuremath{\eq}},
	plural          ={\eq},%%% im Mathematikmodus - überflüssig
	description     ={
		Eine \glos{Metarelation}:~ \textdots\ \emph{gleich} (ist dasselbe wie; ist identisch zu) \textdots
		\\-- Siehe \glos{Gleichheit}.
		\\-- Zur Definition \vrefseesubsub{subsub:Vergleiche} und \vrefseesub{sub:ausJunktorDef}.
	}
}
\newcommand*      {\symequiv}[1][]{\glsSym[#1]{equiv}{\equiv}}
\newglossaryentry     {equiv}{
	name={\ensuremath{\equiv}},
	plural          ={\equiv},%%% im Mathematikmodus - überflüssig
	description     ={
		Eine \glos{Metarelation}:~ \textdots\ \emph{äquivalent zu} (ist das gleiche wie; ist so wie) \textdots
		\\-- Siehe \glos{Äquivalenz}.
		\\-- Zur Definition \vrefseesubsub{subsub:Vergleiche} und \vrefseesub{sub:ausJunktorDef}.
	}
}
\newcommand*      {\symmetaand}[1][]{\glsSym[#1]{metaand}{\metaand}}
\newglossaryentry     {metaand}{
	name={\ensuremath{\metaand}},
	plural          ={\metaand},%%% im Mathematikmodus - überflüssig
	description     ={
		Eine \glos{Metaoperation}:~ \textdots\ \emph{und} \textdots
		\\-- Zur Definition \vrefseesub{sub:AussagenUndMetaoperationen}.
	}
}
\newcommand*      {\symmetadefeq}[1][]{\glsSym[#1]{metadefeq}{\metadefeq}}
\newglossaryentry     {metadefeq}{
	name={\ensuremath{\metadefeq}},
	plural          ={\metadefeq},%%% im Mathematikmodus - überflüssig
	description     ={
		Eine \glos{Metadefinition}:~ \textdots\ \emph{definitionsgemäß genau dann wenn} \textdots
	}
}
\newcommand*      {\symmetaequiv}[1][]{\glsSym[#1]{metaequiv}{\metaequiv}}
\newglossaryentry     {metaequiv}{
	name={\ensuremath{\metaequiv}},
	plural          ={\metaequiv},%%% im Mathematikmodus - überflüssig
	description     ={
		Eine \glos{Metarelation}:~ \textdots\ \emph{genau dann wenn} \textdots
		\\-- Zur Definition \vrefseesub{sub:AussagenUndMetaoperationen}.
	}
}
\newcommand*      {\symmetaimp}[1][]{\glsSym[#1]{metaimp}{\metaimp}}
\newglossaryentry     {metaimp}{
	name={\ensuremath{\metaimp}},
	plural          ={\metaimp},%%% im Mathematikmodus - überflüssig
	description     ={
		Eine \glos{Metarelation}:~ \textdots\ \emph{dann auch} \textdots , die Umkehrrelation zu $\metarep$
		\\-- Zur Definition \vrefseesub{sub:AussagenUndMetaoperationen}.
	}
}
\newcommand*      {\symmetanot}[1][]{\glsSym[#1]{metanot}{\metanot}}
\newglossaryentry     {metanot}{
	name={\ensuremath{\metanot}},
	plural          ={\metanot},%%% im Mathematikmodus - überflüssig
	description     ={
		Eine unäre \glos{Metaoperation}:~ \textdots emph{gilt nicht}
		\\-- Zur Definition \vrefseesub{sub:AussagenUndMetaoperationen}.
	}
}
\newcommand*      {\symmetaor}[1][]{\glsSym[#1]{metaor}{\metaor}}
\newglossaryentry     {metaor}{
	name={\ensuremath{\metaor}},
	plural          ={\metaor},%%% im Mathematikmodus - überflüssig
	description     ={
		Eine \glos{Metaoperation}:~ \textdots\ \emph{oder} \textdots
		\\-- Zur Definition \vrefseesub{sub:AussagenUndMetaoperationen}.
	}
}
\newcommand*      {\symmetarep}[1][]{\glsSym[#1]{metarep}{\metarep}}
\newglossaryentry     {metarep}{
	name={\ensuremath{\metarep}},
	plural          ={\metarep},%%% im Mathematikmodus - überflüssig
	description     ={
		Eine \glos{Metarelation}:~ \textdots\ \emph{sofern} \textdots , die Umkehrrelation zu $\metaimp$
		\\-- Zur Definition \vrefseesub{sub:AussagenUndMetaoperationen}.
	}
}
\newcommand*      {\symne}[1][]{\glsSym[#1]{ne}{\ne}}
\newglossaryentry     {ne}{
	name={\ensuremath{\ne}},
	plural          ={\ne},%%% im Mathematikmodus - überflüssig
	description     ={
		Eine \glos{(Meta-)Operation}:~ \textdots\ \emph{ungleich} (nicht dasselbe wie; nicht identisch zu) \textdots
	}
}
\newcommand*      {\symnequiv}[1][]{\glsSym[#1]{nequiv}{\nequiv}}
\newglossaryentry     {nequiv}{
	name={\ensuremath{\nequiv}},
	plural          ={\nequiv},%%% im Mathematikmodus - überflüssig
	description     ={
		Eine \glos{Metarelation}:~ \textdots\ \emph{nicht äquivalent} (ist nicht das gleiche wie; ist nicht so wie) \textdots
	}
}
\newcommand*      {\symsubst}[1][]{\glsSym[#1]{subst}{\subst}}
\newglossaryentry     {subst}{
	name={\ensuremath{\subst}},
	plural          ={\subst},%%% im Mathematikmodus - überflüssig
	description     ={
		\glos{Substitution}:~ \textdots\ \emph{substituiert durch} \textdots\
		\\-- Zur Definition \vrefseesub{sub:Identitätsregeln}.
	}
}
\newcommand*      {\symswap}[1][]{\glsSym[#1]{swap}{\swap}}
\newglossaryentry     {swap}{
	name={\ensuremath{\swap}},
	plural          ={\swap},%%% im Mathematikmodus - überflüssig
	description     ={
		\glos{Vertauschung}:~ \textdots\ \emph{vertauscht mit} \textdots\
		\\-- Zur Definition \vrefseesub{sub:Identitätsregeln}.
	}
}
\newcommand*      {\symsrand}[1][]{\glsSym[#1]{srand}{\srand}}
\newglossaryentry     {srand}{
	name={\ensuremath{\srand}},
	plural          ={\srand},%%% im Mathematikmodus - überflüssig
	description     ={
		Eine \glos{Metaoperation}:~ \textdots\ \emph{und} \textdots\
		\\-- Wird nur bei den \glos{Schlussregeln} verwendet.
	}
}

% aussagenlogische Operationen (Junktoren) -------------------------------------

\newcommand*      {\symland}[1][]{\glsSym[#1]{land}{\land}}
\newglossaryentry     {land}{
	name={\ensuremath{\land}},
	plural          ={\land},%%% im Mathematikmodus - überflüssig
	description     ={
		Ein binärer \glos{Junktor}:~ \textdots\ \emph{und} \textdots\
		\\-- Zur Definition \vrefseetab{tab:Symbole}.
	}
}
\newcommand*      {\symlequiv}[1][]{\glsSym[#1]{lequiv}{\lequiv}}
\newglossaryentry     {lequiv}{
	name={\ensuremath{\lequiv}},
	plural          ={\lequiv},%%% im Mathematikmodus - überflüssig
	description     ={
		Ein binärer \glos{Junktor}:~ \textdots\ \emph{genau dann wenn} \textdots\
		\\-- Zur Definition \vrefseetab{tab:Symbole}.
	}
}
\newcommand*      {\symlfalse}[1][]{\glsSym[#1]{lfalse}{\lfalse}}
\newglossaryentry     {lfalse}{
	name={\ensuremath{\lfalse}},
	plural          ={\lfalse},%%% im Mathematikmodus - überflüssig
	description     ={
		Ein 0-stelliger \glos{Junktor}, \textdh\ eine aussagenlogische Konstante (\glos{Wahrheitswert}): \emph{$\falsch$}
		\\-- Zur Definition \vrefseetab{tab:Symbole}.
	}
}
\newcommand*      {\symlimp}[1][]{\glsSym[#1]{limp}{\limp}}
\newglossaryentry     {limp}{
	name={\ensuremath{\limp}},
	plural          ={\limp},%%% im Mathematikmodus - überflüssig
	description     ={
		Ein binärer \glos{Junktor}:~ \emph{Aus} \textdots\ \emph{folgt} \textdots\
		\\-- Zur Definition \vrefseetab{tab:Symbole}.
	}
}
\newcommand*      {\symlnand}[1][]{\glsSym[#1]{lnand}{\lnand}}
\newglossaryentry     {lnand}{
	name={\ensuremath{\lnand}},
	plural          ={\lnand},%%% im Mathematikmodus - überflüssig
	description     ={
		Ein binärer \glos{Junktor}:~ \emph{Nicht zugleich}\textdots\ \emph{und} \textdots\
		\\-- Zur Definition \vrefseetab{tab:Symbole}.
	}
}
\newcommand*      {\symlnor}[1][]{\glsSym[#1]{lnor}{\lnor}}
\newglossaryentry     {lnor}{
	name={\ensuremath{\lnor}},
	plural          ={\lnor},%%% im Mathematikmodus - überflüssig
	description     ={
		Ein binärer \glos{Junktor}:~ \emph{weder} \textdots\ \emph{noch} \textdots\
	}
		\\-- Zur Definition \vrefseetab{tab:Symbole}.
}
\newcommand*      {\symlnot}[1][]{\glsSym[#1]{lnot}{\lnot}}
\newglossaryentry     {lnot}{
	name={\ensuremath{\lnot}},
	plural          ={\lnot},%%% im Mathematikmodus - überflüssig
	description     ={
		Ein unärer \glos{Junktor}:~ \emph{Nicht} \textdots\
		\\-- Zur Definition \vrefseetab{tab:Symbole}.
	}
}
\newcommand*      {\symlor}[1][]{\glsSym[#1]{lor}{\lor}}
\newglossaryentry     {lor}{
	name={\ensuremath{\lor}},
	plural          ={\lor},%%% im Mathematikmodus - überflüssig
	description     ={
		Ein binärer \glos{Junktor}:~ \textdots\ \emph{oder} \textdots\
	}
	\\-- Zur Definition \vrefseetab{tab:Symbole}.
}
\newcommand*      {\symlrep}[1][]{\glsSym[#1]{lrep}{\lrep}}
\newglossaryentry     {lrep}{
	name={\ensuremath{\lrep}},
	plural          ={\lrep},%%% im Mathematikmodus - überflüssig
	description     ={
		Ein binärer \glos{Junktor}:~ \textdots\ \emph{folgt aus} \textdots\
		\\-- Zur Definition \vrefseetab{tab:Symbole}.
	}
}
\newcommand*      {\symltrue}[1][]{\glsSym[#1]{ltrue}{\ltrue}}
\newglossaryentry     {ltrue}{
	name={\ensuremath{\ltrue}},
	plural          ={\ltrue},%%% im Mathematikmodus - überflüssig
	description     ={
		Ein 0-stelliger \glos{Junktor}, \textdh\ eine aussagenlogische Konstante (\glos{Wahrheitswert}): \emph{$\wahr$}
		\\-- Zur Definition \vrefseetab{tab:Symbole}.
	}
}
\newcommand*      {\symlxor}[1][]{\glsSym[#1]{lxor}{\lxor}}
\newglossaryentry     {lxor}{
	name={\ensuremath{\lxor}},
	plural          ={\lxor},%%% im Mathematikmodus - überflüssig
	description     ={
		Ein binärer \glos{Junktor}:~ \emph{entweder} \textdots\ \emph{oder} \textdots\
		\\-- Zur Definition \vrefseetab{tab:Symbole}.
	}
}

% sonstige mathematische Symbole -----------------------------------------------

\DeclareMathOperator*{\len}{len}
\newcommand*      {\symlen}[1][]{\glsSym[#1]{len}{\len}}
\newglossaryentry     {len}{
	name={\ensuremath{\len}},
	plural          ={\len},%%% im Mathematikmodus - überflüssig
	description     ={
		$\len(\vec{a})$ ist die Länge, \textdh\ die Anzahl der Elemente eines Vektors.
	}
}
\newcommand*      {\symnsubset}[1][]{\glsSym[#1]{nsubset}{\nsubset}}
\newglossaryentry     {nsubset}{
	name={\ensuremath{\nsubset}},
	plural          ={\nsubset},%%% im Mathematikmodus - überflüssig
	description     ={
		Teilmengenbeziehung:~ \textdots\ \emph{ist keine echte Teilmenge von} \textdots\
	}
}
\newcommand*      {\symnsupset}[1][]{\glsSym[#1]{nsupset}{\nsupset}}
\newglossaryentry     {nsupset}{
	name={\ensuremath{\nsupset}},
	plural          ={\nsupset},%%% im Mathematikmodus - überflüssig
	description     ={
		Teilmengenbeziehung:~ \textdots\ \emph{ist keine echte Obermenge von} \textdots\
	}
}
\newcommand*      {\symPot}[1][]{\glsSym[#1]{Pot}{\Pot}}
\newglossaryentry     {Pot}{
	name={\ensuremath{\Pot}},
	plural          ={\Pot},%%% im Mathematikmodus - überflüssig
	description     ={
		\glos{Potenzmenge}.
	}
}
\newcommand*      {\symPotf}[1][]{\glsSym[#1]{Potf}{\Potf}}
\newglossaryentry     {Potf}{
	name={\ensuremath{\Potf}},
	plural          ={\Potf},%%% im Mathematikmodus - überflüssig
	description     ={
		Menge der endlichen Teilmengen.
	}
}
\newcommand*      {\symRel}[1][]{\glsSym[#1]{Rel}{\Rel}}
\newglossaryentry     {Rel}{
	name={\ensuremath{\Rel}},
	plural          ={\Rel},%%% im Mathematikmodus - überflüssig
	description     ={
		Menge der binären Relationen.
	}
}
\newcommand*      {\symRelf}[1][]{\glsSym[#1]{Relf}{\Relf}}
\newglossaryentry     {Relf}{
	name={\ensuremath{\Relf}},
	plural          ={\Relf},%%% im Mathematikmodus - überflüssig
	description     ={
		Menge der endlichen binären Relationen.
	}
}
\DeclareMathOperator*{\Set}{Set}
\newcommand*      {\symSet}[1][]{\glsSym[#1]{Set}{\Set}}
\newglossaryentry     {Set}{
	name={\ensuremath{\Set}},
	plural          ={\Set},%%% im Mathematikmodus - überflüssig
	description     ={
		$\Set(\vec{a})$ ist die Menge der Elemente eines Vektors.
	}
}
\newcommand*      {\symsubset}[1][]{\glsSym[#1]{subset}{\subset}}
\newglossaryentry     {subset}{
	name={\ensuremath{\subset}},
	plural          ={\subset},%%% im Mathematikmodus - überflüssig
	description     ={
		Teilmengenbeziehung:~ \textdots\ \emph{ist echte Teilmenge von} \textdots\
		; Insbesondere kann keine \Gleichheit\ bestehen.
		In der Literatur wird $\subset$ oft im Sinne von $\subseteq$ verwendet.
		\\-- Zur Definition \vrefseesub{sub:Bezeichnungen}.
	}
}
\newcommand*      {\symsupset}[1][]{\glsSym[#1]{supset}{\supset}}
\newglossaryentry     {supset}{
	name={\ensuremath{\supset}},
	plural          ={\supset},%%% im Mathematikmodus - überflüssig
	description     ={
		Teilmengenbeziehung:~ \textdots\ \emph{ist echte Obermenge von} \textdots\
		; Insbesondere kann keine \Gleichheit\ bestehen.
		In der Literatur wird $\supset$ oft im Sinne von $\supseteq$ verwendet.
	}
}
\newcommand*      {\symsubseteq}[1][]{\glsSym[#1]{subseteq}{\subseteq}}
\newglossaryentry     {subseteq}{
	name={\ensuremath{\subseteq}},
	plural          ={\subseteq},%%% im Mathematikmodus - überflüssig
	description     ={
		Teilmengenbeziehung:~ \textdots\ \emph{ist Teilmenge von} \textdots\
		; Insbesondere kann \Gleichheit\ bestehen.
		\\-- Zur Definition \vrefseesub{sub:Bezeichnungen}.
	}
}
\newcommand*      {\symtupelSet}[1][]{\glsSym[#1]{tupelSet}{\tupelSet}}
\newglossaryentry     {tupelSet}{
	name={\ensuremath{\tupelSet}},
	plural          ={\tupelSet},%%% im Mathematikmodus - überflüssig
	description     ={
		\glos{Tupelmenge}.
	}
}

% Schlussregeln ----------------------------------------------------------------

\newcommand*    {\AR}{\ensuremath{\text{AR}}}% Argument für \tag
%%%\newcommand* {\tagAR}                {(\AR)}% Verweis
\newglossaryentry{AR}{
	name       ={\AR},
	plural     ={\AR},%%% im Mathematikmodus - überflüssig
	description={
		\glos{Anfangsregel}.
	}
}
\newcommand*    {\FS}{\ensuremath{\text{FS}}}% Argument für \tag
%%%\newcommand* {\tagFS}                {(\FS)}% Verweis
\newglossaryentry{FS}{
	name       ={\FS},
	plural     ={\FS},%%% im Mathematikmodus - überflüssig
	description={
		\glos{formaler Satz}.
	}
}
\newcommand*    {\MR}{\ensuremath{\text{MR}}}% Argument für \tag
%%%\newcommand* {\tagMR}                {(\MR)}% Verweis
\newglossaryentry{MR}{
	name       ={\MR},
	plural     ={\MR},%%% im Mathematikmodus - überflüssig
	description={
		\glos{Monotonieregel}.
	}
}
\newcommand*    {\SR}{\ensuremath{\text{SR}}}% Argument für \tag
%%%\newcommand* {\tagSR}                {(\SR)}% Verweis
\newglossaryentry{SR}{
	name       ={\SR},
	plural     ={\SR},%%% im Mathematikmodus - überflüssig
	description={
		\glos{Schnittregel} (Modus ponens)
	}
}
\newcommand*    {\TR}{\ensuremath{\text{TR}}}% Argument für \tag
%%%\newcommand* {\tagTR}                {(\TR)}% Verweis
\newglossaryentry{TR}{
	name       ={\TR},
	plural     ={\TR},%%% im Mathematikmodus - überflüssig
	description={
		\glos{Abtrennungsregel}.
	}
}
\newcommand*    {\eqB}{\ensuremath{\eq\text{B}}}% Argument für \tag
%%%\newcommand* {\tageqB}           {(\eqB)}% Verweis
\newglossaryentry{eqB}{
	name       ={\eqB},
	plural     ={\eqB},%%% im Mathematikmodus - überflüssig
	description={
		Beseitigung von \chrqt{$\eq$}.
	}
}
\newcommand*    {\eqE}{\ensuremath{\eq\text{E}}}% Argument für \tag
%%%\newcommand* {\tageqE}           {(\eqE)}% Verweis
\newglossaryentry{eqE}{
	name       ={\eqE},
	plural     ={\eqE},%%% im Mathematikmodus - überflüssig
	description={
		Einführung von \chrqt{$\eq$}.
	}
}
\newcommand*    {\andB}{\ensuremath{\land\text{B}}}% Argument für \tag
%%%\newcommand* {\tagandB}            {(\andB)}% Verweis
\newglossaryentry{andB}{
	name       ={\andB},
	plural     ={\andB},%%% im Mathematikmodus - überflüssig
	description={
		Beseitigung von \chrqt{$\land$}.
	}
}
\newcommand*    {\andE}{\ensuremath{\land\text{E}}}% Argument für \tag
%%%\newcommand* {\tagandE}            {(\andE)}% Verweis
\newglossaryentry{andE}{
	name       ={\andE},
	plural     ={\andE},%%% im Mathematikmodus - überflüssig
	description={
		Einführung von \chrqt{$\land$}.
	}
}
\newcommand*    {\impB}{\ensuremath{\limp\text{B}}}% Argument für \tag
%%%\newcommand* {\tagimpB}            {(\impB)}% Verweis
\newglossaryentry{impB}{
	name       ={\impB},
	plural     ={\impB},%%% im Mathematikmodus - überflüssig
	description={
		Beseitigung von \chrqt{$\limp$}.
	}
}
\newcommand*    {\impE}{\ensuremath{\limp\text{E}}}% Argument für \tag
%%%\newcommand* {\tagimpE}            {(\impE)}% Verweis
\newglossaryentry{impE}{
	name       ={\impE},
	plural     ={\impE},%%% im Mathematikmodus - überflüssig
	description={
		Einführung von \chrqt{$\limp$}.
	}
}
\newcommand*    {\nota}{\ensuremath{\lnot\text{1}}}% Argument für \tag
%%%\newcommand* {\tagnota}            {(\nota)}% Verweis
\newglossaryentry{nota}{
	name       ={\nota},
	plural     ={\nota},%%% im Mathematikmodus - überflüssig
	description={
		Einführung/Beseitigung von \chrqt{$\lnot$} Teil 1.
	}
}
\newcommand*    {\notb}{\ensuremath{\lnot\text{2}}}% Argument für \tag
%%%\newcommand* {\tagnotb}            {(\notb)}% Verweis
\newglossaryentry{notb}{
	name       ={\notb},
	plural     ={\notb},%%% im Mathematikmodus - überflüssig
	description={
		Einführung/Beseitigung von \chrqt{$\lnot$} Teil 2.
	}
}
\newcommand*    {\notc}{\ensuremath{\lnot\text{3}}}% Argument für \tag
%%%\newcommand* {\tagnotc}            {(\notc)}% Verweis
\newglossaryentry{notc}{
	name       ={\notc},
	plural     ={\notc},%%% im Mathematikmodus - überflüssig
	description={
		Beweistechnik \enquote{Indirekter \glos{Beweis}}.
	}
}
\newcommand*    {\notd}{\ensuremath{\lnot\text{4}}}% Argument für \tag
%%%\newcommand* {\tagnotd}            {(\notd)}% Verweis
\newglossaryentry{notd}{
	name       ={\notd},
	plural     ={\notd},%%% im Mathematikmodus - überflüssig
	description={
		Reductio ad absurdum (Indirekter \glos{Beweis}).
	}
}
%%%\newcommand*    {\orB}{\ensuremath{\lor\text{B}}}% Argument für \tag
%%%\newcommand* {\tagorB}            {(\orB)}% Verweis
%%%\newglossaryentry{orB}{
%%%	name       ={\orB},
%%%	plural     ={\orB},%%% im Mathematikmodus - überflüssig
%%%	description={
%%%		Beseitigung von \chrqt{$\lor$}.
%%%	}
%%%}
%%%\newcommand*    {\orE}{\ensuremath{\lor\text{E}}}% Argument für \tag
%%%\newcommand* {\tagorE}            {(\orE)}% Verweis
%%%\newglossaryentry{orE}{
%%%	name       ={\orE},
%%%	plural     ={\orE},%%% im Mathematikmodus - überflüssig
%%%	description={
%%%		Einführung von \chrqt{$\lor$}.
%%%	}
%%%}

% Fachbegriffe #################################################################

%A === A === A === A === A === A === A === A === A === A === A === A === A === A

\newcommand*{\ASBA}[1][]{\glsIdx  [#1]{ASBA}}
\newacronym{ASBA}{ASBA}{
	Programmsystem, das \textbf{A}xiome, \textbf{S}ätze, \textbf{B}eweise und \textbf{A}uswertungen behandeln kann.
}
\newcommand*    {\ableitbar} [1][]{\glsIdx  [#1]{ableitbar}}
\newcommand*    {\ableitbare}[1][]{\glsIdxPl[#1]{ableitbar}}
\newglossaryentry{ableitbar}{
	name        ={ableitbar},
	plural      ={ableitbare},
	description     ={
		Wenn sich eine \glos{Formel} $\beta$ aus einer anderen \glos{Formel} $\alpha$ mittels \glos{zulässiger Transformationen} ableiten lässt, heißt $\beta$ \glos{ableitbar} aus $\alpha$.
		Sprechweise: \seqqt{$ \alpha \text{ ableitbar } \beta $}.
		Eine oder beide \glos{Formeln} $\alpha$ \textbzw\ $\beta$ dürfen dabei durch \glos{Formelmengen} ersetzt werden.
		\\-- Siehe \glos{Ableitungsrelation} und $\derive$.
		\\-- Synonym: \glos{beweisbar}.
	}
}
\newcommand*    {\Ableitung}  [1][]{\glsIdx  [#1]{Ableitung}}
\newcommand*    {\Ableitungen}[1][]{\glsIdxPl[#1]{Ableitung}}
\newglossaryentry{Ableitung}{
	name        ={Ableitung},
	plural      ={Ableitungen},
	description     ={
		Eine \glos{Aussage} $A \derive B$ \textbzw\ allgemeiner $A \derive_R B$.
		Dies entspricht einem Element $(A,B)$ einer \glos{Ableitungsrelation} $\derive$ \textbzw\ $\derive_R$.
		Die semantische Aussage ist, das die \glos{Formeln} von $B$ aus den \glos{Formeln} von $A$ abgeleitet werden können.
	}
}
%%%\newcommand*    {\Ableitungsmenge} [1][]{\glsIdx  [#1]{Ableitungsmenge}}
%%%\newcommand*    {\Ableitungsmengen}[1][]{\glsIdxPl[#1]{Ableitungsmenge}}
%%%\newglossaryentry{Ableitungsmenge}{
%%%	name        ={Ableitungsmenge},
%%%	plural      ={Ableitungsmengen},
%%%	description     ={
%%%		Eine Menge von \glos{Ableitungen}, letztlich nichts anderes als eine \glos{ABleitungsrelation}.
%%%	}
%%%}
\newcommand*    {\Ableitungsrelation}  [1][]{\glsIdx  [#1]{Ableitungsrelation}}
\newcommand*    {\Ableitungsrelationen}[1][]{\glsIdxPl[#1]{Ableitungsrelation}}
\newglossaryentry{Ableitungsrelation}{
	name        ={Ableitungsrelation},
	plural      ={Ableitungsrelationen},
	description     ={
		Eine binäre \glos{Relation} $\derive$ \textbzw\ allgemeiner $\derive_R$ aus $\formulaSetSet$.
		Siehe auch \glos{Ableitung}
	}
}
\newcommand*    {\Abtrennungsregel}[1][]{\glsIdx  [#1]{Abtrennungsregel}}
\newglossaryentry{Abtrennungsregel}{
	name        ={Abtrennungsregel},
	description     ={
		Eine \glos{Schlussregel} -- siehe~\glos{TR}.
	}
}
\newcommand*    {\Aequivalenz}  [1][]{\glsIdy  [#1]{Aequivalenz}{Äquivalenz}}
\newcommand*    {\Aequivalenzen}[1][]{\glsIdyPl[#1]{Aequivalenz}{Äquivalenz}}
\newglossaryentry{Aequivalenz}{
	name        ={Äquivalenz},
	plural      ={Äquivalenzen},
	description     ={
		Eine \glos{Gleichheitsrelation}:
		Zwei Objekte $A$ und $B$ sind \emph{gleich} (äquivalent), $A \equiv B$, wenn sie in den \glos{interessierenden Eigenschaften} für $\equiv$ übereinstimmen.
		\\-- Zur Definition \vrefseesubsub{subsub:Vergleiche}.
	}
}
\newcommand*    {\Aequivalenzrelation}  [1][]{\glsIdy  [#1]{Aequivalenzrelation}{Äquivalenzrelation}}
\newcommand*    {\Aequivalenzrelationen}[1][]{\glsIdyPl[#1]{Aequivalenzrelation}{Äquivalenzrelation}}
\newglossaryentry{Aequivalenzrelation}{
	name        ={Äquivalenzrelation},
	plural      ={Äquivalenzrelationen},
	description     ={
		Eine binäre \glos{Relation} $\sim$ auf einer Menge $M$ mit folgenden Eigenschaften:
		\begin{description}
			\item [reflexiv] ($a \sim a$)
			\item [transitiv] ($((a \sim b) \metaand (b \sim c)) \metaimp (a \sim c)$)
			\item[symmetrisch] ($(a \sim b) \metaimp (b \sim a)$)
		\end{description}
		jeweils für alle Elemente $a$, $b$ und $c$ aus $M$.
		\\-- \vrefSeesubsub{subsub:Vergleiche}.
	}
}
\newcommand*    {\allgemeingueltigeSchlussregel}  [1][]{\glsIdy  [#1]{allgemeingueltige-Schlussregel}{allgemeingültige Schlussregel}}
\newcommand*    {\allgemeingueltigenSchlussregel} [1][]{\glsIdZBg[#1]{allgemeingueltige-Schlussregel}{allgemeingültigen Schlussregel}}
\newcommand*    {\allgemeingueltigenSchlussregeln}[1][]{\glsIdZBg[#1]{allgemeingueltige-Schlussregel}{allgemeingültigen Schlussregeln}}
\newglossaryentry{allgemeingueltige-Schlussregel}{
	name        ={allgemeingültige Schlussregel},
	plural      ={allgemeingültige Schlussregeln},
	description     ={
		Eine \glos{Schlussregel} die aus den \glos{Basisregeln} und schon bekannten \glos{allgemeingültigen Schlussregeln} abgeleitet werden kann.
		\\-- Zur Definition \vrefseesub{sub:Schlussregeln}.
	}
}
\newcommand*    {\Anfangsregel}[1][]{\glsIdx  [#1]{Anfangsregel}}
\newglossaryentry{Anfangsregel}{
	name        ={Anfangsregel},
	description     ={
		Eine \glos{Schlussregel} um beginnen zu können -- siehe~\glos{AR}.
	}
}
\newcommand*    {\atomar} [1][]{\glsIdx  [#1]{atomar}}
\newcommand*    {\atomare}[1][]{\glsIdxPl[#1]{atomar}}
\newglossaryentry{atomar}{
	name        ={atomar},
	plural      ={atomare},
	description     ={
		Synonym zu \glos{unzerlegbar}, siehe dort; vergleiche auch \glos{zerlegbar}.
	}
}
%%%\newcommand*    {\atomareAussage} [1][]{\glsIdy  [#1]{atomare-Aussage}{atomare Aussage}}
%%%\newcommand*    {\atomareAussagen}[1][]{\glsIdyPl[#1]{atomare-Aussage}{atomare Aussage}}
%%%\newglossaryentry{atomare-Aussage}{
%%%	name        ={atomare Aussage},
%%%	plural      ={atomare Aussagen},
%%%	description     ={
%%%		Eine \glos{unzerlegbare} \glos{Aussage}.
%%%	}
%%%}
%%%\newcommand*    {\atomareFormel} [1][]{\glsIdy  [#1]{atomare-Formel}{atomare Formel}}
%%%\newcommand*    {\atomareFormeln}[1][]{\glsIdyPl[#1]{atomare-Formel}{atomare Formel}}
%%%\newglossaryentry{atomare-Formel}{
%%%	name        ={atomare Formel},
%%%	plural      ={atomare Formeln},
%%%	description     ={
%%%		Eine \glos{unzerlegbare} \glos{Formel}.
%%%	}
%%%}
\newcommand*    {\Ausgabeschema}  [1][]{\glsIdx  [#1]{Ausgabeschema}}
\newcommand*    {\Ausgabeschemata}[1][]{\glsIdxPl[#1]{Ausgabeschema}}
\newglossaryentry{Ausgabeschema}{
	name        ={Ausgabeschema},
	plural      ={Ausgabeschemata},
	description     ={
		Ein Schema, mit dem bestimmte mathematische \glos{Objekte} ausgegeben werden sollen.
	}
}
\newcommand*    {\Aussage} [1][]{\glsIdx  [#1]{Aussage}}
\newcommand*    {\Aussagen}[1][]{\glsIdxPl[#1]{Aussage}}
\newglossaryentry{Aussage}{
	name        ={Aussage},
	plural      ={Aussagen},
	description     ={
		Eine \glos{Aussage} in natürlicher Sprache oder als \glos{Formel}, die einen \glos{Wahrheitswert} liefert.
		\\-- Zur Definition \vrefseesub{sub:AussagenUndMetaoperationen}.
	}
}
\newcommand*    {\Aussagenlogik}[1][]{\glsIdx  [#1]{Aussagenlogik}}
\newglossaryentry{Aussagenlogik}{
	name        ={Aussagenlogik},
	description     ={
		-- Zur Definition \vrefseesec{sec:Aussagenlogik}.
	}
}
\newcommand*    {\Axiom}  [1][]{\glsIdx  [#1]{Axiom}}
\newcommand*    {\Axiome} [1][]{\glsIdxPl[#1]{Axiom}}
\newcommand*    {\Axiomen}[1][]{\glsIdxPl[#1]{Axiom}n}
\newglossaryentry{Axiom}{
	name        ={Axiom},
	plural      ={Axiome},
	description     ={
		Eine \glos{Formel}, die unbewiesen als wahr angesehen wird.
		\\-- Zur Definition \vrefseesub{sub:Schlussregeln} und \vref{sub:ausAxiome}.
	}
}
\newcommand*    {\Axiomensystem} [1][]{\glsIdx  [#1]{Axiomensystem}}
\newcommand*    {\Axiomensysteme}[1][]{\glsIdxPl[#1]{Axiomensystem}}
\newglossaryentry{Axiomensystem}{
	name        ={Axiomensystem},
	plural      ={Axiomensysteme},
	description     ={
		Eine Menge von \glos{Axiomen}.
		\\-- Zur Definition \vrefseesub{sub:Schlussregeln} und \vref{sub:ausAxiome}.
	}
}

%B === B === B === B === B === B === B === B === B === B === B === B === B === B

\newcommand*    {\Basisregel} [1][]{\glsIdx  [#1]{Basisregel}}
\newcommand*    {\Basisregeln}[1][]{\glsIdxPl[#1]{Basisregel}}
\newglossaryentry{Basisregel}{
	name        ={Basisregel},
	plural      ={Basisregeln},
	description     ={
		Eine \glos{Schlussregel}, die nicht mehr auf andere zurückgeführt wird.
		Obwohl das auch auf die \glos{Identitätsregeln} zutrifft, werden diese hier aber nicht dazu gezählt.
		\\-- Zur Definition \vrefseesub{sub:Basisregeln}.
	}
}
\newcommand*    {\beschraenkt}  [1][]{\glsIdy  [#1]{beschraenkt}{beschränkt}}
\newcommand*    {\beschraenkte} [1][]{\glsIdyPl[#1]{beschraenkt}{beschränkt}}
\newcommand*    {\beschraenkten}[1][]{\glsIdyPl[#1]{beschraenkt}{beschränkt}n}
\newglossaryentry{beschraenkt}{
	name        ={beschränkt},
	plural      ={beschränkte},
	description     ={
		Eine \glos{Schlussregel} heißt \emph{beschränkt}, wenn sie nur endlich viele Voraussetzungen und Folgerungen hat.
	}
}
\newcommand*    {\Beweis}  [1][]{\glsIdx  [#1]{Beweis}}
\newcommand*    {\Beweise} [1][]{\glsIdxPl[#1]{Beweis}}
\newcommand*    {\Beweises}[1][]{\glsIdx  [#1]{Beweis}es}
\newcommand*    {\Beweisen}[1][]{\glsIdxPl[#1]{Beweis}n}
\newglossaryentry{Beweis}{
	name        ={Beweis},
	plural      ={Beweise},
	description     ={
		Eine zulässige Ableitung von \glos{Folgerungen} aus gegebenen \glos{Voraussetzungen}.
		\\-- \vrefSeesec{sec:BeweiseASBA}.
	}
}
\newcommand*    {\beweisbar} [1][]{\glsIdx  [#1]{beweisbar}}
\newcommand*    {\beweisbare}[1][]{\glsIdxPl[#1]{beweisbar}}
\newglossaryentry{beweisbar}{
	name        ={beweisbar},
	plural      ={beweisbare},
	description     ={
		Synonym zu \glos{ableitbar}.
	}
}
\newcommand*    {\Beweisschritt}  [1][]{\glsIdx  [#1]{Beweisschritt}}
\newcommand*    {\Beweisschritte} [1][]{\glsIdxPl[#1]{Beweisschritt}}
\newcommand*    {\Beweisschritten}[1][]{\glsIdxPl[#1]{Beweisschritt}n}
\newglossaryentry{Beweisschritt}{
	name        ={Beweisschritt},
	plural      ={Beweisschritte},
	description     ={
		Eine Vorschrift, wie aus vorgegebenen \glos{Aussagen} (den \glos{Voraussetzungen}) weitere (die \glos{Folgerungen}) folgen.
		\\-- Zur Definition \vrefseesub{sub:Beweisschritte}.
	}
}
\newcommand*    {\Beweisschrittfolge} [1][]{\glsIdx  [#1]{Beweisschrittfolge}}
\newcommand*    {\Beweisschrittfolgen}[1][]{\glsIdxPl[#1]{Beweisschrittfolge}}
\newglossaryentry{Beweisschrittfolge}{
	name        ={Beweisschrittfolge},
	plural      ={Beweisschrittfolgen},
	description     ={
		Eine Folge von \glos{Beweisschritten}.
		\\-- Zur Definition \vrefseesub{sub:Beweisschritte}.
	}
}
\newcommand*    {\Beweisschrittmenge} [1][]{\glsIdx  [#1]{Beweisschrittmenge}}
\newcommand*    {\Beweisschrittmengen}[1][]{\glsIdxPl[#1]{Beweisschrittmenge}}
\newglossaryentry{Beweisschrittmenge}{
	name        ={Beweisschrittmenge},
	plural      ={Beweisschrittmengen},
	description     ={
		Eine Menge von \glos{Beweisschritten}, insbesondere die Menge der Glieder einer \glos{Beweisschrittfolge}.
		\\-- Zur Definition \vrefseesub{sub:Beweisschritte}.
	}
}
\newcommand*    {\BoolscheSignatur} [1][]{\glsIdy  [#1]{Boolsche-Signatur}{Boolsche Signatur}}
\newcommand*    {\BoolschenSignatur}[1][]{\glsIdyPl[#1]{Boolsche-Signatur}{Boolsche Signatur}}
\newglossaryentry{Boolsche-Signatur}{
	name        ={Boolsche Signatur},
	plural      ={Boolschen Signatur},
	description     ={
		Die \glos{logische Signatur} $\{\lnot, \land, \lor\}$.
	}
}

%D === D === D === D === D === D === D === D === D === D === D === D === D === D

\newcommand*    {\Definition}  [1][]{\glsIdx  [#1]{Definition}}
\newcommand*    {\Definitionen}[1][]{\glsIdxPl[#1]{Definition}}
\newglossaryentry{Definition}{
	name        ={Definition},
	plural      ={Definitionen},
	description     ={
		Eine Definition mit Hilfe des \emph{Definitionssymbols} \chrqt{$\defeq$}.
		\seqqt{$A \defeq B$} steht für \enquote{$A$ \emph{ist definitionsgemäß gleich} $B$} für \glos{Objekte} $A$ und $B$.
		Gewissermaßen ist $A$ nur eine andere Schreibweise für $B$.
		\\-- Man vergleiche auch den Begriff \enquote{\glos{Metadefinition}} und das zugehörige \glos{Symbol} \chrqt{$\metadefeq$}.
		\\-- Zur Definition \vrefseesub{subsub:Definitionen}.
	}
}
\newcommand*    {\Definitionsbereich} [1][]{\glsIdx  [#1]{Definitionsbereich}}
\newcommand*    {\Definitionsbereiche}[1][]{\glsIdxPl[#1]{Definitionsbereich}}
\newglossaryentry{Definitionsbereich}{
	name        ={Definitionsbereich},
	plural      ={Definitionsbereiche},
	description     ={
		einer \glos{Funktion}.
		\\-- Zur genaueren Definition \vrefseesub{sub:weitereBezeichnungen}.
	}
}

%E === E === E === E === E === E === E === E === E === E === E === E === E === E

\newcommand*    {\Ergebnis}  [1][]{\glsIdx  [#1]{Ergebnis}}
\newcommand*    {\Ergebnisse}[1][]{\glsIdxPl[#1]{Ergebnis}}
\newglossaryentry{Ergebnis}{
	name        ={Ergebnis},
	plural      ={Ergebnisse},
	description     ={
		Ein \emph{Ergebnis} eines \glos{Beweises}.
		\\-- Standardsymbol: $\outcome$;
		zur Definition \vrefseesub{sub:Beweise}.
	}
}
\newcommand*    {\Ergebnismenge} [1][]{\glsIdx  [#1]{Ergebnismenge}}
\newcommand*    {\Ergebnismengen}[1][]{\glsIdxPl[#1]{Ergebnismenge}}
\newglossaryentry{Ergebnismenge}{
	name        ={Ergebnismenge},
	plural      ={Ergebnismengen},
	description     ={
		Die Menge der \emph{Ergebnisse} eines \glos{Beweises}.
		\\-- Standardsymbol: $\outcomeSet$;
		zur Definition \vrefseesub{sub:Beweise}.
	}
}

%F === F === F === F === F === F === F === F === F === F === F === F === F === F

\newcommand*    {\Fachbegriff}  [1][]{\glsIdx  [#1]{Fachbegriff}}
\newcommand*    {\Fachbegriffe} [1][]{\glsIdxPl[#1]{Fachbegriff}}
\newcommand*    {\Fachbegriffen}[1][]{\glsIdxPl[#1]{Fachbegriff}n}
\newglossaryentry{Fachbegriff}{
	name        ={Fachbegriff},
	plural      ={Fachbegriffe},
	description     ={
		Ein Name für einen mathematischen Begriff.
	}
}
\newcommand*    {\Fachgebiet}  [1][]{\glsIdx  [#1]{Fachgebiet}}
\newcommand*    {\Fachgebiete} [1][]{\glsIdxPl[#1]{Fachgebiet}}
\newcommand*    {\Fachgebieten}[1][]{\glsIdxPl[#1]{Fachgebiet}n}
\newglossaryentry{Fachgebiet}{
	name        ={Fachgebiet},
	plural      ={Fachgebiete},
	description     ={
		Ein Teil der Mathematik mit einer zugehörigen Basis aus \glos{Axiomen}, \glos{Sätzen}, \glos{Fachbegriffen} und Darstellungsweisen.
	}
}
\newcommand*    {\Folgerung}  [1][]{\glsIdx  [#1]{Folgerung}}
\newcommand*    {\Folgerungen}[1][]{\glsIdxPl[#1]{Folgerung}}
\newglossaryentry{Folgerung}{
	name        ={Folgerung},
	plural      ={Folgerungen},
	description     ={
		Die \emph{Folgerungen} einer \glos{Schlussregel} $\frac{\prerequisiteSet}{\conclusionSet}$ sind die Elemente von $\conclusionSet$.
		\\-- Standardsymbol: $\conclusion$; zur Definition \vrefseesub{sub:Schlussregeln}.
	}
}
%%%\newcommand*    {\Folgerungsmenge} [1][]{\glsIdx  [#1]{Folgerungsmenge}}
%%%\newcommand*    {\Folgerungsmengen}[1][]{\glsIdxPl[#1]{Folgerungsmenge}}
%%%\newglossaryentry{Folgerungsmenge}{
%%%	name        ={Folgerungsmenge},
%%%	plural      ={Folgerungsmengen},
%%%	description     ={
%%%		Die Menge der \glos{Folgerungen} einer \glos{Schlussregel} \textbzw\ eines \glos{Beweises}.
%%%		\\-- Standardsymbol: $\conclusionSet$; zur Definition \vrefseesub{:Schlussregeln}.
%%%	}
%%%}
\newcommand*    {\formalerSatz} [1][]{\glsIdy  [#1]{formaler-Satz}{formaler Satz}}
\newcommand*    {\formaleSaetze}[1][]{\glsIdyPl[#1]{formaler-Satz}{formaler Satz}}
\newcommand*    {\formalenSatz} [1][]{\glsIdZBg[#1]{formaler-Satz}{formalen Satz}}
\newglossaryentry{formaler-Satz}{
	name        ={formaler Satz},
	plural      ={formale Sätze},
	description     ={
		Formale Darstellung eines mathematischen Satzes.
		\\-- Siehe~\glos{FS}; zur Definition \vrefseesub{sub:Schlussregeln}.
	}
}
\newcommand*    {\Formel} [1][]{\glsIdx  [#1]{Formel}}
\newcommand*    {\Formeln}[1][]{\glsIdxPl[#1]{Formel}}
\newglossaryentry{Formel}{
	name        ={Formel},
	plural      ={Formeln},
	description     ={
		Unter einer \glos{Formel} verstehen wir stets eine mathematische \glos{Formel}.
		Diese kann aus einem einzigen \glos{Symbol} bestehen (\glos{atomare Formel}), andererseits aber auch mehrdimensional sein, lässt sich dann aber mittels geeigneter \glos{Definitionen} immer eindeutig als eine \glos{Zeichenfolge} schreiben.
		\glos{Sätze}, \glos{Beweise} und \glos{Schlussregeln} betrachten wir \emph{nicht} als \glos{Formeln}.
		\\-- Zur Definition \vrefseesub{sub:Bezeichnungen}
		\\-- Zur Definition \vrefseesubsub{subsub:Formeln}.
	}
}
\newcommand*    {\Formelmenge} [1][]{\glsIdx  [#1]{Formelmenge}}
\newcommand*    {\Formelmengen}[1][]{\glsIdxPl[#1]{Formelmenge}}
\newglossaryentry{Formelmenge}{
	name        ={Formelmenge},
	plural      ={Formelmengen},
	description     ={
		Eine Menge von \glos{Formeln}, oft mit $\formulaSet$ bezeichnet.
		Man nennt $\formulaSet$ auch eine \glos{Sprache} und ihre Elemente \glos{Worte}, insbesondere dann, wenn es eindeutige Regeln zur Konstruktion von $\formulaSet$ gibt.
		Wir bevorzugen \enquote{\glos{Formel}} und \enquote{\glos{Formelmenge}}.
	}
}
\newcommand*    {\Funktion}  [1][]{\glsIdx  [#1]{Funktion}}
\newcommand*    {\Funktionen}[1][]{\glsIdxPl[#1]{Funktion}}
\newglossaryentry{Funktion}{
	name        ={Funktion},
	plural      ={Funktionen},
	description     ={
		Eine \emph{$n$-stellige Funktion} $f$ von einer Menge $A = A_1 \times \dots \times A_n$, dem \emph{Definitionsbereich}, in eine Menge $B$, den \emph{Zielbereich}, ist eine ($n$+1)-stellige \glos{Relation} $(G,A_1,\dots,A_n,B)$ derart, dass es für jedes $\vec{a} = (a_1,\dots,a_n)$ mit $a_i \in A_i$ genau ein $b \in B$ gibt mit $(a_1,\dots,a_n,b) \in f$.
		Dieses $b$ wird auch mit \seqqt{$f(a_1,\dots,a_n)$} , \seqqt{$f a_1 \dots a_n$} , \seqqt{$f(\vec{a})$} oder \seqqt{$f\vec{a}$} bezeichnet.
		\\Schreibweise: \seqqt{$f : A \rightarrow B$} \textbzw\ \seqqt{$f : A_1 \times \dots \times A_n \rightarrow B$}
		\\-- Zur Definition \vrefseesec{sub:weitereBezeichnungen}.
	}
}
\newcommand*    {\Funktionswert} [1][]{\glsIdx  [#1]{Funktionswert}}
\newcommand*    {\Funktionswerte}[1][]{\glsIdxPl[#1]{Funktionswert}}
\newglossaryentry{Funktionswert}{
	name        ={Funktionswert},
	plural      ={Funktionswerte},
	description     ={
		einer \glos{Funktion}.
		\\-- Zur genaueren Definition \vrefseesub{sub:weitereBezeichnungen}.
	}
}

%G === G === G === G === G === G === G === G === G === G === G === G === G === G

\newcommand*    {\Gleichheit}[1][]{\glsIdx  [#1]{Gleichheit}}
\newglossaryentry{Gleichheit}{
	name        ={Gleichheit},
	description     ={
		Eine \glos{Gleichheitsrelation}:
		Zwei Objekte $A$ und $B$ sind \emph{identisch}, $A \eq B$, wenn sie in den \glos{interessierenden Eigenschaften} für $\eq$ übereinstimmen.
		\\-- Zur Definition \vrefseesubsub{subsub:Vergleiche}
	}
}
\newcommand*    {\Gleichheitsrelation}  [1][]{\glsIdx  [#1]{Gleichheitsrelation}}
\newcommand*    {\Gleichheitsrelationen}[1][]{\glsIdxPl[#1]{Gleichheitsrelation}}
\newglossaryentry{Gleichheitsrelation}{
	name        ={Gleichheitsrelation},
	plural      ={Gleichheitsrelationen},
	description     ={
		Eine mit \glos{Gleichheit} verwandte \glos{Relation}: $\eq$, $\ne$, $\equiv$ und $\nequiv$.
	}
}
\newcommand*    {\Graph}  [1][]{\glsIdx  [#1]{Graph}}
\newcommand*    {\Graphen}[1][]{\glsIdxPl[#1]{Graph}}
\newglossaryentry{Graph}{
	name        ={Graph},
	plural      ={Graphen},
	description     ={
		einer \glos{Funktion} oder \glos{Relation}.
		\\-- Zur genaueren Definition \vrefseesub{sub:weitereBezeichnungen}.
	}
}

%I === I === I === I === I === I === I === I === I === I === I === I === I === I

\newcommand*    {\Identitaetsregel} [1][]{\glsIdy  [#1]{Identitaetsregel}{Identitätsregel}}
\newcommand*    {\Identitaetsregeln}[1][]{\glsIdyPl[#1]{Identitaetsregel}{Identitätsregel}}
\newglossaryentry{Identitaetsregel}{
	name        ={Identitätsregel},
	plural      ={Identitätsregeln},
	description     ={
		Eigentlich eine \glos{Basisregel} zur Identität.
		Da die \glos{Identitätsregeln} nur zur Rechtfertigung der \glos{Substitution} verwendet werden, werden sie hier nicht zu den \glos{Basisregeln} gezählt.
		\\-- Zur Definition \vrefseesub{sub:Identitätsregeln}.
	}
}
\newcommand*    {\interessierendeEigenschaft}   [1][]{\glsIdy  [#1]{interessierende-Eigenschaft}{interessierende Eigenschaft}}
\newcommand*    {\interessierendeEigenschaften} [1][]{\glsIdyPl[#1]{interessierende-Eigenschaft}{interessierende Eigenschaft}}
\newcommand*    {\interessierendenEigenschaft}  [1][]{\glsIdZBg[#1]{interessierende-Eigenschaft}{interessierenden Eigenschaft}}
\newcommand*    {\interessierendenEigenschaften}[1][]{\glsIdZBg[#1]{interessierende-Eigenschaft}{interessierenden Eigenschaften}}
\newglossaryentry{interessierende-Eigenschaft}{
	name        ={interessierende Eigenschaft},
	plural      ={interessierende Eigenschaften},
	description     ={
		Solche Eigenschaften von \glos{Objekten}, die im aktuellen Zusammenhang von Interesse sind, \textzB\ einen bestimmten Wert zu haben, Element einer bestimmten Menge zu sein, ein bestimmtes \glos{Objekt} zu bezeichnen, usw.
	}
}

%J === J === J === J === J === J === J === J === J === J === J === J === J === J

\newcommand*    {\Junktor}  [1][]{\glsIdx  [#1]{Junktor}}
\newcommand*    {\Junktoren}[1][]{\glsIdxPl[#1]{Junktor}}
\newglossaryentry{Junktor}{
	name        ={Junktor},
	plural      ={Junktoren},
	description     ={
		Eine aussagenlogische \glos{Operation}.
		Da die Werte einer aussagenlogischen \glos{Operation} \glos{Wahrheitswerte} sind, kann man einen \glos{Junktor} auch als \glos{Relation} verstehen.
		\\-- Zur Definition \vrefseesub{sub:weitereBezeichnungen}
		\\-- Zur Definition \vrefseesub{sub:ausJunktorDef}.
	}
}
\newcommand*    {\Junktorsymbol} [1][]{\glsIdx  [#1]{Junktorsymbol}}
\newcommand*    {\Junktorsymbole}[1][]{\glsIdxPl[#1]{Junktorsymbol}}
\newglossaryentry{Junktorsymbol}{
	name        ={Junktorsymbol},
	plural      ={Junktorsymbole},
	description     ={
		Ein \glos{Symbol} für einen \glos{Junktor}.%
		\footnote{%
			Entsprechend \emph{Funktionssymbol}, \emph{Operatorsymbol}, \emph{Relationssymbol}, usw.
		}
	}
}

%K === K === K === K === K === K === K === K === K === K === K === K === K === K

\newcommand*    {\Kontraposition}[1][]{\glsIdx  [#1]{Kontraposition}}
\newglossaryentry{Kontraposition}{
	name        ={Kontraposition},
	description     ={
		Die allgemeingültige \glos{Aussage}: $ (\alpha \limp \beta) \limp (\lnot\beta \limp \lnot\alpha) $.
	}
}
\newcommand*    {\Kontravalenz}[1][]{\glsIdx  [#1]{Kontravalenz}}
\newglossaryentry{Kontravalenz}{
	name        ={Kontravalenz},
	description     ={
		Eine \glos{Gleichheitsrelation}:
		Zwei Objekte $A$ und $B$ sind \emph{nicht gleich} (nicht äquivalent), $A \nequiv B$, wenn sie in mindestens einer \glos{interessierenden Eigenschaft} für $\equiv$ nicht übereinstimmen.
		\\-- Zur Definition \vrefseesubsub{subsub:Vergleiche}.
	}
}

%L === L === L === L === L === L === L === L === L === L === L === L === L === L

\newcommand*    {\logischeSignatur}  [1][]{\glsIdy  [#1]{logische-Signatur}{logische Signatur}}
\newcommand*    {\logischeSignaturen}[1][]{\glsIdyPl[#1]{logische-Signatur}{logische Signatur}}
\newcommand*    {\logischenSignatur} [1][]{\glsIdZBg[#1]{logische-Signatur}{logischen Signatur}}
\newglossaryentry{logische-Signatur}{
	name        ={logische Signatur},
	plural      ={logische Signaturen},
	description     ={
		Eine Teilmenge von $\alJun$, ausreichend um damit alle anderen Elemente von $\alJun$ zu definieren.
	}
}

%M === M === M === M === M === M === M === M === M === M === M === M === M === M

\newcommand*    {\Mengenlehre}[1][]{\glsIdx  [#1]{Mengenlehre}}
\newglossaryentry{Mengenlehre}{
	name={Mengenlehre},
	description     ={
		-- Zur Definition \vrefseesec{sec:Mengenlehre}.
	}
}
\newcommand*    {\Metadefinition}  [1][]{\glsIdx  [#1]{Metadefinition}}
\newcommand*    {\Metadefinitionen}[1][]{\glsIdxPl[#1]{Metadefinition}}
\newglossaryentry{Metadefinition}{
	name        ={Metadefinition},
	plural      ={Metadefinitionen},
	description     ={
		Eine \glos{Definition} in der \glos{Metasprache} mit Hilfe des \emph{Metadefinitionssymbols} \chrqt{$\metadefeq$}.
		\seqqt{$A \metadefeq B$} steht für \enquote{$A$ \emph{ist definitionsgemäß gleich} $B$} für \glos{Aussagen} $A$ und $B$.
		Gewissermaßen ist $A$ nur eine andere Schreibweise für $B$.
		\\-- Man vergleiche auch den Begriff \enquote{\glos{Definition}} und das zugehörige \glos{Symbol} \chrqt{$\defeq$}.
		\\-- Zur Definition \vrefseesubsub{subsub:Definitionen}.
	}
}
\newcommand*    {\Metaoperation}  [1][]{\glsIdx  [#1]{Metaoperation}}
\newcommand*    {\Metaoperationen}[1][]{\glsIdxPl[#1]{Metaoperation}}
\newglossaryentry{Metaoperation}{
	name        ={Metaoperation},
	plural      ={Metaoperationen},
	description     ={
		Eine \glos{Operation} der \glos{Metasprache}: $\metaand$, $\metaor$ oder $\srand$.
		\\-- Zur Definition \vrefseesub{sub:AussagenUndMetaoperationen}.
	}
}
\newcommand*    {\Metarelation}  [1][]{\glsIdx  [#1]{Metarelation}}
\newcommand*    {\Metarelationen}[1][]{\glsIdxPl[#1]{Metarelation}}
\newglossaryentry{Metarelation}{
	name        ={Metarelation},
	plural      ={Metarelationen},
	description     ={
		Eine \glos{Relation} der \glos{Metasprache}: $\metaimp$, $\metarep$ oder $\metaequiv$.
		\\-- Zur Definition \vrefseesub{sub:AussagenUndMetaoperationen}.
	}
}
\newcommand*    {\Metasprache} [1][]{\glsIdx  [#1]{Metasprache}}
\newcommand*    {\Metasprachen}[1][]{\glsIdxPl[#1]{Metasprache}}
\newglossaryentry{Metasprache}{
	name        ={Metasprache},
	plural      ={Metasprachen},
	description     ={
		Eine Sprache, in der \glos{Aussagen} über Elemente einer anderen Sprache getroffen werden können.
		In diesem Dokument ist dies immer die normale Sprache.
		\\-- \vrefSeesec{sec:Metasprache}.
	}
}
\newcommand*    {\Monotonieregel}[1][]{\glsIdx  [#1]{Monotonieregel}}
\newglossaryentry{Monotonieregel}{
	name        ={Monotonieregel},
	description     ={
		Eine \glos{Schlussregel}. -- siehe~\glos{MR}.
	}
}

%O === O === O === O === O === O === O === O === O === O === O === O === O === O

\newcommand*    {\Objekt} [1][]{\glsIdx  [#1]{Objekt}}
\newcommand*    {\Objekte}[1][]{\glsIdxPl[#1]{Objekt}}
\newcommand*    {\Objekts}[1][]{\glsIdx  [#1]{Objekt}s}
\newglossaryentry{Objekt}{
	name        ={Objekt},
	plural      ={Objekte},
	description     ={
		\glos{Symbole}, \glos{Formeln} und \glos{Aussagen} sowie Mengen, \glos{Zeichenfolgen}, Zahlen; ganz allgemein reale oder gedachte Dinge an sich.
		\\-- Zur Definition \vrefseesub{sub:Bezeichnungen}.
	}
}
\newcommand*    {\Operation}  [1][]{\glsIdx  [#1]{Operation}}
\newcommand*    {\Operationen}[1][]{\glsIdxPl[#1]{Operation}}
\newglossaryentry{Operation}{
	name        ={Operation},
	plural      ={Operationen},
	description     ={
		Eine -- meistens binäre, \textdh\ zweiwertige -- Funktion $M^n \rightarrow M$.
		Für eine binäre \glos{Operation} $\opbsp : M \times M \rightarrow M$ schreibt man meistens $x \opbsp y$ statt $\opbsp(x,y)$.
		\\-- Zur Definition \vrefseesub{sub:weitereBezeichnungen}
		\\-- \vrefSeesub{sub:Beispielsymbole} und \vref{sub:Operationen}.
	}
}
\newcommand*    {\Operationssymbol} [1][]{\glsIdx  [#1]{Operationssymbol}}
\newcommand*    {\Operationssymbole}[1][]{\glsIdxPl[#1]{Operationssymbol}}
\newglossaryentry{Operationssymbol}{
	name        ={Operationssymbol},
	plural      ={Operationssymbole},
	description     ={
		Ein \glos{Symbol} für eine \glos{Operation}.
	}
}

%P === P === P === P === P === P === P === P === P === P === P === P === P === P

\newcommand*    {\PolnischeNotation}  [1][]{\glsIdy  [#1]{PolnischeNotation}{Polnische Notation}}
\newcommand*    {\PolnischeNotationen}[1][]{\glsIdyPl[#1]{PolnischeNotation}{Polnische Notation}}
\newcommand*    {\PolnischenNotation} [1][]{\glsIdZBg[#1]{PolnischeNotation}{Polnischen Notation}}
\newglossaryentry{PolnischeNotation}{
	name        ={Polnische Notation},
	plural      ={Polnische Notationen},
	description     ={
		Bei der \emph{Polnischen Notation} stehen die Operanden \textbzw\ Argumente von \glos{Relationen} und \glos{Funktionen} stets rechts von den Relations- und Funktionssymbolen.
		Dadurch kann auf Gliederungszeichen wie Klammern und Kommata verzichtet werden.
		Noch einfacher für Computer ist die \emph{umgekehrte Polnische Notation}, bei der die Operanden und Argumente links von den Symbolen stehen.
	}
}
\newcommand*    {\Potenzmenge} [1][]{\glsIdx  [#1]{Potenzmenge}}
\newcommand*    {\Potenzmengen}[1][]{\glsIdxPl[#1]{Potenzmenge}}
\newglossaryentry{Potenzmenge}{
	name        ={Potenzmenge},
	plural      ={Potenzmengen},
	description     ={
		Die \emph{Potenzmenge} $\Pot(M)$ einer Menge $M$ ist die Menge ihrer Teilmengen.
		\\-- Zur Definition \vrefseesub{sub:Bezeichnungen}.
	}
}
\newcommand*    {\Praedikat} [1][]{\glsIdy  [#1]{Praedikat}{Prädikat}}
\newcommand*    {\Praedikate}[1][]{\glsIdyPl[#1]{Praedikat}{Prädikat}}
\newglossaryentry{Praedikat}{
	name        ={Prädikat},
	plural      ={Prädikate},
	description     ={
		Ein Element der \glos{Prädikatenlogik}.
		\\-- Zur Definition \vrefseesec{sec:Prädikatenlogik}.
		\\\textZB\ kann man eine Gruppe als ein zweistelliges \glos{Prädikat} $\mathrm{Gruppe}(G,+)$ definieren, in dem $G$ eine Menge und $+$ eine \glos{Operation}, \textdh\ eine binäre (zweistellige) Funktion $ +: G \times G \rightarrow G $ ist, so dass die Gruppenaxiome erfüllt sind.
	}
}
\newcommand*    {\Praedikatenlogik}[1][]{\glsIdy  [#1]{Praedikatenlogik}{Prädikatenlogik}}
\newglossaryentry{Praedikatenlogik}{
	name={Prädikatenlogik},
	description     ={
		-- Zur Definition \vrefseesec{sec:Prädikatenlogik}.
	}
}

%R === R === R === R === R === R === R === R === R === R === R === R === R === R

\newcommand*    {\Relation}  [1][]{\glsIdx  [#1]{Relation}}
\newcommand*    {\Relationen}[1][]{\glsIdxPl[#1]{Relation}}
\newglossaryentry{Relation}{
	name        ={Relation},
	plural      ={Relationen},
	description     ={
		Eine \emph{$n$-stellige Relation} $R$ ist ein (1+$n$)-Tupel $(G,A_1,\dots,A_n$) mit $G \subseteq A_1 \times \dots \times A_n)$.
		\\-- Zur genaueren Definition \vrefseesub{sub:weitereBezeichnungen}
 		\\-- \vrefSeesub{sub:Beispielsymbole} und \vref{sub:Gleichheit}.
	}
}

%S === S === S === S === S === S === S === S === S === S === S === S === S === S

\newcommand*    {\Satz}   [1][]{\glsIdx  [#1]{Satz}}
\newcommand*    {\Saetze} [1][]{\glsIdxPl[#1]{Satz}}
\newcommand*    {\Satzes} [1][]{\glsIdx  [#1]{Satz}e}
\newcommand*    {\Saetzen}[1][]{\glsIdxPl[#1]{Satz}n}
\newglossaryentry{Satz}{
	name        ={Satz},
	plural      ={Sätze},
	description     ={
		Eine mathematische \glos{Aussage}, dass bestimmte \glos{Folgerungen} aus gegebenen \glos{Voraussetzungen} abgeleitet werden können.
	}
}
\newcommand*    {\Schlussregel} [1][]{\glsIdx  [#1]{Schlussregel}}
\newcommand*    {\Schlussregeln}[1][]{\glsIdxPl[#1]{Schlussregel}}
\newglossaryentry{Schlussregel}{
	name        ={Schlussregel},
	plural      ={Schlussregeln},
	description     ={
		Eine \glos{Schlussregel} $\frac{\prerequisiteSet}{\conclusionSet}$ entspricht der \glos{Aussage}:
		Wenn alle \glos{Voraussetzungen} $\prerequisite$ aus $\prerequisiteSet$ zutreffen, dann auch alle \glos{Folgerungen} $\conclusion$ aus $\conclusionSet$.
		Wenn diese \glos{Aussage} zutrifft, kann die Schlussregel zur \glos{zulässigen Transformation} von \glos{Formeln} dienen.
		\\-- Zur Definition \vrefseesub{sub:Schlussregeln}.
	}
}
\newcommand*    {\Schlussregelmenge} [1][]{\glsIdx  [#1]{Schlussregelmenge}}
\newcommand*    {\Schlussregelmengen}[1][]{\glsIdxPl[#1]{Schlussregelmenge}}
\newglossaryentry{Schlussregelmenge}{
	name        ={Schlussregelmenge},
	plural      ={Schlussregelmengen},
	description     ={
		Eine Menge von \glos{Schlussregeln}, meistens mit $\conclusionruleSet$ bezeichnet.
		\\-- Zur Definition \vrefseesub{:Schlussregeln}.
	}
}
\newcommand*    {\Schnittregel}[1][]{\glsIdx  [#1]{Schnittregel}}
\newglossaryentry{Schnittregel}{
	name        ={Schnittregel},
	plural      ={Schnittregeln},
	description     ={
		Eine \glos{allgemeingültige Schlussregel}.
		\\-- Siehe~\glos{SR}.
	}
}
\newcommand*    {\Sprache} [1][]{\glsIdx  [#1]{Sprache}}
\newcommand*    {\Sprachen}[1][]{\glsIdxPl[#1]{Sprache}}
\newglossaryentry{Sprache}{
	name        ={Sprache},
	plural      ={Sprachen},
	description     ={
		-- Siehe \glos{Formelmenge}.
	}
}
\newcommand*    {\Stelligkeit}  [1][]{\glsIdx  [#1]{Stelligkeit}}
\newcommand*    {\Stelligkeiten}[1][]{\glsIdxPl[#1]{Stelligkeit}}
\newglossaryentry{Stelligkeit}{
	name        ={Stelligkeit},
	plural      ={Stelligkeiten},
	description     ={
		einer \glos{Funktion} oder \glos{Relation}.
		\\-- Zur genaueren Definition \vrefseesub{sub:weitereBezeichnungen}.
	}
}
\newcommand*    {\Substitution}  [1][]{\glsIdx  [#1]{Substitution}}
\newcommand*    {\Substitutionen}[1][]{\glsIdxPl[#1]{Substitution}}
\newglossaryentry{Substitution}{
	name        ={Substitution},
	plural      ={Substitutionen},
	description     ={
		Eine \glos{Funktion} zur Transformation einer \glos{Formel} mittels Substitution in eine gleichwertige.
		\\-- Zur Definition \vrefseesub{sub:Beweise}.
	}
}
\newcommand*    {\Substitutionsmenge} [1][]{\glsIdx  [#1]{Substitutionsmenge}}
\newcommand*    {\Substitutionsmengen}[1][]{\glsIdxPl[#1]{Substitutionsmenge}}
\newglossaryentry{Substitutionsmenge}{
	name        ={Substitutionsmenge},
	plural      ={Substitutionsmengen},
	description     ={
		Eine Menge von \glos{Substitutionen}, meistens mit $\substitutionSet$ bezeichnet.
	}
}
\newcommand*    {\Symbol}  [1][]{\glsIdx  [#1]{Symbol}}
\newcommand*    {\Symbole} [1][]{\glsIdxPl[#1]{Symbol}}
\newcommand*    {\Symbols} [1][]{\glsIdx  [#1]{Symbol}s}
\newcommand*    {\Symbolen}[1][]{\glsIdxPl[#1]{Symbol}n}
\newglossaryentry{Symbol}{
	name        ={Symbol},
	plural      ={Symbole},
	description     ={
		Ein \emph{einfaches} \glos{Symbol} ist ein druckbares typographisches Zeichen.
		Ein \emph{zusammengesetztes} \glos{Symbol} besteht aus mehreren einfachen \glos{Symbolen}.
		In beiden Fällen wird ein \glos{Symbol} als \emph{\glos{unzerlegbar}} angesehen.
		\\-- Zur Definition \vrefseesec{sec:Notationen}.
	}
}

%T === T === T === T === T === T === T === T === T === T === T === T === T === T

\newcommand*    {\Traegermenge} [1][]{\glsIdy  [#1]{Traegermenge}{Trägermenge}}
\newcommand*    {\Traegermengen}[1][]{\glsIdyPl[#1]{Traegermenge}{Trägermenge}}
\newglossaryentry{Traegermenge}{
	name        ={Trägermenge},
	plural      ={Trägermengen},
	description     ={
		einer \glos{Relation}.
		\\-- Zur genaueren Definition \vrefseesub{sub:weitereBezeichnungen}.
	}
}
\newcommand*    {\Transformation}  [1][]{\glsIdx  [#1]{Transformation}}
\newcommand*    {\Transformationen}[1][]{\glsIdxPl[#1]{Transformation}}
\newglossaryentry{Transformation}{
	name        ={Transformation},
	plural      ={Transformationen},
	description     ={
		Eine Umformung oder Erzeugung einer \glos{Formel} aus einer vorgegebenen Menge von \glos{Formeln},
		\textdh\ die Anwendung einer \glos{Schlussregel}.
	}
}
\newcommand*    {\Transformationsfolge} [1][]{\glsIdx  [#1]{Transformationsfolge}}
\newcommand*    {\Transformationsfolgen}[1][]{\glsIdxPl[#1]{Transformationsfolge}}
\newglossaryentry{Transformationsfolge}{
	name        ={Transformationsfolge},
	plural      ={Transformationsfolgen},
	description     ={
		Eine Folge von \glos{Transformationen}.
		\\-- Zur Definition \vrefseesub{sub:Beweisschritte}.
	}
}
\newcommand*    {\Tupel} [1][]{\glsIdx  [#1]{Tupel}}
\newglossaryentry{Tupel}{
	name        ={Tupel},
	plural      ={Tupel},
	description     ={
		Ein $n$-\emph{Tupel} $\vec{a}$ ist eine endliche Folge $\vec{a} = (a_1, \dots, a_n)$ seiner Komponenten $a_i$.
		\\-- Zur Definition \vrefseesub{sub:weitereBezeichnungen}.
	}
}
\newcommand*    {\Tupelmenge} [1][]{\glsIdx  [#1]{Tupelmenge}}
\newcommand*    {\Tupelmengen}[1][]{\glsIdxPl[#1]{Tupelmenge}}
\newglossaryentry{Tupelmenge}{
	name        ={Tupelmenge},
	plural      ={Tupelmengen},
	description     ={
		Die \emph{Tupelmenge} $\tupelSet(M)$ einer Menge $M$ ist die Menge aller $n$-Tupel aus $M^n$ für alle $n \in \INo$.
		\\-- Zur Definition \vrefseesub{sub:Bezeichnungen}.
	}
}

%U === U === U === U === U === U === U === U === U === U === U === U === U === U

\newcommand*    {\Umkehrrelation}  [1][]{\glsIdx  [#1]{Umkehrrelation}}
\newcommand*    {\Umkehrrelationen}[1][]{\glsIdxPl[#1]{Umkehrrelation}}
\newglossaryentry{Umkehrrelation}{
	name        ={Umkehrrelation},
	plural      ={Umkehrrelationen},
	description     ={
		Die \glos{Umkehrrelation} zu einer binären \glos{Relation} $(G,A,B)$ ist die \glos{Relation} $(\{(b,a)|(a,b) \in G\},B,A)$.
		Üblicherweise wird das zugehörige Relationssymbol gespiegelt.
	}
}
\newcommand*    {\Ungleichheit}[1][]{\glsIdx  [#1]{Ungleichheit}}
\newglossaryentry{Ungleichheit}{
	name        ={Ungleichheit},
	description     ={
		Eine \glos{Gleichheitsrelation}:
		Zwei Objekte $A$ und $B$ sind \emph{nicht identisch}, $A \ne B$, wenn sie in mindestens einer \glos{interessierenden Eigenschaft} für $\eq$ nicht übereinstimmen.
		\\-- Zur Definition \vrefseesubsub{subsub:Vergleiche}.
	}
}
\newcommand*    {\unzerlegbar} [1][]{\glsIdx  [#1]{unzerlegbar}}
\newcommand*    {\unzerlegbare}[1][]{\glsIdxPl[#1]{unzerlegbar}}
\newglossaryentry{unzerlegbar}{
	name        ={unzerlegbar},
	plural      ={unzerlegbare},
	description     ={
		Eine \glos{Aussage}, die keine \glos{Metaoperation}, und eine \glos{Formel}, die keine \glos{Operation} und keine \glos{Relation} enthält, heißt \glos{unzerlegbar}.
		\\-- Synonym: \glos{atomar}; vergleiche auch \glos{zerlegbar}.
	}
}

%V === V === V === V === V === V === V === V === V === V === V === V === V === V

\newcommand*    {\vergleichbar} [1][]{\glsIdx  [#1]{vergleichbar}}
\newcommand*    {\vergleichbare}[1][]{\glsIdxPl[#1]{vergleichbar}}
\newglossaryentry{vergleichbar}{
	name        ={vergleichbar},
	plural      ={vergleichbare},
	description     ={
		Zwei \glos{Objekte} $A$ und $B$ sind \glos{vergleichbar}, wenn beide von derselben Art sind, \textdh\ wenn beide \textzB\ jeweils Mengen, \glos{Zeichenfolgen}, Zahlen, \textusw\ sind.
		Dabei muss bei \glos{Formeln} zwischen der \glos{Formel} an sich und dem Ergebnis der \glos{Formel} unterschieden werden.
		\\-- Zur Definition \vrefseesub{subsub:Vergleichbar}.
	}
}
%TODO überarbeiten: Vertauschung muss überarbeitet werden
\newcommand*    {\Vertauschung}  [1][]{\glsIdx  [#1]{Vertauschung}}
\newcommand*    {\Vertauschungen}[1][]{\glsIdxPl[#1]{Vertauschung}}
\newglossaryentry{Vertauschung}{
	name        ={Vertauschung},
	plural      ={Vertauschungen},
	description     ={
		Die \glos{Vertauschung} von zwei unabhängigen Teil-\glos{Formeln} ($\alpha$ und $\beta$) in einer anderen \glos{Formel} ($\gamma$)
		\\-- Formal: $\gamma(\alpha\swap\beta)$.
		Die \glos{Vertauschung} ist eine spezielle Form der \glos{Substitution}.
		%TODO Makro für Verweis
		\\-- Zur Definition siehe~\eqref{def:Vertauschung} \vrefinsub{sub:Identitätsregeln}.
	}
}
\newcommand*    {\Voraussetzung}  [1][]{\glsIdx  [#1]{Voraussetzung}}
\newcommand*    {\Voraussetzungen}[1][]{\glsIdxPl[#1]{Voraussetzung}}
\newglossaryentry{Voraussetzung}{
	name        ={Voraussetzung},
	plural      ={Voraussetzungen},
	description     ={
		Die \glos{Voraussetzungen} einer \glos{Schlussregel} $\frac{\prerequisiteSet}{\conclusionSet}$ sind die Elemente von $\prerequisiteSet$.
		\\-- Standardsymbol: $\prerequisite$; zur Definition \vrefseesub{sub:Schlussregeln}.
	}
}
\newcommand*    {\Voraussetzungsmenge} [1][]{\glsIdx  [#1]{Voraussetzungsmenge}}
\newcommand*    {\Voraussetzungsmengen}[1][]{\glsIdxPl[#1]{Voraussetzungsmenge}}
\newglossaryentry{Voraussetzungsmenge}{
	name        ={Voraussetzungsmenge},
	plural      ={Voraussetzungsmengen},
	description     ={
		Die Menge der \glos{Voraussetzungen} einer \glos{Schlussregel} \textbzw\ eines \glos{Beweises}.
		\\-- Standardsymbol: $\prerequisiteSet$; zur Definition \vrefseesub{:Schlussregeln}.
	}
}

%W === W === W === W === W === W === W === W === W === W === W === W === W === W

\newcommand*    {\Wahrheitswert}  [1][]{\glsIdx  [#1]{Wahrheitswert}}
\newcommand*    {\Wahrheitswerte} [1][]{\glsIdxPl[#1]{Wahrheitswert}}
\newcommand*    {\Wahrheitswerten}[1][]{\glsIdxPl[#1]{Wahrheitswert}n}
\newglossaryentry{Wahrheitswert}{
	name        ={Wahrheitswert},
	plural      ={Wahrheitswerte},
	description     ={
		Die Werte \chrqt{$\ltrue$} und \chrqt{$\lfalse$}, oft auch mit \chrqt{$\wahr$} \textbzw\ \chrqt{$\falsch$}, \chrqt{$\mathrm{true}$} \textbzw\ \chrqt{$\mathrm{false}$} oder einfach \chrqt{$1$} \textbzw\ \chrqt{$0$} bezeichnet.
	}
}
\newcommand*    {\Wort}   [1][]{\glsIdx  [#1]{Wort}}
\newcommand*    {\Worte}  [1][]{\glsIdxPl[#1]{Wort}}
\newcommand*    {\Woerter}[1][]{\glsIdxPl[#1]{Wort}}
\newglossaryentry{Wort}{
	name        ={Wort},
	plural      ={Wörter},
	description     ={
		Ein Element einer \glos{Sprache}.
		In dem Fall Synonym zu \glodictionary{Formel}.
		\\-- Siehe \glos{Formelmenge}.
	}
}

%Z === Z === Z === Z === Z === Z === Z === Z === Z === Z === Z === Z === Z === Z

\newcommand*    {\Zeichenfolge} [1][]{\glsIdx  [#1]{Zeichenfolge}}
\newcommand*    {\Zeichenfolgen}[1][]{\glsIdxPl[#1]{Zeichenfolge}}
\newglossaryentry{Zeichenfolge}{
	name        ={Zeichenfolge},
	plural      ={Zeichenfolgen},
	description     ={
		Folgen von \glos{Symbolen}, wobei Leerstellen und sonstiger Zwischenraum nicht zählen und nur zur besseren Darstellung dienen.
		Dabei sind als spezielle \glos{Symbole} auch \glos{Zeichenketten} erlaubt, solange die Zerlegung eindeutig bleibt.
		\textZB\ kann \chrqt{sin} als ein einzelnes \glos{Symbol} -- für die Sinusfunktion -- aufgefasst werden, aber auch als Folge der Buchstaben \chrqt{s}, \chrqt{i} und \chrqt{n}.
		\glos{Formeln} werden immer als \glos{Zeichenfolgen} aufgefasst.
		%TODO Makro für große Version
		\\-- Siehe auch \glos{Zeichenkette}.
		\\-- Zur Definition \vrefseesub{subsub:Definitionen}.
	}
}
\newcommand*    {\Zeichenkette} [1][]{\glsIdx  [#1]{Zeichenkette}}
\newcommand*    {\Zeichenketten}[1][]{\glsIdxPl[#1]{Zeichenkette}}
\newglossaryentry{Zeichenkette}{
	name        ={Zeichenkette},
	plural      ={Zeichenketten},
	description     ={
		Folgen von \glos{Symbolen}, auch Leerstellen und sonstigem Zwischenraum.
		\\-- Siehe auch \glos{Zeichenfolge}.
		\\-- Zur Definition \vrefseesub{subsub:Definitionen}.
	}
}
\newcommand*    {\zerlegbar} [1][]{\glsIdx  [#1]{zerlegbar}}
\newcommand*    {\zerlegbare}[1][]{\glsIdxPl[#1]{zerlegbar}}
\newcommand*    {\Zerlegbare}[1][]{\GlsIdxPl[#1]{zerlegbar}}
\newglossaryentry{zerlegbar}{
	name        ={zerlegbar},
	plural      ={zerlegbare},
	description     ={
		Eine \glos{Aussage}, die eine \glos{Metaoperation}, und eine \glos{Formel}, die eine \glos{Operation} oder eine \glos{Relation} enthält, heißen \glos{zerlegbar}.
		\\-- Vergleiche auch \glos{unzerlegbar}.
	}
}
\newcommand*    {\Ziel} [1][]{\glsIdx  [#1]{Ziel}}
\newcommand*    {\Ziele}[1][]{\glsIdxPl[#1]{Ziel}}
\newglossaryentry{Ziel}{
	name        ={Ziel},
	plural      ={Ziele},
	description     ={
		In diesem Dokument sind \emph{Ziele} die Anforderungen an \glos{ASBA}.
	}
}
\newcommand*    {\Zielbereich} [1][]{\glsIdx  [#1]{Zielbereich}}
\newcommand*    {\Zielbereiche}[1][]{\glsIdxPl[#1]{Zielbereich}}
\newglossaryentry{Zielbereich}{
	name        ={Zielbereich},
	plural      ={Zielbereiche},
	description     ={
		einer \glos{Funktion}.
		\\-- Zur genaueren Definition \vrefseesub{sub:weitereBezeichnungen}.
	}
}
%TODO überarbeiten: zulässige Transformation muss voraussichtlich überarbeitet werden
\newcommand*    {\zulaessigeTransformation}   [1][]{\glsIdy  [#1]{zulaessige-Transformation}{zulässige Transformation}}
\newcommand*    {\zulaessigeTransformationen} [1][]{\glsIdyPl[#1]{zulaessige-Transformation}{zulässige Transformation}}
\newcommand*    {\zulaessigenTransformation}  [1][]{\glsIdZBg[#1]{zulaessige-Transformation}{zulässigen Transformation}}
\newcommand*    {\zulaessigenTransformationen}[1][]{\glsIdZBg[#1]{zulaessige-Transformation}{zulässigen Transformationen}}
\newcommand*    {\zulaessigerTransformationen}[1][]{\glsIdZBg[#1]{zulaessige-Transformation}{zulässiger Transformationen}}
\newglossaryentry{zulaessige-Transformation}{
	name        ={zulässige Transformation},
	plural      ={zulässige Transformationen},
	description     ={
		Eine \glos{Transformation} aus einer vorgegebenen Menge von \glos{Transformationen} oder eine daraus zulässigerweise abgeleitete \glos{Transformation}.
	}
}
