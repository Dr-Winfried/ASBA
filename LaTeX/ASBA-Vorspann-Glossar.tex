%%############################################################################%%
%%                                                                            %%
%% Datei:  ASBA-Vorspann-Glossar.tex                                          %%
%% Inhalt: Vorspann Glossareinträge für ASBA                                  %%
%%                                                                            %%
%% Copyright (C) 2017  Winfried Teschers                                      %%
%%                                                                            %%
%% This program is free software: you can redistribute it and/or modify       %%
%% it under the terms of the GNU Affero General Public License as published   %%
%% by the Free Software Foundation, either version 3 of the License, or       %%
%% (at your option) any later version.                                        %%
%%                                                                            %%
%% This program is distributed in the hope that it will be useful,            %%
%% but WITHOUT ANY WARRANTY; without even the implied warranty of             %%
%% MERCHANTABILITY or FITNESS FOR A PARTICULAR PURPOSE.  See the              %%
%% GNU Affero General Public License for more details.                        %%
%%                                                                            %%
%% You should have received a copy of the GNU Affero General Public License   %%
%% along with this program.  If not, see <http://www.gnu.org/licenses/>.      %%
%%                                                                            %%
%% Dr. Winfried Teschers                                                      %%
%% Anton-Günther-Straße 26c                                                   %%
%% 91083 Baiersdorf                                                           %%
%% Germany                                                                    %%
%%                                                                            %%
%% e-mail: winfried.teschers@t-online.de                                      %%
%%                                                                            %%
%%############################################################################%%

% !TeX root = ASBA.tex
% !TeX encoding = UTF-8
% !TeX spellcheck = de_DE

\newif\ifmitGlossarZusatzFlg% Schalter ob Glossar zu berücksichtigen ist; AUS
\mitGlossarZusatzFlgtrue% ... jetzt EIN
\newcommand*{\glsdeskOhneZusatz}[1]{%
	\mitGlossarZusatzFlgfalse%
	\glsdesc*{#1}%
	\mitGlossarZusatzFlgtrue%
}
\newcommand{\GlossarZusatz}[1]{\ifmitGlossarZusatzFlg #1\else\fi}% ohne *!

% Elemente, die keine Glossareinträge sind und dafür nicht gebraucht werden,
% werden in "ASBA-Vorspann.tex" und "ASBA-Mathematik-Vorspann.tex" definiert.

\newglossary[nlg]{symbols}{not}{ntn}{Symbolverzeichnis}

% Fonts für die Liste der Seitenangaben
\newcommand*    {\hyperTxt}[1]             {\hyperrm{#1}}%  für       Definitionen
\newcommand*{\likehyperTxt}[1]        {\linkcolor{#1}}%  simuliert Definitionen
\newcommand*    {\hyperDef}[1]{\textbf     {\hypersf{#1}}}% für       Definitionen
\newcommand*{\likehyperDef}[1]{\textbf{\linkcolor{#1}}}% simuliert Definitionen
\GlsAddXdyAttribute{hyperDef}% damit xindy damit umgehen kann
\GlsAddXdyAttribute{hyperTxt}% damit xindy damit umgehen kann

\GlsSetXdyMinRangeLength{2}% Seitenbereiche ab ...
\makeglossaries
\setacronymstyle{long-sc-short}
\renewcommand*{\glsnumberformat}[1]{\hyperTxt{#1}}% Standardformat für Seitenliste

% Makros für neue Symbole und Begriffe =========================================

% neue Symbole -----------------------------------------------------------------
% Ausgabe und Aufnahme ins Symbolverzeichnis, ohne Link, mit Hervorhebung der Seitennummer
% {<Glossary Key>}
\newcommand*{\glsTag}[1]{\tag{\gls*{#1}}\glsadd[format=hyperDef]{#1}}% ohne Link ins Symbolverzeichnis
% Ausgabe und Aufnahme und Link ins Symbolverzeichnis mit Hervorhebung der Seitennummer
% {<Makro>} - An <Makro> muss [<Font-Makro>] angehängt werden können
\newcommand*{\defSym}   [1]          {#1[format=hyperDef]}
\newcommand*{\defSymUna}[1]  {\defSym{#1}\;}%  unär: folgender  Abstand
\newcommand*{\defSymBin}[1]{\;\defSym{#1}\;}% binär: umgebender Abstand

% neue Begriffe ----------------------------------------------------------------
% Ausgabe und Aufnahme und Link ins Glossar mit Hervorhebung der Seitennummer
% {<Makro>} - An <Makro> muss [format=<Font-Makro>] angehängt werden können
\newcommand*{\defTxt}[1]{\defFt{#1[format=hyperDef]}}

% Automatischer Index ==========================================================
\newcommand*{\addIdx}[2][]{% {Index-Eintrag} - Index hinzufügen
	\newterm[name={#2},#1]{ind-#2}
	\glsadd               {ind-#2}
}
\newcommand*{\idx}[2][]{% [Text]{Index-Eintrag} - Ausgabe und Index hinzufügen
	\addIdx{#2}
	\def\ArgumentEins{#1}
	\def\ASBAundefined{}
	\ifx\ArgumentEins\ASBAundefined #2\else #1\fi%  wenn Text leer, Index-Eintrag ausgeben
}

% Synonyme =====================================================================

% [<Begriff>]{Makro für den <Begriff>}{<Begriff-Label>}{<Makro für das Synonym>}
\newcommand*{\newsynonym}[4][]{
	\def\ASBAundefined{}
	\def\ArgumentEins{#1}
	\ifx\ArgumentEins\ASBAundefined
		\newcommand*     {#2}[1][]{\glstext[##1]{#3}}
		\newglossaryentry{#3}{
			name        ={#3 \addIdx[
				name    ={#3}]                  {#3}},
			text        ={#3},
			description ={Synonym zu #4.}
		}
	\else
		\newcommand*     {#2}[1][]{\glstext[##1]{#3}}
		\newglossaryentry{#3}{
			name        ={#1 \addIdx[
				name    ={#1}]                  {#3}},
			text        ={#1},
			description ={Synonym zu #4.}
		}
	\fi
}

% Glossar-Einträge #############################################################

% ### Symbolverzeichnis: Symbole ###############################################

% ==============================================================================
% \* - Ausgabe als Symbol und Eintrag (und Verweis) ins Symbolverzeichnis
% Fachbegriffe =================================================================

\iftestFlg

\newcommand*    {\dummy} [1][]{\glstext[#1]{dummy}}
\newglossaryentry{dummy}{
	text       ={\ensuremath{\#}},
	name       ={\ensuremath{\#} \addIdx[
		name   ={\ensuremath{dummy}},
		sort   ={dummy}]               {dummy}},
	sort       ={=},
	type       ={symbols},
	description={
		\todo{Beschreibung fehlt noch}% TO
	}
}

\else \fi

% ==============================================================================
% \Bsp* - Ausgabe als Symbol und Eintrag und Verweis ins Symbolverzeichnis
\newglossaryentry{Glo-Beispielsymbole}{
	name  ={Beispielsymbole für Operationen und Relationen},% ==================
	sort  ={= 0 0 0},
	type  ={symbols},
	description={}
}

\newcommand*              {\BspOpU}[1][]{\glstext[#1]{BspOpU}}
\newglossaryentry          {BspOpU}{
	name  ={\ensuremath{\RawBspOpU}},
	sort  ={= 0 1 1},
	type  ={symbols},
	description={
		\Beispielsymbol\ für eine \unaere\ \Operation.
	}
}

\newcommand*              {\BspOpB}[1][]{\glstext[#1]{BspOpB}}
\newglossaryentry          {BspOpB}{
	name  ={\ensuremath{\RawBspOpB}},
	sort  ={= 0 1 2},
	type  ={symbols},
	description={
		\Beispielsymbol\ für eine \binaere\ \Operation.
	}
}

\newcommand*              {\BspRel}[1][]{\glstext[#1]{BspRel}}
\newglossaryentry          {BspRel}{
	name  ={\ensuremath{\RawBspRel}},
	sort  ={= 0 2 1},
	type  ={symbols},
	description={
		\Beispielsymbol\ für eine \binaere\ \Relation.
	}
}

\newcommand*              {\BspRelEq}[1][]{\glstext[#1]{BspRelEq}}
\newglossaryentry          {BspRelEq}{
	name  ={\ensuremath{\RawBspRelEq}},
	sort  ={= 0 2 2},
	type  ={symbols},
	description={
		\Beispielsymbol\ für eine \binaere\ \Relation.
	}
}

\newcommand*              {\BspRelBck}[1][]{\glstext[#1]{BspRelBck}}
\newglossaryentry          {BspRelBck}{
	name  ={\ensuremath{\RawBspRelBck}},
	sort  ={= 0 2 3},
	type  ={symbols},
	description={
		Die \Umkehrrelation\ von \BspRel.
	}
}

\newcommand*              {\BspRelBckEq}[1][]{\glstext[#1]{BspRelBckEq}}
\newglossaryentry          {BspRelBckEq}{
	name  ={\ensuremath{\RawBspRelBckEq}},
	sort  ={= 0 2 4},
	type  ={symbols},
	description={
		Die \Umkehrrelation\ von \BspRelEq.
	}
}

\newcommand*              {\BspRelN}[1][]{\glstext[#1]{BspRelN}}
\newglossaryentry          {BspRelN}{
	name  ={\ensuremath{\RawBspRelN}},
	sort  ={= 0 3 1},
	type  ={symbols},
	description={
		Die \Negation\ von \BspRel.
	}
}

\newcommand*              {\BspRelEqN}[1][]{\glstext[#1]{BspRelEqN}}
\newglossaryentry          {BspRelEqN}{
	name  ={\ensuremath{\RawBspRelEqN}},
	sort  ={= 0 3 2},
	type  ={symbols},
	description={
		Die \Negation\ von \BspRelEq.
	}
}

\newcommand*              {\BspRelBckN}[1][]{\glstext[#1]{BspRelBckN}}
\newglossaryentry          {BspRelBckN}{
	name  ={\ensuremath{\RawBspRelBckN}},
	sort  ={= 0 3 3},
	type  ={symbols},
	description={
		Die \Negation\ von \BspRelBck; gleichzeitig die \Umkehrrelation\ von \BspRelN.
	}
}

\newcommand*              {\BspRelBckEqN}[1][]{\glstext[#1]{BspRelBckEqN}}
\newglossaryentry          {BspRelBckEqN}{
	name  ={\ensuremath{\RawBspRelBckEqN}},
	sort  ={= 0 3 4},
	type  ={symbols},
	description={
		Die \Negation\ von \BspRelBckEq; gleichzeitig die \Umkehrrelation\ von \BspRelEqN.
	}
}

% ==============================================================================
% \Mts* - Ausgabe als Symbol und Eintrag und Verweis ins Symbolverzeichnis
\newglossaryentry{Glo-Metasymbole}{
	name  ={Metaoperationen, -relationen u.a},% ==================================
	sort  ={= 1 0 0},
	type  ={symbols},
	description={
		Im Folgenden seien $A$ und $B$ \Aussagen\ in den \metasprachlichenAusdruecken\ $\BspOpU A$ \textbzw $A \BspOpB B$.
	}
}

\newcommand*              {\MtsNot}[1][]{\glstext[#1]{MtsNot}}
\newglossaryentry          {MtsNot}{
	name  ={\ensuremath{\RawMtsNot}},
	sort  ={= 1 1 1},
	see   ={OjkNot},
	type  ={symbols},
	description={
		Eine \unaere\ \Metaoperation:~ \defFt{nicht} $A$
	}
}

\newcommand*              {\MtsAnd}[1][]{\glstext[#1]{MtsAnd}}
\newglossaryentry          {MtsAnd}{
	name  ={\ensuremath{\RawMtsAnd}},
	sort  ={= 1 1 2},
	see   ={OjkAnd},
	type  ={symbols},
	description={
		Eine \binaere\ \Metaoperation:~ $A$ \defFt{und} $B$
	}
}

\newcommand*              {\MtsOr}[1][]{\glstext[#1]{MtsOr}}
\newglossaryentry          {MtsOr}{
	name  ={\ensuremath{\RawMtsOr}},
	sort  ={= 1 1 3},
	see   ={OjkOr},
	type  ={symbols},
	description={
		Eine \binaere\ \Metaoperation:~ $A$ \defFt{oder} $B$
	}
}

\newcommand*              {\MtsImp}[1][]{\glstext[#1]{MtsImp}}
\newglossaryentry          {MtsImp}{
	name  ={\ensuremath{\RawMtsImp}},
	sort  ={= 1 2 1},
	see   ={OjkImp},
	type  ={symbols},
	description={
		Eine \binaere\ \Metarelation:~ wenn $A$ \defFt{dann} $B$
	}
}

\newcommand*              {\MtsRep}[1][]{\glstext[#1]{MtsRep}}
\newglossaryentry          {MtsRep}{
	name  ={\ensuremath{\RawMtsRep}},
	sort  ={= 1 2 2},
	see   ={OjkRep},
	type  ={symbols},
	description={
		Eine \binaere\ \Metarelation:~ $A$ \defFt{wenn} $B$
		; die \Umkehrrelation\ von \MtsImp.
	}
}

\newcommand*              {\MtsEquiv}[1][]{\glstext[#1]{MtsEquiv}}
\newglossaryentry          {MtsEquiv}{
	name  ={\ensuremath{\RawMtsEquiv}},
	sort  ={= 1 2 3},
	see   ={OjkEquiv},
	type  ={symbols},
	description={
		Eine \binaere\ \Metarelation:~ $A$ genau \defFt{dann wenn} $B$
	}
}

\newcommand*              {\MtsEq}[1][]{\glstext[#1]{MtsEq}}
\newglossaryentry          {MtsEq}{
	name  ={\ensuremath{\RawMtsEq}},
	sort  ={= 1 3 1},
	see   ={OjkEq,Gleichheit},
	type  ={symbols},
	description={
		Eine \binaere\ \Metarelation:~ $A$ ist \defFt{gleich}\alternativii{dasselbe wie}{identisch zu} $B$
	}
}

\newcommand*              {\MtsEqN}[1][]{\glstext[#1]{MtsEqN}}
\newglossaryentry          {MtsEqN}{
	name  ={\ensuremath{\RawMtsEqN}},
	sort  ={= 1 3 2},
	see   ={OjkEqN,Ungleichheit},
	type  ={symbols},
	description={
		Eine \binaere\ \Metarelation:~ $A$ ist \defFt{ungleich}\alternativii{nicht gleich}{nicht dasselbe wie}{nicht identisch zu} $B$
		; Die \Negation\ von \MtsEq.
	}
}

\newcommand*              {\MtsAequiv}[1][]{\glstext[#1]{MtsAequiv}}
\newglossaryentry          {MtsAequiv}{
	name  ={\ensuremath{\RawMtsAequiv}},
	sort  ={= 1 3 3},
	see   ={Aequivalenz},
	type  ={symbols},
	description={
		Eine \binaere\ \Metarelation:~ $A$ \defFt{äquivalent}\alternativii{so wie}{ähnlich} $B$
	}
}

\newcommand*              {\MtsAequivN}[1][]{\glstext[#1]{MtsAequivN}}
\newglossaryentry          {MtsAequivN}{
	name  ={\ensuremath{\RawMtsNAequiv}},
	sort  ={= 1 3 4},
	see   ={Aequivalenz},
	type  ={symbols},
	description={
		Eine \binaere\ \Metarelation:~ $A$ \defFt{nicht äquivalent}\alternativii{nicht so wie}{nicht ähnlich} $B$
		; Die \Negation\ von \MtsAequiv.
	}
}

\newcommand*              {\MtsDefEquiv}[1][]{\glstext[#1]{MtsDefEquiv}}
\newglossaryentry          {MtsDefEquiv}{
	name  ={\ensuremath{\RawMtsDefEquiv}},
	sort  ={= 1 4 1},
	type  ={symbols},
	description={
		\Metadefinition:~ $A$ \defFt{definitionsgemäß} genau \defFt{dann wenn} $B$
	}
}

\newcommand*              {\MtsDefEq}[1][]{\glstext[#1]{MtsDefEq}}
\newglossaryentry          {MtsDefEq}{
	name  ={\ensuremath{\RawMtsDefEq}},
	sort  ={= 1 4 2},
	type  ={symbols},
	description={
		\Definition:~ $A$ \defFt{definitionsgemäß gleich}\alternativii{dasselbe wie}{identisch zu} $B$
	}
}

\newcommand*              {\MtsUnd}[1][]{\glstext[#1]{MtsUnd}}
\newglossaryentry          {MtsUnd}{
	name  ={\ensuremath{\RawMtsUnd}},
	sort  ={= 1 5 1},
	see   ={MtsAnd,OjkAnd},
	type  ={symbols},
	description={
		Eine \binaere\ \Metaoperation\footnote{nur für Schlussregeln}:~ $A$ \defFt{und} $B$
	}
}

\newcommand*              {\MtsDerive}[1][]{\glstext[#1]{MtsDerive}}
\newglossaryentry          {MtsDerive}{
	name  ={\ensuremath{\RawMtsDerive}},
	sort  ={= 1 5 2},
	type  ={symbols},
	description={
		\Ableitungsrelation:~ $A$ \defFt{\ableitbar}\synonym{\beweisbar} $B$
	}
}

\newcommand*              {\MtsDeriveR}[1][]{\glstext[#1]{MtsDeriveR}}
\newglossaryentry          {MtsDeriveR}{
	name  ={\ensuremath{\RawMtsDerive_R}},
	sort  ={= 1 5 2R},
	type  ={symbols},
	description={
		Die \Darstellung\ einer \Relation\ $R \MtsIn \MtsRelAllDerive$ als \Ableitungsrelation.
	}
}

\newcommand*              {\MtsSubst}[1][]{\glstext[#1]{MtsSubst}}
\newglossaryentry          {MtsSubst}{
	name  ={\ensuremath{\RawMtsSubst}},
	sort  ={= 1 5 3},
	type  ={symbols},
	description={
		\Ersetzung:~ \textdots\ \defFt{substituiert durch} \textdots% ToDo Was sind die Operanden?
	}
}

\newcommand*              {\MtsSwap}[1][]{\glstext[#1]{MtsSwap}}
\newglossaryentry          {MtsSwap}{
	name  ={\ensuremath{\RawMtsSwap}},
	sort  ={= 1 5 4},
	type  ={symbols},
	description={
		\Vertauschung:~ \textdots\ \defFt{vertauscht mit} \textdots% ToDo Was sind die Operanden?
	}
}

% ==============================================================================
% \Mts* - Ausgabe als Symbol und Eintrag und Verweis ins Symbolverzeichnis
\newglossaryentry{Glo-Elementrelationen}{
	name  ={Elementrelationen},% =================================================
	sort  ={= 3 0 0},
	type  ={symbols},
	description={
		Im Folgenden sei $x$ ein \Element\ und $M$ eine \Menge\ in den \metasprachlichenAusdruecken\ $x \BspOpB M$ \textbzw\ $M \BspOpB x$.
	}
}

\newcommand*              {\MtsIn}[1][]{\glstext[#1]{MtsIn}}
\newglossaryentry          {MtsIn}{
	name  ={\ensuremath{\RawMtsIn}},
	sort  ={= 3 1 1},
	type  ={symbols},
	description={
		Eine \Elementrelation:~ $x$ ist \defFt{Element aus}%
		\alternativi[;
			\enquote{$a$ aus $A$} kann hier nur heißen: Element $a$ aus der Menge $A$, \enquote{$a$ von $A$} könnte \textzB\ auch \enquote{Komponente von} meinen.
		]{von}
		$M$;
		die grundlegende \Relation\ der \Mengenlehre.
	}
}

\newcommand*              {\MtsNi}[1][]{\glstext[#1]{MtsNi}}
\newglossaryentry          {MtsNi}{
	name  ={\ensuremath{\RawMtsNi}},
	sort  ={= 3 1 2},
	type  ={symbols},
	description={
		Eine \Elementrelation:~ $M$ \defFt{enthält} $x$;
		die \Umkehrrelation\ von \MtsIn.
	}
}

\newcommand*              {\MtsInN}[1][]{\glstext[#1]{MtsInN}}
\newglossaryentry          {MtsInN}{
	name  ={\ensuremath{\RawMtsInN}},
	sort  ={= 3 2 1},
	type  ={symbols},
	description={
		Eine \Elementrelation:~ $x$ ist \defFt{kein Element aus} $M$;
		die \Negation\ von \MtsIn.
	}
}

\newcommand*              {\MtsNiN}[1][]{\glstext[#1]{MtsNiN}}
\newglossaryentry          {MtsNiN}{
	name  ={\ensuremath{\RawMtsNiN}},
	sort  ={= 3 2 2},
	type  ={symbols},
	description={
		Eine \Elementrelation:~ $M$ \defFt{enthält} $x$ \defFt{nicht};
		die \Negation\ von\ \MtsNi; gleichzeitig die \Umkehrrelation\ von \MtsInN
	}
}

% ==============================================================================
% \Mts* - Ausgabe als Symbol und Eintrag und Verweis ins Symbolverzeichnis
\newglossaryentry{Glo-Mengenoperationen}{
	name  ={Mengenrelationen und -operationen},% =================================
	sort  ={= 4 0 0},
	type  ={symbols},
	description={
		\footnote{In diesem Dokument \Metarelationen\ und -\Moperationen.}
		Im Folgenden seien $M$ und $N$ \Mengen\ in den \metasprachlichenAusdruecken\ $M \BspOpB N$.
	}
}

\newcommand*              {\MtsSubset}[1][]{\glstext[#1]{MtsSubset}}
\newglossaryentry          {MtsSubset}{
	name  ={\ensuremath{\RawMtsSubset}},
	sort  ={= 4 1 1},
	type  ={symbols},
	description={
		Eine \Mengenrelation:~ $M$ ist \defFt{\echteTeilmenge\ von} $N$;
		es kann \emph{keine} \Gleichheit\ bestehen.\\
		Ursprünglich wurde \MtsSubset\ im Sinne von \MtsSubsetEq\ verwendet.
	}
}

\newcommand*              {\MtsSubsetEq}[1][]{\glstext[#1]{MtsSubsetEq}}
\newglossaryentry          {MtsSubsetEq}{
	name  ={\ensuremath{\RawMtsSubsetEq}},
	sort  ={= 4 1 2},
	type  ={symbols},
	description={
		Eine \Mengenrelation:~ $M$ ist \defFt{\Teilmenge\ von} $N$;
		es \emph{kann} \Gleichheit\ bestehen.\\
	}
}

\newcommand*              {\MtsSubsetN}[1][]{\glstext[#1]{MtsSubsetN}}
\newglossaryentry          {MtsSubsetN}{
	name  ={\ensuremath{\RawMtsSubsetN}},
	sort  ={= 4 1 3},
	type  ={symbols},
	description={
		Eine \Mengenrelation:~ $M$ ist \defFt{keine \echteTeilmenge\ von} $N$;
		es \emph{kann} aber \Gleichheit\ bestehen.\\
		Die \Negation\ von \MtsSubset.
	}
}

\newcommand*              {\MtsSubsetEqN}[1][]{\glstext[#1]{MtsSubsetEqN}}
\newglossaryentry          {MtsSubsetEqN}{
	name  ={\ensuremath{\RawMtsSubsetEqN}},
	sort  ={= 4 1 4},
	type  ={symbols},
	description={
		Eine \Mengenrelation:~ $M$ ist \defFt{keine \Teilmenge\ von} $N$;
		es kann auch \emph{keine} \Gleichheit\ bestehen.\\
		Die \Negation\ von \MtsSubsetEq.
	}
}

\newcommand*              {\MtsSupset}[1][]{\glstext[#1]{MtsSupset}}
\newglossaryentry          {MtsSupset}{
	name  ={\ensuremath{\RawMtsSupset}},
	sort  ={= 4 2 1},
	type  ={symbols},
	description={
		Eine \Mengenrelation:~ $M$ ist \defFt{\echteObermenge\ von} $N$;
		es kann \emph{keine} \Gleichheit\ bestehen.\\
		Die \Umkehrrelation\ von \MtsSubset.
		Ursprünglich wurde \MtsSupset\ im Sinne von \MtsSupsetEq\ verwendet.
	}
}

\newcommand*              {\MtsSupsetEq}[1][]{\glstext[#1]{MtsSupsetEq}}
\newglossaryentry          {MtsSupsetEq}{
	name  ={\ensuremath{\RawMtsSupsetEq}},
	sort  ={= 4 2 2},
	type  ={symbols},
	description={
		Eine \Mengenrelation:~ $M$ ist \defFt{\Obermenge\ von} $N$;
		es \emph{kann} \Gleichheit\ bestehen.\\
		Die \Umkehrrelation\ von \MtsSubsetEq.
	}
}

\newcommand*              {\MtsSupsetN}[1][]{\glstext[#1]{MtsSupsetN}}
\newglossaryentry          {MtsSupsetN}{
	name  ={\ensuremath{\RawMtsSupsetN}},
	sort  ={= 4 2 3},
	type  ={symbols},
	description={
		Eine \Mengenrelation:~ $M$ ist \defFt{keine \echteObermenge\ von} $N$;
		es \emph{kann} aber \Gleichheit\ bestehen.\\
		Die \Negation\ von \MtsSupset; gleichzeitig die \Umkehrrelation\ von \MtsSubsetN.
	}
}

\newcommand*              {\MtsSupsetEqN}[1][]{\glstext[#1]{MtsSupsetEqN}}
\newglossaryentry          {MtsSupsetEqN}{
	name  ={\ensuremath{\RawMtsSupsetEqN}},
	sort  ={= 4 2 4},
	type  ={symbols},
	description={
		Eine \Mengenrelation:~ $M$ ist \defFt{keine \Obermenge\ von} $N$;
		es kann auch \emph{keine} \Gleichheit\ bestehen.\\
		Die \Negation\ von \MtsSupsetEq; gleichzeitig die \Umkehrrelation\ von \MtsSubsetEqN.
	}
}

\newcommand*              {\MtsCap}[1][]{\glstext[#1]{MtsCap}}
\newglossaryentry          {MtsCap}{
	name  ={\ensuremath{\RawMtsCap}},
	sort  ={= 4 3 1},
	type  ={symbols},
	description={
		Eine \Mengenoperation:~ \defFt{\Durchschnitt} von $M$ und $N$.
		\\$M \MtsCap N \MtsDefEq \MengeDef{x}{(x \MtsIn M) \MtsAnd (x \MtsIn N)}$
	}
}

\newcommand*              {\MtsCup}[1][]{\glstext[#1]{MtsCup}}
\newglossaryentry          {MtsCup}{
	name  ={\ensuremath{\RawMtsCup}},
	sort  ={= 4 3 2},
	type  ={symbols},
	description={
		Eine \Mengenoperation:~ \defFt{\Vereinigung} von $M$ und $N$.
		\\$M \MtsCup N \MtsDefEq \MengeDef{x}{(x \MtsIn M) \MtsOr (x \MtsIn N)}$
	}
}

\newcommand*              {\MtsSetminus}[1][]{\glstext[#1]{MtsSetminus}}
\newglossaryentry          {MtsSetminus}{
	name  ={\ensuremath{\RawMtsSetminus}},
	sort  ={= 4 3 3},
	type  ={symbols},
	description={
		Eine \Mengenoperation:~ \defFt{\Differenz} von $M$ und $N$.
		\\$M \MtsSetminus N \MtsDefEq \MengeDef{x}{(x \MtsIn M) \MtsAnd (x \MtsInN N)}$
	}
}

\newcommand*              {\MtsTimes}[1][]{\glstext[#1]{MtsTimes}}
\newglossaryentry          {MtsTimes}{
	name  ={\ensuremath{\RawMtsTimes}},
	sort  ={= 4 3 4},
	type  ={symbols},
	description={
		Eine \Mengenoperation:~ \defFt{\kartesischesProdukt}\synonym{\Mengenprodukt}) von $M$ und $N$.
		\\$M \MtsTimes N \MtsDefEq \MengeDef{(x,y)}{(x \MtsIn M) \MtsAnd (y \MtsIn N)}$
	}
}

% ==============================================================================
% \MtsSeq* - Ausgabe als Symbol und Eintrag und Verweis ins Symbolverzeichnis
\newglossaryentry{Glo-Komponentenrelationen}{
	name  ={Komponentenrelationen},% ================================================
	sort  ={= 5 0 0},
	type  ={symbols},
	description={
		Im Folgenden sei $x$ eine \Komponente\ und $F$ eine \Folge\ in den \metasprachlichenAusdruecken\ $x \BspOpB F$ \textbzw\ $F \BspOpB x$.
	}
}

\newcommand*              {\MtsSeqIn}[1][]{\glstext[#1]{MtsSeqIn}}
\newglossaryentry          {MtsSeqIn}{
	name  ={\ensuremath{\RawMtsSeqIn}},
	sort  ={= 5 1 1},
	type  ={symbols},
	description={
		Eine \Komponentenrelation:~ $x$ ist \defFt{Komponente aus}%
		\alternativi[;
		\enquote{$a$ aus $A$} kann hier nur heißen: Element $a$ aus der Menge $A$, \enquote{$a$ von $A$} könnte \textzB\ auch \enquote{Komponente von} meinen.
		]{von}
		$F$;
		die grundlegende \Relation\ der \Mengenlehre.
	}
}

\newcommand*              {\MtsSeqNi}[1][]{\glstext[#1]{MtsSeqNi}}
\newglossaryentry          {MtsSeqNi}{
	name  ={\ensuremath{\RawMtsSeqNi}},
	sort  ={= 5 1 2},
	type  ={symbols},
	description={
		Eine \Komponentenrelation:~ $F$ \defFt{enthält} $x$ nicht als \Komponente;
		die \Umkehrrelation\ von \MtsSeqIn.
	}
}

\newcommand*              {\MtsSeqInN}[1][]{\glstext[#1]{MtsSeqInN}}
\newglossaryentry          {MtsSeqInN}{
	name  ={\ensuremath{\RawMtsSeqInN}},
	sort  ={= 5 2 1},
	type  ={symbols},
	description={
		Eine \Komponentenrelation:~ $x$ ist \defFt{keine Komponente aus} $F$;
		die \Negation\ von \MtsSeqIn.
	}
}

\newcommand*              {\MtsSeqNiN}[1][]{\glstext[#1]{MtsSeqNiN}}
\newglossaryentry          {MtsSeqNiN}{
	name  ={\ensuremath{\RawMtsSeqNiN}},
	sort  ={= 5 2 2},
	type  ={symbols},
	description={
		Eine \Komponentenrelation:~ $F$ \defFt{enthält} $x$ \defFt{nicht} als \Komponente;
		die \Negation\ von\ \MtsSeqNi; gleichzeitig die \Umkehrrelation\ von \MtsSeqInN
	}
}

% ==============================================================================
% \Mts* - Ausgabe als Symbol und Eintrag und Verweis ins Symbolverzeichnis
\newglossaryentry{Glo-Folgenrelationen}{
	name  ={Folgenrelationen},% ================================================
	sort  ={= 6 0 0},
	type  ={symbols},
	description={
		Im Folgenden seien $F$ und $G$ \Folgen\ in den \metasprachlichenAusdruecken\ $F \BspOpB G$.
	}
}

\newcommand*              {\MtsSubseq}[1][]{\glstext[#1]{MtsSubseq}}
\newglossaryentry          {MtsSubseq}{
	name  ={\ensuremath{\RawMtsSubseq}},
	sort  ={= 6 1 1},
	type  ={symbols},
	description={
		Eine \Folgenrelation:~ $F$ ist \defFt{\echteTeilfolge\ von} $G$;
		es kann \emph{keine} \Gleichheit\ bestehen.
	}
}

\newcommand*              {\MtsSubseqEq}[1][]{\glstext[#1]{MtsSubseqEq}}
\newglossaryentry          {MtsSubseqEq}{
	name  ={\ensuremath{\RawMtsSubseqEq}},
	sort  ={= 6 1 2},
	type  ={symbols},
	description={
		Eine \Folgenrelation:~ $F$ ist \defFt{\Teilfolge\ von} $G$;
		es \emph{kann} \Gleichheit\ bestehen.
	}
}

\newcommand*              {\MtsSubseqN}[1][]{\glstext[#1]{MtsSubseqN}}
\newglossaryentry          {MtsSubseqN}{
	name  ={\ensuremath{\RawMtsSubseqN}},
	sort  ={= 6 1 3},
	type  ={symbols},
	description={
		Eine \Folgenrelation:~ $F$ ist \defFt{keine \echteTeilfolge\ von} $G$;
		es \emph{kann} aber \Gleichheit\ bestehen.\\
		Die \Negation\ von \MtsSubseq.
	}
}

\newcommand*              {\MtsSubseqEqN}[1][]{\glstext[#1]{MtsSubseqEqN}}
\newglossaryentry          {MtsSubseqEqN}{
	name  ={\ensuremath{\RawMtsSubseqEqN}},
	sort  ={= 6 1 4},
	type  ={symbols},
	description={
		Eine \Folgenrelation:~ $F$ ist \defFt{keine \Teilfolge\ von} $G$;
		es kann auch \emph{keine} \Gleichheit\ bestehen.\\
		Die \Negation\ von \MtsSubseqEq.
	}
}

\newcommand*              {\MtsSupseq}[1][]{\glstext[#1]{MtsSupseq}}
\newglossaryentry          {MtsSupseq}{
	name  ={\ensuremath{\RawMtsSupseq}},
	sort  ={= 6 2 1},
	type  ={symbols},
	description={
		Eine \Folgenrelation:~ $F$ ist \defFt{\echteOberfolge\ von} $G$;
		es kann \emph{keine} \Gleichheit\ bestehen.\\
		Die \Umkehrrelation\ von \MtsSubseq.
	}
}

\newcommand*              {\MtsSupseqEq}[1][]{\glstext[#1]{MtsSupseqEq}}
\newglossaryentry          {MtsSupseqEq}{
	name  ={\ensuremath{\RawMtsSupseqEq}},
	sort  ={= 6 2 2},
	type  ={symbols},
	description={
		Eine \Folgenrelation:~ $F$ ist \defFt{\Oberfolge\ von} $G$;
		es \emph{kann} \Gleichheit\ bestehen.\\
		Die \Umkehrrelation\ von \MtsSubseqEq.
	}
}

\newcommand*              {\MtsSupseqN}[1][]{\glstext[#1]{MtsSupseqN}}
\newglossaryentry          {MtsSupseqN}{
	name  ={\ensuremath{\RawMtsSupseqN}},
	sort  ={= 6 2 3},
	type  ={symbols},
	description={
		Eine \Folgenrelation:~ $F$ ist \defFt{keine \echteOberfolge\ von} $G$;
		es \emph{kann} aber \Gleichheit\ bestehen.\\
		Die \Negation\ von \MtsSupseq; gleichzeitig die \Umkehrrelation\ von \MtsSubseq.
	}
}

\newcommand*              {\MtsSupseqEqN}[1][]{\glstext[#1]{MtsSupseqEqN}}
\newglossaryentry          {MtsSupseqEqN}{
	name  ={\ensuremath{\RawMtsSupseqEqN}},
	sort  ={= 6 2 4},
	type  ={symbols},
	description={
		Eine \Folgenrelation:~ $F$ ist \defFt{keine \Oberfolge\ von} $G$;
		es kann auch \emph{keine} \Gleichheit\ bestehen.\\
		Die \Negation\ von \MtsSupseqEq; gleichzeitig die \Umkehrrelation\ von \MtsSubseqEq.
	}
}

% ==============================================================================
% \Ojk* - Ausgabe als Symbol und Eintrag und Verweis ins Symbolverzeichnis
\newglossaryentry{Glo-Objektsymbole}{
	name  ={Junktoren},% =========================================================
	sort  ={= 7 0 0},
	type  ={symbols},
	description={
		\footnote{In diesem Dokument \aussagenlogischeKonstante, -\aRelationen\ und -\aOperationen, \textdh\ \Objektkonstante, -\Orelationen\ und -\Ooperationen.}
		Im Folgenden seien $A$ und $B$ \logischeAussagen\ in den \logischenAusdruecken\ $\BspOpU A$ \textbzw $A \BspOpB B$.
	}
}

\newcommand*              {\OjkFalse}[1][]{\glstext[#1]{OjkFalse}}
\newglossaryentry          {OjkFalse}{
	name  ={\ensuremath{\RawOjkFalse}},
	sort  ={= 7 0 1},
	see   ={MtsFalse},
	type  ={symbols},
	description={
		Ein $0$-\stelliger\ \Junktor, \textdh\ eine \aussagenlogischeKonstante\ mit dem \Wahrheitswert\ \TxtFalse.
	}
}

\newcommand*              {\OjkTrue}[1][]{\glstext[#1]{OjkTrue}}
\newglossaryentry          {OjkTrue}{
	name  ={\ensuremath{\RawOjkTrue}},
	sort  ={= 7 0 2},
	see   ={MtsTrue},
	type  ={symbols},
	description={
		Ein $0$-\stelliger\ \Junktor, \textdh\ eine \aussagenlogischeKonstante\ mit dem \Wahrheitswert\ \TxtTrue.
	}
}

\newcommand*              {\OjkNot}[1][]{\glstext[#1]{OjkNot}}
\newglossaryentry          {OjkNot}{
	name  ={\ensuremath{\RawOjkNot}},
	sort  ={= 7 1 1},
	see   ={MtsNot},
	type  ={symbols},
	description={
		Ein \unaererJunktor:~ \defFt{nicht} $A$.
	}
}

\newcommand*              {\OjkAnd}[1][]{\glstext[#1]{OjkAnd}}
\newglossaryentry          {OjkAnd}{
	name  ={\ensuremath{\RawOjkAnd}},
	sort  ={= 7 1 2},
	see   ={OjkNand,MtsAnd},
	type  ={symbols},
	description={
		Ein \binaererJunktor:~ $A$ \defFt{und} $B$.
	}
}

\newcommand*              {\OjkOr}[1][]{\glstext[#1]{OjkOr}}
\newglossaryentry          {OjkOr}{
	name  ={\ensuremath{\RawOjkOr}},
	sort  ={= 7 1 3},
	see   ={OjkNor,OjkXor,MtsOr},
	type  ={symbols},
	description={
		Ein \binaererJunktor:~ $A$ \defFt{oder} $B$.
	}
}

\newcommand*              {\OjkImp}[1][]{\glstext[#1]{OjkImp}}
\newglossaryentry          {OjkImp}{
	name  ={\ensuremath{\RawOjkImp}},
	sort  ={= 7 2 1},
	see   ={MtsImp},
	type  ={symbols},
	description={
		Ein \binaererJunktor:~ wenn $A$ \defFt{dann} $B$.
	}
}

\newcommand*              {\OjkRep}[1][]{\glstext[#1]{OjkRep}}
\newglossaryentry          {OjkRep}{
	name  ={\ensuremath{\RawOjkRep}},
	sort  ={= 7 2 2},
	see   ={MtsRep},
	type  ={symbols},
	description={
		Ein \binaererJunktor:~ $A$ \defFt{wenn} $B$.
	}
}

\newcommand*              {\OjkEquiv}[1][]{\glstext[#1]{OjkEquiv}}
\newglossaryentry          {OjkEquiv}{
	name  ={\ensuremath{\RawOjkEquiv}},
	sort  ={= 7 2 3},
	see   ={MtsEquiv},
	type  ={symbols},
	description={
		Ein \binaererJunktor:~ $A$ genau \defFt{dann wenn} $B$.
	}
}

\newcommand*              {\OjkNand}[1][]{\glstext[#1]{OjkNand}}
\newglossaryentry          {OjkNand}{
	name  ={\ensuremath{\RawOjkNand}},
	sort  ={= 7 3 1},
	see   ={OjkAnd},
	type  ={symbols},
	description={
		Ein \binaererJunktor:~ \defFt{nicht} ($A$ \defFt{und} $B$)\alternativi{sowohl~~~als auch}.
	}
}

\newcommand*              {\OjkNor}[1][]{\glstext[#1]{OjkNor}}
\newglossaryentry          {OjkNor}{
	name  ={\ensuremath{\RawOjkNor}},
	sort  ={= 7 3 2},
	see   ={OjkOr,OjkXor},
	type  ={symbols},
	description={
		Ein \binaererJunktor:~ \defFt{nicht} ($A$ \defFt{oder} $B$)\alternativi{weder~~~noch}.
	}
}

\newcommand*              {\OjkXor}[1][]{\glstext[#1]{OjkXor}}
\newglossaryentry          {OjkXor}{
	name  ={\ensuremath{\RawOjkXor}},
	sort  ={= 7 3 3},
	see   ={OjkOr,OjkNor},
	type  ={symbols},
	description={
		Ein \binaererJunktor:~ \defFt{entweder} $A$ \defFt{oder} $B$.
	}
}

\newcommand*              {\OjkEq}[1][]{\glstext[#1]{OjkEq}}
%ToDo prüfen
\newglossaryentry          {OjkEq}{
	name  ={\ensuremath{\RawOjkEq}},
	sort  ={= 7 4 1},
	see   ={MtsEq},
	type  ={symbols},
	description={
		Logische \Gleichheit:~ $A$ ist \defFt{gleich} $B$.
	}
}

\newcommand*              {\OjkEqN}[1][]{\glstext[#1]{OjkEqN}}
%ToDo prüfen
\newglossaryentry          {OjkEqN}{
	name  ={\ensuremath{\RawOjkEqN}},
	sort  ={= 7 4 2},
	see   ={MtsEqN},
	type  ={symbols},
	description={
		Logische \Ungleichheit:~ $A$ ist \defFt{ungleich} $B$.
	}
}

% ==============================================================================
% \Mts* - Ausgabe als Symbol und Eintrag und Verweis ins Symbolverzeichnis
% \Ojk* - Ausgabe als Symbol und Eintrag und Verweis ins Symbolverzeichnis
\newglossaryentry{Glo-Quantoren}{
	name  ={Quantoren},% =======================================================
	sort  ={= 8 0 0},
	type  ={symbols},
	description={
		$x$ steht jeweils für eine \metasprachlicheV\ \textbzw\ \logischeV\ \Variable\ und $A$ für eine \Aussage\ \textbzw\ \Formel.
	}
}

\newcommand*              {\MtsForall}[1][]{\glstext[#1]{MtsForall}}
\newglossaryentry          {MtsForall}{
	name  ={\ensuremath{\RawMtsForall}},
	sort  ={= 8 1 1},
	see   ={OjkForall},
	type  ={symbols},
	description={
		Ein \metasprachlicherQuantor: \defFt{für alle} $x$ \defFt{gilt} $A$.
	}
}

\newcommand*              {\MtsExists}[1][]{\glstext[#1]{MtsExists}}
\newglossaryentry          {MtsExists}{
	name  ={\ensuremath{\RawMtsExists}},
	sort  ={= 8 1 2},
	see   ={OjkExists},
	type  ={symbols},
	description={
		Ein \metasprachlicherQuantor: \defFt{es gibt ein} $x$ \defFt{so dass} $A$.
	}
}

\newcommand*              {\MtsExione}[1][]{\glstext[#1]{MtsExione}}
\newglossaryentry          {MtsExione}{
	name  ={\ensuremath{\RawMtsExione}},
	sort  ={= 8 1 3},
	see   ={OjkExione},
	type  ={symbols},
	description={
		Ein \metasprachlicherQuantor: \defFt{es gibt genau ein} $x$ \defFt{so dass} $A$.
	}
}

\newcommand*              {\OjkForall}[1][]{\glstext[#1]{OjkForall}}
\newglossaryentry          {OjkForall}{
	name  ={\ensuremath{\RawOjkForall}},
	sort  ={= 8 2 1},
	see   ={MtsForall},
	type  ={symbols},
	description={
		Ein \logischerQuantor: \defFt{für alle} $x$ \defFt{gilt} $A$.
	}
}

\newcommand*              {\OjkExists}[1][]{\glstext[#1]{OjkExists}}
\newglossaryentry          {OjkExists}{
	name  ={\ensuremath{\RawOjkExists}},
	sort  ={= 8 2 2},
	see   ={MtsExists},
	type  ={symbols},
	description={
		Ein \logischerQuantor: \defFt{es gibt ein} $x$ \defFt{so dass} $A$.
	}
}

\newcommand*              {\OjkExione}[1][]{\glstext[#1]{OjkExione}}
\newglossaryentry          {OjkExione}{
	name  ={\ensuremath{\RawOjkExione}},
	sort  ={= 8 2 3},
	see   ={MtsExione},
	type  ={symbols},
	description={
		Ein \logischerQuantor: \defFt{es gibt genau ein} $x$ \defFt{so dass} $A$.
	}
}

% ==============================================================================
% \sym* - Ausgabe als geklammertes Symbol und Eintrag ins Symbolverzeichnis
% \gls* - wie \sym*            und zusätzlich Verweis ins Symbolverzeichnis
% \tag* - Tag in einer Formel setzen      und Eintrag ins Symbolverzeichnis
% Verweise als geklammertes Symbol auf die Formel mit dem Tag:
%   \ref    {def-*} -->  \*
%   \eqref  {def-*} --> (\*)
%   \vreffor{def-*} --> (\*) auf Seite <n>
\newglossaryentry{Glo-Schlussregeln}{
	name       =     {Schlussregeln},% =========================================
	sort    ={= 9},
	type       ={symbols},
	description={}
}

\newcommand*    {\AR}{\ensuremath  {\text{AR}}}
\newcommand* {\glsAR} [1][]{\glstext [#1]{AR}}
\newcommand* {\symAR} [1][]{\glsuserv[#1]{AR}}
\newcommand* {\tagAR}      {\glsTag      {AR}}
\newglossaryentry{AR}{
	name      ={(\AR)},
	user5      ={\AR},
	sort    ={= 9 AR},
	user6      ={},% Dummy für \glsTag
	type       ={symbols},
	description={
		Eine \Schlussregel: \Anfangsregel.
	}
}

\newcommand*    {\FS}{\ensuremath  {\text{FS}}}
\newcommand* {\glsFS} [1][]{\glstext [#1]{FS}}
\newcommand* {\symFS} [1][]{\glsuserv[#1]{FS}}
\newcommand* {\tagFS}      {\glsTag      {FS}}
\newglossaryentry{FS}{
	name      ={(\FS)},
	user5      ={\FS},
	sort    ={= 9 FS},
	user6      ={},% Dummy für \glsTag
	type       ={symbols},
	description={
		Eine \Schlussregel: \formalerSatz.
	}
}

\newcommand*    {\MR}{\ensuremath  {\text{MR}}}
\newcommand* {\glsMR} [1][]{\glstext [#1]{MR}}
\newcommand* {\symMR} [1][]{\glsuserv[#1]{MR}}
\newcommand* {\tagMR}      {\glsTag      {MR}}
\newglossaryentry{MR}{
	name      ={(\MR)},
	user5      ={\MR},
	sort    ={= 9 MR},
	user6      ={},% Dummy für \glsTag
	type       ={symbols},
	description={
		Eine \Schlussregel: \Monotonieregel.
	}
}

\newcommand*    {\SR}{\ensuremath  {\text{SR}}}
\newcommand* {\glsSR} [1][]{\glstext [#1]{SR}}
\newcommand* {\symSR} [1][]{\glsuserv[#1]{SR}}
\newcommand* {\tagSR}      {\glsTag      {SR}}
\newglossaryentry{SR}{
	name      ={(\SR)},
	user5      ={\SR},
	sort    ={= 9 SR},
	user6      ={},% Dummy für \glsTag
	type       ={symbols},
	description={
		Eine \Schlussregel: \Schnittregel.
	}
}

\newcommand*    {\TR}{\ensuremath  {\text{TR}}}
\newcommand* {\glsTR} [1][]{\glstext [#1]{TR}}
\newcommand* {\symTR} [1][]{\glsuserv[#1]{TR}}
\newcommand* {\tagTR}      {\glsTag      {TR}}
%ToDo prüfen
\newglossaryentry{TR}{
	name      ={(\TR)},
	user5      ={\TR},
	sort    ={= 9 TR},
	user6      ={},% Dummy für \glsTag
	type       ={symbols},
	description={
		Eine \Schlussregel: \Abtrennungsregel.
	}
}

\newcommand*    {\andB}{\ensuremath{\RawOjkAnd\text{B}}}
\newcommand* {\glsandB}[1][]{\glstext [#1]{andB}}
\newcommand* {\symandB}[1][]{\glsuserv[#1]{andB}}
\newcommand* {\tagandB}     {\glsTag      {andB}}
%ToDo prüfen
\newglossaryentry{andB}{
	name      ={(\andB)},
	user5      ={\andB},
	user6      ={},% Dummy für \glsTag
	sort       ={= 9 1 1},
	type       ={symbols},
	description={
		Eine \Schlussregel: Beseitigung von \OjkAnd.
	}
}

\newcommand*    {\andE}{\ensuremath{\RawOjkAnd\text{E}}}
\newcommand* {\glsandE}[1][]{\glstext [#1]{andE}}
\newcommand* {\symandE}[1][]{\glsuserv[#1]{andE}}
\newcommand* {\tagandE}     {\glsTag      {andE}}
\newglossaryentry{andE}{
	name      ={(\andE)},
	user5      ={\andE},
	user6      ={},% Dummy für \glsTag
	sort       ={= 9 1 2},
	type       ={symbols},
	description={
		Eine \Schlussregel: Einführung von \OjkAnd.
	}
}

\newcommand*    {\orB}{\ensuremath{\RawOjkOr\text{B}}}
\newcommand* {\glsorB}[1][]{\glstext [#1]{orB}}
\newcommand* {\symorB}[1][]{\glsuserv[#1]{orB}}
\newcommand* {\tagorB}     {\glsTag      {orB}}
\newglossaryentry{orB}{
	name      ={(\orB)},
	user5      ={\orB},
	user6      ={},% Dummy für \glsTag
	sort       ={= 9 2 1},
	type       ={symbols},
	description={
		Eine \Schlussregel: Beseitigung von \OjkOr.
	}
}

\newcommand*    {\orE}{\ensuremath{\RawOjkOr\text{E}}}
\newcommand* {\glsorE}[1][]{\glstext [#1]{orE}}
\newcommand* {\symorE}[1][]{\glsuserv[#1]{orE}}
\newcommand* {\tagorE}     {\glsTag      {orE}}
\newglossaryentry{orE}{
	name      ={(\orE)},
	user5      ={\orE},
	user6      ={},% Dummy für \glsTag
	sort       ={= 9 2 2},
	type       ={symbols},
	description={
		Eine \Schlussregel: Einführung von \OjkOr.
	}
}

\newcommand*    {\impB}{\ensuremath{\RawOjkImp\text{B}}}
\newcommand* {\glsimpB}[1][]{\glstext [#1]{impB}}
\newcommand* {\symimpB}[1][]{\glsuserv[#1]{impB}}
\newcommand* {\tagimpB}     {\glsTag      {impB}}
\newglossaryentry{impB}{
	name      ={(\impB)},
	user5      ={\impB},
	user6      ={},% Dummy für \glsTag
	sort       ={= 9 3 1},
	type       ={symbols},
	description={
		Eine \Schlussregel: Beseitigung von \OjkImp.
	}
}

\newcommand*    {\impE}{\ensuremath{\RawOjkImp\text{E}}}
\newcommand* {\glsimpE}[1][]{\glstext [#1]{impE}}
\newcommand* {\symimpE}[1][]{\glsuserv[#1]{impE}}
\newcommand* {\tagimpE}     {\glsTag      {impE}}
\newglossaryentry{impE}{
	name      ={(\impE)},
	user5      ={\impE},
	user6      ={},% Dummy für \glsTag
	sort       ={= 9 3 2},
	type       ={symbols},
	description={
		Eine \Schlussregel: Einführung von \OjkImp.
	}
}

\newcommand*    {\nota}{\ensuremath{\RawOjkNot\text{1}}}
\newcommand* {\glsnota}[1][]{\glstext [#1]{nota}}
\newcommand* {\symnota}[1][]{\glsuserv[#1]{nota}}
\newcommand* {\tagnota}     {\glsTag      {nota}}
\newglossaryentry{nota}{
	name      ={(\nota)},
	user5      ={\nota},
	user6      ={},% Dummy für \glsTag
	sort       ={= 9 4 1},
	type       ={symbols},
	description={
		Eine \Schlussregel: Einführung/Beseitigung von \OjkNot\ Teil 1.
	}
}

\newcommand*    {\notb}{\ensuremath{\RawOjkNot\text{2}}}
\newcommand* {\glsnotb}[1][]{\glstext [#1]{notb}}
\newcommand* {\symnotb}[1][]{\glsuserv[#1]{notb}}
\newcommand* {\tagnotb}     {\glsTag      {notb}}
\newglossaryentry{notb}{
	name      ={(\notb)},
	user5      ={\notb},
	user6      ={},% Dummy für \glsTag
	sort       ={= 9 4 2},
	type       ={symbols},
	description={
		Eine \Schlussregel: Einführung/Beseitigung von \OjkNot\ Teil 2.
	}
}

\newcommand*    {\notc}{\ensuremath{\RawOjkNot\text{3}}}
\newcommand* {\glsnotc}[1][]{\glstext [#1]{notc}}
\newcommand* {\symnotc}[1][]{\glsuserv[#1]{notc}}
\newcommand* {\tagnotc}     {\glsTag      {notc}}
\newglossaryentry{notc}{
	name      ={(\notc)},
	user5      ={\notc},
	user6      ={},% Dummy für \glsTag
	sort       ={= 9 4 3},
	type       ={symbols},
	description={
		Eine \Schlussregel: Beweistechnik „Indirekter \Beweis“.
	}
}

\newcommand*    {\notd}{\ensuremath{\RawOjkNot\text{4}}}
\newcommand* {\glsnotd}[1][]{\glstext [#1]{notd}}
\newcommand* {\symnotd}[1][]{\glsuserv[#1]{notd}}
\newcommand* {\tagnotd}     {\glsTag      {notd}}
\newglossaryentry{notd}{
	name      ={(\notd)},
	user5      ={\notd},
	user6      ={},% Dummy für \glsTag
	sort       ={= 9 4 4},
	type       ={symbols},
	description={
		Eine \Schlussregel: Reductio ad absurdum (Indirekter \Beweis).
	}
}

\newcommand*    {\eqB}{\ensuremath{\RawOjkEq\text{B}}}
\newcommand* {\glseqB}[1][]{\glstext [#1]{eqB}}
\newcommand* {\symeqB}[1][]{\glsuserv[#1]{eqB}}
\newcommand* {\tageqB}     {\glsTag      {eqB}}
\newglossaryentry{eqB}{
	name      ={(\eqB)},
	user5      ={\eqB},
	user6      ={},% Dummy für \glsTag
	sort       ={= 9 5 1},
	type       ={symbols},
	description={
		Eine \Schlussregel: Beseitigung von \OjkEq.
	}
}

\newcommand*    {\eqE}{\ensuremath{\RawOjkEq\text{E}}}
\newcommand* {\glseqE}[1][]{\glstext [#1]{eqE}}
\newcommand* {\symeqE}[1][]{\glsuserv[#1]{eqE}}
\newcommand* {\tageqE}     {\glsTag      {eqE}}
\newglossaryentry{eqE}{
	name      ={(\eqE)},
	user5      ={\eqE},
	user6      ={},% Dummy für \glsTag
	sort       ={= 9 5 2},
	type       ={symbols},
	description={
		Eine \Schlussregel: Einführung von \OjkEq.
	}
}

% ### Symbolverzeichnis und Index ##############################################
% Anmerkung:
%   Eigentlich gehören die weiteren aufgeführten Symbole alle zur Metasprache.
%   Solche, die zur Bildung von aussagen- und prädikatenlogischen Formeln
%   dienen, sind trotzdem mit 'Ojk' statt 'Mts' markiert.

% ==============================================================================
\newglossaryentry{Glo-TextSymbole}{
	name       ={Text-Symbole},% ===============================================
	sort       ={A},
	type       ={symbols},
	description={
		Die folgenden Symbole sind alphabetisch geordnet und auch im Index aufgeführt.
		$\square$ dient zur Verdeutlichung, an welche Stelle die Indizes gehören.
	}
}

% ==============================================================================
% \StrMtsIdx* - Ausgabe als Text-Symbol und Eintrag und Verweis ins Symbolverzeichnis
% \Mts* - Ausgabe als Text-Symbol und Eintrag und Verweis ins Symbolverzeichnis
% Operationen mit Namen (Buchstaben) ===========================================

\newcommand*             {\LtrMtsIdxEndlich}       {e}% nur endliche Elemente
\newcommand*                {\MtsIdxEndlich} [1][]{\glstext[#1]{MtsIdxEndlich}}
\newglossaryentry            {MtsIdxEndlich}{
	text    ={\ensuremath{\RawMtsIdxEndlich}},
	name    ={\ensuremath{\square_{\RawMtsIdxEndlich}} \addIdx[
		name={\ensuremath{\RawMtsIdxEndlich}},
		sort={e Ind}]                                          {MtsIdxEndlich}},
	sort    ={e Ind},%       \LtrIdxEndlich Index
	type    ={symbols},
	description={
		Eine \Operation\ mittels eines Index:
		\[
			X_{\MtsIdxEndlich} \MtsDefEq
			\begin{cases}
				\MengeDef{M  \MtsIn X}{|M|                \quad \MtsIn \MtsINo}
				& \text{, für eine \Menge\ $X$ von \Mengen}     \\
				\MengeDef{R\;\MtsIn X}{|R_{\MtsIdxGraph}| \quad \MtsIn \MtsINo}
				& \text{, für eine \Menge\ $X$ von \Relationen} \\
				\MengeDef{F\;\MtsIn X}{\MtsLen(F)               \MtsIn \MtsINo}
				& \text{, für eine \Menge\ $X$ von \Folgen}
			\end{cases}
		\]
	}
}

\newcommand*             {\LtrMtsIdxGraph}         {g}% Graph von
\newcommand*                {\MtsIdxGraph} [1][]{\glstext[#1]{MtsIdxGraph}}
\newglossaryentry            {MtsIdxGraph}{
	text    ={\ensuremath{\RawMtsIdxGraph}},
	name    ={\ensuremath{\square_{\RawMtsIdxGraph}} \addIdx[
		name={\ensuremath{\RawMtsIdxGraph}},
		sort={g Ind}]                                        {MtsIdxGraph}},
	sort    ={g Ind},%    \LtrMtsIdxGraph Index
	type    ={symbols},
	description={
		Eine \Operation\ mittels eines Index:
		$X_{\MtsIdxGraph} \MtsDefEq \MtsGraph(X)$ für \Funktionen\ und \Relationen\ $X$.
	}
}

\newcommand*             {\LtrMtsIdxPolnisch}      {p}% in Polnischer Notation
\newcommand*                {\MtsIdxPolnisch} [1][]{\glstext[#1]{MtsIdxPolnisch}}
\newglossaryentry            {MtsIdxPolnisch}{
	text    ={\ensuremath{\RawMtsIdxPolnisch}},
	name    ={\ensuremath{\square^{\RawMtsIdxPolnisch}} \addIdx[
		name={\ensuremath{\RawMtsIdxPolnisch}},
		sort={p Ind}]                                           {MtsIdxPolnisch}},
	sort    ={p Ind},%    \LtrMtsIdxPolnisch Index
	type    ={symbols},
	description={
		Eine \Operation\ mittels eines Index: Für eine \Menge\ $L$ von \Formeln\ und eine \Formel\ $\alpha$ ist\\
		$L^{\MtsIdxPolnisch} \MtsDefEq \MengeDef{\alpha^{\MtsIdxPolnisch}}{\alpha \MtsIn L}$.
		mit $\alpha^{\MtsIdxPolnisch} \MtsDefEq ( \alpha$ umgewandelt in \PolnischeNotation ).
	}
}

\newcommand*             {\StrMtsDb}               {dom}% Definitionsbereich [domain]
\newcommand*                {\MtsDb}[1][]{\glstext[#1]{MtsDb}}
\newglossaryentry            {MtsDb}{
	text    ={\ensuremath{\RawMtsDb}},
	name    ={\ensuremath{\RawMtsDb} \addIdx[
		name={\ensuremath{\RawMtsDb}},
		sort={dom}]                                   {MtsDb}},
	sort    ={dom},%      \StrMtsDb
	type    ={symbols},
	description={
		Für eine \Funktion\ \FunktionDef{f}{A}{B} ist $\MtsDb(f) \MtsDefEq A$, der \Definitionsbereich\ von $f$.
	}
}

\newcommand*             {\LtrMtsFol}              {F}%  Folgenmenge
\newcommand*                {\MtsFol}[1][]{\glstext[#1]{MtsFol}}
\newglossaryentry            {MtsFol}{
	text    ={\ensuremath{\RawMtsFol}},
	name    ={\ensuremath{\RawMtsFol} \addIdx[
		name={\ensuremath{\RawMtsFol}},
		sort={F}]                                      {MtsFol}},
	sort    ={F},%        \LtrMtsFol
	see     ={MtsFolf},
	type    ={symbols},
	description={
		$\MtsFol(M) \MtsDefEq \MengeDef{F}{F \text{ ist \Folge\ über } M}$.
	}
}

\newcommand*                {\MtsFolf}[1][]{\glstext[#1]{MtsFolf}}
\newglossaryentry            {MtsFolf}{
	text    ={\ensuremath{\RawMtsFolf}},
	name    ={\ensuremath{\RawMtsFolf} \addIdx[
		name={\ensuremath{\RawMtsFolf}},
		sort={F e}]                                     {MtsFolf}},
	sort    ={F e},%      \LtrMtsFol \LtrIdxEndlich
	see     ={MtsFol,Folgenmenge},
	type    ={symbols},
	description={
		$\MtsFol(M) \MtsDefEq \MengeDef{F \MtsIn \MtsFol(M)}{\MtsLen(F) \MtsIn \MtsINo}$.
	}
}

\newcommand*             {\StrMtsGraph}            {graph}% Graph; Funktionen/Relationen
\newcommand*                {\MtsGraph}[1][]{\glstext[#1]{MtsGraph}}
\newglossaryentry            {MtsGraph}{
	text    ={\ensuremath{\RawMtsGraph}},
	name    ={\ensuremath{\RawMtsGraph} \addIdx[
		name={\ensuremath{\RawMtsGraph}},
		sort={graph}]                                    {MtsGraph}},
	sort    ={graph},%    \StrMtsGraph
	see     ={Graph},
	type    ={symbols},
	description={
		Für eine \Relation\ $R = (G, A_1, \dots, A_n)$ ist $\MtsGraph(R) \MtsDefEq G$.\\
		Für eine \Funktion\ \FunktionDef{f}{A}{B} ist $\MtsGraph(f) \MtsDefEq \MengeDef{(a,f(a))}{a \MtsIn A}$.
	}
}

\newcommand*             {\StrMtsLen}              {len}% Länge [length] (Tupel)
\newcommand*                {\MtsLen}[1][]{\glstext[#1]{MtsLen}}
\newglossaryentry            {MtsLen}{
	text    ={\ensuremath{\RawMtsLen}},
	name    ={\ensuremath{\RawMtsLen} \addIdx[
		name={\ensuremath{\RawMtsLen}},
		sort={len}]                                    {MtsLen}},
	sort    ={len},%      \StrMtsLen
	type    ={symbols},
	description={
		$\MtsLen(\vec{a}) \MtsDefEq$ Anzahl der \Komponenten\ einer endlichen \Folge\, \textdh\ eines \Tupels\ $\vec{a}$
	}
}

\newcommand*             {\LtrMtsPot}              {P}%  Potenzmenge
\newcommand*                {\MtsPot}[1][]{\glstext[#1]{MtsPot}}
\newglossaryentry            {MtsPot}{
	text    ={\ensuremath{\RawMtsPot}},
	name    ={\ensuremath{\RawMtsPot} \addIdx[
		name={\ensuremath{\RawMtsPot}},
		sort={P}]                                      {MtsPot}},
	sort    ={P},%        \LtrMtsPot
	see     ={MtsPotf},
	type    ={symbols},
	description={
		$\MtsPot(M) \MtsDefEq \MengeDef{N}{N \MtsSubsetEq M}$, die \Potenzmenge\ einer \Menge\ $M$.
	}
}

\newcommand*                {\MtsPotf}[1][]{\glstext[#1]{MtsPotf}}
\newglossaryentry            {MtsPotf}{
	text    ={\ensuremath{\RawMtsPotf}},
	name    ={\ensuremath{\RawMtsPotf} \addIdx[
		name={\ensuremath{\RawMtsPotf}},
		sort={P e}]                                     {MtsPotf}},
	sort    ={P e},%      \LtrMtsPot \LtrIdxEndlich
	type    ={symbols},
	description={
		$\MtsPot(M) \MtsDefEq \MengeDef{N \MtsIn \MtsPot(M)}{|N| \MtsIn \MtsINo}$.
	}
}

\newcommand*             {\StrMtsQb}               {src}% Quellbereich [source]
\newcommand*                {\MtsQb}[1][]{\glstext[#1]{MtsQb}}
\newglossaryentry            {MtsQb}{
	text    ={\ensuremath{\RawMtsQb}},
	name    ={\ensuremath{\RawMtsQb} \addIdx[
		name={\ensuremath{\RawMtsQb}},
		sort={src}]                                   {MtsQb}},
	sort    ={src},%      \StrMtsQb
	type    ={symbols},
	description={
		Für eine \Funktion\ \FunktionDef{f}{A}{B} ist $\MtsQb(f) \MtsDefEq \MengeDef{a \in A}{f(a) \text{ existiert}}$ der \Quellbereich\ von $f$.
	}
}

\newcommand*             {\LtrMtsRel}              {R}% Menge der Relationen
\newcommand*                {\MtsRel}[1][]{\glstext[#1]{MtsRel}}
\newglossaryentry            {MtsRel}{
	text    ={\ensuremath{\RawMtsRel}},
	name    ={\ensuremath{\RawMtsRel} \addIdx[
		name={\ensuremath{\RawMtsRel}},
		sort={R}]                                      {MtsRel}},
	sort    ={R},%        \LtrMtsRel
	see     ={MtsRelf,Relation},
	type    ={symbols},
	description={
		\Menge\ der \binaeren\ \Relationen. --- noch prüfen%ToDo prüfen
	}
}

\newcommand*                {\MtsRelf}[1][]{\glstext[#1]{MtsRelf}}%
\newglossaryentry            {MtsRelf}{
	text    ={\ensuremath{\RawMtsRelf}},
	name    ={\ensuremath{\RawMtsRelf} \addIdx[
		name={\ensuremath{\RawMtsRelf}},
		sort={R e}]                                     {MtsRelf}},
	sort    ={R e},%      \LtrMtsRel \LtrIdxEndlich
	type    ={symbols},
	description={
		$\MtsRelf(M) \MtsDefEq \MengeDef{R \MtsIn \MtsRel(M)}{|R| \MtsIn \MtsINo}$
	}
}

\newcommand*             {\StrMtsSet}              {set}% Komponentenmenge (Tupel/Folge)
\newcommand*                {\MtsSet}[1][]{\glstext[#1]{MtsSet}}
\newglossaryentry            {MtsSet}{
	text    ={\ensuremath{\RawMtsSet}},
	name    ={\ensuremath{\RawMtsSet} \addIdx[
		name={\ensuremath{\RawMtsSet} (Menge)},
		sort={Set}]                                    {MtsSet}},
	sort    ={Set},%      \StrMtsSet
	see     ={Komponentenmenge,Folge,Tupel},
	type    ={symbols},
	description={
		$\MtsSet(\vec{a}) \MtsDefEq \MengeDef{a}{a \MtsSeqIn \vec{a}}$.
	}
}

\newcommand*             {\StrMtsStel}             {stel}% [Stel]ligkeit Funktionen/Relationen
\newcommand*                {\MtsStelF}[1][]{\glstext[#1]{MtsStelF}}
%ToDo prüfen
\newglossaryentry            {MtsStelF}{
	text    ={\ensuremath{\RawMtsStelF}},
	name    ={\ensuremath{\RawMtsStelF} \addIdx[
		name={\ensuremath{\RawMtsStelF}},
		sort={stel f}]                                   {MtsStelF}},
	sort    ={stel f},%   \StrMtsStel f
	see     ={Stelligkeit,Funktion},
	type    ={symbols},
	description={
		$\MtsStelF(f) \MtsDefEq n$ für $\FunktionDef{f}{A_1 \MtsTimes \dots \MtsTimes A_n}{B}$.
	}
}

\newcommand*                {\MtsStelR}[1][]{\glstext[#1]{MtsStelR}}
\newglossaryentry            {MtsStelR}{
	text    ={\ensuremath{\RawMtsStelR}},
	name    ={\ensuremath{\RawMtsStelR} \addIdx[
		name={\ensuremath{\RawMtsStelR}},
		sort={stel r}]                                   {MtsStelR}},
	sort    ={stel r},%   \StrMtsStel r
	see     ={Stelligkeit,Relation},
	type    ={symbols},
	description={
		$\MtsStelR(R) \MtsDefEq n$ für $R \MtsSubsetEq A_1 \MtsTimes \dots \MtsTimes A_n$.
	}
}

\newcommand*             {\StrMtsTraeger}          {car}% Trägermenge [carrier] (Relation)
\newcommand*                {\MtsTraeger}[1][]{\glstext[#1]{MtsTraeger}}
\newglossaryentry            {MtsTraeger}{
	text    ={\ensuremath{\RawMtsTraeger}},
	name    ={\ensuremath{\RawMtsTraeger} \addIdx[
		name={\ensuremath{\RawMtsTraeger}},
		sort={car}]                                        {MtsTraeger}},
	sort    ={car},%      \StrMtsTraeger
	see     ={Traegermenge},
	type    ={symbols},
	description={
		Für eine \Relation%
		\footnote{%
			\Funktionen\ sind spezielle \Relationen.
			Für eine \Funktion\ $\FunktionDef{f}{A_1 \MtsTimes \dots \MtsTimes A_n}{B}$ gilt demnach:
			\\$\MtsTraeger(f) \MtsDefEq A_1 \MtsTimes \dots \MtsTimes A_n \MtsTimes B$;
			\quad $\MtsTraeger_i(f) \MtsDefEq A_i$ für $1 \le i \le n$;
			\quad $\MtsTraeger_{n+1}(f) \MtsDefEq B$
		}
		$R = (G, A_1, \dots, A_n)$ ist
		$\MtsTraeger(R) \MtsDefEq A_1 \MtsTimes \dots \MtsTimes A_n$ und
		$\MtsTraeger_i(R) \MtsDefEq A_i$ für $1 \le i \le n$.
	}
}

\newcommand*             {\StrMtsWb}               {ran}% Wertebereich [range]
\newcommand*                {\MtsWb}[1][]{\glstext[#1]{MtsWb}}
\newglossaryentry            {MtsWb}{
	text    ={\ensuremath{\RawMtsWb}},
	name    ={\ensuremath{\RawMtsWb} \addIdx[
		name={\ensuremath{\RawMtsWb}},
		sort={ran}]                                   {MtsWb}},
	sort    ={ran},%      \StrMtsWb
	type    ={symbols},
	description={
		Für eine \Funktion\ \FunktionDef{f}{A}{B} ist $\MtsWb(f) \MtsDefEq \MengeDef{f(a)}{a \in A}$ der \Wertebereich\ von $f$.
	}
}

\newcommand*             {\StrMtsZb}               {tar}% Zielbereich [target]
\newcommand*                {\MtsZb}[1][]{\glstext[#1]{MtsZb}}
\newglossaryentry            {MtsZb}{
	text    ={\ensuremath{\RawMtsZb}},
	name    ={\ensuremath{\RawMtsZb} \addIdx[
		name={\ensuremath{\RawMtsZb}},
		sort={tar}]                                   {MtsZb}},
	sort    ={tar},%      \StrMtsZb
	type    ={symbols},
	description={
		Für eine \Funktion\ \FunktionDef{f}{A}{B} ist $\MtsZb(f) \MtsDefEq B$ der \Zielbereich\ von $f$.
	}
}

% ==============================================================================
% \Mts* - Ausgabe als Text-Symbol und Eintrag und Verweis ins Symbolverzeichnis
% Mengen und Elemente ==========================================================

\newcommand*             {\LtrMtsAxiom}            {X}%        A[x]iom
\newcommand*                {\MtsAxiom}[1][]{\glstext[#1]{MtsAxiom}}
\newglossaryentry            {MtsAxiom}{
	text    ={\ensuremath{\RawMtsAxiom}},
	name    ={\ensuremath{\RawMtsAxiom} \addIdx[
		name={\ensuremath{\RawMtsAxiom} (Element)},
		sort={X Ele}]                                    {MtsAxiom}},
	sort    ={X Ele},%    \LtrMtsAxiom   Element
	type    ={symbols},
	description={
		Ein \Axiom.
	}
}

\newcommand*                {\MtsAxiomSet}[1][]{\glstext[#1]{MtsAxiomSet}}
\newglossaryentry            {MtsAxiomSet}{
	text    ={\ensuremath{\RawMtsAxiomSet}},
	name    ={\ensuremath{\RawMtsAxiomSet} \addIdx[
		name={\ensuremath{\RawMtsAxiomSet} (Menge)},
		sort={X Men}]                                       {MtsAxiomSet}},
	sort    ={X Men},%    \LtrMtsAxiom      Menge
	type    ={symbols},
	description={
		Eine \Menge\ von \Axiomen.
	}
}

\newcommand*             {\LtrMtsBeweisschritt}    {b}%              Beweisschritt
\newcommand*                {\MtsBeweisschritt}[1][]{\glstext[#1]{MtsBeweisschritt}}
\newglossaryentry            {MtsBeweisschritt}{
	text    ={\ensuremath{\RawMtsBeweisschritt}},
	name    ={\ensuremath{\RawMtsBeweisschritt} \addIdx[
		name={\ensuremath{\RawMtsBeweisschritt} (Element)},
		sort={b Ele}]                                            {MtsBeweisschritt}},
	sort    ={b Ele},%    \LtrMtsBeweisschritt   Element
	type    ={symbols},
	description={
		Ein \Beweisschritt.
	}
}

\newcommand*                {\MtsBeweisschrittTup}[1][]{\glstext[#1]{MtsBeweisschrittTup}}
\newglossaryentry            {MtsBeweisschrittTup}{
	text    ={\ensuremath{\RawMtsBeweisschrittTup}},
	name    ={\ensuremath{\RawMtsBeweisschrittTup} \addIdx[
		name={\ensuremath{\RawMtsBeweisschrittTup} (Tupel)},
		sort={b Tup}]                                               {MtsBeweisschrittTup}},
	sort    ={b Tup},%    \LtrMtsBeweisschritt      Tupel
	type    ={symbols},
	description={
		Ein \Tupel\ von \Beweisschritten.
	}
}

\newcommand*             {\LtrMtsBeweisschrittSet} {B}% Menge der       Beweisschritte
\newcommand*                {\MtsBeweisschrittSet}[1][]{\glstext[#1]{MtsBeweisschrittSet}}
\newglossaryentry            {MtsBeweisschrittSet}{
	text    ={\ensuremath{\RawMtsBeweisschrittSet}},
	name    ={\ensuremath{\RawMtsBeweisschrittSet} \addIdx[
		name={\ensuremath{\RawMtsBeweisschrittSet} (Menge)},
		sort={B Men}]                                               {MtsBeweisschrittSet}},
	sort    ={B Men},%    \LtrMtsBeweisschrittSet   Menge
	type    ={symbols},
	description={
		Eine \Menge\ von \Beweisschritten.
	}
}

\newcommand*             {\LtrMtsErgebnis}         {e}% result; Ergebnis
\newcommand*                {\MtsErgebnis}[1][]{\glstext[#1]{MtsErgebnis}}
\newglossaryentry            {MtsErgebnis}{
	text    ={\ensuremath{\RawMtsErgebnis}},
	name    ={\ensuremath{\RawMtsErgebnis} \addIdx[
		name={\ensuremath{\RawMtsErgebnis} (Element)},
		sort={r Ele}]                                       {MtsErgebnis}},
	sort    ={r Ele},%    \LtrMtsErgebnis   Element
	type    ={symbols},
	description={
		Ein \Ergebnis.
	}
}

\newcommand*             {\LtrMtsErgebnisSet}      {E}% resultset; Ergebnismeng
\newcommand*                {\MtsErgebnisSet}[1][]{\glstext[#1]{MtsErgebnisSet}}
\newglossaryentry            {MtsErgebnisSet}{
	text    ={\ensuremath{\RawMtsErgebnisSet}},
	name    ={\ensuremath{\RawMtsErgebnisSet} \addIdx[
		name={\ensuremath{\RawMtsErgebnisSet} (Menge)},
		sort={R Men}]                                          {MtsErgebnisSet}},
	sort    ={R Men},%    \LtrMtsErgebnisSet   Menge
	type    ={symbols},
	description={
		Eine \Menge\ von \Ergebnissen.
	}
}

\newcommand*                {\MtsErgebnisRel}[1][]{\glstext[#1]{MtsErgebnisRel}}
\newglossaryentry            {MtsErgebnisRel}{
	text    ={\ensuremath{\RawMtsErgebnisRel}},
	name    ={\ensuremath{\RawMtsErgebnisRel} \addIdx[
		name={\ensuremath{\RawMtsErgebnisRel} (Relation)},
		sort={R Rel}]                                          {MtsErgebnisRel}},
	sort    ={R Rel},%    \LtrMtsErgebnisSet   Relation
	type    ={symbols},
	description={
		Eine \Relation\ (aufgefasst als \Menge) von \Ergebnissen.
	}
}

\newcommand*             {\LtrMtsErsetzung}        {E}% Substitution; Ersetzung
\newcommand*                {\MtsErsetzung}[1][]{\glstext[#1]{MtsErsetzung}}
\newglossaryentry            {MtsErsetzung}{
	text    ={\ensuremath{\RawMtsErsetzung}},
	name    ={\ensuremath{\RawMtsErsetzung} \addIdx[
		name={\ensuremath{\RawMtsErsetzung} (Element)},
		sort={E Ele}]                                        {MtsErsetzung}},
	sort    ={E Ele},%    \LtrMtsErsetzung   Element
	see     ={MtsErsetzungSet},
	type    ={symbols},
	description={
		Ein \Ersetzung.
	}
}

\newcommand*                {\MtsErsetzungSet}[1][]{\glstext[#1]{MtsErsetzungSet}}
\newglossaryentry            {MtsErsetzungSet}{
	text    ={\ensuremath{\RawMtsErsetzungSet}},
	name    ={\ensuremath{\RawMtsErsetzungSet} \addIdx[
		name={\ensuremath{\RawMtsErsetzungSet} (Menge)},
		sort={E Men}]                                           {MtsErsetzungSet}},
	sort    ={E Men},%    \LtrMtsErsetzung      Menge
	see     ={MtsErsetzung},
	type    ={symbols},
	description={
		Eine \Menge\ von \Ersetzungen.
	}
}

\newcommand*             {\LtrMtsKonklusion}        {k}% eine     Konklusion
\newcommand*                {\MtsKonklusion}[1][]{\glstext[#1]{MtsKonklusion}}
\newglossaryentry            {MtsKonklusion}{
	text    ={\ensuremath{\RawMtsKonklusion}},
	name    ={\ensuremath{\RawMtsKonklusion} \addIdx[
		name={\ensuremath{\RawMtsKonklusion} (Element)},
		sort={k Ele}]                                        {MtsKonklusion}},
	sort    ={k Ele},%    \LtrMtsKonklusion   Element
	type    ={symbols},
	description={
		Eine \Konklusion.
	}
}

\newcommand*             {\LtrMtsKonklusionSet}     {K}% Menge von   Konklusionen
\newcommand*                {\MtsKonklusionSet}[1][]{\glstext[#1]{MtsKonklusionSet}}
\newglossaryentry            {MtsKonklusionSet}{
	text    ={\ensuremath{\RawMtsKonklusionSet}},
	name    ={\ensuremath{\RawMtsKonklusionSet} \addIdx[
		name={\ensuremath{\RawMtsKonklusionSet} (Menge)},
		sort={K Men}]                                           {MtsKonklusionSet}},
	sort    ={K Men},%    \LtrMtsKonklusionSet   Menge
	type    ={symbols},
	description={
		Eine \Menge\ von \Konklusionen.
	}
}

\newcommand*                {\MtsKonklusionRel}[1][]{\glstext[#1]{MtsKonklusionRel}}
\newglossaryentry            {MtsKonklusionRel}{
	text    ={\ensuremath{\RawMtsKonklusionRel}},
	name    ={\ensuremath{\RawMtsKonklusionRel} \addIdx[
		name={\ensuremath{\RawMtsKonklusionRel} (Relation)},
		sort={K Rel}]                                            {MtsKonklusionRel}},
	sort    ={K Rel},%    \LtrMtsKonklusionSet   Relation
	type    ={symbols},
	description={
		Eine \Relation\ (aufgefasst als \Menge) von \Konklusionen.
	}
}

\newcommand*                {\MtsEmptyset}[1][]{\glstext[#1]{MtsEmptyset}}
\newglossaryentry            {MtsEmptyset}{
	text    ={\ensuremath{\RawMtsEmptyset}},
	name    ={\ensuremath{\RawMtsEmptyset} \addIdx[
		name={\ensuremath{\RawMtsEmptyset}},
		sort={O}]                                           {MtsEmptyset}},
	sort    ={O},
	type    ={symbols},
	description={
		Die \leereMenge, \textdh\ die einzige \Menge\ ohne \Elemente; auch mit $\{\}$ bezeichnet.
	}
}

\newcommand*             {\LtrMtsIN}               {N}% Natürliche Zahlen
\newcommand*                {\MtsIN}[1][]{\glstext[#1]{MtsIN}}
\newglossaryentry            {MtsIN}{
	text    ={\ensuremath{\RawMtsIN}},
	name    ={\ensuremath{\RawMtsIN} \addIdx[
		name={\ensuremath{\RawMtsIN}},
		sort={N}]                                     {MtsIN}},
	sort    ={N},%        \LtrMtsIN
	type    ={symbols},
	description={
		Die \Menge\ der \natuerlichenZahlen\ ohne 0.
	}
}

\newcommand*                {\MtsINo}[1][]{\glstext[#1]{MtsINo}}
\newglossaryentry            {MtsINo}{
	text    ={\ensuremath{\RawMtsINo}},
	name    ={\ensuremath{\RawMtsINo} \addIdx[
		name={\ensuremath{\RawMtsINo}},
		sort={N 0}]                                    {MtsINo}},
	sort    ={N 0},%      \LtrMtsIN 0
	type    ={symbols},
	description={
		Die \Menge\ der \natuerlichenZahlen\ (mit 0).
	}
}

\newcommand*                {\MtsMn}[1][]{\glstext[#1]{MtsMn}}
\newglossaryentry            {MtsMn}{
	text    ={\ensuremath{\RawMtsMn}},
	name    ={\ensuremath{\RawMtsMn} \addIdx[
		name={\ensuremath{\RawMtsMn}},
		sort={M n}]                                   {MtsMn}},
	sort    ={M n},
	see     ={Tupel},
	type    ={symbols},
	description={
		Das \kartesischeProdukt\ $M \MtsTimes \dots \MtsTimes M$ aus $n$ Mengen $M$ mit $n \MtsIn \MtsINo$.
	}
}

\newcommand*                {\MtsMo}[1][]{\glstext[#1]{MtsMo}}
\newglossaryentry            {MtsMo}{
	text    ={\ensuremath{\RawMtsMo}},
	name    ={\ensuremath{\RawMtsMo} \addIdx[
		name={\ensuremath{\RawMtsMo}},
		sort={M 0}]                                   {MtsMo}},
	sort    ={M 0},
	type    ={symbols},
	description={
		$\{()\}$, wobei $()$ das $0$-\Tupel\ ist.
	}
}

\newcommand*             {\LtrMtsPraemisse}    {p}% Eine Voraussetzung; Prämisse
\newcommand*                {\MtsPraemisse}[1][]{\glstext[#1]{MtsPraemisse}}
\newglossaryentry            {MtsPraemisse}{
	text    ={\ensuremath{\RawMtsPraemisse}},
	name    ={\ensuremath{\RawMtsPraemisse} \addIdx[
		name={\ensuremath{\RawMtsPraemisse} (Element)},
		sort={p Ele}]                                        {MtsPraemisse}},
	sort    ={p Ele},%    \LtrMtsPraemisse   Element
	type    ={symbols},
	description={
		Eine \Praemisse.
	}
}

\newcommand*             {\LtrMtsPraemisseSet} {P}% Menge der Voraussetzungen; Prämissen
\newcommand*                {\MtsPraemisseSet}[1][]{\glstext[#1]{MtsPraemisseSet}}
\newglossaryentry            {MtsPraemisseSet}{
	text    ={\ensuremath{\RawMtsPraemisseSet}},
	name    ={\ensuremath{\RawMtsPraemisseSet} \addIdx[
		name={\ensuremath{\RawMtsPraemisseSet} (Menge)},
		sort={P Men}]                                           {MtsPraemisseSet}},
	sort    ={P Men},%    \LtrMtsPraemisseSet   Menge
	type    ={symbols},
	description={
		Eine \Menge\ von \Praemissen.
	}
}

\newcommand*                {\MtsPraemisseRel}[1][]{\glstext[#1]{MtsPraemisseRel}}
\newglossaryentry            {MtsPraemisseRel}{
	text    ={\ensuremath{\RawMtsPraemisseRel}},
	name    ={\ensuremath{\RawMtsPraemisseRel} \addIdx[
		name={\ensuremath{\RawMtsPraemisseRel} (Relation)},
		sort={P Rel}]                                           {MtsPraemisseRel}},
	sort    ={P Rel},%    \LtrMtsPraemisseSet   Relation
	type    ={symbols},
	description={
		Eine \Relation\ (aufgefasst als \Menge) von \Praemissen.
	}
}

\newcommand*             {\LtrMtsSchlussregel}     {C}% conclusionrule; Schlussregel
\newcommand*                {\MtsSchlussregel}[1][]{\glstext[#1]{MtsSchlussregel}}
\newglossaryentry            {MtsSchlussregel}{
	text    ={\ensuremath{\RawMtsSchlussregel}},
	name    ={\ensuremath{\RawMtsSchlussregel} \addIdx[
		name={\ensuremath{\RawMtsSchlussregel} (Element)},
		sort={C Ele}]                                           {MtsSchlussregel}},
	sort    ={C Ele},%    \LtrMtsSchlussregel   Element
	type    ={symbols},
	description={
		Eine \Schlussregel.
	}
}

\newcommand*                {\MtsSchlussregelSet}[1][]{\glstext[#1]{MtsSchlussregelSet}}
\newglossaryentry            {MtsSchlussregelSet}{
	text    ={\ensuremath{\RawMtsSchlussregelSet}},
	name    ={\ensuremath{\RawMtsSchlussregelSet} \addIdx[
		name={\ensuremath{\RawMtsSchlussregelSet} (Menge)},
		sort={C Men}]                                              {MtsSchlussregelSet}},
	sort    ={C Men},%    \LtrMtsSchlussregel      Menge
	type    ={symbols},
	description={
		Eine \Menge\ von \Schlussregeln.
	}
}

\newcommand*             {\LtrMtsSprache}          {L}%      Sprache; language; \LtrOjkFor
\newcommand*                {\MtsSprache}[1][]{\glstext[#1]{MtsSprache}}
\newglossaryentry            {MtsSprache}{
	text    ={\ensuremath{\RawMtsSprache}},
	name    ={\ensuremath{\RawMtsSprache} \addIdx[
		name={\ensuremath{\RawMtsSprache}},
		sort={L}]                                          {MtsSprache}},
	sort    ={L},%        \LtrMtsSprache
	see     ={Formelmenge},
	type    ={symbols},
	description={
		Eine \Sprache.
	}
}

\newcommand*             {\LtrMtsTup}              {T}% sequenz; Menge der Tupel
\newcommand*                {\MtsTup}[1][]{\glstext[#1]{MtsTup}}
\newglossaryentry            {MtsTup}{
	text    ={\ensuremath{\RawMtsTup}},
	name    ={\ensuremath{\RawMtsTup} \addIdx[
		name={\ensuremath{\RawMtsTup}},
		sort={T}]                                      {MtsTup}},
	sort    ={T},%        \LtrMtsTup
	see     ={Tupelmenge},
	type    ={symbols},
	description={
		Eine \Mengenoperation: $\MtsTup(M)$ ist die \Menge\ aller \Tupel\ von $M$.
	}
}

\newcommand*             {\LtrMtsTransformation}       {T}% Transformation, Transformation,
\newcommand*                {\MtsTransformation}[1][]{\glstext[#1]{MtsTransformation}}
\newglossaryentry            {MtsTransformation}{
	text    ={\ensuremath{\RawMtsTransformation}},
	name    ={\ensuremath{\RawMtsTransformation} \addIdx[
		name={\ensuremath{\RawMtsTransformation} (Element)},
		sort={T Ele}]                                             {MtsTransformation}},
	sort    ={T Ele},%    \LtrMtsTransformation   Element
	type    ={symbols},
	description={
		Eine \Transformation.
	}
}

\newcommand*                {\MtsTransformationTup}[1][]{\glstext[#1]{MtsTransformationTup}}
\newglossaryentry            {MtsTransformationTup}{
	text    ={\ensuremath{\RawMtsTransformationTup}},
	name    ={\ensuremath{\RawMtsTransformationTup} \addIdx[
		name={\ensuremath{\RawMtsTransformationTup} (Tupel)},
		sort={T Tup}]                                                {MtsTransformationTup}},
	sort    ={T Tup},%    \LtrMtsTransformation      Tupel
	type    ={symbols},
	description={
		Eine \Menge\ von \Transformationen.
	}
}

% ==============================================================================
% \Ojk* - Ausgabe als Text-Symbol und Eintrag und Verweis ins Symbolverzeichnis
% Symbole für die Konstruktiuon von logischen Formeln ==========================

\newcommand*             {\LtrOjkABC}              {A}
\newcommand*                {\OjkABC}[1][]{\glstext[#1]{OjkABC}}
\newglossaryentry            {OjkABC}{
	text    ={\ensuremath{\RawOjkABC}},
	name    ={\ensuremath{\RawOjkABC} \addIdx[
		name={\ensuremath{\RawOjkABC}},
		sort={A a}]                                    {OjkABC}},
	sort    ={A a},%      \LtrOjkABC ('a' wegen Kollision mit 'Glo-TextSymbole')
	type    ={symbols},
	description={
		Das \Alphabet\ der \aussagenlogischenSprache.
	}
}

\newcommand*                {\OjkABCx}[1][]{\glstext[#1]{OjkABCx}}
\newglossaryentry            {OjkABCx}{
	text    ={\ensuremath{\RawOjkABC_x}},
	name    ={\ensuremath{\RawOjkABC_x} \addIdx[
		name={\ensuremath{\RawOjkABC_x}},
		sort={A x}]                                     {OjkABCx}},
	sort    ={A x},%      \LtrOjkABC x
	type    ={symbols},
	description={
		Eine \Teilmenge\ des \Alphabets\ \OjkABC\ der \aussagenlogischenSprache.
	}
}

\newcommand*             {\LtrOjkFor}              {L}% Sprache; language; \LtrMtsSprache
\newcommand*                {\OjkFor}[1][]{\glstext[#1]{OjkFor}}
\newglossaryentry            {OjkFor}{
	text    ={\ensuremath{\RawOjkFor}},
	name    ={\ensuremath{\RawOjkFor} \addIdx[
		name={\ensuremath{\RawOjkFor}},
		sort={L A}]                                    {OjkFor}},
	sort    ={L A},%      \LtrOjkFor \LtrMtsIdxLogisch
	type    ={symbols},
	description={
		Eine \Formelmenge: Die \Menge\ der \aussagenlogischenFormeln\ mit \Klammerung.
	}
}

\newcommand*                {\OjkForp}[1][]{\glstext[#1]{OjkForp}}
\newglossaryentry            {OjkForp}{
	text    ={\ensuremath{\RawOjkForp}},
	name    ={\ensuremath{\RawOjkForp} \addIdx[
		name={\ensuremath{\RawOjkForp}},
		sort={L Ap}]                                    {OjkForp}},
	sort    ={L Ap},%     \LtrOjkFor \LtrMtsIdxLogisch\LtrMtsIdxPolnisch
	type    ={symbols},
	description={
		Eine \Formelmenge: Die \Menge\ der \aussagenlogischenFormeln\ in \PolnischerNotation.
	}
}

\newcommand*                {\OjkForx}[1][]{\glstext[#1]{OjkForx}}
\newglossaryentry            {OjkForx}{
	text    ={\ensuremath{\RawOjkFor_x}},
	name    ={\ensuremath{\RawOjkFor_x} \addIdx[
		name={\ensuremath{\RawOjkFor_x}},
		sort={L A x}]                                   {OjkForx}},
	sort    ={L A x},%    \LtrOjkFor \LtrMtsIdxLogisch x
	type    ={symbols},
	description={
		Eine \Formelmenge: Eine \Teilmenge\ der \Menge\ \OjkFor\ der \aussagenlogischenFormeln\ mit \Klammerung.
	}
}

\newcommand*                {\OjkForpx}[1][]{\glstext[#1]{OjkForpx}}
\newglossaryentry            {OjkForpx}{
	text    ={\ensuremath{\RawOjkForp_x}},
	name    ={\ensuremath{\RawOjkForp_x} \addIdx[
		name={\ensuremath{\RawOjkForp_x}},
		sort={L Ap x}]                                   {OjkForpx}},
	sort    ={L Ap x},%   \LtrOjkFor \LtrMtsIdxLogisch\LtrMtsIdxPolnisch x
	type    ={symbols},
	description={
		Eine \Formelmenge: Eine \Teilmenge\ der \Menge\ \OjkForp\ der \aussagenlogischenFormel\ in \PolnischerNotation.
	}
}

\newcommand*             {\LtrOjkJun}              {J}% Junktoren
\newcommand*                {\OjkJun}[1][]{\glstext[#1]{OjkJun}}
\newglossaryentry            {OjkJun}{
	text    ={\ensuremath{\RawOjkJun}},
	name    ={\ensuremath{\RawOjkJun} \addIdx[
		name={\ensuremath{\RawOjkJun}},
		sort={J}]                                      {OjkJun}},
	sort    ={J},%        \LtrOjkJun
	see     ={Junktor},
	type    ={symbols},
	description={
		Die \Menge\ der \Junktorsymbole.
	}
}

\newcommand*                {\OjkJunx}[1][]{\glstext[#1]{OjkJunx}}
\newglossaryentry            {OjkJunx}{
	text    ={\ensuremath{\RawOjkJun_x}},
	name    ={\ensuremath{\RawOjkJun_x} \addIdx[
		name={\ensuremath{\RawOjkJun_x}},
		sort={J x}]                                     {OjkJunx}},
	sort    ={J x},%      \LtrOjkJun x
	type    ={symbols},
	description={
		Eine \Teilmenge\ der \Menge\ \OjkJun\ der \Junktorsymbole.
	}
}

\newcommand*                {\OjkBin}[1][]{\glstext[#1]{OjkBin}}
\newglossaryentry            {OjkBin}{
	text    ={\ensuremath{\RawOjkBin}},
	name    ={\ensuremath{\RawOjkBin} \addIdx[
		name={\ensuremath{\RawOjkBin}},
		sort={J b}]                                    {OjkBin}},
	sort    ={J b},%      \LtrOjkJun \StrMtsIdxBin
	type    ={symbols},
	description={
		Die \Menge\ der \binaerenJunktoren.
	}
}

\newcommand*                {\OjkCon}[1][]{\glstext[#1]{OjkCon}}
\newglossaryentry            {OjkCon}{
	text    ={\ensuremath{\RawOjkCon}},
	name    ={\ensuremath{\RawOjkCon} \addIdx[
		name={\ensuremath{\RawOjkCon}},
		sort={J c}]                                    {OjkCon}},
	sort    ={J c},%      \LtrOjkJun \StrMtsIdxCon
	type    ={symbols},
	description={
		Die \Menge\ der \aussagenlogischenKonstanten.
	}
}

\newcommand*                {\OjkUna}[1][]{\glstext[#1]{OjkUna}}
\newglossaryentry            {OjkUna}{
	text    ={\ensuremath{\RawOjkUna}},
	name    ={\ensuremath{\RawOjkUna} \addIdx[
		name={\ensuremath{\RawOjkUna}},
		sort={J u}]                                    {OjkUna}},
	sort    ={J u},%      \LtrOjkJun \StrMtsIdxUna
	type    ={symbols},
	description={
		Die \Menge\ der \unaerenJunktoren.
	}
}

\newcommand*             {\LtrOjkvar}              {q}% Name aussagenlogische Variable
\newcommand*                {\Ojkvar}[1][]{\glstext[#1]{Ojkvar}}
\newglossaryentry            {Ojkvar}{
	text    ={\ensuremath{\RawOjkvar}},
	name    ={\ensuremath{\RawOjkvar} \addIdx[
		name={\ensuremath{\RawOjkvar}},
		sort={q}]                                      {Ojkvar}},
	sort    ={q},%        \LtrOjkvar
	see     ={Aussagenlogik},
	type    ={symbols},
	description={
		Die \Elemente\ aus \OjkVar\ sind die \aussagenlogischenVariablen.
	}
}

\newcommand*             {\LtrOjkVar}              {Q}% Menge aussagenlogische Variable
\newcommand*                {\OjkVar}[1][]{\glstext[#1]{OjkVar}}
\newglossaryentry            {OjkVar}{
	text    ={\ensuremath{\RawOjkVar}},
	name    ={\ensuremath{\RawOjkVar} \addIdx[
		name={\ensuremath{\RawOjkVar}},
		sort={Q}]                                      {OjkVar}},
	sort    ={Q},%        \LtrOjkVar
	see     ={Aussagenlogik},
	type    ={symbols},
	description={
		$\OjkVar \MtsDefEq \MengeDef{\Ojkvar_i}{i \in \MtsINo}$,
		die \Menge\ der \aussagenlogischenVariablen.
	}
}

% ==============================================================================
% \Mts* - Ausgabe als Text-Symbol und Eintrag und Verweis ins Symbolverzeichnis
% Wahrheitswerte ===============================================================

\newcommand*             {\StrMtsFalse}            {false}
\newcommand*                {\MtsFalse}[1][]{\glstext[#1]{MtsFalse}}
%ToDo prüfen
\newglossaryentry            {MtsFalse}{
	text    ={\ensuremath{\RawMtsFalse}},
	name    ={\ensuremath{\RawMtsFalse} \addIdx[
		name={\ensuremath{\RawMtsFalse}},
		sort={false}]                                    {MtsFalse}},
	sort    ={false},%    \StrMtsFalse
	see     ={OjkFalse,MtsTrue},
	type    ={symbols},
	description={
		Der \metasprachlicheWahrheitswert\ \TxtFalse\ als \Symbol.
	}
}

\newcommand*             {\StrMtsTrue}             {true}
\newcommand*                {\MtsTrue}[1][]{\glstext[#1]{MtsTrue}}
%ToDo prüfen
\newglossaryentry            {MtsTrue}{
	text    ={\ensuremath{\RawMtsTrue}},
	name    ={\ensuremath{\RawMtsTrue} \addIdx[
		name={\ensuremath{\RawMtsTrue}},
		sort={true}]                                    {MtsTrue}},
	sort    ={true},%     \StrMtsTrue
	see     ={OjkTrue,MtsFalse},
	type    ={symbols},
	description={
		Der \metasprachlicheWahrheitswert\ \TxtTrue\ als \Symbol.
	}
}
