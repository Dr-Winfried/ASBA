%%############################################################################%%
%%                                                                            %%
%% Datei:  ASBA-Vorspann-Glossary.tex                                         %%
%% Inhalt: Vorspann Glossareinträge für ASBA                                  %%
%%                                                                            %%
%% Copyright (C) 2017  Winfried Teschers                                      %%
%%                                                                            %%
%% This program is free software: you can redistribute it and/or modify       %%
%% it under the terms of the GNU Affero General Public License as published   %%
%% by the Free Software Foundation, either version 3 of the License, or       %%
%% (at your option) any later version.                                        %%
%%                                                                            %%
%% This program is distributed in the hope that it will be useful,            %%
%% but WITHOUT ANY WARRANTY; without even the implied warranty of             %%
%% MERCHANTABILITY or FITNESS FOR A PARTICULAR PURPOSE.  See the              %%
%% GNU Affero General Public License for more details.                        %%
%%                                                                            %%
%% You should have received a copy of the GNU Affero General Public License   %%
%% along with this program.  If not, see <http://www.gnu.org/licenses/>.      %%
%%                                                                            %%
%% Dr. Winfried Teschers                                                      %%
%% Anton-Günther-Straße 26c                                                   %%
%% 91083 Baiersdorf                                                           %%
%% Germany                                                                    %%
%%                                                                            %%
%% e-mail: winfried.teschers@t-online.de                                      %%
%%                                                                            %%
%%############################################################################%%

% !TeX root = ASBA.tex
% !TeX encoding = UTF-8
% !TeX spellcheck = de_DE

%TODO Im Index und Glossar prüfen: Haben alle Einträge einen Verweise auf die Definition?

% Elemente, die keine Glossareinträge sind und dafür nicht gebraucht werden,
% werden in "ASBA-Vorspann.tex" und "ASBA-Mathematik-Vorspann.tex" definiert.

\newglossary[nlg]{symbols}{not}{ntn}{Symbole}

% Fonts für die Liste der Seitenangaben
\newcommand*{\hyperDef}[1]{\textbf{\hypersf{#1}}}% für Definitionen
\newcommand*{\hyperTxt}[1]{\hyperrm{#1}}%   normal für Texte
\GlsAddXdyAttribute{hyperDef}% damit xindy damit umgehen kann
\GlsAddXdyAttribute{hyperTxt}% damit xindy damit umgehen kann

\GlsSetXdyMinRangeLength{2}% Seitenbereiche ab ...
\makeglossaries
\setacronymstyle{long-sc-short}
\renewcommand*{\glsnumberformat}[1]{\hyperTxt{#1}}% Standardformat für Seitenliste

% Makros für neue Symbole und Begriffe =========================================

% neue Symbole -----------------------------------------------------------------
% Ausgabe und Aufnahme ins Glossar - - - - - - - - - - - - - - - - - - - - - - -
% [<Font-Makro>] {<Glossary Key>}
\newcommand*{\glsSym}[2][]{\glssymbol [#1]             {#2}}% mit  Link ins Glossar
\newcommand*{\glsTag}[1]  {\glssymbol*[format=hyperDef]{#1}}% ohne Link ins Glossar
% ... mit Hervorhebung der Seitennummer- - - - - - - - - - - - - - - - - - - - -
% {<Makro>} - An <Makro> muss [<Font-Makro>] angehängt werden können
\newcommand*{\defSym}   [1]          {#1[format=hyperDef]}
\newcommand*{\defSymUna}[1]  {\defSym{#1}\;}%  unär: folgender  Abstand
\newcommand*{\defSymBin}[1]{\;\defSym{#1}\;}% binär: umgebender Abstand

% neue Begriffe ----------------------------------------------------------------
% hervorgehoben, aber kein Verweis - - - - - - - - - - - - - - - - - - - - - - -
% {<Begriff>}
\newcommand*{\defFt}[1]{\textbf{#1}}
% nur Eintrag im Index - - - - - - - - - - - - - - - - - - - - - - - - - - - - -
% [Font-Makro] {<Index-Eintrag>}
\newcommand*{\idx}[2][]{\sindex[idx]{#2|#1}}
% Ausgabe und Aufnahme in Index und Glossar- - - - - - - - - - - - - - - - - - -
% [Font-Makro] {Golossary key}
% 1. Buchstabe klein
\newcommand*{\glsIdx}   [2][hyperTxt]{\idx[#1]{\glsentrytext{#2}}\glstext   [format=#1]{#2}}
\newcommand*{\glsIdxI}  [2][hyperTxt]{\idx[#1]{\glsentrytext{#2}}\glsuseri  [format=#1]{#2}}
\newcommand*{\glsIdxII} [2][hyperTxt]{\idx[#1]{\glsentrytext{#2}}\glsuserii [format=#1]{#2}}
\newcommand*{\glsIdxIII}[2][hyperTxt]{\idx[#1]{\glsentrytext{#2}}\glsuseriii[format=#1]{#2}}
\newcommand*{\glsIdxIV} [2][hyperTxt]{\idx[#1]{\glsentrytext{#2}}\glsuseriv [format=#1]{#2}}
\newcommand*{\glsIdxV}  [2][hyperTxt]{\idx[#1]{\glsentrytext{#2}}\glsuserv  [format=#1]{#2}}
\newcommand*{\glsIdxVI} [2][hyperTxt]{\idx[#1]{\glsentrytext{#2}}\glsuservi [format=#1]{#2}}
\newcommand*{\glsIdxPl} [2][hyperTxt]{\idx[#1]{\glsentrytext{#2}}\glspl     [format=#1]{#2}}
% 1. Buchstabe groß
\newcommand*{\GlsIdx}   [2][hyperTxt]{\idx[#1]{\glsentrytext{#2}}\Glstext   [format=#1]{#2}}
\newcommand*{\GlsIdxPl} [2][hyperTxt]{\idx[#1]{\glsentrytext{#2}}\Glspl     [format=#1]{#2}}
% ... mit Hervorhebung der Seitennummer- - - - - - - - - - - - - - - - - - - - -
% {<Makro>} - An <Makro> muss [<Font-Makro>] angehängt werden können
\newcommand*{\defTxt}[1]{\defFt{#1[hyperDef]}}% Seitennummer hervorheben
% normal und Verweis ins Glossar - - - - - - - - - - - - - - - - - - - - - - - -

% Glossar-Einträge #############################################################

\newglossaryentry{dummy}{name={\ensuremath{\text{|--}}},type={symbols},description={Dummy Symbol}}%%% Nur zum Testen
\newglossaryentry{Dummy}{name={Dummy},description={Dummy Text}}%%% Nur zum Testen

% Symbole für Beispieloperationen und -relationen ------------------------------
% \Bsp* - Ausgabe als Symbol und Aufnahme in Symbolliste und Glossar

\newcommand*              {\BspOpU}[1][]{\glsSym[#1]{BspOpU}}
\newglossaryentry          {BspOpU}{
	name  ={\ensuremath{\RawBspOpU}},
	symbol={\ensuremath{\RawBspOpU}},
	sort  ={= 0 1 1},
	type  ={symbols},
	description={
		Beispielsymbol für eine unäre \Operation.
	}
}
\newcommand*              {\BspOpB}[1][]{\glsSym[#1]{BspOpB}}
\newglossaryentry          {BspOpB}{
	name  ={\ensuremath{\RawBspOpB}},
	symbol={\ensuremath{\RawBspOpB}},
	sort  ={= 0 1 2},
	type  ={symbols},
	description={
		Beispielsymbol für eine binäre \Operation.
	}
}
%Nummerierung ändern (Verneinung ans Ende)
\newcommand*              {\BspRel}[1][]{\glsSym[#1]{BspRel}}
\newglossaryentry          {BspRel}{
	name  ={\ensuremath{\RawBspRel}},
	symbol={\ensuremath{\RawBspRel}},
	sort  ={= 0 2 1},
	type  ={symbols},
	description={
		Beispielsymbol für eine binäre \Relation\ mit \Umkehrrelation\ \BspRelBck.
	}
}
\newcommand*              {\BspRelEq}[1][]{\glsSym[#1]{BspRelEq}}
\newglossaryentry          {BspRelEq}{
	name  ={\ensuremath{\RawBspRelEq}},
	symbol={\ensuremath{\RawBspRelEq}},
	sort  ={= 0 2 2},
	type  ={symbols},
	description={
		Beispielsymbol für eine binäre \Relation\ mit \Gleichheit\ und \Umkehrrelation\ \BspRelBckEq.
	}
}
\newcommand*              {\BspRelBck}[1][]{\glsSym[#1]{BspRelBck}}
\newglossaryentry          {BspRelBck}{
	name  ={\ensuremath{\RawBspRelBck}},
	symbol={\ensuremath{\RawBspRelBck}},
	sort  ={= 0 2 3},
	type  ={symbols},
	description={
		Beispielsymbol für eine binäre \Relation\ mit \Umkehrrelation\ \BspRel.
	}
}
\newcommand*              {\BspRelBckEq}[1][]{\glsSym[#1]{BspRelBckEq}}
\newglossaryentry          {BspRelBckEq}{
	name  ={\ensuremath{\RawBspRelBckEq}},
	symbol={\ensuremath{\RawBspRelBckEq}},
	sort  ={= 0 2 4},
	type  ={symbols},
	description={
		Beispielsymbol für eine binäre \Relation\ mit \Gleichheit\ und \Umkehrrelation\ \BspRelEq.
	}
}
\newcommand*              {\BspRelN}[1][]{\glsSym[#1]{BspRelN}}
\newglossaryentry          {BspRelN}{
	name  ={\ensuremath{\RawBspRelN}},
	symbol={\ensuremath{\RawBspRelN}},
	sort  ={= 0 3 1},
	type  ={symbols},
	description={
		Verneinung von \BspRel.
	}
}
\newcommand*              {\BspRelEqN}[1][]{\glsSym[#1]{BspRelEqN}}
\newglossaryentry          {BspRelEqN}{
	name  ={\ensuremath{\RawBspRelEqN}},
	symbol={\ensuremath{\RawBspRelEqN}},
	sort  ={= 0 3 2},
	type  ={symbols},
	description={
		Verneinung von \BspRelEq.
	}
}
\newcommand*              {\BspRelBckN}[1][]{\glsSym[#1]{BspRelBckN}}
\newglossaryentry          {BspRelBckN}{
	name  ={\ensuremath{\RawBspRelBckN}},
	symbol={\ensuremath{\RawBspRelBckN}},
	sort  ={= 0 3 3},
	type  ={symbols},
	description={
		Verneinung von \BspRelBck.
	}
}
\newcommand*              {\BspRelBckEqN}[1][]{\glsSym[#1]{BspRelBckEqN}}
\newglossaryentry          {BspRelBckEqN}{
	name  ={\ensuremath{\RawBspRelBckEqN}},
	symbol={\ensuremath{\RawBspRelBckEqN}},
	sort  ={= 0 3 4},
	type  ={symbols},
	description={
		Verneinung von \BspRelBckEq.
	}
}

% Meta-Symbole -----------------------------------------------------------------
% \Mts* - Ausgabe als Symbol und Aufnahme in Symbolliste und Glossar

\newcommand*              {\MtsNot}[1][]{\glsSym[#1]{MtsNot}}
\newglossaryentry          {MtsNot}{
	name  ={\ensuremath{\RawMtsNot}},
	symbol={\ensuremath{\RawMtsNot}},
	sort  ={= 1 1 1},
	see   ={FrmNot},
	type  ={symbols},
	description={
		Eine unäre \Metaoperation:~ \textdots\ \emph{gilt nicht}.
	}
}
\newcommand*              {\MtsAnd}[1][]{\glsSym[#1]{MtsAnd}}
\newglossaryentry          {MtsAnd}{
	name  ={\ensuremath{\RawMtsAnd}},
	symbol={\ensuremath{\RawMtsAnd}},
	sort  ={= 1 1 2},
	see   ={FrmAnd},
	type  ={symbols},
	description={
		Eine \Metaoperation:~ \textdots\ \emph{und} \textdots
	}
}
\newcommand*              {\MtsOr}[1][]{\glsSym[#1]{MtsOr}}
\newglossaryentry          {MtsOr}{
	name  ={\ensuremath{\RawMtsOr}},
	symbol={\ensuremath{\RawMtsOr}},
	sort  ={= 1 1 3},
	see   ={FrmOr},
	type  ={symbols},
	description={
		Eine \Metaoperation:~ \textdots\ \emph{oder} \textdots
	}
}
\newcommand*              {\MtsImp}[1][]{\glsSym[#1]{MtsImp}}
\newglossaryentry          {MtsImp}{
	name  ={\ensuremath{\RawMtsImp}},
	symbol={\ensuremath{\RawMtsImp}},
	sort  ={= 1 2 1},
	see   ={FrmImp},
	type  ={symbols},
	description={
		Eine \Metarelation:~ \textdots\ \emph{dann auch} \textdots, die \Umkehrrelation\ zu \MtsRep.
	}
}
\newcommand*              {\MtsRep}[1][]{\glsSym[#1]{MtsRep}}
\newglossaryentry          {MtsRep}{
	name  ={\ensuremath{\RawMtsRep}},
	symbol={\ensuremath{\RawMtsRep}},
	sort  ={= 1 2 2},
	see   ={FrmRep},
	type  ={symbols},
	description={
		Eine \Metarelation:~ \textdots\ \emph{sofern} \textdots , die \Umkehrrelation\ zu \MtsImp.
	}
}
\newcommand*              {\MtsEquiv}[1][]{\glsSym[#1]{MtsEquiv}}
\newglossaryentry          {MtsEquiv}{
	name  ={\ensuremath{\RawMtsEquiv}},
	symbol={\ensuremath{\RawMtsEquiv}},
	sort  ={= 1 2 3},
	see   ={FrmEquiv},
	type  ={symbols},
	description={
		Eine \Metarelation:~ \textdots\ \emph{genau dann wenn} \textdots
	}
}
\newcommand*              {\MtsEq}[1][]{\glsSym[#1]{MtsEq}}
\newglossaryentry          {MtsEq}{
	name  ={\ensuremath{\RawMtsEq}},
	symbol={\ensuremath{\RawMtsEq}},
	sort  ={= 1 3 1},
	see   ={FrmEq,Gleichheit},
	type  ={symbols},
	description={
		Eine \Metarelation:~ \textdots\ \emph{ist gleich} (dasselbe wie; identisch zu) \textdots
	}
}
\newcommand*              {\MtsEqN}[1][]{\glsSym[#1]{MtsEqN}}
\newglossaryentry          {MtsEqN}{
	name  ={\ensuremath{\RawMtsEqN}},
	symbol={\ensuremath{\RawMtsEqN}},
	sort  ={= 1 3 2},
	see   ={FrmEqN},%%% Ungleichheit?
	type  ={symbols},
	description={
		Eine \Metarelation:~ \textdots\ \emph{ist ungleich} (nicht dasselbe wie, nicht identisch zu) \textdots
	}
}
\newcommand*              {\MtsAequiv}[1][]{\glsSym[#1]{MtsAequiv}}
\newglossaryentry          {MtsAequiv}{
	name  ={\ensuremath{\RawMtsAequiv}},
	symbol={\ensuremath{\RawMtsAequiv}},
	sort  ={= 1 3 3},
	see   ={Aequivalenz},
	type  ={symbols},
	description={
		Eine \Metarelation:~ \textdots\ \emph{äquivalent} (so wie; ähnlich) \textdots
	}
}
\newcommand*              {\MtsAequivN}[1][]{\glsSym[#1]{MtsAequivN}}
\newglossaryentry          {MtsAequivN}{
	name  ={\ensuremath{\RawMtsNAequiv}},
	symbol={\ensuremath{\RawMtsNAequiv}},
	sort  ={= 1 3 4},
	see   ={Aequivalenz},
	type  ={symbols},
	description={
		Eine \Metarelation:~ \textdots\ \emph{nicht äquivalent} (nicht so wie; nicht ähnlich) \textdots
	}
}
\newcommand*              {\MtsDefEquiv}[1][]{\glsSym[#1]{MtsDefEquiv}}
\newglossaryentry          {MtsDefEquiv}{
	name  ={\ensuremath{\RawMtsDefEquiv}},
	symbol={\ensuremath{\RawMtsDefEquiv}},
	sort  ={= 1 4 1},
	type  ={symbols},
	description={
		\Metadefinition:~ \textdots\ \emph{definitionsgemäß genau dann wenn} \textdots
	}
}
\newcommand*              {\MtsDefEq}[1][]{\glsSym[#1]{MtsDefEq}}
\newglossaryentry          {MtsDefEq}{
	name  ={\ensuremath{\RawMtsDefEq}},
	symbol={\ensuremath{\RawMtsDefEq}},
	sort  ={= 1 4 2},
	type  ={symbols},
	description={
		\Definition:~ \textdots\ \emph{definitionsgemäß gleich} (dasselbe wie; identisch zu) \textdots
	}
}
\newcommand*              {\MtsUnd}[1][]{\glsSym[#1]{MtsUnd}}
%ToDo prüfen
\newglossaryentry          {MtsUnd}{
	name  ={\ensuremath{\RawMtsUnd}},
	symbol={\ensuremath{\RawMtsUnd}},
	sort  ={= 1 5 1},
	see   ={MtsAnd,FrmAnd},
	type  ={symbols},
	description={
		Eine \Metaoperation\ (nur für Schlussregeln):~ \textdots\ \emph{und} \textdots
	}
}
\newcommand*              {\MtsDerive}[1][]{\glsSym[#1]{MtsDerive}}
\newglossaryentry          {MtsDerive}{
	name  ={\ensuremath{\RawMtsDerive}},
	symbol={\ensuremath{\RawMtsDerive}},
	sort  ={= 1 5 2},
	see   ={MtsDeriveR},
	type  ={symbols},
	description={
		\Ableitungsrelation:~ \textdots\ \emph{\ableitbar} (\beweisbar) \textdots
	}
}
\newcommand*              {\MtsDeriveR}[1][]{\glsSym[#1]{MtsDeriveR}}
\newglossaryentry          {MtsDeriveR}{
	name  ={\ensuremath{\RawMtsDerive_R}},
	symbol={\ensuremath{\RawMtsDerive_R}},
	sort  ={= 1 5 2R},
	type  ={symbols},
	description={
		Eine Darstellung der \Relation\ $R$ aus $\MtsRel(\MtsPot(\MtsSprache))$ als \Ableitungsrelation.
	}
}
\newcommand*              {\MtsSubst}[1][]{\glsSym[#1]{MtsSubst}}
%%%ToDo prüfen
\newglossaryentry          {MtsSubst}{
	name  ={\ensuremath{\RawMtsSubst}},
	symbol={\ensuremath{\RawMtsSubst}},
	sort  ={= 1 5 3},
	type  ={symbols},
	description={
		\Ersetzung:~ \textdots\ \emph{substituiert durch} \textdots
	}
}
\newcommand*              {\MtsSwap}[1][]{\glsSym[#1]{MtsSwap}}
%%%ToDo prüfen
\newglossaryentry          {MtsSwap}{
	name  ={\ensuremath{\RawMtsSwap}},
	symbol={\ensuremath{\RawMtsSwap}},
	sort  ={= 1 5 4},
	type  ={symbols},
	description={
		\Vertauschung:~ \textdots\ \emph{vertauscht mit} \textdots
	}
}

% aussagenlogische Operationen, dargestellt mit Symbolen -----------------------
% \Frm* - Ausgabe als Symbol und Aufnahme in Symbolliste und Glossar

\newcommand*              {\FrmFalse}[1][]{\glsSym[#1]{FrmFalse}}
\newglossaryentry          {FrmFalse}{
	name  ={\ensuremath{\RawFrmFalse}},
	symbol={\ensuremath{\RawFrmFalse}},
	sort  ={= 2 0 1},
	see   ={MtsFalse},
	type  ={symbols},
	description={
		Ein 0-stelliger \Junktor, \textdh\ eine aussagenlogische Konstante mit dem \Wahrheitswert\ \TxtFalse.
	}
}
\newcommand*              {\FrmTrue}[1][]{\glsSym[#1]{FrmTrue}}
\newglossaryentry          {FrmTrue}{
	name  ={\ensuremath{\RawFrmTrue}},
	symbol={\ensuremath{\RawFrmTrue}},
	sort  ={= 2 0 2},
	see   ={MtsTrue},
	type  ={symbols},
	description={
		Ein 0-stelliger \Junktor, \textdh\ eine aussagenlogische Konstante mit dem \Wahrheitswert\ \TxtTrue.
	}
}
\newcommand*              {\FrmNot}[1][]{\glsSym[#1]{FrmNot}}
\newglossaryentry          {FrmNot}{
	name  ={\ensuremath{\RawFrmNot}},
	symbol={\ensuremath{\RawFrmNot}},
	sort  ={= 2 1 1},
	see   ={MtsNot},
	type  ={symbols},
	description={
		Ein unärer \Junktor:~ \emph{nicht} \textdots
	}
}
\newcommand*              {\FrmAnd}[1][]{\glsSym[#1]{FrmAnd}}
\newglossaryentry          {FrmAnd}{
	name  ={\ensuremath{\RawFrmAnd}},
	symbol={\ensuremath{\RawFrmAnd}},
	sort  ={= 2 1 2},
	see   ={FrmNand,MtsAnd},
	type  ={symbols},
	description={
		Ein binärer \Junktor:~ \textdots\ \emph{und} \textdots
	}
}
\newcommand*              {\FrmOr}[1][]{\glsSym[#1]{FrmOr}}
\newglossaryentry          {FrmOr}{
	name  ={\ensuremath{\RawFrmOr}},
	symbol={\ensuremath{\RawFrmOr}},
	sort  ={= 2 1 3},
	see   ={FrmNor,FrmXor,MtsOr},
	type  ={symbols},
	description={
		Ein binärer \Junktor:~ \textdots\ \emph{oder} \textdots
	}
}
\newcommand*              {\FrmImp}[1][]{\glsSym[#1]{FrmImp}}
\newglossaryentry          {FrmImp}{
	name  ={\ensuremath{\RawFrmImp}},
	symbol={\ensuremath{\RawFrmImp}},
	sort  ={= 2 2 1},
	see   ={MtsImp},
	type  ={symbols},
	description={
		Ein binärer \Junktor:~ \emph{aus} \textdots\ \emph{folgt} \textdots
	}
}
\newcommand*              {\FrmRep}[1][]{\glsSym[#1]{FrmRep}}
\newglossaryentry          {FrmRep}{
	name  ={\ensuremath{\RawFrmRep}},
	symbol={\ensuremath{\RawFrmRep}},
	sort  ={= 2 2 2},
	see   ={MtsRep},
	type  ={symbols},
	description={
		Ein binärer \Junktor:~ \textdots\ \emph{folgt aus} \textdots
	}
}
\newcommand*              {\FrmEquiv}[1][]{\glsSym[#1]{FrmEquiv}}
\newglossaryentry          {FrmEquiv}{
	name  ={\ensuremath{\RawFrmEquiv}},
	symbol={\ensuremath{\RawFrmEquiv}},
	sort  ={= 2 2 3},
	see   ={MtsEquiv},
	type  ={symbols},
	description={
		Ein binärer \Junktor:~ \textdots\ \emph{genau dann wenn} \textdots
	}
}
\newcommand*              {\FrmNand}[1][]{\glsSym[#1]{FrmNand}}
\newglossaryentry          {FrmNand}{
	name  ={\ensuremath{\RawFrmNand}},
	symbol={\ensuremath{\RawFrmNand}},
	sort  ={= 2 3 1},
	see   ={FrmAnd},
	type  ={symbols},
	description={
		Ein binärer \Junktor:~ \emph{nicht zugleich} \textdots\ \emph{und} \textdots
	}
}
\newcommand*              {\FrmNor}[1][]{\glsSym[#1]{FrmNor}}
\newglossaryentry          {FrmNor}{
	name  ={\ensuremath{\RawFrmNor}},
	symbol={\ensuremath{\RawFrmNor}},
	sort  ={= 2 3 2},
	see   ={FrmOr,FrmXor},
	type  ={symbols},
	description={
		Ein binärer \Junktor:~ \emph{weder} \textdots\ \emph{noch} \textdots
	}
}
\newcommand*              {\FrmXor}[1][]{\glsSym[#1]{FrmXor}}
\newglossaryentry          {FrmXor}{
	name  ={\ensuremath{\RawFrmXor}},
	symbol={\ensuremath{\RawFrmXor}},
	sort  ={= 2 3 3},
	see   ={FrmOr,FrmNor},
	type  ={symbols},
	description={
		Ein binärer \Junktor:~ \emph{entweder} \textdots\ \emph{oder} \textdots
	}
}
\newcommand*              {\FrmEq}[1][]{\glsSym[#1]{FrmEq}}
%ToDo prüfen
\newglossaryentry          {FrmEq}{
	name  ={\ensuremath{\RawFrmEq}},
	symbol={\ensuremath{\RawFrmEq}},
	sort  ={= 2 4 1},
	see   ={MtsEq},
	type  ={symbols},
	description={
		Logische Gleichheit:~ \textdots\ \emph{gleich} \textdots
	}
}
\newcommand*              {\FrmEqN}[1][]{\glsSym[#1]{FrmEqN}}
%ToDo prüfen
\newglossaryentry          {FrmEqN}{
	name  ={\ensuremath{\RawFrmEqN}},
	symbol={\ensuremath{\RawFrmEqN}},
	sort  ={= 2 4 2},
	see   ={MtsEqN},
	type  ={symbols},
	description={
		Logische Ungleichheit:~ \textdots\ \emph{ungleich} \textdots
	}
}

% Mengen-Operatoren ------------------------------------------------------------
% \sym* - Ausgabe als Symbol und Aufnahme in Symbolliste und Glossar
%TODO ### Überprüfung hier fortsetzen.

\newcommand*              {\MtsIn}[1][]{\glsSym[#1]{MtsIn}}
\newglossaryentry          {MtsIn}{
	name  ={\ensuremath{\RawMtsIn}},
	symbol={\ensuremath{\RawMtsIn}},
	sort  ={= 3 0 1},
	type  ={symbols},
	description={
		Ein Mengenoperator:~ \textdots\ \emph{ist Element aus} (der Menge) \textdots
	}
}
\newcommand*              {\MtsNi}[1][]{\glsSym[#1]{MtsNi}}
\newglossaryentry          {MtsNi}{
	name  ={\ensuremath{\RawMtsNi}},
	symbol={\ensuremath{\RawMtsNi}},
	sort  ={= 3 0 2},
	type  ={symbols},
	description={
		Ein Mengenoperator:~ \textdots\ \emph{ist Element aus} (der Menge) \textdots
	}
}
\newcommand*              {\MtsSubset}[1][]{\glsSym[#1]{MtsSubset}}
\newglossaryentry          {MtsSubset}{
	name  ={\ensuremath{\RawMtsSubset}},
	symbol={\ensuremath{\RawMtsSubset}},
	sort  ={= 3 1 1},
	type  ={symbols},
	description={
		Mengenoperator:~ (die Menge) \textdots\ \emph{ist echte Teilmenge von} (der Menge) \textdots
		; es kann keine \Gleichheit\ bestehen.
		In der Literatur wird \MtsSubset\ oft im Sinne von \MtsSubsetEq\ verwendet.
	}
}
\newcommand*              {\MtsSubsetEq}[1][]{\glsSym[#1]{MtsSubsetEq}}
\newglossaryentry          {MtsSubsetEq}{
	name  ={\ensuremath{\RawMtsSubsetEq}},
	symbol={\ensuremath{\RawMtsSubsetEq}},
	sort  ={= 3 1 2},
	type  ={symbols},
	description={
		Ein Mengenoperator:~ (die Menge) \textdots\ \emph{ist Teilmenge von} (der Menge) \textdots
		; es kann \Gleichheit\ bestehen.
	}
}
\newcommand*              {\MtsSubsetN}[1][]{\glsSym[#1]{MtsSubsetN}}
\newglossaryentry          {MtsSubsetN}{
	name  ={\ensuremath{\RawMtsSubsetN}},
	symbol={\ensuremath{\RawMtsSubsetN}},
	sort  ={= 3 1 3},
	type  ={symbols},
	description={
		Ein Mengenoperator:~ (die Menge) \textdots\ \emph{ist keine echte Teilmenge von} (der Menge) \textdots
	}
}
\newcommand*              {\MtsSupset}[1][]{\glsSym[#1]{MtsSupset}}
\newglossaryentry          {MtsSupset}{
	name  ={\ensuremath{\RawMtsSupset}},
	symbol={\ensuremath{\RawMtsSupset}},
	sort  ={= 3 2 1},
	type  ={symbols},
	description={
		Ein Mengenoperator:~ (die Menge) \textdots\ \emph{ist echte Obermenge von} (der Menge) \textdots
		; es kann keine \Gleichheit\ bestehen.
		In der Literatur wird \MtsSupset\ oft im Sinne von \MtsSupsetEq\ verwendet.
	}
}
\newcommand*              {\MtsSupsetEq}[1][]{\glsSym[#1]{MtsSupsetEq}}
\newglossaryentry          {MtsSupsetEq}{
	name  ={\ensuremath{\RawMtsSupsetEq}},
	symbol={\ensuremath{\RawMtsSupsetEq}},
	sort  ={= 3 2 2},
	type  ={symbols},
	description={
		Ein Mengenoperator:~ (die Menge) \textdots\ \emph{ist Obermenge von} (der Menge) \textdots
		; es kann \Gleichheit\ bestehen.
	}
}
\newcommand*              {\MtsSupsetN}[1][]{\glsSym[#1]{MtsSupsetN}}
\newglossaryentry          {MtsSupsetN}{
	name  ={\ensuremath{\RawMtsSupsetN}},
	symbol={\ensuremath{\RawMtsSupsetN}},
	sort  ={= 3 2 3},
	type  ={symbols},
	description={
		Teilmengenbeziehung:~ (die Menge) \textdots\ \emph{ist keine echte Obermenge von} (der Menge) \textdots
	}
}

% Schlussregeln ----------------------------------------------------------------
% \*    - Ausgabe sowohl im Text- als auch Mathematik-Modus
% \sym* - Ausgabe als Symbol und Eintrag in Symbolliste und Glossar
% \tag* - wie \sym*, aber ohne Verweis ins Glossar
% Verweise:
%   \ref    {def-*} -->  \*
%   \eqref  {def-*} --> (\*)
%   \vreffor{def-*} --> (\*) auf Seite n

\newcommand*    {\AR}{\ensuremath{\text{AR}}}
\newcommand* {\symAR}[1][]{\glsSym[#1]{AR}}
\newcommand* {\tagAR}     {\glsTag    {AR}}
\newglossaryentry{AR}{
	name      ={(\AR)},
	symbol     ={\AR},
	sort    ={= 9 AR},
	see        ={Anfangsregel},
	type       ={symbols},
	description={
		Eine \Schlussregel.
	}
}
\newcommand*    {\FS}{\ensuremath{\text{FS}}}
\newcommand* {\symFS}[1][]{\glsSym[#1]{FS}}
\newcommand* {\tagFS}     {\glsTag    {FS}}
\newglossaryentry{FS}{
	name      ={(\FS)},
	symbol     ={\FS},
	sort    ={= 9 FS},
	see        ={formalerSatz},
	type       ={symbols},
	description={
		Eine \Schlussregel.
	}
}
\newcommand*    {\MR}{\ensuremath{\text{MR}}}
\newcommand* {\symMR}[1][]{\glsSym[#1]{MR}}
\newcommand* {\tagMR}     {\glsTag    {MR}}
\newglossaryentry{MR}{
	name      ={(\MR)},
	symbol     ={\MR},
	sort    ={= 9 MR},
	see        ={Monotonieregel},
	type       ={symbols},
	description={
		Eine \Schlussregel.
	}
}
\newcommand*    {\SR}{\ensuremath{\text{SR}}}
\newcommand* {\symSR}[1][]{\glsSym[#1]{SR}}
\newcommand* {\tagSR}     {\glsTag    {SR}}
\newglossaryentry{SR}{
	name      ={(\SR)},
	symbol     ={\SR},
	sort    ={= 9 SR},
	see        ={Schnittregel},
	type       ={symbols},
	description={
		Eine \Schlussregel.
	}
}
\newcommand*    {\TR}{\ensuremath{\text{TR}}}
\newcommand* {\symTR}[1][]{\glsSym[#1]{TR}}
\newcommand* {\tagTR}     {\glsTag    {TR}}
%ToDo prüfen
\newglossaryentry{TR}{
	name      ={(\TR)},
	symbol     ={\TR},
	sort    ={= 9 TR},
	see        ={Abtrennungsregel},
	type       ={symbols},
	description={
		Eine \Schlussregel.
	}
}
\newcommand*    {\andB}{\ensuremath{\FrmAnd\text{B}}}
\newcommand* {\symandB}[1][]{\glsSym[#1]{andB}}
\newcommand* {\tagandB}     {\glsTag    {andB}}
%ToDo prüfen
\newglossaryentry{andB}{
	name      ={(\andB)},
	symbol     ={\andB},
	see         ={andE},
	sort     ={= 9 1 B},
	type       ={symbols},
	description={
		Eine \Schlussregel: Beseitigung von \FrmAnd.
	}
}
\newcommand*    {\andE}{\ensuremath{\FrmAnd\text{E}}}
\newcommand* {\symandE}[1][]{\glsSym[#1]{andE}}
\newcommand* {\tagandE}     {\glsTag    {andE}}
\newglossaryentry{andE}{
	name      ={(\andE)},
	symbol     ={\andE},
	see         ={andB},
	sort     ={= 9 1 E},
	type       ={symbols},
	description={
		Eine \Schlussregel: Einführung von \FrmAnd.
	}
}
\newcommand*    {\orB}{\ensuremath{\RawFrmOr\text{B}}}
\newcommand* {\symorB}[1][]{\glsSym[#1]{orB}}
\newcommand* {\tagorB}     {\glsTag    {orB}}
\newglossaryentry{orB}{
	name      ={(\orB)},
	symbol     ={\orB},
	see         ={orE},
	sort    ={= 9 2 B},
	type       ={symbols},
	description={
		Eine \Schlussregel: Beseitigung von \FrmOr.
	}
}
\newcommand*    {\orE}{\ensuremath{\RawFrmOr\text{E}}}
\newcommand* {\symorE}[1][]{\glsSym[#1]{orE}}
\newcommand* {\tagorE}     {\glsTag    {orE}}
\newglossaryentry{orE}{
	name      ={(\orE)},
	symbol     ={\orE},
	see         ={orB},
	sort    ={= 9 2 E},
	type       ={symbols},
	description={
		Eine \Schlussregel: Einführung von \FrmOr.
	}
}
\newcommand*    {\impB}{\ensuremath{\RawFrmImp\text{B}}}
\newcommand* {\symimpB}[1][]{\glsSym[#1]{impB}}
\newcommand* {\tagimpB}     {\glsTag    {impB}}
\newglossaryentry{impB}{
	name      ={(\impB)},
	symbol     ={\impB},
	see         ={impE},
	sort     ={= 9 3 B},
	type       ={symbols},
	description={
		Eine \Schlussregel: Beseitigung von \FrmImp.
	}
}
\newcommand*    {\impE}{\ensuremath{\RawFrmImp\text{E}}}
\newcommand* {\symimpE}[1][]{\glsSym[#1]{impE}}
\newcommand* {\tagimpE}     {\glsTag    {impE}}
\newglossaryentry{impE}{
	name      ={(\impE)},
	symbol     ={\impE},
	see         ={impB},
	sort     ={= 9 3 E},
	type       ={symbols},
	description={
		Eine \Schlussregel: Einführung von \FrmImp.
	}
}
\newcommand*    {\nota}{\ensuremath{\RawFrmNot\text{1}}}
\newcommand* {\symnota}[1][]{\glsSym[#1]{nota}}
\newcommand* {\tagnota}     {\glsTag    {nota}}
\newglossaryentry{nota}{
	name      ={(\nota)},
	symbol     ={\nota},
	sort     ={= 9 4 1},
	see        ={notb,notc,notd},
	type       ={symbols},
	description={
		Eine \Schlussregel: Einführung/Beseitigung von \FrmNot\ Teil 1.
	}
}
\newcommand*    {\notb}{\ensuremath{\RawFrmNot\text{2}}}
\newcommand* {\symnotb}[1][]{\glsSym[#1]{notb}}
\newcommand* {\tagnotb}     {\glsTag    {notb}}
\newglossaryentry{notb}{
	name      ={(\notb)},
	symbol     ={\notb},
	sort     ={= 9 4 2},
	see        ={nota,notc,notd},
	type       ={symbols},
	description={
		Eine \Schlussregel: Einführung/Beseitigung von \FrmNot\ Teil 2.
	}
}
\newcommand*    {\notc}{\ensuremath{\RawFrmNot\text{3}}}
\newcommand* {\symnotc}[1][]{\glsSym[#1]{notc}}
\newcommand* {\tagnotc}     {\glsTag    {notc}}
\newglossaryentry{notc}{
	name      ={(\notc)},
	symbol     ={\notc},
	sort     ={= 9 4 3},
	see        ={nota,notb,notd},
	type       ={symbols},
	description={
		Eine \Schlussregel: Beweistechnik "`Indirekter \Beweis"'.
	}
}
\newcommand*    {\notd}{\ensuremath{\RawFrmNot\text{4}}}
\newcommand* {\symnotd}[1][]{\glsSym[#1]{notd}}
\newcommand* {\tagnotd}     {\glsTag    {notd}}
\newglossaryentry{notd}{
	name      ={(\notd)},
	symbol     ={\notd},
	sort     ={= 9 4 4},
	see        ={nota,notb,notc},
	type       ={symbols},
	description={
		Eine \Schlussregel: Reductio ad absurdum (Indirekter \Beweis).
	}
}
\newcommand*    {\eqB}{\ensuremath{\RawFrmEq\text{B}}}
\newcommand* {\symeqB}[1][]{\glsSym[#1]{eqB}}
\newcommand* {\tageqB}     {\glsTag    {eqB}}
\newglossaryentry{eqB}{
	name      ={(\eqB)},
	symbol     ={\eqB},
	see         ={eqE},
	sort    ={= 9 5 B},
	type       ={symbols},
	description={
		Eine \Schlussregel: Beseitigung von \FrmEq.
	}
}
\newcommand*    {\eqE}{\ensuremath{\RawFrmEq\text{E}}}
\newcommand* {\symeqE}[1][]{\glsSym[#1]{eqE}}
\newcommand* {\tageqE}     {\glsTag    {eqE}}
\newglossaryentry{eqE}{
	name      ={(\eqE)},
	symbol     ={\eqE},
	see         ={eqB},
	sort    ={= 9 5 E},
	type       ={symbols},
	description={
		Eine \Schlussregel: Einführung von \FrmEq.
	}
}

% Symbole für Mengen und Elemente ----------------------------------------------
% \Mts* - Ausgabe als Symbol und Aufnahme in Symbolliste und Glossar
% \Frm* - Ausgabe als Symbol und Aufnahme in Symbolliste und Glossar
% Anmerkung:
%   Eigentlich gehören die hier aufgeführten Mengen alle zur Metasprache.
%   Mengen, die zur Bildung von aussagen- und prädikatenlogischen Formeln dienen,
%   sind trotzdem mit 'Frm' statt 'Mts' markiert.

\newcommand*              {\MtsFiniteLtr}          {e}% endliche Menge
\newcommand*              {\FrmLogischLtr}         {A}% die Aussagenlogik betreffend
\newcommand*              {\FrmPolnischLtr}        {p}% ...in Polnischer Notation

\newcommand*              {\MtsINLtr}              {N}% Natürliche Zahlen
\newcommand*              {\MtsIN}[1][]{\glsSym[#1]{MtsIN}}
\newglossaryentry          {MtsIN}{
	name  ={\ensuremath{\RawMtsIN}},
	symbol={\ensuremath{\RawMtsIN}},
	sort  ={N},%           \MtsINLtr
	see   ={MtsINo},
	type  ={symbols},
	description={
		Die Menge der natürlichen Zahlen ohne 0.
	}
}
\newcommand*              {\MtsINo}[1][]{\glsSym[#1]{MtsINo}}
\newglossaryentry          {MtsINo}{
	name  ={\ensuremath{\RawMtsINo}},
	symbol={\ensuremath{\RawMtsINo}},
	sort  ={N 0},%         \MtsINLtr 0
	see   ={MtsIN},
	type  ={symbols},
	description={
		Die Menge der natürlichen Zahlen einschließlich 0.
	}
}
\newcommand*              {\FrmABCLtr}             {A}% Alphabet der aussagenlog. Sprache
\newcommand*              {\FrmABC}[1][]{\glsSym[#1]{FrmABC}}
\newglossaryentry          {FrmABC}{
	name  ={\ensuremath{\RawFrmABC}},
	symbol={\ensuremath{\RawFrmABC}},
	sort  ={A},%           \FrmABCLtr
	type  ={symbols},
	description={
		Das Alphabet der aussagenlogischen \Sprache.
	}
}
\newcommand*              {\FrmABCx}[1][]{\glsSym[#1]{FrmABCx}}
\newglossaryentry          {FrmABCx}{
	name  ={\ensuremath{\RawFrmABC_x}},
	symbol={\ensuremath{\RawFrmABC_x}},
	sort  ={A x},%         \FrmABCLtr x
	see   ={FrmABC},
	type  ={symbols},
	description={
		Eine Teilmenge des Alphabets \FrmABC\ der aussagenlogischen \Sprache.
	}
}
\newcommand*              {\FrmBinIdx}             {b}% binäre Junktoren
\newcommand*              {\FrmBin}[1][]{\glsSym[#1]{FrmBin}}
\newglossaryentry          {FrmBin}{
	%TODO \RawFrmBin = \RawFrmJun_{\IdxFt{\FrmBinIdx}} führt zu Fehlern
	name  ={\ensuremath{\SetFt{J}_{\IdxFt{\FrmBinIdx}}}},
	symbol={\ensuremath{\RawFrmBin}},
	sort  ={J b},%         \FrmJunLtr \FrmBinIdx
	see   ={FrmJun,Junktor},
	type  ={symbols},
	description={
		Die Menge der binären \Junktoren.
	}
}
\newcommand*              {\FrmConIdx}             {c}% Konstantensymbole
\newcommand*              {\FrmCon}[1][]{\glsSym[#1]{FrmCon}}
\newglossaryentry          {FrmCon}{
	%TODO \RawFrmCon = \RawFrmJun_{\IdxFt{\FrmConIdx}} führt zu Fehlern
	name  ={\ensuremath{\SetFt{J}_{\IdxFt{\FrmConIdx}}}},
	symbol={\ensuremath{\RawFrmCon}},
	sort  ={J c},%         \FrmJunLtr \FrmConIdx
	see   ={FrmJun,Junktor},
	type  ={symbols},
	description={
		Die Menge der aussagenlogischen Konstanten.
	}
}
\newcommand*              {\FrmForLtr}             {L}% language, Sprache; siehe auch \MtsSpracheLtr
\newcommand*              {\FrmFor}[1][]{\glsSym[#1]{FrmFor}}
\newglossaryentry          {FrmFor}{
	name  ={\ensuremath{\RawFrmFor}},
	symbol={\ensuremath{\RawFrmFor}},
	sort  ={L A},%         \FrmForLtr \FrmLogischLtr
	see   ={Formel,Formelmenge},
	type  ={symbols},
	description={
		Die Menge der \aussagenlogischenFormeln\ mit Klammerung.
	}
}
\newcommand*              {\FrmForp}[1][]{\glsSym[#1]{FrmForp}}
\newglossaryentry          {FrmForp}{
	name  ={\ensuremath{\RawFrmForp}},
	symbol={\ensuremath{\RawFrmForp}},
	sort  ={L Ap},%        \FrmForLtr \FrmLogischLtr\FrmPolnischLtr
	type  ={symbols},
	see   ={Formel,Formelmenge},
	description={
		Die Menge der aussagenlogischen \Formeln\ in \PolnischerNotation.
	}
}
\newcommand*              {\FrmForx}[1][]{\glsSym[#1]{FrmForx}}
\newglossaryentry          {FrmForx}{
	name  ={\ensuremath{\RawFrmFor_x}},
	symbol={\ensuremath{\RawFrmFor_x}},
	sort  ={L A x},%       \FrmForLtr \FrmLogischLtr x
	see   ={FrmFor,Formel,Formelmenge},
	type  ={symbols},
	description={
		Eine Teilmenge der Menge \FrmFor\ der aussagenlogischen \Formeln\ mit Klammerung.
	}
}
\newcommand*              {\FrmForpx}[1][]{\glsSym[#1]{FrmForpx}}
\newglossaryentry          {FrmForpx}{
	name  ={\ensuremath{\RawFrmForp_x}},
	symbol={\ensuremath{\RawFrmForp_x}},
	sort  ={L Ap x},%      \FrmForLtr \FrmLogischLtr\FrmPolnischLtr x
	see   ={FrmForp,Formel,Formelmenge},
	type  ={symbols},
	description={
		Eine Teilmenge der Menge \FrmForp\ der aussagenlogischen \Formel\ in \PolnischerNotation.
	}
}
\newcommand*              {\FrmJunLtr}             {J}% Junktoren
\newcommand*              {\FrmJun}[1][]{\glsSym[#1]{FrmJun}}
\newglossaryentry          {FrmJun}{
	name  ={\ensuremath{\RawFrmJun}},
	symbol={\ensuremath{\RawFrmJun}},
	sort  ={J},%           \FrmJunLtr
	see   ={Junktor,Junktorsymbol},
	type  ={symbols},
	description={
		Die Menge der \Junktorsymbole.
	}
}
\newcommand*              {\FrmJunx}[1][]{\glsSym[#1]{FrmJunx}}
\newglossaryentry          {FrmJunx}{
	name  ={\ensuremath{\RawFrmJun_x}},
	symbol={\ensuremath{\RawFrmJun_x}},
	sort  ={J x},%         \FrmJunLtr x
	see   ={FrmJun},
	type  ={symbols},
	description={
		Eine Teilmenge der Menge \FrmJun\ der \Junktorsymbole.
	}
}
\newcommand*              {\FrmUnaIdx}             {u}% unäre Junktoren
\newcommand*              {\FrmUna}[1][]{\glsSym[#1]{FrmUna}}
\newglossaryentry          {FrmUna}{
	name  ={\ensuremath{\RawFrmUna}},
	symbol={\ensuremath{\RawFrmUna}},
	sort  ={J u},%         \FrmJunLtr \FrmUnaIdx
	see   ={FrmJun,Junktor},
	type  ={symbols},
	description={
		Die Menge der unären \Junktoren.
	}
}
\newcommand*              {\FrmvarLtr}             {q}% Name von aussagenlogischen Variablen
\newcommand*              {\Frmvar}[1][]{\glsSym[#1]{Frmvar}}
\newglossaryentry          {Frmvar}{
	name  ={\ensuremath{\RawFrmvar}},
	symbol={\ensuremath{\RawFrmvar}},
	sort  ={q},%           \FrmvarLtr
	see   ={FrmVar,Aussagenlogik},
	type  ={symbols},
	description={
		Die $\Frmvar_i$ für $i \in \MtsINo$ sind die aussagenlogischen Variablen.
	}
}
\newcommand*              {\FrmVarLtr}             {Q}% Menge der aussagenlogischen Variablen
\newcommand*              {\FrmVar}[1][]{\glsSym[#1]{FrmVar}}
\newglossaryentry          {FrmVar}{
	name  ={\ensuremath{\RawFrmVar}},
	symbol={\ensuremath{\RawFrmVar}},
	sort  ={Q},%           \FrmVarLtr
	see   ={Frmvar,Aussagenlogik},
	type  ={symbols},
	description={
		Die Menge der aussagenlogischen Variablen $\Frmvar_i$ für $i \in \MtsINo$.
	}
}
\newcommand*              {\MtsAxiomLtr}           {X}%  A[x]iom
\newcommand*              {\MtsAxiom}[1][]{\glsSym[#1]{MtsAxiom}}
\newglossaryentry          {MtsAxiom}{
	name  ={\ensuremath{\RawMtsAxiom}},
	symbol={\ensuremath{\RawMtsAxiom}},
	sort  ={X Element},%   \MtsAxiomLtr Element
	see   ={MtsAxiomSet,Axiom},
	type  ={symbols},
	description={
		Ein \Axiom.
	}
}
\newcommand*              {\MtsAxiomSet}[1][]{\glsSym[#1]{MtsAxiomSet}}
\newglossaryentry          {MtsAxiomSet}{
	name  ={\ensuremath{\RawMtsAxiomSet}},
	symbol={\ensuremath{\RawMtsAxiomSet}},
	sort  ={X Menge},%     \MtsAxiomLtr Menge
	see   ={MtsAxiom,Axiom},
	type  ={symbols},
	description={
		Eine Menge von \Axiomen.
	}
}
\newcommand*              {\MtsBeweisschrittLtr}   {b}%           Beweisschritt
\newcommand*              {\MtsBeweisschritt}[1][]{\glsSym[#1]{MtsBeweisschritt}}
\newglossaryentry          {MtsBeweisschritt}{
	name  ={\ensuremath{\RawMtsBeweisschritt}},
	symbol={\ensuremath{\RawMtsBeweisschritt}},
	sort  ={b Element},%  \MtsBeweisschrittLtr Element
	see   ={MtsBeweisschrittTup,MtsBeweisschrittSet,Beweisschritt},
	type  ={symbols},
	description={
		Ein \Beweisschritt.
	}
}
\newcommand*              {\MtsBeweisschrittTup}[1][]{\glsSym[#1]{MtsBeweisschrittTup}}
\newglossaryentry          {MtsBeweisschrittTup}{
	name  ={\ensuremath{\RawMtsBeweisschrittTup}},
	symbol={\ensuremath{\RawMtsBeweisschrittTup}},
	sort  ={b Tupel},%     \MtsBeweisschrittLtr Tupel
	see   ={MtsBeweisschritt,MtsBeweisschrittSet,Beweisschritt},
	type  ={symbols},
	description={
		Eine Menge von \Beweisschritten.
	}
}
\newcommand*              {\MtsBeweisschrittSetLtr}{B}% Menge der    Beweisschritte
\newcommand*              {\MtsBeweisschrittSet}[1][]{\glsSym[#1]{MtsBeweisschrittSet}}
\newglossaryentry          {MtsBeweisschrittSet}{
	name  ={\ensuremath{\RawMtsBeweisschrittSet}},
	symbol={\ensuremath{\RawMtsBeweisschrittSet}},
	sort  ={B Menge},%     \MtsBeweisschrittSetLtr Menge
	see   ={MtsBeweisschritt,MtsBeweisschrittTup,Beweisschritt},
	type  ={symbols},
	description={
		Eine Menge von \Beweisschritten.
	}
}
\newcommand*              {\MtsErsetzungLtr}       {E}%   Ersetzung, Substitution
\newcommand*              {\MtsErsetzung}[1][]{\glsSym[#1]{MtsErsetzung}}
\newglossaryentry          {MtsErsetzung}{
	name  ={\ensuremath{\RawMtsErsetzung}},
	symbol={\ensuremath{\RawMtsErsetzung}},
	sort  ={E Element},%   \MtsErsetzungLtr Element
	see   ={MtsErsetzungSet,Ersetzung},
	type  ={symbols},
	description={
		Ein \Ersetzung.
	}
}
\newcommand*              {\MtsErsetzungSet}[1][]{\glsSym[#1]{MtsErsetzungSet}}
\newglossaryentry          {MtsErsetzungSet}{
	name  ={\ensuremath{\RawMtsErsetzungSet}},
	symbol={\ensuremath{\RawMtsErsetzungSet}},
	sort  ={E Menge},%     \MtsErsetzungLtr Menge
	see   ={MtsErsetzung,Ersetzung},
	type  ={symbols},
	description={
		Eine Menge von \Ersetzungen.
	}
}
\newcommand*              {\MtsFolgerungLtr}       {f}% eine         Folgerung
\newcommand*              {\MtsFolgerung}[1][]{\glsSym[#1]{MtsFolgerung}}
\newglossaryentry          {MtsFolgerung}{
	name  ={\ensuremath{\RawMtsFolgerung}},
	symbol={\ensuremath{\RawMtsFolgerung}},
	sort  ={F Element},%   \MtsFolgerungLtr Element
	see   ={MtsFolgerungSet,MtsFolgerungRel,Folgerung},
	type  ={symbols},
	description={
		Eine \Folgerung.
	}
}
\newcommand*              {\MtsFolgerungSetLtr}    {F}% Menge von    Folgerungen
\newcommand*              {\MtsFolgerungSet}[1][]{\glsSym[#1]{MtsFolgerungSet}}
\newglossaryentry          {MtsFolgerungSet}{
	name  ={\ensuremath{\RawMtsFolgerungSet}},
	symbol={\ensuremath{\RawMtsFolgerungSet}},
	sort  ={F Menge},%     \MtsFolgerungSetLtr Menge
	see   ={MtsFolgerung,MtsFolgerungRel,Folgerung},
	type  ={symbols},
	description={
		Eine Menge von \Folgerungen.
	}
}
\newcommand*              {\MtsFolgerungRel}[1][]{\glsSym[#1]{MtsFolgerungRel}}
\newglossaryentry          {MtsFolgerungRel}{
	name  ={\ensuremath{\RawMtsFolgerungRel}},
	symbol={\ensuremath{\RawMtsFolgerungRel}},
	sort  ={F Relation},%  \MtsFolgerungSetLtr Relation
	see   ={MtsFolgerung,MtsFolgerungSet,Folgerung},
	type  ={symbols},
	description={
		Eine Relation (als Menge aufgefasst) von \Folgerungen.
	}
}
\newcommand*              {\MtsMn}[1][]{\glsSym[#1]{MtsMn}}
\newglossaryentry          {MtsMn}{
	name  ={\ensuremath{\RawMtsMn}},
	symbol={\ensuremath{\RawMtsMn}},
	sort  ={M n},
	see   ={MtsMo,Tupel},
	type  ={symbols},
	description={
		Das kartesische Produkt $M \times \dots \times M$ aus $n$ Mengen $M$ mit $n \in \MtsINo$.
	}
}
\newcommand*              {\MtsMo}[1][]{\glsSym[#1]{MtsMo}}
\newglossaryentry          {MtsMo}{
	name  ={\ensuremath{\RawMtsMo}},
	symbol={\ensuremath{\RawMtsMo}},
	sort  ={M 0},
	see   ={MtsMn,Tupel},
	type  ={symbols},
	description={
		$\{()\}$, wobei $()$ das $0$-\Tupel\ ist.
	}
}
\newcommand*              {\MtsErgebnisLtr}        {e}%      Ergebnis; outcome
\newcommand*              {\MtsErgebnis}[1][]{\glsSym[#1]{MtsErgebnis}}
\newglossaryentry          {MtsErgebnis}{
	name  ={\ensuremath{\RawMtsErgebnis}},
	symbol={\ensuremath{\RawMtsErgebnis}},
	sort  ={O Element},%   \MtsErgebnisLtr Element
	see   ={MtsErgebnisSet,MtsErgebnisRel,Ergebnis},
	type  ={symbols},
	description={
		Ein \Ergebnis.
	}
}
\newcommand*              {\MtsErgebnisSetLtr}     {E}%         Ergebnismenge; outcomeset
\newcommand*              {\MtsErgebnisSet}[1][]{\glsSym[#1]{MtsErgebnisSet}}
\newglossaryentry          {MtsErgebnisSet}{
	name  ={\ensuremath{\RawMtsErgebnisSet}},
	symbol={\ensuremath{\RawMtsErgebnisSet}},
	sort  ={O Menge},%     \MtsErgebnisSetLtr Menge
	see   ={MtsErgebnis,MtsErgebnisRel,Ergebnis},
	type  ={symbols},
	description={
		Eine Menge von \Ergebnissen.
	}
}
\newcommand*              {\MtsErgebnisRel}[1][]{\glsSym[#1]{MtsErgebnisRel}}
\newglossaryentry          {MtsErgebnisRel}{
	name  ={\ensuremath{\RawMtsErgebnisRel}},
	symbol={\ensuremath{\RawMtsErgebnisRel}},
	sort  ={O Relation},%  \MtsErgebnisSetLtr Relation
	see   ={MtsErgebnis,MtsErgebnisSet,Ergebnis},
	type  ={symbols},
	description={
		Eine Relation (aufgefasst als Menge) von \Ergebnissen.
	}
}
\newcommand*              {\MtsSchlussregelLtr}    {C}%          Schlussregel; conclusionrule
\newcommand*              {\MtsSchlussregel}[1][]{\glsSym[#1]{MtsSchlussregel}}
\newglossaryentry          {MtsSchlussregel}{
	name  ={\ensuremath{\RawMtsSchlussregel}},
	symbol={\ensuremath{\RawMtsSchlussregel}},
	sort  ={C Element},%   \MtsSchlussregelLtr Element
	see   ={MtsSchlussregelSet,Schlussregel},
	type  ={symbols},
	description={
		Eine \Schlussregel.
	}
}
\newcommand*              {\MtsSchlussregelSet}[1][]{\glsSym[#1]{MtsSchlussregelSet}}
\newglossaryentry          {MtsSchlussregelSet}{
	name  ={\ensuremath{\RawMtsSchlussregelSet}},
	symbol={\ensuremath{\RawMtsSchlussregelSet}},
	sort  ={C Menge},%     \MtsSchlussregelLtr Menge
	see   ={MtsSchlussregel,Schlussregel},
	type  ={symbols},
	description={Eine Menge von \Schlussregeln.}
}
\newcommand*              {\MtsSpracheLtr}         {L}% language,Sprache; siehe auch \FrmForLtr
\newcommand*              {\MtsSprache}[1][]{\glsSym[#1]{MtsSprache}}
\newglossaryentry          {MtsSprache}{
	name  ={\ensuremath{\RawMtsSprache}},
	symbol={\ensuremath{\RawMtsSprache}},
	sort  ={L},%           \MtsSpracheLtr
	see   ={Formelmenge},
	type  ={symbols},
	description={
		Eine \Formelmenge.
	}
}
\newcommand*              {\MtsTupLtr}             {T}% sequenz; Menge der Tupel
\newcommand*              {\MtsTup}[1][]{\glsSym[#1]{MtsTup}}
\newglossaryentry          {MtsTup}{
	name  ={\ensuremath{\RawMtsTup}},
	symbol={\ensuremath{\RawMtsTup}},
	sort  ={T},%           \MtsTupLtr
	see   ={Tupel,Tupelmenge},
	type  ={symbols},
	description={
		Ein Mengenoperator: $\MtsTup(M)$ ist die Menge aller \Tupel\ von $M$.
	}
}
\newcommand*              {\MtsUmwandlungLtr}      {T}%        Umwandlung, Transformation
\newcommand*              {\MtsUmwandlung}[1][]{\glsSym[#1]{MtsUmwandlung}}
\newglossaryentry          {MtsUmwandlung}{
	name  ={\ensuremath{\RawMtsUmwandlung}},
	symbol={\ensuremath{\RawMtsUmwandlung}},
	sort  ={T Element},%   \MtsUmwandlungLtr Element
	see   ={MtsUmwandlungTup,Umwandlung},
	type  ={symbols},
	description={
		Eine \Umwandlung.
	}
}
\newcommand*              {\MtsUmwandlungTup}[1][]{\glsSym[#1]{MtsUmwandlungTup}}
\newglossaryentry          {MtsUmwandlungTup}{
	name  ={\ensuremath{\RawMtsUmwandlungTup}},
	symbol={\ensuremath{\RawMtsUmwandlungTup}},
	sort  ={T Tupel},%     \MtsUmwandlungLtr Tupel
	see   ={MtsUmwandlung,Umwandlung},
	type  ={symbols},
	description={
		Eine Menge von \Umwandlungen.
	}
}
\newcommand*              {\MtsVoraussetzungLtr}   {v}% Eine      Voraussetzung
\newcommand*              {\MtsVoraussetzung}[1][]{\glsSym[#1]{MtsVoraussetzung}}
\newglossaryentry          {MtsVoraussetzung}{
	name  ={\ensuremath{\RawMtsVoraussetzung}},
	symbol={\ensuremath{\RawMtsVoraussetzung}},
	sort  ={V Element},%   \MtsVoraussetzungLtr Element
	see   ={MtsVoraussetzungSet,MtsVoraussetzungRel,Voraussetzung},
	type  ={symbols},
	description={
		Eine \Voraussetzung.
	}
}
\newcommand*              {\MtsVoraussetzungSetLtr}{V}% Menge der    Voraussetzungen
\newcommand*              {\MtsVoraussetzungSet}[1][]{\glsSym[#1]{MtsVoraussetzungSet}}
\newglossaryentry          {MtsVoraussetzungSet}{
	name  ={\ensuremath{\RawMtsVoraussetzungSet}},
	symbol={\ensuremath{\RawMtsVoraussetzungSet}},
	sort  ={V Menge},%     \MtsVoraussetzungSetLtr Menge
	see   ={MtsVoraussetzung,MtsVoraussetzungRel,Voraussetzung},
	type  ={symbols},
	description={
		Eine Menge von \Voraussetzungen.
	}
}
\newcommand*              {\MtsVoraussetzungRel}[1][]{\glsSym[#1]{MtsVoraussetzungRel}}
\newglossaryentry          {MtsVoraussetzungRel}{
	name  ={\ensuremath{\RawMtsVoraussetzungRel}},
	symbol={\ensuremath{\RawMtsVoraussetzungRel}},
	sort  ={V Relation},%  \MtsVoraussetzungSetLtr Relation
	see   ={MtsVoraussetzung,MtsVoraussetzungSet,Voraussetzung},
	type  ={symbols},
	description={
		Eine Relation (aufgefasst als Menge) von \Voraussetzungen.
	}
}

% Operationen mit Namen (Buchstaben) -------------------------------------------
% \sym* - Ausgabe als Symbol und Aufnahme in Symbolliste und Glossar

\newcommand*              {\MtsDbTxt}              {dom}% [dom]ain; Definitionsbereich einer Funktion
\newcommand*              {\MtsDb}[1][]{\glsSym[#1]{MtsDb}}
%ToDo prüfen
\newglossaryentry          {MtsDb}{
	name  ={\ensuremath{\RawMtsDb}},
	symbol={\ensuremath{\RawMtsDb}},
	sort  ={dom},%         \MtsDbTxt
	see   ={Quellbereich,Funktion},
	type  ={symbols},
	description={
		$\MtsDb(f)$ für $f : A \rightarrow B$ ist die Menge $A$
	}
}
\newcommand*              {\MtsGraphTxt}           {graph}% Graph; Funktionen/Relationen
\newcommand*              {\MtsGraph}[1][]{\glsSym[#1]{MtsGraph}}
%ToDo prüfen
\newglossaryentry          {MtsGraph}{
	name  ={\ensuremath{\RawMtsGraph}},
	symbol={\ensuremath{\RawMtsGraph}},
	sort  ={graph},%       \MtsGraphTxt
	see   ={Funktion,Relation,Graph},
	type  ={symbols},
	description ={
		$\MtsGraph(R)$ ist der \Graph\ der \Funktion\ \textbzw\ Relation $R$.
	}
}
\newcommand*              {\MtsLenTxt}             {len}% Länge ([len]gth) Tupel/Folge
\newcommand*              {\MtsLen}[1][]{\glsSym[#1]{MtsLen}}
%ToDo prüfen
\newglossaryentry          {MtsLen}{
	name  ={\ensuremath{\RawMtsLen}},
	symbol={\ensuremath{\RawMtsLen}},
	sort  ={len},%         \MtsLenTxt
	see   ={Folge,Tupel},
	type  ={symbols},
	description={
		$\MtsLen(\vec{a})$ ist die Länge, \textdh\ die Anzahl der \Komponenten\ einer \Folge\ \textbzw\ eines \Tupels.
	}
}
\newcommand*              {\MtsPotLtr}             {P}% Potenzmenge
\newcommand*              {\MtsPot}[1][]{\glsSym[#1]{MtsPot}}
%ToDo prüfen
\newglossaryentry          {MtsPot}{
	name  ={\ensuremath{\RawMtsPot}},
	symbol={\ensuremath{\RawMtsPot}},
	sort  ={P},%           \MtsPotLtr
	see   ={MtsPotf},
	type  ={symbols},
	description={
		Menge der Teilmengen (\Potenzmenge).
	}
}
\newcommand*              {\MtsPotf}[1][]{\glsSym[#1]{Dummy}}
%ToDo prüfen
%%%%\newglossaryentry          {MtsPotf}{
%%%%	name  ={\ensuremath{\RawMtsPotf}},
%%%%	symbol={\ensuremath{\RawMtsPotf}},
%%%%	sort  ={P e},%         \MtsPotLtr \MtsFiniteLtr
%%%%	see   ={MtsPot,Potenzmenge},
%%%%	type  ={symbols},
%%%%	description={
%%%%		Menge der endlichen Teilmengen.
%%%%	}
%%%%}
\newcommand*              {\MtsQbTxt}              {src}% Quellbereich ([s]ou[rc]e) einer partiellen Fkt.
\newcommand*              {\MtsQb}[1][]{\glsSym[#1]{MtsQb}}
%ToDo prüfen
\newglossaryentry          {MtsQb}{
	name  ={\ensuremath{\RawMtsQb}},
	symbol={\ensuremath{\RawMtsQb}},
	sort  ={src},%         \MtsQbTxt
	see   ={Definitionsbereich,Funktion},
	type  ={symbols},
	description={
		$\MtsQb(f)$ für $f : A \rightarrow B$ ist die Menge $\{a \in A | f(a) \text{ existiert\}}$.
	}
}
\newcommand*              {\MtsRelLtr}             {R}% Menge der Relationen
\newcommand*              {\MtsRel}[1][]{\glsSym[#1]{MtsRel}}
%ToDo prüfen
\newglossaryentry          {MtsRel}{
	name  ={\ensuremath{\RawMtsRel}},
	symbol={\ensuremath{\RawMtsRel}},
	sort  ={R},%           \MtsRelLtr
	see   ={MtsRelf,Relation},
	type  ={symbols},
	description={
		Menge der binären Relationen.
	}
}
\newcommand*              {\MtsRelf}[1][]{\glsSym[#1]{dummy}}
%ToDo prüfen
%%%%\newglossaryentry          {MtsRelf}{
%%%%	name  ={\ensuremath{\RawMtsRelf}},
%%%%	symbol={\ensuremath{\RawMtsRelf}},
%%%%	sort  ={R e},%         \MtsRelLtr\MtsFiniteLtr
%%%%	see   ={MtsRel,Relation},
%%%%	type  ={symbols},
%%%%	description={
%%%%		Menge der endlichen binären Relationen.
%%%%	}
%%%%}
\newcommand*              {\MtsSetTxt}             {set}% Komponentenmenge Tupel/Folge
\newcommand*              {\MtsSet}[1][]{\glsSym[#1]{MtsSet}}
%ToDo prüfen
\newglossaryentry          {MtsSet}{
	name  ={\ensuremath{\RawMtsSet}},
	symbol={\ensuremath{\RawMtsSet}},
	sort  ={Set},%         \MtsSetTxt
	see   ={Folge,Tupel},
	type  ={symbols},
	description={
		$\MtsSet(\vec{a})$ ist die Menge der \Komponenten\ eine \Folge\ \textbzw\ eines \Tupels.
	}
}
\newcommand*              {\MtsStelTxt}            {stel}% [Stel]ligkeit Funktionen/Relationen
\newcommand*              {\MtsStelF}[1][]{\glsSym[#1]{MtsStelF}}
%ToDo prüfen
\newglossaryentry          {MtsStelF}{
	name  ={\ensuremath{\RawMtsStelF}},
	symbol={\ensuremath{\RawMtsStelF}},
	sort  ={stel f},%   \MtsStelTxt             f
	see   ={Funktion,MtsStelR},
	type  ={symbols},
	description={
		\Stelligkeit\ einer \Funktion.
	}
}
\newcommand*              {\MtsStelR}[1][]{\glsSym[#1]{MtsStelR}}
%ToDo prüfen
\newglossaryentry          {MtsStelR}{
	name  ={\ensuremath{\RawMtsStelR}},
	symbol={\ensuremath{\RawMtsStelR}},
	sort  ={stel r},%   \MtsStelTxt             r
	see   ={Relation,MtsStelF},
	type  ={symbols},
	description={
		\Stelligkeit\ einer \Relation.
	}
}
\newcommand*              {\MtsTraegerTxt}         {car}% ([car]rier) Trägermenge einer Relation
\newcommand*              {\MtsTraeger}[1][]{\glsSym[#1]{MtsTraeger}}
%ToDo prüfen
\newglossaryentry          {MtsTraeger}{
	name  ={\ensuremath{\RawMtsTraeger}},
	symbol={\ensuremath{\RawMtsTraeger}},
	sort  ={car},%         \MtsTraegerTxt
	see   ={Traegermenge,Relation},
	type  ={symbols},
	description={
		$\MtsTraeger_i(R)$ für $R \MtsSubsetEq A_1 \times \cdots \times A_n$ ist die \Traegermenge\ $A_i$ für $i$ von $1$ bis $n$.
	}
}
\newcommand*              {\MtsWbTxt}              {ran}% Wertebereich ([ran]ge) einer Funktion
\newcommand*              {\MtsWb}[1][]{\glsSym[#1]{MtsWb}}
%ToDo prüfen
\newglossaryentry          {MtsWb}{
	name  ={\ensuremath{\RawMtsWb}},
	symbol={\ensuremath{\RawMtsWb}},
	sort  ={ran},%         \MtsWbTxt
	see   ={Zielbereich,Funktion},
	type  ={symbols},
	description={
		$\MtsWb(f)$ für $f : A \rightarrow B$ ist die Menge $\{f(a) | a \in A\}$.
	}
}
\newcommand*              {\MtsZbTxt}              {tar}% Zielbereich ([tar]get) einer Funktion
\newcommand*              {\MtsZb}[1][]{\glsSym[#1]{MtsZb}}
%ToDo prüfen
\newglossaryentry          {MtsZb}{
	name  ={\ensuremath{\RawMtsZb}},
	symbol={\ensuremath{\RawMtsZb}},
	sort  ={tar},%         \MtsZbTxt
	see   ={Wertebereich,Funktion},
	type  ={symbols},
	description={
		$\MtsZb(f)$ für $f : A \rightarrow B$ ist die Menge $B$
	}
}

% Individuelle Bezeichnungen ---------------------------------------------------
% \Txt* - Ausgabe als Text und Aufnahme in Symbolverzeichnis bzw. Glossar und Index

\newcommand*              {\MtsFalseTxt}           {false}
\newcommand*              {\MtsFalse}[1][]{\glsSym[#1]{MtsFalse}}
%ToDo prüfen
\newglossaryentry          {MtsFalse}{
	name  ={\ensuremath{\RawMtsFalse}},
	symbol={\ensuremath{\RawMtsFalse}},
	sort  ={false},%       \MtsFalseTxt
	see   ={FrmFalse,MtsTrue},
	type  ={symbols},
	description={
		Der metasprachliche \Wahrheitswert \TxtFalse\ als \Symbol.
	}
}
\newcommand*              {\MtsTrueTxt}            {true}
\newcommand*              {\MtsTrue}[1][]{\glsSym[#1]{MtsTrue}}
%ToDo prüfen
\newglossaryentry          {MtsTrue}{
	name  ={\ensuremath{\RawMtsTrue}},
	symbol={\ensuremath{\RawMtsTrue}},
	sort  ={true},%        \MtsTrueTxt
	see   ={FrmTrue,MtsFalse},
	type  ={symbols},
	description={
		Der metasprachliche \Wahrheitswert \TxtTrue\ als \Symbol.
	}
}
\newcommand*              {\TxtFalseTxt}           {falsch}
\newcommand*              {\TxtFalse}[1][]{\glsIdx[#1]{Dummy}}
\renewcommand*{\TxtFalse}{falsch}
%ToDo prüfen
\newglossaryentry          {TxtFalse}{
	name       =       {\RawTxtFalse},
	sort       ={falsch},% \TxtFalseTxt
	see        ={TxtTrue,MtsFalse,FrmFalse},
	description={
		Ein metasprachlicher \Wahrheitswert\ in Textform.
	}
}

\newcommand*              {\TxtTrueTxt}            {wahr}
\newcommand*              {\TxtTrue}[1][]{\glsIdx[#1]{Dummy}}
\renewcommand*{\TxtTrue}{wahr}
%ToDo prüfen
\newglossaryentry          {TxtTrue}{
	name       =       {\RawTxtTrue},
	sort       ={wahr},%   \TxtTrueTxt
	see        ={TxtFalse,MtsTrue,FrmTrue},
	description={
		Ein metasprachlicher \Wahrheitswert\ in Textform.
	}
}

% Fachbegriffe -----------------------------------------------------------------
% Hilfsmakros:       Glossary-   Index-Eintrag  Textausgabe
%   \glsIdx  {key}   name        name           text
%   \glsIdxI {key}   name        name           user1
%   \GlsIdxPl{key}   name        name           Plural

%A === A === A === A === A === A === A === A === A === A === A === A === A === A

\newcommand*    {\ASBA}[1][hyperTxt]{\textbf{\glsIdx[#1]{ASBA}}}
\newglossaryentry{ASBA}{
	name        ={ASBA},
	description ={
		ist ein Akronym für "`\textbf{A}xiome, \textbf{S}ätze, \textbf{B}eweise und \textbf{A}uswertungen"'.
		Es bezeichnet das in diesem Dokument beschriebene Programmsystem, das zu eingegebenen \Axiomen, \Saetzen\ und \Beweisen\ letztere prüft, Auswertungen generiert und unter Zuhilfenahme gegebener \Ausgabeschemata\ eine Ausgabe im \LaTeX-Format in mathematisch üblicher Schreibweise mit \Formeln\ erstellt.
	}
}
\newcommand*    {\ableitbar} [1][hyperTxt]{\glsIdx  [#1]{ableitbar}}
\newcommand*    {\ableitbare}[1][hyperTxt]{\glsIdxPl[#1]{ableitbar}}
%ToDo prüfen
\newglossaryentry{ableitbar}{
	name        ={ableitbar},
	plural      ={ableitbare},
	see         ={Ableitungsrelation},
	description ={
		Synonym zu \beweisbar\ ---
		Wenn sich eine \Formel\ $\beta$ aus einer anderen \Formel\ $\alpha$ mittels \zulaessiger\ \Umwandlungen\ ableiten lässt, heißt $\beta$ \ableitbar\ aus $\alpha$.
		Sprechweise: \seqqt{$ \alpha \text{ ableitbar } \beta $}.
		Eine oder beide \Formeln\ $\alpha$ \textbzw\ $\beta$ dürfen dabei durch \Formelmengen\ ersetzt werden.
	}
}
\newcommand*    {\Ableitung}  [1][hyperTxt]{\glsIdx  [#1]{Ableitung}}
\newcommand*    {\Ableitungen}[1][hyperTxt]{\glsIdxPl[#1]{Ableitung}}
%ToDo prüfen
\newglossaryentry{Ableitung}{
	name        ={Ableitung},
	plural      ={Ableitungen},
	see         ={Ableitungsmenge,Ableitungsrelation},
	description ={
		Eine \Aussage\ $A \MtsDerive B$ \textbzw\ allgemeiner $A \MtsDeriveR B$ mit $A,B \MtsSubsetEq \MtsSprache$.
		Dies entspricht einem Element $(A,B)$ einer \Ableitungsrelation\ \MtsDerive\ \textbzw\ \MtsDeriveR (\textdh\ $(A,B) \in R$.
		Die semantische Aussage ist die, das die \Formeln\ aus $B$ aus den \Formeln\ aus $A$ abgeleitet werden können.
	}
}
\newcommand*    {\Ableitungsmenge} [1][hyperTxt]{\glsIdx  [#1]{Ableitungsmenge}}
\newcommand*    {\Ableitungsmengen}[1][hyperTxt]{\glsIdxPl[#1]{Ableitungsmenge}}
%ToDo prüfen
\newglossaryentry{Ableitungsmenge}{
	name        ={Ableitungsmenge},
	plural      ={Ableitungsmengen},
	description ={
		Eine Menge von \Ableitungen, letztlich nichts anderes als eine \Ableitungsrelation.
	}
}
\newcommand*    {\Ableitungsrelation}  [1][hyperTxt]{\glsIdx  [#1]{Ableitungsrelation}}
\newcommand*    {\Ableitungsrelationen}[1][hyperTxt]{\glsIdxPl[#1]{Ableitungsrelation}}
%ToDo prüfen
\newglossaryentry{Ableitungsrelation}{
	name        ={Ableitungsrelation},
	plural      ={Ableitungsrelationen},
	see         ={Ableitung},
	description ={
		Eine binäre \Relation\ \MtsDerive\ aus \MtsAllDerive.
		Für $R \in \MtsAllDerive$ auch mit \MtsDeriveR\ bezeichnet.
	}
}
\newcommand*    {\Abtrennungsregel}[1][hyperTxt]{\glsIdx[#1]{Abtrennungsregel}}
%ToDo prüfen
\newglossaryentry{Abtrennungsregel}{
	name        ={Abtrennungsregel},
	see         ={TR},
	description ={
		Eine \Schlussregel.
	}
}
\newcommand*    {\Aequivalenz}  [1][hyperTxt]{\glsIdx  [#1]{Aequivalenz}}
\newcommand*    {\Aequivalenzen}[1][hyperTxt]{\glsIdxPl[#1]{Aequivalenz}}
%ToDo prüfen
\newglossaryentry{Aequivalenz}{
	name        ={Äquivalenz},
	plural      ={Äquivalenzen},
	see         ={MtsAequiv},
	description ={
		Eine \Gleichheitsrelation:
		Zwei Objekte $A$ und $B$ sind \emph{äquivalent}\alternativ{ähnlich}, $A \MtsAequiv B$, wenn sie in den \interessierendenEigenschaften\ für \MtsAequiv\ übereinstimmen.
	}
}
\newcommand*    {\Aequivalenzrelation}  [1][hyperTxt]{\glsIdx  [#1]{Aequivalenzrelation}}
\newcommand*    {\Aequivalenzrelationen}[1][hyperTxt]{\glsIdxPl[#1]{Aequivalenzrelation}}
%ToDo prüfen
\longnewglossaryentry{Aequivalenzrelation}{
	name            ={Äquivalenzrelation},
	plural          ={Äquivalenzrelationen}
}{
	Eine binäre \Relation\ \BspRel\ auf einer Menge $M$ mit folgenden Eigenschaften:

	\textbf{reflexiv}    : $\quad   a \BspRel a$ \\
	\textbf{transitiv}   : $\quad ((a \BspRel b) \MtsAnd (b \BspRel c)) \MtsImp (a \BspRel c)$\\
	\textbf{symmetrisch} : $\quad  (a \BspRel b) \MtsImp (b \BspRel a)$

	jeweils für alle Elemente $a$, $b$ und $c$ aus $M$.
}
\newcommand*    {\Anfangsregel}[1][hyperTxt]{\glsIdx  [#1]{Anfangsregel}}
%ToDo prüfen
\newglossaryentry{Anfangsregel}{
	name        ={Anfangsregel},
	description ={
		Die \Schlussregel\ \AR\ um anfangen zu können.
	}
}
\newcommand*    {\atomar} [1][hyperTxt]{\glsIdx  [#1]{atomar}}
\newcommand*    {\atomare}[1][hyperTxt]{\glsIdxPl[#1]{atomar}}
%ToDo prüfen
\newglossaryentry{atomar}{
	name        ={atomar},
	plural      ={atomare},
	see         ={zerlegbar,unzerlegbar},
	description ={
		Synonym zu \unzerlegbar.
		Das Attribut betrifft \Aussagen\ und \Formeln.
	}
}
\newcommand*    {\Ausgabeschema}  [1][hyperTxt]{\glsIdx  [#1]{Ausgabeschema}}
\newcommand*    {\Ausgabeschemata}[1][hyperTxt]{\glsIdxPl[#1]{Ausgabeschema}}
%ToDo prüfen
\newglossaryentry{Ausgabeschema}{
	name        ={Ausgabeschema},
	plural      ={Ausgabeschemata},
	description ={
		Ein Schema, mit dem bestimmte mathematische \Objekte\ ausgegeben werden sollen.
	}
}
\newcommand*    {\Aussage} [1][hyperTxt]{\glsIdx  [#1]{Aussage}}
\newcommand*    {\Aussagen}[1][hyperTxt]{\glsIdxPl[#1]{Aussage}}
%ToDo prüfen
\newglossaryentry{Aussage}{
	name        ={Aussage},
	plural      ={Aussagen},
	description ={
		Eine \Aussage\ in natürlicher Sprache oder als \Formel, die einen \Wahrheitswert\ liefert.
	}
}
\newcommand*        {\Aussagenlogik}[1][hyperTxt]{\glsIdx [#1]{Aussagenlogik}}
\newcommand*        {\AussagenL}    [1][hyperTxt]{\glsIdxI[#1]{Aussagenlogik}}
\longnewglossaryentry{Aussagenlogik}{
	name            ={Aussagenlogik},
	user1           ={Aussagen-},
	see             ={Logik,Praedikatenlogik}
}{
	\Wikipedia~\cite{bib:Aussagenlogik} schreibt dazu (Zitat ohne Verweise ins Internet):

	Die \textbf{Aussagenlogik} ist ein Teilgebiet der Logik, das sich mit Aussagen und deren Verknüpfung durch Junktoren befasst, ausgehend von strukturlosen Elementaraussagen (Atomen), denen ein Wahrheitswert zugeordnet wird. In der \textit{klassischen Aussagenlogik} wird jeder Aussage genau einer der zwei Wahrheitswerte „wahr“ und „falsch“ zugeordnet. Der Wahrheitswert einer zusammengesetzten Aussage lässt sich ohne zusätzliche Informationen aus den Wahrheitswerten ihrer Teilaussagen bestimmen.\par
}
\newcommand*    {\Axiom}  [1][hyperTxt]{\glsIdx  [#1]{Axiom}}
\newcommand*    {\Axiome} [1][hyperTxt]{\glsIdxPl[#1]{Axiom}}
\newcommand*    {\Axiomen}[1][hyperTxt]{\glsIdxPl[#1]{Axiom}[n]}
%ToDo prüfen
\newglossaryentry{Axiom}{
	name        ={Axiom},
	plural      ={Axiome},
	see         ={MtsAxiom,MtsAxiomSet},
	description ={
		Eine \Formel, die unbewiesen als wahr angesehen wird.
	}
}
\newcommand*    {\Axiomensystem} [1][hyperTxt]{\glsIdx  [#1]{Axiomensystem}}
\newcommand*    {\Axiomensysteme}[1][hyperTxt]{\glsIdxPl[#1]{Axiomensystem}}
%ToDo prüfen
\newglossaryentry{Axiomensystem}{
	name        ={Axiomensystem},
	plural      ={Axiomensysteme},
	description ={
		Eine Menge von \Axiomen.
	}
}

%B === B === B === B === B === B === B === B === B === B === B === B === B === B

\newcommand*    {\Basisregel} [1][hyperTxt]{\glsIdx  [#1]{Basisregel}}
\newcommand*    {\Basisregeln}[1][hyperTxt]{\glsIdxPl[#1]{Basisregel}}
%ToDo prüfen
\newglossaryentry{Basisregel}{
	name        ={Basisregel},
	plural      ={Basisregeln},
	description ={
		Eine \Schlussregel, die nicht mehr auf andere zurückgeführt wird.
		Obwohl das auch auf die \Identitaetsregeln\ zutrifft, werden diese hier aber nicht dazu gezählt.
	}
}
\newcommand*    {\beschraenkt}  [1][hyperTxt]{\glsIdx  [#1]{beschraenkt}}
\newcommand*    {\beschraenkte} [1][hyperTxt]{\glsIdxPl[#1]{beschraenkt}}
\newcommand*    {\beschraenkten}[1][hyperTxt]{\glsIdxPl[#1]{beschraenkt}[n]}
%ToDo prüfen
\newglossaryentry{beschraenkt}{
	name        ={beschränkt},
	plural      ={beschränkte},
	description ={
		Eine \Schlussregel\ heißt \beschraenkt, wenn sie nur endlich viele Voraussetzungen und Folgerungen hat.
	}
}
\newcommand*    {\Beweis}  [1][hyperTxt]{\glsIdx  [#1]{Beweis}}
\newcommand*    {\Beweise} [1][hyperTxt]{\glsIdxPl[#1]{Beweis}}
\newcommand*    {\Beweises}[1][hyperTxt]{\glsIdx  [#1]{Beweis}[es]}
\newcommand*    {\Beweisen}[1][hyperTxt]{\glsIdxPl[#1]{Beweis}[n]}
%ToDo prüfen
\newglossaryentry{Beweis}{
	name        ={Beweis},
	plural      ={Beweise},
	description ={
		Eine zulässige Ableitung von \Folgerungen\ aus gegebenen \Voraussetzungen.
	}
}
\newcommand*    {\beweisbar} [1][hyperTxt]{\glsIdx  [#1]{beweisbar}}
\newcommand*    {\beweisbare}[1][hyperTxt]{\glsIdxPl[#1]{beweisbar}}
%ToDo prüfen
\newglossaryentry{beweisbar}{
	name        ={beweisbar},
	plural      ={beweisbare},
	description ={
		Synonym zu \ableitbar.
	}
}
\newcommand*    {\Beweisschritt}  [1][hyperTxt]{\glsIdx  [#1]{Beweisschritt}}
\newcommand*    {\Beweisschritte} [1][hyperTxt]{\glsIdxPl[#1]{Beweisschritt}}
\newcommand*    {\Beweisschritten}[1][hyperTxt]{\glsIdxPl[#1]{Beweisschritt}[n]}
%ToDo prüfen
\newglossaryentry{Beweisschritt}{
	name        ={Beweisschritt},
	plural      ={Beweisschritte},
	see         ={MtsBeweisschritt,MtsBeweisschrittSet,MtsBeweisschrittTup},
	symbol      ={\ensuremath{\RawMtsBeweisschritt}},
	description ={
		Eine Vorschrift, wie aus vorgegebenen \Aussagen\ (den \Voraussetzungen) weitere (die \Folgerungen) folgen.
	}
}
\newcommand*    {\Beweisschrittfolge} [1][hyperTxt]{\glsIdx  [#1]{Beweisschrittfolge}}
\newcommand*    {\Beweisschrittfolgen}[1][hyperTxt]{\glsIdxPl[#1]{Beweisschrittfolge}}
%ToDo prüfen
\newglossaryentry{Beweisschrittfolge}{
	name        ={Beweisschrittfolge},
	plural      ={Beweisschrittfolgen},
	description ={
		Eine Folge von \Beweisschritten.
	}
}
\newcommand*    {\Beweisschrittmenge} [1][hyperTxt]{\glsIdx  [#1]{Beweisschrittmenge}}
\newcommand*    {\Beweisschrittmengen}[1][hyperTxt]{\glsIdxPl[#1]{Beweisschrittmenge}}
%ToDo prüfen
\newglossaryentry{Beweisschrittmenge}{
	name        ={Beweisschrittmenge},
	plural      ={Beweisschrittmengen},
	description ={
		Eine Menge von \Beweisschritten, insbesondere die Menge der Glieder einer \Beweisschrittfolge.
	}
}
\newcommand*    {\binaer} [1][hyperTxt]{\glsIdx  [#1]{binaer}}
\newcommand*    {\binaere}[1][hyperTxt]{\glsIdxPl[#1]{binaer}}
%ToDo prüfen
\newglossaryentry{binaer}{
	name        ={binär},
	plural      ={binäre},
	see         ={unaer},
	description ={
		Eine \Operation\, \Funktion\ oder \Relation\ heißt \defFt{binär}, wenn ihre \Stelligkeit\ gleich 2 ist.
	}
}

%D === D === D === D === D === D === D === D === D === D === D === D === D === D

\newcommand*    {\Definition}  [1][hyperTxt]{\glsIdx  [#1]{Definition}}
\newcommand*    {\Definitionen}[1][hyperTxt]{\glsIdxPl[#1]{Definition}}
%ToDo prüfen
\newglossaryentry{Definition}{
	name        ={Definition},
	plural      ={Definitionen},
	see         ={Metadefinition},
	description ={
		Eine Definition mit Hilfe des Symbols \chrqt{\MtsDefEq}.
		\seqqt{$A \MtsDefEq B$} steht für \standsfor{$A$ \emph{ist definitionsgemäß gleich} $B$} für \Objekte\ $A$ und $B$.
		Gewissermaßen ist $A$ nur eine andere Schreibweise für $B$.
	}
}
\newcommand*    {\Definitionsbereich} [1][hyperTxt]{\glsIdx  [#1]{Definitionsbereich}}
\newcommand*    {\Definitionsbereiche}[1][hyperTxt]{\glsIdxPl[#1]{Definitionsbereich}}
%ToDo prüfen
\newglossaryentry{Definitionsbereich}{
	name        ={Definitionsbereich},
	plural      ={Definitionsbereiche},
	symbol      ={\MtsDb},
	see         = {MtsDb,Quellbereich,Funktion},
	description ={
		einer \Funktion.
	}
}

%E === E === E === E === E === E === E === E === E === E === E === E === E === E

\newcommand*    {\interessierendeEigenschaft}   [1][hyperTxt]{\glsIdx  [#1]{interessierendeEigenschaft}}
\newcommand*    {\interessierendenEigenschaft}  [1][hyperTxt]{\glsIdxPl[#1]{interessierendeEigenschaft}}
\newcommand*    {\interessierendenEigenschaften}[1][hyperTxt]{\glsIdxPl[#1]{interessierendeEigenschaft}[en]}
%ToDo prüfen
\newglossaryentry{interessierendeEigenschaft}{
	name        =                 {Eigenschaft, interessierende},
	sort        =                 {Eigenschaft, interessierende},
	text        ={interessierende  Eigenschaft},
	plural      ={interessierenden Eigenschaft},% Dativ
	description ={
		Solche Eigenschaften von \Objekten, die im aktuellen Zusammenhang von Interesse sind, \textzB\ einen bestimmten Wert zu haben, Element einer bestimmten Menge zu sein, ein bestimmtes \Objekt\ zu bezeichnen, usw.
	}
}
\newcommand*    {\Ergebnis}   [1][hyperTxt]{\glsIdx  [#1]{Ergebnis}}
\newcommand*    {\Ergebnisse} [1][hyperTxt]{\glsIdxPl[#1]{Ergebnis}}
\newcommand*    {\Ergebnissen}[1][hyperTxt]{\glsIdxPl[#1]{Ergebnis}[n]}
%ToDo prüfen
\newglossaryentry{Ergebnis}{
	name        ={Ergebnis},
	plural      ={Ergebnisse},
	see         ={MtsErgebnis,MtsErgebnisSet,MtsErgebnisRel},
	description ={
		Eine \Ableitung:
		Ein \Ergebnis\ eines \Beweises.
	}
}
\newcommand*    {\Ergebnismenge} [1][hyperTxt]{\glsIdx  [#1]{Ergebnismenge}}
\newcommand*    {\Ergebnismengen}[1][hyperTxt]{\glsIdxPl[#1]{Ergebnismenge}}
%ToDo prüfen
\newglossaryentry{Ergebnismenge}{
	name        ={Ergebnismenge},
	plural      ={Ergebnismengen},
	description ={
		Eine \Ableitungsmenge:
		Die Menge \MtsErgebnisSet\ der \Ergebnisse\ eines \Beweises.
	}
}
\newcommand*    {\Ersetzung}  [1][hyperTxt]{\glsIdx  [#1]{Ersetzung}}
\newcommand*    {\Ersetzungen}[1][hyperTxt]{\glsIdxPl[#1]{Ersetzung}}
%ToDo prüfen
\newglossaryentry{Ersetzung}{
	name        ={Ersetzung},
	plural      ={Ersetzungen},
	description ={
		Eine \Funktion\ zur \Umwandlung\ einer \Formel\ mittels \Ersetzung\ in eine gleichwertige.
		Die \Ersetzung\ heißt \zulaessig, wenn sie vorgegebene Regeln erfüllt.
	}
}
\newcommand*    {\Ersetzungsmenge} [1][hyperTxt]{\glsIdx  [#1]{Ersetzungsmenge}}
\newcommand*    {\Ersetzungsmengen}[1][hyperTxt]{\glsIdxPl[#1]{Ersetzungsmenge}}
%ToDo prüfen
\newglossaryentry{Ersetzungsmenge}{
	name        ={Ersetzungsmenge},
	plural      ={Ersetzungsmengen},
	description ={
		Eine Menge von \Ersetzungen, meistens mit \MtsErsetzungSet bezeichnet.
	}
}

%F === F === F === F === F === F === F === F === F === F === F === F === F === F

\newcommand*    {\Fachbegriff}  [1][hyperTxt]{\glsIdx  [#1]{Fachbegriff}}
\newcommand*    {\Fachbegriffe} [1][hyperTxt]{\glsIdxPl[#1]{Fachbegriff}}
\newcommand*    {\Fachbegriffen}[1][hyperTxt]{\glsIdxPl[#1]{Fachbegriff}[n]}
%ToDo prüfen
\newglossaryentry{Fachbegriff}{
	name        ={Fachbegriff},
	plural      ={Fachbegriffe},
	description ={
		Ein Name für einen mathematischen Begriff.
	}
}
\newcommand*    {\Fachgebiet}  [1][hyperTxt]{\glsIdx  [#1]{Fachgebiet}}
\newcommand*    {\Fachgebiete} [1][hyperTxt]{\glsIdxPl[#1]{Fachgebiet}}
\newcommand*    {\Fachgebieten}[1][hyperTxt]{\glsIdxPl[#1]{Fachgebiet}[n]}
%ToDo prüfen
\newglossaryentry{Fachgebiet}{
	name        ={Fachgebiet},
	plural      ={Fachgebiete},
	description ={
		Ein Teil der Mathematik mit einer zugehörigen Basis aus \Axiomen, \Saetzen, \Fachbegriffen\ und Darstellungsweisen.
	}
}
\newcommand*    {\Folge} [1][hyperTxt]{\glsIdx  [#1]{Folge}}
\newcommand*    {\Folgen}[1][hyperTxt]{\glsIdxPl[#1]{Folge}}
%ToDo prüfen
\newglossaryentry{Folge}{
	name        ={Folge},
	plural      ={Folgen},
	see         ={MtsLen,leereFolge,Tupel},
	description ={
		Ein \Folge\alternativ{Sequenz} $\vec{a}$ ist eine Aneinanderreihung von \defFt{\Komponenten} $a_i$, $i \in \MtsINo$, geschrieben $(a_1, a_2, \dots)$.
		Sind alle \Komponenten\ Elemente einer Menge $M$, so heißt $\vec{a}$ ein \Folge\ \defFt{auf} $M$.
		Bricht die \Folge\ ab, \textdh\ gibt es ein $n \in \MtsINo$ mit $\vec{a} = (a_1, \dots, a_n)$, so heißt die \Folge\ \defFt{endlich} von der \defFt{Länge} $n$.
		Ist die Länge $n = 0$, so sprechen wir von der \defFt{\leerenFolge} und bezeichnen sie mit \seqqt{$()$}.
		Eine endliche \Folge\ der Länge $n$ heißt auch \defFt{$n$-\Tupel} und die leere \Folge\ demnach \defFt{$0$-\Tupel}.
	}
}
\newcommand*    {\leereFolge} [1][hyperTxt]{\glsIdx  [#1]{leereFolge}}
\newcommand*    {\leereFolgen}[1][hyperTxt]{\glsIdx  [#1]{leereFolge}[n]}
\newcommand*    {\leerenFolge}[1][hyperTxt]{\glsIdxPl[#1]{leereFolge}}
%ToDo prüfen
\newglossaryentry{leereFolge}{
	name        =           {-, leere},
	sort        =       {Folge, leere},
	text        ={leere  Folge},
	plural      ={leeren Folge},% Singular Dativ
	see         ={MtsLen,Folge,Tupel},
	description ={
		Eine \Folge\ heißt \defFt{leer}, wenn ihre Länge $0$ ist, \textdh\ wenn sie keine \Komponenten\ besitzt.
	}
}
\newcommand*    {\Folgerung}  [1][hyperTxt]{\glsIdx  [#1]{Folgerung}}
\newcommand*    {\Folgerungen}[1][hyperTxt]{\glsIdxPl[#1]{Folgerung}}
%ToDo prüfen
\newglossaryentry{Folgerung}{
	name        ={Folgerung},
	plural      ={Folgerungen},
	see         ={Schlussregel},
	description ={
		Eine \Ableitung:
		Die \Folgerungen\ einer \Schlussregel\ $\frac{\MtsVoraussetzungSet}{\MtsFolgerungSet}$ \textbzw\ $\frac{\MtsVoraussetzungSet}{\MtsFolgerungSet}$ sind die Elemente aus \MtsFolgerungSet\ \textbzw\ \MtsFolgerungRel.
		Die \Voraussetzungen\ werden normalerweise mit $\MtsVoraussetzung_i$ bezeichnet.
	}
}
\newcommand*    {\Folgerungsmenge} [1][hyperTxt]{\glsIdx  [#1]{Folgerungsmenge}}
\newcommand*    {\Folgerungsmengen}[1][hyperTxt]{\glsIdxPl[#1]{Folgerungsmenge}}
%ToDo prüfen
\newglossaryentry{Folgerungsmenge}{
	name        ={Folgerungsmenge},
	plural      ={Folgerungsmengen},
	description ={
		Eine \Ableitungsmenge:
		Die Menge \MtsFolgerungSet\ der \Folgerungen\ einer \Schlussregel\ \textbzw\ eines \Beweises.
	}
}
\newcommand*    {\Formel} [1][hyperTxt]{\glsIdx  [#1]{Formel}}
\newcommand*    {\Formeln}[1][hyperTxt]{\glsIdxPl[#1]{Formel}}
%ToDo prüfen
\newglossaryentry{Formel}{
	name        ={Formel},
	plural      ={Formeln},
	description ={
		Unter einer \Formel\ verstehen wir stets eine mathematische \Formel.
		Diese kann aus einem einzigen \Symbol\ bestehen (\atomare\ \Formel), andererseits aber auch mehrdimensional sein, lässt sich dann aber mittels geeigneter \Definitionen\ immer eindeutig als eine \Zeichenfolge\ schreiben.
		\Saetze, \Beweise\ und \Schlussregeln\ betrachten wir \emph{nicht} als \Formeln.
	}
}
\newcommand*    {\allgemeingueltigeFormel} [1][hyperTxt]{\glsIdx  [#1]{allgemeingueltigeFormel}}
\newcommand*    {\allgemeingueltigenFormel}[1][hyperTxt]{\glsIdxPl[#1]{allgemeingueltigeFormel}}
%ToDo prüfen
\newglossaryentry{allgemeingueltigeFormel}{
	name        =                       {-, allgemeingültige},
	sort        =                  {Formel, allgemeingültige},
	text        ={allgemeingültige  Formel},
	plural      ={allgemeingültigen Formel},% Singular Dativ
	description ={
		Eine \Formel\ heißt \defFt{allgemeingültig}, wenn sie aus den \Axiomen\ und \allgemeingueltigenSchlussregeln\ abgeleitet werden kann.
	}
}
\newcommand*    {\aussagenlogischeFormel}  [1][hyperTxt]{\glsIdx  [#1]{aussagenlogischeFormel}}
\newcommand*    {\aussagenlogischenFormeln}[1][hyperTxt]{\glsIdxPl[#1]{aussagenlogischeFormel}[n]}
%ToDo prüfen
\newglossaryentry{aussagenlogischeFormel}{
	name        =                       {-, aussagenlogische},
	sort        =                  {Formel, aussagenlogische},
	text        ={aussagenlogische  Formel},
	plural      ={aussagenlogischen Formel},% Singular Dativ
	description ={
		Eine \Formel\ heißt \defFt{aussagenlogisch}, wenn sie ein Element von \FrmFor\ ist.
	}
}
\newcommand*    {\Formelmenge} [1][hyperTxt]{\glsIdx  [#1]{Formelmenge}}
\newcommand*    {\Formelmengen}[1][hyperTxt]{\glsIdxPl[#1]{Formelmenge}}
%ToDo prüfen
\newglossaryentry{Formelmenge}{
	name        ={Formelmenge},
	plural      ={Formelmengen},
	description ={
		Eine Menge von \Formeln, oft mit \glssymbol{MtsSprache} bezeichnet.
		Man nennt \glssymbol{MtsSprache} auch eine \Sprache\ und ihre Elemente \Woerter, insbesondere dann, wenn es eindeutige Regeln zur Konstruktion von \glssymbol{MtsSprache} gibt.
		Wir bevorzugen "`\Formel"' und "`\Formelmenge"'.
	}
}
\newcommand*    {\Funktion}  [1][hyperTxt]{\glsIdx  [#1]{Funktion}}
\newcommand*    {\Funktionen}[1][hyperTxt]{\glsIdxPl[#1]{Funktion}}
%ToDo prüfen
\newglossaryentry{Funktion}{
	name        ={Funktion},
	plural      ={Funktionen},
	description ={
		Eine \defFt{$n$-stellige Funktion} $f$ von einer Menge $A = A_1 \times \dots \times A_n$, dem \Definitionsbereich, in eine Menge $B$, den \Zielbereich, ist eine ($n$+1)-stellige \Relation\ $(G,A_1,\dots,A_n,B)$ derart, dass es für jedes $\vec{a} = (a_1,\dots,a_n)$ mit $a_i \in A_i$ genau ein $b \in B$ gibt mit $(a_1,\dots,a_n,b) \in f$.
		Dieses $b$ wird auch mit \seqqt{$f(a_1,\dots,a_n)$} , \seqqt{$f a_1 \dots a_n$} , \seqqt{$f(\vec{a})$} oder \seqqt{$f\vec{a}$} bezeichnet.
		\\Schreibweise: \seqqt{$f : A \rightarrow B$} \textbzw\ \seqqt{$f : A_1 \times \dots \times A_n \rightarrow B$}
	}
}
\newcommand*    {\Funktionswert} [1][hyperTxt]{\glsIdx  [#1]{Funktionswert}}
\newcommand*    {\Funktionswerte}[1][hyperTxt]{\glsIdxPl[#1]{Funktionswert}}
%ToDo prüfen
\newglossaryentry{Funktionswert}{
	name        ={Funktionswert},
	plural      ={Funktionswerte},
	description ={
		einer \Funktion.
	}
}

%G === G === G === G === G === G === G === G === G === G === G === G === G === G

\newcommand*    {\Gleichheit}[1][hyperTxt]{\glsIdx[#1]{Gleichheit}}
%ToDo prüfen
\newglossaryentry{Gleichheit}{
	name        ={Gleichheit},
	description ={
		Eine \Gleichheitsrelation:
		Zwei Objekte $A$ und $B$ sind \emph{gleich} (dasselbe; identisch), $A \MtsEq B$, wenn sie in den \interessierendenEigenschaften\ für \MtsEq\ übereinstimmen.
	}
}
\newcommand*    {\Gleichheitsrelation}  [1][hyperTxt]{\glsIdx  [#1]{Gleichheitsrelation}}
\newcommand*    {\Gleichheitsrelationen}[1][hyperTxt]{\glsIdxPl[#1]{Gleichheitsrelation}}
%ToDo prüfen
\newglossaryentry{Gleichheitsrelation}{
	name        ={Gleichheitsrelation},
	plural      ={Gleichheitsrelationen},
	description ={
		Eine mit \Gleichheit\ verwandte \Relation: \MtsEq, \MtsEqN, \MtsAequiv\ und \MtsAequivN.
	}
}
\newcommand*    {\Graph}  [1][hyperTxt]{\glsIdx  [#1]{Graph}}
\newcommand*    {\Graphen}[1][hyperTxt]{\glsIdxPl[#1]{Graph}}
%ToDo prüfen
\newglossaryentry{Graph}{
	name        ={Graph},
	plural      ={Graphen},
	symbol  ={\MtsGraph},
	see      ={MtsGraph},
	description ={
		einer \Funktion\ oder \Relation.
	}
}

%I === I === I === I === I === I === I === I === I === I === I === I === I === I

\newcommand*    {\Identitaetsregel} [1][hyperTxt]{\glsIdx  [#1]{Identitaetsregel}}
\newcommand*    {\Identitaetsregeln}[1][hyperTxt]{\glsIdxPl[#1]{Identitaetsregel}}
%ToDo prüfen
\newglossaryentry{Identitaetsregel}{
	name        ={Identitätsregel},
	plural      ={Identitätsregeln},
	description ={
		Eigentlich eine \Basisregel\ zur Identität.
		Da die \Identitaetsregeln\ nur zur Rechtfertigung der \Ersetzung\ verwendet werden, werden sie hier nicht zu den \Basisregeln\ gezählt.
	}
}

%J === J === J === J === J === J === J === J === J === J === J === J === J === J

\newcommand*    {\Junktor}  [1][hyperTxt]{\glsIdx  [#1]{Junktor}}
\newcommand*    {\Junktoren}[1][hyperTxt]{\glsIdxPl[#1]{Junktor}}
%ToDo prüfen
\newglossaryentry{Junktor}{
	name        ={Junktor},
	plural      ={Junktoren},
	description ={
		Eine aussagenlogische \Operation.
		Da die Werte einer aussagenlogischen \Operation\ \Wahrheitswerte\ sind, kann man einen \Junktor\ auch als \Relation\ verstehen.
	}
}
\newcommand*    {\Junktorsymbol} [1][hyperTxt]{\glsIdx  [#1]{Junktorsymbol}}
\newcommand*    {\Junktorsymbole}[1][hyperTxt]{\glsIdxPl[#1]{Junktorsymbol}}
%ToDo prüfen
\newglossaryentry{Junktorsymbol}{
	name        ={Junktorsymbol},
	plural      ={Junktorsymbole},
	description ={
		Ein \Symbol\ für einen \Junktor.%
		\footnote{%
			Entsprechend \emph{Funktionssymbol}, \emph{Operationssymbol}, \emph{Relationssymbol}, usw.
		}
	}
}

%K === K === K === K === K === K === K === K === K === K === K === K === K === K

\newcommand*    {\Komponente} [1][hyperTxt]{\glsIdx  [#1]{Komponente}}
\newcommand*    {\Komponenten}[1][hyperTxt]{\glsIdxPl[#1]{Komponente}}
%ToDo prüfen
\newglossaryentry{Komponente}{
	name        ={Komponente},
	plural      ={Komponenten},
	see         ={Folge,Tupel},
	description ={
		Die \Komponenten\ einer \Folge\ $\vec{a} = (a_1, a_2, \dots)$ sind die $a_i$.
		$a_i$ heißt die \defFt{$i$-te \Komponente} von $\vec{a}$.
	}
}
\newcommand*    {\Kontraposition}[1][hyperTxt]{\glsIdx[#1]{Kontraposition}}
%ToDo prüfen
\newglossaryentry{Kontraposition}{
	name        ={Kontraposition},
	description ={
		Die allgemeingültige \Aussage: $ (\alpha \FrmImp \beta) \FrmImp (\FrmNot\beta \FrmImp \FrmNot\alpha) $.
	}
}
\newcommand*    {\Kontravalenz}[1][hyperTxt]{\glsIdx[#1]{Kontravalenz}}
%ToDo prüfen
\newglossaryentry{Kontravalenz}{
	name        ={Kontravalenz},
	description ={
		Eine \Gleichheitsrelation:
		Zwei Objekte $A$ und $B$ sind \emph{nicht äquivalent} (nicht ähnlich), $A \MtsAequivN B$, wenn sie in mindestens einer \interessierendenEigenschaft\ für \MtsAequiv\ nicht übereinstimmen.
	}
}

%L === L === L === L === L === L === L === L === L === L === L === L === L === L

\newcommand*        {\Logik}[1][hyperTxt]{\glsIdx[#1]{Logik}}
\longnewglossaryentry{Logik}{
	name            ={Logik},
	see             ={Aussagenlogik,Praedikatenlogik}
}{
	\Wikipedia~\cite{bib:Logik} schreibt dazu (Zitat ohne altgriechischen Text, Fußnote und Verweise ins Internet):

	Mit \textbf{Logik} (von altgriechisch [\dots]‚denkende Kunst‘, ‚Vorgehensweise‘) oder auch \textbf{Folgerichtigkeit} wird im Allgemeinen das vernünftige Schlussfolgern und im Besonderen dessen Lehre – die \textbf{Schlussfolgerungslehre} oder auch \textbf{Denklehre} – bezeichnet. In der Logik wird die Struktur von Argumenten im Hinblick auf ihre Gültigkeit untersucht, unabhängig vom Inhalt der Aussagen. Bereits in diesem Sinne spricht man auch von „formaler“ Logik. Traditionell ist die Logik ein Teil der Philosophie. Ursprünglich hat sich die traditionelle Logik in Nachbarschaft zur Rhetorik entwickelt. Seit dem 20. Jahrhundert versteht man unter Logik überwiegend symbolische Logik, die auch als grundlegende Strukturwissenschaft, z. B. innerhalb der Mathematik und der theoretischen Informatik, behandelt wird.\par
}

%M === M === M === M === M === M === M === M === M === M === M === M === M === M

\newcommand*        {\Mengenlehre}[1][hyperTxt]{\glsIdx[#1]{Mengenlehre}}
%ToDo prüfen
\longnewglossaryentry{Mengenlehre}{
	name            ={Mengenlehre}
}{
	\Wikipedia~\cite{bib:Mengenlehre} schreibt dazu (Zitat ohne Verweise ins Internet):

	Die \textbf{Mengenlehre} ist ein grundlegendes Teilgebiet der Mathematik, das sich mit der Untersuchung von Mengen, also von Zusammenfassungen von Objekten, beschäftigt. Die gesamte Mathematik, wie sie heute üblicherweise gelehrt wird, ist in der Sprache der Mengenlehre formuliert und baut auf den Axiomen der Mengenlehre auf. Die meisten mathematischen Objekte, die in Teilbereichen wie Algebra, Analysis, Geometrie, Stochastik oder Topologie behandelt werden, um nur einige wenige zu nennen, lassen sich als Mengen definieren. Gemessen daran ist die Mengenlehre eine recht junge Wissenschaft; erst nach der Überwindung der Grundlagenkrise der Mathematik im frühen 20. Jahrhundert konnte die Mengenlehre ihren heutigen, zentralen und grundlegenden Platz in der Mathematik einnehmen.\par
}
\newcommand*    {\Metadefinition}  [1][hyperTxt]{\glsIdx  [#1]{Metadefinition}}
\newcommand*    {\Metadefinitionen}[1][hyperTxt]{\glsIdxPl[#1]{Metadefinition}}
%ToDo prüfen
\newglossaryentry{Metadefinition}{
	name        ={Metadefinition},
	plural      ={Metadefinitionen},
	see         ={Definition},
	description ={
		Eine \Definition\ in \Metasprache\ mit Hilfe des \emph{Metadefinitionssymbols} \chrqt{\MtsDefEquiv}.
		\seqqt{$A \MtsDefEquiv B$} steht für \standsfor{$A$ \emph{ist definitionsgemäß äquivalent zu} $B$} für \Aussagen\ $A$ und $B$.
		Gewissermaßen ist $A$ nur eine andere Schreibweise für $B$.
	}
}
\newcommand*    {\Metaformel} [1][hyperTxt]{\glsIdx  [#1]{Metaformel}}
\newcommand*    {\Metaformeln}[1][hyperTxt]{\glsIdxPl[#1]{Metaformel}}
\newglossaryentry{Metaformel}{
	name        ={Metaformel},
	plural      ={Metaformeln},
	description ={
		Eine \Formel\ der \formalisiertenMetasprache.
	}
}
\newcommand*    {\Metaoperation}  [1][hyperTxt]{\glsIdx  [#1]{Metaoperation}}
\newcommand*    {\Metaoperationen}[1][hyperTxt]{\glsIdxPl[#1]{Metaoperation}}
%ToDo prüfen
\newglossaryentry{Metaoperation}{
	name        ={Metaoperation},
	plural      ={Metaoperationen},
	description ={
		Eine \Operation\ der \Metasprache: \MtsAnd, \MtsOr\ oder \MtsUnd.
	}
}
\newcommand*    {\Metarelation}  [1][hyperTxt]{\glsIdx  [#1]{Metarelation}}
\newcommand*       {\Mrelationen}[1][hyperTxt]{\glsIdxI [#1]{Metarelation}}
\newcommand*    {\Metarelationen}[1][hyperTxt]{\glsIdxPl[#1]{Metarelation}}
%ToDo prüfen
\newglossaryentry{Metarelation}{
	name        ={Metarelation},
	user1       =   {-relationen},
	plural      ={Metarelationen},
	description ={
		Eine \Relation\ der \Metasprache: \MtsImp, \MtsRep\ oder \MtsEquiv.
	}
}
\newcommand*    {\Metasprache} [1][hyperTxt]{\glsIdx  [#1]{Metasprache}}
\newcommand*    {\Metasprachen}[1][hyperTxt]{\glsIdxPl[#1]{Metasprache}}
%ToDo prüfen
\newglossaryentry{Metasprache}{
	name        ={Metasprache},
	plural      ={Metasprachen},
	see         ={Objektsprache},
	description ={
		Eine \Sprache, in der \Aussagen\ über Elemente einer anderen \Sprache\ getroffen werden können.
		In diesem Dokument ist dies immer die normale Umgangssprache.
	}
}
\newcommand*    {\formalisierteMetasprache} [1][hyperTxt]{\glsIdx  [#1]{formalisierteMetasprache}}
\newcommand*    {\formalisiertenMetasprache}[1][hyperTxt]{\glsIdx  [#1]{formalisierteMetasprache}}
%ToDo prüfen
\newglossaryentry{formalisierteMetasprache}{
	name        =                         {-, formalisierte},
	sort        =               {Metasprache, formalisierte},
	text        ={formalisierte  Metasprache},
	plural      ={formalisierten Metasprache},
	description ={
		Eine \Metasprache, deren Ausdrucksmittel \Formeln\ sind.
		In diesem Dokument gehören die meisten \Formeln\ dazu und werden daher als \Metaformeln\ bezeichnet.
		Die Definition der Bedeutung der \Metaformeln\ ist mehr beschreibend und nicht so exakt wie bei den \Formeln\ der Mathematik, den hier sogenannten \Objektformeln.
	}
}
\newcommand*    {\Metavariable} [1][hyperTxt]{\glsIdx [#1]{Metavariable}}
\newcommand*       {\Mvariablen}[1][hyperTxt]{\glsIdxI[#1]{Metavariable}}
%ToDo prüfen
\newglossaryentry{Metavariable}{
	name        ={Metavariable},
	user1       =   {-variablen},
	description ={
		Eine \Variable\ der \formalisiertenMetasprache.
	}
}
\newcommand*    {\Monotonieregel}[1][hyperTxt]{\glsIdx  [#1]{Monotonieregel}}
%ToDo prüfen
\newglossaryentry{Monotonieregel}{
	name        ={Monotonieregel},
	see         ={MR},
	description ={
		Eine \Schlussregel.
	}
}

%N === N === N === N === N === N === N === N === N === N === N === N === N === N

\newcommand*    {\Negation}  [1][hyperTxt]{\glsIdx  [#1]{Negation}}
\newcommand*    {\Negationen}[1][hyperTxt]{\glsIdxPl[#1]{Negation}}
%ToDo prüfen
\newglossaryentry{Negation}{
	name        ={Negation},
	plural      ={Negationen},
	description ={
		Die \Negation\ (zu) einer binären \Relation\ $(G,A,B)$ ist die \Relation\ $(H,A,B)$ mit $H = (A \times B) \setminus G\}$.
		Üblicherweise wird das zugehörige Relationssymbol mit einem schrägen oder vertikalen Strich durchgestrichen.
	}
}

%O === O === O === O === O === O === O === O === O === O === O === O === O === O

\newcommand*    {\Objekt}  [1][hyperTxt]{\glsIdx  [#1]{Objekt}}
\newcommand*    {\Objekts} [1][hyperTxt]{\glsIdx  [#1]{Objekt}[s]}
\newcommand*    {\Objekte} [1][hyperTxt]{\glsIdxPl[#1]{Objekt}}
\newcommand*    {\Objekten}[1][hyperTxt]{\glsIdxPl[#1]{Objekt}[n]}
%ToDo prüfen
\newglossaryentry{Objekt}{
	name        ={Objekt},
	plural      ={Objekte},
	description ={
		\Symbole, \Formeln\ und \Aussagen\ sowie Mengen, \Zeichenfolgen, Zahlen; ganz allgemein reale oder gedachte Dinge an sich.
	}
}
\newcommand*    {\Objektformel} [1][hyperTxt]{\glsIdx  [#1]{Objektformel}}
\newcommand*    {\Objektformeln}[1][hyperTxt]{\glsIdxPl[#1]{Objektformel}}
%ToDo prüfen
\newglossaryentry{Objektformel}{
	name        ={Objektformel},
	plural      ={Objektformeln},
	description ={
		Eine \Formel\ der \Objektsprache.
	}
}
\newcommand*    {\Objektsprache} [1][hyperTxt]{\glsIdx[#1]{Objektsprache}}
%ToDo prüfen
\newglossaryentry{Objektsprache}{
	name        ={Objektsprache},
	description ={
		Je nach der aktuellen (mathematischen) Umgebung die \Formeln\ der \Aussagenlogik, der \Praedikatenlogik\ oder der \Mengenlehre.
	}
}
\newcommand*    {\Operation}  [1][hyperTxt]{\glsIdx  [#1]{Operation}}
\newcommand*    {\Operationen}[1][hyperTxt]{\glsIdxPl[#1]{Operation}}
%ToDo prüfen
\newglossaryentry{Operation}{
	name        ={Operation},
	plural      ={Operationen},
	description ={
		Eine --- meistens binäre, \textdh\ zweiwertige --- Funktion $M^n \rightarrow M$.
		Für eine binäre \Operation\ $\BspOpB : M \times M \rightarrow M$ schreibt man meistens $x \BspOpB y$ statt $\BspOpB(x,y)$.
	}
}
\newcommand*    {\Operationssymbol} [1][hyperTxt]{\glsIdx  [#1]{Operationssymbol}}
\newcommand*    {\Operationssymbole}[1][hyperTxt]{\glsIdxPl[#1]{Operationssymbol}}
%ToDo prüfen
\newglossaryentry{Operationssymbol}{
	name        ={Operationssymbol},
	plural      ={Operationssymbole},
	description ={
		Ein \Symbol\ für eine \Operation.
	}
}

%P === P === P === P === P === P === P === P === P === P === P === P === P === P

\newcommand*    {\PolnischeNotation}  [1][hyperTxt]{\glsIdx  [#1]{PolnischeNotation}}
\newcommand*    {\PolnischeNotationen}[1][hyperTxt]{\glsIdx  [#1]{PolnischeNotation}[en]}
\newcommand*    {\PolnischerNotation} [1][hyperTxt]{\glsIdxI [#1]{PolnischeNotation}}
\newcommand*    {\PolnischenNotation} [1][hyperTxt]{\glsIdxPl[#1]{PolnischeNotation}[en]}
%ToDo prüfen
\newglossaryentry{PolnischeNotation}{
	name        =           {Notation, Polnische},
	text        ={Polnische  Notation},
	user1       ={Polnischer Notation},% Singular Genitiv
	plural      ={Polnischen Notation},% Singular Dativ
	description ={
		Bei der \PolnischenNotation\ stehen die Operanden \textbzw\ Argumente von \Relationen\ und \Funktionen\ stets rechts von den Relations- und Funktionssymbolen.
		Dadurch kann auf Gliederungszeichen wie Klammern und Kommata verzichtet werden.
		Noch einfacher für Computer ist die \defFt{umgekehrte} \PolnischeNotation, bei der die Operanden und Argumente links von den Symbolen stehen.
	}
}
\newcommand*    {\Potenzmenge} [1][hyperTxt]{\glsIdx  [#1]{Potenzmenge}}
\newcommand*    {\Potenzmengen}[1][hyperTxt]{\glsIdxPl[#1]{Potenzmenge}}
%ToDo prüfen
\newglossaryentry{Potenzmenge}{
	name        ={Potenzmenge},
	plural      ={Potenzmengen},
	description ={
		Die \Potenzmenge\ $\MtsPot(M)$ einer Menge $M$ ist die Menge ihrer Teilmengen.
	}
}
\newcommand*    {\Praedikat} [1][hyperTxt]{\glsIdx  [#1]{Praedikat}}
\newcommand*    {\Praedikate}[1][hyperTxt]{\glsIdxPl[#1]{Praedikat}}
%ToDo prüfen
\newglossaryentry{Praedikat}{
	name        ={Prädikat},
	plural      ={Prädikate},
	description ={
		Ein Element der \Praedikatenlogik. ---
		\textZB\ kann man eine Gruppe als ein zweistelliges \Praedikat\ $\mathrm{Gruppe}(G,+)$ definieren, in dem $G$ eine Menge und $+$ eine \Operation, \textdh\ eine binäre (zweistellige) Funktion $ +: G \times G \rightarrow G $ ist, so dass die Gruppenaxiome erfüllt sind.
	}
}
\newcommand*        {\Praedikatenlogik}[1][hyperTxt]{\glsIdx  [#1]{Praedikatenlogik}}
\longnewglossaryentry{Praedikatenlogik}{
	name            ={Prädikatenlogik},
	see             ={Aussagenlogik,Logik}
}{
	\Wikipedia~\cite{bib:Praedikatenlogik} schreibt dazu (Zitat ohne Verweise ins Internet):

	Die \textbf{Prädikatenlogiken} (auch \textbf{Quantorenlogiken}) bilden eine Familie logischer Systeme, die es erlauben, einen weiten und in der Praxis vieler Wissenschaften und deren Anwendungen wichtigen Bereich von Argumenten zu formalisieren und auf ihre Gültigkeit zu überprüfen. Auf Grund dieser Eigenschaft spielt die Prädikatenlogik eine große Rolle in der Logik sowie in Mathematik, Informatik, Linguistik und Philosophie.\par
}

%Q === Q === Q === Q === Q === Q === Q === Q === Q === Q === Q === Q === Q === Q

\newcommand*    {\Quellbereich} [1][hyperTxt]{\glsIdx  [#1]{Quellbereich}}
\newcommand*    {\Quellbereiche}[1][hyperTxt]{\glsIdxPl[#1]{Quellbereich}}
%ToDo prüfen
\newglossaryentry{Quellbereich}{
	name        ={Quellbereich},
	plural      ={Quellbereiche},
	symbol      ={\MtsQb},
	see         = {MtsQb,Definitionsbereich,Funktion},
	description ={
		einer partiellen \Funktion.
	}
}

%R === R === R === R === R === R === R === R === R === R === R === R === R === R

\newcommand*    {\Relation}  [1][hyperTxt]{\glsIdx  [#1]{Relation}}
\newcommand*    {\Relationen}[1][hyperTxt]{\glsIdxPl[#1]{Relation}}
%ToDo prüfen
\newglossaryentry{Relation}{
	name        ={Relation},
	plural      ={Relationen},
	description ={
		Eine \defFt{$n$-stellige} \Relation\ $R$ ist ein (1+$n$)-\Tupel\ $(G,A_1,\dots,A_n$) mit $G \MtsSubsetEq A_1 \times \dots \times A_n)$.
	}
}

%S === S === S === S === S === S === S === S === S === S === S === S === S === S

\newcommand*    {\Satz}   [1][hyperTxt]{\glsIdx  [#1]{Satz}}
\newcommand*    {\Satzes} [1][hyperTxt]{\glsIdx  [#1]{Satz}[s]}
\newcommand*    {\Saetze} [1][hyperTxt]{\glsIdxPl[#1]{Satz}}
\newcommand*    {\Saetzen}[1][hyperTxt]{\glsIdxPl[#1]{Satz}[n]}
%ToDo prüfen
\newglossaryentry{Satz}{
	name        ={Satz},
	plural      ={Sätze},
	description ={
		Eine mathematische \Aussage, dass bestimmte \Folgerungen\ aus gegebenen \Voraussetzungen\ abgeleitet werden können.
	}
}
\newcommand*    {\formalerSatz} [1][hyperTxt]{\glsIdx  [#1]{formalerSatz}}
\newcommand*    {\formalenSatz} [1][hyperTxt]{\glsIdxPl[#1]{formalerSatz}}
%ToDo prüfen
\newglossaryentry{formalerSatz}{
	name        =            {-, formal},
	sort        =         {Satz, formal},
	text        ={formaler Satz},
	plural      ={formalen Satz},% Singular Dativ
	see         ={FS},
	description ={
		Formale Darstellung eines mathematischen \Satzes.
	}
}
\newcommand*    {\Schlussregel} [1][hyperTxt]{\glsIdx  [#1]{Schlussregel}}
\newcommand*    {\Schlussregeln}[1][hyperTxt]{\glsIdxPl[#1]{Schlussregel}}
%ToDo prüfen
\newglossaryentry{Schlussregel}{
	name        ={Schlussregel},
	plural      ={Schlussregeln},
	see         ={MtsSchlussregel,MtsSchlussregelSet,allgemeingueltig},
	description ={
		Eine \Schlussregel\ $\frac{\MtsVoraussetzungSet}{\MtsFolgerungSet}$ entspricht der \Aussage:
		\begin{quote}
			\enquote{Wenn alle \Voraussetzungen\ \MtsVoraussetzung\ aus \MtsVoraussetzungSet\ zutreffen, dann auch alle \Folgerungen\ \MtsFolgerung\ aus \MtsFolgerungSet.}
		\end{quote}
		Wenn diese \Aussage\ zutrifft, kann die Schlussregel zur \zulaessigen\ \Umwandlung\ von \Formel\ dienen.
	}
}
\newcommand*    {\allgemeingueltigeSchlussregel}  [1][hyperTxt]{\glsIdx  [#1]{allgemeingueltigeSchlussregel}}
\newcommand*    {\allgemeingueltigeSchlussregeln} [1][hyperTxt]{\glsIdx  [#1]{allgemeingueltigeSchlussregel}[n]}
\newcommand*    {\allgemeingueltigenSchlussregel} [1][hyperTxt]{\glsIdxPl[#1]{allgemeingueltigeSchlussregel}}
\newcommand*    {\allgemeingueltigenSchlussregeln}[1][hyperTxt]{\glsIdxPl[#1]{allgemeingueltigeSchlussregel}[n]}
%ToDo prüfen
\newglossaryentry{allgemeingueltigeSchlussregel}{
	name        =                             {-, allgemeingültig},
	sort        =                  {Schlussregel, allgemeingültig},
	text        ={allgemeingültige  Schlussregel},
	plural      ={allgemeingültigen Schlussregel},% Singular Dativ
	description ={
		Eine \Schlussregel\ heißt \defFt{allgemeingültig}, wenn sie aus den \Basisregeln\ und schon bekannten \allgemeingueltigenSchlussregeln\ abgeleitet werden kann.
	}
}
\newcommand*    {\Schlussregelmenge} [1][hyperTxt]{\glsIdx  [#1]{Schlussregelmenge}}
\newcommand*    {\Schlussregelmengen}[1][hyperTxt]{\glsIdxPl[#1]{Schlussregelmenge}}
%ToDo prüfen
\newglossaryentry{Schlussregelmenge}{
	name        ={Schlussregelmenge},
	plural      ={Schlussregelmengen},
	symbol      ={\ensuremath{\RawMtsSchlussregelSet}},
	see         ={MtsSchlussregelSet},
	description ={
		Eine Menge von \Schlussregeln, meistens mit \MtsSchlussregelSet\ bezeichnet.
	}
}
\newcommand*    {\Schnittregel}[1][hyperTxt]{\glsIdx[#1]{Schnittregel}}
%ToDo prüfen
\newglossaryentry{Schnittregel}{
	name        ={Schnittregel},
	plural      ={Schnittregeln},
	see         ={SR},
	description ={
		Eine \allgemeingueltigeSchlussregel.
	}
}
%ToDo prüfen
\newcommand*        {\Signatur}[1][hyperTxt]{\glsIdx[#1]{Signatur}}
\longnewglossaryentry{Signatur}{
	name            ={Signatur},
	plural          ={Signaturen}
}{
	\Wikipedia~\cite{bib:Signatur} schreibt (Zitat ohne Verweise ins Internet):

	In der mathematischen Logik besteht eine \textbf{Signatur} aus der Menge der Symbole, die in der betrachteten Sprache zu den üblichen, rein logischen Symbolen hinzukommt, und einer Abbildung, die jedem Symbol der Signatur eine Stelligkeit eindeutig zuordnet. Während die logischen Symbole wie  $ \forall ,\exists ,\land ,\lor ,\rightarrow ,\leftrightarrow ,\neg $ stets als „für alle“, „es gibt ein“, „und“, „oder“, „folgt“, „äquivalent zu“ bzw. „nicht“ interpretiert werden, können durch die semantische Interpretation der Symbole der Signatur verschiedene Strukturen (insbesondere Modelle von Aussagen der Logik) unterschieden werden. Die Signatur ist der spezifische Teil einer elementaren Sprache.\par
}
\newcommand*    {\BoolescheSignatur} [1][hyperTxt]{\glsIdx  [#1]{BoolescheSignatur}}
\newcommand*    {\BooleschenSignatur}[1][hyperTxt]{\glsIdxPl[#1]{BoolescheSignatur}}
%ToDo prüfen
\newglossaryentry{BoolescheSignatur}{
	name        =                  {-, Boolesche},
	sort        =           {Signatur, Boolesche},
	text        ={Boolesche  Signatur},
	plural      ={Booleschen Signatur},% Singular Dativ
	description ={
		Die \logischeSignatur\ $\{\FrmNot, \FrmAnd, \FrmOr\}$.
	}
}
\newcommand*    {\logischeSignatur}  [1][hyperTxt]{\glsIdx  [#1]{logischeSignatur}}
\newcommand*    {\logischeSignaturen}[1][hyperTxt]{\glsIdx  [#1]{logischeSignatur}[en]}
\newcommand*    {\logischenSignatur} [1][hyperTxt]{\glsIdxPl[#1]{logischeSignatur}}
%ToDo prüfen
\newglossaryentry{logischeSignatur}{
	name        =                {-, logische},
	sort        =         {Signatur, logische},
	text        ={logische Signatur},
	plural      ={logischen Signatur},% Dativ
	description ={
		Abweichend von der Definition von \Signatur\ in \Wikipedia\ ist eine \defFt{logische Signatur} eine Teilmenge von \FrmJun, ausreichend um damit und mit \FrmVar\ und Klammerung alle anderen Elemente aus \FrmJun\ zu definieren.
	}
}
\newcommand*    {\Sprache} [1][hyperTxt]{\glsIdx  [#1]{Sprache}}
\newcommand*    {\Sprachen}[1][hyperTxt]{\glsIdxPl[#1]{Sprache}}
%ToDo prüfen
\newglossaryentry{Sprache}{
	name        ={Sprache},
	plural      ={Sprachen},
	description ={
		--- Siehe \Formelmenge.
	}
}
\newcommand*    {\Stelligkeit}  [1][hyperTxt]{\glsIdx  [#1]{Stelligkeit}}
\newcommand*    {\Stelligkeiten}[1][hyperTxt]{\glsIdxPl[#1]{Stelligkeit}}
%ToDo prüfen
\newglossaryentry{Stelligkeit}{
	name        ={Stelligkeit},
	plural      ={Stelligkeiten},
	see         ={MtsStelF,MtsStelR},
	description ={
		einer \Funktion\ oder \Relation.
	}
}
\newcommand*    {\Symbol}  [1][hyperTxt]{\glsIdx  [#1]{Symbol}}
\newcommand*    {\Symbols} [1][hyperTxt]{\glsIdx  [#1]{Symbol}[s]}
\newcommand*    {\Symbole} [1][hyperTxt]{\glsIdxPl[#1]{Symbol}}
\newcommand*    {\Symbolen}[1][hyperTxt]{\glsIdxPl[#1]{Symbol}[n]}
%ToDo prüfen
\newglossaryentry{Symbol}{
	name        ={Symbol},
	plural      ={Symbole},
	description ={
		Ein \defFt{einfaches} \Symbol\ ist ein druckbares typographisches Zeichen.
		Ein \defFt{zusammengesetztes} \Symbol\ besteht aus mehreren einfachen \Symbolen.
		In beiden Fällen wird ein \Symbol\ als \unzerlegbar\ angesehen.
	}
}

%T === T === T === T === T === T === T === T === T === T === T === T === T === T

\newcommand*    {\Traegermenge} [1][hyperTxt]{\glsIdx  [#1]{Traegermenge}}
\newcommand*    {\Traegermengen}[1][hyperTxt]{\glsIdxPl[#1]{Traegermenge}}
%ToDo prüfen
\newglossaryentry{Traegermenge}{
	name        ={Trägermenge},
	plural      ={Trägermengen},
	symbol  ={\MtsTraeger},
	see         ={MtsTraeger},
	description ={
		einer \Relation.
	}
}
\newcommand*    {\Tupel} [1][hyperTxt]{\glsIdx  [#1]{Tupel}}
\newcommand*    {\Tupels}[1][hyperTxt]{\glsIdx  [#1]{Tupel}[s]}
%ToDo prüfen
\newglossaryentry{Tupel}{
	name        ={Tupel},
	description ={
		Ein $n$-\Tupel\alternativ{Vektor} $\vec{a}$ ist eine endliche Folge\alternativ{Sequenz} $(a_1, \dots, a_n)$ \defFt{von} seinen \defFt{Komponenten} $a_i$.
		Sind alle Komponenten Elemente einer Menge $M$, so heißt $\vec{a}$ ein $n$-\Tupel\ \defFt{auf} $M$.
	}
}
\newcommand*    {\Tupelmenge} [1][hyperTxt]{\glsIdx  [#1]{Tupelmenge}}
\newcommand*    {\Tupelmengen}[1][hyperTxt]{\glsIdxPl[#1]{Tupelmenge}}
%ToDo prüfen
\newglossaryentry{Tupelmenge}{
	name        ={Tupelmenge},
	plural      ={Tupelmengen},
	description ={
		Die \Tupelmenge\ $\MtsTup(M)$ einer Menge $M$ ist die Menge aller $n$-Tupel aus $M^n$ für alle $n \in \MtsINo$.
	}
}

%U === U === U === U === U === U === U === U === U === U === U === U === U === U

\newcommand*    {\Umkehrrelation}  [1][hyperTxt]{\glsIdx  [#1]{Umkehrrelation}}
\newcommand*    {\Umkehrrelationen}[1][hyperTxt]{\glsIdxPl[#1]{Umkehrrelation}}
%ToDo prüfen
\newglossaryentry{Umkehrrelation}{
	name        ={Umkehrrelation},
	plural      ={Umkehrrelationen},
	description ={
		Die \Umkehrrelation\ zu einer binären \Relation\ $(G,A,B)$ ist die \Relation\ $(H,B,A)$ mit $H = \{(b,a)|(a,b) \in G\}$.
		Üblicherweise wird das zugehörige Relationssymbol gespiegelt.
	}
}
\newcommand*    {\Umwandlung}  [1][hyperTxt]{\glsIdx  [#1]{Umwandlung}}
\newcommand*    {\Umwandlungen}[1][hyperTxt]{\glsIdxPl[#1]{Umwandlung}}
%ToDo prüfen
\newglossaryentry{Umwandlung}{
	name        ={Umwandlung},
	plural      ={Umwandlungen},
	see         ={MtsUmwandlung,MtsUmwandlungTup,zulaessigeUmwandlung},
	description ={
		Eine Umformung oder Erzeugung einer \Formel\ aus einer vorgegebenen Menge von \Formeln, \textdh\ die Anwendung einer \Schlussregel.
	}
}
\newcommand*    {\zulaessigeUmwandlung}   [1][hyperTxt]{\glsIdx  [#1]{zulaessigeUmwandlung}}
\newcommand*    {\zulaessigeUmwandlungen} [1][hyperTxt]{\glsIdx  [#1]{zulaessigeUmwandlung}[en]}
\newcommand*    {\zulaessigerUmwandlungen}[1][hyperTxt]{\glsIdxI [#1]{zulaessigeUmwandlung}[en]}
\newcommand*    {\zulaessigenUmwandlung}  [1][hyperTxt]{\glsIdxPl[#1]{zulaessigeUmwandlung}}
\newcommand*    {\zulaessigenUmwandlungen}[1][hyperTxt]{\glsIdxPl[#1]{zulaessigeUmwandlung}[en]}
%ToDo prüfen
\newglossaryentry{zulaessigeUmwandlung}{
	name        =                    {-, zulässig},
	sort        =           {Umwandlung, zulässig},
	text        ={zulässige  Umwandlung},
	user1       ={zulässiger Umwandlung},% Singular Genitiv
	plural      ={zulässigen Umwandlung},% Singular Dativ
	see         ={Umwandlung},
	description ={
		Eine \Umwandlung\ heißt \defFt{zulässig}, wenn sie Element einer vorgegebenen Menge von \Umwandlungen\ oder eine daraus zulässigerweise abgeleitete \Umwandlung\ ist.
	}
}
\newcommand*    {\Umwandlungsfolge} [1][hyperTxt]{\glsIdx  [#1]{Umwandlungsfolge}}
\newcommand*    {\Umwandlungsfolgen}[1][hyperTxt]{\glsIdxPl[#1]{Umwandlungsfolge}}
%ToDo prüfen
\newglossaryentry{Umwandlungsfolge}{
	name        ={Umwandlungsfolge},
	plural      ={Umwandlungsfolgen},
	see         ={MtsUmwandlung,MtsUmwandlungTup,Umwandlung},
	description ={
		Eine Folge von \Umwandlungen.
	}
}
\newcommand*    {\Ungleichheit}[1][hyperTxt]{\glsIdx[#1]{Ungleichheit}}
%ToDo prüfen
\newglossaryentry{Ungleichheit}{
	name        ={Ungleichheit},
	description ={
		Eine \Gleichheitsrelation:
		Zwei Objekte $A$ und $B$ sind \emph{nicht gleich} (nicht dasselbe; nicht identisch), $A \MtsEqN B$, wenn sie in mindestens einer \interessierendenEigenschaft\ für \MtsEq\ nicht übereinstimmen.
	}
}
\newcommand*    {\unaer} [1][hyperTxt]{\glsIdx  [#1]{unaer}}
\newcommand*    {\unaere}[1][hyperTxt]{\glsIdxPl[#1]{unaer}}
%ToDo prüfen
\newglossaryentry{unaer}{
	name        ={unär},
	plural      ={unäre},
	see         ={binaer},
	description ={
		Eine \Operation\, \Funktion\ oder \Relation\ heißt \defFt{unär}, wenn ihre \Stelligkeit\ gleich 1 ist.
	}
}
\newcommand*    {\unzerlegbar} [1][hyperTxt]{\glsIdx  [#1]{unzerlegbar}}
\newcommand*    {\unzerlegbare}[1][hyperTxt]{\glsIdxPl[#1]{unzerlegbar}}
%ToDo prüfen
\newglossaryentry{unzerlegbar}{
	name        ={unzerlegbar},
	plural      ={unzerlegbare},
	see         ={zerlegbar},
	description ={
		Synonym: \atomar\ ---
		Eine \Aussage, die keine \Metaoperation, \textbzw\ eine \Formel, die keine \Operation\ und keine \Relation\ enthält, heißt \defFt{unzerlegbar}.
	}
}

%V === V === V === V === V === V === V === V === V === V === V === V === V === V

\newcommand*        {\Variable} [1][hyperTxt]{\glsIdx  [#1]{Variable}}
\newcommand*        {\Variablen}[1][hyperTxt]{\glsIdxPl[#1]{Variable}}
\longnewglossaryentry{Variable}{
	name            ={Variable},
	plural          ={Variablen}
}{
	\Wikipedia~\cite{bib:Variable} schreibt dazu (Zitat ohne Fußnoten und Verweise ins Internet):

	Eine \textbf{Variable} ist ein Name für eine Leerstelle in einem logischen oder mathematischen Ausdruck.[1] Der Begriff leitet sich vom lateinischen Adjektiv \textit{variabilis} (veränderlich) ab. Gleichwertig werden auch die Begriffe \textit{Platzhalter} oder \textit{Veränderliche} benutzt. Als „Variable“ dienten früher Wörter oder Symbole, heute verwendet man zur mathematischen Notation in der Regel Buchstaben als Zeichen. Wird anstelle der Variablen ein konkretes Objekt eingesetzt, so ist darauf zu achten, dass überall dort, wo die Variable auftritt, auch dasselbe Objekt benutzt wird.\par
}
\newcommand*    {\vergleichbar} [1][hyperTxt]{\glsIdx  [#1]{vergleichbar}}
\newcommand*    {\Vergleichbar} [1][hyperTxt]{\GlsIdx  [#1]{vergleichbar}}
\newcommand*    {\vergleichbare}[1][hyperTxt]{\glsIdxPl[#1]{vergleichbar}}
%ToDo prüfen
\newglossaryentry{vergleichbar}{
	name        ={vergleichbar},
	plural      ={vergleichbare},
	description ={
		Zwei \Objekte\ $A$ und $B$ sind \vergleichbar, wenn beide von derselben Art sind, \textdh\ wenn beide \textzB\ jeweils Mengen, \Zeichenfolgen, Zahlen, \textusw\ sind.
		Dabei muss bei \Formeln\ zwischen der \Formel\ an sich und ihrem \emph{Wert} oder \emph{Ergebnis} unterschieden werden.
	}
}
\newcommand*    {\Vertauschung}  [1][hyperTxt]{\glsIdx  [#1]{Vertauschung}}
\newcommand*    {\Vertauschungen}[1][hyperTxt]{\glsIdxPl[#1]{Vertauschung}}
%ToDo prüfen
\newglossaryentry{Vertauschung}{
	name        ={Vertauschung},
	plural      ={Vertauschungen},
	description ={
		Die \emph{Vertauschung} von zwei unabhängigen Teil-\Formeln\ ($\alpha$ und $\beta$) in einer anderen \Formel\ ($\gamma$)
		\\--- Formal: $\gamma(\alpha \MtsSwap \beta)$.
		Die \emph{Vertauschung} ist eine spezielle Form der \Ersetzung.
	}
}
\newcommand*    {\Voraussetzung}  [1][hyperTxt]{\glsIdx  [#1]{Voraussetzung}}
\newcommand*    {\Voraussetzungen}[1][hyperTxt]{\glsIdxPl[#1]{Voraussetzung}}
%ToDo prüfen
\newglossaryentry{Voraussetzung}{
	name        ={Voraussetzung},
	plural      ={Voraussetzungen},
	see         ={Schlussregel},
	description ={
		Eine \Ableitung:
		Die \Voraussetzungen\ einer \Schlussregel\ $\frac{\MtsVoraussetzungSet}{\MtsFolgerungSet}$ \textbzw\ $\frac{\MtsVoraussetzungSet}{\MtsFolgerungSet}$ sind die Elemente aus \MtsVoraussetzungSet\ \textbzw\ \MtsVoraussetzungRel.
		Die \Voraussetzungen\ werden normalerweise mit $\MtsVoraussetzung_i$ bezeichnet.
	}
}
\newcommand*    {\Voraussetzungsmenge} [1][hyperTxt]{\glsIdx  [#1]{Voraussetzungsmenge}}
\newcommand*    {\Voraussetzungsmengen}[1][hyperTxt]{\glsIdxPl[#1]{Voraussetzungsmenge}}
%ToDo prüfen
\newglossaryentry{Voraussetzungsmenge}{
	name        ={Voraussetzungsmenge},
	plural      ={Voraussetzungsmengen},
	description ={
		Eine \Ableitungsmenge:
		Die Menge \MtsVoraussetzungSet\ der \Voraussetzungen\ einer \Schlussregel\ \textbzw\ eines \Beweises.
	}
}

%W === W === W === W === W === W === W === W === W === W === W === W === W === W

\newcommand*    {\Wahrheitswert}  [1][hyperTxt]{\glsIdx  [#1]{Wahrheitswert}}
\newcommand*    {\Wahrheitswerte} [1][hyperTxt]{\glsIdxPl[#1]{Wahrheitswert}}
\newcommand*    {\Wahrheitswerten}[1][hyperTxt]{\glsIdxPl[#1]{Wahrheitswert}n}
%ToDo prüfen
\newglossaryentry{Wahrheitswert}{
	name        ={Wahrheitswert},
	plural      ={Wahrheitswerte},
	description ={
		Die Werte \chrqt{\FrmTrue} und \chrqt{\FrmFalse}, oft auch mit \chrqt{\TxtTrue} und \chrqt{\TxtFalse}, \chrqt{$\RawMtsTrue$} und \chrqt{$\RawMtsFalse$} oder einfach \chrqt{$1$} und \chrqt{$0$} bezeichnet.
	}
}
\newcommand*    {\Wertebereich} [1][hyperTxt]{\glsIdx  [#1]{Wertebereich}}
\newcommand*    {\Wertebereiche}[1][hyperTxt]{\glsIdxPl[#1]{Wertebereich}}
%ToDo prüfen
\newglossaryentry{Wertebereich}{
	name        ={Wertebereich},
	plural      ={Wertebereiche},
	symbol      ={\MtsWb},
	see         = {MtsWb,Zielbereich,Funktion},
	description ={
		einer \Funktion.
	}
}
\newcommand*    {\Wikipedia}[1][hyperTxt]{\glsIdx[#1]{Wikipedia}}
\newglossaryentry{Wikipedia}{
	name        ={Wikipedia},
	description ={
		\Wikipedia~\cite{bib:Wikipedia} schreibt dazu (Zitat): "`Wikipedia ist ein Projekt zum Aufbau einer [Internet-\nobreak]Enzyklopädie aus freien Inhalten."'
	}
}
\newcommand*    {\Wort}   [1][hyperTxt]{\glsIdx  [#1]{Wort}}
\newcommand*    {\Worte}  [1][hyperTxt]{\glsIdxPl[#1]{Wort}}
\newcommand*    {\Woerter}[1][hyperTxt]{\glsIdxPl[#1]{Wort}}
%ToDo prüfen
\newglossaryentry{Wort}{
	name        ={Wort},
	plural      ={Wörter},
	see         ={Formelmenge},
	description ={
		Synonym: \Formel\ ---
		Ein Element einer \Sprache.
	}
}

%Z === Z === Z === Z === Z === Z === Z === Z === Z === Z === Z === Z === Z === Z

\newcommand*    {\Zeichenfolge} [1][hyperTxt]{\glsIdx  [#1]{Zeichenfolge}}
\newcommand*    {\Zeichenfolgen}[1][hyperTxt]{\glsIdxPl[#1]{Zeichenfolge}}
%ToDo prüfen
\newglossaryentry{Zeichenfolge}{
	name        ={Zeichenfolge},
	plural      ={Zeichenfolgen},
	see         ={Zeichenkette},
	description ={
		Eine Folge von \Symbolen, wobei Leerstellen und sonstiger Zwischenraum nicht zählen und nur zur besseren Darstellung dienen.
		Dabei sind als spezielle \Symbole\ auch \Zeichenketten\ erlaubt, solange die Zerlegung eindeutig bleibt.
		\textZB\ kann \chrqt{sin} als ein einzelnes \Symbol\ --- für die Sinusfunktion --- aufgefasst werden, aber auch als Folge von den Buchstaben \chrqt{s}, \chrqt{i} und \chrqt{n}.
		\Formeln\ werden immer als \Zeichenfolgen\ aufgefasst.
	}
}
\newcommand*    {\Zeichenkette} [1][hyperTxt]{\glsIdx  [#1]{Zeichenkette}}
\newcommand*    {\Zeichenketten}[1][hyperTxt]{\glsIdxPl[#1]{Zeichenkette}}
%ToDo prüfen
\newglossaryentry{Zeichenkette}{
	name        ={Zeichenkette},
	plural      ={Zeichenketten},
	see         ={Zeichenfolge},
	description ={
		Eine Folge von (typographischen) Zeichen, auch Leerstellen und sonstigem Zwischenraum.
	}
}
\newcommand*    {\zerlegbar} [1][hyperTxt]{\glsIdx  [#1]{zerlegbar}}
\newcommand*    {\zerlegbare}[1][hyperTxt]{\glsIdxPl[#1]{zerlegbar}}
\newcommand*    {\Zerlegbare}[1][hyperTxt]{\GlsIdxPl[#1]{zerlegbar}}
%ToDo prüfen
\newglossaryentry{zerlegbar}{
	name        ={zerlegbar},
	plural      ={zerlegbare},
	see         ={unzerlegbar},
	description ={
		Eine \Aussage, die eine \Metaoperation, \textbzw\ eine \Formel, die eine \Operation\ oder eine \Relation\ enthält, heißen \zerlegbar.
	}
}
\newcommand*    {\Ziel} [1][hyperTxt]{\glsIdx  [#1]{Ziel}}
\newcommand*    {\Ziele}[1][hyperTxt]{\glsIdxPl[#1]{Ziel}}
%ToDo prüfen
\newglossaryentry{Ziel}{
	name        ={Ziel},
	plural      ={Ziele},
	description ={
		Ein \defFt{Ziel} ist in diesem Dokument eine Anforderungen an \ASBA.
	}
}
\newcommand*    {\Zielbereich} [1][hyperTxt]{\glsIdx  [#1]{Zielbereich}}
\newcommand*    {\Zielbereiche}[1][hyperTxt]{\glsIdxPl[#1]{Zielbereich}}
%ToDo prüfen
\newglossaryentry{Zielbereich}{
	name        ={Zielbereich},
	plural      ={Zielbereiche},
	symbol      ={\MtsZb},
	see         = {MtsZb,Wertebereich,Funktion},
	description ={
		einer \Funktion.
	}
}
\newcommand*    {\zulaessig}  [1][hyperTxt]{\glsIdx  [#1]{zulaessig}}
\newcommand*    {\zulaessige} [1][hyperTxt]{\glsIdxPl[#1]{zulaessig}}
\newcommand*    {\zulaessigen}[1][hyperTxt]{\glsIdx  [#1]{zulaessig}en}
\newcommand*    {\zulaessiger}[1][hyperTxt]{\glsIdxPl[#1]{zulaessig}r}
%ToDo prüfen
\newglossaryentry{zulaessig}{
	name        ={zulässig},
	plural      ={zulässige},
	see         ={Formel,Umwandlung,Ersetzung},
	description ={
		Eine Eigenschaft von \Formel, \Umwandlung\ und \Ersetzung.
	}
}
