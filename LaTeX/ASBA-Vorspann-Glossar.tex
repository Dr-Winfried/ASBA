%%############################################################################%%
%%                                                                            %%
%% Datei:  ASBA-Vorspann-Glossar.tex                                          %%
%% Inhalt: Vorspann Glossareinträge für ASBA                                  %%
%%                                                                            %%
%% Copyright (C) 2017  Winfried Teschers                                      %%
%%                                                                            %%
%% This program is free software: you can redistribute it and/or modify       %%
%% it under the terms of the GNU Affero General Public License as published   %%
%% by the Free Software Foundation, either version 3 of the License, or       %%
%% (at your option) any later version.                                        %%
%%                                                                            %%
%% This program is distributed in the hope that it will be useful,            %%
%% but WITHOUT ANY WARRANTY; without even the implied warranty of             %%
%% MERCHANTABILITY or FITNESS FOR A PARTICULAR PURPOSE.  See the              %%
%% GNU Affero General Public License for more details.                        %%
%%                                                                            %%
%% You should have received a copy of the GNU Affero General Public License   %%
%% along with this program.  If not, see <http://www.gnu.org/licenses/>.      %%
%%                                                                            %%
%% Dr. Winfried Teschers                                                      %%
%% Anton-Günther-Straße 26c                                                   %%
%% 91083 Baiersdorf                                                           %%
%% Germany                                                                    %%
%%                                                                            %%
%% e-mail: winfried.teschers@t-online.de                                      %%
%%                                                                            %%
%%############################################################################%%
{%%%%
% !TeX root = ASBA.tex
% !TeX encoding = UTF-8
% !TeX spellcheck = de_DE

\newglossary[nlg]{symbols}{not}{ntn}{Symbolverzeichnis}

% Elemente, die keine Glossareinträge sind und dafür nicht gebraucht werden,
% werden in "ASBA-Vorspann.tex" und "ASBA-Vorspann-Logik.tex" definiert.

\newif\ifmitGlossarZusatzFlg% Schalter ob Glossar zu berücksichtigen ist; AUS
\mitGlossarZusatzFlgtrue% ... jetzt EIN
\newcommand*{\glsdeskOhneZusatz}[1]{%
	\mitGlossarZusatzFlgfalse%
	\glsentrydesc{#1}%
	\mitGlossarZusatzFlgtrue%
}
\newcommand{\GlossarZusatz}[1]{\ifmitGlossarZusatzFlg #1\else\fi}% ohne *!

\newcommand*{\SymbolAmRand}[1]{\marginpar{\raggedright{\glsentrysymbol{#1}}}}

\newcommand*{\baueBeschreibung}[3][\hspace{0.5em}]{% [Zwischentext]{glo-Label}{Symbolart}
	#3:\hspace{0.5em}\glsentrysymbol{#2}#1%
	\def\ArgumentEins{#1}%
	\def\ASBAundefined{}% 'user6' nur mit Zwischentext
	\ifx\ArgumentEins\ASBAundefined\else[\glsentryuservi{#2}]\fi%
	\SymbolAmRand{#2}%
}

\newenvironment{wikicite}[2][\Wikipedia]{%
	#1 \cite{#2} schreibt dazu:
	\begin{quote}\sffamily
}    {\end{quote}}

% Text-Fonts
\newcommand*{\citeFt}[1]{\textsf{#1}}% Zitate
\newcommand*{\defFt} [1]{\textsf{\textbf{#1}}}% Definitionen
\newcommand*{\optFt} [1]{\textit{#1}}% optionale Teile von Sprechweisen
\newcommand*{\manFt} [1]{\textbf{#1}}% mandatory (notwendige) Teile von Sprechweisen
\newcommand*{\GloFt} [1]{\textsf{\textbf{#1}}}% Nur im Glossar: Wenn eine Bezeichnung in ihrer Definition auftaucht
\newcommand*{\gloFt} [1]{\textsf{#1}}% Nur im Glossar: Verweis ohne Link ins Glossar
\newcommand*{\likehyperDef}[1]{\defFt{\linkcolor{#1}}}% simuliert Verweise bei Definitionen
\newcommand*{\likehyperTxt}[1]       {\linkcolor{#1}}%  simuliert Verweise ins Glossar

% Fonts für die Liste der Seitenangaben
\newcommand*      {\hyperDef}[1]{\textsf{\hyperbf{#1}}}% für Seitenangaben bei Definitionen
\newcommand*      {\hyperTxt}[1]        {\hyperrm{#1}}%  für Seitenangaben im Glossar
\GlsAddXdyAttribute{hyperDef}% damit xindy damit umgehen kann
\GlsAddXdyAttribute{hyperTxt}% damit xindy damit umgehen kann

\GlsSetXdyMinRangeLength{2}% Seitenbereiche ab ...
\makeglossaries
\setacronymstyle{long-sc-short}
\renewcommand*{\glsnumberformat}[1]{\hyperTxt{#1}}% Standardformat für Seitenliste
\renewcommand*{\glsnumlistlastsep}{, }% Liste auch am Ende mit Komma trennen

% Makros für neue Bezeichnungen ================================================
% [<Anhang>]{<Makro>}{<gls-Makro>}{<glo-Label>}
\newcommand*{\newVerweis}  [4][]{\newcommand*{#2}[1][]{#3[##1]{#4}[#1]}}
\newcommand*{\dummyVerweis}[4][]{\newcommand*{#2}[1][]{\linkcolor{#4#1}}}

% neue Benennungen -------------------------------------------------------------
% Ausgabe und Aufnahme und Link ins Glossar mit Hervorhebung der Seitennummer
% {<Makro>} - An <Makro> muss [format=<Font-Makro>] angehängt werden können
\newcommand*{\defTxt}[1]{\defFt{#1[format=hyperDef]}}
\newcommand*{\defGlo}[1]{\defFt{#1[format=hyperDef]}\GD}
\newcommand*{\GD}{*}

% neue Symbole -----------------------------------------------------------------
% Ausgabe und Aufnahme ins Symbolverzeichnis, ohne Link, mit Hervorhebung der Seitennummer
% {<Glossary Key>}
\newcommand*{\glsTag}[1]{\tag{\gls*{#1}}\glsadd[format=hyperDef]{#1}}% ohne Link ins Symbolverzeichnis
% Ausgabe und Aufnahme und Link ins Symbolverzeichnis mit Hervorhebung der Seitennummer
% {<Makro>} - An <Makro> muss [<Font-Makro>] angehängt werden können
\newcommand*{\defSym}   [1]          {#1[format=hyperDef]}
\newcommand*{\defSymUna}[1]  {\defSym{#1}\;}%  unär: folgender  Abstand
\newcommand*{\defSymBin}[1]{\;\defSym{#1}\;}% binär: umgebender Abstand

% Automatischer Index ==========================================================
\newcommand*{\addIdx}[2][]{% [gloKeys]{Index-Eintrag} - Index hinzufügen
	\newterm[name={#2},#1]{ind-#2}
	\glsadd               {ind-#2}
}
\newcommand*{\idx}[2][]{% [Text]{Index-Eintrag} - Ausgabe und Index hinzufügen
	\addIdx{#2}
	\def\ArgumentEins{#1}
	\def\ASBAundefined{}
	\ifx\ArgumentEins\ASBAundefined #2\else #1\fi%  wenn Text leer, Index-Eintrag ausgeben
}

% Synonyme =====================================================================

% [<Benennung>]{Makro für die <Benennungen>}{<Benennungen-Label>}{<Makro für das Synonym>}
\newcommand*{\newsynonym}[4][]{
	\def\ASBAundefined{}
	\def\ArgumentEins{#1}
	\ifx\ArgumentEins\ASBAundefined
		\newcommand*     {#2}[1][]{\glstext[##1]{#3}}
		\newglossaryentry{#3}{
			name        ={#3 \addIdx[
				name    ={#3}]                  {#3}},
			text        ={#3},
			description ={Synonym zu #4.}
		}
	\else
		\newcommand*     {#2}[1][]{\glstext[##1]{#3}}
		\newglossaryentry{#3}{
			name        ={#1 \addIdx[
				name    ={#1}]                  {#3}},
			text        ={#1},
			description ={Synonym zu #4.}
		}
	\fi
}

% Glossar-Einträge #############################################################

% ### Symbolverzeichnis: Symbole ###############################################

% ==============================================================================
% \* - Ausgabe als Symbol und Eintrag (und Verweis) ins Symbolverzeichnis
% Fachbegriffe =================================================================

\iftestFlg% Definition von Dummy Glossareinträgen
	\newVerweis     {\dummy} {\glstext}{dummy}
	\newglossaryentry{dummy}{
		text       ={\ensuremath{\#}},
		name       ={\ensuremath{\#} \addIdx[
			name   ={\ensuremath{dummy}},
			sort   ={dummy}]           {dummy}},
		sort       ={=},
		type       ={symbols},
		symbol     ={},
		user6      ={},
		see        ={},
		description={
			\todo{Beschreibung fehlt noch}% Todo=dummy
		}
	}
\else\fi

% ==============================================================================
% \Bsp* - Ausgabe als Symbol und Eintrag und Verweis ins Symbolverzeichnis
\newglossaryentry{Glo-Beispielsymbole}{
	name     ={\gloFt{Beispielsymbole} für \gloFt{Operationen} und \gloFt{Relationen}},% =====
	sort  ={= 0 0 0},
	type  ={symbols},
	see   ={Beispielsymbol,Objekt,Operation,Relation},
	description={Im Folgenden seien $A$ und $B$ passende Objekte.}
}

\newVerweis               {\BspOpU}{\glstext}{BspOpU}
\newglossaryentry          {BspOpU}{
	name  ={\ensuremath{\RawBspOpU}},
	sort  ={= 0 1 1},
	type  ={symbols},
	symbol={\ensuremath{\RawBspOpU A}},
	see   ={Beispielsymbol,Operation,unaer},
	description={
		\baueBeschreibung[]{BspOpU}{\gloFt{Beispielsymbol} für eine \gloFt{unäre} \gloFt{Operation}}.
	}
}

\newVerweis               {\BspOpB}{\glstext}{BspOpB}
\newglossaryentry          {BspOpB}{
	name  ={\ensuremath{\RawBspOpB}},
	sort  ={= 0 1 2},
	type  ={symbols},
	symbol={\ensuremath{A \RawBspOpB B}},
	see   ={Beispielsymbol,binaer,Operation},
	description={
		\baueBeschreibung[]{BspOpB}{\gloFt{Beispielsymbol} für eine \gloFt{binäre} \gloFt{Operation}}.
	}
}

\newVerweis               {\BspRel}{\glstext}{BspRel}
\newglossaryentry          {BspRel}{
	name  ={\ensuremath{\RawBspRel}},
	sort  ={= 0 2 1},
	type  ={symbols},
	symbol={\ensuremath{A \RawBspRel B}},
	see   ={Beispielsymbol,binaer,Relation},
	description={
		\baueBeschreibung[]{BspRel}{\gloFt{Beispielsymbol} für eine \gloFt{binäre} \gloFt{Relation}}.
	}
}

\newVerweis               {\BspRelEq}{\glstext}{BspRelEq}
\newglossaryentry          {BspRelEq}{
	name  ={\ensuremath{\RawBspRelEq}},
	sort  ={= 0 2 2},
	type  ={symbols},
	symbol={\ensuremath{A \RawBspRelEq B}},
	see   ={Beispielsymbol,binaer,Relation},
	description={
		\baueBeschreibung[]{BspRelEq}{\gloFt{Beispielsymbol} für eine \gloFt{binäre} \gloFt{Relation}}.
	}
}

\newVerweis               {\BspRelBck}{\glstext}{BspRelBck}
\newglossaryentry          {BspRelBck}{
	name  ={\ensuremath{\RawBspRelBck}},
	sort  ={= 0 2 3},
	type  ={symbols},
	symbol={\ensuremath{(A \RawBspRelBck B) \RawMtsDefEquiv (B \RawBspRel A)}},
	see   ={BspRel,MtsDefEquiv,Beispielsymbol,binaer,Relation,Umkehrrelation},
	description={
		\baueBeschreibung[]{BspRelBck}{\gloFt{Beispielsymbol} für eine \gloFt{binäre} \gloFt{Relation}}.
		\\Die \gloFt{Umkehrrelation} von $\RawBspRel$.
	}
}

\newVerweis               {\BspRelBckEq}{\glstext}{BspRelBckEq}
\newglossaryentry          {BspRelBckEq}{
	name  ={\ensuremath{\RawBspRelBckEq}},
	sort  ={= 0 2 4},
	type  ={symbols},
	symbol={\ensuremath{(A \RawBspRelBckEq B) \RawMtsDefEquiv (B \RawBspRelEq A)}},
	see   ={BspRelEq,MtsDefEquiv,Beispielsymbol,binaer,Relation,Umkehrrelation},
	description={
		\baueBeschreibung[]{BspRelBckEq}{\gloFt{Beispielsymbol} für eine \gloFt{binäre} \gloFt{Relation}}.
		\\Die \gloFt{Umkehrrelation} von $\RawBspRelEq$.
	}
}

\newVerweis               {\BspRelN}{\glstext}{BspRelN}
\newglossaryentry          {BspRelN}{
	name  ={\ensuremath{\RawBspRelN}},
	sort  ={= 0 3 1},
	type  ={symbols},
	symbol={\ensuremath{(A \RawBspRelN B) \RawMtsDefEquiv \RawMtsNot (A \RawBspRel B)}},
	see   ={BspRel,MtsNot,MtsDefEquiv,Beispielsymbol,binaer,Negation,Relation},
	description={
		\baueBeschreibung[]{BspRelN}{\gloFt{Beispielsymbol} für eine \gloFt{binäre} \gloFt{Relation}}.
		\\Die \gloFt{Negation} von $\RawBspRel$.
	}
}

\newVerweis               {\BspRelEqN}{\glstext}{BspRelEqN}
\newglossaryentry          {BspRelEqN}{
	name  ={\ensuremath{\RawBspRelEqN}},
	sort  ={= 0 3 2},
	type  ={symbols},
	symbol={\ensuremath{(A \RawBspRelEqN B) \RawMtsDefEquiv \RawMtsNot (A \RawBspRelEq B)}},
	see   ={BspRelEq,MtsNot,MtsDefEquiv,Beispielsymbol,binaer,Negation,Relation},
	description={
		\baueBeschreibung[]{BspRelEqN}{\gloFt{Beispielsymbol} für eine \gloFt{binäre} \gloFt{Relation}}.
		\\Die \gloFt{Negation} von $\RawBspRelEq$.
	}
}

\newVerweis               {\BspRelBckN}{\glstext}{BspRelBckN}
\newglossaryentry          {BspRelBckN}{
	name  ={\ensuremath{\RawBspRelBckN}},
	sort  ={= 0 3 3},
	type  ={symbols},
	symbol={\ensuremath{(A \RawBspRelBckN B) \RawMtsDefEquiv \RawMtsNot (B \RawBspRel A)}},
	see   ={BspRel,MtsNot,MtsDefEquiv,Beispielsymbol,binaer,Negation,Relation,Umkehrrelation},
	description={
		\baueBeschreibung[]{BspRelBckN}{\gloFt{Beispielsymbol} für eine \gloFt{binäre} \gloFt{Relation}}.
		\\Die \gloFt{Negation} der \gloFt{Umkehrrelation} und gleichzeitig die \gloFt{Umkehrrelation} der \gloFt{Negation} von $\RawBspRel$.
	}
}

\newVerweis               {\BspRelBckEqN}{\glstext}{BspRelBckEqN}
\newglossaryentry          {BspRelBckEqN}{
	name  ={\ensuremath{\RawBspRelBckEqN}},
	sort  ={= 0 3 4},
	type  ={symbols},
	symbol={\ensuremath{(A \RawBspRelBckEqN B) \RawMtsDefEquiv \RawMtsNot (B \RawBspRelEq A)}},
	see   ={BspRelEq,MtsNot,MtsDefEquiv,Beispielsymbol,binaer,Negation,Relation,Umkehrrelation},
	description={
		\baueBeschreibung[]{BspRelBckEqN}{\gloFt{Beispielsymbol} für eine \gloFt{binäre} \gloFt{Relation}}.
		\\Die \gloFt{Negation} der \gloFt{Umkehrrelation} und gleichzeitig die \gloFt{Umkehrrelation} der \gloFt{Negation} von $\RawBspRelEq$.
	}
}

% ==============================================================================
% \Mts* - Ausgabe als Symbol und Eintrag und Verweis ins Symbolverzeichnis
% In 'symbol' steht die formale Definition, in 'user6' die Benennung, ggf.
% in einer Formel: Wörter, die weggelassen werden können, kursiv
\newglossaryentry{Glo-MetaAussagen}{
	name  ={\gloFt{Metaoperationen} und \gloFt{-relationen} mit \gloFt{Aussagen}},% =====
	sort  ={= 1 0 0},
	type  ={symbols},
	see   ={metasprachlicheAussage,Metaoperation,Metarelation},
	description={Im Folgenden seien $A$ und $B$ beliebige \gloFt{metasprachliche Aussagen}.}
}

\newVerweis               {\MtsNot}{\glstext}{MtsNot}
\newglossaryentry          {MtsNot}{
	name  ={\ensuremath{\RawMtsNot}},
	sort  ={= 1 1 1},
	type  ={symbols},
	symbol={\ensuremath{\RawMtsNot A}},
	user6 ={\optFt{es gilt} \manFt{nicht} $A$},
	see   ={OjkNot,Metaoperation,unaer},
	description={\baueBeschreibung{MtsNot}{Eine \gloFt{unäre} \gloFt{Metaoperation}}.}
}

\newVerweis               {\MtsAnd}{\glstext}{MtsAnd}
\newglossaryentry          {MtsAnd}{
	name  ={\ensuremath{\RawMtsAnd}},
	sort  ={= 1 1 2},
	type  ={symbols},
	symbol={\ensuremath{(A \RawMtsAnd B)}},
	user6 ={\optFt{es gilt} $A$ \manFt{und} $B$},
	see   ={MtsUnd,OjkAnd,binaer,Metaoperation},
	description={\baueBeschreibung{MtsAnd}{Eine \gloFt{binäre} \gloFt{Metaoperation}}.}
}

\newVerweis               {\MtsOr}{\glstext}{MtsOr}
\newglossaryentry          {MtsOr}{
	name  ={\ensuremath{\RawMtsOr}},
	sort  ={= 1 1 3},
	type  ={symbols},
	symbol={\ensuremath{(A \RawMtsOr B)}},
	user6 ={\optFt{es gilt} $A$ \manFt{oder} $B$},
	see   ={OjkOr,binaer,Metaoperation},
	description={\baueBeschreibung{MtsOr}{Eine \gloFt{binäre} \gloFt{Metaoperation}}.}
}

\newVerweis               {\MtsUnd}{\glstext}{MtsUnd}
\newglossaryentry          {MtsUnd}{
	name  ={\ensuremath{\RawMtsUnd}},
	sort  ={= 1 2 1},
	type  ={symbols},
	symbol={\ensuremath{(A \RawMtsUnd B) \RawMtsDefEquiv (A \RawMtsAnd B)}},
	user6 ={\optFt{es gilt }$A$ \manFt{und} $B$},
	see   ={MtsAnd,MtsDefEquiv,OjkAnd,binaer,Metaoperation,Schlussregel},
	description={
		\baueBeschreibung{MtsUnd}{Eine \gloFt{binäre} \gloFt{Metaoperation}}.
		\\Nur in \gloFt{Schlussregeln}!
	}
}

\newVerweis               {\MtsImp}{\glstext}{MtsImp}
\newglossaryentry          {MtsImp}{
	name  ={\ensuremath{\RawMtsImp}},
	sort  ={= 1 3 1},
	type  ={symbols},
	symbol={\ensuremath{(A \RawMtsImp B)}},
	user6 ={\optFt{wenn} $A$ \optFt{gilt,} \manFt{dann} \optFt{gilt auch} $B$},
	see   ={OjkImp,binaer,Metarelation},
	description={\baueBeschreibung{MtsImp}{Eine \gloFt{binäre} \gloFt{Metarelation}}.}
}

\newVerweis               {\MtsRep}{\glstext}{MtsRep}
\newglossaryentry          {MtsRep}{
	name  ={\ensuremath{\RawMtsRep}},
	sort  ={= 1 3 2},
	type  ={symbols},
	symbol={\ensuremath{(A \RawMtsRep B) \RawMtsDefEquiv (B \RawMtsImp A)}},
	user6 ={$A$ \optFt{gilt dann,} \manFt{wenn} $B$ \optFt{gilt}},
	see   ={MtsImp,MtsDefEquiv,OjkRep,binaer,Metarelation,Umkehrrelation},
	description={
		\baueBeschreibung{MtsRep}{Eine \gloFt{binäre} \gloFt{Metarelation}}.
		\\Die \gloFt{Umkehrrelation} von $\RawMtsImp$.
	}
}

\newVerweis               {\MtsEquiv}{\glstext}{MtsEquiv}
\newglossaryentry          {MtsEquiv}{
	name  ={\ensuremath{\RawMtsEquiv}},
	sort  ={= 1 3 3},
	type  ={symbols},
	symbol={\ensuremath{(A \RawMtsEquiv B) \RawMtsDefEquiv ((A \RawMtsImp B) \RawMtsAnd (B \RawMtsImp A))}},
	user6 ={$A$ \optFt{gilt genau} \manFt{dann wenn} $B$ \optFt{gilt}},
	see   ={MtsAnd,MtsImp,MtsDefEquiv,OjkEquiv,binaer,Metarelation},
	description={\baueBeschreibung{MtsEquiv}{Eine \gloFt{binäre} \gloFt{Metarelation}}.}
}

\newVerweis               {\MtsDefEquiv}{\glstext}{MtsDefEquiv}
\newglossaryentry          {MtsDefEquiv}{
	name  ={\ensuremath{\RawMtsDefEquiv}},
	sort  ={= 1 4 1},
	type  ={symbols},
	symbol={\ensuremath{(A \RawMtsDefEquiv B)}},
	user6 ={$A$ \optFt{gilt} \manFt{definitionsgemäß} \optFt{genau} \manFt{dann wenn} $B$ \optFt{gilt}},
	see   ={Aussagedefinition,binaer,Metarelation},
	description={\baueBeschreibung{MtsDefEquiv}{Die \GloFt{Aussagedefinition} (eine \gloFt{binäre} \gloFt{Metarelation})}.}
}

\newglossaryentry{Glo-MetaObjekte}{
	name  ={\gloFt{Metaoperationen} und \gloFt{-relationen} mit \gloFt{Objekten}},% =====
	sort  ={= 1 5 0},
	type  ={symbols},
	see   ={Metaoperation,Metarelation,metasprachlichesObjekt},
	description={Im Folgenden seien $A$ und $B$ beliebige \gloFt{metasprachliche Objekte}.}
}

\newVerweis               {\MtsEq}{\glstext}{MtsEq}
\newglossaryentry          {MtsEq}{
	name  ={\ensuremath{\RawMtsEq}},
	sort  ={= 1 6 1},
	see   ={OjkEq,binaer,Gleichheit,Metarelation},
	type  ={symbols},
	symbol={\ensuremath{(A \RawMtsEq B)}},
	user6 ={$A$ \optFt{ist} \manFt{gleich} $B$},
	description={\baueBeschreibung{MtsEq}{Eine \gloFt{binäre} \gloFt{Metarelation}}\alternativMtsEq.}
}
\newcommand*{\alternativMtsEq}{\alternativii{dasselbe wie}{identisch zu}}

\newVerweis               {\MtsEqN}{\glstext}{MtsEqN}
\newglossaryentry          {MtsEqN}{
	name  ={\ensuremath{\RawMtsEqN}},
	sort  ={= 1 6 2},
	see   ={MtsNot,MtsDefEquiv,MtsEq,OjkEqN,binaer,Metarelation,Negation,Ungleichheit},
	type  ={symbols},
	symbol={\ensuremath{(A \RawMtsEqN B) \RawMtsDefEquiv \RawMtsNot (A \RawMtsEq B)}},
	user6 ={$A$ \optFt{ist} \manFt{ungleich} $B$},
	description={
		\baueBeschreibung{MtsEqN}{Eine \gloFt{binäre} \gloFt{Metarelation}}\alternativMtsEqN.
		\\Die \gloFt{Negation} von $\RawMtsEq$.
	}
}
\newcommand*{\alternativMtsEqN}{\alternativiii{nicht gleich}{nicht dasselbe wie}{nicht identisch zu}}

\newVerweis               {\MtsAequiv}{\glstext}{MtsAequiv}
\newglossaryentry          {MtsAequiv}{
	name  ={\ensuremath{\RawMtsAequiv}},
	sort  ={= 1 7 1},
	see   ={Aequivalenz,binaer,Metarelation},
	type  ={symbols},
	symbol={\ensuremath{(A \RawMtsAequiv B)}},
	user6 ={$A$ \optFt{ist} \manFt{äquivalent} \optFt{zu} $B$},
	description={\baueBeschreibung{MtsAequiv}{Eine \gloFt{binäre} \gloFt{Metarelation}}\alternativMtsAequiv.}
}
\newcommand*{\alternativMtsAequiv}{\alternativii{so wie}{ähnlich}}

\newVerweis               {\MtsAequivN}{\glstext}{MtsAequivN}
\newglossaryentry          {MtsAequivN}{
	name  ={\ensuremath{\RawMtsAequivN}},
	sort  ={= 1 7 2},
	see   ={MtsNot,MtsDefEquiv,MtsAequiv,Aequivalenz,binaer,Metarelation},
	type  ={symbols},
	symbol={\ensuremath{(A \RawMtsAequivN B) \RawMtsDefEquiv \RawMtsNot (A \RawMtsAequiv B)}},
	user6 ={$A$ \optFt{ist} \manFt{nicht äquivalent} \optFt{zu} $B$},
	description={\baueBeschreibung{MtsAequivN}{Eine \gloFt{binäre} \gloFt{Metarelation}}\alternativMtsAequivN}
}
\newcommand*{\alternativMtsAequivN}{\alternativii{nicht so wie}{nicht ähnlich}}

\newVerweis               {\MtsDefEq}{\glstext}{MtsDefEq}
\newglossaryentry          {MtsDefEq}{
	name  ={\ensuremath{\RawMtsDefEq}},
	sort  ={= 1 7 3},
	type  ={symbols},
	symbol={\ensuremath{(A \RawMtsDefEq B)}},
	user6 ={$A$ \optFt{ist} \manFt{definitionsgemäß gleich} $B$},
	see   ={binaer,Metarelation,Objektdefinition},
	description={\baueBeschreibung{MtsDefEq}{Die \GloFt{Objektdefinition} (eine \gloFt{binäre} \gloFt{Metarelation})}\alternativMtsDefEq.}
}
\newcommand*{\alternativMtsDefEq}{\alternativii{dasselbe wie}{identisch zu}}

\newglossaryentry{Glo-MetaSonstige}{
	name  ={Sonstige \gloFt{Metaoperationen} und \gloFt{-relationen}},% =========
	sort  ={= 1 8 0},
	type  ={symbols},
	see   ={metasprachlicheAussage,Menge},
	description={Im Folgenden seien $A$ und $B$ \gloFt{metasprachliche Aussagen} oder \gloFt{Mengen} davon und $\alpha$ und $\beta$ ???.}
}

\newVerweis               {\MtsDerive}{\glstext}{MtsDerive}
\newglossaryentry          {MtsDerive}{
	name  ={\ensuremath{\RawMtsDerive}},
	sort  ={= 1 8 1},
	type  ={symbols},
	symbol={\ensuremath{(A \RawMtsDerive B)}},
	user6 ={$A$ \optFt{ist} \manFt{ableitbar} \optFt{aus} $B$},
	see   ={ableitbar,Ableitung,Ableitungsrelation,binaer,Darstellung,Metarelation,Relation},
	description={
		\baueBeschreibung{MtsDerive}{Die \GloFt{Ableitungsrelation} (eine \gloFt{binäre} \gloFt{Metarelation})}\alternativMtsDerive.
	}
}
\newcommand*{\alternativMtsDerive}{\synonym{\beweisbar}}

\newVerweis               {\MtsDeriveR}{\glstext}{MtsDeriveR}
\newglossaryentry          {MtsDeriveR}{
	name  ={\ensuremath{\RawMtsDerive_R}},
	sort  ={= 1 8 2},
	type  ={symbols},
	symbol={\ensuremath{(A \RawMtsDerive_R B) \RawMtsDefEquiv ((A,B) \RawMtsIn R_{\RawMtsIdxGraph})}},
	user6 ={$A$ \optFt{ist} \manFt{$R$-ableitbar} \optFt{aus} $B$},
	see   ={MtsDefEquiv,MtsIn,MtsIdxGraph,MtsSprache,MtsPot,MtsRel,ableitbar,Ableitung,Ableitungsrelation,beweisbar,binaer,Metarelation,Relation},
	description={
		\baueBeschreibung{MtsDeriveR}{Die $R$\GloFt{-Ableitungsrelation} (eine \gloFt{binäre} \gloFt{Metarelation})}\alternativMtsDeriveR.
		\\Die Darstellung einer \gloFt{Relation} $R \RawMtsIn \RawMtsRelAllDerive$ als \gloFt{Ableitungsrelation}.
	}
}
\newcommand*{\alternativMtsDeriveR}{\synonym{$R$-\gloFt{beweisbar}}}

\newVerweis               {\MtsSubst}{\glstext}{MtsSubst}
\newglossaryentry          {MtsSubst}{% ToDo Was sind die Operanden?
	name  ={\ensuremath{\RawMtsSubst}},
	sort  ={= 1 8 3},
	type  ={symbols},
	symbol={\ensuremath{(\alpha \RawMtsSubst \beta)}},
	user6 ={$\alpha$ \optFt{wird} \manFt{ersetzt durch} $\beta$},
	see   ={Ersetzung},
	description={
		\baueBeschreibung{MtsSubst}{Die \GloFt{Ersetzung}}.\alternativMtsSubst
	}
}
\newcommand*{\alternativMtsSubst}{\alternativi{substituiert durch}}

\newVerweis               {\MtsSwap}{\glstext}{MtsSwap}
\newglossaryentry          {MtsSwap}{% ToDo Was sind die Operanden?
	name  ={\ensuremath{\RawMtsSwap}},
	sort  ={= 1 8 4},
	type  ={symbols},
	symbol={\ensuremath{(\alpha \RawMtsSwap \beta)}},
	user6 ={$\alpha$ \optFt{wird} \manFt{vertauscht mit} $\beta$},
	see   ={Vertauschung},
	description={\baueBeschreibung{MtsSwap}{Die \GloFt{Vertauschung}}.}
}

% ==============================================================================
% \Mts* - Ausgabe als Symbol und Eintrag und Verweis ins Symbolverzeichnis
\newglossaryentry{Glo-Elementrelationen}{
	name     ={\gloFt{Elementrelationen}},% =============================
	sort            ={= 3 0 0},
	type            ={symbols},
	see             ={Element,Menge},
	description     ={Im Folgenden sei $x$ ein beliebiges \gloFt{Element} und $M$ eine beliebige \gloFt{Menge}.}
}

\newVerweis               {\MtsIn}{\glstext}{MtsIn}
\newglossaryentry          {MtsIn}{
	name  ={\ensuremath{\RawMtsIn}},
	sort  ={= 3 1 1},
	type  ={symbols},
	symbol={\ensuremath{(x \RawMtsIn M)}},
	user6 ={$x$ \optFt{ist ein Element} \manFt{aus} $M$},
	see   ={Element,Elementrelation,Komponente,Mengenlehre,Relation},
	description={
		\baueBeschreibung{MtsIn}{Eine \gloFt{Elementrelation}}\alternativMtsIn.
		\\Die grundlegende \gloFt{Relation} der \gloFt{Mengenlehre}.
	}
}
\newcommand*{\alternativMtsIn}{\alternativi[;
	\enquote{$a$ von $M$} könnte \textzB\ auch \enquote{(\gloFt{Komponente}) $a$ von der \gloFt{Folge} $M$} meinen.
	Daher bevorzugen wir für \gloFt{Elemente} \enquote{aus}.
]{von}}

\newVerweis               {\MtsNi}{\glstext}{MtsNi}
\newglossaryentry          {MtsNi}{
	name  ={\ensuremath{\RawMtsNi}},
	sort  ={= 3 1 2},
	type  ={symbols},
	symbol={\ensuremath{(M \RawMtsNi x) \RawMtsDefEquiv (x \RawMtsIn M)}},
	user6 ={$M$ \manFt{enthält} $x$ \optFt{als Element}},
	see   ={MtsDefEquiv,MtsIn,Element,Elementrelation,Umkehrrelation},
	description={
		\baueBeschreibung{MtsNi}{Eine \gloFt{Elementrelation}}.
		\\Die \gloFt{Umkehrrelation} von $\RawMtsIn$.
	}
}

\newVerweis               {\MtsInN}{\glstext}{MtsInN}
\newglossaryentry          {MtsInN}{
	name  ={\ensuremath{\RawMtsInN}},
	sort  ={= 3 2 1},
	type  ={symbols},
	user6 ={$x$ \optFt{ist} \manFt{nicht aus} $M$},
	symbol={\ensuremath{(x \RawMtsInN M) \RawMtsDefEquiv \RawMtsNot (x \RawMtsIn M)}},
	see   ={MtsNot,MtsDefEquiv,MtsIn,Elementrelation,Negation},
	description={
		\baueBeschreibung{MtsInN}{Eine \gloFt{Elementrelation}}\alternativMtsInN.
		\\Die \gloFt{Negation} von $\RawMtsIn$.
	}
}
\newcommand*{\alternativMtsInN}{\alternativi{kein Element aus}}

\newVerweis               {\MtsNiN}{\glstext}{MtsNiN}
\newglossaryentry          {MtsNiN}{
	name  ={\ensuremath{\RawMtsNiN}},
	sort  ={= 3 2 2},
	type  ={symbols},
	symbol={\ensuremath{(M \RawMtsNiN x) \RawMtsDefEquiv \RawMtsNot (x \RawMtsIn M)}},
	user6 ={$M$ \manFt{enthält} $x$ \manFt{nicht} \optFt{als Element}},
	see   ={MtsNot,MtsDefEquiv,MtsIn,Elementrelation,Negation,Umkehrrelation},
	description={
		\baueBeschreibung{MtsNiN}{Eine \gloFt{Elementrelation}}.
		\\Die \gloFt{Negation} der \gloFt{Umkehrrelation} und gleichzeitig die \gloFt{Umkehrrelation} der \gloFt{Negation} von $\RawMtsIn$.
	}
}

% ==============================================================================
% \Mts* - Ausgabe als Symbol und Eintrag und Verweis ins Symbolverzeichnis
\newglossaryentry{Glo-Mengenoperationen}{
	name  ={\gloFt{Mengenrelationen} und \gloFt{-operationen}},% ===============
	sort  ={= 4 0 0},
	type  ={symbols},
	see   ={Menge,Metaoperation,Metarelation},
	description={
		\footnote{In diesem Dokument \gloFt{Metarelationen} und \gloFt{-operationen}.}
		Im Folgenden seien $M$ und $N$ beliebige \gloFt{Mengen}.
	}
}

\newVerweis               {\MtsSubset}{\glstext}{MtsSubset}
\newglossaryentry          {MtsSubset}{
	name  ={\ensuremath{\RawMtsSubset}},
	sort  ={= 4 1 1},
	type  ={symbols},
	symbol={\ensuremath{(M \RawMtsSubset N) \RawMtsDefEquiv ((M \RawMtsSubsetEq N) \RawMtsAnd (M \RawMtsEqN N))}},
	user6 ={$M$ \optFt{ist eine} \manFt{echte Teilmenge} \optFt{von} $N$},
	see   ={MtsAnd,MtsDefEquiv,MtsEqN,MtsSubsetEq,Mengenrelation,echteTeilmenge},
	description={
		\baueBeschreibung{MtsSubset}{Eine \gloFt{Mengenrelation}}.
		\\Ursprünglich wurde $\RawMtsSubset$ im Sinne von $\RawMtsSubsetEq$ verwendet.
	}
}

\newVerweis               {\MtsSubsetEq}{\glstext}{MtsSubsetEq}
\newglossaryentry          {MtsSubsetEq}{
	name  ={\ensuremath{\RawMtsSubsetEq}},
	sort  ={= 4 1 2},
	type  ={symbols},
	symbol={\ensuremath{(M \RawMtsSubsetEq N) \RawMtsDefEquiv \RawMtsForall x:((x \RawMtsIn M) \RawMtsImp (x \RawMtsIn N))}},
	user6 ={$M$ \optFt{ist eine} \manFt{Teilmenge} \optFt{von} $N$},
	see   ={MtsImp,MtsDefEquiv,MtsIn,MtsForall,Mengenrelation,Teilmenge},
	description={\baueBeschreibung{MtsSubsetEq}{Eine \gloFt{Mengenrelation}}.}
}

\newVerweis               {\MtsSupset}{\glstext}{MtsSupset}
\newglossaryentry          {MtsSupset}{
	name  ={\ensuremath{\RawMtsSupset}},
	sort  ={= 4 1 3},
	type  ={symbols},
	symbol={\ensuremath{(M \RawMtsSupset N) \RawMtsDefEquiv (N \RawMtsSubset M)}},
	user6 ={$M$ \optFt{ist eine} \manFt{echte Obermenge von} $N$},
	see   ={MtsDefEquiv,MtsSubset,MtsSupsetEq,Mengenrelation,echteObermenge,Umkehrrelation},
	description={
		\baueBeschreibung{MtsSupset}{Eine \gloFt{Mengenrelation}}.
		\\Die \gloFt{Umkehrrelation} von $\RawMtsSubset$.
		Ursprünglich wurde $\RawMtsSupset$ im Sinne von $\RawMtsSupsetEq$ verwendet.
	}
}

\newVerweis               {\MtsSupsetEq}{\glstext}{MtsSupsetEq}
\newglossaryentry          {MtsSupsetEq}{
	name  ={\ensuremath{\RawMtsSupsetEq}},
	sort  ={= 4 1 4},
	type  ={symbols},
	symbol={\ensuremath{(M \RawMtsSupsetEq N) \RawMtsDefEquiv (N \RawMtsSubsetEq M)}},
	user6 ={$M$ \optFt{ist eine} \manFt{Obermenge von} $N$},
	see   ={MtsDefEquiv,MtsSubsetEq,Mengenrelation,Obermenge,Umkehrrelation},
	description={
		\baueBeschreibung{MtsSupsetEq}{Eine \gloFt{Mengenrelation}}.
		\\Die \gloFt{Umkehrrelation} von $\RawMtsSubsetEq$.
	}
}

\newVerweis               {\MtsSubsetN}{\glstext}{MtsSubsetN}
\newglossaryentry          {MtsSubsetN}{
	name  ={\ensuremath{\RawMtsSubsetN}},
	sort  ={= 4 2 1},
	type  ={symbols},
	symbol={\ensuremath{(M \RawMtsSubsetN N) \RawMtsDefEquiv \RawMtsNot (M \RawMtsSubset N)}},
	user6 ={$M$ \optFt{ist} \manFt{keine echte Teilmenge} \optFt{von} $N$},
	see   ={MtsNot,MtsDefEquiv,MtsSubset,Mengenrelation,Negation,echteTeilmenge},
	description={
		\baueBeschreibung{MtsSubsetN}{Eine \gloFt{Mengenrelation}}.
		\\Die \gloFt{Negation} von $\RawMtsSubset$.
	}
}

\newVerweis               {\MtsSubsetEqN}{\glstext}{MtsSubsetEqN}
\newglossaryentry          {MtsSubsetEqN}{
	name  ={\ensuremath{\RawMtsSubsetEqN}},
	sort  ={= 4 2 2},
	type  ={symbols},
	symbol={\ensuremath{(M \RawMtsSubsetEqN N) \RawMtsDefEquiv \RawMtsNot (M \RawMtsSubsetEq N)}},
	user6 ={$M$ \optFt{ist} \manFt{keine Teilmenge} \optFt{von} $N$},
	see   ={MtsNot,MtsDefEquiv,MtsSubsetEq,Mengenrelation,Negation,Teilmenge},
	description={
		\baueBeschreibung{MtsSubsetEqN}{Eine \gloFt{Mengenrelation}}.
		\\Die \gloFt{Negation} von $\RawMtsSubsetEq$.
	}
}

\newVerweis               {\MtsSupsetN}{\glstext}{MtsSupsetN}
\newglossaryentry          {MtsSupsetN}{
	name  ={\ensuremath{\RawMtsSupsetN}},
	sort  ={= 4 2 3},
	type  ={symbols},
	symbol={\ensuremath{(M \RawMtsSupsetN N) \RawMtsDefEquiv \RawMtsNot (N \RawMtsSubset M)}},
	user6 ={$M$ \optFt{ist} \manFt{keine echte Obermenge von} $N$},
	see   ={MtsNot,MtsDefEquiv,MtsSubset,Mengenrelation,Negation,Umkehrrelation},
	description={
		\baueBeschreibung{MtsSupsetN}{Eine \gloFt{Mengenrelation}}.
		\\Die \gloFt{Negation} der \gloFt{Umkehrrelation} und gleichzeitig die \gloFt{Umkehrrelation} der \gloFt{Negation} von $\RawMtsSubset$.
	}
}

\newVerweis               {\MtsSupsetEqN}{\glstext}{MtsSupsetEqN}
\newglossaryentry          {MtsSupsetEqN}{
	name  ={\ensuremath{\RawMtsSupsetEqN}},
	sort  ={= 4 2 4},
	type  ={symbols},
	symbol={\ensuremath{(M \RawMtsSupsetEqN N) \RawMtsDefEquiv \RawMtsNot (N \RawMtsSubsetEq M)}},
	user6 ={$M$ \optFt{ist} \manFt{keine Obermenge von} $N$},
	see   ={MtsNot,MtsDefEquiv,MtsSubsetEq,Mengenrelation,Negation,Umkehrrelation},
	description={
		\baueBeschreibung{MtsSupsetEqN}{Eine \gloFt{Mengenrelation}}.
		\\Die \gloFt{Negation} der \gloFt{Umkehrrelation} und gleichzeitig die \gloFt{Umkehrrelation} der \gloFt{Negation} von $\RawMtsSubsetEq$.
	}
}

\newVerweis               {\MtsCap}{\glstext}{MtsCap}
\newglossaryentry          {MtsCap}{
	name  ={\ensuremath{\RawMtsCap}},
	sort  ={= 4 3 1},
	type  ={symbols},
	symbol={\ensuremath{M \RawMtsCap N \RawMtsDefEq \RawMengeDef{x}{(x \RawMtsIn M) \RawMtsAnd (x \RawMtsIn N)}}},
	user6 ={\optFt{Der} \manFt{Durchschnitt von} $M$ \manFt{und} $N$},
	see   ={MtsAnd,MtsDefEq,MtsIn,Durchschnitt,Menge,Mengenoperation},
	description={\baueBeschreibung{MtsCap}{Eine \gloFt{Mengenoperation}}.}
}

\newVerweis               {\MtsCup}{\glstext}{MtsCup}
\newglossaryentry          {MtsCup}{
	name  ={\ensuremath{\RawMtsCup}},
	sort  ={= 4 3 2},
	type  ={symbols},
	symbol={\ensuremath{M \RawMtsCup N \RawMtsDefEq \RawMengeDef{x}{(x \RawMtsIn M) \RawMtsOr (x \RawMtsIn N)}}},
	user6 ={\optFt{Die} \manFt{Vereinigung von} $M$ \manFt{und} $N$},
	see   ={MtsOr,MtsDefEq,MtsIn,Menge,Mengenoperation,Vereinigung},
	description={\baueBeschreibung{MtsCup}{Eine \gloFt{Mengenoperation}}.}
}

\newVerweis               {\MtsSetminus}{\glstext}{MtsSetminus}
\newglossaryentry          {MtsSetminus}{
	name  ={\ensuremath{\RawMtsSetminus}},
	sort  ={= 4 3 3},
	type  ={symbols},
	symbol={\ensuremath{M \RawMtsSetminus N \RawMtsDefEq \RawMengeDef{x}{(x \RawMtsIn M) \RawMtsAnd (x \RawMtsInN N)}}},
	user6 ={\optFt{Die} \manFt{Differenz von} $M$ \manFt{und} $N$},
	see   ={MtsAnd,MtsDefEq,MtsIn,MtsInN,Differenz,Menge,Mengenoperation},
	description={\baueBeschreibung{MtsSetminus}{Eine \gloFt{Mengenoperation}}.}
}

%%% Todo Formel wird fälschlicherweise blau auf den Rand geschrieben
\newVerweis               {\MtsTimes}{\glstext}{MtsTimes}
\newglossaryentry          {MtsTimes}{
	name  ={\ensuremath{\RawMtsTimes}},
	sort  ={= 4 3 4},
	type  ={symbols},
	symbol={\ensuremath{M \RawMtsTimes N \RawMtsDefEq \RawMengeDef{(x,y)}{(x \RawMtsIn M) \RawMtsAnd (y \RawMtsIn N)}}},
	user6 ={\optFt{Das} \manFt{kartesische Produkt von} $M$ \manFt{und} $N$},
	see   ={MtsAnd,MtsDefEq,MtsIn,kartesischesProdukt,Menge,Mengenoperation,Mengenprodukt},
	description={
		\baueBeschreibung{MtsTimes}{Eine \gloFt{Mengenoperation}}\synonymMtsTimes.
	}
}
\newcommand*{\synonymMtsTimes}{\synonym{\gloFt{Mengenprodukt}}}

% ==============================================================================
% \MtsSeq* - Ausgabe als Symbol und Eintrag und Verweis ins Symbolverzeichnis
\newglossaryentry{Glo-Komponentenrelationen}{
	name     ={\gloFt{Komponentenrelationen}},% ===================================
	sort  ={= 5 0 0},
	type  ={symbols},
	see   ={Folge,Komponentenrelation},
	description={
		Im Folgenden sei $x$ eine beliebige \gloFt{Komponente} und $F$ eine beliebige \gloFt{Folge}.
	}
}

\newVerweis               {\MtsSeqIn}{\glstext}{MtsSeqIn}
\newglossaryentry          {MtsSeqIn}{
	name  ={\ensuremath{\RawMtsSeqIn}},
	sort  ={= 5 1 1},
	type  ={symbols},
	symbol={\ensuremath{(x \RawMtsSeqIn F)}},
	user6 ={$x$ \optFt{ist eine} \manFt{Komponente von} $F$},
	see   ={Element,Komponente,Komponentenrelation},
	description={
		\baueBeschreibung{MtsSeqIn}{Eine \gloFt{Komponentenrelation}}.\alternativMtsSeqIn
	}
}
\newcommand*{\alternativMtsSeqIn}{\alternativi[;
	\enquote{$x$ aus $F$} könnte \textzB\ auch \enquote{(\gloFt{Element}) $x$ aus der Menge $F$} meinen.
	Daher bevorzugen wir für \gloFt{Komponenten} \enquote{von}.
]{aus}}

\newVerweis               {\MtsSeqNi}{\glstext}{MtsSeqNi}
\newglossaryentry          {MtsSeqNi}{
	name  ={\ensuremath{\RawMtsSeqNi}},
	sort  ={= 5 1 2},
	type  ={symbols},
	symbol={\ensuremath{(F \RawMtsSeqNi x) \RawMtsDefEq (x \RawMtsSeqIn F)}},
	user6 ={$F$ \manFt{enthält} $x$ \optFt{als Komponente}},
	see   ={MtsDefEq,MtsSeqIn,Komponente,Komponentenrelation,Umkehrrelation},
	description={
		\baueBeschreibung{MtsSeqNi}{Eine \gloFt{Komponentenrelation}}.
		\\Die \gloFt{Umkehrrelation} von $\RawMtsSeqIn$.
	}
}

\newVerweis               {\MtsSeqInN}{\glstext}{MtsSeqInN}
\newglossaryentry          {MtsSeqInN}{
	name  ={\ensuremath{\RawMtsSeqInN}},
	sort  ={= 5 2 1},
	type  ={symbols},
	symbol={\ensuremath{(x \RawMtsSeqInN F) \RawMtsDefEq \RawMtsNot (x \RawMtsSeqIn F)}},
	user6 ={$x$ \optFt{ist} \manFt{keine Komponente aus} $F$},
	see   ={MtsNot,MtsDefEq,MtsSeqIn,Komponente,Komponentenrelation,Negation},
	description={
		\baueBeschreibung{MtsSeqInN}{Eine \gloFt{Komponentenrelation}}.
		\\Die \gloFt{Negation} von $\RawMtsSeqIn$.
	}
}

\newVerweis               {\MtsSeqNiN}{\glstext}{MtsSeqNiN}
\newglossaryentry          {MtsSeqNiN}{
	name  ={\ensuremath{\RawMtsSeqNiN}},
	sort  ={= 5 2 2},
	type  ={symbols},
	symbol={\ensuremath{(F \RawMtsSeqNiN x) \RawMtsDefEq \RawMtsNot (x \RawMtsSeqIn F)}},
	user6 ={$F$ \manFt{enthält} $x$ \manFt{nicht} \optFt{als Komponente}},
	see   ={MtsNot,MtsDefEq,MtsSeqIn,Komponente,Komponentenrelation,Negation,Umkehrrelation},
	description={
		\baueBeschreibung{MtsSeqNiN}{Eine \gloFt{Komponentenrelation}}.
		\\Die \gloFt{Negation} der \gloFt{Umkehrrelation} und gleichzeitig die \gloFt{Umkehrrelation} der \gloFt{Negation} von $\RawMtsSeqIn$.
	}
}

% ==============================================================================
% \Mts* - Ausgabe als Symbol und Eintrag und Verweis ins Symbolverzeichnis
\newglossaryentry{Glo-Folgenrelationen}{
	name  ={\gloFt{Folgenoperationen} und \gloFt{-relationen}},% ===============================
	sort  ={= 6 0 0},
	see   ={Folge,Folgenoperation,Folgenrelation,Tupel},
	type  ={symbols},
	description={
		Im Folgenden seien $\vec{a}$ eine endliche und $\vec{b}$, $\vec{c}$ und $\vec{d}$ beliebige \gloFt{Folgen}.
	}
}

\newVerweis               {\MtsCat}{\glstext}{MtsCat}
\newglossaryentry          {MtsCat}{
	name  ={\ensuremath{\RawMtsCat}},
	sort  ={= 6 0 1},
	type  ={symbols},
	symbol={\ensuremath{\{a_1, \dots, a_n\} \RawMtsCat \{c_1, c_2, \dots\} \RawMtsDefEq \{a_1, \dots, a_n, c_1, c_2, \dots\}}},
	user6 ={$\vec{a}$ \manFt{verkettet} \optFt{mit} $\vec{c}$},
	see   ={MtsDefEq,Folge,Folgenoperation,Verkettung},
	description={\baueBeschreibung{MtsCat}{Eine \gloFt{Folgenoperation}}.}
}

\newVerweis               {\MtsSubseq}{\glstext}{MtsSubseq}
\newglossaryentry          {MtsSubseq}{
	name  ={\ensuremath{\RawMtsSubseq}},
	sort  ={= 6 1 1},
	type  ={symbols},
	symbol={\ensuremath{(\vec{c} \RawMtsSubseq \vec{d}) \RawMtsDefEquiv ((\vec{c} \RawMtsSubseqEq \vec{d}) \RawMtsAnd (\vec{c} \RawMtsEqN \vec{d}))}},
	user6 ={$\vec{c}$ \optFt{ist eine} \manFt{echte Teilfolge} \optFt{von} $\vec{d}$},
	see   ={MtsAnd,MtsDefEquiv,MtsEqN,MtsSubseqEq,Folgenrelation,echteTeilfolge},
	description={\baueBeschreibung{MtsSubseq}{Eine \gloFt{Folgenrelation}}.}
}

\newVerweis               {\MtsSubseqEq}{\glstext}{MtsSubseqEq}
\newglossaryentry          {MtsSubseqEq}{
	name  ={\ensuremath{\RawMtsSubseqEq}},
	sort  ={= 6 1 2},
	type  ={symbols},
	symbol={\ensuremath{
			(\vec{c} \RawMtsSubseqEq \vec{d}) \RawMtsDefEquiv (
				(\RawMtsExists \vec{a} :         (\vec{a} \RawMtsCat \vec{c})                   \RawMtsEq \vec{d}) \RawMtsOr
				(\RawMtsExists \vec{a},\vec{b} : (\vec{a} \RawMtsCat \vec{c} \RawMtsCat \vec{b}) \RawMtsEq \vec{d})
			)
		}
	},
	user6 ={$\vec{c}$ \optFt{ist eine} \manFt{Teilfolge von} $\vec{d}$},
	see   ={MtsDefEquiv,MtsEq,MtsExists,Folge,Folgenrelation,Teilfolge},
	description={
		\baueBeschreibung{MtsSubseqEq}{Eine \gloFt{Folgenrelation}}%
		\footnote{In letzterem Fall muss $\vec{c}$ eine endliche \gloFt{Folge} sein.}.
	}
}

\newVerweis               {\MtsSupseq}{\glstext}{MtsSupseq}
\newglossaryentry          {MtsSupseq}{
	name  ={\ensuremath{\RawMtsSupseq}},
	sort  ={= 6 1 3},
	type  ={symbols},
	symbol={\ensuremath{(\vec{c} \RawMtsSupseq \vec{d}) \RawMtsDefEquiv (\vec{d} \RawMtsSubseq \vec{c})}},
	user6 ={$\vec{c}$ \optFt{ist eine} \manFt{echte Oberfolge von} $\vec{d}$},
	see   ={MtsDefEquiv,MtsSubseq,Folgenrelation,echteOberfolge,Umkehrrelation},
	description={
		\baueBeschreibung{MtsSupseq}{Eine \gloFt{Folgenrelation}}.
		\\Die \gloFt{Umkehrrelation} von $\RawMtsSubseq$.
	}
}

\newVerweis               {\MtsSupseqEq}{\glstext}{MtsSupseqEq}
\newglossaryentry          {MtsSupseqEq}{
	name  ={\ensuremath{\RawMtsSupseqEq}},
	sort  ={= 6 1 4},
	type  ={symbols},
	symbol={\ensuremath{(\vec{c} \RawMtsSupseqEq \vec{d}) \RawMtsDefEquiv (\vec{d} \RawMtsSubseqEq \vec{c})}},
	user6 ={$\vec{c}$ \optFt{ist eine} \manFt{Oberfolge von} $\vec{d}$},
	see   ={MtsDefEquiv,MtsSubseqEq,Folgenrelation,Oberfolge,Umkehrrelation},
	description={
		\baueBeschreibung{MtsSupseqEq}{Eine \gloFt{Folgenrelation}}.
		\\Die \gloFt{Umkehrrelation} von $\RawMtsSubseqEq$.
	}
}

\newVerweis               {\MtsSubseqN}{\glstext}{MtsSubseqN}
\newglossaryentry          {MtsSubseqN}{
	name  ={\ensuremath{\RawMtsSubseqN}},
	sort  ={= 6 2 1},
	type  ={symbols},
	symbol={\ensuremath{(\vec{c} \RawMtsSubseqN \vec{d}) \RawMtsDefEquiv \RawMtsNot (\vec{c} \RawMtsSubseq \vec{d})}},
	user6 ={$\vec{c}$ \optFt{ist} \manFt{keine echte Teilfolge von} $\vec{d}$},
	see   ={MtsNot,MtsDefEquiv,MtsSubseq,Folgenrelation,Negation,echteTeilfolge},
	description={
		\baueBeschreibung{MtsSubseqN}{Eine \gloFt{Folgenrelation}}.
		\\Die \gloFt{Negation} von $\RawMtsSubseq$.
	}
}

\newVerweis               {\MtsSubseqEqN}{\glstext}{MtsSubseqEqN}
\newglossaryentry          {MtsSubseqEqN}{
	name  ={\ensuremath{\RawMtsSubseqEqN}},
	sort  ={= 6 2 2},
	type  ={symbols},
	symbol={\ensuremath{(\vec{c} \RawMtsSubseqEqN \vec{d}) \RawMtsDefEquiv \RawMtsNot (\vec{c} \RawMtsSubseqEq \vec{d})}},
	user6 ={$\vec{c}$ \optFt{ist} \manFt{keine Teilfolge von} $\vec{d}$},
	see   ={MtsNot,MtsDefEquiv,MtsSubseqEq,Folgenrelation,Negation,Teilfolge},
	description={
		\baueBeschreibung{MtsSubseqEqN}{Eine \gloFt{Folgenrelation}}.
		\\Die \gloFt{Negation} von $\RawMtsSubseqEq$.
	}
}

\newVerweis               {\MtsSupseqN}{\glstext}{MtsSupseqN}
\newglossaryentry          {MtsSupseqN}{
	name  ={\ensuremath{\RawMtsSupseqN}},
	sort  ={= 6 2 3},
	type  ={symbols},
	symbol={\ensuremath{(\vec{c} \RawMtsSupseqN \vec{d}) \RawMtsDefEquiv \RawMtsNot (\vec{d} \RawMtsSubseq \vec{c})}},
	user6 ={$\vec{c}$ \optFt{ist} \manFt{keine echte Oberfolge von} $\vec{d}$},
	see   ={MtsNot,MtsDefEquiv,MtsSubseq,Folgenrelation,Negation,echteOberfolge,Umkehrrelation},
	description={
		\baueBeschreibung{MtsSupseqN}{Eine \gloFt{Folgenrelation}}.
		\\Die \gloFt{Negation} der \gloFt{Umkehrrelation} und gleichzeitig die \gloFt{Umkehrrelation} der \gloFt{Negation} von $\RawMtsSubseq$.
	}
}

\newVerweis               {\MtsSupseqEqN}{\glstext}{MtsSupseqEqN}
\newglossaryentry          {MtsSupseqEqN}{
	name  ={\ensuremath{\RawMtsSupseqEqN}},
	sort  ={= 6 2 4},
	type  ={symbols},
	symbol={\ensuremath{(\vec{c} \RawMtsSupseqEqN \vec{d}) \RawMtsDefEquiv \RawMtsNot (\vec{d} \RawMtsSubseqEq \vec{c})}},
	user6 ={$\vec{c}$ \optFt{ist} \manFt{keine Oberfolge von} $\vec{d}$},
	see   ={MtsNot,MtsDefEquiv,MtsSubseqEq,Folgenrelation,Negation,Oberfolge,Umkehrrelation},
	description={
		\baueBeschreibung{MtsSupseqEqN}{Eine \gloFt{Folgenrelation}}.
		\\Die \gloFt{Negation} der \gloFt{Umkehrrelation} und gleichzeitig die \gloFt{Umkehrrelation} der \gloFt{Negation} von $\RawMtsSubseqEq$.
	}
}

% ==============================================================================
% \Ojk* - Ausgabe als Symbol und Eintrag und Verweis ins Symbolverzeichnis
\newglossaryentry{Glo-Objektsymbole}{
	name  ={\gloFt{Junktoren}},% ===============================================
	sort  ={= 7 0 0},
	type  ={symbols},
	see   ={logischeAussage,Junktor,Objektkonstante,Objektoperation,Objektrelation,aussagenlogischeOperation,aussagenlogischeRelation},
	description={
		\footnote{In diesem Dokument \gloFt{aussagenlogische Konstante}, \gloFt{Relationen} und \gloFt{Operationen}, \textdh\ \gloFt{Objektkonstante}, \gloFt{-relationen} und \gloFt{-operationen}.}
		Im Folgenden seien $A$ und $B$ beliebige \gloFt{logische Aussagen}.
	}
}

% TODO ### Definierende Formel in 'symbol' eintragen --- Symbol am Seitenrand ausgeben: mit \baueBeschreibung oder \SymbolAmRand

\newVerweis               {\OjkFalse}{\glstext}{OjkFalse}
\newglossaryentry          {OjkFalse}{
	name  ={\ensuremath{\RawOjkFalse}},
	sort  ={= 7 0 1},
	type  ={symbols},
	symbol={\ensuremath{Wert(\RawOjkFalse) \RawMtsDefEq \RawTxtFalse}},
	see   ={MtsFalse,TxtFalse,Junktor,aussagenlogischeKonstante,stellig,Wahrheitswert},
	description={\SymbolAmRand{OjkFalse}%
		Ein $0$-\gloFt{stelliger} \gls{Junktor}, \textdh\ eine \gloFt{aussagenlogische Konstante} mit dem \gloFt{Wahrheitswert} \RawTxtFalse.
	}
}

\newVerweis               {\OjkTrue}{\glstext}{OjkTrue}
\newglossaryentry          {OjkTrue}{
	name  ={\ensuremath{\RawOjkTrue}},
	sort  ={= 7 0 2},
	type  ={symbols},
	symbol={\ensuremath{Wert(\RawOjkTrue) \RawMtsDefEq \RawTxtTrue }},
	see   ={MtsTrue,TxtTrue,Junktor,aussagenlogischeKonstante,stellig,Wahrheitswert},
	description={\SymbolAmRand{OjkFalse}%
		Ein $0$-\gloFt{stelliger} \gloFt{Junktor}, \textdh\ eine \gloFt{aussagenlogische Konstante} mit dem \gloFt{Wahrheitswert} \RawTxtTrue.
	}
}

\newVerweis               {\OjkNot}{\glstext}{OjkNot}
\newglossaryentry          {OjkNot}{
	name  ={\ensuremath{\RawOjkNot}},
	sort  ={= 7 1 1},
	type  ={symbols},
	symbol={\ensuremath{\RawOjkNot A \RawMtsDefEq \text{\defFt{nicht}} A}},
	see   ={MtsNot,unaererJunktor},
	description={\baueBeschreibung[]{OjkNot}{Ein \gloFt{unärer Junktor}}.}
}

\newVerweis               {\OjkAnd}{\glstext}{OjkAnd}
\newglossaryentry          {OjkAnd}{
	name  ={\ensuremath{\RawOjkAnd}},
	sort  ={= 7 1 2},
	type  ={symbols},
	symbol={\ensuremath{}},
	user6 ={},
	see   ={MtsAnd,OjkNand,binaererJunktor},
	description={
		Ein \gloFt{binärer Junktor}:~ $A$ \defFt{und} $B$.
	}
}

% TODO ### direkte glossarinterne Verweise abbauen

\newVerweis               {\OjkOr}{\glstext}{OjkOr}
\newglossaryentry          {OjkOr}{
	name  ={\ensuremath{\RawOjkOr}},
	sort  ={= 7 1 3},
	type  ={symbols},
	symbol={\ensuremath{}},
	user6 ={},
	see   ={MtsOr,OjkNor,OjkXor},
	description={
		Ein \binaererJunktor:~ $A$ \defFt{oder} $B$.
	}
}

\newVerweis               {\OjkImp}{\glstext}{OjkImp}
\newglossaryentry          {OjkImp}{
	name  ={\ensuremath{\RawOjkImp}},
	sort  ={= 7 2 1},
	type  ={symbols},
	symbol={\ensuremath{}},
	user6 ={},
	see   ={MtsImp},
	description={
		Ein \binaererJunktor:~ wenn $A$ \defFt{dann} $B$.
	}
}

\newVerweis               {\OjkRep}{\glstext}{OjkRep}
\newglossaryentry          {OjkRep}{
	name  ={\ensuremath{\RawOjkRep}},
	sort  ={= 7 2 2},
	type  ={symbols},
	symbol={\ensuremath{}},
	user6 ={},
	see   ={MtsRep},
	description={
		Ein \binaererJunktor:~ $A$ \defFt{wenn} $B$.
	}
}

\newVerweis               {\OjkEquiv}{\glstext}{OjkEquiv}
\newglossaryentry          {OjkEquiv}{
	name  ={\ensuremath{\RawOjkEquiv}},
	sort  ={= 7 2 3},
	type  ={symbols},
	symbol={\ensuremath{}},
	user6 ={},
	see   ={MtsEquiv},
	description={
		Ein \binaererJunktor:~ $A$ genau \defFt{dann wenn} $B$.
	}
}

\newVerweis               {\OjkNand}{\glstext}{OjkNand}
\newglossaryentry          {OjkNand}{
	name  ={\ensuremath{\RawOjkNand}},
	sort  ={= 7 3 1},
	type  ={symbols},
	symbol={\ensuremath{}},
	user6 ={},
	see   ={OjkAnd},
	description={
		Ein \binaererJunktor:~ \defFt{nicht} ($A$ \defFt{und} $B$)\alternativOjkNand.
	}
}
\newcommand*{\alternativOjkNand}{\alternativi{sowohl~~~als auch}}

\newVerweis               {\OjkNor}{\glstext}{OjkNor}
\newglossaryentry          {OjkNor}{
	name  ={\ensuremath{\RawOjkNor}},
	sort  ={= 7 3 2},
	type  ={symbols},
	symbol={\ensuremath{}},
	user6 ={},
	see   ={OjkOr,OjkXor},
	description={
		Ein \binaererJunktor:~ \defFt{nicht} ($A$ \defFt{oder} $B$)\alternativOjkNor.
	}
}
\newcommand*{\alternativOjkNor}{\alternativi{weder~~~noch}}

\newVerweis               {\OjkXor}{\glstext}{OjkXor}
\newglossaryentry          {OjkXor}{
	name  ={\ensuremath{\RawOjkXor}},
	sort  ={= 7 3 3},
	type  ={symbols},
	symbol={\ensuremath{}},
	user6 ={},
	see   ={OjkOr,OjkNor},
	description={
		Ein \binaererJunktor:~ \defFt{entweder} $A$ \defFt{oder} $B$.
	}
}

\newVerweis               {\OjkEq}{\glstext}{OjkEq}
\newglossaryentry          {OjkEq}{
	name  ={\ensuremath{\RawOjkEq}},
	sort  ={= 7 4 1},
	type  ={symbols},
	symbol={\ensuremath{}},
	user6 ={},
	see   ={MtsEq},
	description={
		Logische \Gleichheit:~ $A$ ist \defFt{gleich} $B$.
	}
}

\newVerweis               {\OjkEqN}{\glstext}{OjkEqN}
\newglossaryentry          {OjkEqN}{
	name  ={\ensuremath{\RawOjkEqN}},
	sort  ={= 7 4 2},
	type  ={symbols},
	symbol={\ensuremath{}},
	user6 ={},
	see   ={MtsEqN},
	description={
		Logische \Ungleichheit:~ $A$ ist \defFt{ungleich} $B$.
	}
}

% ==============================================================================
% \Mts* - Ausgabe als Symbol und Eintrag und Verweis ins Symbolverzeichnis
% \Ojk* - Ausgabe als Symbol und Eintrag und Verweis ins Symbolverzeichnis
\newglossaryentry{Glo-Quantoren}{
	name     ={\gloFt{Quantoren}},% ===============================================
	sort  ={= 8 0 0},
	type  ={symbols},
	symbol={\ensuremath{}},
	user6 ={},
	see   ={Quantor},
	description={
		$x$ steht jeweils für eine \metasprachlicheV\ \textbzw\ \logischeV\ \Variable\ und $A$ für eine \Aussage\ \textbzw\ \Formel.
	}
}

\newVerweis               {\MtsForall}{\glstext}{MtsForall}
\newglossaryentry          {MtsForall}{
	name  ={\ensuremath{\RawMtsForall}},
	sort  ={= 8 1 1},
	type  ={symbols},
	symbol={\ensuremath{}},
	user6 ={},
	see   ={OjkForall,Allquantor},
	description={
		Ein \metasprachlicherQuantor: \defFt{für alle} $x$ \defFt{gilt} $A$.
	}
}

\newVerweis               {\MtsExists}{\glstext}{MtsExists}
\newglossaryentry          {MtsExists}{
	name  ={\ensuremath{\RawMtsExists}},
	sort  ={= 8 1 2},
	type  ={symbols},
	symbol={\ensuremath{}},
	user6 ={},
	see   ={OjkExists,Existenzquantor},
	description={
		Ein \metasprachlicherQuantor: \defFt{es gibt ein} $x$ \defFt{so dass} $A$.
	}
}

\newVerweis               {\MtsExione}{\glstext}{MtsExione}
\newglossaryentry          {MtsExione}{
	name  ={\ensuremath{\RawMtsExione}},
	sort  ={= 8 1 3},
	type  ={symbols},
	symbol={\ensuremath{}},
	user6 ={},
	see   ={OjkExione,Existenzquantor},
	description={
		Ein \metasprachlicherQuantor: \defFt{es gibt genau ein} $x$ \defFt{so dass} $A$.
	}
}

\newVerweis               {\OjkForall}{\glstext}{OjkForall}
\newglossaryentry          {OjkForall}{
	name  ={\ensuremath{\RawOjkForall}},
	sort  ={= 8 2 1},
	type  ={symbols},
	symbol={\ensuremath{}},
	user6 ={},
	see   ={MtsForall,Allquantor},
	description={
		Ein \logischerQuantor: \defFt{für alle} $x$ \defFt{gilt} $A$.
	}
}

\newVerweis               {\OjkExists}{\glstext}{OjkExists}
\newglossaryentry          {OjkExists}{
	name  ={\ensuremath{\RawOjkExists}},
	sort  ={= 8 2 2},
	type  ={symbols},
	symbol={\ensuremath{}},
	user6 ={},
	see   ={MtsExists,Existenzquantor},
	description={
		Ein \logischerQuantor: \defFt{es gibt ein} $x$ \defFt{so dass} $A$.
	}
}

\newVerweis               {\OjkExione}{\glstext}{OjkExione}
\newglossaryentry          {OjkExione}{
	name  ={\ensuremath{\RawOjkExione}},
	sort  ={= 8 2 3},
	type  ={symbols},
	symbol={\ensuremath{}},
	user6 ={},
	see   ={MtsExione,Existenzquantor},
	description={
		Ein \logischerQuantor: \defFt{es gibt genau ein} $x$ \defFt{so dass} $A$.
	}
}

% ==============================================================================
% \sym* - Ausgabe als geklammertes Symbol und Eintrag ins Symbolverzeichnis
% \gls* - wie \sym*            und zusätzlich Verweis ins Symbolverzeichnis
% \tag* - Tag in einer Formel setzen      und Eintrag ins Symbolverzeichnis
% Verweise als geklammertes Symbol auf die Formel mit dem Tag:
%   \ref    {def:*} -->  \*
%   \eqref  {def:*} --> (\*)
%   \vreffor{def:*} --> (\*) auf Seite <n>
\newglossaryentry{Glo-Schlussregeln}{
	name     ={\gloFt{Schlussregeln}},% =========================================
	sort    ={= 9},
	type       ={symbols},
	description={}
}

\newcommand*    {\AR}{\ensuremath{\text{AR}}}
\newVerweis  {\glsAR}{\glstext }       {AR}
\newVerweis  {\symAR}{\glsuserv}       {AR}
\newcommand* {\tagAR}{\glsTag          {AR}}
\newglossaryentry{AR}{
	name      ={(\AR)},
	user5      ={\AR},
	sort    ={= 9 AR},
	user6      ={},% Dummy für \glsTag
	type       ={symbols},
	see        ={Anfangsregel,Schlussregel},
	description={
		Eine Schlussregel: Anfangsregel.
	}
}

\newcommand*    {\FS}{\ensuremath{\text{FS}}}
\newVerweis  {\glsFS}{\glstext }       {FS}
\newVerweis  {\symFS}{\glsuserv}       {FS}
\newcommand* {\tagFS}{\glsTag          {FS}}
\newglossaryentry{FS}{
	name      ={(\FS)},
	user5      ={\FS},
	sort    ={= 9 FS},
	user6      ={},% Dummy für \glsTag
	type       ={symbols},
	see        ={formalerSatz,Schlussregel},
	description={
		Eine Schlussregel: formalerSatz.
	}
}

\newcommand*    {\MR}{\ensuremath{\text{MR}}}
\newVerweis  {\glsMR}{\glstext }       {MR}
\newVerweis  {\symMR}{\glsuserv}       {MR}
\newcommand* {\tagMR}{\glsTag          {MR}}
\newglossaryentry{MR}{
	name      ={(\MR)},
	user5      ={\MR},
	sort    ={= 9 MR},
	user6      ={},% Dummy für \glsTag
	type       ={symbols},
	see        ={Monotonieregel,Schlussregel},
	description={
		Eine Schlussregel: Monotonieregel.
	}
}

\newcommand*    {\SR}{\ensuremath{\text{SR}}}
\newVerweis  {\glsSR}{\glstext }       {SR}
\newVerweis  {\symSR}{\glsuserv}       {SR}
\newcommand* {\tagSR}{\glsTag          {SR}}
\newglossaryentry{SR}{
	name      ={(\SR)},
	user5      ={\SR},
	sort    ={= 9 SR},
	user6      ={},% Dummy für \glsTag
	type       ={symbols},
	see        ={Schlussregel,Schnittregel},
	description={
		Eine Schlussregel: Schnittregel.
	}
}

\newcommand*    {\TR}{\ensuremath{\text{TR}}}
\newVerweis  {\glsTR}{\glstext }       {TR}
\newVerweis  {\symTR}{\glsuserv}       {TR}
\newcommand* {\tagTR}{\glsTag          {TR}}
\newglossaryentry{TR}{
	name      ={(\TR)},
	user5      ={\TR},
	sort    ={= 9 TR},
	user6      ={},% Dummy für \glsTag
	type       ={symbols},
	see        ={Abtrennungsregel,Schlussregel},
	description={
		Eine Schlussregel: Abtrennungsregel.
	}
}

\newcommand*    {\andB}{\ensuremath{\RawOjkAnd\text{B}}}
\newVerweis  {\glsandB}{\glstext }{andB}
\newVerweis  {\symandB}{\glsuserv}{andB}
\newcommand* {\tagandB}{\glsTag   {andB}}
\newglossaryentry{andB}{
	name      ={(\andB)},
	user5      ={\andB},
	user6      ={},% Dummy für \glsTag
	sort       ={= 9 1 1},
	type       ={symbols},
	see        ={OjkAnd,Schlussregel},
	description={
		Eine Schlussregel: Beseitigung von $\RawOjkAnd$.
	}
}

\newcommand*    {\andE}{\ensuremath{\RawOjkAnd\text{E}}}
\newVerweis  {\glsandE}{\glstext }        {andE}
\newVerweis  {\symandE}{\glsuserv}        {andE}
\newcommand* {\tagandE}{\glsTag           {andE}}
\newglossaryentry{andE}{
	name      ={(\andE)},
	user5      ={\andE},
	user6      ={},% Dummy für \glsTag
	sort       ={= 9 1 2},
	type       ={symbols},
	see        ={OjkAnd,Schlussregel},
	description={
		Eine Schlussregel: Einführung von $\RawOjkAnd$.
	}
}

\newcommand*    {\orB}{\ensuremath{\RawOjkOr\text{B}}}
\newVerweis  {\glsorB}{\glstext }        {orB}
\newVerweis  {\symorB}{\glsuserv}        {orB}
\newcommand* {\tagorB}{\glsTag           {orB}}
\newglossaryentry{orB}{
	name      ={(\orB)},
	user5      ={\orB},
	user6      ={},% Dummy für \glsTag
	sort       ={= 9 2 1},
	type       ={symbols},
	see        ={OjkOr,Schlussregel},
	description={
		Eine Schlussregel: Beseitigung von $\RawOjkOr$.
	}
}

\newcommand*    {\orE}{\ensuremath{\RawOjkOr\text{E}}}
\newVerweis  {\glsorE}{\glstext }        {orE}
\newVerweis  {\symorE}{\glsuserv}        {orE}
\newcommand* {\tagorE}{\glsTag           {orE}}
\newglossaryentry{orE}{
	name      ={(\orE)},
	user5      ={\orE},
	user6      ={},% Dummy für \glsTag
	sort       ={= 9 2 2},
	type       ={symbols},
	see        ={OjkOr,Schlussregel},
	description={
		Eine Schlussregel: Einführung von $\RawOjkOr$.
	}
}

\newcommand*    {\impB}{\ensuremath{\RawOjkImp\text{B}}}
\newVerweis  {\glsimpB}{\glstext }        {impB}
\newVerweis  {\symimpB}{\glsuserv}        {impB}
\newcommand* {\tagimpB}{\glsTag           {impB}}
\newglossaryentry{impB}{
	name      ={(\impB)},
	user5      ={\impB},
	user6      ={},% Dummy für \glsTag
	sort       ={= 9 3 1},
	type       ={symbols},
	see        ={OjkImp,Schlussregel},
	description={
		Eine Schlussregel: Beseitigung von $\RawOjkImp$.
	}
}

\newcommand*    {\impE}{\ensuremath{\RawOjkImp\text{E}}}
\newVerweis  {\glsimpE}{\glstext }        {impE}
\newVerweis  {\symimpE}{\glsuserv}        {impE}
\newcommand* {\tagimpE}{\glsTag           {impE}}
\newglossaryentry{impE}{
	name      ={(\impE)},
	user5      ={\impE},
	user6      ={},% Dummy für \glsTag
	sort       ={= 9 3 2},
	type       ={symbols},
	see        ={OjkImp,Schlussregel},
	description={
		Eine Schlussregel: Einführung von $\RawOjkImp$.
	}
}

\newcommand*    {\nota}{\ensuremath{\RawOjkNot\text{1}}}
\newVerweis  {\glsnota}{\glstext }        {nota}
\newVerweis  {\symnota}{\glsuserv}        {nota}
\newcommand* {\tagnota}{\glsTag           {nota}}
\newglossaryentry{nota}{
	name      ={(\nota)},
	user5      ={\nota},
	user6      ={},% Dummy für \glsTag
	sort       ={= 9 4 1},
	type       ={symbols},
	see        ={OjkNot,Schlussregel},
	description={
		Eine Schlussregel: Einführung/Beseitigung von $\RawOjkNot$ Teil 1.
	}
}

\newcommand*    {\notb}{\ensuremath{\RawOjkNot\text{2}}}
\newVerweis  {\glsnotb}{\glstext }        {notb}
\newVerweis  {\symnotb}{\glsuserv}        {notb}
\newcommand* {\tagnotb}{\glsTag           {notb}}
\newglossaryentry{notb}{
	name      ={(\notb)},
	user5      ={\notb},
	user6      ={},% Dummy für \glsTag
	sort       ={= 9 4 2},
	type       ={symbols},
	see        ={OjkNot,Schlussregel},
	description={
		Eine Schlussregel: Einführung/Beseitigung von $\RawOjkNot$ Teil 2.
	}
}

\newcommand*    {\notc}{\ensuremath{\RawOjkNot\text{3}}}
\newVerweis  {\glsnotc}{\glstext }        {notc}
\newVerweis  {\symnotc}{\glsuserv}        {notc}
\newcommand* {\tagnotc}{\glsTag           {notc}}
\newglossaryentry{notc}{
	name      ={(\notc)},
	user5      ={\notc},
	user6      ={},% Dummy für \glsTag
	sort       ={= 9 4 3},
	type       ={symbols},
	see        ={Beweis,Schlussregel},
	description={
		Eine Schlussregel: Beweistechnik „Indirekter Beweis“.
	}
}

\newcommand*    {\notd}{\ensuremath{\RawOjkNot\text{4}}}
\newVerweis  {\glsnotd}{\glstext }        {notd}
\newVerweis  {\symnotd}{\glsuserv}        {notd}
\newcommand* {\tagnotd}{\glsTag           {notd}}
\newglossaryentry{notd}{
	name      ={(\notd)},
	user5      ={\notd},
	user6      ={},% Dummy für \glsTag
	sort       ={= 9 4 4},
	type       ={symbols},
	see        ={Beweis,Schlussregel},
	description={
		Eine Schlussregel: Reductio ad absurdum (Indirekter Beweis).
	}
}

\newcommand*    {\eqB}{\ensuremath{\RawOjkEq\text{B}}}
\newVerweis  {\glseqB}{\glstext }        {eqB}
\newVerweis  {\symeqB}{\glsuserv}        {eqB}
\newcommand* {\tageqB}{\glsTag           {eqB}}
\newglossaryentry{eqB}{
	name      ={(\eqB)},
	user5      ={\eqB},
	user6      ={},% Dummy für \glsTag
	sort       ={= 9 5 1},
	type       ={symbols},
	see        ={OjkEq,Schlussregel},
	description={
		Eine Schlussregel: Beseitigung von $\RawOjkEq$.
	}
}

\newcommand*    {\eqE}{\ensuremath{\RawOjkEq\text{E}}}
\newVerweis  {\glseqE}{\glstext }        {eqE}
\newVerweis  {\symeqE}{\glsuserv}        {eqE}
\newcommand* {\tageqE}{\glsTag           {eqE}}
\newglossaryentry{eqE}{
	name      ={(\eqE)},
	user5      ={\eqE},
	user6      ={},% Dummy für \glsTag
	sort       ={= 9 5 2},
	type       ={symbols},
	see        ={OjkEq,Schlussregel},
	description={
		Eine Schlussregel: Einführung von $\RawOjkEq$.
	}
}

% ### Symbolverzeichnis und Index ##############################################
% Anmerkung:
%   Eigentlich gehören die weiteren aufgeführten Symbole alle zur Metasprache.
%   Solche, die zur Bildung von aussagen- und prädikatenlogischen Formeln
%   dienen, sind trotzdem mit 'Ojk' statt 'Mts' markiert.

% ==============================================================================
\newglossaryentry{Glo-TextSymbole}{
	name    ={Text-\gloFt{Symbole}},% ==========================================
	sort       ={A},
	type       ={symbols},
	see        ={Symbol},
	description={
		Die folgenden \gloFt{Symbole} sind alphabetisch geordnet und auch im Index aufgeführt.
		$\square$ dient zur Verdeutlichung, an welche Stelle die Indizes gehören.
	}
}

% ==============================================================================
% \StrMtsIdx* - Ausgabe als Text-Symbol und Eintrag und Verweis ins Symbolverzeichnis
% \Mts* - Ausgabe als Text-Symbol und Eintrag und Verweis ins Symbolverzeichnis
% Operationen mit Namen (Buchstaben) ===========================================

%TODO Wert definieren
\newcommand*{\Wert}{Wert}
\newcommand*{\Werten}{Werten}
\newcommand*             {\StrMtsValue}            {val}% Definitionsbereich [domain]
\newVerweis                 {\MtsValue}{\glstext}{MtsValue}
\newglossaryentry            {MtsValue}{
	text    ={\ensuremath{\RawMtsValue}},
	name    ={\ensuremath{\RawMtsValue} \addIdx[
		name={\ensuremath{\RawMtsValue}},
		sort={val}]                              {MtsValue}},
	sort    ={dom},%      \StrMtsValue
	type    ={symbols},
	symbol  ={\ensuremath{\RawMtsValue}},
	see     ={},
	description={
		Der \Wert einer \Formel, nachdem die \Variablen mit \Werten belegt wurden.
	}
}

\newcommand*             {\LtrMtsIdxEndlich}       {e}% nur endliche Elemente
\newVerweis                 {\MtsIdxEndlich} {\glstext}{MtsIdxEndlich}
\newglossaryentry            {MtsIdxEndlich}{
	text    ={\ensuremath{\RawMtsIdxEndlich}},
	name    ={\ensuremath{\square_{\RawMtsIdxEndlich}} \addIdx[
		name={\ensuremath{\RawMtsIdxEndlich}},
		sort={e Ind}]                                  {MtsIdxEndlich}},
	sort    ={e Ind},%       \LtrIdxEndlich Index
	type    ={symbols},
	symbol  ={},
	user6   ={},
	see     ={Menge},
	description={
		Eine \Operation\ mittels eines Index:
		\[
			X_{\RawMtsIdxEndlich} \RawMtsDefEq
			\begin{cases}
				\RawMengeDef{M  \RawMtsIn X}{|M|                   \quad \RawMtsIn \RawMtsINo}
				& \text{, für eine \Menge\ $X$ von \Mengen}     \\
				\RawMengeDef{R\;\RawMtsIn X}{|R_{\RawMtsIdxGraph}| \quad \RawMtsIn \RawMtsINo}
				& \text{, für eine \Menge\ $X$ von \Relationen} \\
				\RawMengeDef{F\;\RawMtsIn X}{\RawMtsLen(F)               \RawMtsIn \RawMtsINo}
				& \text{, für eine \Menge\ $X$ von \Folgen}
			\end{cases}
		\]
	}
}

\newcommand*             {\LtrMtsIdxGraph}         {g}% Graph von
\newVerweis                 {\MtsIdxGraph} {\glstext}{MtsIdxGraph}
\newglossaryentry            {MtsIdxGraph}{
	text    ={\ensuremath{\RawMtsIdxGraph}},
	name    ={\ensuremath{\square_{\RawMtsIdxGraph}} \addIdx[
		name={\ensuremath{\RawMtsIdxGraph}},
		sort={g Ind}]                                {MtsIdxGraph}},
	sort    ={g Ind},%    \LtrMtsIdxGraph Index
	type    ={symbols},
	symbol  ={},
	user6   ={},
	see     ={},
	description={
		Eine \Operation\ mittels eines Index:
		$X_{\RawMtsIdxGraph} \RawMtsDefEq \RawMtsGraph(X)$ für \Funktionen\ und \Relationen\ $X$.
	}
}

\newcommand*             {\LtrMtsIdxPolnisch}      {p}% in Polnischer Notation
\newVerweis                 {\MtsIdxPolnisch} {\glstext}{MtsIdxPolnisch}
\newglossaryentry            {MtsIdxPolnisch}{
	text    ={\ensuremath{\RawMtsIdxPolnisch}},
	name    ={\ensuremath{\square^{\RawMtsIdxPolnisch}} \addIdx[
		name={\ensuremath{\RawMtsIdxPolnisch}},
		sort={p Ind}]                                   {MtsIdxPolnisch}},
	sort    ={p Ind},%    \LtrMtsIdxPolnisch Index
	type    ={symbols},
	symbol  ={},
	user6   ={}, 
	see     ={Menge},
	description={
		Eine \Operation\ mittels eines Index: Für eine \Menge\ $L$ von \Formeln\ und eine \Formel\ $\alpha$ ist\\
		$L^{\RawMtsIdxPolnisch} \RawMtsDefEq \RawMengeDef{\alpha^{\RawMtsIdxPolnisch}}{\alpha \RawMtsIn L}$.
		mit $\alpha^{\RawMtsIdxPolnisch} \RawMtsDefEq ( \alpha$ umgewandelt in \PolnischeNotation ).
	}
}

\newcommand*             {\StrMtsDb}               {dom}% Definitionsbereich [domain]
\newVerweis                 {\MtsDb}{\glstext}{MtsDb}
\newglossaryentry            {MtsDb}{
	text    ={\ensuremath{\RawMtsDb}},
	name    ={\ensuremath{\RawMtsDb} \addIdx[
		name={\ensuremath{\RawMtsDb}},
		sort={dom}]                           {MtsDb}},
	sort    ={dom},%      \StrMtsDb
	type    ={symbols},
	symbol  ={},
	user6   ={},
	see     ={},
	description={
		Für eine \Funktion\ \FunktionDef{f}{A}{B} ist $\RawMtsDb(f) \RawMtsDefEq A$, der \Definitionsbereich\ von $f$.
	}
}

\newcommand*             {\LtrMtsFol}              {F}%  Folgenmenge
\newVerweis                 {\MtsFol}{\glstext}{MtsFol}
\newglossaryentry            {MtsFol}{
	text    ={\ensuremath{\RawMtsFol}},
	name    ={\ensuremath{\RawMtsFol} \addIdx[
		name={\ensuremath{\RawMtsFol}},
		sort={F}]                              {MtsFol}},
	sort    ={F},%        \LtrMtsFol
	type    ={symbols},
	symbol  ={},
	user6   ={},
	see     ={MtsFolf,Menge},
	description={
		$\RawMtsFol(M) \RawMtsDefEq \RawMengeDef{F}{F \text{ ist \Folge\ über } M}$.
	}
}

\newVerweis                 {\MtsFolf}{\glstext}{MtsFolf}
\newglossaryentry            {MtsFolf}{
	text    ={\ensuremath{\RawMtsFolf}},
	name    ={\ensuremath{\RawMtsFolf} \addIdx[
		name={\ensuremath{\RawMtsFolf}},
		sort={F e}]                             {MtsFolf}},
	sort    ={F e},%      \LtrMtsFol \LtrIdxEndlich
	type    ={symbols},
	symbol  ={},
	user6   ={},
	see     ={MtsFol,Folgenmenge,Menge},
	description={
		$\RawMtsFol(M) \RawMtsDefEq \RawMengeDef{F \RawMtsIn \RawMtsFol(M)}{\RawMtsLen(F) \RawMtsIn \RawMtsINo}$.
	}
}

\newcommand*             {\StrMtsGraph}            {graph}% Graph; Funktionen/Relationen
\newVerweis                 {\MtsGraph}{\glstext}{MtsGraph}
\newglossaryentry            {MtsGraph}{
	text    ={\ensuremath{\RawMtsGraph}},
	name    ={\ensuremath{\RawMtsGraph} \addIdx[
		name={\ensuremath{\RawMtsGraph}},
		sort={graph}]                            {MtsGraph}},
	sort    ={graph},%    \StrMtsGraph
	type    ={symbols},
	symbol  ={},
	user6   ={},
	see     ={Graph,Menge},
	description={
		Für eine \Relation\ $R = (G, A_1, \dots, A_n)$ ist $\RawMtsGraph(R) \RawMtsDefEq G$.\\
		Für eine \Funktion\ \FunktionDef{f}{A}{B} ist $\RawMtsGraph(f) \RawMtsDefEq \RawMengeDef{(a,f(a))}{a \RawMtsIn A}$.
	}
}

\newcommand*             {\StrMtsLen}              {len}% Länge [length] (Tupel)
\newVerweis                 {\MtsLen}{\glstext}{MtsLen}
\newglossaryentry            {MtsLen}{
	text    ={\ensuremath{\RawMtsLen}},
	name    ={\ensuremath{\RawMtsLen} \addIdx[
		name={\ensuremath{\RawMtsLen}},
		sort={len}]                            {MtsLen}},
	sort    ={len},%      \StrMtsLen
	type    ={symbols},
	symbol  ={},
	user6   ={},
	see     ={},
	description={
		$\RawMtsLen(\vec{a}) \RawMtsDefEq$ Anzahl der \Komponenten\ einer endlichen \Folge\, \textdh\ eines \Tupels\ $\vec{a}$
	}
}

\newcommand*             {\LtrMtsPot}              {P}%  Potenzmenge
\newVerweis                 {\MtsPot}{\glstext}{MtsPot}
\newglossaryentry            {MtsPot}{
	text    ={\ensuremath{\RawMtsPot}},
	name    ={\ensuremath{\RawMtsPot} \addIdx[
		name={\ensuremath{\RawMtsPot}},
		sort={P}]                              {MtsPot}},
	sort    ={P},%        \LtrMtsPot
	type    ={symbols},
	symbol  ={},
	user6   ={},
	see     ={MtsPotf,Menge},
	description={
		$\RawMtsPot(M) \RawMtsDefEq \RawMengeDef{N}{N \RawMtsSubsetEq M}$, die \Potenzmenge\ einer \Menge\ $M$.
	}
}

\newVerweis                 {\MtsPotf}{\glstext}{MtsPotf}
\newglossaryentry            {MtsPotf}{
	text    ={\ensuremath{\RawMtsPotf}},
	name    ={\ensuremath{\RawMtsPotf} \addIdx[
		name={\ensuremath{\RawMtsPotf}},
		sort={P e}]                             {MtsPotf}},
	sort    ={P e},%      \LtrMtsPot \LtrIdxEndlich
	type    ={symbols},
	symbol  ={},
	user6   ={},
	see     ={Menge},
	description={
		$\RawMtsPot(M) \RawMtsDefEq \RawMengeDef{N \RawMtsIn \RawMtsPot(M)}{|N| \RawMtsIn \RawMtsINo}$.
	}
}

\newcommand*             {\StrMtsQb}               {src}% Quellbereich [source]
\newVerweis                 {\MtsQb}{\glstext}{MtsQb}
\newglossaryentry            {MtsQb}{
	text    ={\ensuremath{\RawMtsQb}},
	name    ={\ensuremath{\RawMtsQb} \addIdx[
		name={\ensuremath{\RawMtsQb}},
		sort={src}]                           {MtsQb}},
	sort    ={src},%      \StrMtsQb
	type    ={symbols},
	symbol  ={},
	user6   ={},
	see     ={Menge},
	description={
		Für eine \Funktion\ \FunktionDef{f}{A}{B} ist $\RawMtsQb(f) \RawMtsDefEq \RawMengeDef{a \in A}{f(a) \text{ existiert}}$ der \Quellbereich\ von $f$.
	}
}

\newcommand*             {\LtrMtsRel}              {R}% Menge der Relationen
\newVerweis                 {\MtsRel}{\glstext}{MtsRel}
\newglossaryentry            {MtsRel}{
	text    ={\ensuremath{\RawMtsRel}},
	name    ={\ensuremath{\RawMtsRel} \addIdx[
		name={\ensuremath{\RawMtsRel}},
		sort={R}]                              {MtsRel}},
	sort    ={R},%        \LtrMtsRel
	type    ={symbols},
	symbol  ={},
	user6   ={},
	see     ={MtsRelf,Relation},
	description={
		Für eine \Menge\ $M$ ist RAWMtsRel\ RAWMtsDefEq\ die \Menge\ der \binaeren\ \Relationen\ in $M$.
	}
}

\newVerweis                 {\MtsRelf}{\glstext}{MtsRelf}%
\newglossaryentry            {MtsRelf}{
	text    ={\ensuremath{\RawMtsRelf}},
	name    ={\ensuremath{\RawMtsRelf} \addIdx[
		name={\ensuremath{\RawMtsRelf}},
		sort={R e}]                             {MtsRelf}},
	sort    ={R e},%      \LtrMtsRel \LtrIdxEndlich
	type    ={symbols},
	symbol  ={},
	user6   ={},
	see     ={Menge},
	description={
		$\RawMtsRelf(M) \RawMtsDefEq \RawMengeDef{R \RawMtsIn \RawMtsRel(M)}{|R_{\RawMtsIdxEndlich}| \RawMtsIn \RawMtsINo}$
	}
}

\newcommand*             {\StrMtsSet}              {set}% Komponentenmenge (Tupel/Folge)
\newVerweis                 {\MtsSet}{\glstext}{MtsSet}
\newglossaryentry            {MtsSet}{
	text    ={\ensuremath{\RawMtsSet}},
	name    ={\ensuremath{\RawMtsSet} \addIdx[
		name={\ensuremath{\RawMtsSet} (Menge)},
		sort={Set}]                            {MtsSet}},
	sort    ={Set},%      \StrMtsSet
	type    ={symbols},
	symbol  ={},
	user6   ={},
	see     ={Folge,Komponentenmenge,Menge,Tupel},
	description={
		$\RawMtsSet(\vec{a}) \RawMtsDefEq \RawMengeDef{a}{a \RawMtsSeqIn \vec{a}}$.
	}
}

\newcommand*             {\StrMtsStel}             {stel}% [Stel]ligkeit Funktionen/Relationen
\newVerweis                 {\MtsStelF}{\glstext}{MtsStelF}
\newglossaryentry            {MtsStelF}{
	text    ={\ensuremath{\RawMtsStelF}},
	name    ={\ensuremath{\RawMtsStelF} \addIdx[
		name={\ensuremath{\RawMtsStelF}},
		sort={stel f}]                           {MtsStelF}},
	sort    ={stel f},%   \StrMtsStel f
	type    ={symbols},
	symbol  ={},
	user6   ={},
	see     ={Funktion,Stelligkeit},
	description={
		$\RawMtsStelF(f) \RawMtsDefEq n$ für $\FunktionDef{f}{A_1 \RawMtsTimes \dots \RawMtsTimes A_n}{B}$.
	}
}

\newVerweis                 {\MtsStelR}{\glstext}{MtsStelR}
\newglossaryentry            {MtsStelR}{
	text    ={\ensuremath{\RawMtsStelR}},
	name    ={\ensuremath{\RawMtsStelR} \addIdx[
		name={\ensuremath{\RawMtsStelR}},
		sort={stel r}]                           {MtsStelR}},
	sort    ={stel r},%   \StrMtsStel r
	type    ={symbols},
	symbol  ={},
	user6   ={},
	see     ={Relation,Stelligkeit},
	description={
		$\RawMtsStelR(R) \RawMtsDefEq n$ für $R \RawMtsSubsetEq A_1 \RawMtsTimes \dots \RawMtsTimes A_n$.
	}
}

\newcommand*             {\StrMtsTraeger}          {car}% Trägermenge [carrier] (Relation)
\newVerweis                 {\MtsTraeger}{\glstext}{MtsTraeger}
\newglossaryentry            {MtsTraeger}{
	text    ={\ensuremath{\RawMtsTraeger}},
	name    ={\ensuremath{\RawMtsTraeger} \addIdx[
		name={\ensuremath{\RawMtsTraeger}},
		sort={car}]                                {MtsTraeger}},
	sort    ={car},%      \StrMtsTraeger
	type    ={symbols},
	symbol  ={},
	user6   ={},
	see     ={Traegermenge},
	description={
		Für eine \Relation%
		\footnote{%
			\Funktionen\ sind spezielle \Relationen.
			Für eine \Funktion\ $\FunktionDef{f}{A_1 \RawMtsTimes \dots \RawMtsTimes A_n}{B}$ gilt demnach:
			\\$\RawMtsTraeger(f) \RawMtsDefEq A_1 \RawMtsTimes \dots \RawMtsTimes A_n \RawMtsTimes B$;
			\quad $\RawMtsTraeger_i(f) \RawMtsDefEq A_i$ für $1 \le i \le n$;
			\quad $\RawMtsTraeger_{n+1}(f) \RawMtsDefEq B$
		}
		$R = (G, A_1, \dots, A_n)$ ist
		$\RawMtsTraeger(R) \RawMtsDefEq A_1 \RawMtsTimes \dots \RawMtsTimes A_n$ und
		\ifmarginparFlg\newline\else\fi
		$\RawMtsTraeger_i(R) \RawMtsDefEq A_i$ für $1 \le i \le n$.
	}
}

\newcommand*             {\LtrMtsUniversum}        {U}%   Diskursuniversum [Universe of Discourse]
\newVerweis                 {\MtsUniversum}{\glstext}{MtsUniversum}
\newglossaryentry            {MtsUniversum}{
	text    ={\ensuremath{\RawMtsUniversum}},
	name    ={\ensuremath{\RawMtsUniversum} \addIdx[
		name={\ensuremath{\RawMtsUniversum}},
		sort={U}]                                    {MtsUniversum}},
	sort    ={U},%        \LtrMtsUniversum
	type    ={symbols},
	symbol  ={},
	user6   ={},
	see     ={},
	description={
		RAWMtsUniversum\ ist das \Diskursuniversum.
	}
}

\newcommand*             {\StrMtsWb}               {ran}% Wertebereich [range]
\newVerweis                 {\MtsWb}{\glstext}{MtsWb}
\newglossaryentry            {MtsWb}{
	text    ={\ensuremath{\RawMtsWb}},
	name    ={\ensuremath{\RawMtsWb} \addIdx[
		name={\ensuremath{\RawMtsWb}},
		sort={ran}]                           {MtsWb}},
	sort    ={ran},%      \StrMtsWb
	type    ={symbols},
	symbol  ={},
	user6   ={},
	see     ={Menge},
	description={
		Für eine \Funktion\ \FunktionDef{f}{A}{B} ist $\RawMtsWb(f) \RawMtsDefEq \RawMengeDef{f(a)}{a \in A}$ der \Wertebereich\ von $f$.
	}
}

\newcommand*             {\StrMtsZb}               {tar}% Zielbereich [target]
\newVerweis                 {\MtsZb}{\glstext}{MtsZb}
\newglossaryentry            {MtsZb}{
	text    ={\ensuremath{\RawMtsZb}},
	name    ={\ensuremath{\RawMtsZb} \addIdx[
		name={\ensuremath{\RawMtsZb}},
		sort={tar}]                           {MtsZb}},
	sort    ={tar},%      \StrMtsZb
	type    ={symbols},
	symbol  ={},
	user6   ={},
	see     ={},
	description={
		Für eine \Funktion\ \FunktionDef{f}{A}{B} ist $\RawMtsZb(f) \RawMtsDefEq B$ der \Zielbereich\ von $f$.
	}
}

% ==============================================================================
% \Mts* - Ausgabe als Text-Symbol und Eintrag und Verweis ins Symbolverzeichnis
% Mengen und Elemente ==========================================================

\newcommand*             {\LtrMtsAxiom}            {X}%        A[x]iom
\newVerweis                 {\MtsAxiom}{\glstext}{MtsAxiom}
\newglossaryentry            {MtsAxiom}{
	text    ={\ensuremath{\RawMtsAxiom}},
	name    ={\ensuremath{\RawMtsAxiom} \addIdx[
		name={\ensuremath{\RawMtsAxiom} (Element)},
		sort={X Ele}]                            {MtsAxiom}},
	sort    ={X Ele},%    \LtrMtsAxiom   Element
	type    ={symbols},
	symbol  ={},
	user6   ={},
	see     ={},
	description={
		Ein \Axiom.
	}
}

\newVerweis                 {\MtsAxiomSet}{\glstext}{MtsAxiomSet}
\newglossaryentry            {MtsAxiomSet}{
	text    ={\ensuremath{\RawMtsAxiomSet}},
	name    ={\ensuremath{\RawMtsAxiomSet} \addIdx[
		name={\ensuremath{\RawMtsAxiomSet} (Menge)},
		sort={X Men}]                               {MtsAxiomSet}},
	sort    ={X Men},%    \LtrMtsAxiom      Menge
	type    ={symbols},
	symbol  ={},
	user6   ={},
	see     ={},
	description={
		Eine \Menge\ von \Axiomen.
	}
}

\newcommand*             {\LtrMtsBeweisschritt}    {b}%              Beweisschritt
\newVerweis                 {\MtsBeweisschritt}{\glstext}{MtsBeweisschritt}
\newglossaryentry            {MtsBeweisschritt}{
	text    ={\ensuremath{\RawMtsBeweisschritt}},
	name    ={\ensuremath{\RawMtsBeweisschritt} \addIdx[
		name={\ensuremath{\RawMtsBeweisschritt} (Element)},
		sort={b Ele}]                                    {MtsBeweisschritt}},
	sort    ={b Ele},%    \LtrMtsBeweisschritt   Element
	type    ={symbols},
	symbol  ={},
	user6   ={},
	see     ={},
	description={
		Ein \Beweisschritt.
	}
}

\newVerweis                 {\MtsBeweisschrittTup}{\glstext}{MtsBeweisschrittTup}
\newglossaryentry            {MtsBeweisschrittTup}{
	text    ={\ensuremath{\RawMtsBeweisschrittTup}},
	name    ={\ensuremath{\RawMtsBeweisschrittTup} \addIdx[
		name={\ensuremath{\RawMtsBeweisschrittTup} (Tupel)},
		sort={b Tup}]                                       {MtsBeweisschrittTup}},
	sort    ={b Tup},%    \LtrMtsBeweisschritt      Tupel
	type    ={symbols},
	symbol  ={},
	user6   ={},
	see     ={},
	description={
		Ein \Tupel\ von \Beweisschritten.
	}
}

\newcommand*             {\LtrMtsBeweisschrittSet} {B}% Menge der       Beweisschritte
\newVerweis                 {\MtsBeweisschrittSet}{\glstext}{MtsBeweisschrittSet}
\newglossaryentry            {MtsBeweisschrittSet}{
	text    ={\ensuremath{\RawMtsBeweisschrittSet}},
	name    ={\ensuremath{\RawMtsBeweisschrittSet} \addIdx[
		name={\ensuremath{\RawMtsBeweisschrittSet} (Menge)},
		sort={B Men}]                                       {MtsBeweisschrittSet}},
	sort    ={B Men},%    \LtrMtsBeweisschrittSet   Menge
	type    ={symbols},
	symbol  ={},
	user6   ={},
	see     ={},
	description={
		Eine \Menge\ von \Beweisschritten.
	}
}

\newcommand*             {\LtrMtsErgebnis}         {e}% result; Ergebnis
\newVerweis                 {\MtsErgebnis}{\glstext}{MtsErgebnis}
\newglossaryentry            {MtsErgebnis}{
	text    ={\ensuremath{\RawMtsErgebnis}},
	name    ={\ensuremath{\RawMtsErgebnis} \addIdx[
		name={\ensuremath{\RawMtsErgebnis} (Element)},
		sort={r Ele}]                               {MtsErgebnis}},
	sort    ={r Ele},%    \LtrMtsErgebnis   Element
	type    ={symbols},
	symbol  ={},
	user6   ={},
	see     ={},
	description={
		Ein \Ergebnis.
	}
}

\newcommand*             {\LtrMtsErgebnisSet}      {E}% resultset; Ergebnismeng
\newVerweis                 {\MtsErgebnisSet}{\glstext}{MtsErgebnisSet}
\newglossaryentry            {MtsErgebnisSet}{
	text    ={\ensuremath{\RawMtsErgebnisSet}},
	name    ={\ensuremath{\RawMtsErgebnisSet} \addIdx[
		name={\ensuremath{\RawMtsErgebnisSet} (Menge)},
		sort={R Men}]                                  {MtsErgebnisSet}},
	sort    ={R Men},%    \LtrMtsErgebnisSet   Menge
	type    ={symbols},
	symbol  ={},
	user6   ={},
	see     ={},
	description={
		Eine \Menge\ von \Ergebnissen.
	}
}

\newVerweis                 {\MtsErgebnisRel}{\glstext}{MtsErgebnisRel}
\newglossaryentry            {MtsErgebnisRel}{
	text    ={\ensuremath{\RawMtsErgebnisRel}},
	name    ={\ensuremath{\RawMtsErgebnisRel} \addIdx[
		name={\ensuremath{\RawMtsErgebnisRel} (Relation)},
		sort={R Rel}]                                  {MtsErgebnisRel}},
	sort    ={R Rel},%    \LtrMtsErgebnisSet   Relation
	type    ={symbols},
	symbol  ={},
	user6   ={},
	see     ={},
	description={
		Eine \Relation\ (aufgefasst als \Menge) von \Ergebnissen.
	}
}

\newcommand*             {\LtrMtsErsetzung}        {E}% Substitution; Ersetzung
\newVerweis                 {\MtsErsetzung}{\glstext}{MtsErsetzung}
\newglossaryentry            {MtsErsetzung}{
	text    ={\ensuremath{\RawMtsErsetzung}},
	name    ={\ensuremath{\RawMtsErsetzung} \addIdx[
		name={\ensuremath{\RawMtsErsetzung} (Element)},
		sort={E Ele}]                                {MtsErsetzung}},
	sort    ={E Ele},%    \LtrMtsErsetzung   Element
	type    ={symbols},
	symbol  ={},
	user6   ={},
	see     ={MtsErsetzungSet},
	description={
		Eine \Ersetzung.
	}
}

\newVerweis                 {\MtsErsetzungSet}{\glstext}{MtsErsetzungSet}
\newglossaryentry            {MtsErsetzungSet}{
	text    ={\ensuremath{\RawMtsErsetzungSet}},
	name    ={\ensuremath{\RawMtsErsetzungSet} \addIdx[
		name={\ensuremath{\RawMtsErsetzungSet} (Menge)},
		sort={E Men}]                                   {MtsErsetzungSet}},
	sort    ={E Men},%    \LtrMtsErsetzung      Menge
	type    ={symbols},
	symbol  ={},
	user6   ={},
	see     ={MtsErsetzung},
	description={
		Eine \Menge\ von \Ersetzungen.
	}
}

\newcommand*             {\LtrMtsKonklusion}       {k}% eine     Konklusion
\newVerweis                 {\MtsKonklusion}{\glstext}{MtsKonklusion}
\newglossaryentry            {MtsKonklusion}{
	text    ={\ensuremath{\RawMtsKonklusion}},
	name    ={\ensuremath{\RawMtsKonklusion} \addIdx[
		name={\ensuremath{\RawMtsKonklusion} (Element)},
		sort={k Ele}]                                {MtsKonklusion}},
	sort    ={k Ele},%    \LtrMtsKonklusion   Element
	type    ={symbols},
	symbol  ={},
	user6   ={},
	see     ={},
	description={
		Eine \Konklusion.
	}
}

\newcommand*             {\LtrMtsKonklusionSet}    {K}% Menge von   Konklusionen
\newVerweis                 {\MtsKonklusionSet}{\glstext}{MtsKonklusionSet}
\newglossaryentry            {MtsKonklusionSet}{
	text    ={\ensuremath{\RawMtsKonklusionSet}},
	name    ={\ensuremath{\RawMtsKonklusionSet} \addIdx[
		name={\ensuremath{\RawMtsKonklusionSet} (Menge)},
		sort={K Men}]                                   {MtsKonklusionSet}},
	sort    ={K Men},%    \LtrMtsKonklusionSet   Menge
	type    ={symbols},
	symbol  ={},
	user6   ={},
	see     ={},
	description={
		Eine \Menge\ von \Konklusionen.
	}
}

\newVerweis                 {\MtsKonklusionRel}{\glstext}{MtsKonklusionRel}
\newglossaryentry            {MtsKonklusionRel}{
	text    ={\ensuremath{\RawMtsKonklusionRel}},
	name    ={\ensuremath{\RawMtsKonklusionRel} \addIdx[
		name={\ensuremath{\RawMtsKonklusionRel} (Relation)},
		sort={K Rel}]                                    {MtsKonklusionRel}},
	sort    ={K Rel},%    \LtrMtsKonklusionSet   Relation
	type    ={symbols},
	symbol  ={},
	user6   ={},
	see     ={},
	description={
		Eine \Relation\ (aufgefasst als \Menge) von \Konklusionen.
	}
}

\newVerweis                 {\MtsEmptyset}{\glstext}{MtsEmptyset}
\newglossaryentry            {MtsEmptyset}{
	text    ={\ensuremath{\RawMtsEmptyset}},
	name    ={\ensuremath{\RawMtsEmptyset} \addIdx[
		name={\ensuremath{\RawMtsEmptyset}},
		sort={O}]                                   {MtsEmptyset}},
	sort    ={O},
	type    ={symbols},
	symbol  ={},
	user6   ={},
	see     ={},
	description={
		Die \leereMenge, \textdh\ die einzige \Menge\ ohne \Elemente; auch mit $\{\}$ bezeichnet.
	}
}

\newcommand*             {\LtrMtsIN}               {N}% Natürliche Zahlen
\newVerweis                 {\MtsIN}{\glstext}{MtsIN}
\newglossaryentry            {MtsIN}{
	text    ={\ensuremath{\RawMtsIN}},
	name    ={\ensuremath{\RawMtsIN} \addIdx[
		name={\ensuremath{\RawMtsIN}},
		sort={N}]                             {MtsIN}},
	sort    ={N},%        \LtrMtsIN
	type    ={symbols},
	symbol  ={},
	user6   ={},
	see     ={},
	description={
		Die \Menge\ der \natuerlichenZahlen\ ohne 0.
	}
}

\newVerweis                 {\MtsINo}{\glstext}{MtsINo}
\newglossaryentry            {MtsINo}{
	text    ={\ensuremath{\RawMtsINo}},
	name    ={\ensuremath{\RawMtsINo} \addIdx[
		name={\ensuremath{\RawMtsINo}},
		sort={N 0}]                            {MtsINo}},
	sort    ={N 0},%      \LtrMtsIN 0
	type    ={symbols},
	symbol  ={},
	user6   ={},
	see     ={},
	description={
		Die \Menge\ der \natuerlichenZahlen\ (mit 0).
	}
}

\newVerweis                 {\MtsMn}{\glstext}{MtsMn}
\newglossaryentry            {MtsMn}{
	text    ={\ensuremath{\RawMtsMn}},
	name    ={\ensuremath{\RawMtsMn} \addIdx[
		name={\ensuremath{\RawMtsMn}},
		sort={M n}]                           {MtsMn}},
	sort    ={M n},
	type    ={symbols},
	symbol  ={},
	user6   ={},
	see     ={Tupel},
	description={
		Das \kartesischeProdukt\ $M \RawMtsTimes \dots \RawMtsTimes M$ aus $n$ Mengen $M$ mit $n \RawMtsIn \RawMtsINo$.
	}
}

\newVerweis                 {\MtsMo}{\glstext}{MtsMo}
\newglossaryentry            {MtsMo}{
	text    ={\ensuremath{\RawMtsMo}},
	name    ={\ensuremath{\RawMtsMo} \addIdx[
		name={\ensuremath{\RawMtsMo}},
		sort={M 0}]                           {MtsMo}},
	sort    ={M 0},
	type    ={symbols},
	symbol  ={},
	user6   ={},
	see     ={},
	description={
		$\{()\}$, wobei $()$ das $0$-\Tupel\ ist.
	}
}

\newcommand*             {\LtrMtsPraemisse}        {p}% Eine Voraussetzung; Prämisse
\newVerweis                 {\MtsPraemisse}{\glstext}{MtsPraemisse}
\newglossaryentry            {MtsPraemisse}{
	text    ={\ensuremath{\RawMtsPraemisse}},
	name    ={\ensuremath{\RawMtsPraemisse} \addIdx[
		name={\ensuremath{\RawMtsPraemisse} (Element)},
		sort={p Ele}]                                {MtsPraemisse}},
	sort    ={p Ele},%    \LtrMtsPraemisse   Element
	type    ={symbols},
	symbol  ={},
	user6   ={},
	see     ={},
	description={
		Eine \Praemisse.
	}
}

\newcommand*             {\LtrMtsPraemisseSet}     {P}% Menge der Voraussetzungen; Prämissen
\newVerweis                 {\MtsPraemisseSet}{\glstext}{MtsPraemisseSet}
\newglossaryentry            {MtsPraemisseSet}{
	text    ={\ensuremath{\RawMtsPraemisseSet}},
	name    ={\ensuremath{\RawMtsPraemisseSet} \addIdx[
		name={\ensuremath{\RawMtsPraemisseSet} (Menge)},
		sort={P Men}]                                   {MtsPraemisseSet}},
	sort    ={P Men},%    \LtrMtsPraemisseSet   Menge
	type    ={symbols},
	symbol  ={},
	user6   ={},
	see     ={},
	description={
		Eine \Menge\ von \Praemissen.
	}
}

\newVerweis                 {\MtsPraemisseRel}{\glstext}{MtsPraemisseRel}
\newglossaryentry            {MtsPraemisseRel}{
	text    ={\ensuremath{\RawMtsPraemisseRel}},
	name    ={\ensuremath{\RawMtsPraemisseRel} \addIdx[
		name={\ensuremath{\RawMtsPraemisseRel} (Relation)},
		sort={P Rel}]                                   {MtsPraemisseRel}},
	sort    ={P Rel},%    \LtrMtsPraemisseSet   Relation
	type    ={symbols},
	symbol  ={},
	user6   ={},
	see     ={},
	description={
		Eine \Relation\ (aufgefasst als \Menge) von \Praemissen.
	}
}

\newcommand*             {\LtrMtsSchlussregel}     {C}% conclusionrule; Schlussregel
\newVerweis                 {\MtsSchlussregel}{\glstext}{MtsSchlussregel}
\newglossaryentry            {MtsSchlussregel}{
	text    ={\ensuremath{\RawMtsSchlussregel}},
	name    ={\ensuremath{\RawMtsSchlussregel} \addIdx[
		name={\ensuremath{\RawMtsSchlussregel} (Element)},
		sort={C Ele}]                                   {MtsSchlussregel}},
	sort    ={C Ele},%    \LtrMtsSchlussregel   Element
	type    ={symbols},
	symbol  ={},
	user6   ={},
	see     ={},
	description={
		Eine \Schlussregel.
	}
}

\newVerweis                 {\MtsSchlussregelSet}{\glstext}{MtsSchlussregelSet}
\newglossaryentry            {MtsSchlussregelSet}{
	text    ={\ensuremath{\RawMtsSchlussregelSet}},
	name    ={\ensuremath{\RawMtsSchlussregelSet} \addIdx[
		name={\ensuremath{\RawMtsSchlussregelSet} (Menge)},
		sort={C Men}]                                      {MtsSchlussregelSet}},
	sort    ={C Men},%    \LtrMtsSchlussregel      Menge
	type    ={symbols},
	symbol  ={},
	user6   ={},
	see     ={},
	description={
		Eine \Menge\ von \Schlussregeln.
	}
}

\newcommand*             {\LtrMtsSprache}          {L}%      Sprache; language; \LtrOjkFor
\newVerweis                 {\MtsSprache}{\glstext}{MtsSprache}
\newglossaryentry            {MtsSprache}{
	text    ={\ensuremath{\RawMtsSprache}},
	name    ={\ensuremath{\RawMtsSprache} \addIdx[
		name={\ensuremath{\RawMtsSprache}},
		sort={L}]                                  {MtsSprache}},
	sort    ={L},%        \LtrMtsSprache
	type    ={symbols},
	symbol  ={},
	user6   ={},
	see     ={Formelmenge},
	description={
		Eine \Sprache.
	}
}

\newcommand*             {\LtrMtsTup}              {T}% sequenz; Menge der Tupel
\newVerweis                 {\MtsTup}{\glstext}{MtsTup}
\newglossaryentry            {MtsTup}{
	text    ={\ensuremath{\RawMtsTup}},
	name    ={\ensuremath{\RawMtsTup} \addIdx[
		name={\ensuremath{\RawMtsTup}},
		sort={T}]                              {MtsTup}},
	sort    ={T},%        \LtrMtsTup
	type    ={symbols},
	symbol  ={},
	user6   ={},
	see     ={Tupelmenge},
	description={
		Eine \Mengenoperation: $\RawMtsTup(M)$ ist die \Menge\ aller \Tupel\ von $M$.
	}
}

\newcommand*             {\LtrMtsTransformation}   {T}% Transformation, Transformation,
\newVerweis                 {\MtsTransformation}{\glstext}{MtsTransformation}
\newglossaryentry            {MtsTransformation}{
	text    ={\ensuremath{\RawMtsTransformation}},
	name    ={\ensuremath{\RawMtsTransformation} \addIdx[
		name={\ensuremath{\RawMtsTransformation} (Element)},
		sort={T Ele}]                                     {MtsTransformation}},
	sort    ={T Ele},%    \LtrMtsTransformation   Element
	type    ={symbols},
	symbol  ={},
	user6   ={},
	see     ={},
	description={
		Eine \Transformation.
	}
}

\newVerweis                 {\MtsTransformationTup}{\glstext}{MtsTransformationTup}
\newglossaryentry            {MtsTransformationTup}{
	text    ={\ensuremath{\RawMtsTransformationTup}},
	name    ={\ensuremath{\RawMtsTransformationTup} \addIdx[
		name={\ensuremath{\RawMtsTransformationTup} (Tupel)},
		sort={T Tup}]                                        {MtsTransformationTup}},
	sort    ={T Tup},%    \LtrMtsTransformation      Tupel
	type    ={symbols},
	symbol  ={},
	user6   ={},
	see     ={},
	description={
		Eine \Menge\ von \Transformationen.
	}
}

% ==============================================================================
% \Ojk* - Ausgabe als Text-Symbol und Eintrag und Verweis ins Symbolverzeichnis
% Symbole für die Konstruktiuon von logischen Formeln ==========================

\newcommand*             {\LtrOjkABC}              {A}
\newVerweis                 {\OjkABC}{\glstext}{OjkABC}
\newglossaryentry            {OjkABC}{
	text    ={\ensuremath{\RawOjkABC}},
	name    ={\ensuremath{\RawOjkABC} \addIdx[
		name={\ensuremath{\RawOjkABC}},
		sort={A a}]                            {OjkABC}},
	sort    ={A a},%      \LtrOjkABC ('a' wegen Kollision mit 'Glo-TextSymbole')
	type    ={symbols},
	symbol  ={},
	user6   ={},
	see     ={},
	description={
		Das \Alphabet\ der \aussagenlogischenSprache.
	}
}

\newVerweis                 {\OjkABCx}{\glstext}{OjkABCx}
\newglossaryentry            {OjkABCx}{
	text    ={\ensuremath{\RawOjkABC_x}},
	name    ={\ensuremath{\RawOjkABC_x} \addIdx[
		name={\ensuremath{\RawOjkABC_x}},
		sort={A x}]                             {OjkABCx}},
	sort    ={A x},%      \LtrOjkABC x
	type    ={symbols},
	symbol  ={},
	user6   ={},
	see     ={},
	description={
		Eine \Teilmenge\ des \Alphabets\ \OjkABC\ der \aussagenlogischenSprache.
	}
}

\newcommand*             {\LtrOjkFor}              {L}% Sprache; language; \LtrMtsSprache
\newVerweis                 {\OjkFor}{\glstext}{OjkFor}
\newglossaryentry            {OjkFor}{
	text    ={\ensuremath{\RawOjkFor}},
	name    ={\ensuremath{\RawOjkFor} \addIdx[
		name={\ensuremath{\RawOjkFor}},
		sort={L A}]                            {OjkFor}},
	sort    ={L A},%      \LtrOjkFor \LtrMtsIdxLogisch
	type    ={symbols},
	symbol  ={},
	user6   ={},
	see     ={},
	description={
		Eine \Formelmenge: Die \Menge\ der \aussagenlogischenFormeln\ mit \Klammerung.
	}
}

\newVerweis                 {\OjkForp}{\glstext}{OjkForp}
\newglossaryentry            {OjkForp}{
	text    ={\ensuremath{\RawOjkForp}},
	name    ={\ensuremath{\RawOjkForp} \addIdx[
		name={\ensuremath{\RawOjkForp}},
		sort={L Ap}]                            {OjkForp}},
	sort    ={L Ap},%     \LtrOjkFor \LtrMtsIdxLogisch\LtrMtsIdxPolnisch
	type    ={symbols},
	symbol  ={},
	user6   ={},
	see     ={},
	description={
		Eine \Formelmenge: Die \Menge\ der \aussagenlogischenFormeln\ in \PolnischerNotation.
	}
}

\newVerweis                 {\OjkForx}{\glstext}{OjkForx}
\newglossaryentry            {OjkForx}{
	text    ={\ensuremath{\RawOjkFor_x}},
	name    ={\ensuremath{\RawOjkFor_x} \addIdx[
		name={\ensuremath{\RawOjkFor_x}},
		sort={L A x}]                           {OjkForx}},
	sort    ={L A x},%    \LtrOjkFor \LtrMtsIdxLogisch x
	type    ={symbols},
	symbol  ={},
	user6   ={},
	see     ={},
	description={
		Eine \Formelmenge: Eine \Teilmenge\ der \Menge\ \OjkFor\ der \aussagenlogischenFormeln\ mit \Klammerung.
	}
}

\newVerweis                 {\OjkForpx}{\glstext}{OjkForpx}
\newglossaryentry            {OjkForpx}{
	text    ={\ensuremath{\RawOjkForp_x}},
	name    ={\ensuremath{\RawOjkForp_x} \addIdx[
		name={\ensuremath{\RawOjkForp_x}},
		sort={L Ap x}]                           {OjkForpx}},
	sort    ={L Ap x},%   \LtrOjkFor \LtrMtsIdxLogisch\LtrMtsIdxPolnisch x
	type    ={symbols},
	symbol  ={},
	user6   ={},
	see     ={},
	description={
		Eine \Formelmenge: Eine \Teilmenge\ der \Menge\ \OjkForp\ der \aussagenlogischenFormel\ in \PolnischerNotation.
	}
}

\newcommand*             {\LtrOjkJun}              {J}% Junktoren
\newVerweis                 {\OjkJun}{\glstext}{OjkJun}
\newglossaryentry            {OjkJun}{
	text    ={\ensuremath{\RawOjkJun}},
	name    ={\ensuremath{\RawOjkJun} \addIdx[
		name={\ensuremath{\RawOjkJun}},
		sort={J}]                              {OjkJun}},
	sort    ={J},%        \LtrOjkJun
	type    ={symbols},
	symbol  ={},
	user6   ={},
	see     ={Junktor},
	description={
		Die \Menge\ der \Junktorsymbole.
	}
}

\newVerweis                 {\OjkJunx}{\glstext}{OjkJunx}
\newglossaryentry            {OjkJunx}{
	text    ={\ensuremath{\RawOjkJun_x}},
	name    ={\ensuremath{\RawOjkJun_x} \addIdx[
		name={\ensuremath{\RawOjkJun_x}},
		sort={J x}]                             {OjkJunx}},
	sort    ={J x},%      \LtrOjkJun x
	type    ={symbols},
	symbol  ={},
	user6   ={},
	see     ={},
	description={
		Eine \Teilmenge\ der \Menge\ \OjkJun\ der \Junktorsymbole.
	}
}

\newVerweis                 {\OjkBin}{\glstext}{OjkBin}
\newglossaryentry            {OjkBin}{
	text    ={\ensuremath{\RawOjkBin}},
	name    ={\ensuremath{\RawOjkBin} \addIdx[
		name={\ensuremath{\RawOjkBin}},
		sort={J b}]                            {OjkBin}},
	sort    ={J b},%      \LtrOjkJun \StrMtsIdxBin
	type    ={symbols},
	symbol  ={},
	user6   ={},
	see     ={},
	description={
		Die \Menge\ der \binaerenJunktoren.
	}
}

\newVerweis                 {\OjkCon}{\glstext}{OjkCon}
\newglossaryentry            {OjkCon}{
	text    ={\ensuremath{\RawOjkCon}},
	name    ={\ensuremath{\RawOjkCon} \addIdx[
		name={\ensuremath{\RawOjkCon}},
		sort={J c}]                            {OjkCon}},
	sort    ={J c},%      \LtrOjkJun \StrMtsIdxCon
	type    ={symbols},
	symbol  ={},
	user6   ={},
	see     ={},
	description={
		Die \Menge\ der \aussagenlogischenKonstanten.
	}
}

\newVerweis                 {\OjkUna}{\glstext}{OjkUna}
\newglossaryentry            {OjkUna}{
	text    ={\ensuremath{\RawOjkUna}},
	name    ={\ensuremath{\RawOjkUna} \addIdx[
		name={\ensuremath{\RawOjkUna}},
		sort={J u}]                            {OjkUna}},
	sort    ={J u},%      \LtrOjkJun \StrMtsIdxUna
	type    ={symbols},
	symbol  ={},
	user6   ={},
	see     ={},
	description={
		Die \Menge\ der \unaerenJunktoren.
	}
}

\newcommand*             {\LtrOjkvar}              {q}% Name aussagenlogische Variable
\newVerweis                 {\Ojkvar}{\glstext}{Ojkvar}
\newglossaryentry            {Ojkvar}{
	text    ={\ensuremath{\RawOjkvar}},
	name    ={\ensuremath{\RawOjkvar} \addIdx[
		name={\ensuremath{\RawOjkvar}},
		sort={q}]                              {Ojkvar}},
	sort    ={q},%        \LtrOjkvar
	type    ={symbols},
	symbol  ={},
	user6   ={},
	see     ={Aussagenlogik},
	description={
		Die \Elemente\ aus \OjkVar\ sind die \aussagenlogischenVariablen.
	}
}

\newcommand*             {\LtrOjkVar}              {Q}% Menge aussagenlogische Variable
\newVerweis                 {\OjkVar}{\glstext}{OjkVar}
\newglossaryentry            {OjkVar}{
	text    ={\ensuremath{\RawOjkVar}},
	name    ={\ensuremath{\RawOjkVar} \addIdx[
		name={\ensuremath{\RawOjkVar}},
		sort={Q}]                              {OjkVar}},
	sort    ={Q},%        \LtrOjkVar
	type    ={symbols},
	symbol  ={},
	user6   ={},
	see     ={Aussagenlogik,Menge},
	description={
		$\OjkVar \RawMtsDefEq \RawMengeDef{\Ojkvar_i}{i \in \RawMtsINo}$,
		die \Menge\ der \aussagenlogischenVariablen.
	}
}

% ==============================================================================
% \Mts* - Ausgabe als Text-Symbol und Eintrag und Verweis ins Symbolverzeichnis
% Wahrheitswerte ===============================================================

\newcommand*             {\StrMtsFalse}            {false}
\newVerweis                 {\MtsFalse}{\glstext}{MtsFalse}
\newglossaryentry            {MtsFalse}{
	text    ={\ensuremath{\RawMtsFalse}},
	name    ={\ensuremath{\RawMtsFalse} \addIdx[
		name={\ensuremath{\RawMtsFalse}},
		sort={false}]                            {MtsFalse}},
	sort    ={false},%    \StrMtsFalse
	type    ={symbols},
	symbol  ={},
	user6   ={},
	see     ={MtsTrue,OjkFalse},
	description={
		Der \metasprachlicheWahrheitswert\ \TxtFalse\ als \Symbol.
	}
}

\newcommand*             {\StrMtsTrue}             {true}
\newVerweis                 {\MtsTrue}{\glstext}{MtsTrue}
\newglossaryentry            {MtsTrue}{
	text    ={\ensuremath{\RawMtsTrue}},
	name    ={\ensuremath{\RawMtsTrue} \addIdx[
		name={\ensuremath{\RawMtsTrue}},
		sort={true}]                            {MtsTrue}},
	sort    ={true},%     \StrMtsTrue
	type    ={symbols},
	symbol  ={},
	user6   ={},
	see     ={MtsFalse,OjkTrue},
	description={
		Der \metasprachlicheWahrheitswert\ \TxtTrue\ als \Symbol.
	}
}
