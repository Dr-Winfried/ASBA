%%############################################################################%%
%%                                                                            %%
%% Datei:  ASBA-Vorspann-Glossary.tex                                         %%
%% Inhalt: Vorspann Glossareinträge für ASBA                                  %%
%%                                                                            %%
%% Copyright (C) 2017  Winfried Teschers                                      %%
%%                                                                            %%
%% This program is free software: you can redistribute it and/or modify       %%
%% it under the terms of the GNU Affero General Public License as published   %%
%% by the Free Software Foundation, either version 3 of the License, or       %%
%% (at your option) any later version.                                        %%
%%                                                                            %%
%% This program is distributed in the hope that it will be useful,            %%
%% but WITHOUT ANY WARRANTY; without even the implied warranty of             %%
%% MERCHANTABILITY or FITNESS FOR A PARTICULAR PURPOSE.  See the              %%
%% GNU Affero General Public License for more details.                        %%
%%                                                                            %%
%% You should have received a copy of the GNU Affero General Public License   %%
%% along with this program.  If not, see <http://www.gnu.org/licenses/>.      %%
%%                                                                            %%
%% Dr. Winfried Teschers                                                      %%
%% Anton-Günther-Straße 26c                                                   %%
%% 91083 Baiersdorf                                                           %%
%% Germany                                                                    %%
%%                                                                            %%
%% e-mail: winfried.teschers@t-online.de                                      %%
%%                                                                            %%
%%############################################################################%%

% !TeX root = ASBA.tex
% !TeX encoding = UTF-8
% !TeX spellcheck = de_DE

% Elemente, die keine Glossareinträge sind, werden in "ASBA-Vorspann.tex" und "ASBA-Mathematik-Vorspann.tex" definiert.

\newcommand*{\glos}[1]{\textsf{#1}}% Schriftart für Verweise ins Glossar

% Symbole für Mengen -----------------------------------------------------------

\newglossaryentry{gsN}{
	name      ={$\gsN$},
	plural     ={\gsN},% im Mathematikmodus
	description ={Die Menge der natürlichen Zahlen ohne 0.}
}
\newglossaryentry{gsNo}{
	name      ={$\gsNo$},
	%TODO falscher Zeichensatz?
	plural     ={\gsNo},% im Mathematikmodus
	description ={Die Menge der natürlichen Zahlen einschließlich 0.}
}
\newglossaryentry{alABC}{
	name      ={$\alABC$},
	plural     ={\alABC},% im Mathematikmodus
	description ={Das Alphabet der aussagenlogischen Sprache.}
}
\newglossaryentry{alABCx}{
	name      ={$\alABC_x$},
	plural     ={\alABC_x},% im Mathematikmodus
	description ={
		Eine Teilmenge des Alphabets $\alABC$ der aussagenlogischen Sprache.
	}
}
\newglossaryentry{alBin}{
	name      ={$\alBin$},
	plural     ={\alBin},% im Mathematikmodus
	description ={Die Menge der aussagenlogischen, binären Operatoren.}
}
\newglossaryentry{alCon}{
	name      ={$\alCon$},
	plural     ={\alCon},% im Mathematikmodus
	description ={Die Menge der aussagenlogischen Konstanten.}
}
\newglossaryentry{alFor}{
	name      ={$\alFor$},
	plural     ={\alFor},% im Mathematikmodus
	description ={Die Menge der aussagenlogischen \glos{Formeln} mit Klammerung.}
}
\newglossaryentry{alForp}{
	name      ={$\alForp$},
	plural     ={\alForp},% im Mathematikmodus
	description ={
		Die Menge der aussagenlogischen \glos{Formeln} in polnischer Notation.
	}
}
\newglossaryentry{alForx}{
	name      ={$\alFor_x$},
	plural     ={\alFor_x},% im Mathematikmodus
	description ={
		Eine Teilmenge der Menge $\alFor$ der aussagenlogischen \glos{Formeln} mit Klammerung.
	}
}
\newglossaryentry{alForxp}{
	name      ={$\alForp_x$},
	plural     ={\alForp_x},% im Mathematikmodus
	description ={
		Eine Teilmenge der Menge $\alFor$ der aussagenlogischen \glos{Formeln} in polnischer Notation.
	}
}
\newglossaryentry{alJun}{
	name      ={$\alJun$},
	plural     ={\alJun},% im Mathematikmodus
	description ={Die Menge der aussagenlogischen \glos{Operatorsymbole}.}
}
\newglossaryentry{alJunx}{
	name      ={$\alJun_x$},
	plural     ={\alJun_x},% im Mathematikmodus
	description ={
		Eine Teilmenge der Menge $\alJun$ der aussagenlogischen Operatoren.
	}
}
\newglossaryentry{alUna}{
	name      ={$\alUna$},
	plural     ={\alUna},% im Mathematikmodus
	description ={Die Menge der aussagenlogischen unären Operatoren.}
}
\newglossaryentry{alVar}{
	name      ={$\alVar$},
	plural     ={\alVar},% im Mathematikmodus
	description ={Die Menge der aussagenlogischen Variablen.}
}

% Symbole für Beispieloperatoren -----------------------------------------------

\newglossaryentry{lrelbsp}{
	name      ={$\lrelbsp$},
	plural     ={\lrelbsp},% im Mathematikmodus
	description ={
		Beispielsymbol für eine Relation mit Umkehrrelation $\rrelbsp$.
	}
}
\newglossaryentry{lreleqbsp}{
	name      ={$\lreleqbsp$},
	plural     ={\lreleqbsp},% im Mathematikmodus
	description ={
		Beispielsymbol für eine Relation mit Gleichheit und Umkehrrelation $\rreleqbsp$.
	}
}
\newglossaryentry{relbsp}{
	name      ={$\relbsp$},
	plural     ={\relbsp},% im Mathematikmodus
	description ={Beispielsymbol für eine Relation.}
}
\newglossaryentry{releqbsp}{
	name      ={$\releqbsp$},
	plural     ={\releqbsp},% im Mathematikmodus
	description ={Beispielsymbol für eine Relation mit Gleichheit.}
}
\newglossaryentry{relnbsp}{
	name      ={$\relnbsp$},
	plural     ={\relnbsp},% im Mathematikmodus
	description ={Verneinung von $\relbsp$.}
}
\newglossaryentry{rrelbsp}{
	name      ={$\rrelbsp$},
	plural     ={\rrelbsp},% im Mathematikmodus
	description ={
		Beispielsymbol für eine Relation mit Umkehrrelation $\lrelbsp$.
	}
}
\newglossaryentry{rreleqbsp}{
	name      ={$\rreleqbsp$},
	plural     ={\rreleqbsp},% im Mathematikmodus
	description ={
		Beispielsymbol für eine Relation mit Gleichheit und Umkehrrelation $\lreleqbsp$.
	}
}

% Meta-Symbole -----------------------------------------------------------------

\newglossaryentry{defeq}{
	name      ={$:=$},
	plural      ={:=},% im Mathematikmodus
	description ={Definition: \textdots\ definitionsgemäß gleich \textdots}
}
\newglossaryentry{derive}{
	name      ={$\derivesym$},
	plural     ={\derive},% im Mathematikmodus
	description ={
		\glos{Ableitungsrelation}: \textdots\ ableitbar \textdots
		-- siehe \glos{ableitbar}.
	}
}
\newglossaryentry{eq}{
	name      ={$\eq$},
	plural     ={\eq},% im Mathematikmodus
	description ={
		Eine \glos{Metarelation}: \textdots\ gleich (ist dasselbe wie, ist identisch zu) \textdots
	}
}
\newglossaryentry{equiv}{
	name      ={$\equiv$},
	plural     ={\equiv},% im Mathematikmodus
	description ={
		Eine \glos{Metarelation}: \textdots\ äquivalent zu (ist das gleiche wie, ist so wie) \textdots
	}
}
\newglossaryentry{metaand}{
	name      ={$\metaandsym$},
	plural     ={\metaand},% im Mathematikmodus
	description ={Ein \glos{Metaoperator}: \textdots\ und \textdots}
}
\newglossaryentry{metadefeq}{
	name      ={$\metadefeq$},
	plural     ={\metadefeq},% im Mathematikmodus
	description ={
		\glos{Metadefinition}: \textdots\ definitionsgemäß gleich (definitionsgemäß genau dann, wenn) \textdots
	}
}
\newglossaryentry{metaequiv}{
	name      ={$\metaequiv$},
	plural     ={\metaequiv},% im Mathematikmodus
	description ={Eine \glos{Metarelation}: \textdots\ genau dann wenn \textdots}
}
\newglossaryentry{metaimp}{
	name      ={$\metaimp$},
	plural     ={\metaimp},% im Mathematikmodus
	description ={Eine \glos{Metarelation}: \textdots\ dann auch \textdots}
}
\newglossaryentry{metaor}{
	name      ={$\metaorsym$},
	plural     ={\metaor},% im Mathematikmodus
	description ={Ein \glos{Metaoperator}: \textdots\ oder \textdots}
}
\newglossaryentry{metarep}{
	name      ={$\metarep$},
	plural     ={\metarep},% im Mathematikmodus
	description ={Eine \glos{Metarelation}: \textdots\ sofern \textdots}
}
\newglossaryentry{ne}{
	name      ={$\ne$},
	plural     ={\ne},% im Mathematikmodus
	description ={
		Eine (Meta-)Operator: \textdots\ ungleich (nicht dasselbe wie, nicht identisch zu) \textdots
	}
}
\newglossaryentry{nequiv}{
	name      ={$\nequiv$},
	plural     ={\nequiv},% im Mathematikmodus
	description ={
		Eine \glos{Metarelation}: \textdots\ nicht äquivalent (ist nicht das gleiche wie, ist nicht so wie) \textdots
	}
}
\newglossaryentry{subst}{
	name      ={$\subst$},
	plural     ={\subst},% im Mathematikmodus
	description ={
		\glos{Substitution}: \textdots\ substituiert durch \textdots\
		-- siehe die Definition \vrefinsub{sub:Identitätsregeln}.
	}
}
\newglossaryentry{swap}{
	name      ={$\swap$},
	plural     ={\swap},% im Mathematikmodus
	description ={
		\glos{Vertauschung}: \textdots\ vertauscht mit \textdots\
		-- siehe die Definition \vrefinsub{sub:Identitätsregeln}.
	}
}
\newglossaryentry{srand}{
	name      ={$\srand$},
	plural     ={\srand},% im Mathematikmodus
	description ={
		Ein \glos{Metaoperator}: \textdots\ und \textdots\
		-- wird nur bei den \glos{Schlussregeln} verwendet.
	}
}

% sonstige mathematische Symbole -----------------------------------------------

\newglossaryentry{lfalse}{
	name      ={$\lfalse$},
	plural     ={\lfalse},% im Mathematikmodus
	description ={Eine aussagenlogische Konstante (\glos{Wahrheitswert}): Falsch.}
}
\newglossaryentry{ltrue}{
	name      ={$\ltrue$},
	plural     ={\ltrue},% im Mathematikmodus
	description ={Eine aussagenlogische Konstante (\glos{Wahrheitswert}): Wahr.}
}

% Schlussregeln ----------------------------------------------------------------

\newcommand* {\tagAR}{AR}% Argument für \tag - im Textmodus
\newcommand*    {\AR}{(\text{AR})}% im Mathematikmodus
\newglossaryentry{AR}{
	name      ={$\AR$},
	plural     ={\AR},% im Mathematikmodus
	description ={\glos{Anfangsregel}}
}
\newcommand* {\tagFS}{FS}% Argument für \tag - im Textmodus
\newcommand*    {\FS}{(\text{FS})}% im Mathematikmodus
\newglossaryentry{FS}{
	name      ={$\FS$},
	plural     ={\FS},% im Mathematikmodus
	description ={\glos{formaler Satz}}
}
\newcommand* {\tagMR}{MR}% Argument für \tag - im Textmodus
\newcommand*    {\MR}{(\text{MR})}% im Mathematikmodus
\newglossaryentry{MR}{
	name      ={$\MR$},
	plural     ={\MR},% im Mathematikmodus
	description ={\glos{Monotonieregel}}
}
\newcommand* {\tagSR}{SR}% Argument für \tag - im Textmodus
\newcommand*    {\SR}{(\text{SR})}% im Mathematikmodus
\newglossaryentry{SR}{
	name      ={$\SR$},
	plural     ={\SR},% im Mathematikmodus
	description ={\glos{Schnittregel} (Modus ponens)}
}
\newcommand* {\tagTR}{TR}% Argument für \tag - im Textmodus
\newcommand*    {\TR}{(\text{TR})}% im Mathematikmodus
\newglossaryentry{TR}{
	name      ={$\TR$},
	plural     ={\TR},% im Mathematikmodus
	description ={\glos{Abtrennungsregel}}
}
\newcommand* {\tageqB}{$\eq$B}% Argument für \tag - im Textmodus
\newcommand*    {\eqB}{(\eq\text{B})}% im Mathematikmodus
\newglossaryentry{eqB}{
	name      ={$\eqB$},
	plural     ={\eqB},% im Mathematikmodus
	description ={Beseitigung von \chrqt{$\eq$}}
}
\newcommand* {\tageqE}{$\eq$E}% Argument für \tag - im Textmodus
\newcommand*    {\eqE}{(\eq\text{E})}% im Mathematikmodus
\newglossaryentry{eqE}{
	name      ={$\eqE$},
	plural     ={\eqE},% im Mathematikmodus
	description ={Einführung von \chrqt{$\eq$}}
}
\newcommand* {\tagandB}{$\land$B}% Argument für \tag - im Textmodus
\newcommand*    {\andB}{(\land\text{B})}% im Mathematikmodus
\newglossaryentry{andB}{
	name      ={$\andB$},
	plural     ={\andB},% im Mathematikmodus
	description ={Beseitigung von \chrqt{$\land$}}
}
\newcommand* {\tagandE}{$\land$E}% Argument für \tag - im Textmodus
\newcommand*    {\andE}{(\land\text{E})}% im Mathematikmodus
\newglossaryentry{andE}{
	name      ={$\andE$},
	plural     ={\andE},% im Mathematikmodus
	description ={Einführung von \chrqt{$\land$}}
}
\newcommand* {\tagimpB}{$\limp$B}% Argument für \tag - im Textmodus
\newcommand*    {\impB}{(\limp\text{B})}% im Mathematikmodus
\newglossaryentry{impB}{
	name      ={$\impB$},
	plural     ={\impB},% im Mathematikmodus
	description ={Beseitigung von \chrqt{$\limp$}}
}
\newcommand* {\tagimpE}{$\limp$E}% Argument für \tag - im Textmodus
\newcommand*    {\impE}{(\limp\text{E})}% im Mathematikmodus
\newglossaryentry{impE}{
	name      ={$\impE$},
	plural     ={\impE},% im Mathematikmodus
	description ={Einführung von \chrqt{$\limp$}}
}
\newcommand* {\tagnota}{$\lnot$1}% Argument für \tag - im Textmodus
\newcommand*    {\nota}{(\lnot\text{1})}% im Mathematikmodus
\newglossaryentry{nota}{
	name={ $\nota$},
	plural     ={\nota},% im Mathematikmodus
	description ={Einführung/Beseitigung von \chrqt{$\lnot$} Teil 1}
}
\newcommand* {\tagnotb}{$\lnot$2}% Argument für \tag - im Textmodus
\newcommand*    {\notb}{(\lnot\text{2})}% im Mathematikmodus
\newglossaryentry{notb}{
	name      ={$\notb$},
	plural     ={\notb},% im Mathematikmodus
	description ={Einführung/Beseitigung von \chrqt{$\lnot$} Teil 2}
}
\newcommand* {\tagnotc}{$\lnot$3}% Argument für \tag - im Textmodus
\newcommand*    {\notc}{(\lnot\text{3})}% im Mathematikmodus
\newglossaryentry{notc}{
	name      ={$\notc$},
	plural     ={\notc},% im Mathematikmodus
	description ={Beweistechnik \enquote{Indirekter \glos{Beweis}}}
}
\newcommand* {\tagnotd}{$\lnot$4}% Argument für \tag - im Textmodus
\newcommand*    {\notd}{(\lnot\text{4})}% im Mathematikmodus
\newglossaryentry{notd}{% statt "notE"
	name      ={$\notd$},
	plural     ={\notd},% im Mathematikmodus
	description ={Reductio ad absurdum (Indirekter \glos{Beweis})}
}
%%%\newcommand* {\tagorB}{$\lor$B}% Argument für \tag - im Textmodus
%%%\newcommand*    {\orB}{(\lor\text{B})}% im Mathematikmodus
%%%\newglossaryentry{orB}{
%%%	name      ={$\obB$},
%%%	plural     ={\orB},% im Mathematikmodus
%%%	description ={Beseitigung von \chrqt{$\lor$}}
%%%}
%%%\newcommand* {\tagorE}{$\lor$E}% Argument für \tag - im Textmodus
%%%\newcommand*    {\orE}{(\lor\text{E})}% im Mathematikmodus
%%%\newglossaryentry{orE}{
%%%	name      ={$\obE$},
%%%	plural     ={\orE},% im Mathematikmodus
%%%	description ={Beseitigung von \chrqt{$\lor$}}
%%%}

% Fachbegriffe #################################################################

%A === A === A === A === A === A === A === A === A === A === A === A === A === A

\newcommand*{\ASBA}{\glsIdx{ASBA}}
\newacronym{ASBA}{ASBA}{
	Programmsystem, das \textbf{A}xiome, \textbf{S}ätze, \textbf{B}eweise und \textbf{A}uswertungen behandeln kann.
}
\newcommand*    {\ableitbar} {\glsIdx  {ableitbar}}
\newcommand*    {\ableitbare}{\glsIdxPl{ableitbar}}
\newglossaryentry{ableitbar}{
	name        ={ableitbar},
	plural      ={ableitbare},
	description ={
		Wenn sich eine \glos{Formel} $\beta$ aus einer anderen \glos{Formel} $\alpha$ mittels zulässiger Transaktionen ableiten lässt, heißt $\beta$ \glos{ableitbar} aus $\alpha$.
		Sprechweise: \seqqt{$ \alpha \text{ ableitbar } \beta $}.
		Eine oder beide \glos{Formeln} $\alpha$ \textbzw\ $\beta$ dürfen dabei durch \glos{Formelmengen} ersetzt werden.
		-- siehe \glos{Ableitungsrelation} und \chrqt{$\derivesym$}.
		\newline
		Synonym: \glos{beweisbar}.%
	}
}
\newcommand*    {\Ableitungsrelation}{\glsIdx{Ableitungsrelation}}
\newglossaryentry{Ableitungsrelation}{
	name        ={Ableitungsrelation},
	description ={Die Relation \chrqt{$\derivesym$}.}
}
\newcommand*    {\Abtrennungsregel}{\glsIdx{Abtrennungsregel}}
\newglossaryentry{Abtrennungsregel}{
	name        ={Abtrennungsregel},
	description ={Eine \glos{Schlussregel} -- siehe~\glos{TR}}
}
\newcommand*    {\allgemeingueltigeSchlussregel}  {\glsIdx  {allgemeingueltige-Schlussregel}}
\newcommand*    {\allgemeingueltigenSchlussregel} {\glsIdxBg{allgemeingueltige-Schlussregel}{allgemeingültigen Schlussregel}}
\newcommand*    {\allgemeingueltigeSchlussregeln} {\glsIdxPl{allgemeingueltige-Schlussregel}}
\newcommand*    {\allgemeingueltigenSchlussregeln}{\glsIdxBg{allgemeingueltige-Schlussregel}{allgemeingültigen Schlussregeln}}
\newglossaryentry{allgemeingueltige-Schlussregel}{
	name        ={allgemeingültige Schlussregel},
	plural      ={allgemeingültige Schlussregeln},
	description ={
		Eine \glos{Schlussregel} die aus den \glos{Basisregeln} und den schon bekannten \glos{allgemeingültigen Schlussregeln} abgeleitet werden kann.
	}
}
\newcommand*    {\Anfangsregel}{\glsIdx{Anfangsregel}}
\newglossaryentry{Anfangsregel}{
	name        ={Anfangsregel},
	description ={
		Eine \glos{Schlussregel} um beginnen zu können -- siehe~\glos{AR}.}
}
\newcommand*    {\atomareFormel} {\glsIdx  {atomare-Formel}}
\newcommand*    {\atomareFormeln}{\glsIdxPl{atomare-Formel}}
\newglossaryentry{atomare-Formel}{
	name        ={atomare Formel},
	plural      ={atomare Formeln},
	description ={Eine \glos{Formel}, die sich nicht weiter zerlegen lässt.}
}
\newcommand*    {\Ausgabeschema}  {\glsIdx  {Ausgabeschema}}
\newcommand*    {\Ausgabeschemata}{\glsIdxPl{Ausgabeschema}}
\newglossaryentry{Ausgabeschema}{
	name        ={Ausgabeschema},
	plural      ={Ausgabeschemata},
	description ={
		Ein Schema, mit dem bestimmte mathematische \glos{Objekte} ausgegeben werden sollen.
	}
}
\newcommand*    {\Aussage} {\glsIdx  {Aussage}}
\newcommand*    {\Aussagen}{\glsIdxPl{Aussage}}
\newglossaryentry{Aussage}{
	name        ={Aussage},
	plural      ={Aussagen},
	description ={
		Eine \glos{Aussage} in natürlicher Sprache oder als \glos{Formel}, die einen \glos{Wahrheitswert} liefert.
	}
}
\newcommand*    {\Aussagenlogik}{\glsIdx{Aussagenlogik}}
\newglossaryentry{Aussagenlogik}{
	name        ={Aussagenlogik},
	description ={-- \vrefseesec{sec:Aussagenlogik}.}
}
\newcommand*    {\Axiom}  {\glsIdx  {Axiom}}
\newcommand*    {\Axiome} {\glsIdxPl{Axiom}}
\newcommand*    {\Axiomen}{\glsIdxBg{Axiom}{Axiomen}}
\newglossaryentry{Axiom}{
	name        ={Axiom},
	plural      ={Axiome},
	description ={Eine \glos{Formel}, die unbewiesen als wahr angesehen wird.}
}
\newcommand*    {\Axiomensystem} {\glsIdx  {Axiomensystem}}
\newcommand*    {\Axiomensysteme}{\glsIdxPl{Axiomensystem}}
\newglossaryentry{Axiomensystem}{
	name        ={Axiomensystem},
	plural      ={Axiomensysteme},
	description ={Eine Menge von \glos{Axiomen}.}
}

%B === B === B === B === B === B === B === B === B === B === B === B === B === B

\newcommand*    {\Basisregel} {\glsIdx  {Basisregel}}
\newcommand*    {\Basisregeln}{\glsIdxPl{Basisregel}}
\newglossaryentry{Basisregel}{
	name        ={Basisregel},
	plural      ={Basisregeln},
	description ={
		Eine \glos{Schlussregel}, die nicht mehr auf andere zurückgeführt wird.
		Obwohl das auch auf die \glos{Identitätsregeln} zutrifft, werden diese hier aber nicht dazu gezählt.
	}
}
\newcommand*    {\Beweis}  {\glsIdx  {Beweis}}
\newcommand*    {\Beweises}{\glsIdxBg{Beweis}{Beweises}}
\newcommand*    {\Beweise} {\glsIdxPl{Beweis}}
\newcommand*    {\Beweisen}{\glsIdxBg{Beweis}{Beweisen}}
\newglossaryentry{Beweis}{
	name        ={Beweis},
	plural      ={Beweise},
	description ={
		Eine zulässige Ableitung von \glos{Folgerungen} aus gegebenen \glos{Voraussetzungen}.
		-- 	\newline\vrefseesec{sec:BeweiseASBA}.
	}
}
\newcommand*    {\beweisbar} {\glsIdx  {beweisbar}}
\newcommand*    {\beweisbare}{\glsIdxPl{beweisbar}}
\newglossaryentry{beweisbar}{
	name        ={beweisbar},
	plural      ={beweisbare},
	description ={Synonym zu \glos{ableitbar}.}
}
\newcommand*    {\Beweisschritt}  {\glsIdx  {Beweisschritt}}
\newcommand*    {\Beweisschritte} {\glsIdxPl{Beweisschritt}}
\newcommand*    {\Beweisschritten}{\glsIdxBg{Beweisschritt}{Beweisschritten}}
\newglossaryentry{Beweisschritt}{
	name        ={Beweisschritt},
	plural      ={Beweisschritte},
	description ={
		Eine Vorschrift, wie aus vorgegebenen \glos{Aussagen} (den \glos{Voraussetzungen}) weitere (die \glos{Folgerungen}) folgen.
	}
}
\newcommand*    {\Beweisschrittfolge} {\glsIdx  {Beweisschrittfolge}}
\newcommand*    {\Beweisschrittfolgen}{\glsIdxPl{Beweisschrittfolge}}
\newglossaryentry{Beweisschrittfolge}{
	name        ={Beweisschrittfolge},
	plural      ={Beweisschrittfolgen},
	description ={Eine Folge von \glos{Beweisschritten}.}
}
\newcommand*    {\Beweisschrittmenge} {\glsIdx  {Beweisschrittmenge}}
\newcommand*    {\Beweisschrittmengen}{\glsIdxPl{Beweisschrittmenge}}
\newglossaryentry{Beweisschrittmenge}{
	name        ={Beweisschrittmenge},
	plural      ={Beweisschrittmengen},
	description ={
		Eine Menge von \glos{Beweisschritten}, insbesondere die Menge der Glieder einer \glos{Beweisschrittfolge}.
	}
}
\newcommand*    {\BoolscheSignatur} {\glsIdx  {BoolscheSignatur}}
\newcommand*    {\BoolschenSignatur}{\glsIdxPl{BoolscheSignatur}}
\newglossaryentry{BoolscheSignatur}{
	name        ={Boolsche Signatur},
	plural      ={Boolschen Signatur},
	description ={Die \glos{logische Signatur} $\{\lnot, \land, \lor\}$.}
}

%F === F === F === F === F === F === F === F === F === F === F === F === F === F

\newcommand*    {\Fachbegriff}  {\glsIdx  {Fachbegriff}}
\newcommand*    {\Fachbegriffe} {\glsIdxPl{Fachbegriff}}
\newcommand*    {\Fachbegriffen}{\glsIdxBg{Fachbegriff}{Fachbegriffen}}
\newglossaryentry{Fachbegriff}{
	name        ={Fachbegriff},
	plural      ={Fachbegriffe},
	description ={Ein Name für einen mathematischen Begriff.}
}
\newcommand*    {\Fachgebiet}  {\glsIdx  {Fachgebiet}}
\newcommand*    {\Fachgebiete} {\glsIdxPl{Fachgebiet}}
\newcommand*    {\Fachgebieten}{\glsIdxBg{Fachgebiet}{Fachgebieten}}
\newglossaryentry{Fachgebiet}{
	name        ={Fachgebiet},
	plural      ={Fachgebiete},
	description ={
		Ein Teil der Mathematik mit einer zugehörigen Basis aus \glos{Axiomen}, \glos{Sätzen}, \glos{Fachbegriffen} und Darstellungsweisen.
	}
}
\newcommand*    {\Folgerung}  {\glsIdx  {Folgerung}}
\newcommand*    {\Folgerungen}{\glsIdxPl{Folgerung}}
\newglossaryentry{Folgerung}{
	name        ={Folgerung},
	plural      ={Folgerungen},
	description ={
		Die Folgerungen einer \glos{Schlussregel} $\frac{\prerequisiteset}{\conclusionset}$ sind die Elemente von $\conclusionset$.
	}
}
%%%\newcommand*    {\Folgerungsmenge} {\glsIdx  {Folgerungsmenge}}
%%%\newcommand*    {\Folgerungsmengen}{\glsIdxPl{Folgerungsmenge}}
%%%\newglossaryentry{Folgerungsmenge}{
%%%	name        ={Folgerungsmenge},
%%%	plural      ={Folgerungsmengen},
%%%	description ={
%%%		Die Menge der \glos{Folgerungen} einer \glos{Schlussregel} \textbzw\ eines \glos{Beweises}.
%%%	}
%%%}
\newcommand*    {\formalerSatz} {\glsIdx  {formaler-Satz}}
\newcommand*    {\formaleSaetze}{\glsIdxPl{formaler-Satz}}
\newglossaryentry{formaler-Satz}{
	name        ={formaler Satz},
	plural      ={formale Sätze},
	description ={
		Formale Darstellung eines mathematischen Satzes. -- siehe~\glos{FS}.
	}
}
\newcommand*    {\Formel} {\glsIdx  {Formel}}
\newcommand*    {\Formeln}{\glsIdxPl{Formel}}
\newglossaryentry{Formel}{
	name        ={Formel},
	plural      ={Formeln},
	description ={
		Unter einer \glos{Formel} verstehen wir stets eine mathematische \glos{Formel}.
		Diese kann auch mehrdimensional sein, lässt sich aber mittels geeigneter Definitionen immer eindeutig als eine \glos{Zeichenfolge} schreiben.
		\glos{Sätze}, \glos{Beweise} und \glos{Schlussregeln} betrachten wir \emph{nicht} als \glos{Formeln}.
	}
}
\newcommand*    {\Formelmenge} {\glsIdx  {Formelmenge}}
\newcommand*    {\Formelmengen}{\glsIdxPl{Formelmenge}}
\newglossaryentry{Formelmenge}{
	name        ={Formelmenge},
	plural      ={Formelmengen},
	description ={
		Eine Menge von \glos{Formeln} $\formula$ \textbzw\ \glos{Worten}, oft mit $\formulaset$ bezeichnet.
		Man nennt $\formulaset$ auch eine \glos{Sprache}, insbesondere dann, wenn es eindeutige Regeln zur Konstruktion von $\formulaset$ gibt.
	}
}

%G === G === G === G === G === G === G === G === G === G === G === G === G === G

\newcommand*    {\Gleichheitsrelation}  {\glsIdx  {Gleichheitsrelation}}
\newcommand*    {\Gleichheitsrelationen}{\glsIdxPl{Gleichheitsrelation}}
\newglossaryentry{Gleichheitsrelation}{
	name        ={Gleichheitsrelation},
	plural      ={Gleichheitsrelationen},
	description ={
		Eine mit Gleichheit verwandte Relation: $\eq$, $\ne$, $\equiv$ und $\nequiv$.
	}
}

%I === I === I === I === I === I === I === I === I === I === I === I === I === I

%TODO Identitätsregel nötig?
\newcommand*    {\Identitaetsregel} {\glsIdx  {Identitaetsregel}}
\newcommand*    {\Identitaetsregeln}{\glsIdxPl{Identitaetsregel}}
\newglossaryentry{Identitaetsregel}{
	name        ={Identitätsregel},
	plural      ={Identitätsregeln},
	description ={
		Eigentlich eine \glos{Basisregel} zur Identität.
		Da die \glos{Identitätsregeln} nur zur Rechtfertigung der \glos{Substitution} verwendet werden, werden sie hier nicht zu den \glos{Basisregeln} gezählt.
	}
}
\newcommand*    {\interessierendeEigenschaft}   {\glsIdx  {interessierende-Eigenschaft}}
\newcommand*    {\interessierendeEigenschaften} {\glsIdxPl{interessierende-Eigenschaft}}
\newcommand*    {\interessierendenEigenschaft}  {\glsIdxBg{interessierende-Eigenschaft}{interessierenden Eigenschaft}}
\newcommand*    {\interessierendenEigenschaften}{\glsIdxBg{interessierende-Eigenschaft}{interessierenden Eigenschaften}}
\newglossaryentry{interessierende-Eigenschaft}{
	name        ={interessierende Eigenschaft},
	plural      ={interessierende Eigenschaften},
	description ={
		Solche Eigenschaften von \glos{Objekten}, die im aktuellen Zusammenhang von Interesse sind, \textzB\ einen bestimmten Wert zu haben, Element einer bestimmten Menge zu sein, ein bestimmtes Objekt zu bezeichnen, usw.
	}
}

%J === J === J === J === J === J === J === J === J === J === J === J === J === J

\newcommand*    {\Junktor}  {\glsIdx  {Junktor}}
\newcommand*    {\Junktoren}{\glsIdxPl{Junktor}}
\newglossaryentry{Junktor}{
	name        ={Junktor},
	plural      ={Junktore},
	description ={
		Ein aussagenlogischer \glos{Operator}.
		Da die Werte einer aussagenlogischen \glos{Operation} Wahrheitswerte sind, kann man einen \glos{Junktor} auch als \glos{Relation} verstehen.
	}
}
\newcommand*    {\Junktorsymbol} {\glsIdx  {Junktorsymbol}}
\newcommand*    {\Junktorsymbole}{\glsIdxPl{Junktorsymbol}}
\newglossaryentry{Junktorsymbol}{
	name        ={Junktorsymbol},
	plural      ={Junktorsymbole},
	description ={
		Ein Symbol für einen \glos{Junktor}.
	}
}

%K === K === K === K === K === K === K === K === K === K === K === K === K === K

%TODO === weiter: Alle Fachbegriffe in Glossar und Index eintragen
\newglossaryentry{Kontraposition}{
	name        ={Kontraposition},
	plural      ={Kontraposition},
	description ={
		Die allgemeingültige \glos{Aussage}: $ (\alpha \limp \beta) \limp (\lnot\beta \limp \lnot\alpha) $.
	}
}

%L === L === L === L === L === L === L === L === L === L === L === L === L === L

\newglossaryentry{logische-Signatur}{
	name        ={logische Signatur},
	plural      ={logische Signaturen},
	description ={
		Eine Teilmenge von $\alJun$, die ausreicht um alle anderen Elemente von $\alJun$ zu definieren.
	}
}

%M === M === M === M === M === M === M === M === M === M === M === M === M === M

\newglossaryentry{Mengenlehre}{
	name={Mengenlehre},
	description ={-- \vrefseesec{sec:Mengenlehre}.}
}
\newglossaryentry{Metaoperator}{
	name        ={Metaoperator},
	plural      ={Metaoperatoren},
	description ={
		Ein Operator der \glos{Metasprache}: $\metaandsym$, $\metaorsym$ und $\srand$.
	}
}
\newglossaryentry{Metarelation}{
	name        ={Metarelation},
	plural      ={Metarelationen},
	description ={
		Eine Relation der \glos{Metasprache}: $\metaimp$, $\metarep$ und $\metaequiv$.
	}
}
\newglossaryentry{Metasprache}{
	name        ={Metasprache},
	plural      ={Metasprachen},
	description ={
		Eine Sprache, in der \glos{Aussagen} über Elemente einer anderen Sprache getroffen werden können.
		In diesem Dokument ist dies immer die normale Sprache.
		-- \vrefseesec{sec:Metasprache}.
	}
}
\newglossaryentry{Monotonieregel}{
	name        ={Monotonieregel},
	plural      ={Monotonieregeln},
	description ={
		Eine \glos{Schlussregel}. -- siehe~\glos{MR}.
	}
}

%O === O === O === O === O === O === O === O === O === O === O === O === O === O

\newglossaryentry{Objekt}{
	name        ={Objekt},
	plural      ={Objekte},
	description ={
		\glos{Symbole}, \glos{Formeln} und \glos{Aussagen} sowie Mengen, \glos{Zeichenfolgen}, Zahlen; ganz allgemein reale oder gedachte Dinge an sich.
	}
}
\newcommand*    {\Operator}  {\glsIdx  {Operator}}
\newcommand*    {\Operatoren}{\glsIdxPl{Operator}}
\newglossaryentry{Operator}{
	name        ={Operator},
	plural      ={Operatoren},
	description ={
		Eine -- meistens binäre, \textdh\ zweiwertige -- Funktion $M \times M \times \dots \times M \rightarrow M$.
		Für einen binären \glos{Operator} $\opbsp : M \times M \rightarrow M$ schreibt man meistens $x \opbsp y$ statt $\opbsp(x,y)$ oder $\opbsp x y$.
	}
}
\newcommand*    {\Operatorsymbol} {\glsIdx  {Operatorsymbol}}
\newcommand*    {\Operatorsymbole}{\glsIdxPl{Operatorsymbol}}
\newglossaryentry{Operatorsymbol}{
	name        ={Operatorsymbol},
	plural      ={Operatorsymbole},
	description ={
		Ein Symbol für einen \glos{Operator}.
	}
}

%P === P === P === P === P === P === P === P === P === P === P === P === P === P

\newglossaryentry{Praedikat}{
	name        ={Prädikat},
	plural      ={Prädikate},
	description ={
		Ein Element der \glos{Prädikatenlogik} -- \vrefseesec{sec:Prädikatenlogik}.\\
		\textZB\ kann man eine Gruppe als ein zweistelliges Prädikat $\mathrm{Gruppe}(G,+)$ definieren, in dem $G$ eine Menge und $+$ eine Operation, \textdh\ eine (zweistellige) Funktion $ +: G \times G \rightarrow G $ ist, so dass die Gruppenaxiome erfüllt sind.
	}
}
\newcommand*    {\Praedikatenlogik}{\glsIdx{Praedikatenlogik}}
\newglossaryentry{Praedikatenlogik}{
	name={Prädikatenlogik},
	description ={-- \vrefseesec{sec:Prädikatenlogik}.}
}

%R === R === R === R === R === R === R === R === R === R === R === R === R === R

\newcommand*    {\Relation}  {\glsIdx  {Relation}}
\newcommand*    {\Relationen}{\glsIdxPl{Relation}}
\newglossaryentry{Relation}{
	name        ={Relation},
	plural      ={Relationen},
	description ={
		Eine -- meistens binäre, \textdh\ zweiwertige -- Funktion $M \times M \times \dots \times M \rightarrow \{wahr, falsch\}$.
		Für eine binäre \glos{Relation} $\relbsp : M \times M \rightarrow \{wahr, falsch\}$ schreibt man meistens $x \relbsp y$ statt $\relbsp(x,y)$ oder $\relbsp x y$.
	}
}
\newcommand*    {\Relationssymbol} {\glsIdx  {Relationssymbol}}
\newcommand*    {\Relationssymbole}{\glsIdxPl{Relationssymbol}}
\newglossaryentry{Relationssymbol}{
	name        ={Relationssymbol},
	plural      ={Relationssymbole},
	description ={
		Ein Symbol für eine \glos{Relation}.
	}
}

%S === S === S === S === S === S === S === S === S === S === S === S === S === S

\newcommand*    {\Satz}   {\glsIdx  {Satz}}
\newcommand*    {\Satzes} {\glsIdxBg{Satz}{Satzes}}
\newcommand*    {\Saetze} {\glsIdxPl{Satz}}
\newcommand*    {\Saetzen}{\glsIdxBg{Satz}{Sätzen}}
\newglossaryentry{Satz}{
	name        ={Satz},
	plural      ={Sätze},
	description ={
		Eine mathematische \glos{Aussage}, dass bestimmte \glos{Folgerungen} aus gegebenen \glos{Voraussetzungen} abgeleitet werden können.
	}
}
\newglossaryentry{Schlussregel}{
	name        ={Schlussregel},
	plural      ={Schlussregeln},
	description ={
		Eine \glos{Schlussregel} $\frac{\prerequisiteset}{\conclusionset}$ entspricht der \glos{Aussage}:
		Wenn alle \glos{Voraussetzungen} $\prerequisite$ aus $\prerequisiteset$ zutreffen, dann auch alle \glos{Folgerungen} $\conclusion$ aus $\conclusionset$.
		Wenn diese \glos{Aussage} zutrifft, kann die Schlussregel zur zulässigen Umwandlung von \glos{Formeln} dienen.
	}
}
\newglossaryentry{Schlussregelmenge}{
	name        ={Schlussregelmenge},
	plural      ={Schlussregelmengen},
	description ={
		Eine Menge von \glos{Schlussregeln}, meistens mit $\conclusionruleset$ bezeichnet.
	}
}
\newglossaryentry{Schnittregel}{
	name        ={Schnittregel},
	plural      ={Schnittregeln},
	description ={Eine \glos{allgemeingültige Schlussregel}. -- siehe~\glos{SR}}.
}
\newglossaryentry{Sprache}{
	name        ={Sprache},
	plural      ={Sprachen},
	description ={-- siehe \glos{Formelmenge}.}
}
\newglossaryentry{Substitution}{ %TODO muss überarbeitet werden
	name        ={Substitution},
	plural      ={Substitutionen},
	description ={
		Die Ersetzung von einem, mehreren oder allen Teil-\glos{Formeln} ($\alpha$) in einer anderen \glos{Formel} ($\gamma$) durch eine dritte \glos{Formel} ($\beta$)
		-- formal: $\gamma(\alpha\subst\beta)$.
		Wenn alle $\alpha$ in $\gamma$ durch $\beta$ ersetzt werden, ist die \glos{Substitution} \emph{vollständig}.
		-- \vrefseesub{sub:Identitätsregeln}.
	}
}
%TODO Substitutionsmenge
\newglossaryentry{Symbol}{
	name        ={Symbol},
	plural      ={Symbole},
	description ={
		Ein \emph{einfaches} \glos{Symbol} ist ein druckbares typographisches Zeichen.
		Ein \emph{zusammengesetztes} \glos{Symbol} besteht aus mehreren einfachen Symbolen.
		In beiden Fällen wird ein Symbol als \emph{unzerlegbar} angesehen.
		-- \vrefseesec{sec:Notationen}.
	}
}

%T === T === T === T === T === T === T === T === T === T === T === T === T === T

\newglossaryentry{Transformation}{
	name        ={Transformation},
	plural      ={Transformationen},
	description ={
		Eine Umformung oder Erzeugung einer \glos{Formel} aus einer vorgegebenen Menge von \glos{Formeln},
		\textdh\ die Anwendung einer \glos{Schlussregel}.
	}
}
\newglossaryentry{Transformationsmenge}{
	name        ={Transformationsmenge},
	plural      ={Transformationsmenge},
	description ={
		Eine Menge von \glos{Transformationen}.
	}
}

%V === V === V === V === V === V === V === V === V === V === V === V === V === V

\newglossaryentry{vergleichbar}{
	name        ={vergleichbar},
	plural      ={vergleichbare},
	description ={
		Zwei \glos{Objekte} $A$ und $B$ sind vergleichbar, wenn beide von derselben Art sind, \textdh\ wenn beide \textzB\ jeweils Mengen, \glos{Zeichenfolgen}, Zahlen, \textusw\ sind.
		Dabei muss bei \glos{Formeln} zwischen der \glos{Formel} an sich und dem Ergebnis der \glos{Formel} unterschieden werden.
		-- \vrefseesec{subsub:Vergleichbar}.
	}
}
\newglossaryentry{Vertauschung}{ %TODO muss überarbeitet werden
	name        ={Vertauschung},
	plural      ={Vertauschungen},
	description ={
		Die \glos{Vertauschung} von zwei unabhängigen Teil-\glos{Formeln} ($\alpha$ und $\beta$) in einer anderen \glos{Formel} ($\gamma$)
		-- formal: $\gamma(\alpha\swap\beta)$.
		Die Vertauschung ist eine spezielle Form der \glos{Substitution}.
		-- siehe die Definition~\eqref{def:Vertauschung} \vrefinsub{sub:Identitätsregeln}.
	}
}
\newcommand*    {\Voraussetzung}  {\glsIdx  {Voraussetzung}}
\newcommand*    {\Voraussetzungen}{\glsIdxPl{Voraussetzung}}
\newglossaryentry{Voraussetzung}{
	name        ={Voraussetzung},
	plural      ={Voraussetzungen},
	description ={
		Die \glos{Voraussetzungen} einer \glos{Schlussregel} $\frac{\prerequisiteset}{\conclusionset}$ sind die Elemente von $\prerequisiteset$.
	}
}
%%%\newcommand*    {\Voraussetzungsmenge} {\glsIdx  {Voraussetzungsmenge}}
%%%\newcommand*    {\Voraussetzungsmengen}{\glsIdxPl{Voraussetzungsmenge}}
%%%\newglossaryentry{Voraussetzungsmenge}{
%%%	name        ={Voraussetzungsmenge},
%%%	plural      ={Voraussetzungsmengen},
%%%	description ={
%%%		Die Menge der \glos{Voraussetzungen} einer \glos{Schlussregel} \textbzw\ eines \glos{Beweises}.
%%%	}
%%%}

%W === W === W === W === W === W === W === W === W === W === W === W === W === W

\newglossaryentry{Wahrheitswert}{
	name        ={Wahrheitswert},
	plural      ={Wahrheitswerte},
	description ={
		Die Werte \chrqt{$\ltrue$} und \chrqt{$\lfalse$}, oft auch mit \chrqt{$\mathrm{wahr}$} und \chrqt{$\mathrm{falsch}$}, \chrqt{$\mathrm{true}$} und \chrqt{$\mathrm{false}$} oder einfach \chrqt{$1$} und \chrqt{$0$} bezeichnet.
	}
}
%TODO Wort definieren

%Z === Z === Z === Z === Z === Z === Z === Z === Z === Z === Z === Z === Z === Z

\newglossaryentry{Zeichenfolge}{
	name        ={Zeichenfolge},
	plural      ={Zeichenfolgen},
	description ={
		Folgen von unzerlegbaren Zeichen und \glos{Symbolen}, wobei Leerstellen und sonstiger Zwischenraum nicht zählen und nur zur besseren Darstellung dienen.
		Dabei sind als spezielle \glos{Symbole} auch \glos{Zeichenketten} erlaubt, solange die Zerlegung eindeutig bleibt.
		\textZB\ kann \chrqt{sin} als ein einzelnes Symbol -- für die Sinusfunktion -- aufgefasst werden, aber auch als Folge der Buchstaben \chrqt{s}, \chrqt{i} und \chrqt{n}.
		\glos{Formeln} werden immer als \glos{Zeichenfolgen} aufgefasst.
	}
}
\newglossaryentry{Zeichenkette}{
	name        ={Zeichenkette},
	plural      ={Zeichenketten},
	description ={
		Folgen von unzerlegbaren Zeichen, auch Leerstellen und sonstigem Zwischenraum.
		-- siehe auch \glos{Zeichenfolge}.
	}
}
\newglossaryentry{zerlegbar}{
	name        ={zerlegbar},
	description ={
		Eine \glos{Aussage}, die einen \glos{Metaoperator}, \textbzw\ eine \glos{Formel}, die einen \glos{Operator} oder eine \glos{Relation} enthält, heißt \glos{zerlegbar}.
	}
}
\newglossaryentry{zulaessige-Transformation}{%TODO ggf. überarbeiten
	name        ={zulässige Transformation},
	plural      ={zulässige Transformationen},
	description ={
		Eine \glos{Transformation} aus einer vorgegebenen Menge von \glos{Transformationen} oder eine daraus zulässiger weise abgeleitete \glos{Transformation}.
	}
}
