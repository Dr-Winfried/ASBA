%%############################################################################%%
%%                                                                            %%
%% Datei:  ASBA-Anhang.tex                                                    %%
%% Inhalt: Anhang                                                             %%
%%                                                                            %%
%% Copyright (C) 2017  Winfried Teschers                                      %%
%%                                                                            %%
%% This program is free software: you can redistribute it and/or modify       %%
%% it under the terms of the GNU Affero General Public License as published   %%
%% by the Free Software Foundation, either version 3 of the License, or       %%
%% (at your option) any later version.                                        %%
%%                                                                            %%
%% This program is distributed in the hope that it will be useful,            %%
%% but WITHOUT ANY WARRANTY; without even the implied warranty of             %%
%% MERCHANTABILITY or FITNESS FOR A PARTICULAR PURPOSE.  See the              %%
%% GNU Affero General Public License for more details.                        %%
%%                                                                            %%
%% You should have received a copy of the GNU Affero General Public License   %%
%% along with this program.  If not, see <http://www.gnu.org/licenses/>.      %%
%%                                                                            %%
%% Dr. Winfried Teschers                                                      %%
%% Anton-Günther-Straße 26c                                                   %%
%% 91083 Baiersdorf                                                           %%
%% Germany                                                                    %%
%%                                                                            %%
%% e-mail: winfried.teschers@t-online.de                                      %%
%%                                                                            %%
%%############################################################################%%

% !TeX root = ASBA.tex
% !TeX encoding = UTF-8
% !TeX spellcheck = de_DE

\appendix
\renewcommand*{\Chaptername}{\appendixname}

\chapter     {Anhang}% #########################################################
\beginchapter{Anhang}
\label   {cha:Anhang}

\section     {Werkzeuge}% ======================================================
\beginsection{Werkzeuge}
\label   {sec:Werkzeuge}

Da dies ein Open Source Projekt sein soll, müssen alle Werkzeuge, die zum Ablauf der Software erforderlich sind, ebenfalls Open Source sein.
Für die reine Entwicklung sollte das auch gelten, muss es aber nicht.

\paragraph{Werkzeuge zur Übersetzung der Quelldateien}% --------------------

\begin{enumerate}

	\item\label{Werkzeug:LaTeX}
	Ein Übersetzer für \LaTeX\ Quellcode (*.tex).
	--- Verwendet wird \emph{MiK\TeX}.

	\item\label{Werkzeug:Cpp}
	Ein Übersetzer für C++ Quellcode (*.c, *.cpp, *.h, *.hpp).
	--- Verwendet wird \emph{Visual Studio Community 2017}.

	\setcounter{Enumi}{\value{enumi}}% Nummerierung wird fortgesetzt.
\end{enumerate}
%
Nicht unbedingt nötig, aber sinnvoll:
\begin{enumerate}
	\setcounter{enumi}{\value{Enumi}}% Nummerierung wird fortgesetzt.

	\item\label{Werkzeug:Dokumentation}
	Ein Dokumentationssystem für in C++ Quellcode und darin enthaltene Doxygen Kommentare (*.c, *.cpp, *.h, *.hpp).
	--- Verwendet wird \emph{Doxygen} mit Konfigurationsdatei "`Doxyfile"'.

	\item\label{Werkzeug:Konfigurationsmanagement}
	Ein Konfigurationsmanagementsystem zur Verwaltung der Quelldateien.
	--- Verwendet wird \emph{GitHub}.

	\setcounter{Enumi}{\value{enumi}}% Nummerierung wird fortgesetzt.
\end{enumerate}

\paragraph{Werkzeuge für die Entwicklung}% -------------------------------------

\begin{enumerate}
	\setcounter{enumi}{\value{Enumi}}% Nummerierung wird fortgesetzt.

	\item\label{Werkzeug:GitHub}\emph{GitHub} als Online Konfigurationsmanagementsystem zur Zusammenarbeit verschiedener Entwickler.
	\\\tourl{https://github.com/}
	--- Lizenz \citesee{bib:GPLii}

	\item\label{Werkzeug:Git}GitHub benötigt \emph{Git} als Konfigurationsmanagementsystem.
	\\\tourl{https://git-scm.com/}
	--- Lizenz \citesee{bib:GPLii}

	\item\label{Werkzeug:MiKTeX}\emph{MiK\TeX} für Dokumentation und Ausgaben in \LaTeX.
	\\\tourl{https://miktex.org/}
	--- Lizenz \citesee{bib:MiKTeX}

	\item\label{Werkzeug:VSC}angedacht: \emph{Visual Studio Community 2017}%
	\footnote{%
		Visual Studio Community ist zwar nicht Open Source, darf aber zur Entwicklung von Open Source Software
		unentgeltlich verwendet werden.
	}
	(\emph{VS}) als Entwicklungsumgebung für C++.
	\\\tourl{https://www.visualstudio.com/downloads/}
	--- Lizenz \citesee{bib:EULA}

	\item\label{Werkzeug:VSC DB}angedacht: In \emph{Visual Studio Community 2015} integrierte Datenbank für \Ausgabeschemata, \Saetze, \Beweise, \Fachbegriffe\ und \Fachgebiete.
	--- Lizenz \citesee{bib:EULA}

	\item\label{Werkzeug:RapidXml}angedacht: \emph{RapidXml} für Ein- und Ausgabe in XML.
	\\\tourl{http://rapidxml.sourceforge.net/index.htm}
	--- Lizenz \citesee{bib:BSLi} oder wahlweise~\cite{bib:MIT}
	\footnote{%
		RapidXml stellt eine C++ Header-Datei zur Verfügung.
		Wenn diese im Quellcode eines Programms enthalten ist, gilt das ganze Programm als Open Source.
		Wenn diese Header-Datei nur in einer Bibliothek innerhalb eines Projekts verwendet wird, so gilt nur diese Bibliothek als Open Source.
	}

	\item\label{Werkzeug:Doxygen}angedacht: \emph{Doxygen} als Dokumentationssystem für C++.
	\\\tourl{http://www.stack.nl/~dimitri/doxygen/}
	--- Lizenz \citesee{bib:GPLii}

	\item\label{Werkzeug:Ghostscript}angedacht: Doxygen benötigt \emph{Ghostscript} als Interpreter für Postscript und PDF.
	\\\tourl{http://ghostscript.com/}
	--- Lizenz \citesee{bib:AGPL}

	\item\label{Werkzeug:Graphviz}angedacht: Doxygen benötigt \emph{Graphviz} mit \emph{Dot} zur Erzeugung und Visualisierung von Graphen.
	\\\tourl{http://www.graphviz.org/Home.php}
	--- Lizenz \citesee{bib:EPL}

	\setcounter{Enumi}{\value{enumi}}% Nummerierung wird fortgesetzt.
\end{enumerate}

\paragraph{Werkzeuge zur Bearbeitung der Quelldateien}% ------------------------

\begin{enumerate}
	\setcounter{enumi}{\value{Enumi}}% Nummerierung wird fortgesetzt.

	\item\label{Werkzeug:TeXstudio}\emph{\TeX studio} als Editor für \LaTeX.
	\\\tourl{http://www.texstudio.org/}
	--- Lizenz \citesee{bib:GPLii}
	\\\TeX studio benötigt einen Interpreter für Perl:

	\item\label{Werkzeug:Perl}\emph{Strawberry Perl} als Interpreter für Perl.
	\\\tourl{http://strawberryperl.com/}
	--- Lizenz: Various OSI-compatible Open Source licenses, or given to the public domain

	\item\label{Werkzeug:Notepadpp}\emph{Notepad++} als Text-Editor.
	\\\tourl{https://notepad-plus-plus.org/}
	--- Lizenz \citesee{bib:GPLi}

	\item\label{Werkzeug:WinMerge}\emph{WinMerge} zum Vergleich von Dateien und Verzeichnissen.
	\\\tourl{http://winmerge.org/}
	--- Lizenz \citesee{bib:GPLi}

	\setcounter{Enumi}{\value{enumi}}% Nummerierung wird fortgesetzt.
\end{enumerate}

\color{gray}%%% Anfang grauer Text ---------------------------------------------
\paragraph{Im Projekt \emph{qedeq} verwendete Werkzeuge}% ----------------------

\begin{itemize}
	\setcounter{enumi}{\value{Enumi}}% Nummerierung wird fortgesetzt.

	\item\label{Werkzeug:Java}\emph{Java} als Programmiersprache und Laufzeitumgebung.
	\\\tourl{https://www.java.com/de/download/win10.jsp}
	--- Lizenz \citesee{bib:JavaSE}

	\item\label{Werkzeug:Apache Ant}\emph{Apache Ant} als Java Bibliothek und Kommandozeilen-Werkzeug
	um Java Programme zu erzeugen.
	\\\tourl{http://ant.apache.org/}
	--- Lizenz \citesee{bib:Apacheii}

	\item\label{Werkzeug:Checkstyle}\emph{Checkstyle} zur statischen Code-Analyse für Java.
	\\\tourl{http://checkstyle.sourceforge.net/}
	--- Lizenz \citesee{bib:LGPLii}

	\item\label{Werkzeug:Clover}\emph{Clover}%
	\footnote{%
		Clover ist proprietäre Software, aber auf Anfrage frei für 30 Tage.
		Danach ist eine einmalige Lizenzgebühr fällig.
	}
	als Testwerkzeug zur Analyse der Code-Abdeckung.
	\\\tourl{https://www.atlassian.com/software/clover/}
	--- Lizenz \citesee{bib:Clover}

	\item\label{Werkzeug:Eclipse Java}\emph{Eclipse IDE for Java Developers} als Entwicklungsumgebung für Java.
	\\\tourl{http://www.eclipse.org/downloads/packages/eclipse-ide-java-developers/neon1a/}
	--- Lizenz \citesee{bib:OSI}

	\item\label{Werkzeug:JUnit}\emph{JUnit} zur Erzeugung von wiederholbaren Tests.
	\\\tourl{http://junit.org/junit4/}
	--- Lizenz \citesee{bib:EPL}

	\item\label{Werkzeug:Xerces2}\emph{Xerces2} als XML-Parser in Java.
	\\\tourl{http://xerces.apache.org/xerces2-j/}
	--- Lizenzen \citesee{bib:Apacheii, bib:SAX, bib:WDCDL, bib:WDCSNL}
\end{itemize}
\color{black}%%% Ende  grauer Text ---------------------------------------------

\section     [Die Struktur ausgewählter Begriffe]{Die Struktur ausgewählter \Begriffe}
\beginsection{Die Struktur ausgewählter Begriffe}
\label   {sec:Begriffsstruktur}

\begin{table}[H]
	\centering
	\begin{threeparttable}
		\setlength\extrarowheight{3pt}
		\begin{tabularx}{\linewidth}{c@{\extracolsep{\fill}}|c|c|c|c|}
			\hline% ------------------------------------------------------------
			\multicolumn{5}{c|}{\textbf{\Objekt}}\Tnote{1}
			\\
			\hline% ------------------------------------------------------------
		\end{tabularx}
		\begin{tablenotes}
			\footnotesize
			\item[1] Fußnote zur Tabelle
		\end{tablenotes}
	\end{threeparttable}
	\caption{\Bezeichnungen}
	\label{tab:Objekte}% Erst nach '\caption'!
\end{table}

\begin{table}[H]
	\begin{threeparttable}
		\setlength\extrarowheight{3pt}
		\begin{tabularx}{\linewidth}{c@{\extracolsep{\fill}}|c|c|c|c|}
			\hline% ------------------------------------------------------------
			\multicolumn{3}{c|}{\textbf{\Metasprache}}&
			\multicolumn{2}{c|}{\textbf{\Objektsprache}}
			\\
			\textbf{natürliche Sprache} & \multicolumn{2}{c|}{\textbf{\formaleMetasprache}}
			& \textbf{\Aussagenlogik} & \textbf{\Praedikatenlogik}
			\\
			\hline% ------------------------------------------------------------
			& \multicolumn{4}{c|}{\Symbole}
			\\
			& \multicolumn{2}{c|}{\Metasymbol}
			& \multicolumn{2}{c|}{\Objektsymbol}
			\\
			\hline% ------------------------------------------------------------
			& \multicolumn{4}{c|}{Beispielsymbole}
			\\
			\unaere\  \Operation
			& \multicolumn{4}{c|}{\BspOpU}
			\\
			\binaere\ \Operation
			& \multicolumn{4}{c|}{\BspOpB}
			\\
			\binaere\ \Relationen
			& \multicolumn{4}{c|}{\BspRel  \quad \BspRelEq  \quad \BspRelBck  \quad \BspRelBckEq \quad \BspRelN \quad \BspRelEqN \quad \BspRelBckN \quad \BspRelBckEqN}
			\\
			\hline% ------------------------------------------------------------
			\multicolumn{5}{c|}{\Wahrheitswerte}
			\\
			~                    \TxtTrue \quad \TxtFalse
			&\multicolumn{2}{c|}{\MtsTrue \quad \MtsFalse}
			&\multicolumn{2}{c|}{\OjkTrue \quad \OjkFalse}
			\\
			\hline% ------------------------------------------------------------
			& \multicolumn{4}{c|}{\Operation \quad \Relation \quad \Umkehrrelation \quad \Negation}
			\\
			& \multicolumn{2}{c|}{\Metaoperation \quad \Metarelation}
			& \multicolumn{2}{c|}{\Junktor}
			\\
			\hline% ------------------------------------------------------------
			~                       nicht
			& \multicolumn{2}{c|}{\MtsNot}
			& \multicolumn{2}{c|}{\OjkNot}
			\\
			~                         und \quad   oder \quad    dann
			& \multicolumn{2}{c|}{\MtsAnd \quad \MtsOr \quad \MtsImp}
			& \multicolumn{2}{c|}{\OjkAnd \quad \OjkOr \quad \OjkImp}
			\\
			~                     dann wenn \quad    wenn
			& \multicolumn{2}{c|}{\MtsEquiv \quad \MtsRep}
			& \multicolumn{2}{c|}{\OjkEquiv \quad \OjkRep}
			\\
			~                         und\Tnote{1} \quad entweder oder
			& \multicolumn{2}{c|}{\MtsUnd}
			& \multicolumn{2}{c|}{                             \OjkXor}
			\\
			~                    nicht und \quad nicht oder
			& \multicolumn{2}{c|}{ }
			& \multicolumn{2}{c|}{\OjkNand \quad \OjkNor}
			\\
			\hline% ------------------------------------------------------------
			~                     gleich \quad ungleich
			& \multicolumn{2}{c|}{\MtsEq \quad \MtsEqN}
			& \multicolumn{2}{c|}{\OjkEq \quad \OjkEqN}
			\\
			definitionsgemäß            gleich
			& \multicolumn{2}{c|}{\MtsDefEquiv}
			& \multicolumn{2}{c|}{ }
			\\
			definitionsgemäß         gleich
			& \multicolumn{2}{c|}{\MtsDefEq}
			& \multicolumn{2}{c|}{ }
			\\
			\hline% ------------------------------------------------------------
			\Quantoren
			& \multicolumn{2}{c|}{$\MtsForall \quad \MtsExists \quad \MtsExione$}
			&                   & $\OjkForall \quad \OjkExists \quad \OjkExione$
			\\
			\hline% ------------------------------------------------------------
			\Ersetzung \quad \Vertauschung
			& \multicolumn{2}{c|}{\MtsSubst \quad \MtsSwap}
			& \multicolumn{2}{c|}{ }
			\\
			\Ableitungsrelationen:
			& \multicolumn{2}{c|}{\MtsDerive \quad \MtsDeriveR \quad \MtsPraemisseRel \quad \MtsKonklusionRel \quad \MtsErgebnisRel}
			& \multicolumn{2}{c|}{ }
			\\
			\hline% ------------------------------------------------------------
			\Elementrelationen:
			& \multicolumn{2}{c|}{\MtsIn \quad \MtsNi \quad \MtsInN \quad \MtsNiN}
			& \multicolumn{2}{c|}{ }
			\\
			\Mengenrelationen:
			& \multicolumn{2}{c|}{\MtsSubset \quad \MtsSubsetEq \quad \MtsSubsetN \quad \MtsSubsetEqN \quad \MtsSupset \quad \MtsSupsetEq \quad \MtsSupsetN \quad \MtsSupsetEqN}
			& \multicolumn{2}{c|}{ }
			\\
			\Komponentenrelationen:
			& \multicolumn{2}{c|}{\MtsSeqIn \quad \MtsSeqNi \quad \MtsSeqInN \quad \MtsSeqNiN}
			& \multicolumn{2}{c|}{ }
			\\
			\Folgenrelationen:
			& \multicolumn{2}{c|}{\MtsSubseq \quad \MtsSubseqEq \quad \MtsSubseqN \quad \MtsSubseqEqN \quad \MtsSupseq \quad \MtsSupseqEq \quad \MtsSupseqN \quad \MtsSupseqEqN}
			& \multicolumn{2}{c|}{ }
			\\
			ausgewählte Mengen
			& \multicolumn{2}{c|}{\MtsIN \quad \MtsINo \quad \MtsUniversum \quad \Sprache }
			& \multicolumn{2}{c|}{ }
			\\
			\hline% ------------------------------------------------------------
			& \textbf{\unaer} & \textbf{\binaer}
			& \multicolumn{2}{c|}{ }
			\\
			\Mengenoperationen
			& \MtsPot \quad \MtsPotf \quad \MtsRel \quad \MtsRelf & \MtsCap \quad \MtsCup \quad \MtsSetminus \quad \MtsTimes
			& \multicolumn{2}{c|}{ }
			\\
			& \MtsFol \quad \MtsFolf \quad \MtsTup &
			& \multicolumn{2}{c|}{ }
			\\
			\hline% ------------------------------------------------------------
			\unaere\ \Operationen\ auf:
			& \textbf{\Relationen} & \textbf{\Funktionen}
			& \multicolumn{2}{c|}{ }
			\\
			& \MtsStelR            & \MtsStelF
			& \multicolumn{2}{c|}{ }
			\\
			\DefinitionsB- \quad \Zielbereich
			&                      & \MtsDb \quad \MtsZb
			& \multicolumn{2}{c|}{ }
			\\
			\QuellB- \quad \Wertebereich
			&                      & \MtsQb \quad \MtsWb
			& \multicolumn{2}{c|}{ }
			\\
			\Traegermenge
			& \multicolumn{2}{c|}{$\MtsTraeger \quad \MtsTraeger_i$}
			& \multicolumn{2}{c|}{ }
			\\
			\Graph
			& \multicolumn{2}{c|}{ \MtsGraph }
			& \multicolumn{2}{c|}{ }
			\\
			\hline% ------------------------------------------------------------
			\unaere\ \Operationen\ auf:
			& \multicolumn{2}{c|}{ \Folgen \quad \Tupel }
			& \multicolumn{2}{c|}{ }
			\\
			& \multicolumn{2}{c|}{ \MtsLen \quad \MtsSet }
			& \multicolumn{2}{c|}{ }
			\\
			\hline% ------------------------------------------------------------
		\end{tabularx}
		\begin{tablenotes}
			\footnotesize
			\item[] Die erste Spalte beschreibt die anderen Spalten.
			Die \textbf{fettgedruckten} Teile, und nur diese, gelten als Überschriften.
			\item[1] nur in Schlussregeln
		\end{tablenotes}
	\end{threeparttable}
	\caption{Ausgewählte \Bezeichnungen}
	\label{tab:Benennungen}% Erst nach '\caption'!
\end{table}


\section     {Offene Aufgaben}% ================================================
\beginsection{Offene Aufgaben}
\label   {sec:OffeneAufgaben}

\begin{enumerate}
	\item TODOs bearbeiten.
	\item Eingabeprogramm erstellen (liest XML).
	\item Prüfprogramm erstellen.
	\item Ausgabeprogramm erstellen (schreibt XML).
	\item Formelausgabe erstellen (erzeugt \LaTeX{} aus XML).
	\item \Axiome\ sammeln und eingeben.
	\item \Saetze\ sammeln und eingeben.
	\item \Beweise\ sammeln und eingeben.
	\item \Fachbegriffe\ und Symbole sammeln und eingeben.
	\item \Fachgebiete\ sammeln und eingeben.
	\item \Ausgabeschemata\ sammeln und eingeben.
\end{enumerate}

\Endchapter
