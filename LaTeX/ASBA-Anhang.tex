%%############################################################################%%
%%                                                                            %%
%% Datei:  ASBA-Anhang.tex                                                    %%
%% Inhalt: Anhang                                                             %%
%%                                                                            %%
%% Copyright (C) 2017  Winfried Teschers                                      %%
%%                                                                            %%
%% This program is free software: you can redistribute it and/or modify       %%
%% it under the terms of the GNU Affero General Public License as published   %%
%% by the Free Software Foundation, either version 3 of the License, or       %%
%% (at your option) any later version.                                        %%
%%                                                                            %%
%% This program is distributed in the hope that it will be useful,            %%
%% but WITHOUT ANY WARRANTY; without even the implied warranty of             %%
%% MERCHANTABILITY or FITNESS FOR A PARTICULAR PURPOSE.  See the              %%
%% GNU Affero General Public License for more details.                        %%
%%                                                                            %%
%% You should have received a copy of the GNU Affero General Public License   %%
%% along with this program.  If not, see <http://www.gnu.org/licenses/>.      %%
%%                                                                            %%
%% Dr. Winfried Teschers                                                      %%
%% Anton-Günther-Straße 26c                                                   %%
%% 91083 Baiersdorf                                                           %%
%% Germany                                                                    %%
%%                                                                            %%
%% e-mail: winfried.teschers@t-online.de                                      %%
%%                                                                            %%
%%############################################################################%%

% !TeX root = ASBA.tex
% !TeX encoding = UTF-8
% !TeX spellcheck = de_DE

\appendix
\renewcommand*{\Chaptername}{\appendixname}

\chapter{Anhang}% ##############################################################
\beginchapter{Anhang}
\label{cha:Anhang}

\section{Werkzeuge}% ===========================================================
\beginsection{Werkzeuge}
\label{sec:Werkzeuge}

Da dies ein Open Source Projekt sein soll, müssen alle Werkzeuge, die zum Ablauf der Software erforderlich sind, ebenfalls Open Source sein.
Für die reine Entwicklung sollte das auch gelten, muss es aber nicht.

\paragraph{Werkzeuge zur Übersetzung der Quelldateien}% --------------------

\begin{enumerate}

	\item\label{Werkzeug:LaTeX}
	Ein Übersetzer für \LaTeX Quellcode (*.tex).
	-- Verwendet wird \emph{MiK\TeX}.

	\item\label{Werkzeug:C++}
	Ein Übersetzer für C++ Quellcode (*.c, *.cpp, *.h, *.hpp).
	-- Verwendet wird \emph{Visual Studio Community 2017}.

	\setcounter{Enumi}{\value{enumi}}% Nummerierung wird fortgesetzt.
\end{enumerate}
%
Nicht unbedingt nötig, aber sinnvoll:
\begin{enumerate}
	\setcounter{enumi}{\value{Enumi}}% Nummerierung wird fortgesetzt.

	\item\label{Werkzeug:Dokumentation}
	Ein Dokumentationssystem für in C++ Quellcode und darin enthaltene Doxygen Kommentare (*.c, *.cpp, *.h, *.hpp).
	-- Verwendet wird \emph{Doxygen} mit Konfigurationsdatei \enquote{Doxyfile}.

	\item\label{Werkzeug:Konfigurationsmanagement}
	Ein Konfigurationsmanagementsystem zur Verwaltung der Quelldateien.
	-- Verwendet wird \emph{GitHub}.

	\setcounter{Enumi}{\value{enumi}}% Nummerierung wird fortgesetzt.
\end{enumerate}

\paragraph{Werkzeuge für die Entwicklung}% -------------------------------------

\begin{enumerate}
	\setcounter{enumi}{\value{Enumi}}% Nummerierung wird fortgesetzt.

	\item\label{Werkzeug:GitHub}\emph{GitHub} als Online Konfigurationsmanagementsystem zur Zusammenarbeit verschiedener Entwickler.
	\tourl{https://github.com/}
	-- Lizenz siehe~\cite{bib:GPLii}

	\item\label{Werkzeug:Git}GitHub benötigt \emph{Git} als Konfigurationsmanagementsystem.
	\tourl{https://git-scm.com/}
	-- Lizenz siehe~\cite{bib:GPLii}

	\item\label{Werkzeug:MiKTeX}\emph{MiK\TeX} für Dokumentation und Ausgaben in \LaTeX.
	\tourl{https://miktex.org/}
	-- Lizenz siehe~\cite{bib:MiKTeX}

	\item\label{Werkzeug:VSC}angedacht: \emph{Visual Studio Community 2017}%
	\footnote{%
		Visual Studio Community ist zwar nicht Open Source, darf aber zur Entwicklung von Open Source Software
		unentgeltlich verwendet werden.%
	}
	(\emph{VS}) als Entwicklungsumgebung für C++.
	\tourl{https://www.visualstudio.com/downloads/}
	-- Lizenz siehe~\cite{bib:EULA}

	\item\label{Werkzeug:VSC DB}angedacht: In \emph{Visual Studio Community 2015} integrierte Datenbank für \glsIdxPl{Axiom}, \glsIdxPl{Satz}, \glsIdxPl{Beweis}, \glsIdxPl{Fachbegriff} und \glsIdxPl{Fachgebiet}.
	-- Lizenz siehe~\cite{bib:EULA}

	\item\label{Werkzeug:RapidXml}angedacht: \emph{RapidXml} für Ein- und Ausgabe in XML.
	\tourl{http://rapidxml.sourceforge.net/index.htm}
	-- Lizenz siehe wahlweise~\cite{bib:BSLi} oder~\cite{bib:MIT}
	\footnote{%
		RapidXml stellt eine C++ Header-Datei zur Verfügung.
		Wenn diese im Quellcode eines Programms enthalten ist, gilt das ganze Programm als Open Source.
		Wenn diese Header-Datei nur in einer Bibliothek innerhalb eines Projekts verwendet wird, so gilt nur diese Bibliothek als Open Source.%
	}

	\item\label{Werkzeug:Doxygen}angedacht: \emph{Doxygen} als Dokumentationssystem für C++.
	\tourl{http://www.stack.nl/~dimitri/doxygen/}
	-- Lizenz siehe~\cite{bib:GPLii}

	\item\label{Werkzeug:Ghostscript}angedacht: Doxygen benötigt \emph{Ghostscript} als Interpreter für Postscript und PDF.
	\tourl{http://ghostscript.com/}
	-- Lizenz siehe~\cite{bib:AGPL}

	\item\label{Werkzeug:Graphviz}angedacht: Doxygen benötigt \emph{Graphviz} mit \emph{Dot} zur Erzeugung und Visualisierung von Graphen.
	\tourl{http://www.graphviz.org/Home.php}
	-- Lizenz siehe~\cite{bib:EPL}

	\setcounter{Enumi}{\value{enumi}}% Nummerierung wird fortgesetzt.
\end{enumerate}

\paragraph{Werkzeuge zur Bearbeitung der Quelldateien}% ------------------------

\begin{enumerate}
	\setcounter{enumi}{\value{Enumi}}% Nummerierung wird fortgesetzt.

	\item\label{Werkzeug:TeXstudio}\emph{\TeX studio} als Editor für \LaTeX.
	\tourl{http://www.texstudio.org/}
	-- Lizenz siehe~\cite{bib:GPLii}\\
	\TeX studio benötigt einen Interpreter für Perl:

	\item\label{Werkzeug:Perl}\emph{Strawberry Perl} als Interpreter für Perl.
	\tourl{http://strawberryperl.com/}
	-- Lizenz: Various OSI-compatible Open Source licenses, or given to the public domain

	\item\label{Werkzeug:Notepadpp}\emph{Notepad++} als Text-Editor.
	\tourl{https://notepad-plus-plus.org/}
	-- Lizenz siehe~\cite{bib:GPLi}

	\item\label{Werkzeug:WinMerge}\emph{WinMerge} zum Vergleich von Dateien und Verzeichnissen.
	\tourl{http://winmerge.org/}
	-- Lizenz siehe~\cite{bib:GPLi}

	\setcounter{Enumi}{\value{enumi}}% Nummerierung wird fortgesetzt.
\end{enumerate}

%%%	\paragraph{Im Projekt \emph{qedeq} verwendete Werkzeuge}% ----------------------
%%%
%%%	\begin{itemize}
%%%		\setcounter{enumi}{\value{Enumi}}% Nummerierung wird fortgesetzt.
%%%
%%%		\item\label{Werkzeug:Java}\emph{Java} als Programmiersprache und Laufzeitumgebung.
%%%		\tourl{https://www.java.com/de/download/win10.jsp}
%%%		-- Lizenz siehe~\cite{bib:JavaSE}
%%%
%%%		\item\label{Werkzeug:Apache Ant}\emph{Apache Ant} als Java Bibliothek und Kommandozeilen-Werkzeug
%%%		um Java Programme zu erzeugen.
%%%		\tourl{http://ant.apache.org/}
%%%		-- Lizenz siehe~\cite{bib:Apacheii}
%%%
%%%		\item\label{Werkzeug:Checkstyle}\emph{Checkstyle} zur statischen Code-Analyse für Java.
%%%		\tourl{http://checkstyle.sourceforge.net/}
%%%		-- Lizenz siehe~\cite{bib:LGPLii}
%%%
%%%		\item\label{Werkzeug:Clover}\emph{Clover}%
%%%		\footnote{%
%%%			Clover ist proprietäre Software, aber auf Anfrage frei für 30 Tage.
%%%			Danach ist eine einmalige Lizenzgebühr fällig.%
%%%		}
%%%		als Testwerkzeug zur Analyse der Code-Abdeckung.
%%%		\tourl{https://www.atlassian.com/software/clover/}
%%%		-- Lizenz siehe~\cite{bib:Clover}
%%%
%%%		\item\label{Werkzeug:Eclipse Java}\emph{Eclipse IDE for Java Developers} als Entwicklungsumgebung für Java.
%%%		\tourl{http://www.eclipse.org/downloads/packages/eclipse-ide-java-developers/neon1a/}
%%%		-- Lizenz siehe~\cite{bib:OSI}
%%%
%%%		\item\label{Werkzeug:JUnit}\emph{JUnit} zur Erzeugung von wiederholbaren Tests.
%%%		\tourl{http://junit.org/junit4/}
%%%		-- Lizenz siehe~\cite{bib:EPL}
%%%
%%%		\item\label{Werkzeug:Xerces2}\emph{Xerces2} als XML-Parser in Java.
%%%		\tourl{http://xerces.apache.org/xerces2-j/}
%%%		-- Lizenzen siehe~\cite{bib:Apacheii, bib:SAX, bib:WDCDL, bib:WDCSNL}
%%%
%%%		\setcounter{Enumi}{\value{enumi}}% Nummerierung wird fortgesetzt.
%%%	\end{itemize}

\section{Offene Aufgaben}% =====================================================
\beginsection{Offene Aufgaben}
\label{sec:Offene Aufgaben}

\begin{enumerate}
	\item TODOs bearbeiten.
	\item Eingabeprogramm erstellen (liest XML).
	\item Prüfprogramm erstellen.
	\item Ausgabeprogramm erstellen (schreibt XML).
	\item Formelausgabe erstellen (erzeugt \LaTeX{} aus XML).
	\item Axiome sammeln und eingeben.
	\item Sätze sammeln und eingeben.
	\item Beweise sammeln und eingeben.
	\item Fachbegriffe und Symbole sammeln und eingeben.
	\item Fachgebiete sammeln und eingeben.
	\item Ausgabeschemata sammeln und eingeben.
\end{enumerate}

\Endchapter
