%%############################################################################%%
%%                                                                            %%
%% Datei:  ASBA-Design.tex                                                    %%
%% Inhalt: Kapitel "Design"                                                   %%
%%                                                                            %%
%% Copyright (C) 2017  Winfried Teschers                                      %%
%%                                                                            %%
%% This program is free software: you can redistribute it and/or modify       %%
%% it under the terms of the GNU Affero General Public License as published   %%
%% by the Free Software Foundation, either version 3 of the License, or       %%
%% (at your option) any later version.                                        %%
%%                                                                            %%
%% This program is distributed in the hope that it will be useful,            %%
%% but WITHOUT ANY WARRANTY; without even the implied warranty of             %%
%% MERCHANTABILITY or FITNESS FOR A PARTICULAR PURPOSE.  See the              %%
%% GNU Affero General Public License for more details.                        %%
%%                                                                            %%
%% You should have received a copy of the GNU Affero General Public License   %%
%% along with this program.  If not, see <http://www.gnu.org/licenses/>.      %%
%%                                                                            %%
%% Dr. Winfried Teschers                                                      %%
%% Anton-Günther-Straße 26c                                                   %%
%% 91083 Baiersdorf                                                           %%
%% Germany                                                                    %%
%%                                                                            %%
%% e-mail: winfried.teschers@t-online.de                                      %%
%%                                                                            %%
%%############################################################################%%

% !TeX root = ASBA.tex
% !TeX encoding = UTF-8
% !TeX spellcheck = de_DE

\chapter{Design}% ##############################################################
\beginchapter{Design}
\label{cha:Design}

Dieses Projekt soll Open Source sein.
Daher gilt für die Dokumente die \emph{GNU Free Documentation License (FDL)} und für die Software die \emph{GNU Affero General Public License (APGL)}.
Die \emph{GNU General Public License (GPL)} reicht für die Software nicht aus, da das Programm auch mittels eines Servers betrieben werden kann und soll.
Damit das Projekt gegebenenfalls durch verschiedene Entwickler gleichzeitig bearbeitet werden kann und wegen des Konfigurationsmanagements wurde es als ein GitHub Projekt erstellt (siehe~\cite{bib:ASBA}).

Wenn die Lizenzen nicht mitgeliefert wurden, können sie unter \url{http://www.gnu.org/licenses/} gefunden werden.

\section{Anforderungen}% =======================================================
\beginsection{Anforderungen}
\label{sec:Anforderungen}

Die Anforderungen ergeben sich zunächst aus den Zielen \vrefinsec{sec:Ziele}.
Die beiden Ziele~\ref{Ziel:Daten}~\emph{Daten} und~\ref{Ziel:Lizenz}~\emph{Lizenz} sind für die Entwicklung von \ASBA\ von sekundärer Bedeutung und werden daher in diesem \sectionname\ nicht übernommen.
Die anderen Ziele werden noch verfeinert.

\todo{Ziele aus Abschnitt "'Ziele"' in Anforderungen umwandeln.}%%%
%TODO Ziele aus Abschnitt "'Ziele"' in Anforderungen umwandeln. %%%
%
\begin{enumerate}

	\item \label{Anforderung:Form} \emph{Form}:
	Die Daten liegt in formaler, geprüfter Form vor.
	(\vrefseeziel{Ziel:Form})

	\item \label{Anforderung:Eingaben} \emph{Eingaben}:
	Die Eingabe von Daten erfolgt in einer formalen Syntax unter Verwendung der üblichen mathematischen Schreibweise.
	Folgende Daten können eingegeben werden:
	\begin{enumerate}
		\item \Axiome
		\item \Saetze
		\item \Beweise
		\item \Fachbegriffe
		\item \Fachgebiete
		\item \Ausgabeschemata
	\end{enumerate}
	Dabei sind alle Begriffe nur innerhalb eines Fachgebiets und seiner untergeordneten \Fachgebiete\ gültig, solange sie nicht umdefiniert werden.
	Das oberste \Fachgebiet\ ist die ganze Mathematik.
	-- \vrefseeziel{Ziel:Eingaben}

	\item \label{Anforderung:Prüfung} \emph{Prüfung}:
	Vorhandene \Beweise\ können automatisch geprüft werden.
	-- \vrefseeziel{Ziel:Prüfung}

	\item \label{Anforderung:Ausgaben} \emph{Ausgaben}:
	Die Ausgabe kann in einer eindeutigen, formalen Syntax gemäß vorhandener \Ausgabeschemata\ erfolgen.
	-- \vrefseeziel{Ziel:Ausgaben} - \emph{Ausgabe in polnischer Notation}

	\item \label{Anforderung:Auswertungen} \emph{Auswertungen}:
	Zusätzlich zur Ausgabe der Daten sind verschiedene Auswertungen möglich.
	Insbesondere kann zu jedem \Beweis\ angegeben werden, wie lang er ist und welche \Axiome\ und Sätze%
	\footnote{Sätze, die quasi als \Axiome\ verwendet werden.}
	er benötigt.
	-- \vrefseeziel{Ziel:Auswertungen}

	\item \label{Anforderung:Anpassbarkeit} \emph{Anpassbarkeit}:
	\Fachbegriffe\ und die Darstellung bei der Ausgabe können mit Hilfe von -- gegebenenfalls unbenannten -- untergeordneten \Fachgebieten\ angepasst werden.
	-- \vrefseeziel{Ziel:Anpassbarkeit}

	\item \label{Anforderung:Individualität} \emph{Individualität}:
	\Axiome\ und Sätze können für jeden \Beweis\ individuell vorausgesetzt werden.
	Dabei sind fachgebietsspezifische \Fachbegriffe\ erlaubt.
	-- \vrefseeziel{Ziel:Individualität})

	\item \label{Anforderung:Internet} \emph{Internet}:
	Die Daten können auf mehrere Dateien verteilt sein.
	Ein Teil davon -- oder sogar alle -- können im Internet liegen.
	-- \vrefseeziel{Ziel:Internet}

	\item \label{Anforderung:Kommunikation} \emph{Kommunikation}:
	Die Kommunikation mit \ASBA\ kann mit den \Fachbegriffen\ der einzelnen \Fachgebiete\ erfolgen.
	-- \vrefseeziel{Ziel:Kommunikation}

	\item \label{Anforderung:Zugriff} \emph{Zugriff}:
	Der Zugriff auf \ASBA\ kann lokal und über das Internet erfolgen.
	-- \vrefseeziel{Ziel:Zugriff}

	\item \label{Anforderung:Unabhängigkeit} \emph{Unabhängigkeit}:
	\ASBA\ kann offline und online arbeiten.
	-- \vrefseeziel{Ziel:Unabhängigkeit}

	\item \label{Anforderung:Rekursion} \emph{Rekursion}:
	Es kann rekursiv über alle verwendeten Dateien -- auch solchen, die im Internet liegen -- ausgewertet werden.
	-- \vrefseeziel{Ziel:Rekursion}

	\item \label{Anforderung:Bedienbarkeit} \emph{Bedienbarkeit}:
	\ASBA\ ist einfach zu bedienen.
	-- \vrefseeziel{Ziel:Bedienbarkeit}

\end{enumerate}

\section{Axiome}% ==============================================================
\beginsection{\Axiome}
\label{sec:Axiome}
\todo{Axiome auswählen und definieren.}%%%
%TODO Axiome auswählen und definieren. %%%

\section{Beweise}% =============================================================
\beginsection{\Beweise}
\label{sec:Beweise}
\todo{Schlussregeln auswählen und Beweise definieren.}%%%
%TODO Schlussregeln auswählen und Beweise definieren. %%%

\section{Datenstruktur}% =======================================================
\beginsection{Datenstruktur}
\label{sec:Datenstruktur}
\todo{Datenstruktur abstrakt und in XML definieren.}%%%
%TODO Datenstruktur abstrakt und in XML definieren. %%%

\section{Bausteine}% ===========================================================
\beginsection{Bausteine}
\label{sec:Bausteine}
\todo{Bausteine? definieren.}%%%
%TODO Bausteine? definieren. %%%

\Endchapter
