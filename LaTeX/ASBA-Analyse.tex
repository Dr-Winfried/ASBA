%%############################################################################%%
%%                                                                            %%
%% Datei:  ASBA-Analyse.tex                                                   %%
%% Inhalt: Kapitel "Analyse"                                                  %%
%%                                                                            %%
%% Copyright (C) 2017  Winfried Teschers                                      %%
%%                                                                            %%
%% This program is free software: you can redistribute it and/or modify       %%
%% it under the terms of the GNU Affero General Public License as published   %%
%% by the Free Software Foundation, either version 3 of the License, or       %%
%% (at your option) any later version.                                        %%
%%                                                                            %%
%% This program is distributed in the hope that it will be useful,            %%
%% but WITHOUT ANY WARRANTY; without even the implied warranty of             %%
%% MERCHANTABILITY or FITNESS FOR A PARTICULAR PURPOSE.  See the              %%
%% GNU Affero General Public License for more details.                        %%
%%                                                                            %%
%% You should have received a copy of the GNU Affero General Public License   %%
%% along with this program.  If not, see <http://www.gnu.org/licenses/>.      %%
%%                                                                            %%
%% Dr. Winfried Teschers                                                      %%
%% Anton-Günther-Straße 26c                                                   %%
%% 91083 Baiersdorf                                                           %%
%% Germany                                                                    %%
%%                                                                            %%
%% e-mail: winfried.teschers@t-online.de                                      %%
%%                                                                            %%
%%############################################################################%%

% !TeX root = ASBA.tex
% !TeX encoding = UTF-8
% !TeX spellcheck = de_DE

\chapter{Analyse}% #############################################################
\beginchapter{Analyse}
\label{cha:Analyse}

In der Mathematik gibt es eine unüberschaubare Menge an \glsIdxPl{Axiom}n, \glsIdxPl{Satz}n, \glsIdxPl{Beweis}n, \emph{\glsIdxPl{Fachbegriff}n}%
\footnote{%
	\emph{Fachbegriffe} sind Namen für mathematische Elemente und Konstruktionen, \textzB\ \glsIdxPl{Axiom}, \glsIdxPl{Satz}, \glsIdxPl{Beweis} und \glsIdxPl{Fachgebiet}.
	\emph{Symbole} können als spezielle \glsIdxPl{Fachbegriff} aufgefasst werden.
}
und \emph{\glsIdxPl{Fachgebiet}n}%
\footnote{%
	Ein \emph{\glsIdx{Fachgebiet}} ist ein Teil der Mathematik mit einer zugehörigen Basis an \glsIdxPl{Axiom}n, \glsIdxPl{Satz}n, \glsIdxPl{Fachbegriff}n und Darstellungen, \textzB\ \emph{Logik}, \emph{Mengenlehre} und \emph{Gruppentheorie}.
	Ein \glsIdx{Fachgebiet} kann sehr klein sein und im Extremfall kein einziges Element enthalten.
	\emph{Umgebung} wäre eine bessere Bezeichnung, ist aber schon ein verbreiteter \glsIdx{Fachbegriff}, so dass hier die Bezeichnung \glsIdx{Fachgebiet} verwendet wird.

	Statt \enquote{\glsIdx{Fachgebiet}} könnte man auch \enquote{Theorie} nehmen.
	An Theorien werden aber bestimmte Anforderungen gestellt, die aber nicht notwendigerweise überprüft werden sollen.
	Theorien sind allerdings \textiAlg\ auch \glsIdxPl{Fachgebiet}.
}.
Zu den meisten \glsIdxPl{Fachgebiet}n gibt es noch ungelöste Probleme.

Es fehlt ein System, das einen Überblick bietet und die Möglichkeit, \glsIdxPl{Beweis} automatisch zu überprüfen.
Außerdem sollte all dies in üblicher mathematischer Schreibweise ein- und ausgegeben werden können.
In diesem Dokument werden die Grundlagen für das zu entwickelnde \ASBA\ behandelt.

Ein Programmsystem mit ähnlicher Aufgabenstellung findet sich im GitHub Projekt \emph{Hilbert~II} (\seename~\cite{bib:HilbertII, bib:qedeq}).
Einige Ideen sind von dort übernommen worden.

\section{Fragen}% ==============================================================
\beginsection{Fragen}
\label{sec:Fragen}

Einige der Fragen, die in diesem Zusammenhang auftauchen,
werden nun formuliert:
\begin{enumerate}
	%
	\item \label{Frage:Grundlagen} \emph{Grundlagen}:
	Was sind die Grundlagen?
	\textZB\ welche Logik und Mengenlehre.
	%
	\item \label{Frage:Basis} \emph{Basis}:
	Welche wichtigen \glsIdxPl{Axiom}, \glsIdxPl{Satz}, \glsIdxPl{Beweis}, \glsIdxPl{Fachbegriff} und \glsIdxPl{Fachgebiet} gibt es?
	Welche davon sind Standard?
	%
	\item \label{Frage:Axiome} \emph{\glsIdxPl{Axiom}}:
	Welche \glsIdxPl{Axiom} werden bei einem \glsIdx{Satz} oder \glsIdx{Beweis} vorausgesetzt?
	Allgemein anerkannte oder auch strittige, wie \textzB\ den \emph{\glsIdx{Satz} vom ausgeschlossenen Dritten} (\emph{tertium non datur}) oder das \emph{Auswahlaxiom}.
	%
	\item \label{Frage:Beweis} \emph{\glsIdx{Beweis}}:
	Ist ein \glsIdx{Beweis} fehlerfrei?
	%
	\item \label{Frage:Konstruktion} \emph{Konstruktion}:
	Gibt es einen konstruktiven \glsIdx{Beweis}?
	%
	\item \label{Frage:Vergleiche} \emph{Vergleiche}:
	Welcher \glsIdx{Beweis} ist besser?
	Nach welchem Kriterium?
	\textZB\ elegant, kurz, einsichtig oder wenige \glsIdxPl{Axiom}.
	Was heißt eigentlich \emph{elegant}?
	%
	\item \label{Frage:Definitionen} \emph{Definitionen}:
	Was ist mit einem \glsIdx{Fachbegriff} jeweils genau gemeint?
	\textZB\ \emph{Stetigkeit}, \emph{Integral} und \emph{Analysis}.
	%
	\item \label{Frage:Abhängigkeiten} \emph{Abhängigkeiten}:
	Wie heißt ein \glsIdx{Fachbegriff} in einer anderen Sprache?
	Ist wirklich dasselbe gemeint?
	Was ist mit \glsIdxPl{Fachbegriff}n in verschiedenen \glsIdxPl{Fachgebiet}n?
	%
	\item \label{Frage:Überblick} \emph{Überblick}:
	Ist ein \glsIdx{Axiom}, \glsIdx{Satz}, \glsIdx{Beweis} oder \glsIdx{Fachbegriff} schon einmal -- \textggf\ abweichend -- definiert, formuliert oder bewiesen worden?
	%
	\item \label{Frage:Darstellung} \emph{Darstellung}:
	Wie kann man einen \glsIdx{Satz} und den zugehörigen \glsIdx{Beweis} -- \textggf\ auch spezifisch für ein \glsIdx{Fachgebiet} -- darstellen?
	%
	\item \label{Frage:Forschung} \emph{Forschung}:
	Welche Probleme gibt es noch zu erforschen.
	%
\end{enumerate}

\section{Eigenschaften}% =======================================================
\beginsection{Eigenschaften}
\label{sec:Eigenschaften}

\ASBA\ soll ausgehend von den Fragen in \vrefsec{sec:Fragen} entwickelt werden, und die folgenden Eigenschaften haben:
\begin{enumerate}
	%
	\item \label{Eigenschaft:Daten} \emph{Daten}:
	\glsIdxPl{Axiom}, \glsIdxPl{Satz}, \glsIdxPl{Beweis}, \glsIdxPl{Fachbegriff} und \glsIdxPl{Fachgebiet} können in formaler Form gespeichert werden -- auch nicht oder unvollständig bewiesene \glsIdxPl{Satz}.
	Dabei soll die übliche mathematische Schreibweise verwendet werden können.
	%
	\item \label{Eigenschaft:Definitionen} \emph{Definitionen}:
	Es können \glsIdxPl{Fachbegriff} für \glsIdxPl{Axiom}, \glsIdxPl{Satz}, \glsIdxPl{Beweis} und \glsIdxPl{Fachgebiet} -- letztere mit eigenen \glsIdxPl{Axiom}n, \glsIdxPl{Satz}n, \glsIdxPl{Beweis}n, \glsIdxPl{Fachbegriff}n und über- oder untergeordneten \glsIdxPl{Fachgebiet}n -- definiert werden.
	Die Definitionen dürfen wiederum an dieser Stelle schon bekannte \glsIdxPl{Fachbegriff} und \glsIdxPl{Fachgebiet} verwenden.
	%
	\item \label{Eigenschaft:Prüfung} \emph{Prüfung}:
	Vorhandene \glsIdxPl{Beweis} können automatisch geprüft werden.
	%
	\item \label{Eigenschaft:Ausgaben} \emph{Ausgaben}:
	Die \glsIdxPl{Axiom}, \glsIdxPl{Satz} und \glsIdxPl{Beweis} können in üblicher Schreibweise -- abhängig von Sprache und \glsIdx{Fachgebiet} -- ausgegeben werden.
	%
	\item \label{Eigenschaft:Auswertungen} \emph{Auswertungen}:
	Zusätzlich zur Ausgabe der gespeicherten Daten sind verschiedene Auswertungen möglich, unter anderem für die meisten der unter \vrefsec{sec:Fragen} behandelten Fragen.
	%
	\setcounter{Enumi}{\value{enumi}}% Nummerierung wird fortgesetzt.
\end{enumerate}
%
Damit \ASBA\ nicht umsonst erstellt wird und möglichst breite Verwendung findet, werden noch zwei Punkte angefügt:
\begin{enumerate}
	\setcounter{enumi}{\value{Enumi}}% Nummerierung wird fortgesetzt.
	%
	\item \label{Eigenschaft:Lizenz} \emph{Lizenz}:
	Die Software ist \emph{Open Source}.
	%
	\item \label{Eigenschaft:Akzeptanz} \emph{Akzeptanz}:
	\ASBA\ wird von Mathematikern akzeptiert und verwendet.
\end{enumerate}
%
\vreftab{tab:Fragen->Eigenschaften} zeigt, wie sich die Eigenschaften zu den Fragen \vrefinsec{sec:Fragen} verhalten.
Mit einem X werden die Spalten einer Zeile markiert, deren zugehörige Eigenschaften zur Beantwortung der entsprechenden Frage beitragen sollen.
Idealerweise sollte die Erfüllung aller angegebenen Eigenschaften alle gestellten Fragen beantworten, was allerdings illusorisch ist.
%
% Abstände für die nächsten drei Tabellen
\newcommand*{\vsL}{\hspace{-1.0cm}}  % für 1-stellige Zahlen
\newcommand*{\vsl}{\hspace{-6pt}\vsL}% für 2-stellige Zahlen
\newcommand*{\vsc}{\hspace{6pt}}     % für die gedrehten Überschriften
%
\begin{table}[!htb]
	\begin{tabularx}{\linewidth-10.95pt}
		{@{\hspace{.5cm}}rl@{\extracolsep{\fill}}|*{7}{c}@{\hspace{1cm}}|}
		\multicolumn{2}{l|}{\diagbox[height=3.0cm,width=4.5cm]%
			{\textbf{Frage}\\~}{\\\textbf{Eigenschaft}}}
		&\rotatebox{90}{%
			\mbox{\vsL\ref{Eigenschaft:Daten}        \vsc Daten        }}
		&\rotatebox{90}{%
			\mbox{\vsL\ref{Eigenschaft:Definitionen} \vsc Definitionen }}
		&\rotatebox{90}{%
			\mbox{\vsL\ref{Eigenschaft:Prüfung}      \vsc Prüfung      }}
		&\rotatebox{90}{%
			\mbox{\vsL\ref{Eigenschaft:Ausgaben}     \vsc Ausgaben     }}
		&\rotatebox{90}{%
			\mbox{\vsL\ref{Eigenschaft:Auswertungen} \vsc Auswertungen }}
		&\rotatebox{90}{%
			\mbox{\vsL\ref{Eigenschaft:Lizenz}       \vsc Lizenz       }}
		&\rotatebox{90}{%
			\mbox{\vsL\ref{Eigenschaft:Akzeptanz}    \vsc Akzeptanz    }}
		\\\hline
		\ref{Frage:Grundlagen}      & Grundlagen
		& X & X & - & X & X & - & - \\
		\ref{Frage:Basis}           & Basis
		& X & X & - & X & X & - & - \\
		\ref{Frage:Axiome}          & \glsIdxPl{Axiom}
		& X & X & - & X & X & - & - \\
		\hdashline[2pt/2pt]
		\ref{Frage:Beweis}          & \glsIdx{Beweis}
		& X & - & X & X & - & - & - \\
		\ref{Frage:Konstruktion}    & Konstruktion
		& X & - & - & X & - & - & - \\
		\ref{Frage:Vergleiche}      & Vergleiche
		& X & - & - & - & X & - & - \\
		\hdashline[2pt/2pt]
		\ref{Frage:Definitionen}    & Definitionen
		& X & X & - & X & - & - & - \\
		\ref{Frage:Abhängigkeiten}  & Abhängigkeiten
		& X & - & - & X & - & - & - \\
		\ref{Frage:Überblick}       & Überblick
		& X & - & - & - & X & - & - \\
		\hdashline[2pt/2pt]
		\ref{Frage:Darstellung}     & Darstellung
		& - & X & - & X & - & - & - \\
		\ref{Frage:Forschung}       & Forschung
		& X & - & - & - & X & - & - \\
		\hline
	\end{tabularx}
	\caption{\ref{sec:Fragen} Fragen
		$\to$ \ref{sec:Eigenschaften} Eigenschaften}
	\label{tab:Fragen->Eigenschaften}
\end{table}

\section{Ziele}% ===============================================================
\beginsection{Ziele}
\label{sec:Ziele}

Um die Eigenschaften von \vrefsec{sec:Eigenschaften} zu erreichen, werden für \ASBA\ die folgenden Ziele%
\footnote{%
	Es sind eigentlich Anforderungen.
	Da dieser Begriff auch \vrefincha{cha:Design} verwendet wird, werden die Anforderungen hier \emph{\Idx{Ziel}e} genannt.
}
gesetzt:
\begin{enumerate}
	%
	\item \label{Ziel:Daten} \emph{Daten}:
	Es enthält möglichst viele wichtige \glsIdxPl{Axiom}, \glsIdxPl{Satz}, \glsIdxPl{Beweis}, \glsIdxPl{Fachbegriff}, \glsIdxPl{Fachgebiet} und \glsIdxPl{Ausgabeschema}%
	\footnote{%
		Um den Punkt~\ref{Eigenschaft:Ausgaben} \vrefvonsec{sec:Eigenschaften} erfüllen zu können, werden noch fachgebietsspezifische \glsIdxPl{Ausgabeschema} benötigt, welche die Art der Ausgaben beschreiben.
	}.
	%
	\item \label{Ziel:Form} \emph{Form}:
	Die Datenn liegt in formaler, geprüfter Form vor.
	%
	\item \label{Ziel:Eingaben} \emph{Eingaben}:
	Die Eingabe von Daten erfolgt in einer formalen Syntax unter Verwendung der üblichen mathematischen Schreibweise.
	%
	\item \label{Ziel:Prüfung} \emph{Prüfung}:
	Vorhandene \glsIdxPl{Beweis} können automatisch geprüft werden.
	%
	\item \label{Ziel:Ausgaben} \emph{Ausgaben}:
	Die Ausgabe kann in einer eindeutigen, formalen Syntax gemäß vorhandener \glsIdxPl{Ausgabeschema} erfolgen.
	%
	\item \label{Ziel:Auswertungen} \emph{Auswertungen}:
	Zusätzlich zur Ausgabe der Daten sind verschiedene Auswertungen möglich.
	Insbesondere kann zu jedem \glsIdx{Beweis} angegeben werden, wie lang er ist und welche \glsIdxPl{Axiom} und \glsIdxPl{Satz}%
	\footnote{
		\glsIdxPl{Satz}, die quasi als \glsIdxPl{Axiom} verwendet werden.
	}
	er benötigt.
	%
	\item \label{Ziel:Anpassbarkeit} \emph{Anpassbarkeit}:
	\glsIdxPl{Fachbegriff} und die Darstellung bei der Ausgabe können mit Hilfe von -- gegebenenfalls unbenannten -- untergeordneten \glsIdx{Fachgebiet}en angepasst werden.
	%
	\item \label{Ziel:Individualität} \emph{Individualität}:
	\glsIdxPl{Axiom} und \glsIdxPl{Satz} können für jeden \glsIdx{Beweis} individuell vorausgesetzt werden.
	Dabei sind fachgebietsspezifische \glsIdxPl{Fachbegriff} erlaubt.
	%
	\item \label{Ziel:Internet} \emph{Internet}:
	Die Daten können auf mehrere Dateien verteilt sein.
	Ein Teil davon -- oder sogar alle -- können im Internet liegen.
	%
	\item \label{Ziel:Kommunikation} \emph{Kommunikation}:
	Die Kommunikation mit \ASBA\ kann mit den \glsIdxPl{Fachbegriff}n der einzelnen \glsIdxPl{Fachgebiet} erfolgen.
	%
	\item \label{Ziel:Zugriff} \emph{Zugriff}:
	Der Zugriff auf \ASBA\ kann lokal und über das Internet erfolgen.
	%
	\item \label{Ziel:Unabhängigkeit} \emph{Unabhängigkeit}:
	\ASBA\ kann online und offline arbeiten.
	%
	\item \label{Ziel:Rekursion} \emph{Rekursion}:
	Es kann rekursiv über alle verwendeten Dateien -- auch solchen, die im Internet liegen -- ausgewertet werden.
	%
	\item \label{Ziel:Bedienbarkeit} \emph{Bedienbarkeit}:
	\ASBA\ ist einfach zu bedienen.
	%
	\item \label{Ziel:Lizenz} \emph{Lizenz}:
	Die Software ist \emph{Open Source}.
	%
	\setcounter{Enumi}{\value{enumi}}% Nummerierung wird fortgesetzt.
\end{enumerate}
%
Der Punkt \ref{Ziel:Zwischenspeicher} wurde noch eingefügt, damit \ASBA\ effizient arbeiten kann und um die Akzeptanz zu erhöhen:
\begin{enumerate}
	\setcounter{enumi}{\value{Enumi}}% Nummerierung wird fortgesetzt.
	%
	\item \label{Ziel:Zwischenspeicher} \emph{Zwischenspeicher}:
	Wichtige Auswertungen können an vorhandenen Dateien angehängt oder separat in eigenen Dateien gespeichert werden.
	%
\end{enumerate}
%
\vrefDtab{tab:Eigenschaften->Ziele} zeigt wieder, wie sich die Ziele zu den Eigenschaften \vrefinsec{sec:Eigenschaften} verhalten.
Mit einem X werden wieder die Spalten einer Zeile markiert, deren zugehörige Ziele zur Sicherstellung der entsprechenden Eigenschaft beitragen sollen.
Idealerweise sollte durch Erreichen aller aufgestellten Ziele \gls{ASBA} alle angegebenen Eigenschaften aufweisen, was wahrscheinlich ebenfalls illusorisch ist.
%
\begin{table}[!htb]
	\begin{tabularx}{\linewidth-10.95pt}
		{@{\hspace{.3cm}}rl@{\extracolsep{\fill}}|*{16}{c}@{\hspace{0.4cm}}|}
		\multicolumn{2}{l|}{\diagbox[height=3.0cm,width=4.0cm]%
			{\textbf{Eigenschaft}\\~}{\\\\\\\textbf{Ziel}}}
		&\rotatebox{90}{%
			\mbox{\vsL\ref{Ziel:Daten}            \vsc Daten            }}
		&\rotatebox{90}{%
			\mbox{\vsL\ref{Ziel:Form}             \vsc Form             }}
		&\rotatebox{90}{%
			\mbox{\vsL\ref{Ziel:Eingaben}         \vsc Eingaben         }}
		&\rotatebox{90}{%
			\mbox{\vsL\ref{Ziel:Prüfung}          \vsc Prüfung          }}
		&\rotatebox{90}{%
			\mbox{\vsL\ref{Ziel:Ausgaben}         \vsc Ausgaben         }}
		&\rotatebox{90}{%
			\mbox{\vsL\ref{Ziel:Auswertungen}     \vsc Auswertungen     }}
		&\rotatebox{90}{%
			\mbox{\vsL\ref{Ziel:Anpassbarkeit}    \vsc Anpassbarkeit    }}
		&\rotatebox{90}{%
			\mbox{\vsL\ref{Ziel:Individualität}   \vsc Individualität   }}
		&\rotatebox{90}{%
			\mbox{\vsL\ref{Ziel:Internet}         \vsc Internet         }}
		&\rotatebox{90}{%
			\mbox{\vsl\ref{Ziel:Kommunikation}    \vsc Kommunikation    }}
		&\rotatebox{90}{%
			\mbox{\vsl\ref{Ziel:Zugriff}          \vsc Zugriff          }}
		&\rotatebox{90}{%
			\mbox{\vsl\ref{Ziel:Unabhängigkeit}   \vsc Unabhängigkeit   }}
		&\rotatebox{90}{%
			\mbox{\vsl\ref{Ziel:Rekursion}        \vsc Rekursion        }}
		&\rotatebox{90}{%
			\mbox{\vsl\ref{Ziel:Bedienbarkeit}    \vsc Bedienbarkeit    }}
		&\rotatebox{90}{%
			\mbox{\vsl\ref{Ziel:Lizenz}           \vsc Lizenz           }}
		&\rotatebox{90}{%
			\mbox{\vsl\ref{Ziel:Zwischenspeicher} \vsc Zwischenspeicher }}
		\\\hline
		\ref{Eigenschaft:Daten}        & Daten%
		& X & X & X & - & - & - & - & - & - & - & - & - & - & - & - &-\\
		\ref{Eigenschaft:Definitionen} & Definitionen%
		& X & - & X & - & - & - & - & - & - & - & - & - & - & - & - &-\\
		\ref{Eigenschaft:Prüfung}      & Prüfung
		& - & - & - & X & - & - & - & - & - & - & - & - & - & - & - &-\\
		\hdashline[2pt/2pt]
		\ref{Eigenschaft:Ausgaben}     & Ausgaben%
		& - & - & - & - & X & - & - & - & - & - & - & - & - & - & - &-\\
		\ref{Eigenschaft:Auswertungen} & Auswertungen%
		& - & - & - & - & - & X & - & - & - & - & - & - & - & - & - &-\\
		\ref{Eigenschaft:Lizenz}       & Lizenz%
		& - & - & - & - & - & - & - & - & - & - & - & - & - & - & X &-\\
		\hdashline[2pt/2pt]
		\ref{Eigenschaft:Akzeptanz}    & Akzeptanz%
		& X & X & X & X & X & X & X & X & X & X & X & X & X & X & X &X\\
		\hline
	\end{tabularx}
	\caption{\ref{sec:Eigenschaften}
		Eigenschaften $\to$ \ref{sec:Ziele} Ziele}
	\label{tab:Eigenschaften->Ziele}% Erst nach '\caption'!
\end{table}

\section{Zusammenfassung}% =====================================================
\beginsection{Zusammenfassung}
\label{sec:Zusammenfassung}

\begin{table}[!htb]
	\begin{tabularx}{\linewidth-10.95pt}
		{@{\hspace{.3cm}}rl@{\extracolsep{\fill}}|*{15}{c}@{\hspace{0.4cm}}|}
		\multicolumn{2}{l|}{\diagbox[height=3.0cm,width=4.0cm]%
			{\textbf{Frage}\\~}{\\\\\\\textbf{Ziel}}}
		&\rotatebox{90}{%
			\mbox{\vsL\ref{Ziel:Daten}            \vsc Daten            }}
		&\rotatebox{90}{%
			\mbox{\vsL\ref{Ziel:Form}             \vsc Form             }}
		&\rotatebox{90}{%
			\mbox{\vsL\ref{Ziel:Eingaben}         \vsc Eingaben         }}
		&\rotatebox{90}{%
			\mbox{\vsL\ref{Ziel:Prüfung}          \vsc Prüfung          }}
		&\rotatebox{90}{%
			\mbox{\vsL\ref{Ziel:Ausgaben}         \vsc Ausgaben         }}
		&\rotatebox{90}{%
			\mbox{\vsL\ref{Ziel:Auswertungen}     \vsc Auswertungen     }}
		&\rotatebox{90}{%
			\mbox{\vsL\ref{Ziel:Anpassbarkeit}    \vsc Anpassbarkeit    }}
		&\rotatebox{90}{%
			\mbox{\vsL\ref{Ziel:Individualität}   \vsc Individualität   }}
		&\rotatebox{90}{%
			\mbox{\vsL\ref{Ziel:Internet}         \vsc Internet         }}
		&\rotatebox{90}{%
			\mbox{\vsl\ref{Ziel:Kommunikation}    \vsc Kommunikation    }}
		&\rotatebox{90}{%
			\mbox{\vsl\ref{Ziel:Zugriff}          \vsc Zugriff          }}
		&\rotatebox{90}{%
			\mbox{\vsl\ref{Ziel:Unabhängigkeit}   \vsc Unabhängigkeit   }}
		&\rotatebox{90}{%
			\mbox{\vsl\ref{Ziel:Rekursion}        \vsc Rekursion        }}
		&\rotatebox{90}{%
			\mbox{\vsl\ref{Ziel:Bedienbarkeit}    \vsc Bedienbarkeit    }}
		&\rotatebox{90}{%
			\mbox{\vsl\ref{Ziel:Lizenz}           \vsc Lizenz           }}
		\\\hline
		\ref{Frage:Grundlagen}      & Grundlagen%
		& X & X & X & - & X & X & x & - & - & - & - & - & - & - & - \\
		\ref{Frage:Basis}           & Basis%
		& X & X & X & - & X & X & x & x & - & - & - & - & - & - & - \\
		\ref{Frage:Axiome}          & \glsIdxPl{Axiom}%
		& X & X & X & - & X & X & x & - & - & - & - & - & - & - & - \\
		\hdashline[2pt/2pt]
		\ref{Frage:Beweis}          & \glsIdx{Beweis}%
		& X & X & X & X & X & - & - & x & - & - & - & - & - & - & - \\
		\ref{Frage:Konstruktion}    & Konstruktion%
		& X & X & X & - & X & - & - & x & - & - & - & - & - & - & - \\
		\ref{Frage:Vergleiche}      & Vergleiche%
		& X & X & X & - & - & X & - & x & - & - & - & - & - & - & - \\
		\hdashline[2pt/2pt]
		\ref{Frage:Definitionen}    & Definitionen%
		& X & X & X & - & X & - & x & - & - & - & - & - & - & - & - \\
		\ref{Frage:Abhängigkeiten}  & Abhängigkeiten%
		& X & X & X & - & X & - & x & - & - & - & - & - & - & - & - \\
		\ref{Frage:Überblick}       & Überblick%
		& X & X & X & - & - & X & x & - & - & - & - & - & - & - & - \\
		\hdashline[2pt/2pt]
		\ref{Frage:Darstellung}     & Darstellung%
		& X & - & X & - & X & - & x & - & - & - & - & - & - & - & - \\
		\ref{Frage:Forschung}       & Forschung%
		& X & X & X & - & - & X & x & - & - & - & - & - & - & - & - \\
		\hline
		\multicolumn{17}{l|}{Die nächsten beiden Punkte
			sind Eigenschaften aus \vrefsec{sec:Eigenschaften}:}\\
		\hline
		\ref{Eigenschaft:Lizenz}    & Lizenz%
		& - & - & - & - & - & - & - & - & - & - & - & - & - & - & X \\
		\ref{Eigenschaft:Akzeptanz} & Akzeptanz%
		& X & X & X & X & X & X & X & X & X & X & X & X & X & X & X \\
		\hline
	\end{tabularx}
	\caption{\ref{sec:Fragen} Fragen $\to$ \ref{sec:Ziele} Ziele}
	\label{tab:Fragen->Ziele}% Erst nach '\caption'!
\end{table}
%
\vrefDtab{tab:Fragen->Ziele} ist eine Kombination der Tabellen~ \ref{tab:Fragen->Eigenschaften} und~\ref{tab:Eigenschaften->Ziele} und zeigt, wie sich die Ziele \vrefinsec{sec:Ziele} zu den Fragen \vrefinsec{sec:Fragen} verhalten.
Auch hier werden mit einem X die Spalten einer Zeile markiert, deren zugehörige Ziele für die Beantwortung der entsprechenden Frage nötig sind.
Mit einem kleinen x werden sie markiert, wenn sie zur Beantwortung der Fragen nicht nötig, aber von Interesse sind.
Idealerweise sollte das Erreichen aller aufgestellten Ziele alle gestellten Fragen beantworten, was natürlich auch illusorisch ist.

\clearpage

\section{Die Umgebung von \textsf{ASBA}}% ======================================
\beginsection{Die Umgebung von \textsf{ASBA}}
\label{sec:Umgebung}

\vrefInfig{fig:Umgebung} wird beschrieben, welche Interaktionen \ASBA\ mit der Umgebung hat, \textdh\ welche Ein- und Ausgaben existieren und woher sie kommen \textbzw\ wohin sie gehen.
%
\begin{figure}[!htb]
	\setlength\unitlength{1cm}
	\begin{picture}(17.0,7.0)(-8.4,-3.5)
	%%%			% Gitter                                            %%%
	%%%			\color{lightgray}%                                  %%%
	%%%			\multiput(-8.4,-3.0)(+0.0,1.0){ 7}{\line(1,0){17.0}}%%%
	%%%			\multiput(-7.9,-3.5)(+1.0,0.0){17}{\line(0,1){ 7.0}}%%%
	\linethickness{1.5pt}
	% Hintergrund (grau) ===============================================
	\color{gray}
	% rechts: externes ASBA mit Pfeilen --------------------------------
	\put(+3.00,-0.50){\framebox(2.40,1.60){\huge\textbf{\ASBA}}}
	\put(+3.00,-0.50){\makebox(2.40,1.50)[t]{\textbild{externes}}}
	\put(+4.00,+1.12){\vector(-1,+2){0.45}}% <--- externes ASBA
	\put(+4.00,+2.05){\vector(+1,-2){0.47}}% ---> externes ASBA
	\put(+3.81,+1.57){\marker{a}}
	% rechts oben: externe Datenbank mit Pfeilen -----------------------
	\put(+7.30,+2.30){\Datenbank{1.20}{0.40}{0.80}{\small externe}{\small Datenbank}}
	\put(+5.41,+1.10){\vector(+1,+1){0.70}}% <--- externes ASBA
	\put(+6.14,+1.39){\vector(-1,-1){0.72}}% ---> externes ASBA
	\put(+5.60,+1.23){\marker{b}}
	% Verbindung Auswertungen ---> Männchen ----------------------------
	\put(+5.60,-2.30){\vector(-1,0){4.05}}%  Auswertungen ---> Männchen
	\put(+3.50,-2.30){\marker{c}}
	% Verbindung Männchen <---> Terminal -------------------------------
	\put(-2.00,-2.00){\vector(+1,0){2.45}}% Männchen <--- Terminal
	\put(+0.40,-2.30){\vector(-1,0){2.40}}% Männchen ---> Terminal
	\put(-0.75,-2.15){\marker{d}}
	% Verbindung Terminal <---> Datei ----------------------------------
	\put(-5.50,-1.50){\vector(+3,-1){1.50}}% Terminal <--- Datei
	\put(-4.01,-2.30){\vector(-3,+1){1.60}}% Terminal ---> Datei
	\put(-5.00,-1.90){\marker{e}}
	% Vordergrund (schwarz) ============================================
	\color{black}
	% rechts oben: Wolke mit Pfeilen -----------------------------------
	\put(+3.40,+3.00){\Wolke{Internet}}
	\put(+2.00,+2.54){\vector(-1,-1){0.98}}% ---> ASBA
	\put(+1.53,+1.53){\vector(+1,+1){0.75}}% <--- ASBA
	\put(+1.50,+1.95){\Marker{1}}
	% links oben: Datenbank mit Pfeilen --------------------------------
	\put(-7.00,+2.00){\Datenbank{1.50}{0.50}{1.00}{Datenbank}{}}
	\put(-5.50,+2.20){\vector(+4,-1){3.95}}% ---> ASBA
	\put(-1.51,+0.90){\vector(-4,+1){4.00}}% <--- ASBA
	\put(-3.70,+1.55){\Marker{2}}
	% links Mitte: Datei mit Pfeilen -----------------------------------
	\put(-7.00,-1.00){\Datei{3.00}{2.00}{Datei}}
	\put(-5.50,-0.80){\vector(+4,+1){3.95}}% ---> ASBA
	\put(-1.51,-0.10){\vector(-4,-1){4.00}}% <--- ASBA
	\put(-3.70,-0.45){\Marker{3}}
	%links unten: Rechner mit Pfeilen ----------------------------------
	\put(-3.00,-2.10){\Terminal{Terminal}}
	\put(-2.50,-1.48){\vector(+1,+1){0.95}}% ---> ASBA
	\put(-1.10,-0.52){\vector(-1,-1){0.95}}% <--- ASBA
	\put(-2.00,-1.00){\Marker{4}}
	% Mitte unten: Männchen mit Pfeilen --------------------------------
	\put(+1.00,-1.80){\Maennchen}
	\put(+0.85,-1.50){\vector(0,+1){1.00}}% ---> ASBA
	\put(+1.15,-0.51){\vector(0,-1){1.00}}% <--- ASBA
	\put(+0.80,-1.02){\Marker{5}}
	% rechts unten: Papier mit Pfeil -----------------------------------
	\put(+5.60,-3.00){\Papier{+2.00}{+0.30}{Ausgabe}}
	\put(+1.51,-0.55){\vector(+4,-1){+4.10}}% <--- ASBA
	\put(+3.25,-1.05){\Marker{6}}
	% Mitte: ASBA ------------------------------------------------------
	\linethickness{3pt}
	\put(-1.5,-0.5){\framebox(3.0,2.0){\Huge\textbf{\ASBA}}}
	\end{picture}
	\caption{Die Umgebung von \ASBA}
	\label{fig:Umgebung}% Erst nach '\caption'!
\end{figure}
%
In den \vrefinfig{fig:Umgebung} abgebildeten Datenflüssen (1) bis (6) und (a) bis (e) werden die folgenden Daten übertragen:
\begin{itemize}
	\newcommand*{\vonnach}  [2]{#1 $\rightarrow$ #2}
	\newcommand*{\nachvon}  [2]{\vonnach{#2}{#1}}
	\newcommand*{\hinundher}[2]{#1 $\leftrightarrow$ #2}
	%
	\item[(1)]\label{dat:Internet}
	\begin{description}
		\item[\vonnach{\ASBA}{Internet}]\label{dat:ausInternet}
		Inhalte der Datenbank.
		\item[\nachvon{\ASBA}{Internet}]\label{dat:inInternet}
		Inhalte der externen Datenbank.
	\end{description}
	%
	\item[(2)]\label{dat:Datenbank}
	\begin{description}
		\item[\vonnach{Datenbank}{\ASBA}]\label{dat:ausDatenbank}
		Inhalte der Datenbank und Antworten auf Datenbankanweisungen.
		\item[\nachvon{Datenbank}{\ASBA}]\label{dat:inDatenbank}
		Inhalte der Datei, der externen Datenbank und Datenbankanweisungen.
	\end{description}
	%
	\item[(3)]\label{dat:Datei}
	\begin{description}
		\item[\vonnach{Datei}{\ASBA}]\label{dat:ausDatei}
		Inhalte der Datei.
		\item[\nachvon{Datei}{\ASBA}]\label{dat:inDatei}
		Die Datei wird um zusätzliche Auswertungen ergänzt, \textzB\ ob die \glsIdxPl{Beweis} korrekt sind, welche \glsIdxPl{Axiom} und \glsIdxPl{Satz} -- auch externe aus dem Internet -- verwendet wurden, Länge des \glsIdx{Beweis}es usw.
	\end{description}
	%
	\item[(4)]\label{dat:Terminal}
	\begin{description}
		\item[\vonnach{Terminal}{\ASBA}]\label{dat:ausTerminal}
		Anweisungen, Daten und Batchprogramme.
		\item[\nachvon{Terminal}{\ASBA}]\label{dat:inTerminal}
		Antworten auf Anweisungen, Auswertungen usw.
	\end{description}
	Außerdem interaktive Ein- und Ausgabe durch einen Anwender, wie in (5) beschrieben.
	%
	\item[(5)]\label{dat:Anwender}
	\begin{description}
		\item[\hinundher{Anwender}{\ASBA}]\label{dat:mitAnwender}
		Interaktive Ein- und Ausgaben durch einen Anwender mit Komponenten von (3), (4) und (6).
		-- Die Kommunikation läuft \textiAlg\ über ein Terminal.
	\end{description}
	%
	\item[(6)]\label{dat:Ausgabe}
	\begin{description}
		\item[\nachvon{Ausgabe}{\ASBA}]\label{dat:inAusgabe}
		Inhalte von Datei und Datenbank in lesbarer Form, \textua\ mit Hilfe von \glsIdxPl{Ausgabeschema} auch in Formelschreibweise.
		Die Ausgabe kann auch in eine Datei erfolgen,
		\textzB\ im \LaTeX-Format.
	\end{description}
	%
	\item[(a)]\label{dat:extInternet}
	Nur die für \ASBA\ relevanten Daten:
	\begin{description}
		\item[\vonnach{Internet}{externes \ASBA}]\label{dat:ausextInternet}
		Inhalte der Datenbank.
		\item[\nachvon{Internet}{externes \ASBA}]\label{dat:inextInternet}
		Inhalte der externen Datenbank.
	\end{description}
	%
	\item[(b)]\label{dat:extDatenbank}
	Nur die für \ASBA\ relevanten Daten:
	\begin{description}
		\item[\vonnach{externe Datenbank}{externes \ASBA}]
		\label{dat:ausextDatenbank} Inhalte der externen Datenbank.
		\item[\nachvon{externe Datenbank}{externes \ASBA}]
		\label{dat:inextDatenbank} Inhalte der Datenbank.
	\end{description}
	%
	\item[(c)]\label{dat:AusgabeAnwender}
	\begin{description}
		\item[\vonnach{Ausgabe}{Anwender}]\label{dat:Ausgabe2Anwender}
		Alle Daten der Ausgabe.
	\end{description}
	%
	\item[(d)] \label{dat:AnwenderTerminal}
	\begin{description}
		\item[\hinundher{Anwender}{Terminal}]\label{dat:Anwender22Terminal}
		Interaktive Ein- und Ausgabe durch einen Anwender, wie in (5) beschrieben.
	\end{description}
	%
	\item[(e)] \label{dat:TerminalDatei}
	\begin{description}
		\item[\hinundher{Terminal}{Datei}]\label{dat:Terminal22Datei}
		Erstellen und Bearbeiten der Datei durch einen Anwender.
		-- siehe (d)
	\end{description}
	%
\end{itemize}
Die Datenflüsse (a) bis (e) erfolgen außerhalb von \ASBA
und werden nicht weiter behandelt.

Die Datenbank und die Datei enthalten im Prinzip die gleichen Daten, wobei sie in der Datei im Textformat in lesbarer Form und in der Datenbank in einem internen Format vorliegen.
Zudem enthält die Datenbank \textiAlg\ sehr viel mehr Daten. Es handelt sich dabei jeweils um die folgenden Daten:
\begin{description}
	%
	\item[\glsIdxPl{Axiom}]         \label{Daten:Axiom}
	Ein \glsIdx{Axiom} ist eine Aussage oder Behauptung, die nicht aus anderen Aussagen abgeleitet werden kann.
	Es können wie bei \glsIdxPl{Satz}n Voraussetzungen vorhanden sein, aber keine \glsIdx{Beweis}.
	%
	\item[\glsIdxPl{Satz}]          \label{Daten:Satz}
	Ein \glsIdx{Satz} besteht aus einer Anzahl von Voraussetzungen, einer Behauptung und einem \glsIdx{Beweis}, der die Behauptung aus den Voraussetzungen ableitet.
	Letztere können \glsIdxPl{Axiom} und andere \glsIdxPl{Satz} sein, auf die dann verwiesen wird.
	%
	\item[\glsIdxPl{Beweis}]        \label{Daten:Beweis}
	Ein \glsIdx{Beweis} besteht aus einer Folge von Beweisschritten, die aus gegebenen Voraussetzungen eine Behauptung ableitet.
	%
	\item[\glsIdxPl{Fachbegriff}]   \label{Daten:Fachbegriff}
	Ein \glsIdx{Fachbegriff} ist ein Name für ein Objekt \textbzw\ eine Eigenschaft in einem bestimmten \glsIdx{Fachgebiet}.
	%
	\item[\glsIdxPl{Fachgebiet}]    \label{Daten:Fachgebiet}
	Ein \glsIdx{Fachgebiet} ist ein Teil der Mathematik mit einer zugehörigen Basis von \glsIdxPl{Axiom}, \glsIdxPl{Satz}, \glsIdxPl{Fachbegriff} und \glsIdxPl{Ausgabeschema}, quasi eine untergeordnete Datenbank.
	%
	\item[\glsIdxPl{Ausgabeschema}] \label{Daten:Ausgabeschema}
	Eine Beschreibung, wie ein bestimmtes mathematisches Objekt ausgegeben werden soll.
	Dies kann \textzB\ ein Stück \LaTeX-Code mit entsprechenden Parametern sein.
	%
	\item[Auswertungen]             \label{Daten:Auswertung}
	Statistische und andere Auswertungen, die bestimmten Elementen der Datei \textbzw\ Datenbank zugeordnet sind.
	\textZB\ können zu einem \glsIdx{Satz} alle für einen \glsIdx{Beweis} notwendigen \glsIdxPl{Axiom} angegeben werden -- als Verweise.
	%
\end{description}
Die Daten können interne und externe Verweise enthalten.

\section{Basis von Beweisen}
\beginsection{Basis von Beweisen}
\label{sec:BeweisBasis}

Da ein Computerprogramm erstellt werden soll, muss die Grundstruktur des Vorgehens bei \glsIdxPl{Beweis}n definiert werden.%
\footnote{\seename~\cite{bib:Kalkuel}}

\begin{description}
	%
	\item[Die logische Darstellung] von mathematischen Aussagen, wozu auch \glsIdxPl{Axiom} und \glsIdxPl{Satz} gehören, erfolgt, da es sich immer um Formeln handelt, an besten mit \emph{Zeichenfolgen}%
	\footnote{%
		Die interne Darstellung der Zeichenfolgen kann zur Optimierung des Programms von der logischen abweichen.
	},
	\textdh\ von Folgen von Zeichen und Symbolen, in denen Zwischenraum -- insbesondere Leerstellen -- nicht zählen.
	Mehrdimensionale Formeln, wie \textzB\ Matrizen, Baumstrukturen, Funktionsschemata und anderes, können auch als (eindimensionale) Zeichenfolgen dargestellt werden.
	\glsIdxPl{Beweis} sind letztendlich nichts anderes, als erlaubte Transformationen dieser Zeichenfolgen.
	%
	\item[Bausteine] sind Grundelemente, auch \defn{Zeichen} oder \defn{(Satz-)Buchstaben} genannt, aus denen die Zeichenfolgen bestehen dürfen, und müssen definiert werden.
	%
	\item[Formationsregeln] dienen zur Festlegung, wie man aus den Bausteinen Ausdrücke erzeugen kann, und müssen ebenfalls definiert werden.
	%
	\item[\glsIdxPl{Satz}] lassen sich als eine Menge von Formeln, den Voraussetzungen, wozu auch \glsIdxPl{Axiom} gehören können, einer weiteren Menge von Formeln (Zeichenfolgen), den Folgerungen, und der Angabe eines \glsIdx{Beweis}es darstellen.
	%
	\item[\glsIdxPl{Beweis}] zu gegebenen Voraussetzungen und Folgerungen lassen sich als Folge von zulässigen Transformationen, beginnend mit den Voraussetzungen und endend mit den Folgerungen, darstellen.
	%
	\item[Transformationsregeln] definieren, welche Transformationen mit gegebenen Formelmengen zulässig sind.%
	\footnote{\seename~\cite{bib:Rautenberg,bib:Schlussregel,bib:NatuerlichesSchliessen}}
	%
\end{description}

\Endchapter
