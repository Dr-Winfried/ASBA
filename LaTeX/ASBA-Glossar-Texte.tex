%%############################################################################%%
%%                                                                            %%
%% Datei:  ASBA-Glossar-Texte.tex                                             %%
%% Inhalt: Vorspann Glossareinträge für ASBA                                  %%
%%                                                                            %%
%% Copyright (C) 2017  Winfried Teschers                                      %%
%%                                                                            %%
%% This program is free software: you can redistribute it and/or modify       %%
%% it under the terms of the GNU Affero General Public License as published   %%
%% by the Free Software Foundation, either version 3 of the License, or       %%
%% (at your option) any later version.                                        %%
%%                                                                            %%
%% This program is distributed in the hope that it will be useful,            %%
%% but WITHOUT ANY WARRANTY; without even the implied warranty of             %%
%% MERCHANTABILITY or FITNESS FOR A PARTICULAR PURPOSE.  See the              %%
%% GNU Affero General Public License for more details.                        %%
%%                                                                            %%
%% You should have received a copy of the GNU Affero General Public License   %%
%% along with this program.  If not, see <http://www.gnu.org/licenses/>.      %%
%%                                                                            %%
%% Dr. Winfried Teschers                                                      %%
%% Anton-Günther-Straße 26c                                                   %%
%% 91083 Baiersdorf                                                           %%
%% Germany                                                                    %%
%%                                                                            %%
%% e-mail: winfried.teschers@t-online.de                                      %%
%%                                                                            %%
%%############################################################################%%

% !TeX root = ASBA.tex
% !TeX encoding = UTF-8
% !TeX spellcheck = de_DE

% ### Glossar und Index ########################################################

% ==============================================================================
% \Txt* - Ausgabe als formatierter Text und Eintrag und Verweis ins Glossar
% Wahrheitswerte ===============================================================

\newcommand*             {\StrTxtFalse}            {falsch}
\newVerweis                 {\TxtFalse}{\glstext}{TxtFalse}
\newglossaryentry            {TxtFalse}{
	text       =         {\RawTxtFalse},
	name       =         {\RawTxtFalse \addIdx[
		name   =         {\RawTxtFalse},
		sort   ={falsch}]    {TxtFalse}},
	sort       ={falsch},%\StrTxtFalse
	see        ={TxtTrue,MtsFalse,OjkFalse},
	description={
		Ein \metasprachlicherWahrheitswert\ in Textform.
	}
}

\newcommand*               {\StrTxtTrue}           {wahr}
\newVerweis                   {\TxtTrue}{\glstext}{TxtTrue}
\newglossaryentry              {TxtTrue}{
	text       =           {\RawTxtTrue},
	name       =           {\RawTxtTrue \addIdx[
		name   =           {\RawTxtTrue},
		sort   ={wahr}]        {TxtTrue}},
	sort       ={wahr},%  \StrTxtTrue
	see        ={TxtFalse,MtsTrue,OjkTrue},
	description={
		Ein \metasprachlicherWahrheitswert\ in Textform.
	}
}

% ==============================================================================
% \* - Ausgabe als Text und Eintrag und Verweis ins Glossar
% Fachbegriffe =================================================================

\iftestFlg% Definition von Dummy Glossareinträgen
	\newVerweis     {\Dummy} {\glstext}{Dummy}
	\newglossaryentry{Dummy}{
		name        ={Dummy \addIdx    {Dummy}},
		text        ={Dummy},
		description ={
			\todo{Beschreibung fehlt noch}% ToDo=Dummy
		}
	}
	\newVerweis     {\dummyDummy} {\glstext}{dummyDummy}
	\newglossaryentry{dummyDummy}{
		name       =        {---, dummy \addIdx[
			name   =        {---, dummy},
			sort   =      {Dummy, dummy}]   {dummyDummy}},
		sort       =      {Dummy, dummy},
		text       ={dummy Dummy},
		description={
			\todo{Beschreibung fehlt noch}% ToDo=dummy Dummy
		}
	}
	\newVerweis         {\WikiDummi} {\glstext}{WikiDummi}
	\longnewglossaryentry{WikiDummi}{
		name            ={WikiDummi \addIdx    {WikiDummi}},
		text            ={WikiDummi},
	}{
		\begin{wikicite}{bib:Wikipedia}
			Inhalt...
		\end{wikicite}
		\todo{Beschreibung fehlt noch}% ToDo=WikiDummi
	}
\else\fi

% TODO ### ab hier Vorkommen prüfen und durch Makro ersetzen
% TODO ### ab hier Selbstreferenzen auflösen
%A === A === A === A === A === A === A === A === A === A === A === A === A === A

\newsynonym{\Abbildung}{Abbildung}{\Funktion}

\newVerweis     {\ableitbar} {\glstext}{ableitbar}
\newVerweis[e]  {\ableitbare}{\glstext}{ableitbar}
\newglossaryentry{ableitbar}{
	name        ={ableitbar \addIdx    {ableitbar}},
	text        ={ableitbar},
	see         ={Ableitungsrelation},
	description ={
		Wenn sich eine \Formel\ $\beta$ aus einer anderen \Formel\ $\alpha$ mittels \zulaessiger\ \Transformationen\ ableiten lässt, heißt $\beta$ \GloFt{ableitbar} aus $\alpha$.
		Sprechweise: $\alpha$ \GloFt{ableitbar}\synonym{\beweisbar} $\beta $.
		Eine oder beide \Formeln\ $\alpha$ \textbzw\ $\beta$ dürfen dabei durch \Formelmengen\ ersetzt werden.
	}
}

\newVerweis         {\Ableitung}  {\glstext}{Ableitung}
\newVerweis[en]     {\Ableitungen}{\glstext}{Ableitung}
\longnewglossaryentry{Ableitung}{
	name            ={Ableitung \addIdx     {Ableitung}},
	text            ={Ableitung},
	see             ={Ableitungsmenge,Ableitungsrelation,Konklusion,Logik,Praemisse,Schlussregel},
}{
	\begin{wikicite}{bib:Ableitung}
		Eine \wikibf{Ableitung}, \wikibf{Herleitung}, oder \wikilink{Deduktion} ist in der \wikilink{Logik} die Gewinnung von \wikilink{Aussagen} aus anderen Aussagen. Dabei werden \wikilink{Schlussregeln} auf \wikilink{Prämissen} angewandt, um zu \wikilink{Konklusionen} zu gelangen. Welche Schlussregeln dabei erlaubt sind, wird durch das verwendete \wikilink{Kalkül} bestimmt.

		Die Ableitung ist zusammen mit der \wikilink{semantischen Konklusion} einer der zwei logischen Methoden, um auf die Konklusion zu kommen.
	\end{wikicite}
	Eine Ableitung ist für \ASBA\ eine \Aussage\ $A \MtsDerive B$ \textbzw\ allgemeiner $A \MtsDeriveR B$ mit $A,B \MtsSubsetEq \MtsSprache$, wobei \MtsSprache\ eine \Sprache\ ist.
	Dies entspricht einem Element $(A,B)$ einer \Ableitungsrelation\ \MtsDerive\ \textbzw\ \MtsDeriveR\ (\textdh\ $(A,B) \in R_{\MtsIdxGraph}$.
	Die semantische Aussage ist die, das die \Formeln\ aus $B$ aus den \Formeln\ aus $A$ abgeleitet werden können.
}

\newVerweis     {\Ableitungsmenge} {\glstext}{Ableitungsmenge}
\newcommand*    {\Ableitungsmengen}[1][]{\glstext[#1]{Ableitungsmenge}[n]}
\newglossaryentry{Ableitungsmenge}{
	name        ={Ableitungsmenge \addIdx    {Ableitungsmenge}},
	text        ={Ableitungsmenge},
	description ={
		Eine \Menge\ von \Ableitungen, letztlich nichts anderes als eine \Ableitungsrelation.
	}
}

\newVerweis     {\Ableitungsrelation}  {\glstext}{Ableitungsrelation}
\newVerweis[en] {\Ableitungsrelationen}{\glstext}{Ableitungsrelation}
\newglossaryentry{Ableitungsrelation}{
	name        ={Ableitungsrelation \addIdx     {Ableitungsrelation}},
	text        ={Ableitungsrelation},
	see         ={Ableitung},
	description ={
		Eine \binaere\ \Relation\ \MtsDerive\ aus \MtsAllDerive.
		Für $R \in \MtsAllDerive$ auch mit \MtsDeriveR\ bezeichnet.
	}
}

\newVerweis     {\Abtrennungsregel}{\glstext}{Abtrennungsregel}
\newglossaryentry{Abtrennungsregel}{
	name        ={Abtrennungsregel \addIdx   {Abtrennungsregel}},
	text        ={Abtrennungsregel},
	see         ={TR},
	description ={
		Eine \Schlussregel.
	}
}

\newVerweis     {\Aequivalenz}  {\glstext}{Aequivalenz}
\newVerweis[en] {\Aequivalenzen}{\glstext}{Aequivalenz}
\newglossaryentry{Aequivalenz}{
	name        ={Äquivalenz \addIdx[
		name    ={Äquivalenz}]            {Aequivalenz}},
	text        ={Äquivalenz},
	see         ={MtsAequiv},
	description ={
		Eine \Gleichheitsrelation:
		Zwei Objekte $A$ und $B$ sind \GloFt{äquivalent}\alternativi{ähnlich}, $A \MtsAequiv B$, wenn sie in den \interessierendenEigenschaften\ für \MtsAequiv\ übereinstimmen.
	}
}

\newVerweis        {\Aequivalenzrelation}  {\glstext}{Aequivalenzrelation}
\newVerweis[en]    {\Aequivalenzrelationen}{\glstext}{Aequivalenzrelation}
\longnewglossaryentry{Aequivalenzrelation}{
	name            ={Äquivalenzrelation \addIdx[
		name        ={Äquivalenzrelation}]           {Aequivalenzrelation}},
	text            ={Äquivalenzrelation},
}{
	Eine \GloFt{Äquivalenzrelation} ist eine \binaere\ \Relation\ auf einer \Menge\ $M$ mit folgenden Eigenschaften
	(dabei sei $\sim$ die \gloFt{Äquivalenzrelation}):
	\begin{align}
		&\text{\textbf{reflexiv }}   &:&&\qquad  &a \sim a \\
		&\text{\textbf{transitiv }}  &:&&\qquad((&a \sim b) \MtsAnd (b \sim c)) \MtsImp (a \sim c)\\
		&\text{\textbf{symmetrisch }}&:&&\qquad (&a \sim b) \MtsImp (b \sim a)
		\formulatoleft
	\end{align}
	jeweils für alle Elemente $a$, $b$ und $c$ aus $M$.
}

\newVerweis     {\Allquantor} {\glstext}{Allquantor}
\newglossaryentry{Allquantor}{
	name        ={Allquantor \addIdx    {Allquantor}},
	text        ={Allquantor},
	description ={
		Man nennt den \Quantor\ \MtsForall\ \textbzw\ \OjkForall\ auch \GloFt{Allquantor}.
	}
}

\newVerweis     {\Existenzquantor} {\glstext}{Existenzquantor}
\newglossaryentry{Existenzquantor}{
	name        ={Existenzquantor \addIdx    {Existenzquantor}},
	text        ={Existenzquantor},
	description ={
		Man nennt den \Quantor\ \MtsExists\ \textbzw\ \OjkExists\ auch \GloFt{Existenzquantor}.
	}
}

\newVerweis     {\Alphabet} {\glstext}{Alphabet}
\newcommand*    {\Alphabets}[1][]{\glstext[#1]{Alphabet}[s]}
\newglossaryentry{Alphabet}{
	name        ={Alphabet \addIdx    {Alphabet}},
	text        ={Alphabet},
	description ={
		\todo{Beschreibung fehlt noch}% ToDo=Alphabet
	}
}

\newVerweis     {\Anfangsregel}{\glstext}{Anfangsregel}
\newglossaryentry{Anfangsregel}{
	name        ={Anfangsregel \addIdx   {Anfangsregel}},
	text        ={Anfangsregel},
	description ={
		Die \Schlussregel\ \glsAR\ um anfangen zu können.
	}
}

\newVerweis     {\ASBA}{\glstext} {ASBA}
\newglossaryentry{ASBA}{
	name        ={ASBA \addIdx   {ASBA}},
	text        ={ASBA},
	description ={
		ist ein Akronym für „\textbf{A}xiome, \textbf{S}ätze, \textbf{B}eweise und \textbf{A}uswertungen“.
		Es bezeichnet das in diesem Dokument beschriebene Programmsystem, das zu eingegebenen \Axiomen, \Saetzen\ und \Beweisen\ letztere prüft, Auswertungen generiert und unter Zuhilfenahme gegebener \Ausgabeschemata\ eine Ausgabe im \LaTeX-Format in mathematisch üblicher Schreibweise mit \Formeln\ erstellt.
	}
}

\newVerweis     {\atomar}  {\glstext}{atomar}
\newVerweis     {\Atomar}  {\Glstext}{atomar}
\newVerweis[e]  {\atomare} {\glstext}{atomar}
\newVerweis[e]  {\Atomare} {\glstext}{atomar}
\newVerweis[en] {\atomaren}{\glstext}{atomar}
\newVerweis[es] {\atomares}{\glstext}{atomar}
\newglossaryentry{atomar}{
	name        ={atomar \addIdx     {atomar}},
	text        ={atomar},
	see         ={zerlegbar},
	description ={
		Das Attribut \GloFt{atomar} kann auf \Aussagen, \Formeln\ und \Symbole\ angewendet werden.
		\Atomar\ sind solche, die keine echten \Teilobjekte\ gleicher \Objektart\ enthalten.
	}
}

\dummyVerweis   {\Ausdruck}{\glstext}{Ausdruck}% ToDo=Ausdruck --> Formel?

\dummyVerweis   {\logischenAusdruecke} {\glsuseri}{logischen Ausdrücke}% ToDo=logischer Ausdruck
\dummyVerweis[n]{\logischenAusdruecken}{\glsuseri}{logischen Ausdrücke}% ToDo=logischer Ausdruck

\dummyVerweis[n]{\metasprachlichenAusdruecken}{\glsuseri}{metasprachlichen Ausdrücke}% ToDo=metasprachlicher Ausdruck

\newVerweis     {\Ausgabeschema}  {\glstext}{Ausgabeschema}
\newVerweis[ta] {\Ausgabeschemata}{\glstext}{Ausgabeschema}
\newglossaryentry{Ausgabeschema}{%%% geprüft
	name        ={Ausgabeschema \addIdx     {Ausgabeschema}},
	text        ={Ausgabeschema},
	description ={\AusgabeschemaDescription}
}
\newcommand*{\AusgabeschemaDescription}{
	Ein \GloFt{Ausgabeschema} ist für \ASBA\ eine Beschreibung, wie ein bestimmtes mathematisches \Objekt\ ausgegeben werden soll.
	Dies kann \textzB\ ein Stück \LaTeX-Code mit entsprechenden Parametern sein.
}

\newVerweis         {\Aussage}{\glstext}{Aussage}
\newVerweis[n]     {\Aussagen}{\glstext}{Aussage}
\longnewglossaryentry{Aussage}{
	name            ={Aussage \addIdx   {Aussage}},
	text            ={Aussage},
}{
	\begin{wikicite}{bib:Aussage}
		Eine \wikibf{Aussage} im Sinn der \wikilink{aristotelischen Logik} ist ein sprachliches Gebilde, von dem es sinnvoll ist zu \wikiit{fragen}, ob es \wikilink{wahr} oder falsch ist (so genanntes Aristotelisches \wikilink{Zweiwertigkeitsprinzip}). Es ist nicht erforderlich, \wikiit{sagen} zu können, ob das Gebilde wahr oder falsch ist. Es genügt, dass die Frage nach Wahrheit („Zutreffen“) oder Falschheit („Nicht-Zutreffen“) sinnvoll ist, – was zum Beispiel bei Fragesätzen, Ausrufen und Wünschen nicht der Fall ist. Aussagen sind somit Sätze, die \wikilink{Sachverhalte} beschreiben und denen man einen \wikilink{Wahrheitswert} zuordnen kann.
	\end{wikicite}
	\GlossarZusatz{
		Das entscheidende Kriterium ist, dass man einer \Aussage\ zumindest im Prinzip einen \Wahrheitswert\ zuordnen kann, \textggf\ nach Ersetzung von Parametern durch konkrete Argumente.
		Dies gilt natürlich auch, wenn \metasprachlicheSymbole\ verwendet werden, weswegen sie in \GloFt{Aussagen} verwendet werden können.
		Da man \logischenAusdruecken\ und \Relationen\ ebenfalls einen \Wahrheitswert\ zuordnen kann%
		\footnote{%
			Zumindest prinzipiell nach Ersetzung von \Variablen\ durch konkrete Werte.
		},
		können wir sie ebenfalls als \Aussagen\ behandeln.
		Es handelt sich dann um \logischeA, im Gegensatz zu \metasprachlichenAussagen.
	}
}

\newVerweis      {\metasprachlicheAussage} {\glstext}       {metasprachlicheAussage}
\newVerweis[n]   {\metasprachlicheAussagen}{\glstext}       {metasprachlicheAussage}
\newVerweis[n]  {\metasprachlichenAussagen}{\glsuseri}      {metasprachlicheAussage}
\newglossaryentry {metasprachlicheAussage}{
	name       =                     {---, metasprachliche \addIdx[
		name   =                     {---, metasprachliche},
		sort   =                 {Aussage, metasprachliche}]{metasprachlicheAussage}},
	sort       =                 {Aussage, metasprachliche},
	text       ={metasprachliche  Aussage},
	user1      ={metasprachlichen Aussage},
	description={
		Die \GloFt{metasprachlichen} \Aussagen\ sind ... \todo{Beschreibung fehlt noch}% ToDo=metasprachliche Aussage
	}
}

\newVerweis     {\logischeAussage} {\glstext} {logischeAussage}
\newVerweis[n]  {\logischeAussagen}{\glstext} {logischeAussage}
\newVerweis     {\logischeA}       {\glsuseri}{logischeAussage}
\newglossaryentry{logischeAussage}{
	name       =             {---, logische \addIdx[
		name   =             {---, logische},
		sort   =         {Aussage, logische}] {logischeAussage}},
	sort       =         {Aussage, logische},
	text       ={logische Aussage},
	user1      ={logische},
	description={
		Die \GloFt{logischen} \Aussagen\ sind ... \todo{Beschreibung fehlt noch}% ToDo=logische Aussage
	}
}

\newVerweis     {\Aussagedefinition}  {\glstext}{Aussagedefinition}
\newVerweis[en] {\Aussagedefinitionen}{\glstext}{Aussagedefinition}
\newglossaryentry{Aussagedefinition}{%ToDo prüfen
	name        ={Aussagedefinition \addIdx     {Aussagedefinition}},
	text        ={Aussagedefinition},
	see         ={Objektdefinition},
	description ={
		Eine \Metadefinition: Die formale Definition einer \Aussage.
		\ifmarginparFlg\newline\else\fi
		\seqqt{$A \MtsDefEquiv B$} steht für \standsfor{$A$ ist \defFt{definitionsgemäß äquivalent zu} $B$} für \Aussagen\ $A$ und $B$.
		Gewissermaßen ist $A$ nur eine andere Schreibweise für $B$.
	}
}

\newVerweis         {\Aussagenlogik}{\glstext} {Aussagenlogik}
\newVerweis         {\AussagenL}    {\glsuseri}{Aussagenlogik}
\longnewglossaryentry{Aussagenlogik}{
	name            ={Aussagenlogik \addIdx    {Aussagenlogik}},
	text            ={Aussagenlogik},
	user1           ={Aussagen-},
	see             ={Aussage,Junktor,Logik,Praedikatenlogik,Wahrheitswert},
}{
	\begin{wikicite}{bib:Aussagenlogik}
		Die \wikibf{Aussagenlogik} ist ein Teilgebiet der \wikilink{Logik}, das sich mit Aussagen und deren Verknüpfung durch \wikilink{Junktoren} befasst, ausgehend von strukturlosen \wikilink{Elementaraussagen} (Atomen), denen ein \wikilink{Wahrheitswert} zugeordnet wird. In der \wikiit{klassischen Aussagenlogik} wird jeder Aussage genau einer der zwei Wahrheitswerte „wahr“ und „falsch“ zugeordnet. Der Wahrheitswert einer zusammengesetzten Aussage lässt sich ohne zusätzliche Informationen aus den Wahrheitswerten ihrer Teilaussagen bestimmen.
	\end{wikicite}
}

\newVerweis     {\Auswertung}  {\glstext}{Auswertung}
\newVerweis[en] {\Auswertungen}{\glstext}{Auswertung}
\newglossaryentry{Auswertung}{%%% geprüft
	name        ={Auswertung \addIdx     {Auswertung}},
	text        ={Auswertung},
	description ={\AuswertungDescription}
}
\newcommand*{\AuswertungDescription}{
	\GloFt{Auswertungen} sind für \ASBA\ statistische und andere \gloFt{Auswertungen}, die bestimmten Elementen der Datei \textbzw\ Datenbank zugeordnet sind.
	\textZB\ können zu einem \Satz\ alle für einen \Beweis\ notwendigen \Axiome\ angegeben werden.
}

\newVerweis     {\Axiom}  {\glstext}{Axiom}
\newVerweis[e]  {\Axiome} {\glstext}{Axiom}
\newVerweis[en] {\Axiomen}{\glstext}{Axiom}
\newglossaryentry{Axiom}{%%% geprüft
	name        ={Axiom \addIdx     {Axiom}},
	text        ={Axiom},
	see         ={MtsAxiom,MtsAxiomSet},
	description ={\AxiomDescription}
}
\newcommand*{\AxiomDescription}{
	Ein \GloFt{Axiom} ist eine \Aussage, die nicht aus anderen Aussagen abgeleitet werden kann.
	Es können wie bei \Saetzen\ \Praemissen\ und \Konklusionen\ vorhanden sein, aber keine \Beweise.
}

\newVerweis     {\Axiomensystem} {\glstext}{Axiomensystem}
\newVerweis[e]  {\Axiomensysteme}{\glstext}{Axiomensystem}
\newglossaryentry{Axiomensystem}{%%% geprüft
	name        ={Axiomensystem \addIdx    {Axiomensystem}},
	text        ={Axiomensysteme},
	description ={
		Eine \Menge\ von \Axiomen.
	}
}

%B === B === B === B === B === B === B === B === B === B === B === B === B === B

\newVerweis     {\Basisregel} {\glstext}{Basisregel}
\newVerweis[n]  {\Basisregeln}{\glstext}{Basisregel}
\newglossaryentry{Basisregel}{
	name        ={Basisregel \addIdx    {Basisregel}},
	text        ={Basisregel},
	description ={
		Eine \Schlussregel, die nicht mehr auf andere zurückgeführt wird.
		Obwohl das auch auf die \Identitaetsregeln\ zutrifft, werden diese hier aber nicht dazu gezählt.
	}
}

\newVerweis     {\Baustein} {\glstext}{Baustein}
\newVerweis[e]  {\Bausteine}{\glstext}{Baustein}
\newglossaryentry{Baustein}{
	name        ={Baustein \addIdx    {Baustein}},
	text        ={Baustein},
	description ={
		\todo{Beschreibung fehlt noch}% ToDo=Baustein
	}
}

\newVerweis         {\Begriff}  {\glstext}{Begriff}
\newVerweis[e]      {\Begriffe} {\glstext}{Begriff}
\newVerweis[en]     {\Begriffen}{\glstext}{Begriff}
\longnewglossaryentry{Begriff}{%%% geprüft
	name            ={Begriff \addIdx     {Begriff}},
	text            ={Begriff},
	see             ={Bezeichnung},
}{
	\begin{wikicite}{bib:Begriff}
		Mit dem Ausdruck \wikibf{Begriff} (\wikilink{mittelhochdeutsch} und \wikilink{frühneuhochdeutsch} \wikiit{begrif} oder \wikiit{begrifunge}) ist allgemein der \wikilink{Bedeutungsinhalt} einer \wikilink{Bezeichnung} angesprochen. Die Abgrenzung zwischen Begriffen und rein gedanklichen (mentalen) Einheiten erfolgt jedoch oft unscharf: Teilweise wird ein \wikiit{Begriff} als „mentale Informationseinheit“ beschrieben, (also genauso wie in der Kognitionswissenschaft das Konzept). Präziser ist die Abgrenzung des \wikiit{Begriffes} als \wikiit{Konzept, das sprachlich benannt ist}, oder geradezu als die \wikiit{Kombination aus einer sprachlichen Bezeichnung und dem entsprechenden Konzept}.

		[\textdots]
	\end{wikicite}
}

\newVerweis     {\Beispielsymbol}{\glstext}{Beispielsymbol}
\newglossaryentry{Beispielsymbol}{
	name        ={Beispielsymbol \addIdx   {Beispielsymbol}},
	text        ={Beispielsymbol},
	see         ={Symbol},
	description ={
		\todo{Beschreibung fehlt noch}% ToDo=Beispielsymbol
	}
}

\newVerweis         {\Benennung}  {\glstext}{Benennung}
\newVerweis[en]     {\Benennungen}{\glstext}{Benennung}
\longnewglossaryentry{Benennung}{%%% geprüft
	name            ={Benennung \addIdx     {Benennung}},
	text            ={Benennung},
	see             ={Bezeichnung,Symbol},
}{
	\begin{wikicite}{bib:Benennung}
		Eine \wikibf{Benennung} ist die \wikilink{Bezeichnung} eines Gegenstandes durch ein \wikilink{Wort} oder mehrere Wörter.[1] Die Benennung gilt in der Sprachwissenschaft und in der \wikilink{Terminologielehre} als die sprachliche Form, mit der \wikilink{Begriffe} ins Bewusstsein gerufen werden.[2] Eine Benennung ist insofern die Versprachlichung einer Vorstellung.[2] Der weiter gefasste Oberbegriff \wikiit{Bezeichnung} beinhaltet demgegenüber, neben der \wikiit{Benennung}, auch nichtsprachliches, wie Nummern, Notationen und Symbole.[3] Bei einer \wikilink{fachsprachlichen} Benennung spricht man auch von einem \wikilink{Fachausdruck} oder Terminus.[2] Benennungen kommen als Einwort- und als \wikilink{Mehrwortbenennungen}, auch Mehrworttermini genannt, vor.

		[\textdots]
	\end{wikicite}
}

\newVerweis     {\beschraenkt}  {\glstext}{beschraenkt}
\newVerweis[e]  {\beschraenkte} {\glstext}{beschraenkt}
\newVerweis[en] {\beschraenkten}{\glstext}{beschraenkt}
\newglossaryentry{beschraenkt}{
	name        ={beschränkt \addIdx[
		name    ={beschränkt}]            {beschraenkt}},
	text        ={beschränkt},
	description ={
		Eine \Schlussregel\ heißt \beschraenkt, wenn sie nur endlich viele Prämissen und Konklusionen hat.
	}
}

\newVerweis         {\Beweis}  {\glstext}{Beweis}
\newVerweis[e]      {\Beweise} {\glstext}{Beweis}
\newVerweis[es]     {\Beweises}{\glstext}{Beweis}
\newVerweis[en]     {\Beweisen}{\glstext}{Beweis}
\longnewglossaryentry{Beweis}{%%% geprüft
	name            ={Beweis \addIdx     {Beweis}},
	text            ={Beweis},
	see             ={Ableitung,Aussage,Axiom},
}{
	\begin{wikicite}{bib:Beweis}
		Ein \wikibf{Beweis} ist in der Mathematik die als fehlerfrei anerkannte Herleitung der Richtigkeit bzw. der Unrichtigkeit einer \wikibf{Aussage} aus einer Menge von \wikilink{Axiomen}, die als wahr vorausgesetzt werden, und anderen Aussagen, die bereits bewiesen sind. Um den Beweis klar vom gültigen Schluss zu unterscheiden, spricht man auch vom \wikibf{axiomatischen Beweis}.

		Umfangreichere Beweise von mathematischen Sätzen werden in der Regel in mehrere kleine Teilbeweise aufgeteilt, siehe dazu \wikilink{Satz} und \wikilink{Hilfssatz}.

		In der \wikilink{Beweistheorie}, einem Teilgebiet der \wikilink{mathematischen Logik}, werden Beweise formal als \wikilink{Ableitungen} aufgefasst und selbst als mathematische Objekte betrachtet, um etwa die Beweisbarkeit oder Unbeweisbarkeit von Sätzen aus gegebenen Axiomen selbst zu beweisen.
	\end{wikicite}
	\BeweisDescription
}
\newcommand*{\BeweisDescription}{
	Ein \GloFt{Beweis} besteht aus einer \Folge\ von \Beweisschritten, die aus gegebenen \Praemissen\ \Konklusionen\ ableitet.
}
\newsynonym{\beweisbar}{beweisbar}{\ableitbar}

\newVerweis     {\Beweisschritt}  {\glstext}{Beweisschritt}
\newVerweis[e]  {\Beweisschritte} {\glstext}{Beweisschritt}
\newVerweis[en] {\Beweisschritten}{\glstext}{Beweisschritt}
\newglossaryentry{Beweisschritt}{
	name        ={Beweisschritt \addIdx     {Beweisschritt}},
	text        ={Beweisschritt},
	see         ={MtsBeweisschritt,MtsBeweisschrittSet,MtsBeweisschrittTup},
	description ={
		Eine Vorschrift, wie aus vorgegebenen \Aussagen\ (den \Praemissen) weitere (die \Konklusionen) folgen.
	}
}

\newVerweis     {\Beweisschrittfolge} {\glstext}{Beweisschrittfolge}
\newVerweis[n]  {\Beweisschrittfolgen}{\glstext}{Beweisschrittfolge}
\newglossaryentry{Beweisschrittfolge}{
	name        ={Beweisschrittfolge \addIdx    {Beweisschrittfolge}},
	text        ={Beweisschrittfolge},
	description ={
		Eine Folge von \Beweisschritten.
	}
}

\newVerweis     {\Beweisschrittmenge} {\glstext}{Beweisschrittmenge}
\newVerweis[n]  {\Beweisschrittmengen}{\glstext}{Beweisschrittmenge}
\newglossaryentry{Beweisschrittmenge}{
	name        ={Beweisschrittmenge \addIdx    {Beweisschrittmenge}},
	text        ={Beweisschrittmenge},
	description ={
		Eine \Menge\ von \Beweisschritten, insbesondere die \Menge\ der Glieder einer \Beweisschrittfolge.
	}
}

\newVerweis         {\Bezeichnung}  {\glstext}{Bezeichnung}
\newVerweis[en]     {\Bezeichnungen}{\glstext}{Bezeichnung}
\longnewglossaryentry{Bezeichnung}{%%% geprüft
	name            ={Bezeichnung \addIdx     {Bezeichnung}},
	text            ={Bezeichnung},
	see             ={Begriff,Benennung,Symbol},
}{
	\begin{wikicite}{bib:Bezeichnung}
		Eine \wikibf{Bezeichnung} ist die Repräsentation eines \wikilink{Begriffs} mit sprachlichen oder anderen Mitteln. Erfolgt diese Repräsentation mittels Wörtern, handelt es sich um eine \wikilink{Benennung}. Eine nichtsprachliche Bezeichnung kann durch ein \wikilink{Symbol} erfolgen.

		[\textdots]
	\end{wikicite}
}

\newVerweis     {\binaer}  {\glstext}{binaer}
\newVerweis[e]  {\binaere} {\glstext}{binaer}
\newVerweis[en] {\binaeren}{\glstext}{binaer}
\newglossaryentry{binaer}{
	name        ={binär \addIdx[
		name    ={binär}]            {binaer}},
	text        ={binär},
	see         ={unaer},
	description ={
		Eine \Operation, \Funktion\ oder \Relation\ heißt \GloFt{binär}, wenn ihre \Stelligkeit\ gleich 2 ist.
	}
}

%D === D === D === D === D === D === D === D === D === D === D === D === D === D

\newVerweis         {\Darstellung}  {\glstext}{Darstellung}
\newVerweis[en]     {\Darstellungen}{\glstext}{Darstellung}
\longnewglossaryentry{Darstellung}{
	name            ={Darstellung \addIdx     {Darstellung}},
	text            ={Darstellung},
}{
	\begin{wikicite}{bib:Darstellung}
		Unter \wikibf{Darstellung} (zur semantischen Wurzel \wikiit{dar}- „öffentlich übergeben“, vergleiche Darbietung, \wikilink{Darlehen}, darreichen) versteht man die Umsetzung von \wikilink{Sachverhalten}, \wikilink{Ereignissen} oder abstrakten Konzepten mittels \wikilink{Zeichen}, performativer \wikilink{Handlungen} oder \wikilink{Modellen}. Historisch reicht die Darstellung von der \wikilink{mündlichen Überlieferung} über das \wikilink{Schauspiel} bis zur \wikilink{Computergrafik} und schließt zahlreiche Vermittlungsmethoden zwischen \wikilink{Text}, \wikilink{Bild} und künstlerischer \wikilink{Aufführung} ein.

		[\textdots]
	\end{wikicite}
	\DarstellungDescription
}
\newcommand*{\DarstellungDescription}{
	Die \GloFt{Darstellung} mathematischer \Objekte\ geschieht auf mehreren Ebenen% ToDo=Darstellung
}

\newVerweis     {\interneDarstellung}{\glstext} {interneDarstellung}
\newglossaryentry{interneDarstellung}{
	name       =                 {---, interne \addIdx[
		name   =                 {---, interne},
		sort   =         {Darstellung, interne}]{interneDarstellung}},
	sort       =         {Darstellung, interne},
	text       ={interne Darstellung},
	description={
		\todo{Beschreibung fehlt noch}% ToDo=interne Darstellung
	}
}

\newVerweis     {\logischeDarstellung}{\glstext} {logischeDarstellung}
\newVerweis    {\logischenD}          {\glsuseri}{logischeDarstellung}
\newglossaryentry{logischeDarstellung}{
	name       =                 {---, logische \addIdx[
		name   =                 {---, logische},
		sort   =         {Darstellung, logische}]{logischeDarstellung}},
	sort       =         {Darstellung, logische},
	text       ={logische Darstellung},
	user1      ={logischen},
	description={
		\todo{Beschreibung fehlt noch}% ToDo=logische Darstellung
	}
}

\newVerweis     {\Darstellungsweise} {\glstext}{Darstellungsweise}
\newVerweis[n]  {\Darstellungsweisen}{\glstext}{Darstellungsweise}
\newglossaryentry{Darstellungsweise}{
	name        ={Darstellungsweise \addIdx    {Darstellungsweise}},
	text        ={Darstellungsweise},
	description ={
		Die Art der \Darstellung\ mathematischer \Objekte.
	}
}

\newVerweis     {\Definitionsbereich} {\glstext} {Definitionsbereich}
\newVerweis[e]  {\Definitionsbereiche}{\glstext} {Definitionsbereich}
\newVerweis     {\DefinitionsB}       {\glsuseri}{Definitionsbereich}
\newglossaryentry{Definitionsbereich}{
	name        ={Definitionsbereich \addIdx     {Definitionsbereich}},
	text        ={Definitionsbereich},
	user1       ={Definitions},
	see         ={MtsDb,Quellbereich,Funktion},
	description ={
		Für eine \Funktion\ \FunktionDef{f}{A}{B} ist $\MtsDb(f)A$ ihr \Definitionsbereich\ (domain).
	}
}

\newVerweis     {\Differenz} {\glstext}{Differenz}
\newglossaryentry{Differenz}{
	name        ={Differenz \addIdx    {Differenz}},
	text        ={Differenz},
	description ={
		Eine \Mengenoperation: \todo{Beschreibung fehlt noch}% ToDo=Differenz von Mengen
	}
}

\newVerweis         {\Diskursuniversum} {\glstext}{Diskursuniversum}
\longnewglossaryentry{Diskursuniversum}{%%% geprüft
	name            ={Diskursuniversum \addIdx    {Diskursuniversum}},
	text            ={Diskursuniversum},
	see             ={Aussage,Begriff,Logik},
}{
	\begin{wikicite}{bib:Diskursuniversum}
		Unter einem \wikibf{Diskursuniversum} versteht man in der \wikilink{Logik} und \wikilink{Sprachphilosophie} die Gesamtheit der Gegenstände, auf die sich Aussagen wie „alle Gegenstände sind … “ (\wikilink{Allaussage}) oder „es gibt keine Gegenstände, die … sind“ (negative \wikilink{Existenzaussage}) beziehen. Solche Aussagen sind nur sinnvoll, wenn die Bedeutung von „Gegenstand“ auf einen bestimmten Bereich, das Diskursuniversum, eingeschränkt wird. Ausmaß und Art der Einschränkung hängen vom Inhalt und vom Zusammenhang der Aussagen ab. Es gibt daher nicht nur ein Diskursuniversum, sondern verschiedene Diskursuniversen.

		Der englische Ausdruck \wikibf{Universe of Discourse} wird auch in der deutschsprachigen Logik- und Informatikliteratur verwendet. Er geht auf \wikilink{Augustus De Morgan} (1847) zurück und bezeichnet den Bereich der Gegenstände (im weitesten Sinn), über die überhaupt geredet werden soll.

		Missverständnisse und Streit entstehen in der Logik wie im Alltag oft dadurch, dass Personen „aneinander vorbei“ von verschiedenen Dingen reden. Jemand behauptet z. B., dass es keine geflügelten Pferde gibt. Sein Widerpart weist dies mit dem Hinweis auf den \wikilink{Pegasus} zurück. Beide bewegen sich gedanklich in verschiedenen Welten. Ihr Streit lässt sich schlichten, wenn sie sich auf ein gemeinsames Diskursuniversum einigen, d. h. aushandeln, wovon die Rede (der \wikilink{Diskurs}) sein soll, ob nur von physisch existierenden Pferden oder auch von \wikilink{Fabelwesen}.

		Auch beim Gebrauch negativer (komplementärer) \wikilink{Begriffe} spielt das Diskursuniversum eine Rolle. Ausdrücke wie „Nichtschwimmer“, „Nichtfachmann“, „Nichtwähler“ können sinnvoll nur auf Personen angewandt werden. Die Nichtwähler bilden mit den Wählern zusammen das auf wahlberechtigte Personen eingeschränkte Diskursuniversum. Die Einschränkung geschieht beim Gebrauch solcher Begriffe automatisch. Wird die Automatik außer Betrieb gesetzt, indem man z. B. einen stillgelegten Schornstein als Nichtraucher bezeichnet, entsteht ein Wortspiel. Allgemein gilt für jeden Begriff: wird er mit dem zugehörigen negativen Begriff vereinigt (genauer: werden deren \wikilink{Extensionen} vereinigt), so bilden beide zusammen das Diskursuniversum oder den Bereich der Anwendungsfälle des positiv bestimmten Komplementärbegriffs:

		[eine Tabelle]

		In der \wikilink{Mengenlehre} entspricht dem Diskursuniversum die \wikilink{Grundmenge}, die Mengen entsprechen den Begriffen, die \wikilink{Komplemente} von Mengen der Negation von Begriffen. In der \wikilink{Prädikatenlogik} entspricht dem Diskursuniversum der Bereich der \wikilink{Definitionsmenge}, den die \wikilink{Gegenstandsvariable} einer \wikilink{quantifizierten} Aussage durchlaufen kann.

		Das \wikiit{Universe of Discourse} wird in der Logik zumeist abgekürzt mit \wikiit{U}, in der Informatik auch mit \wikiit{UoD}.

		Das \wikiit{U} ist in der Regel eine Teilmenge aller existierenden Objekte und insbesondere in der Prädikatenlogik der bei der Verwendung von \wikilink{Quantoren} festgelegte oder vorausgesetzte Objektbereich.
	\end{wikicite}
}

\newVerweis     {\Durchschnitt} {\glstext}{Durchschnitt}
\newglossaryentry{Durchschnitt}{
	name        ={Durchschnitt \addIdx    {Durchschnitt}},
	text        ={Durchschnitt},
	description ={
		Eine \Mengenoperation: \todo{Beschreibung fehlt noch}% ToDo=Durchschnitt von Mengen
	}
}

% TODO ### ab hier weiter prüfen
%E === E === E === E === E === E === E === E === E === E === E === E === E === E

\newVerweis     {\echt} {\glstext}{echt}
\newVerweis[e]  {\echte}{\glstext}{echt}
\newglossaryentry{echt}{
	name        ={echt \addIdx    {echt}},
	text        ={echt},
	description ={
		Attribut für ???% ToDo=echt
	}
}

\newVerweis      {\interessierendeEigenschaft}  {\glstext}      {interessierendeEigenschaft}
\newVerweis     {\interessierendenEigenschaft}  {\glsuseri}     {interessierendeEigenschaft}
\newVerweis[en] {\interessierendenEigenschaften}{\glsuseri}     {interessierendeEigenschaft}
\newglossaryentry {interessierendeEigenschaft}{%ToDo prüfen
	name       =                 {Eigenschaft, interessierende \addIdx[
		name   =                 {Eigenschaft, interessierende},
		sort   =                 {Eigenschaft, interessierende}]{interessierendeEigenschaft}},
	sort       =                 {Eigenschaft, interessierende},
	text       ={interessierende  Eigenschaft},
	user1      ={interessierenden Eigenschaft},
	description={
		Solche Eigenschaften von \Objekten, die im aktuellen Zusammenhang von Interesse sind, \textzB\ einen bestimmten Wert zu haben, Element einer bestimmten \Menge\ zu sein, ein bestimmtes \Objekt\ zu bezeichnen, usw.
	}
}

\newVerweis        {\Element} {\glstext}{Element}
\newVerweis[e]     {\Elemente}{\glstext}{Element}
\longnewglossaryentry{Element}{
	name            ={Element \addIdx   {Element}},
	text            ={Element},
	see             ={Element,Menge,Mengenlehre,Relation},
}{
	\begin{wikicite}{bib:Element}
		Ein \wikibf{Element} in der \wikilink{Mathematik} ist immer im Rahmen der \wikilink{Mengenlehre} oder \wikilink{Klassenlogik} zu verstehen. Die grundlegende \wikilink{Relation}, wenn $x$ ein Element ist und $M$ eine \wikilink{Menge} oder \wikilink{Klasse} ist, lautet:
		\begin{quote}
			„$x$ ist Element von $M$“ oder mit Hilfe des \wikilink{Elementzeichens} „x \MtsIn\ M“.
		\end{quote}
		Die Mengendefinition von \wikilink{Georg Cantor} beschreibt anschaulich, was unter einem Element im Zusammenhang mit einer Menge zu verstehen ist:
		\begin{quote}
			„Unter einer ‚Menge‘ verstehen wir jede Zusammenfassung $M$ von bestimmten wohlunterschiedenen Objekten $m$ unserer Anschauung oder unseres Denkens (welche die ‚Elemente‘ von $M$ genannt werden) zu einem Ganzen.“
		\end{quote}
		Diese anschauliche Mengenauffassung der \wikilink{naiven Mengenlehre} erwies sich als nicht widerspruchsfrei. Heute wird daher eine \wikilink{axiomatische} Mengenlehre benutzt, meist die \wikilink{Zermelo-Fraenkel-Mengenlehre}, teilweise auch eine allgemeinere \wikilink{Klassenlogik}.
	\end{wikicite}
}

\newVerweis     {\Elementoperation}  {\glstext}{Elementoperation}
\newVerweis[en] {\Elementoperationen}{\glstext}{Elementoperation}
\newglossaryentry{Elementoperation}{%ToDo prüfen
	name        ={Elementoperation \addIdx     {Elementoperation}},
	text        ={Elementoperation},
	description ={
		\todo{Beschreibung fehlt noch}% ToDo=Elementoperation
	}
}

\newVerweis     {\Elementrelation}  {\glstext}{Elementrelation}
\newVerweis[en] {\Elementrelationen}{\glstext}{Elementrelation}
\newglossaryentry{Elementrelation}{%ToDo prüfen
	name        ={Elementrelation \addIdx     {Elementrelation}},
	text        ={Elementrelation},
	see         ={Komponentenrelation},
	description ={
		Eine \GloFt{Elementrelation} ist eine Relation zwischen einem \Element\ und einer \Menge: \MtsIn, \MtsNi, \MtsInN und \MtsNiN
	}
}

\newVerweis     {\Ergebnis}   {\glstext}{Ergebnis}
\newVerweis[se] {\Ergebnisse} {\glstext}{Ergebnis}
\newcommand*    {\Ergebnissen}[1][]{\glstext[#1]{Ergebnis}[sen]}
\newglossaryentry{Ergebnis}{%ToDo prüfen
	name        ={Ergebnis \addIdx      {Ergebnis}},
	text        ={Ergebnis},
	see         ={MtsErgebnis,MtsErgebnisSet,MtsErgebnisRel},
	description ={
		Eine \Ableitung:
		Ein \Ergebnis\ eines \Beweises.
	}
}

\newVerweis     {\Ergebnismenge} {\glstext}{Ergebnismenge}
\newVerweis[n]  {\Ergebnismengen}{\glstext}{Ergebnismenge}
\newglossaryentry{Ergebnismenge}{%ToDo prüfen
	name        ={Ergebnismenge \addIdx    {Ergebnismenge}},
	text        ={Ergebnismenge},
	description ={
		Eine \Ableitungsmenge:
		Die \Menge\ \MtsErgebnisSet\ der \Ergebnisse\ eines \Beweises.
	}
}

\newVerweis     {\Ersetzung}  {\glstext}{Ersetzung}
\newVerweis[en] {\Ersetzungen}{\glstext}{Ersetzung}
\newglossaryentry{Ersetzung}{%ToDo prüfen
	name        ={Ersetzung \addIdx     {Ersetzung}},
	text        ={Ersetzung},
	description ={
		Eine \Funktion\ zur \Transformation\ einer \Formel\ mittels \Ersetzung\ in eine gleichwertige.
		Die \Ersetzung\ heißt \zulaessig, wenn sie vorgegebene Regeln erfüllt.
	}
}

\newVerweis     {\Ersetzungsmenge} {\glstext}{Ersetzungsmenge}
\newVerweis[n]  {\Ersetzungsmengen}{\glstext}{Ersetzungsmenge}
\newglossaryentry{Ersetzungsmenge}{%ToDo prüfen
	name        ={Ersetzungsmenge \addIdx    {Ersetzungsmenge}},
	text        ={Ersetzungsmenge},
	description ={
		Eine \Menge\ von \Ersetzungen, meistens mit \MtsErsetzungSet\ bezeichnet.
	}
}

%F === F === F === F === F === F === F === F === F === F === F === F === F === F

\newVerweis         {\Fachbegriff}  {\glstext}{Fachbegriff}
\newVerweis[e]      {\Fachbegriffe} {\glstext}{Fachbegriff}
\newVerweis[en]     {\Fachbegriffen}{\glstext}{Fachbegriff}
\longnewglossaryentry{Fachbegriff}{%%% geprüft
	name            ={Fachbegriff \addIdx     {Fachbegriff}},
	text            ={Fachbegriff},
	see             ={Begriff,Fachgebiet},
}{
	\begin{wikicite}{bib:Terminus}
		Ein \wikibf{Terminus} oder \wikibf{Fachbegriff} ist eine \wikilink{definierte} \wikilink{Benennung} für einen \wikilink{Begriff} innerhalb der \wikilink{Fachsprache} eines \wikilink{Fachgebietes}. Synonyme dazu sind auch \wikibf{Term} oder \wikibf{Terminus technicus} (lateinisch \wikiit{terminus technicus}; \wikilink{Genus} \wikiit{m.}; \wikilink{Pl.} \wikiit{Termini technici}, kurz \wikiit{Termini}). \wikiit{Terminus} kann allerdings neben der rein sprachlichen \wikiit{Benennung} auch den Bedeutungsinhalt, den \wikiit{Begriff} selbst, ansprechen.

		Eine vergleichbare Bezeichnung ist \wikibf{Fachwort}. Ein \wikibf{Fachausdruck} ist ein \wikilink{sprachlicher Ausdruck}, der in einer Fachsprache verwendet wird und dort eine spezielle Bedeutung besitzt. \wikiit{Fachausdruck} gilt gegenüber \wikiit{Fachwort} als ein geeigneteres Ersatzwort für Terminus. Denn ein Terminus kann nicht nur in der Form einer Einwortbenennung, sondern auch als \wikilink{Mehrwortbenennung} (auch \wikiit{Mehrwortterminus}) vorliegen.

		Die Menge aller Termini eines Fachgebietes (die Benennungen aller Begriffe) bildet die jeweilige fachspezifische \wikilink{Terminologie} (den \wikilink{Fachwortschatz}). Mit der Untersuchung und Aufstellung von Terminologien beschäftigt sich die \wikilink{Terminologielehre}. Wenn ein Fachwortschatz standardisiert oder normiert ist, spricht man auch von einem \wikilink{Thesaurus} oder \wikilink{kontrollierten Vokabular} und nennt die darin enthaltenen Termini \wikilink{Deskriptoren}.
	\end{wikicite}
	Hier ist immer ein \Begriff\ aus einem \Fachgebiet\ oder der ganzen Mathematik gemeint.
}
\newcommand*{\FachbegriffDescription}{
	Ein \GloFt{Fachbegriff} ist eine \Benennung\ für einen \Begriff\ aus einem \Fachgebiet.
	Insbesondere kann es auch ein spezielles \Symbol\ sein.
}

\newVerweis         {\Fachgebiet}  {\glstext}{Fachgebiet}
\newVerweis[s]      {\Fachgebiets} {\glstext}{Fachgebiet}
\newVerweis[e]      {\Fachgebiete} {\glstext}{Fachgebiet}
\newVerweis[en]     {\Fachgebieten}{\glstext}{Fachgebiet}
\longnewglossaryentry{Fachgebiet}{%%% geprüft
	name            ={Fachgebiet \addIdx     {Fachgebiet}},
	text            ={Fachgebiet},
}{
	\begin{wikicite}{bib:Fachgebiet}
		\wikibf{Fachgebiet} (auch \wikibf{Fachbereich} oder \wikibf{Fachrichtung} oder \wikibf{Domäne}) ist das auf ein bestimmtes \wikilink{Wissensgebiet} begrenzte \wikilink{Wissen}.
	\end{wikicite}
	\FachgebietDescription
}
\newcommand*{\FachgebietDescription}{
	Ein \GloFt{Fachgebiet} ist für \ASBA\ ein Teilgebiet der Mathematik mit einer zugehörigen Basis aus \Axiomen, \Saetzen, \Fachbegriffen\ und \Darstellungsweisen, \textzB\ \Logik\ und \Mengenlehre.

	Ein \gloFt{Fachgebiet} kann bei \ASBA\ sehr klein sein und im Extremfall kein einziges Element enthalten.
	\emph{Umgebung} wäre vielleicht eine bessere \Bezeichnung, ist aber schon ein verbreiteter \Fachbegriff, so dass hier die \Bezeichnung\ \gloFt{Fachgebiet} verwendet wird.
}

\newVerweis         {\Folge} {\glstext}{Folge}
\newVerweis[n]      {\Folgen}{\glstext}{Folge}
\longnewglossaryentry{Folge}{%%% geprüft
	name            ={Folge \addIdx    {Folge}},
	text            ={Folge},
}{
	\begin{wikicite}{bib:Folge}
		Als \wikibf{Folge} oder \wikibf{Sequenz} wird in der \wikilink{Mathematik} eine Auflistung (\wikilink{Familie}) von endlich oder unendlich vielen fortlaufend nummerierten Objekten (beispielsweise Zahlen) bezeichnet. Dasselbe Objekt kann in einer Folge auch mehrfach auftreten. Das Objekt mit der Nummer $i$, man sagt hier auch: mit dem Index $i$ i, wird $i$-tes Glied oder $i$-te Komponente der Folge genannt. Endliche wie unendliche Folgen finden sich in allen Bereichen der Mathematik. Mit unendlichen Folgen, deren Glieder Zahlen sind, beschäftigt sich vor allem die \wikilink{Analysis}.

		Ist $n$ die Anzahl der Glieder einer endlichen Folge, so spricht man von einer Folge der Länge $n$, einer $n$-gliedrigen Folge oder von einem $n$-Tupel. Die Folge ohne Glieder, deren Index-Bereich also leer ist, wird leere Folge, 0-gliedrige Folge oder 0-Tupel genannt.
	\end{wikicite}
	\FolgeDescription
}
\newcommand*{\FolgeDescription}{
%%%%	Ein \GloFt{Folge}\alternativi{Sequenz} $\vec{a}$ ist eine Aneinanderreihung ihrer \defFt{\Komponenten} $a_i$, $i \in \MtsINo$, geschrieben $[a_1, a_2, \dots]$.
%%%%	Sind alle \Komponenten\ \Elemente\ einer \Menge\ $M$, so heißt $\vec{a}$ eine \gloFt{\Folge} \defFt{auf} $M$ oder eine \gloFt{\Folge} \defFt{von} \Elementen\ aus $M$.
%%%%	Hat die \gloFt{\Folge} nur endlich viele \Komponenten, so heißt sie \defFt{endlich} und die Anzahl $\MtsLen(\vec{a})$ ihrer \Komponenten\ ihre \defFt{Länge}.
%%%%	Ist die Länge gleich $0$, so sprechen wir von der \defFt{\leerenFolge} und bezeichnen sie mit \seqqt{$()$}.
%%%%	Eine endliche \gloFt{\Folge} der Länge $n$ heißt auch \defFt{$n$-\Tupel} und die leere \Folge\ demnach \defFt{$0$-\Tupel}.
}

\newVerweis      {\leereFolge} {\glstext} {leereFolge}
\newVerweis[n]   {\leereFolgen}{\glstext} {leereFolge}
\newVerweis     {\leerenFolge} {\glsuseri}{leereFolge}
\newglossaryentry {leereFolge}{%ToDo prüfen
	name       =         {---, leere \addIdx[
		name   =         {---, leere},
		sort   =       {Folge, leere}]    {leereFolge}},
	sort       =       {Folge, leere},
	text       ={leere  Folge},
	user1      ={leeren Folge},
	see        ={MtsLen,Folge,Tupel},
	description={
		Eine \Folge\ heißt \GloFt{leer}, wenn ihre Länge $0$ ist, \textdh\ wenn sie keine \Komponenten\ besitzt.
	}
}

\newVerweis     {\Folgenmenge}{\glstext}{Folgenmenge}
\newglossaryentry{Folgenmenge}{%ToDo prüfen
	name        ={Folgenmenge \addIdx   {Folgenmenge}},
	text        ={Folgenmenge},
	description ={
		\todo{Beschreibung fehlt noch}% ToDo=Folgenmenge
	}
}

\newVerweis     {\Folgenoperation}  {\glstext}{Folgenoperation}
\newVerweis[en] {\Folgenoperationen}{\glstext}{Folgenoperation}
\newglossaryentry{Folgenoperation}{%ToDo prüfen
	name        ={Folgenoperation \addIdx     {Folgenoperation}},
	text        ={Folgenoperation},
	description ={
		\todo{Beschreibung fehlt noch}% ToDo=Folgenoperation
	}
}

\newVerweis     {\Folgenrelation}  {\glstext}{Folgenrelation}
\newVerweis[en] {\Folgenrelationen}{\glstext}{Folgenrelation}
\newglossaryentry{Folgenrelation}{%ToDo prüfen
	name        ={Folgenrelation \addIdx     {Folgenrelation}},
	text        ={Folgenrelation},
	description ={
		\todo{Beschreibung fehlt noch}% ToDo=Folgenrelation
	}
}

\newcommand*{\Folgerungen}[1][]{\glstext[#1]{Folgerung}[en]}
\newsynonym{\Folgerung}{Folgerung}{\Konklusion}

\newsynonym{\Folgerungsmenge}{Folgerungsmenge}{\Konklusionsmenge}

\newVerweis     {\Formationsregel} {\glstext}{Formationsregel}
\newVerweis[n]  {\Formationsregeln}{\glstext}{Formationsregel}
\newglossaryentry{Formationsregel}{%ToDo prüfen
	name        ={Formationsregel \addIdx    {Formationsregel}},
	text        ={Formationsregel},
	description ={
		\todo{Beschreibung fehlt noch}% ToDo=Formationsregel
	}
}

\newVerweis     {\Formel} {\glstext}{Formel}
\newVerweis[n]  {\Formeln}{\glstext}{Formel}
\newglossaryentry{Formel}{%ToDo prüfen - besser: Formel = Element einer Sprache?
	name        ={Formel \addIdx    {Formel}},
	text        ={Formel},
	description ={
		Unter einer \GloFt{Formel} verstehen wir stets eine mathematische \gloFt{Formel}.
		Diese kann aus einem einzigen \Symbol\ bestehen (\atomare\ \gloFt{Formel}), andererseits aber auch mehrdimensional sein, lässt sich dann aber mittels geeigneter Definitionen immer eindeutig als eine \Symbolfolge\ schreiben.
	}
}

\newVerweis     {\allgemeingueltigeFormel}{\glstext}         {allgemeingueltigeFormel}
\newVerweis    {\allgemeingueltigenFormel}{\glsuseri}        {allgemeingueltigeFormel}
\newglossaryentry{allgemeingueltigeFormel}{%ToDo prüfen
	name       =                     {---, allgemeingültige \addIdx[
		name   =                     {---, allgemeingültige},
		sort   =                  {Formel, allgemeingültige}]{allgemeingueltigeFormel}},
	sort       =                  {Formel, allgemeingültige},
	text       ={allgemeingültige  Formel},
	user1      ={allgemeingültigen Formel},
	description={
		Eine \Formel\ heißt \GloFt{allgemeingültig}, wenn sie aus den \Axiomen\ und \allgemeingueltigenSchlussregeln\ abgeleitet werden kann.
	}
}

\newVerweis      {\aussagenlogischeFormel} {\glstext}        {aussagenlogischeFormel}
\newVerweis[n]   {\aussagenlogischeFormeln}{\glstext}        {aussagenlogischeFormel}
\newVerweis     {\aussagenlogischenFormel} {\glsuseri}       {aussagenlogischeFormel}
\newVerweis[n]  {\aussagenlogischenFormeln}{\glsuseri}       {aussagenlogischeFormel}
\newVerweis      {\aussagenlogischeF}      {\glsuserii}      {aussagenlogischeFormel}
\newglossaryentry {aussagenlogischeFormel}{%ToDo prüfen
	name       =                     {---, aussagenlogische \addIdx[
		name   =                     {---, aussagenlogische},
		sort   =                  {Formel, aussagenlogische}]{aussagenlogischeFormel}},
	sort       =                  {Formel, aussagenlogische},
	text       ={aussagenlogische  Formel},
	user1      ={aussagenlogischen Formel},
	user2      ={aussagenlogische},
	description={
		Eine \Formel\ heißt \GloFt{aussagenlogisch}, wenn sie ein Element von \OjkFor\ ist.
	}
}

\newVerweis      {\praedikatenlogischeFormel} {\glstext}           {praedikatenlogischeFormel}
\newVerweis[n]   {\praedikatenlogischeFormeln}{\glstext}           {praedikatenlogischeFormel}
\newglossaryentry {praedikatenlogischeFormel}{%ToDo prüfen
	name       =                        {---, praedikatenlogische \addIdx[
		name   =                        {---, praedikatenlogische},
		sort   =                     {Formel, praedikatenlogische}]{praedikatenlogischeFormel}},
	sort       =                     {Formel, praedikatenlogische},
	text       ={praedikatenlogische  Formel},
	description={
		Eine \Formel\ heißt \GloFt{prädikatenlogisch}, wenn sie ein Element von \OjkFor\ ist.%TODO andere Menge
	}
}

\newVerweis     {\Formelmenge} {\glstext}{Formelmenge}
\newVerweis[n]  {\Formelmengen}{\glstext}{Formelmenge}
\newglossaryentry{Formelmenge}{%ToDo prüfen
	name        ={Formelmenge \addIdx    {Formelmenge}},
	text        ={Formelmenge},
	description ={
		Eine \Menge\ von \Formeln, oft mit \MtsSprache\ bezeichnet.
		Man nennt \MtsSprache\ auch eine \Sprache\ und ihre Elemente \Woerter, insbesondere dann, wenn es eindeutige Regeln zur Konstruktion von \MtsSprache\ gibt.
		Wir bevorzugen „\Formel“ und „\Formelmenge“.
	}
}

\newVerweis      {\MtsFktSep}    {\glsuserv} {Funktion}
\newVerweis      {\MtsFktArrow}  {\glsuservi}{Funktion}
\newVerweis         {\Funktion}  {\glstext}  {Funktion}
\newVerweis[en]     {\Funktionen}{\glstext}  {Funktion}
\longnewglossaryentry{Funktion}{%ToDo prüfen
	name            ={Funktion \addIdx       {Funktion}},
	text            ={Funktion},
	user5           ={:},
	user6           ={\ensuremath{\RawMtsFktArrow}},
	see             ={Abbildung,Element,Menge,Objekt,Relation},
}{
	\begin{wikicite}{bib:Funktion}
		In der \wikilink{Mathematik} ist eine \wikibf{Funktion} (lateinisch \wikiit{functio}) oder \wikibf{Abbildung} eine Beziehung (\wikilink{Relation}) zwischen zwei \wikilink{Mengen}, die jedem Element der einen Menge (Funktionsargument, unabhängige Variable, $x$-Wert) genau ein Element der anderen Menge (Funktionswert, abhängige Variable, $y$-Wert) zuordnet. Der Funktionsbegriff wird in der Literatur unterschiedlich definiert, jedoch geht man generell von der Vorstellung aus, dass Funktionen \wikilink{mathematischen Objekten} mathematische Objekte zuordnen, zum Beispiel jeder reellen Zahl deren Quadrat.  Das Konzept der Funktion oder Abbildung nimmt in der modernen Mathematik eine zentrale Stellung ein; es enthält als Spezialfälle unter anderem \wikilink{parametrische Kurven}, Skalar- und \wikilink{Vektorfelder}, \wikilink{Transformationen}, \wikilink{Operationen}, \wikilink{Operatoren} und vieles mehr.
	\end{wikicite}
	Eine \GloFt{$n$-\stellige\ Funktion} $f$ von einer \Menge\ $A = A_1 \MtsTimes \dots \MtsTimes A_n$, dem \Definitionsbereich, in eine \Menge\ $B$, den \Zielbereich, ist eine ($n$+1)-\stellige\ \Relation\ $(G,A_1,\dots,A_n,B)$ derart, dass es für jedes $\vec{a} = (a_1,\dots,a_n)$ mit $a_i \in A_i$ genau ein $b \in B$ gibt mit $(a_1,\dots,a_n,b) \in f$.
	Dieses $b$ wird auch mit \seqqt{$f(a_1,\dots,a_n)$} , \seqqt{$f a_1 \dots a_n$} , \seqqt{$f(\vec{a})$} oder \seqqt{$f\vec{a}$} bezeichnet.
	\\Schreibweise: \seqqt{\FunktionDef{f}{A}{B}} \textbzw\ \seqqt{$\FunktionDef{f}{A_1 \MtsTimes \dots \MtsTimes A_n}{B}$}
}

\newVerweis     {\Funktionssymbol}  {\glstext}{Funktionssymbol}
\newVerweis[e]  {\Funktionssymbole} {\glstext}{Funktionssymbol}
\newVerweis[en] {\Funktionssymbolen}{\glstext}{Funktionssymbol}
\newglossaryentry{Funktionssymbol}{%ToDo prüfen
	name        ={Funktionssymbol \addIdx     {Funktionssymbol}},
	text        ={Funktionssymbol},
	description ={
		Ein \Symbol\ für eine \Funktion.
	}
}

\newVerweis     {\Funktionswert} {\glstext}{Funktionswert}
\newVerweis[e]  {\Funktionswerte}{\glstext}{Funktionswert}
\newglossaryentry{Funktionswert}{%ToDo prüfen
	name        ={Funktionswert \addIdx    {Funktionswert}},
	text        ={Funktionswert},
	description ={
		einer \Funktion.
	}
}

%G === G === G === G === G === G === G === G === G === G === G === G === G === G

\newVerweis     {\Gleichheit}{\glstext}{Gleichheit}
\newglossaryentry{Gleichheit}{%ToDo prüfen
	name        ={Gleichheit \addIdx   {Gleichheit}},
	text        ={Gleichheit},
	description ={
		Eine \Gleichheitsrelation:
		Zwei Objekte $A$ und $B$ sind \defFt{gleich} (dasselbe; identisch), $A \MtsEq B$, wenn sie in den \interessierendenEigenschaften\ für \MtsEq\ übereinstimmen.
	}
}

\newVerweis     {\Gleichheitsrelation}  {\glstext}{Gleichheitsrelation}
\newVerweis[en] {\Gleichheitsrelationen}{\glstext}{Gleichheitsrelation}
\newglossaryentry{Gleichheitsrelation}{%ToDo prüfen
	name        ={Gleichheitsrelation \addIdx     {Gleichheitsrelation}},
	text        ={Gleichheitsrelation},
	description ={
		Eine mit \Gleichheit\ verwandte \Relation: \MtsEq, \MtsEqN, \MtsAequiv\ und \MtsAequivN.
	}
}

\newVerweis     {\Gliederungszeichen}{\glstext}{Gliederungszeichen}
\newglossaryentry{Gliederungszeichen}{%ToDo prüfen
	name        ={Gliederungszeichen \addIdx   {Gliederungszeichen}},
	text        ={Gliederungszeichen},
	description ={
		\todo{Beschreibung fehlt noch}% ToDo=Gliederungszeichen
	}
}

\newVerweis     {\Graph}  {\glstext}{Graph}
\newVerweis[en] {\Graphen}{\glstext}{Graph}
\newglossaryentry{Graph}{%ToDo prüfen
	name        ={Graph \addIdx     {Graph}},
	text        ={Graph},
	see      ={MtsGraph},
	description ={
		einer \Funktion\ oder \Relation.
	}
}

%I === I === I === I === I === I === I === I === I === I === I === I === I === I

\newVerweis     {\Identitaetsregel} {\glstext}{Identitaetsregel}
\newVerweis[n]  {\Identitaetsregeln}{\glstext}{Identitaetsregel}
\newglossaryentry{Identitaetsregel}{%ToDo prüfen
	name        ={Identitätsregel \addIdx[
		name    ={Identitätsregel}]           {Identitaetsregel}},
	text        ={Identitätsregel},
	description ={
		Eigentlich eine \Basisregel\ zur Identität.
		Da die \Identitaetsregeln\ nur zur Rechtfertigung der \Ersetzung\ verwendet werden, werden sie hier nicht zu den \Basisregeln\ gezählt.
	}
}

%J === J === J === J === J === J === J === J === J === J === J === J === J === J

\newVerweis         {\Junktor}  {\glstext}{Junktor}
\newVerweis[en]     {\Junktoren}{\glstext}{Junktor}
\longnewglossaryentry{Junktor}{%ToDo prüfen
	name            ={Junktor \addIdx     {Junktor}},
	text            ={Junktor},
	see             ={Metajunktor},
}{
	\begin{wikicite}{bib:Junktor}
		Ein \wikibf{Junktor} (von \wikilink{lat.} \wikiit{iungere} „verknüpfen, verbinden“) ist eine \wikilink{logische Verknüpfung} zwischen Aussagen innerhalb der \wikilink{Aussagenlogik}, also ein logischer \wikilink{Operator}. Junktoren werden auch Konnektive, Konnektoren, Satzoperatoren, Satzverknüpfer, Satzverknüpfungen, Aussagenverknüpfer, logische Bindewörter, Verknüpfungszeichen oder Funktoren genannt und als \wikilink{logische Partikel} klassifiziert.

		Sprachlich wird zwischen der jeweiligen Verknüpfung selbst (zum Beispiel der \wikilink{Konjunktion}) und dem sie bezeichnenden Wort beziehungsweise Sprachzeichen (zum Beispiel dem Wort „und“ beziehungsweise dem Zeichen „\OjkAnd“) oft nicht unterschieden.

		[\textdots]
	\end{wikicite}
	Ein \GloFt{Junktor} ist eine \aussagenlogischeOperation\ oder -\aRelation.
	Da die Werte einer aussagenlogischen \Operation\ \Wahrheitswerte\ sind, kann man einen \Junktor\ auch stets als \Relation\ verstehen.
}

\newVerweis     {\binaererJunktor}  {\glstext} {binaererJunktor}
\newVerweis[en] {\binaerenJunktoren}{\glsuseri}{binaererJunktor}
\newglossaryentry{binaererJunktor}{
	name        =            {---, binärer \addIdx[
		name    =            {---, binärer},
		sort    =        {Junktor, binärer}]   {binaererJunktor}},
	sort        =        {Junktor, binärer},
	text        ={binärer Junktor},
	user1       ={binären Junktor},
	description ={
		\todo{Beschreibung fehlt noch}% ToDo=binärer Junktor
	}
}

\newVerweis     {\unaererJunktor}  {\glstext} {unaererJunktor}
\newVerweis[en] {\unaerenJunktoren}{\glsuseri}{unaererJunktor}
\newglossaryentry{unaererJunktor}{
	name        =           {---, unärer \addIdx[
		name    =           {---, unärer},
		sort    =       {Junktor, unärer}]    {unaererJunktor}},
	sort        =       {Junktor, unärer},
	text        ={unärer Junktor},
	user1       ={unären Junktor},
	description ={
		\todo{Beschreibung fehlt noch}% ToDo=unärer Junktor
	}
}

\newVerweis     {\Junktorsymbol} {\glstext}{Junktorsymbol}
\newVerweis[e]  {\Junktorsymbole}{\glstext}{Junktorsymbol}
\newglossaryentry{Junktorsymbol}{%ToDo prüfen
	name        ={Junktorsymbol \addIdx    {Junktorsymbol}},
	text        ={Junktorsymbol},
	description ={
		Ein \Symbol\ für einen \Junktor.
	}
}

%K === K === K === K === K === K === K === K === K === K === K === K === K === K

\newVerweis         {\Kalkuel} {\glstext}{Kalkuel}
\longnewglossaryentry{Kalkuel}{
	name            ={Kalkuel \addIdx    {Kalkuel}},
	text            ={Kalkül},
	see             ={Axiom,Logik},
}{
	\begin{wikicite}{bib:Kalkuel}
		Als der oder das \wikibf{Kalkül} (französisch \wikiit{calcul} „Rechnung“; von \wikilink{lateinisch} \wikiit{calculus} „\wikilink{Rechenstein}“, „\wikilink{Spielstein}“) versteht man in den formalen Wissenschaften wie \wikilink{Logik} und \wikilink{Mathematik} ein System von Regeln, mit denen sich aus gegebenen Aussagen (\wikilink{Axiomen}) weitere Aussagen ableiten lassen. Kalküle, auf eine Logik selbst angewandt, werden auch Logikkalküle genannt.

		[\textdots]
	\end{wikicite}
	\todo{Beschreibung fehlt noch}% ToDo=Kalkuel
}

\newVerweis     {\Klammerung}{\glstext}{Klammerung}
\newglossaryentry{Klammerung}{
	name        ={Klammerung \addIdx   {Klammerung}},
	text        ={Klammerung},
	description ={
		\todo{Beschreibung fehlt noch}% ToDo=Klammerung
	}
}

\newVerweis         {\Klasse} {\glstext}{Klasse}
\longnewglossaryentry{Klasse}{
	name            ={Klasse \addIdx    {Klasse}},
	text            ={Klasse},
	see             ={Menge,Mengenlehre}
}{
	\begin{wikicite}{bib:Klasse}
		Als \wikibf{Klasse} gilt in der \wikilink{Mathematik}, \wikilink{Klassenlogik} und \wikilink{Mengenlehre} eine Zusammenfassung beliebiger Objekte, definiert durch eine logische Eigenschaft, die alle Objekte der Klasse erfüllen. Vom Klassenbegriff ist der Mengenbegriff zu unterscheiden. Nicht alle Klassen sind automatisch auch Mengen, weil Mengen zusätzliche Bedingungen erfüllen müssen. Mengen sind aber stets Klassen und werden daher auch in der Praxis in Klassenschreibweise notiert.
	\end{wikicite}
	\todo{Beschreibung fehlt noch}% ToDo=Klasse
}

\newVerweis         {\Klassenlogik} {\glstext}{Klassenlogik}
\longnewglossaryentry{Klassenlogik}{
	name            ={Klassenlogik \addIdx    {Klassenlogik}},
	text            ={Klassenlogik},
	see             ={Klasse,Logik},
}{
	\begin{wikicite}{bib:Klassenlogik}
		Die \wikibf{Klassenlogik} ist im weiteren Sinn eine \wikilink{Logik}, deren Objekte als Klassen bezeichnet werden. Im engeren Sinn spricht man von einer Klassenlogik nur dann, wenn \wikilink{Klassen} durch eine Eigenschaft ihrer Elemente beschrieben werden. Diese Klassenlogik ist daher eine Verallgemeinerung der \wikilink{Mengenlehre}, die nur eine eingeschränkte Klassenbildung erlaubt.
	\end{wikicite}
	\todo{Beschreibung fehlt noch}% ToDo=Klassenlogik
}

\newVerweis     {\Komponente} {\glstext}{Komponente}
\newVerweis[n]  {\Komponenten}{\glstext}{Komponente}
\newglossaryentry{Komponente}{%ToDo prüfen
	name        ={Komponente \addIdx    {Komponente}},
	text        ={Komponente},
	see         ={Folge,Tupel},
	description ={
		Die \Komponenten\ einer \Folge\ $\vec{a} = (a_1, a_2, \dots)$ sind die $a_i$.
		$a_i$ heißt die \GloFt{$i$-te \Komponente} von $\vec{a}$.
	}
}

\newVerweis     {\Komponentenmenge}  {\glstext}{Komponentenmenge}
\newglossaryentry{Komponentenmenge}{
	name        ={Komponentenmenge \addIdx     {Komponentenmenge}},
	text        ={Komponentenmenge},
	see         ={Menge},
	description ={
		$\MtsSet(\vec{a}) \MtsDefEq \RawMengeDef{a}{a \MtsSeqIn \vec{a}}$ ist die \GloFt{Komponentenmenge} einer \Folge\ \textbzw\ eines \Tupels\ $\vec{a}$.
	}
}

\newVerweis     {\Komponentenrelation}  {\glstext}{Komponentenrelation}
\newVerweis[en] {\Komponentenrelationen}{\glstext}{Komponentenrelation}
\newglossaryentry{Komponentenrelation}{%ToDo prüfen
	name        ={Komponentenrelation \addIdx     {Komponentenrelation}},
	text        ={Komponentenrelation},
	see         ={Elementrelation},
	description ={
		Eine \GloFt{Komponentenrelation} ist eine Relation zwischen einer (möglichen) \Komponente\ und einer \Folge: \MtsSeqIn, \MtsSeqNi, \MtsSeqInN und \MtsSeqNiN
	}
}

\newVerweis     {\Konklusion}  {\glstext}{Konklusion}
\newVerweis[en] {\Konklusionen}{\glstext}{Konklusion}
\newglossaryentry{Konklusion}{
	name        ={Konklusion \addIdx     {Konklusion}},
	text        ={Konklusion},
	see         ={Schlussregel},
	description ={
		Eine \Ableitung:
		Die \Konklusionen\ einer \Schlussregel\ $\frac{\MtsPraemisseSet}{\MtsKonklusionSet}$ \textbzw\ $\frac{\MtsPraemisseSet}{\MtsKonklusionSet}$ sind die Elemente aus \MtsKonklusionSet\ \textbzw\ \MtsKonklusionRel.
		Die \Konklusionen\ werden normalerweise mit $\MtsKonklusion_i$ bezeichnet.
	}
}

\newVerweis     {\Konklusionsmenge} {\glstext}{Konklusionsmenge}
\newVerweis[n]  {\Konklusionsmengen}{\glstext}{Konklusionsmenge}
\newglossaryentry{Konklusionsmenge}{%ToDo prüfen
	name        ={Konklusionsmenge \addIdx    {Konklusionsmenge}},
	text        ={Konklusionsmenge},
	description ={
		Eine \Ableitungsmenge:
		Die \Menge\ \MtsKonklusionSet\ der \Konklusionen\ einer \Schlussregel\ \textbzw\ eines \Beweises.
	}
}

\newVerweis         {\Konstante} {\glstext}{Konstante}
\newVerweis[n]     {\Konstanten}{\glstext}{Konstante}
\longnewglossaryentry{Konstante}{%ToDo prüfen
	name            ={Konstante \addIdx    {Konstante}},
	text            ={Konstante},
	see             ={Symbol,Variable},
}{
	\begin{wikicite}{bib:Konstante}
		Allgemein ist eine \wikibf{Konstante} (von \wikilink{lateinisch} \wikiit{constans} „feststehend“) ein Zeichen beziehungsweise ein Sprachausdruck mit einer „genau bestimmte[n]Bedeutung, die im Laufe der Überlegungen unverändert bleibt“[1]. Die Konstante ist damit ein Gegenbegriff zur \wikilink{Variablen}.
	\end{wikicite}
}

\newVerweis      {\aussagenlogischeKonstante} {\glstext}        {aussagenlogischeKonstante}
\newVerweis     {\aussagenlogischenKonstante} {\glsuseri}       {aussagenlogischeKonstante}
\newVerweis[n]  {\aussagenlogischenKonstanten}{\glsuseri}       {aussagenlogischeKonstante}
\newglossaryentry {aussagenlogischeKonstante}{%ToDo prüfen
	name       =                        {---, aussagenlogische \addIdx[
		name   =                        {---, aussagenlogische},
		sort   =                  {Konstante, aussagenlogische}]{aussagenlogischeKonstante}},
	sort       =                  {Konstante, aussagenlogische},
	text       ={aussagenlogische  Konstante},
	user1      ={aussagenlogischen Konstante},
	description={
		Eine \Konstante\ heißt \GloFt{aussagenlogisch}, wenn sie ein Element von \OjkCon\ ist.
	}
}

\newVerweis     {\Kontraposition}{\glstext}{Kontraposition}
\newglossaryentry{Kontraposition}{%ToDo prüfen
	name        ={Kontraposition \addIdx   {Kontraposition}},
	text        ={Kontraposition},
	description ={
		Die allgemeingültige \Aussage: $ (\alpha \OjkImp \beta) \OjkImp (\OjkNot\beta \OjkImp \OjkNot\alpha) $.
	}
}

\newVerweis     {\Kontravalenz}{\glstext}{Kontravalenz}
\newglossaryentry{Kontravalenz}{%ToDo prüfen
	name        ={Kontravalenz \addIdx   {Kontravalenz}},
	text        ={Kontravalenz},
	description ={
		Eine \Gleichheitsrelation:
		Zwei Objekte $A$ und $B$ sind \defFt{nicht äquivalent} (nicht ähnlich), $A \MtsAequivN B$, wenn sie in mindestens einer \interessierendenEigenschaft\ für \MtsAequiv\ nicht übereinstimmen.
	}
}

%L === L === L === L === L === L === L === L === L === L === L === L === L === L

\newVerweis         {\Logik}  {\glstext}{Logik}
\newVerweis[en]     {\Logiken}{\glstext}{Logik}
\longnewglossaryentry{Logik}{%%% geprüft
	name            ={Logik \addIdx     {Logik}},
	text            ={Logik},
	see             ={atomar,Aussage,Aussagenlogik,Praedikatenlogik,Schlussregel},
}{
	\begin{wikicite}{bib:Logik}
		Mit \wikibf{Logik} (von \wikilink{altgriechisch}

		[\textdots]‚denkende Kunst‘, ‚Vorgehensweise‘) oder auch \wikibf{Folgerichtigkeit} wird im Allgemeinen das \wikilink{vernünftige Schlussfolgern} und im Besonderen dessen Lehre – die \wikibf{Schlussfolgerungslehre} oder auch \wikibf{Denklehre} – bezeichnet. In der Logik wird die Struktur von \wikilink{Argumenten} im Hinblick auf ihre \wikilink{Gültigkeit} untersucht, unabhängig vom Inhalt der \wikilink{Aussagen}. Bereits in diesem Sinne spricht man auch von „formaler“ Logik. Traditionell ist die Logik ein Teil der \wikilink{Philosophie}. Ursprünglich hat sich die traditionelle Logik in Nachbarschaft zur \wikilink{Rhetorik} entwickelt. Seit dem 20. Jahrhundert versteht man unter Logik überwiegend {symbolische Logik}, die auch als grundlegende \wikilink{Strukturwissenschaft}, z. B. innerhalb der \wikilink{Mathematik} und der \wikilink{theoretischen Informatik}, behandelt wird.

		Die moderne symbolische Logik verwendet statt der \wikilink{natürlichen Sprache} eine \wikilink{künstliche Sprache} (Ein Satz wie \wikiit{Der Apfel ist rot} wird z. B. in der \wikilink{Prädikatenlogik} als $f(a)$ formalisiert, wobei $a$ für \wikiit{Der Apfel} und $f$ für \wikiit{ist rot} steht) und verwendet streng \wikilink{definierte Schlussregeln}. Ein einfaches Beispiel für ein solches \wikilink{formales System} ist die \wikilink{Aussagenlogik} (dabei werden sogenannte \wikilink{atomare Aussagen} durch Buchstaben ersetzt). Die symbolische Logik nennt man auch \wikilink{mathematische Logik} oder formale Logik im engeren Sinn.
	\end{wikicite}
}

\newVerweis         {\mathematischeLogik}{\glstext}       {mathematischeLogik}
\longnewglossaryentry{mathematischeLogik}{%%% geprüft
	name            =                {---, mathematische \addIdx[
		name        =                {---, mathematische},
		sort        =              {Logik, mathematische}]{mathematischeLogik}},
	sort            =              {Logik, mathematische},
	text            ={mathematische Logik},
	see             ={Mengenlehre,Fachgebiet},
}{
	\begin{wikicite}{bib:mathematischeLogik}
		Die \wikibf{mathematische Logik}, auch \wikibf{symbolische Logik}, (alternativer Sprachgebrauch auch \wikiit{Logistik}), ist ein Teilgebiet der \wikilink{Mathematik}, insbesondere als Methode der \wikilink{Metamathematik} und eine Anwendung der modernen \wikilink{formalen Logik}. Oft wird sie wiederum in die Teilgebiete \wikilink{Modelltheorie}, \wikilink{Beweistheorie}, \wikilink{Mengenlehre} und \wikilink{Rekursionstheorie} aufgeteilt. Forschung im Bereich der mathematischen Logik hat zum Studium der \wikilink{Grundlagen der Mathematik} beigetragen und wurde auch durch dieses motiviert. Infolgedessen wurde sie auch unter dem Begriff \wikiit{Metamathematik} bekannt.

		Ein Aspekt der Untersuchungen der mathematischen Logik ist das Studium der Ausdrucksstärke von formalen Logiken und formalen \wikilink{Beweissystemen}. Eine Möglichkeit, die \wikilink{Komplexität} solcher Systeme zu messen, besteht darin, festzustellen, was damit bewiesen oder definiert werden kann.

		Früher wurde die mathematische Logik auch \wikiit{symbolische Logik} (als Gegensatz zur \wikilink{philosophischen Logik}) genannt, wobei jener Name mittlerweile nur noch für gewisse Aspekte der \wikilink{Beweistheorie} verwendet wird.
	\end{wikicite}
}

%M === M === M === M === M === M === M === M === M === M === M === M === M === M

\newVerweis         {\Menge} {\glstext}  {Menge}
\newVerweis[n]      {\Mengen}{\glstext}  {Menge}
\newVerweis      {\MtsSetSep}{\glssymbol}{Menge}
\longnewglossaryentry{Menge}{%%% geprüft
	name            ={Menge \addIdx    {Menge}},
	text            ={Menge},
	symbol          ={\ensuremath{\RawMtsSetSep}},
	see             ={Element,Folge,leereMenge,Mengenlehre,Tupel},
}{
	\begin{wikicite}{bib:Menge}
		Eine \wikibf{Menge} ist ein Verbund, eine Zusammenfassung von einzelnen \wikilink{Elementen}. Die \wikiit{Menge} ist eines der wichtigsten und grundlegenden Konzepte der Mathematik, mit ihrer Betrachtung beschäftigt sich die \wikilink{Mengenlehre}.

		Bei der Beschreibung einer Menge geht es ausschließlich um die Frage, welche Elemente in ihr enthalten sind. Es wird nicht danach gefragt, ob ein Element mehrmals enthalten ist oder ob es eine Reihenfolge unter den Elementen gibt. Eine Menge muss kein Element enthalten – es gibt genau eine Menge ohne Elemente, die „\wikilink{leere Menge}“. In der Mathematik sind die Elemente einer Menge häufig Zahlen, Punkte eines \wikilink{Raumes} oder ihrerseits Mengen. Das Konzept ist jedoch auf beliebige Objekte anwendbar: z. B. in der \wikilink{Statistik} auf Stichproben, in der Medizin auf Patientenakten, am Marktstand auf eine Tüte mit Früchten.

		Ist die Reihenfolge der Elemente von Bedeutung, dann spricht man von einer endlichen oder unendlichen \wikilink{Folge}, wenn sich die Folgenglieder mit den natürlichen Zahlen aufzählen lassen (das erste, das zweite, usw.). Endliche Folgen heißen auch \wikilink{Tupel}. In einem Tupel oder einer Folge können Elemente auch mehrfach vorkommen. Ein Gebilde, das wie eine Menge Elemente enthält, wobei es zusätzlich auf die Anzahl der Exemplare jedes Elements ankommt, jedoch nicht auf die Reihenfolge, heißt \wikilink{Multimenge}.
	\end{wikicite}
}

\newVerweis     {\leereMenge}{\glstext}{leereMenge}
\newglossaryentry{leereMenge}{
	name       =        {---, leere \addIdx[
		name   =        {---, leere},
		sort   =      {Menge, leere}]  {leereMenge}},
	sort       =      {Menge, leere},
	text       ={leere Menge},
	description={
		\MtsEmptyset, die \GloFt{leere Menge}, ist die einzige \Menge\ ohne \Elemente.
		Sie wird auch mit \seqqt{$\{\}$} bezeichnet.
	}
}

\newVerweis         {\Mengenlehre}{\glstext}{Mengenlehre}
\longnewglossaryentry{Mengenlehre}{%ToDo prüfen
	name            ={Mengenlehre \addIdx   {Mengenlehre}},
	text            ={Mengenlehre},
	see             ={Axiom,Fachgebiet,Menge,Objekt},
}{
	\begin{wikicite}{bib:Mengenlehre}
		Die \wikibf{Mengenlehre} ist ein grundlegendes \wikilink{Teilgebiet der Mathematik}, das sich mit der Untersuchung von \wikilink{Mengen}, also von Zusammenfassungen von \wikilink{Objekten}, beschäftigt. Die gesamte Mathematik, wie sie heute üblicherweise gelehrt wird, ist in der Sprache der Mengenlehre formuliert und baut auf den \wikilink{Axiomen der Mengenlehre} auf. Die meisten mathematischen Objekte, die in Teilbereichen wie \wikilink{Algebra}, \wikilink{Analysis}, \wikilink{Geometrie}, \wikilink{Stochastik} oder \wikilink{Topologie} behandelt werden, um nur einige wenige zu nennen, lassen sich als Mengen definieren. Gemessen daran ist die Mengenlehre eine recht junge Wissenschaft; erst nach der Überwindung der \wikilink{Grundlagenkrise der Mathematik} im frühen 20. Jahrhundert konnte die Mengenlehre ihren heutigen, zentralen und grundlegenden Platz in der Mathematik einnehmen.
	\end{wikicite}
}

\newVerweis     {\Mengenoperation}  {\glstext}{Mengenoperation}
\newVerweis[en] {\Mengenoperationen}{\glstext}{Mengenoperation}
\newglossaryentry{Mengenoperation}{
	name        ={Mengenoperation \addIdx     {Mengenoperation}},
	text        ={Mengenoperation},
	description ={
		\todo{Beschreibung fehlt noch}% ToDo=Mengenoperation
	}
}

\newsynonym{\Mengenprodukt}{Mengenprodukt}{\kartesischesProdukt}

\newVerweis     {\Mengenrelation}  {\glstext}{Mengenrelation}
\newVerweis[en] {\Mengenrelationen}{\glstext}{Mengenrelation}
\newglossaryentry{Mengenrelation}{
	name        ={Mengenrelation \addIdx     {Mengenrelation}},
	text        ={Mengenrelation},
	description ={
		\todo{Beschreibung fehlt noch}% ToDo=Mengenrelation
	}
}

\newVerweis     {\Metadefinition}  {\glstext}{Metadefinition}
\newVerweis[en] {\Metadefinitionen}{\glstext}{Metadefinition}
\newglossaryentry{Metadefinition}{
	name        ={Metadefinition \addIdx     {Metadefinition}},
	text        ={Metadefinition},
	description ={
		Eine \Metaoperation: Die formale Definition einer \Aussage\ (\Aussagedefinition) \textbzw\ eines \Objekts\ (\Objektdefinition).
	}
}

\newVerweis     {\Metaformel} {\glstext}{Metaformel}
\newVerweis[n]  {\Metaformeln}{\glstext}{Metaformel}
\newglossaryentry{Metaformel}{
	name        ={Metaformel \addIdx    {Metaformel}},
	text        ={Metaformel},
	description ={
		Eine \Formel\ der \formalenMetasprache.
	}
}

\newVerweis     {\Metajunktor}  {\glstext}{Metajunktor}
\newVerweis[en] {\Metajunktoren}{\glstext}{Metajunktor}
\newglossaryentry{Metajunktor}{
	name        ={Metajunktor \addIdx     {Metajunktor}},
	text        ={Metajunktor},
	see         ={Junktor},
	description ={
		\todo{Beschreibung fehlt noch}% ToDo=Metajunktor
	}
}

\newVerweis     {\Metaoperation}  {\glstext} {Metaoperation}
\newVerweis[en] {\Metaoperationen}{\glstext} {Metaoperation}
\newVerweis[en]    {\Moperationen}{\glsuseri}{Metaoperation}
\newglossaryentry{Metaoperation}{%ToDo prüfen
	name        ={Metaoperation \addIdx      {Metaoperation}},
	text        ={Metaoperation},
	user1       =    {operation},
	see         ={Objektoperation},
	description ={
		Eine \Operation\ der \Metasprache: \MtsAnd, \MtsOr\ oder \MtsUnd.
	}
}

\newVerweis     {\Metarelation}  {\glstext} {Metarelation}
\newVerweis[en] {\Metarelationen}{\glstext} {Metarelation}
\newVerweis[en]    {\Mrelationen}{\glsuseri}{Metarelation}
\newglossaryentry{Metarelation}{%ToDo prüfen
	name        ={Metarelation \addIdx      {Metarelation}},
	text        ={Metarelation},
	user1       =    {relation},
	see         ={Objektrelation},
	description ={
		Eine \Relation\ der \Metasprache: \MtsImp, \MtsRep\ oder \MtsEquiv.
	}
}

\newVerweis     {\Metasprache} {\glstext}{Metasprache}
\newVerweis[n]  {\Metasprachen}{\glstext}{Metasprache}
\newglossaryentry{Metasprache}{%%% geprüft
	name        ={Metasprache \addIdx    {Metasprache}},
	text        ={Metasprache},
	see         ={Objektsprache},
	description ={\MetaspracheDescription}
}
\newcommand*{\MetaspracheDescription}{
	Die \Sprache, in der \Aussagen\ über eine andere \Sprache\ getroffen werden können.
	Hier ist dies immer die normale Umgangssprache.
	Ihre \Syntax\ ist gegeben, \textbzgl\ der \Semantik\ bemühen wir uns um exakte Definitionen der \Begriffe\ und \Bezeichnungen.
}

\newVerweis      {\formaleMetasprache}{\glstext}  {formaleMetasprache}
\newVerweis     {\formalenMetasprache}{\glsuseri} {formaleMetasprache}
\newVerweis     {\formalenM}          {\glsuserii}{formaleMetasprache}
\newglossaryentry {formaleMetasprache}{%%% geprüft
	name       =                 {---, formale \addIdx[
		name   =                 {---, formale},
		sort   =         {Metasprache, formale}]  {formaleMetasprache}},
	sort       =         {Metasprache, formale},
	text       ={formale  Metasprache},
	user1      ={formalen Metasprache},
	user2      ={formalen},
	description={\formaleMetaspracheDescription}
}
\newcommand{\formaleMetaspracheDescription}{
	Die \Metasprache, deren Ausdrucksmittel nur \atomare\ \Aussagen\ und definierte \Metasymbole\ sind.
	Hier ist ihre Syntax und Semantik passend für \ASBA\ definiert, in der Regel parallel zur \Praedikatenlogik.
}

\newVerweis     {\Metasymbol} {\glstext}{Metasymbol}
\newVerweis[e]  {\Metasymbole}{\glstext}{Metasymbol}
\newglossaryentry{Metasymbol}{%ToDo prüfen
	name        ={Metasymbol \addIdx    {Metasymbol}},
	text        ={Metasymbol},
	see         ={Objektsymbol},
	description ={
		Ein \Symbol\ der \formalenMetasprache.
	}
}

\newVerweis     {\Metavariable} {\glstext} {Metavariable}
\newVerweis[n]     {\Mvariablen}{\glsuseri}{Metavariable}
\newglossaryentry{Metavariable}{%ToDo prüfen
	name        ={Metavariable \addIdx     {Metavariable}},
	text        ={Metavariable},
	user1       =    {variable},
	description ={
		Eine \Variable\ der \formalenMetasprache.
	}
}

\newVerweis     {\Monotonieregel}{\glstext}{Monotonieregel}
\newglossaryentry{Monotonieregel}{%ToDo prüfen
	name        ={Monotonieregel \addIdx   {Monotonieregel}},
	text        ={Monotonieregel},
	see         ={MR},
	description ={
		Eine \Schlussregel.
	}
}

%N === N === N === N === N === N === N === N === N === N === N === N === N === N

\newVerweis     {\natuerlicheZahl}  {\glstext} {natuerlicheZahl}
\newVerweis[en]{\natuerlichenZahlen}{\glsuseri}{natuerlicheZahl}
\newglossaryentry{natuerlicheZahl}{%ToDo prüfen
	name       =            {Zahl, natürliche \addIdx[
		name   =            {Zahl, natürliche}]{natuerlicheZahl}},
	text       ={natürliche  Zahl},
	user1      ={natürlichen Zahl},
	description={
		\todo{Beschreibung fehlt noch}% ToDo=natürliche Zahl
	}
}

\newVerweis     {\Negation}  {\glstext}{Negation}
\newVerweis[en] {\Negationen}{\glstext}{Negation}
\newglossaryentry{Negation}{%ToDo prüfen
	name        ={Negation \addIdx     {Negation}},
	text        ={Negation},
	description ={
		Die \GloFt{Negation} \emph{von} einer \binaeren\ \Relation\ $(G,A,B)$ ist die \Relation\ $(H,A,B)$ mit $H = (A \MtsTimes B) \MtsSetminus G\}$.
		Üblicherweise wird das zugehörige \Relationssymbol\ mit einem schrägen oder vertikalen Strich durchgestrichen.
		Die \gloFt{Negation} der \Umkehrrelation\ einer \Relation\ ist gleich der \Umkehrrelation\ ihrer \gloFt{Negation}.
	}
}

%O === O === O === O === O === O === O === O === O === O === O === O === O === O

\newVerweis     {\Oberaussage} {\glstext}{Oberaussage}
\newVerweis[n]  {\Oberaussagen}{\glstext}{Oberaussage}
\newglossaryentry{Oberaussage}{%ToDo prüfen
	name        ={Oberaussage \addIdx    {Oberaussage}},
	text        ={Oberaussage},
	description ={
		Eine \Aussage\ $A$ ist genau dann eine \GloFt{Oberaussage} einer \Aussage\ $B$, wenn $B$ eine \Teilaussage\ von $A$ ist.
	}
}

\newVerweis      {\echteOberaussage}{\glstext} {echteOberaussage}
\newVerweis     {\echtenOberaussage}{\glsuseri}{echteOberaussage}
\newglossaryentry {echteOberaussage}{%ToDo prüfen
	name       =               {---, echte \addIdx[
		name   =               {---, echte},
		sort   =       {Oberaussage, echte}]   {echteOberaussage}},
	sort       =       {Oberaussage, echte},
	text       ={echte  Oberaussage},
	user1      ={echten Oberaussage},
	description={
		Eine \Aussage\ $A$ ist genau dann eine \GloFt{echte Oberaussage} einer \Aussage\ $B$, wenn $B$ eine \echteTeilaussage\ von $A$ ist.
	}
}

\newVerweis     {\Oberfolge} {\glstext}{Oberfolge}
\newVerweis[n]  {\Oberfolgen}{\glstext}{Oberfolge}
\newglossaryentry{Oberfolge}{%ToDo prüfen
	name        ={Oberfolge \addIdx    {Oberfolge}},
	text        ={Oberfolge},
	description ={
		Eine \Folge\ $A$ ist genau dann eine \GloFt{Oberfolge} einer \Folge\ $B$, wenn $B$ eine \Teilfolge\ von $A$ ist.
	}
}

\newVerweis      {\echteOberfolge}{\glstext} {echteOberfolge}
\newVerweis     {\echtenOberfolge}{\glsuseri}{echteOberfolge}
\newglossaryentry {echteOberfolge}{%ToDo prüfen
	name       =              {---, echte \addIdx[
		name   =              {---, echte},
		sort   =       {Oberfolge, echte}]   {echteOberfolge}},
	sort       =       {Oberfolge, echte},
	text       ={echte  Oberfolge},
	user1      ={echten Oberfolge},
	description={
		Eine \Folge\ $A$ ist genau dann eine \GloFt{echte Oberfolge} einer \Folge\ $B$, wenn $B$ eine \echteTeilfolge\ von $A$ ist.
	}
}

\newVerweis     {\Oberformel} {\glstext}{Oberformel}
\newVerweis[n]  {\Oberformeln}{\glstext}{Oberformel}
\newglossaryentry{Oberformel}{%ToDo prüfen
	name        ={Oberformel \addIdx    {Oberformel}},
	text        ={Oberformel},
	description ={
		Eine \Formel\ $A$ ist genau dann eine \GloFt{Oberformel} einer \Formel\ $B$, wenn $B$ eine \Teilformel\ von $A$ ist.
	}
}

\newVerweis      {\echteOberformel}{\glstext} {echteOberformel}
\newVerweis     {\echtenOberformel}{\glsuseri}{echteOberformel}
\newglossaryentry {echteOberformel}{%ToDo prüfen
	name       =              {---, echte \addIdx[
		name   =              {---, echte},
		sort   =       {Oberformel, echte}]   {echteOberformel}},
	sort       =       {Oberformel, echte},
	text       ={echte  Oberformel},
	user1      ={echten Oberformel},
	description={
		Eine \Formel\ $A$ ist genau dann eine \GloFt{echte Oberformel} einer \Formel\ $B$, wenn $B$ eine \echteTeilformel\ von $A$ ist.
	}
}

\newVerweis     {\Obermenge} {\glstext}{Obermenge}
\newVerweis[n]  {\Obermengen}{\glstext}{Obermenge}
\newglossaryentry{Obermenge}{%ToDo prüfen
	name        ={Obermenge \addIdx    {Obermenge}},
	text        ={Obermenge},
	description ={
		Eine \Menge\ $A$ ist genau dann eine \GloFt{Obermenge} einer \Menge\ $B$, wenn $B$ eine \Teilmenge\ von $A$ ist.
	}
}

\newVerweis      {\echteObermenge}{\glstext} {echteObermenge}
\newVerweis     {\echtenObermenge}{\glsuseri}{echteObermenge}
\newglossaryentry {echteObermenge}{%ToDo prüfen
	name       =             {---, echte \addIdx[
		name   =             {---, echte},
		sort   =       {Obermenge, echte}]   {echteObermenge}},
	sort       =       {Obermenge, echte},
	text       ={echte  Obermenge},
	user1      ={echten Obermenge},
	description={
		Eine \Menge\ $A$ ist genau dann eine \GloFt{echte Obermenge} einer \Menge\ $B$, wenn $B$ eine \echteTeilmenge\ von $A$ ist.
	}
}

\newVerweis     {\Oberobjekt} {\glstext}{Oberobjekt}
\newVerweis[e]  {\Oberobjekte}{\glstext}{Oberobjekt}
\newglossaryentry{Oberobjekt}{%ToDo prüfen
	name        ={Oberobjekt \addIdx    {Oberobjekt}},
	text        ={Oberobjekt},
	description ={
		Eine \Objekt\ $A$ ist genau dann ein \GloFt{Oberobjekt} eines \Objekts\ $B$, wenn $B$ ein \Teilobjekt\ von $A$ ist.
	}
}

\newVerweis     {\echtesOberobjekt}{\glstext} {echtesOberobjekt}
\newVerweis     {\echtenOberobjekt}{\glsuseri}{echtesOberobjekt}
\newglossaryentry{echtesOberobjekt}{%ToDo prüfen
	name       =              {---, echtes \addIdx[
		name   =              {---, echtes},
		sort   =       {Oberobjekt, echtes}]  {echtesOberobjekt}},
	sort       =       {Oberobjekt, echtes},
	text       ={echtes Oberobjekt},
	user1      ={echten Oberobjekt},
	description={
		Ein \Objekt\ $A$ ist genau dann ein \GloFt{echtes Oberobjekt} eines \Objekts\ $B$, wenn $B$ ein \echtesTeilobjekt\ von $A$ ist.
	}
}

\newVerweis     {\Obersprache} {\glstext} {Obersprache}
\newVerweis[e]  {\Obersprachee}{\glstext} {Obersprache}
\newglossaryentry{Obersprache}{
	name        ={Obersprache \addIdx     {Obersprache}},
	text        ={Obersprache},
	user1       =    {sprache},
	description ={
		Eine \Sprache\ $A$ ist genau dann eine \GloFt{Obersprache} einer \Sprache\ $B$, wenn $B$ eine \Teilsprache\ von $A$ ist.% ToDo A,B --> L,
	}
}

\newVerweis     {\echteObersprache}{\glstext}  {echteObersprache}
\newglossaryentry{echteObersprache}{
	name        =               {---, echte \addIdx[
		name    =               {---, echte},
		sort    =       {Obersprache, echte}]  {echteObersprache}},
	sort        =       {Obersprache, echte},
	text        ={echte Obersprache},
	user1       ={echten Obersprache},
	user2       =           {sprache},
	description={
		Eine \Sprache\ $A$ ist genau dann eine \GloFt{echte Obersprache} einer \Sprache\ $B$, wenn $B$ eine \echteTeilsprache\ von $A$ ist.% ToDo A,B --> L,
	}
}

\newVerweis     {\Obersymbol} {\glstext}{Obersymbol}
\newVerweis[e]  {\Obersymbole}{\glstext}{Obersymbol}
\newglossaryentry{Obersymbol}{%ToDo prüfen
	name        ={Obersymbol \addIdx    {Obersymbol}},
	text        ={Obersymbol},
	description ={
		Eine \Symbol\ $A$ ist genau dann ein \GloFt{Obersymbol} eines \Symbols\ $B$, wenn $B$ ein \Teilsymbol\ von $A$ ist.
	}
}

\newVerweis     {\echtesObersymbol}{\glstext} {echtesObersymbol}
\newVerweis     {\echtenObersymbol}{\glsuseri}{echtesObersymbol}
\newglossaryentry{echtesObersymbol}{%ToDo prüfen
	name       =              {---, echtes \addIdx[
		name   =              {---, echtes},
		sort   =       {Obersymbol, echtes}]  {echtesObersymbol}},
	sort       =       {Obersymbol, echtes},
	text       ={echtes Obersymbol},
	user1      ={echten Obersymbol},
	description={
		Eine \Symbol\ $A$ ist genau dann ein \GloFt{echtes Obersymbol} eines \Symbols\ $B$, wenn $B$ ein \echtesTeilsymbol\ von $A$ ist.
	}
}

\newVerweis     {\Objekt}  {\glstext}{Objekt}
\newVerweis[e]  {\Objekte} {\glstext}{Objekt}
\newVerweis[s]  {\Objekts} {\glstext}{Objekt}
\newVerweis[en] {\Objekten}{\glstext}{Objekt}
\newglossaryentry{Objekt}{%ToDo prüfen
	name        ={Objekt \addIdx     {Objekt}},
	text        ={Objekt},
	description ={
		\Symbole, \Formeln\ und \Aussagen\ sowie Mengen, \Symbolfolgen, Zahlen; ganz allgemein reale oder gedachte Dinge an sich.
	}
}

\newVerweis     {\metasprachlichesObjekt} {\glstext}        {metasprachlichesObjekt}
\newVerweis      {\metasprachlicheObjekte}{\glspl}          {metasprachlichesObjekt}
\newglossaryentry{metasprachlichesObjekt}{%ToDo prüfen
	name       = {metasprachlichesObjekt \addIdx            {metasprachlichesObjekt}},
	name       =                    {---, metasprachliches \addIdx[
		name   =                    {---, metasprachliches},
		sort   =                 {Objekt, metasprachliches}]{metasprachlichesObjekt}},
	sort       =                 {Objekt, metasprachliches},
	text       ={metasprachliches Objekt},
	plural     ={metasprachliche  Objekte},
	description={
		Ein \Objekt\ der \Metasprache.
	}
}

\newVerweis     {\Objektart}  {\glstext}{Objektart}
\newVerweis[en] {\Objektarten}{\glstext}{Objektart}
\newglossaryentry{Objektart}{
	name        ={Objektart \addIdx     {Objektart}},
	text        ={Objektart},
	description ={
		\todo{Beschreibung fehlt noch}% ToDo=Objektart
	}
}

\newVerweis     {\Objektdefinition}  {\glstext}{Objektdefinition}
\newVerweis[en] {\Objektdefinitionen}{\glstext}{Objektdefinition}
\newglossaryentry{Objektdefinition}{
	name        ={Objektdefinition \addIdx     {Objektdefinition}},
	text        ={Objektdefinition},
	see         ={Aussagedefinition},
	description ={
		Eine \Metadefinition: Die formale Definition eines \Objekts.
		\ifmarginparFlg\newline\else\fi
		\seqqt{$A \MtsDefEq B$} steht für \standsfor{$A$ ist \defFt{definitionsgemäß gleich} $B$} für \Objekte\ $A$ und $B$.
		Gewissermaßen ist $A$ nur eine andere Schreibweise für $B$.
	}
}

\newVerweis     {\Objektformel} {\glstext}{Objektformel}
\newVerweis[n]  {\Objektformeln}{\glstext}{Objektformel}
\newglossaryentry{Objektformel}{
	name        ={Objektformel \addIdx    {Objektformel}},
	text        ={Objektformel},
	description ={
		Eine \Formel\ der \Objektsprache.
	}
}

\newVerweis     {\Objektkonstante} {\glstext}{Objektkonstante}
\newVerweis[n]  {\Objektkonstanten}{\glstext}{Objektkonstante}
\newglossaryentry{Objektkonstante}{
	name        ={Objektkonstante \addIdx    {Objektkonstante}},
	text        ={Objektkonstante},
	description ={
		Eine \Konstante\ der \Objektsprache.
	}
}

\newVerweis     {\Objektoperation}  {\glstext} {Objektoperation}
\newVerweis[en] {\Objektoperationen}{\glstext} {Objektoperation}
\newVerweis[en]      {\Ooperationen}{\glsuseri}{Objektoperation}
\newglossaryentry{Objektoperation}{%ToDo prüfen
	name        ={Objektoperation \addIdx      {Objektoperation}},
	text        ={Objektoperation},
	user1       =      {operation},
	see         ={Metaoperation},
	description ={
		Eine \Operation\ der \Objektsprache: \OjkAnd, \OjkOr.
	}
}

\newVerweis     {\Objektrelation}  {\glstext} {Objektrelation}
\newVerweis[en] {\Objektrelationen}{\glstext} {Objektrelation}
\newVerweis[en]      {\Orelationen}{\glsuseri}{Objektrelation}
\newglossaryentry{Objektrelation}{%ToDo prüfen
	name        ={Objektrelation \addIdx      {Objektrelation}},
	text        ={Objektrelation},
	user1       =      {relation},
	see         ={Metarelation},
	description ={
		Eine \Relation\ der \Objektsprache: \OjkImp, \OjkRep\ oder \OjkEquiv.
	}
}

\newVerweis     {\Objektsprache} {\glstext}{Objektsprache}
\newVerweis[n]  {\Objektsprachen}{\glstext}{Objektsprache}
\newglossaryentry{Objektsprache}{%%% geprüft
	name        ={Objektsprache \addIdx    {Objektsprache}},
	text        ={Objektsprache},
	description ={\ObjektspracheDescription}
}
\newcommand*{\ObjektspracheDescription}{
	Die \Sprache, über die mittels einer (\formalenM) \Metasprache\ "`geredet"' wird.
	Unser \Objekt, mit dem mathematische \Beweise\ formuliert werden sollen, ist die \Logik.
	Demnach sind die Ausdrucksmittel der \Objektsprache\ die der \Logik.
	Wir verwenden hier die \Praedikatenlogik\ oder, als \echteTeilsprache, die \Aussagenlogik.
}

\newVerweis     {\Objektsymbol} {\glstext}{Objektsymbol}
\newVerweis[e]  {\Objektsymbole}{\glstext}{Objektsymbol}
\newglossaryentry{Objektsymbol}{%ToDo prüfen
	name        ={Objektsymbol \addIdx    {Objektsymbol}},
	text        ={Objektsymbol},
	see         ={Metasymbol},
	description ={
		Ein \Symbol\ der \Objektsprache.
	}
}

\newVerweis     {\Operation}  {\glstext}{Operation}
\newVerweis[en] {\Operationen}{\glstext}{Operation}
\newglossaryentry{Operation}{%ToDo prüfen
	name        ={Operation \addIdx     {Operation}},
	text        ={Operation},
	description ={
		Eine \GloFt{Operation} ist eine --- meistens \binaere, \textdh\ zweiwertige --- \Funktion\ $M^n \MtsFktArrow M$.
		Für eine \binaere \Operation\ $\FunktionDef{\BspOpB}{M \MtsTimes M}{M}$ schreibt man meistens $x \BspOpB y$ statt $\BspOpB(x,y)$.
	}
}

\newVerweis      {\aussagenlogischeOperation}  {\glstext}       {aussagenlogischeOperation}
\newVerweis[en]  {\aussagenlogischeOperationen}{\glstext}       {aussagenlogischeOperation}
\newVerweis[en] {\aussagenlogischenOperationen}{\glsuseri}      {aussagenlogischeOperation}
\newVerweis[en]                 {\aOperationen}{\glsuserii}     {aussagenlogischeOperation}
\newglossaryentry {aussagenlogischeOperation}{
	name       =                        {---, aussagenlogische \addIdx[
		name   =                        {---, aussagenlogische},
		sort   =                  {Operation, aussagenlogische}]{aussagenlogischeOperation}},
	sort       =                  {Operation, aussagenlogische},
	text       ={aussagenlogische  Operation},
	user1      ={aussagenlogischen Operation},
	user2      =                  {Operation},
	description={
		Die \GloFt{aussagenlogischen} \Operationen\ sind ...%ToDo=aussagenlogische Operationen
	}
}

\newVerweis     {\Operationssymbol} {\glstext}{Operationssymbol}
\newVerweis[e]  {\Operationssymbole}{\glstext}{Operationssymbol}
\newglossaryentry{Operationssymbol}{%ToDo prüfen
	name        ={Operationssymbol \addIdx    {Operationssymbol}},
	text        ={Operationssymbol},
	description ={
		Ein \Symbol\ für eine \Operation.
	}
}

\newVerweis         {\Ordnungsrelation}  {\glstext}{Ordnungrelation}
\newVerweis[en]     {\Ordnungsrelationen}{\glstext}{Ordnungrelation}
\longnewglossaryentry{Ordnungsrelation}{%ToDo prüfen
	name            ={Ordnungsrelation \addIdx[
		name        ={Ordnungsrelation}]           {Ordnungsrelation}},
	text            ={Ordnungsrelation},
}{
	Eine \GloFt{Ordnungsrelation} ist ein \binaere\ \Relation\ auf einer \Menge\ $M$ mit der folgenden Eigenschaft
	(dabei sei $\preceq$ die \gloFt{Ordnungsrelation}):
	\begin{align}
		&\text{\textbf{transitiv }}:\qquad ((a \preceq b) \MtsAnd (b \preceq c)) \MtsImp (a \preceq c) \formulatoleft
	\end{align}
	jeweils für alle Elemente $a$, $b$ und $c$ aus $M$.
}

%P === P === P === P === P === P === P === P === P === P === P === P === P === P

\newVerweis     {\geordnetesPaar} {\glstext}  {geordnetesPaar}
\newVerweis[e]  {\geordnetenPaare}{\glsuseri} {geordnetesPaar}
\newglossaryentry{geordnetesPaar}{
	name       =           {Paar, geordnetes \addIdx[
		name   =           {Paar, geordnetes}]{geordnetesPaar}},
	text       ={geordnetes Paar},
	user1      ={geordneten Paar},
	description={
		\todo{Beschreibung fehlt noch}% ToDo=geordnetes Paar
	}
}

\newVerweis      {\PolnischeNotation}  {\glstext}  {PolnischeNotation}
\newVerweis[en]  {\PolnischeNotationen}{\glstext}  {PolnischeNotation}
\newVerweis      {\PolnischenNotation} {\glsuseri} {PolnischeNotation}
\newVerweis      {\PolnischerNotation} {\glsuserii}{PolnischeNotation}
\newglossaryentry{PolnischeNotation}{%ToDo prüfen
	name        =           {Notation, Polnische \addIdx[
		name    =           {Notation, Polnische},
		text    ={Polnische  Notation}]            {PolnischeNotation}},
	text        ={Polnische  Notation},
	user1       ={Polnischen Notation},
	user2       ={Polnischer Notation},
	description ={
		Bei der \GloFt{Polnischen Notation} stehen die Argumente von \Relationen\ und \Funktionen\ stets rechts von den \RelationsS- und \Funktionssymbolen.
		Dadurch kann auf \Gliederungszeichen\ wie Klammern und Kommata verzichtet werden.
		Noch einfacher für Computer ist die \GloFt{umgekehrte} \gloFt{Polnische Notation}, bei der die Argumente immer links stehen.
	}
}

\newVerweis     {\Potenzmenge} {\glstext}{Potenzmenge}
\newVerweis[n]  {\Potenzmengen}{\glstext}{Potenzmenge}
\newglossaryentry{Potenzmenge}{%ToDo prüfen
	name        ={Potenzmenge \addIdx    {Potenzmenge}},
	text        ={Potenzmenge},
	description ={
		Die \Potenzmenge\ $\MtsPot(M)$ einer \Menge\ $M$ ist die \Menge\ ihrer \Teilmengen.
	}
}

\newVerweis     {\Praedikat} {\glstext}{Praedikat}
\newVerweis[e]  {\Praedikate}{\glstext}{Praedikat}
\newVerweis[s]  {\Praedikats}{\glstext}{Praedikat}
\newglossaryentry{Praedikat}{%ToDo prüfen
	name        ={Prädikat \addIdx[
		name    ={Prädikat}]           {Praedikat}},
	text        ={Prädikat},
	description ={
		Ein Element der \Praedikatenlogik. ---
		\textZB\ kann man eine Gruppe als ein zwei\stelliges\ \Praedikat\ $\mathrm{Gruppe}(G,+)$ definieren, in dem $G$ eine \Menge\ und $+$ eine \Operation, \textdh\ eine \binaere\ (zwei\stellige) \Funktion\ $ +: G \MtsTimes G \rightarrow G $ ist, so dass die Gruppenaxiome erfüllt sind.
	}
}

\newVerweis         {\Praedikatenlogik}{\glstext}{Praedikatenlogik}
\longnewglossaryentry{Praedikatenlogik}{
	name            ={Prädikatenlogik \addIdx[
		name        ={Prädikatenlogik}]          {Praedikatenlogik}},
	text            ={Prädikatenlogik},
	see             ={Aussagenlogik,Logik},
}{
	\begin{wikicite}{bib:Praedikatenlogik}
		Die \wikibf{Prädikatenlogiken} (auch \wikibf{Quantorenlogiken}) bilden eine Familie \wikilink{logischer} Systeme, die es erlauben, einen weiten und in der Praxis vieler Wissenschaften und deren Anwendungen wichtigen Bereich von Argumenten zu formalisieren und auf ihre Gültigkeit zu überprüfen. Auf Grund dieser Eigenschaft spielt die Prädikatenlogik eine große Rolle in der \wikilink{Logik} sowie in \wikilink{Mathematik}, \wikilink{Informatik}, \wikilink{Linguistik} und \wikilink{Philosophie}.

		[\textdots]
	\end{wikicite}
}

\newVerweis     {\Praemisse}  {\glstext}{Praemisse}
\newVerweis[n]  {\Praemissen}{\glstext}{Praemisse}
\newglossaryentry{Praemisse}{%ToDo prüfen
	name        ={Prämisse \addIdx      {Praemisse}},
	text        ={Prämisse},
	see         ={Schlussregel},
	description ={
		Eine \Ableitung:
		Die \Praemissen\ einer \Schlussregel\ $\frac{\MtsPraemisseSet}{\MtsKonklusionSet}$ \textbzw\ $\frac{\MtsPraemisseSet}{\MtsKonklusionSet}$ sind die Elemente aus \MtsPraemisseSet\ \textbzw\ \MtsPraemisseRel.
		Die \Praemissen\ werden normalerweise mit $\MtsPraemisse_i$ bezeichnet.
	}
}

\newVerweis     {\Praemissenmenge} {\glstext}{Praemissenmenge}
\newVerweis[n]  {\Praemissenmengen}{\glstext}{Praemissenmenge}
\newglossaryentry{Praemissenmenge}{%ToDo prüfen
	name        = {Prämissenmenge \addIdx    {Praemissenmenge}},
	text        = {Prämissenmenge},
	description ={
		Eine \Ableitungsmenge:
		Die \Menge\ \MtsPraemisseSet\ der \Praemissen\ einer \Schlussregel\ \textbzw\ eines \Beweises.
	}
}

\newVerweis         {\kartesischesProdukt}{\glstext}      {kartesischesProdukt}
\newVerweis          {\kartesischeProdukt}{\glsuseri}     {kartesischesProdukt}
\longnewglossaryentry{kartesischesProdukt}{
	name            =             {Produkt, kartesisches \addIdx[
		name        =             {Produkt, kartesisches}]{kartesischesProdukt}},
	text            ={kartesisches Produkt},
	user1           ={kartesische  Produkt},
}{
	\begin{wikicite}{bib:kartesischesProdukt}
		Das \wikibf{kartesische Produkt} oder \wikibf{Mengenprodukt} ist in der Mengenlehre eine grundlegende Konstruktion, aus gegebenen Mengen eine neue Menge zu erzeugen. [\textdots]Das kartesische Produkt zweier Mengen ist die Menge aller geordneten Paare von Elementen der beiden Mengen, wobei die erste Komponente ein Element der ersten Menge und die zweite Komponente ein Element der zweiten Menge ist. Allgemeiner besteht das kartesische Produkt mehrerer Mengen aus der Menge aller Tupel von Elementen der Mengen, wobei die Reihenfolge der Mengen und damit der entsprechenden Elemente fest vorgegeben ist. Die Ergebnismenge des kartesischen Produkts wird auch \wikibf{Produktmenge}, \wikibf{Kreuzmenge} oder \wikibf{Verbindungsmenge} genannt. [\textdots]
	\end{wikicite}
}

%Q === Q === Q === Q === Q === Q === Q === Q === Q === Q === Q === Q === Q === Q

\newVerweis[en]     {\Quantoren}{\glstext}{Quantor}
\newVerweis         {\Quantor}  {\glstext}{Quantor}
\longnewglossaryentry{Quantor}{
	name            ={Quantor \addIdx     {Quantor}},
	text            ={Quantor},
	see             ={Allquantor,Existenzquantor,Junktor,Praedikatenlogik},
}{
	\begin{wikicite}{bib:Quantor}
		Ein \wikibf{Quantor} oder \wikibf{Quantifikator}, die Re-Latinisierung des von \wikilink{C. S. Peirce} eingeführten Ausdrucks „quantifier“, ist ein \wikilink{Operator} der \wikilink{Prädikatenlogik}. Neben den \wikilink{Junktoren} sind die Quantoren Grundzeichen der Prädikatenlogik. Allen Quantoren gemeinsam ist, dass sie \wikilink{Variablen} \wikilink{binden}.

		Die beiden gebräuchlichsten Quantoren sind der \wikiit{Existenzquantor} (in natürlicher Sprache zum Beispiel als „mindestens ein“ ausgedrückt) und der \wikiit{Allquantor} (in natürlicher Sprache zum Beispiel als „alle“ oder „jede/r/s“ ausgedrückt). Andere Arten von Quantoren sind \wikiit{Anzahlquantoren} wie „ein“ oder „zwei“, die sich auf Existenz- beziehungsweise Allquantor zurückführen lassen, und Quantoren wie „manche“, „einige“ oder „viele“, die auf Grund ihrer Unbestimmtheit in der \wikilink{klassischen Logik} nicht verwendet werden.
	\end{wikicite}
	\todo{Beschreibung fehlt noch}% ToDo=Quantor
}

\newVerweis     {\logischerQuantor} {\glstext} {logischerQuantor}
\newglossaryentry{logischerQuantor}{
	name       =              {---, logischer \addIdx[
		name   =              {---, logischer},
		sort   =          {Quantor, logischer}]{logischerQuantor}},
	sort       =          {Quantor, logischer},
	text       ={logischer Quantor},
	description={
		\todo{Beschreibung fehlt noch}% ToDo=logischer Quantor
	}
}

\newVerweis     {\metasprachlicherQuantor} {\glstext}        {metasprachlicherQuantor}
\newglossaryentry{metasprachlicherQuantor}{
	name       =                     {---, metasprachlicher \addIdx[
		name   =                     {---, metasprachlicher},
		sort   =                 {Quantor, metasprachlicher}]{metasprachlicherQuantor}},
	sort       =                 {Quantor, metasprachlicher},
	text       ={metasprachlicher Quantor},
	description={
		\todo{Beschreibung fehlt noch}% ToDo=metasprachlicher Quantor
	}
}

\newVerweis     {\Quellbereich} {\glstext} {Quellbereich}
\newVerweis[e]  {\Quellbereiche}{\glstext} {Quellbereich}
\newVerweis     {\QuellB}       {\glsuseri}{Quellbereich}
\newglossaryentry{Quellbereich}{
	name        ={Quellbereich \addIdx     {Quellbereich}},
	text        ={Quellbereich},
	user1       ={Quell},
	see         ={Definitionsbereich,Menge},
	description ={
		Für die \Funktion \FunktionDef{f}{A}{B} ist die \Menge\ $\MtsQb(f) \MtsDefEq \RawMengeDef{a \in A}{f(a) \text{ existiert}}$ ihr \Quellbereich%
		\footnote{%
			Der \GloFt{Quellbereich} $\MtsQb(f)$ unterscheidet sich nur bei \defFt{partiellen} \Funktionen\ vom \Definitionsbereich\ $\MtsDb(f)$, \textdh\ solchen \Funktionen, für die $f(a)$ nicht für alle $a \MtsIn A$ definiert ist.
		}
		(source).
	}
}

%R === R === R === R === R === R === R === R === R === R === R === R === R === R

\newVerweis         {\Relation}  {\glstext}{Relation}
\newVerweis[en]     {\Relationen}{\glstext}{Relation}
\longnewglossaryentry{Relation}{%ToDo prüfen
	name            ={Relation \addIdx     {Relation}},
	text            ={Relation},
	see             ={Aequivalenzrelation,Begriff,Menge,Objekt,Ordnungsrelation},
}{
	\begin{wikicite}{bib:Relation}
		Eine \wikibf{Relation} (\wikilink{lateinisch} \wikiit{relatio} „Beziehung“, „Verhältnis“) ist allgemein eine Beziehung, die zwischen Dingen bestehen kann. Relationen im Sinne der \wikilink{Mathematik} sind ausschließlich diejenigen Beziehungen, bei denen stets klar ist, ob sie bestehen oder nicht; Objekte können also nicht „bis zu einem gewissen Grade“ in einer Relation zueinander stehen. Damit ist eine einfache \wikilink{mengentheoretische} Definition des Begriffs möglich: Eine Relation $R$ ist eine Menge von $n$-\wikilink{Tupeln}. In der Relation $R$ zueinander stehende Dinge bilden $n$-Tupel, die Element von $R$ sind.

		Wird nicht ausdrücklich etwas anderes angegeben, versteht man unter einer Relation gemeinhin eine zweistellige oder binäre Relation. Bei einer solchen Beziehung bilden dann jeweils zwei Elemente $a$ und $b$ ein \wikilink{geordnetes Paar} $(a,b)$. Stammen dabei $a$ und $b$ aus verschiedenen Grundmengen $A$ und $B$, so heißt die Relation \wikiit{heterogen} oder „Relation \wikiit{zwischen} den Mengen $A$ und $B$.“ Stimmen die Grundmengen überein ($A = B$), dann heißt die Relation \wikiit{homogen} oder „Relation \wikiit{in} bzw. \wikiit{auf} der Menge $A$.“

		Wichtige Spezialfälle, zum Beispiel \wikilink{Äquivalenzrelationen} und \wikilink{Ordnungsrelationen}, sind Relationen \wikiit{auf} einer Menge.

		Heute sehen manche Autoren den Begriff Relation nicht unbedingt als auf Mengen beschränkt an, sondern lassen jede aus geordneten Paaren bestehende \wikilink{Klasse} als Relation gelten.
	\end{wikicite}
	Eine \defFt{$n$-\stellige} \GloFt{Relation} $R$ ist ein (1+$n$)-\Tupel\ $(G,A_1,\dots,A_n)$ mit $G \MtsSubsetEq A_1 \MtsTimes \dots \MtsTimes A_n)$.
}

\newVerweis      {\aussagenlogischeRelation}  {\glstext}       {aussagenlogischeRelation}
\newVerweis[en]  {\aussagenlogischeRelationen}{\glstext}       {aussagenlogischeRelation}
\newVerweis[en] {\aussagenlogischenRelationen}{\glsuseri}      {aussagenlogischeRelation}
\newVerweis                     {\aRelation}  {\glsuserii}     {aussagenlogischeRelation}
\newVerweis[en]         {\aRelationen}{\glsuserii}{aussagenlogischeRelation}
\newglossaryentry {aussagenlogischeRelation}{
	name       =                       {---, aussagenlogische \addIdx[
		name   =                       {---, aussagenlogische},
		sort   =                  {Relation, aussagenlogische}]{aussagenlogischeRelation}},
	sort       =                  {Relation, aussagenlogische},
	text       ={aussagenlogische  Relation},
	user1      ={aussagenlogischen Relation},
	user2      =                  {Relation},
	description={
		Die \GloFt{aussagenlogischen} \Relationen\ sind ...%ToDo=aussagenlogische Relationen
	}
}

\newVerweis     {\Relationssymbol} {\glstext} {Relationssymbol}
\newVerweis[e]  {\Relationssymbole}{\glstext} {Relationssymbol}
\newVerweis     {\RelationsS}      {\glsuseri}{Relationssymbol}
\newglossaryentry{Relationssymbol}{%ToDo prüfen
	name        ={Relationssymbol \addIdx     {Relationssymbol}},
	text        ={Relationssymbol},
	user1       ={Relations},
	description ={
		Ein \Symbol\ für eine \Relation.
	}
}

%S === S === S === S === S === S === S === S === S === S === S === S === S === S

\newVerweis     {\Satz}   {\glstext}{Satz}
\newVerweis[es] {\Satzes} {\glstext}{Satz}
\newVerweis     {\Saetze} {\glspl}  {Satz}
\newVerweis[n]  {\Saetzen}{\glspl}  {Satz}
\newglossaryentry{Satz}{%%% geprüft
	name        ={Satz \addIdx      {Satz}},
	text        ={Satz},
	plural      ={Sätze},
	description ={\SatzDescription}
}
\newcommand*{\SatzDescription}{
	Ein \GloFt{Satz} ist eine \Aussage, bestehend aus einer Anzahl von \Praemissen\ und \Konklusionen\ und einem \Beweis, der die Konklusionen aus den Praemissen ableitet.
}

\newVerweis     {\formalerSatz} {\glstext} {formalerSatz}
\newVerweis     {\formalenSatz} {\glsuseri}{formalerSatz}
\newglossaryentry{formalerSatz}{%ToDo prüfen
	name       =          {---, formaler \addIdx[
		name   =          {---, formaler},
		sort   =         {Satz, formaler}] {formalerSatz}},
	sort       =         {Satz, formaler},
	text       ={formaler Satz},
	user1      ={formalen Satz},
	see        ={FS},
	description={
		Formale \Darstellung\ eines mathematischen \Satzes.
	}
}

\newVerweis         {\Schlussregel} {\glstext}{Schlussregel}
\newVerweis[n]      {\Schlussregeln}{\glstext}{Schlussregel}
\longnewglossaryentry{Schlussregel}{%ToDo prüfen
	name            ={Schlussregel \addIdx    {Schlussregel}},
	text            ={Schlussregel},
	see             ={MtsSchlussregel,MtsSchlussregelSet,Kalkuel},
}{
	\begin{wikicite}{bib:Schlussregel}
		Eine \wikibf{Schlussregel} (oder \wikiit{Inferenzregel}) bezeichnet eine Transformationsregel (Umformungsregel) in einem \wikilink{Kalkül} der \wikilink{formalen Logik}, d. h. eine \wikilink{syntaktische} Regel, nach der es erlaubt ist, von bestehenden Ausdrücken einer formalen Sprache zu neuen Ausdrücken überzugehen. Dieser regelgeleitete Übergang stellt eine \wikilink{Schlussfolgerung} dar.
	\end{wikicite}
	Eine \Schlussregel\ $\frac{\MtsPraemisseSet}{\MtsKonklusionSet}$ entspricht der \Aussage:
	\begin{quote}
		Wenn alle \Praemissen\ $\MtsPraemisse \MtsIn \MtsPraemisseSet$ zutreffen, dann auch alle \Konklusionen\ $\MtsKonklusion \MtsIn \MtsKonklusionSet$.
	\end{quote}
	Wenn diese \Aussage\ zutrifft, kann die Schlussregel zur \zulaessigen\ \Transformation\ von \Formeln\ dienen.
}

\newVerweis     {\allgemeingueltigeSchlussregel} {\glstext}        {allgemeingueltigeSchlussregel}
\newVerweis[n]  {\allgemeingueltigeSchlussregeln}{\glstext}        {allgemeingueltigeSchlussregel}
\newVerweis    {\allgemeingueltigenSchlussregel} {\glsuseri}       {allgemeingueltigeSchlussregel}
\newVerweis[n] {\allgemeingueltigenSchlussregeln}{\glsuseri}       {allgemeingueltigeSchlussregel}
\newglossaryentry{allgemeingueltigeSchlussregel}{%ToDo prüfen
	name       =                           {---, allgemeingültige \addIdx[
		name   =                           {---, allgemeingültige},
		sort   =                  {Schlussregel, allgemeingültige}]{allgemeingueltigeSchlussregel}},
	sort       =                  {Schlussregel, allgemeingültige},
	text       ={allgemeingültige  Schlussregel},
	user1      ={allgemeingültigen Schlussregel},
	description={
		Eine \Schlussregel\ heißt \GloFt{allgemeingültig}, wenn sie aus den \Basisregeln\ und schon bekannten \allgemeingueltigenSchlussregeln\ abgeleitet werden kann.
	}
}

\newVerweis     {\Schlussregelmenge} {\glstext}{Schlussregelmenge}
\newcommand*    {\Schlussregelmengen}[1][]{\glstext[#1]{Schlussregelmenge}n[]}
\newglossaryentry{Schlussregelmenge}{%ToDo prüfen
	name        ={Schlussregelmenge \addIdx    {Schlussregelmenge}},
	text        ={Schlussregelmenge},
	see         ={MtsSchlussregelSet},
	description ={
		Eine \Menge\ von \Schlussregeln, meistens mit \MtsSchlussregelSet\ bezeichnet.
	}
}

\newVerweis     {\Schnittregel}{\glstext}{Schnittregel}
\newglossaryentry{Schnittregel}{%ToDo prüfen
	name        ={Schnittregel \addIdx   {Schnittregel}},
	text        ={Schnittregel},
	see         ={SR},
	description ={
		Eine \allgemeingueltigeSchlussregel.
	}
}

\newVerweis         {\Semantik} {\glstext}{Semantik}
\longnewglossaryentry{Semantik}{%%% geprüft
	name            ={Semantik \addIdx    {Semantik}},
	text            ={Semantik},
}{
	\begin{wikicite}{bib:Wikipedia}
		\wikibf{Semantik} [\textdots], auch \wikibf{Bedeutungslehre}, nennt man die Theorie oder Wissenschaft von der Bedeutung der Zeichen. \wikiit{Zeichen} können hierbei beliebige \wikilink{Symbole} sein, insbesondere aber auch \wikilink{Sätze}, Satzteile, \wikilink{Wörter} oder \wikilink{Wortteile}.

		[\textdots]
	\end{wikicite}
	In der \formalenMetasprache\ und der \Objektsprache\ sind die Zeichen die \Symbole\ und \Formeln.
}

\newVerweis         {\Signatur}{\glstext}{Signatur}
\longnewglossaryentry{Signatur}{%ToDo prüfen
	name            ={Signatur \addIdx   {Signatur}},
	text            ={Signatur},
	see             ={Abbildung,Logik,Praedikatenlogik,Sprache,Stelligkeit,Symbol},
}{
	\begin{wikicite}{bib:Signatur}
		In der \wikilink{mathematischen Logik} besteht eine \wikibf{Signatur} aus der \wikilink{Menge} der \wikilink{Symbole}, die in der betrachteten \wikilink{Sprache} zu den üblichen, rein logischen Symbolen hinzukommt, und einer \wikilink{Abbildung}, die jedem Symbol der Signatur eine \wikilink{Stelligkeit} eindeutig zuordnet. Während die logischen Symbole wie  $\forall ,\exists ,\land ,\lor ,\rightarrow ,\leftrightarrow ,\neg$ stets als „für alle“, „es gibt ein“, „und“, „oder“, „folgt“, „äquivalent zu“ bzw. „nicht“ interpretiert werden, können durch die semantische \wikilink{Interpretation} der Symbole der Signatur verschiedene \wikilink{Strukturen} (insbesondere Modelle von Aussagen der Logik) unterschieden werden. Die Signatur ist der spezifische Teil einer \wikilink{elementaren Sprache}.

		Beispielsweise lässt sich die gesamte \wikilink{Zermelo-Fraenkel-Mengenlehre} in der Sprache der \wikilink{Prädikatenlogik erster Stufe} und dem einzigen Symbol \MtsIn (neben den rein logischen Symbolen) formulieren; in diesem Fall ist die Symbolmenge der Signatur gleich $\{\MtsIn\}$.
	\end{wikicite}
}

\newVerweis      {\BoolescheSignatur}{\glstext}  {BoolescheSignatur}
\newVerweis     {\BooleschenSignatur}{\glsuseri} {BoolescheSignatur}
\newglossaryentry {BoolescheSignatur}{%ToDo prüfen
	name       =                {---, Boolesche \addIdx[
		name   =                {---, Boolesche},
		sort   =           {Signatur, Boolesche}]{BoolescheSignatur}},
	sort       =           {Signatur, Boolesche},
	text       ={Boolesche  Signatur},
	user1      ={Booleschen Signatur},
	description={
		Die \logischeSignatur\ $\{\OjkNot, \OjkAnd, \OjkOr\}$.
	}
}

\newVerweis      {\logischeSignatur}  {\glstext} {logischeSignatur}
\newVerweis[en]  {\logischeSignaturen}{\glstext} {logischeSignatur}
\newVerweis     {\logischenSignatur}  {\glsuseri}{logischeSignatur}
\newglossaryentry {logischeSignatur}{%ToDo prüfen
	name       =               {---, logische \addIdx[
		name   =               {---, logische},
		sort   =          {Signatur, logische}]  {logischeSignatur}},
	sort       =          {Signatur, logische},
	text       ={logische  Signatur},
	user1      ={logischen Signatur},
	description={
		Abweichend von der Definition von \Signatur\ in \Wikipedia\ ist eine \GloFt{logische Signatur} eine \Teilmenge\ von \OjkJun, ausreichend um damit und mit \OjkVar\ und Klammerung alle anderen Elemente aus \OjkJun\ zu definieren.
	}
}

\newVerweis     {\Sprache} {\glstext}{Sprache}
\newVerweis[n]  {\Sprachen}{\glstext}{Sprache}
\newglossaryentry{Sprache}{%ToDo prüfen
	name        ={Sprache \addIdx    {Sprache}},
	text        ={Sprache},
	description ={
		--- Siehe \Formelmenge.
	}
}

\newVerweis      {\aussagenlogischeSprache}{\glstext}         {aussagenlogischeSprache}
\newVerweis     {\aussagenlogischenSprache}{\glsuseri}        {aussagenlogischeSprache}
\newglossaryentry {aussagenlogischeSprache}{%ToDo prüfen
	name       =                      {---, aussagenlogische \addIdx[
		name   =                      {---, aussagenlogische},
		sort   =                  {Sprache, aussagenlogische}]{aussagenlogischeSprache}},
	sort       =                  {Sprache, aussagenlogische},
	text       ={aussagenlogische  Sprache},
	user1      ={aussagenlogischen Sprache},
	description={
		\todo{Beschreibung fehlt noch}% ToDo=aussagenlogische Sprache
	}
}

\newVerweis     {\Sprachebene} {\glstext}{Sprachebene}
\newVerweis[n]  {\Sprachebenen}{\glstext}{Sprachebene}
\newglossaryentry{Sprachebene}{%%% geprüft
	name        ={Sprachebene \addIdx    {Sprachebene}},
	text        ={Sprachebene},
	description ={
		Wir unterscheiden hier drei \GloFt{Sprachebenen}: Die obere Ebene mit der \Metasprache\, die mittlere mit der \formalenMetasprache\ und die untere mit der \Objektsprache.
		Mit einer \Sprache\ einer höheren Ebene kann man \textua\ \Aussagen über \Sprachen\ mit niedrigere Ebene treffen.
	}
}

\newVerweis     {\stellig}  {\glstext}{stellig}
\newVerweis[e]  {\stellige} {\glstext}{stellig}
\newVerweis[es] {\stelliges}{\glstext}{stellig}
\newVerweis[er] {\stelliger}{\glstext}{stellig}
\newglossaryentry{stellig}{
	name        ={$n$-stellig \addIdx[
		name    ={$n$-stellig},
		sort    ={stellig}]           {stellig}},
	sort        ={stellig},
	text        ={stellig},
	see         ={MtsStelF,MtsStelR},
	description ={
		Eine \Funktion, \Relation\ oder ein \Praedikat\ mit der \Stelligkeit\ $n \MtsIn \MtsINo$ nennt man \GloFt{$n$-stellig}.
	}
}

\newVerweis     {\Stelligkeit}  {\glstext}{Stelligkeit}
\newVerweis[en] {\Stelligkeiten}{\glstext}{Stelligkeit}
\newglossaryentry{Stelligkeit}{%ToDo prüfen
	name        ={Stelligkeit \addIdx     {Stelligkeit}},
	text        ={Stelligkeit},
	see         ={MtsStelF,MtsStelR},
	description ={
		einer \Funktion, \Relation\ oder eines \Praedikats.
	}
}

\newVerweis     {\Symbol}  {\glstext} {Symbol}
\newVerweis[e]  {\Symbole} {\glstext} {Symbol}
\newVerweis[s]  {\Symbols} {\glstext} {Symbol}
\newVerweis[en] {\Symbolen}{\glstext} {Symbol}
\newglossaryentry{Symbol}{%ToDo prüfen
	name        ={Symbol \addIdx      {Symbol}},
	text        ={Symbol},
	see         ={Beispielsymbol,Metasymbol,Objektsymbol},
	description ={
		Ein \defFt{einfaches} \GloFt{Symbol} ist ein druckbares typographisches Zeichen, das als Einheit angesehen wird.
		Ein \defFt{zusammengesetztes} \GloFt{Symbol} besteht aus mehreren einfachen \gloFt{Symbolen}.
		Wird ein \gloFt{Symbol}, das kann auch ein zusammengesetztes \gloFt{Symbol} sein, stets als Einheit angesehen, nennen wir es \defTxt{\atomar}\alternativi{unzerlegbar}, andernfalls \defTxt{\zerlegbar}.
		Im Einzelfall muss für ein Symbol definiert werden, ob es zerlegt werden kann oder nicht.
		Ein \emph{einfaches} \gloFt{Symbol} ist offensichtlich immer \atomar.
	}
}

\newVerweis     {\aussagenlogischesSymbol}  {\glstext}       {aussagenlogischesSymbol}
\newVerweis[en] {\aussagenlogischenSymbolen}{\glstext}       {aussagenlogischesSymbol}
\newglossaryentry{aussagenlogischesSymbol}{
	name       =                       {---, aussagenlogisches \addIdx[
		name   =                       {---, aussagenlogisches},
		sort   =                  {Symbol, aussagenlogische}]{aussagenlogischesSymbol}},
	sort       =                  {Symbol, aussagenlogische},
	text       ={aussagenlogisches Symbol},
	user1      ={aussagenlogischen Symbol},
	description={
		Die \GloFt{aussagenlogischen} \Symbole\ sind ...%ToDo=aussagenlogisches Symbol
	}
}

\newVerweis     {\metasprachlichesSymbol} {\glstext}         {metasprachlichesSymbol}
\newVerweis      {\metasprachlicheSymbole}{\glsuseri}        {metasprachlichesSymbol}
\newglossaryentry{metasprachlichesSymbol}{
	name       =                     {---, metasprachliches \addIdx[
		name   =                     {---, metasprachliches},
		sort   =                  {Symbol, metasprachliches}]{metasprachlichesSymbol}},
	sort       =                  {Symbol, metasprachliches},
	text       ={metasprachliches Symbol},
	user1      ={metasprachliche  Symbole},
	description={
		\todo{Beschreibung fehlt noch}% ToDo=metasprachliches Symbol
	}
}

\newVerweis     {\zusammengesetztesSymbol} {\glstext}         {zusammengesetztesSymbol}
\newVerweis      {\zusammengesetzteSymbole}{\glsuseri}        {zusammengesetztesSymbol}
\newglossaryentry{zusammengesetztesSymbol}{
	name       =                     {---, zusammengesetztes \addIdx[
		name   =                     {---, zusammengesetztes},
		sort   =                  {Symbol, zusammengesetztes}]{zusammengesetztesSymbol}},
	sort       =                  {Symbol, zusammengesetztes},
	text       ={zusammengesetztes Symbol},
	user1      ={zusammengesetzte  Symbole},
	description={
		\todo{Beschreibung fehlt noch}% ToDo=zusammengesetztes Symbol
	}
}

\newVerweis     {\Symbolfolge} {\glstext}{Symbolfolge}
\newVerweis[n]  {\Symbolfolgen}{\glstext}{Symbolfolge}
\newglossaryentry{Symbolfolge}{%ToDo prüfen
	name        ={Symbolfolge \addIdx    {Symbolfolge}},
	text        ={Symbolfolge},
	see         ={Zeichenkette},
	description ={
		Eine \GloFt{\Symbolfolge} ist eine \Folge\ von \atomaren\ \Symbolen.
	}
}

\newVerweis         {\Syntax} {\glstext}{Syntax}
\longnewglossaryentry{Syntax}{%%% geprüft
	name            ={Syntax \addIdx    {Syntax}},
	text            ={Syntax},
	see             ={Semantik,Sprache},
}{
	\begin{wikicite}{bib:Wikipedia}
		Unter \wikibf{Syntax} [\textdots] versteht man allgemein ein Regelsystem zur Kombination elementarer Zeichen zu zusammengesetzten Zeichen in natürlichen oder künstlichen Zeichensystemen. Die Zusammenfügungsregeln der Syntax stehen hierbei den Interpretationsregeln der \wikilink{Semantik} gegenüber.

		[\textdots]
	\end{wikicite}
	Wir nennen in der \formalenMetasprache\ und der \Objektsprache\ die elementaren Zeichen \Symbole\ und die zusammengesetzten Zeichen \Formeln.
}

%T === T === T === T === T === T === T === T === T === T === T === T === T === T

\newVerweis     {\Teilaussage} {\glstext} {Teilaussage}
\newVerweis        {\Taussage} {\glsuseri}{Teilaussage}
\newVerweis[n]  {\Teilaussagen}{\glstext} {Teilaussage}
\newglossaryentry{Teilaussage}{%%% geprüft
	name        ={Teilaussage \addIdx     {Teilaussage}},
	text        ={Teilaussage},
	user1       =    {aussage},
	description ={\TeilaussageDescription}
}
\newcommand*{\TeilaussageDescription}{
	Eine \Aussage\ $A$ heißt eine \GloFt{Teilaussage}\synonym{\defTxt{\Unteraussage}} \defFt{von} einer \Aussage\ $B$, wenn sie Teil von $B$ ist und man sie ohne Bedeutungsänderung der Aussage dort klammern könnte.
}

\newVerweis      {\echteTeilaussage}{\glstext}  {echteTeilaussage}
\newVerweis     {\echtenTeilaussage}{\glsuseri} {echteTeilaussage}
\newVerweis             {\eTaussage}{\glsuserii}{echteTeilaussage}
\newglossaryentry {echteTeilaussage}{
	name       =               {---, echte \addIdx[
		name   =               {---, echte},
		sort   =       {Teilaussage, echte}]    {echteTeilaussage}},
	sort       =       {Teilaussage, echte},
	text       ={echte  Teilaussage},
	user1      ={echten Teilaussage},
	user2      =           {aussage},
	description={\echteTeilaussageDescription}
}
\newcommand*{\echteTeilaussageDescription}{
	Eine \Teilaussage\ $A$ von $B$ heißt \GloFt{echte} \Teilaussage\ von $B$, wenn $A$ verschieden von $B$ ist.
}

\newVerweis     {\Teilfolge} {\glstext} {Teilfolge}
\newVerweis        {\Tfolge} {\glsuseri}{Teilfolge}
\newVerweis[n]  {\Teilfolgen}{\glstext} {Teilfolge}
\newglossaryentry{Teilfolge}{
	name        ={Teilfolge \addIdx     {Teilfolge}},
	text        ={Teilfolge},
	user1       =    {folge},
	description ={
		\todo{Beschreibung fehlt noch}% ToDo=Teilfolge
	}
}

\newVerweis      {\echteTeilfolge}{\glstext}  {echteTeilfolge}
\newVerweis     {\echtenTeilfolge}{\glsuseri} {echteTeilfolge}
\newVerweis             {\eTfolge}{\glsuserii}{echteTeilfolge}
\newglossaryentry {echteTeilfolge}{
	name       =             {---, echte \addIdx[
		name   =             {---, echte},
		sort   =       {Teilfolge, echte}]    {echteTeilfolge}},
	sort       =       {Teilfolge, echte},
	text       ={echte  Teilfolge},
	user1      ={echten Teilfolge},
	user2      =           {folge},
	description={
		\todo{Beschreibung fehlt noch}% ToDo=echte Teilfolge
	}
}

\newVerweis     {\Teilformel} {\glstext} {Teilformel}
\newVerweis[n]  {\Teilformeln}{\glstext} {Teilformel}
\newVerweis        {\Tformel} {\glsuseri}{Teilformel}
\newglossaryentry{Teilformel}{
	name        ={Teilformel \addIdx     {Teilformel}},
	text        ={Teilformel},
	user1       =    {formel},
	description ={
		\todo{Beschreibung fehlt noch}% ToDo=Teilformel
	}
}

\newVerweis      {\echteTeilformel}{\glstext}  {echteTeilformel}
\newVerweis     {\echtenTeilformel}{\glsuseri} {echteTeilformel}
\newVerweis             {\eTformel}{\glsuserii}{echteTeilformel}
\newglossaryentry {echteTeilformel}{
	name       =              {---, echte \addIdx[
		name   =              {---, echte},
		sort   =       {Teilformel, echte}]    {echteTeilformel}},
	sort       =       {Teilformel, echte},
	text       ={echte  Teilformel},
	user1      ={echten Teilformel},
	user2      =           {formel},
	description={
		\todo{Beschreibung fehlt noch}% ToDo=echte Teilformel
	}
}

\newVerweis     {\Teilmenge} {\glstext} {Teilmenge}
\newVerweis        {\Tmenge} {\glsuseri}{Teilmenge}
\newVerweis[n]  {\Teilmengen}{\glstext} {Teilmenge}
\newglossaryentry{Teilmenge}{
	name        ={Teilmenge \addIdx     {Teilmenge}},
	text        ={Teilmenge},
	user1       =    {menge},
	description ={
		\todo{Beschreibung fehlt noch}% ToDo=Teilmenge
	}
}

\newVerweis      {\echteTeilmenge}{\glstext}  {echteTeilmenge}
\newVerweis     {\echtenTeilmenge}{\glsuseri} {echteTeilmenge}
\newVerweis             {\eTmenge}{\glsuserii}{echteTeilmenge}
\newglossaryentry {echteTeilmenge}{
	name       =             {---, echte \addIdx[
		name   =             {---, echte},
		sort   =       {Teilmenge, echte}]    {echteTeilmenge}},
	sort       =       {Teilmenge, echte},
	text       ={echte  Teilmenge},
	user1      ={echten Teilmenge},
	user2      =           {menge},
	description={
		\todo{Beschreibung fehlt noch}% ToDo=echte Teilmenge
	}
}

\newVerweis     {\Teilobjekt} {\glstext} {Teilobjekt}
\newVerweis        {\Tobjekt} {\glsuseri}{Teilobjekt}
\newVerweis[e]  {\Teilobjekte}{\glstext} {Teilobjekt}
\newglossaryentry{Teilobjekt}{
	name        ={Teilobjekt \addIdx     {Teilobjekt}},
	text        ={Teilobjekt},
	user1       =    {objekt},
	description ={
		\todo{Beschreibung fehlt noch}% ToDo=Teilobjekt
	}
}

\newVerweis     {\echtesTeilobjekt}{\glstext}  {echtesTeilobjekt}
\newVerweis     {\echtenTeilobjekt}{\glsuseri} {echtesTeilobjekt}
\newVerweis             {\eTobjekt}{\glsuserii}{echtesTeilobjekt}
\newglossaryentry{echtesTeilobjekt}{
	name        =              {---, echtes \addIdx[
		name    =              {---, echtes},
		sort    =       {Teilobjekt, echtes}]  {echtesTeilobjekt}},
	sort        =       {Teilobjekt, echtes},
	text        ={echtes Teilobjekt},
	user1       ={echten Teilobjekt},
	user2       =           {objekt},
	description ={
		\todo{Beschreibung fehlt noch}% ToDo=echtes Teilobjekt
	}
}

\newVerweis     {\Teilsprache} {\glstext} {Teilsprache}
\newglossaryentry{Teilsprache}{
	name        ={Teilsprache \addIdx     {Teilsprache}},
	text        ={Teilsprache},
	description ={
		\todo{Beschreibung fehlt noch}% ToDo=Teilsprache
	}
}

\newVerweis     {\echteTeilsprache}{\glstext}  {echteTeilsprache}
\newglossaryentry{echteTeilsprache}{
	name        =               {---, echte \addIdx[
		name    =               {---, echte},
		sort    =       {Teilsprache, echte}]  {echteTeilsprache}},
	sort        =       {Teilsprache, echte},
	text        ={echte Teilsprache},
	description ={
		\todo{Beschreibung fehlt noch}% ToDo=echte Teilsprache
	}
}

\newVerweis     {\Teilsymbol} {\glstext} {Teilsymbol}
\newVerweis        {\Tsymbol} {\glsuseri}{Teilsymbol}
\newVerweis[e]  {\Teilsymbole}{\glstext} {Teilsymbol}
\newglossaryentry{Teilsymbol}{
	name        ={Teilsymbol \addIdx     {Teilsymbol}},
	text        ={Teilsymbol},
	user1       =    {symbol},
	description ={
		\todo{Beschreibung fehlt noch}% ToDo=Teilsymbol
	}
}

\newVerweis     {\echtesTeilsymbol}{\glstext}  {echtesTeilsymbol}
\newVerweis     {\echtenTeilsymbol}{\glsuseri} {echtesTeilsymbol}
\newVerweis             {\eTsymbol}{\glsuserii}{echtesTeilsymbol}
\newglossaryentry{echtesTeilsymbol}{
	name       =              {---, echtes \addIdx[
		name   =              {---, echtes},
		sort   =       {Teilsymbol, echtes}]   {echtesTeilsymbol}},
	sort       =       {Teilsymbol, echtes},
	text       ={echtes Teilsymbol},
	user1      ={echten Teilsymbol},
	user2      =           {symbol},
	description={
		\todo{Beschreibung fehlt noch}% ToDo=echtes Teilsymbol
	}
}

\newVerweis     {\Traegermenge} {\glstext}{Traegermenge}
\newVerweis[n]  {\Traegermengen}{\glstext}{Traegermenge}
\newglossaryentry{Traegermenge}{%ToDo prüfen
	name        ={Trägermenge \addIdx[
		name    ={Trägermenge}]           {Traegermenge}},
	text        ={Trägermenge},
	see         ={MtsTraeger},
	description ={
		einer \Relation.
	}
}

\newVerweis     {\Transformation}  {\glstext}{Transformation}
\newVerweis[en] {\Transformationen}{\glstext}{Transformation}
\newglossaryentry{Transformation}{%ToDo prüfen
	name        ={Transformation \addIdx     {Transformation}},
	text        ={Transformation},
	see         ={MtsTransformation,MtsTransformationTup,zulaessigeTransformation},
	description ={
		Eine Umformung oder Erzeugung einer \Formel\ aus einer vorgegebenen \Menge\ von \Formeln, \textdh\ die Anwendung einer \Schlussregel.
	}
}

\newVerweis      {\zulaessigeTransformation}  {\glstext}  {zulaessigeTransformation}
\newVerweis[en]  {\zulaessigeTransformationen}{\glstext}  {zulaessigeTransformation}
\newVerweis     {\zulaessigenTransformation}  {\glsuseri} {zulaessigeTransformation}
\newVerweis[en] {\zulaessigenTransformationen}{\glsuseri} {zulaessigeTransformation}
\newVerweis[en] {\zulaessigerTransformationen}{\glsuserii}{zulaessigeTransformation}
\newglossaryentry{zulaessigeTransformation}{%ToDo prüfen
	name        =                      {---, zulässige \addIdx[
		name    =                      {---, zulässige},
		sort    =           {Transformation, zulässige}]  {zulaessigeTransformation}},
	sort        =           {Transformation, zulässige},
	text        ={zulässige  Transformation},
	user1       ={zulässigen Transformation},
	user2       ={zulässiger Transformation},
	description ={
		Eine \Transformation\ heißt \GloFt{zulässig}, wenn sie Element einer vorgegebenen \Menge\ von \Transformationen\ oder eine daraus zulässigerweise abgeleitete \Transformation\ ist.
	}
}

\newVerweis     {\Transformationsfolge} {\glstext}{Transformationsfolge}
\newVerweis[n]  {\Transformationsfolgen}{\glstext}{Transformationsfolge}
\newglossaryentry{Transformationsfolge}{%ToDo prüfen
	name        ={Transformationsfolge \addIdx    {Transformationsfolge}},
	text        ={Transformationsfolge},
	see         ={MtsTransformation,MtsTransformationTup,Transformation},
	description ={
		Eine Folge von \Transformationen.
	}
}

\newVerweis     {\Transformationsregel} {\glstext}{Transformationsregel}
\newVerweis[n]  {\Transformationsregeln}{\glstext}{Transformationsregel}
\newglossaryentry{Transformationsregel}{%ToDo prüfen
	name        ={Transformationsregel \addIdx    {Transformationsregel}},
	text        ={Transformationsregel},
	description ={
		\todo{Beschreibung fehlt noch}% ToDo=Transformationsregel
	}
}

\newVerweis         {\Tupel} {\glstext}{Tupel}
\newVerweis[s]      {\Tupels}{\glstext}{Tupel}
\longnewglossaryentry{Tupel}{%ToDo prüfen
	name            ={Tupel \addIdx    {Tupel}},
	text            ={Tupel},
	see             ={Folge,Komponente,Menge,Objekt,Symbolfolge,Zeichenkette},
}{
	\begin{wikicite}{bib:Tupel}
		\wikibf{Tupel} (abgetrennt von \wikilink{mittellat.} \wikiit{quintuplus} ‚fünffach‘, \wikiit{septuplus} ‚siebenfach‘, \wikiit{centuplus} ‚hundertfach‘ etc.) sind in der \wikilink{Mathematik} neben \wikilink{Mengen} eine wichtige Art und Weise, \wikilink{mathematische Objekte} zusammenzufassen. Ein Tupel besteht aus einer \wikilink{Liste} endlich vieler, nicht notwendigerweise voneinander verschiedener Objekte. Dabei spielt, im Gegensatz zu Mengen, die Reihenfolge der Objekte eine Rolle. Es gibt verschiedene Möglichkeiten, Tupel formal als Mengen darzustellen. Tupel finden in vielen Bereichen der Mathematik Verwendung, zum Beispiel als \wikilink{Koordinaten} von Punkten oder als \wikilink{Vektoren} in mehrdimensionalen \wikilink{Vektorräumen}.

		Von Tupeln unabhängig von ihrer Länge ist selten die Rede. Vielmehr verwendet man das Wort \wikibf{$n$-Tupel} und die im nächsten Abschnitt genannten Spezialfälle davon dann, wenn sich aus dem Zusammenhang die Länge als feste Zahl oder als benannte Konstante wie $n$ ergibt. Betrachtet man dagegen viele endliche Folgen unterschiedlicher Längen von Elementen einer Grundmenge, spricht man von endlichen Folgen oder definiert einen neuen Begriff, der oft mit „Kette“ zusammengesetzt ist, z. B. \wikilink{Zeichenkette}, \wikilink{Additionskette}.

		[\textdots]
	\end{wikicite}
	Ein \GloFt{$n$-Tupel}\alternativi{Vektor} $\vec{a}$ ist eine endliche \Folge\alternativi{Sequenz} $(a_1, \dots, a_n)$ \defFt{von} seinen \defFt{Komponenten} $a_i$.
	Sind alle Komponenten Elemente derselben \Menge\ $M$, so heißt $\vec{a}$ ein $n$-\Tupel\ \defFt{auf} $M$.
}

\newVerweis     {\Tupelmenge} {\glstext}{Tupelmenge}
\newVerweis[n]  {\Tupelmengen}{\glstext}{Tupelmenge}
\newglossaryentry{Tupelmenge}{%ToDo prüfen
	name        ={Tupelmenge \addIdx    {Tupelmenge}},
	text        ={Tupelmenge},
	description ={
		Die \Tupelmenge\ $\MtsTup(M)$ einer \Menge\ $M$ ist die \Menge\ aller $n$-Tupel aus $M^n$ für alle $n \in \MtsINo$.
	}
}

%U === U === U === U === U === U === U === U === U === U === U === U === U === U

\newVerweis     {\Umkehrrelation}  {\glstext}{Umkehrrelation}
\newVerweis[en] {\Umkehrrelationen}{\glstext}{Umkehrrelation}
\newglossaryentry{Umkehrrelation}{%ToDo prüfen
	name        ={Umkehrrelation \addIdx     {Umkehrrelation}},
	text        ={Umkehrrelation},
	see         ={Menge},
	description ={
		Die \GloFt{Umkehrrelation}\alternativiii{konverse Relation}{Konverse }{inverse Relation } \emph{von} einer \binaeren\ \Relation\ $(G,A,B)$ ist die \Relation\ $(H,B,A)$ mit $H = \RawMengeDef{(b,a)}{(a,b) \in G}$.
		Üblicherweise wird das zugehörige \Relationssymbol\ gespiegelt.
		Die \gloFt{Umkehrrelation} der \Negation\ einer \Relation\ ist gleich der \Negation\ ihrer \gloFt{Umkehrrelation}.
	}
}

\newVerweis     {\unaer}  {\glstext}{unaer}
\newVerweis[e]  {\unaere} {\glstext}{unaer}
\newVerweis[en] {\unaeren}{\glstext}{unaer}
\newVerweis[er] {\unaerer}{\glstext}{unaer}
\newglossaryentry{unaer}{%ToDo prüfen
	name        ={unär \addIdx[
		name    ={unär}]            {unaer}},
	text        ={unär},
	see         ={binaer},
	description ={
		Eine \Operation, \Funktion\ oder \Relation\ heißt \GloFt{unär}, wenn ihre \Stelligkeit\ gleich 1 ist.
	}
}

\newVerweis     {\Ungleichheit}{\glstext}{Ungleichheit}
\newglossaryentry{Ungleichheit}{%ToDo prüfen
	name        ={Ungleichheit \addIdx   {Ungleichheit}},
	text        ={Ungleichheit},
	description ={
		Eine \Gleichheitsrelation:
		Zwei Objekte $A$ und $B$ sind \defFt{nicht gleich}\alternativii{nicht dasselbe}{nicht identisch} $A \MtsEqN B$, wenn sie in mindestens einer \interessierendenEigenschaft\ für \MtsEq\ nicht übereinstimmen.
	}
}

\newsynonym{\Unteraussage}{Unteraussage}{\Teilaussage}
\newsynonym{\Unterformel} {Unterformel} {\Teilformel}
\newsynonym{\Untermenge}  {Untermenge}  {\Teilmenge}
\newsynonym{\Unterobjekt} {Unterobjekt} {\Teilobjekt}
\newsynonym{\Untersymbol} {Untersymbol} {\Teilsymbol}

\newsynonym{\unzerlegbar} {unzerlegbar} {\atomar}

%V === V === V === V === V === V === V === V === V === V === V === V === V === V

\newVerweis         {\Variable} {\glstext}{Variable}
\newVerweis[n]      {\Variablen}{\glstext}{Variable}
\longnewglossaryentry{Variable}{
	name            ={Variable \addIdx    {Variable}},
	text            ={Variable},
	see             ={Konstante},
}{
	\begin{wikicite}{bib:Variable}
		Eine \wikibf{Variable} ist ein Name für eine Leerstelle in einem logischen oder mathematischen Ausdruck.[1]Der Begriff leitet sich vom lateinischen \wikilink{Adjektiv} \wikiit{variabilis} (veränderlich) ab. Gleichwertig werden auch die Begriffe \wikiit{Platzhalter} oder \wikiit{Veränderliche} benutzt. Als „Variable“ dienten früher Wörter oder Symbole, heute verwendet man zur \wikilink{mathematischen Notation} in der Regel Buchstaben als Zeichen. Wird anstelle der Variablen ein konkretes Objekt eingesetzt, so ist darauf zu achten, dass überall dort, wo die Variable auftritt, auch dasselbe Objekt benutzt wird.

		[\textdots]
	\end{wikicite}
}

\newVerweis      {\aussagenlogischeVariable} {\glstext}         {aussagenlogischeVariable}
\newVerweis[n]  {\aussagenlogischenVariablen}{\glsuseri}        {aussagenlogischeVariable}
\newVerweis     {\aussagenlogischenV}        {\glsuserii}       {aussagenlogischeVariable}
\newglossaryentry {aussagenlogischeVariable}{
	name       =                       {---, aussagenlogische \addIdx[
		name   =                       {---, aussagenlogische},
		sort   =                  {Variable, aussagenlogische}] {aussagenlogischeVariable}},
	sort       =                  {Variable, aussagenlogische},
	text       ={aussagenlogische  Variable},
	user1      ={aussagenlogischen Variable},
	user2      ={aussagenlogischen},
	description={
		Die \GloFt{aussagenlogischen} \Variablen\ sind die \Elemente\ von \OjkVar.
	}
}

\newVerweis      {\logischeVariable}{\glstext} {logischeVariable}
\newVerweis      {\logischeV}       {\glsuseri}{logischeVariable}
\newglossaryentry {logischeVariable}{
	name       =               {---, logische \addIdx[
		name   =               {---, logische},
		sort   =          {Variable, logische}]{logischeVariable}},
	sort       =          {Variable, logische},
	text       ={logische  Variable},
	user1      ={logische},
	description={
		Die \GloFt{logischen} \Variablen\ entsprechen den \aussagenlogischenV.
	}
}

\newVerweis     {\metasprachlicheVariable}{\glstext}        {metasprachlicheVariable}
\newVerweis     {\metasprachlicheV}       {\glsuseri}       {metasprachlicheVariable}
\newglossaryentry{metasprachlicheVariable}{
	name       =                     {---, metasprachliche \addIdx[
		name   =                     {---, metasprachliche},
		sort   =                {Variable, metasprachliche}]{metasprachlicheVariable}},
	sort       =                {Variable, metasprachliche},
	text       ={metasprachliche Variable},
	user1      ={metasprachliche},
	description={
		Die \GloFt{metasprachlichen} \Variablen\ sind die \Elemente\ von% ToDo=metasprachliche Variable
	}
}

\newVerweis     {\Vereinigung} {\glstext}{Vereinigung}
\newglossaryentry{Vereinigung}{
	name        ={Vereinigung \addIdx    {Vereinigung}},
	text        ={Vereinigung},
	description ={
		Eine \Mengenoperation: \todo{Beschreibung fehlt noch}% ToDo=Vereinigung von Mengen
	}
}

\newVerweis     {\vergleichbar} {\glstext}{vergleichbar}
\newVerweis     {\Vergleichbar} {\Glstext}{vergleichbar}
\newVerweis[e]  {\vergleichbare}{\glstext}{vergleichbar}
\newglossaryentry{vergleichbar}{%ToDo prüfen -  Wert und Ergebnis definieren?
	name        ={vergleichbar \addIdx    {vergleichbar}},
	text        ={vergleichbar},
	description ={
		Zwei \Objekte\ $A$ und $B$ sind \vergleichbar, wenn beide von derselben \Objektart\ sind, \textdh\ wenn beide \textzB\ jeweils Mengen, \Symbolfolgen, Zahlen, \textusw\ sind.
		Dabei muss bei \Formeln\ zwischen der \Formel\ an sich und ihrem \emph{Wert} oder \emph{Ergebnis} unterschieden werden.
	}
}

\newVerweis     {\Verkettung} {\glstext}{Verkettung}
\newglossaryentry{Verkettung}{
	name        ={Verkettung \addIdx    {Verkettung}},
	text        ={Verkettung},
	description ={
		\todo{Beschreibung fehlt noch}% ToDo=Verkettung von Folgen
	}
}

\newVerweis     {\Vertauschung}  {\glstext}{Vertauschung}
\newVerweis[en] {\Vertauschungen}{\glstext}{Vertauschung}
\newglossaryentry{Vertauschung}{%ToDo prüfen
	name        ={Vertauschung \addIdx     {Vertauschung}},
	text        ={Vertauschung},
	description ={
		Die \GloFt{Vertauschung} von zwei unabhängigen Teil-\Formeln\ ($\alpha$ und $\beta$) in einer anderen \Formel\ ($\gamma$)
		\\--- Formal: $\gamma(\alpha \MtsSwap \beta)$.
		Die \gloFt{Vertauschung} ist eine spezielle Form der \Ersetzung.
	}
}

\newsynonym{\Voraussetzung}{Voraussetzung}{\Praemisse}

%W === W === W === W === W === W === W === W === W === W === W === W === W === W

\newVerweis         {\Wahrheitswert}  {\glstext}{Wahrheitswert}
\newVerweis[e]      {\Wahrheitswerte} {\glstext}{Wahrheitswert}
\newVerweis[en]     {\Wahrheitswerten}{\glstext}{Wahrheitswert}
\longnewglossaryentry{Wahrheitswert}{%ToDo prüfen
	name            ={Wahrheitswert \addIdx     {Wahrheitswert}},
	text            ={Wahrheitswert},
	see             ={atomar,Aussage,Element,Junktor,Teilaussage,Logik},
}{
	\begin{wikicite}{bib:Wahrheitswert}
		Ein \wikibf{Wahrheitswert} ist in \wikilink{Logik} und \wikilink{Mathematik} ein \wikiit{logischer Wert}, den eine Aussage in Bezug auf Wahrheit annehmen kann.

		In der zweiwertigen \wikilink{klassischen Logik} kann eine Aussage nur entweder \wikiit{wahr} oder \wikiit{falsch} sein, die Menge der Wahrheitswerte $\{W, F\}$ hat so zwei Elemente. In \wikilink{mehrwertigen Logiken} enthält die \wikilink{Wahrheitswertemenge} mehr als zwei Elemente, z. B. in einer \wikilink{dreiwertigen Logik} oder einer \wikilink{Fuzzy-Logik}, die damit zu den \wikilink{nichtklassischen} zählen. Hier wird dann auch neben Wahrheitswerten von \wikiit{Quasiwahrheitswerten}, \wikiit{Pseudowahrheitswerten} oder \wikiit{Geltungswerten} gesprochen.

		Die Abbildung der Menge von Aussagen einer (meist formalen) Sprache auf die Wahrheitswertemenge wird \wikilink{Wahrheitswertzuordnung}  genannt und ist eine aussagenlogisch spezifische \wikilink{Bewertungsfunktion}. In der klassischen Logik kann auch explizit die Klasse aller wahren Aussagen beziehungsweise die Klasse aller falschen Aussagen definiert werden. Die Abbildung von Wahrheitswerten der (\wikilink{atomaren}) Teilaussagen einer zusammengesetzten Aussage auf die Wahrheitswertemenge heißt \wikilink{Wahrheitswertefunktion} oder Wahrheitsfunktion. Die Wertetabelle dieser \wikilink{Funktion} im mathematischen Sinn wird auch als \wikilink{Wahrheitstafel} bezeichnet und häufig dazu verwendet, die Bedeutung wahrheitsfunktionaler \wikilink{Junktoren} anzugeben.
	\end{wikicite}
	\GlossarZusatz{
		Wir verwenden nur die beiden \GloFt{Wahrheitswerte} der zweiwertigen klassischen \Logik, die wir (in der \Metasprache) mit \chrqt{\TxtTrue} und \chrqt{\TxtFalse} bezeichnen.
		In der \formalenMetasprache\ hingegen verwenden wir \chrqt{\MtsTrue} und \chrqt{\MtsFalse} und in der \Objektsprache\ \chrqt{\OjkTrue} und \chrqt{\OjkFalse}.
		In der Literatur findet man auch einfach \chrqt{$1$} und \chrqt{$0$}.
	}
}

\newVerweis     {\aussagenlogischerWahrheitswert}{\glstext}          {aussagenlogischerWahrheitswert}
\newglossaryentry{aussagenlogischerWahrheitswert}{%%% geprüft
	name       = {aussagenlogischerWahrheitswert \addIdx[
		name   =                            {---, aussagenlogischer},
		sort   =                  {Wahrheitswert, aussagenlogischer}]{aussagenlogischerWahrheitswert}},
	sort       =                  {Wahrheitswert, aussagenlogischer},
	text       ={aussagenlogischer Wahrheitswert},
	description={
		Es gib die beiden \GloFt{aussagenlogischen Wahrheitswerte} \OjkTrue\ und \OjkFalse.
	}
}

\newVerweis     {\metasprachlicherWahrheitswert}{\glstext}         {metasprachlicherWahrheitswert}
\newVerweis      {\metasprachlicheWahrheitswert}{\glsuseri}        {metasprachlicherWahrheitswert}
\newglossaryentry{metasprachlicherWahrheitswert}{%ToDo prüfen
	name       =                           {---, metasprachlicher \addIdx[
		name   =                           {---, metasprachlicher},
		sort   =                 {Wahrheitswert, metasprachlicher}]{metasprachlicherWahrheitswert}},
	sort       =                 {Wahrheitswert, metasprachlicher},
	text       ={metasprachlicher Wahrheitswert},
	user1      ={metasprachliche  Wahrheitswert},
	description={
		Es gib die beiden \GloFt{metasprachlichen Wahrheitswerte} in Textform (\TxtTrue, \TxtFalse) und in der \formalenMetasprache\ (\MtsTrue, \MtsFalse).
	}
}

\newVerweis     {\Wertebereich} {\glstext}{Wertebereich}
\newVerweis[e]  {\Wertebereiche}{\glstext}{Wertebereich}
\newglossaryentry{Wertebereich}{%ToDo prüfen
	name        ={Wertebereich \addIdx    {Wertebereich}},
	text        ={Wertebereich},
	see         ={MtsWb,Zielbereich,Funktion},
	description ={
		einer \Funktion.
	}
}

\newVerweis         {\Wikipedia}{\glstext}{Wikipedia}
\longnewglossaryentry{Wikipedia}{
	name            ={Wikipedia \addIdx   {Wikipedia}},
	text            ={Wikipedia},
}{
	\begin{wikicite}{bib:Wikipedia}
		Wikipedia ist ein Projekt zum Aufbau einer [Internet-\nobreak]Enzyklopädie aus freien Inhalten.
	\end{wikicite}
}

\newVerweis     {\Wort}   {\glstext}{Wort}
\newVerweis[e]  {\Worte}  {\glstext}{Wort}
\newVerweis     {\Woerter}{\glspl}  {Wort}
\newglossaryentry{Wort}{%ToDo prüfen
	name        ={Wort \addIdx      {Wort}},
	text        ={Wort},
	plural      ={Wörter},
	see         ={Formelmenge},
	description ={
		Synonym: \Formel\ ---
		Ein Element einer \Sprache.
	}
}

%Z === Z === Z === Z === Z === Z === Z === Z === Z === Z === Z === Z === Z === Z

\newVerweis     {\Zeichenkette} {\glstext}{Zeichenkette}
\newVerweis[n]  {\Zeichenketten}{\glstext}{Zeichenkette}
\newglossaryentry{Zeichenkette}{%ToDo prüfen
	name        ={Zeichenkette \addIdx    {Zeichenkette}},
	text        ={Zeichenkette},
	see         ={Symbolfolge},
	description ={
		Eine Folge von (typographischen) Zeichen, auch Leerstellen und sonstigem Zwischenraum.
	}
}

\newVerweis     {\zerlegbar}  {\glstext}{zerlegbar}
\newVerweis[e]  {\zerlegbare} {\glstext}{zerlegbar}
\newVerweis[e]  {\Zerlegbare} {\Glstext}{zerlegbar}
\newVerweis[es] {\zerlegbares}{\glstext}{zerlegbar}
\newglossaryentry{zerlegbar}{%ToDo prüfen
	name        ={zerlegbar \addIdx     {zerlegbar}},
	text        ={zerlegbar},
	see         ={atomar},
	description ={
		Eine \Aussage, \Formel, \Folge\ oder \Symbol, die eine \echteTeilaussage,  -\eTfolge, -\eTformel\ \textbzw. -\eTsymbol\ enthalten, heißt \GloFt{zerlegbar}.
	}
}

\newVerweis     {\Ziel} {\glstext}{Ziel}
\newVerweis[e]  {\Ziele}{\glstext}{Ziel}
\newglossaryentry{Ziel}{%ToDo prüfen
	name        ={Ziel \addIdx    {Ziel}},
	text        ={Ziel},
	description ={
		Ein \GloFt{Ziel} ist in diesem Dokument eine Anforderungen an \ASBA.
	}
}

\newVerweis     {\Zielbereich} {\glstext}{Zielbereich}
\newVerweis[e]  {\Zielbereiche}{\glstext}{Zielbereich}
\newglossaryentry{Zielbereich}{%ToDo prüfen
	name        ={Zielbereich \addIdx    {Zielbereich}},
	text        ={Zielbereich},
	see         ={MtsZb,Wertebereich,Funktion},
	description ={
		einer \Funktion.
	}
}

\newVerweis     {\zulaessig}  {\glstext}{zulaessig}
\newVerweis[e]  {\zulaessige} {\glstext}{zulaessig}
\newVerweis[en] {\zulaessigen}{\glstext}{zulaessig}
\newVerweis[er] {\zulaessiger}{\glstext}{zulaessig}
\newglossaryentry{zulaessig}{%ToDo prüfen
	name        ={zulässig \addIdx[
		name    ={zulässig}]            {zulaessig}},
	text        ={zulässig},
	see         ={Formel,Transformation,Ersetzung},
	description ={
		Eine Eigenschaft von \Formel, \Transformation\ und \Ersetzung.
	}
}
