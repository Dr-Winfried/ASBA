%%############################################################################%%
%%                                                                            %%
%% Datei:  ASBA-Glossar-Texte.tex                                             %%
%% Inhalt: Vorspann Glossareinträge für ASBA                                  %%
%%                                                                            %%
%% Copyright (C) 2017  Winfried Teschers                                      %%
%%                                                                            %%
%% This program is free software: you can redistribute it and/or modify       %%
%% it under the terms of the GNU Affero General Public License as published   %%
%% by the Free Software Foundation, either version 3 of the License, or       %%
%% (at your option) any later version.                                        %%
%%                                                                            %%
%% This program is distributed in the hope that it will be useful,            %%
%% but WITHOUT ANY WARRANTY; without even the implied warranty of             %%
%% MERCHANTABILITY or FITNESS FOR A PARTICULAR PURPOSE.  See the              %%
%% GNU Affero General Public License for more details.                        %%
%%                                                                            %%
%% You should have received a copy of the GNU Affero General Public License   %%
%% along with this program.  If not, see <http://www.gnu.org/licenses/>.      %%
%%                                                                            %%
%% Dr. Winfried Teschers                                                      %%
%% Anton-Günther-Straße 26c                                                   %%
%% 91083 Baiersdorf                                                           %%
%% Germany                                                                    %%
%%                                                                            %%
%% e-mail: winfried.teschers@t-online.de                                      %%
%%                                                                            %%
%%############################################################################%%

% !TeX root = ASBA.tex
% !TeX encoding = UTF-8
% !TeX spellcheck = de_DE

% ### Glossar und Index ########################################################

% ==============================================================================
% - Ausgabe als Text und Eintrag und Link ins Glossar
% Fachbegriffe =================================================================

\iftestFlg% Definition von Dummy Glossareinträgen; gleichzeitig Kopiervorlagen
	\newVerweis     {\dummyDummy}{\glstext}{dummyDummy}
	\newglossaryentry{dummyDummy}{
		name       =        {---, dummy \addIdx[% Ausgabe             im Glossar
			name   =        {---, dummy},%        Ausgabe             im Index
			sort   =      {Dummy, dummy}]  {dummyDummy}},%Reihenfolge im Index
		sort       =      {Dummy, dummy},%                Reihenfolge im Glossar
		text       ={dummy Dummy},%               Ausgabe             im Text
		user1      ={},%             alternative Ausgabe im Text für \newVerweis
		user2      ={},%             alternative Ausgabe im Text für \newVerweis
%%%		symbol     ={\ensuremath{Mathmode}},% formal - kein Link ins Glossar
%%%		user6      ={Textmode},%           informell - kein Link ins Glossar
		see        ={},%             Verweise ins Symbolverzeichnis und Glossar
		description={\todoBeschreiben%
%%%			\SymbolAmRand{dummy}%
%%%			\TextAmRand  {dummy}%
		}
	}
	\newVerweis     {\Dummy}{\glstext}{Dummy}
	\newglossaryentry{Dummy}{
		name        ={Dummy \addIdx   {Dummy}},% Ausgabe/Reihenfolge sonst
		text        ={Dummy},%                   Ausgabe             im Text
		user1       ={},%            alternative Ausgabe im Text für \newVerweis
		user2       ={},%            alternative Ausgabe im Text für \newVerweis
%%%		symbol      ={\ensuremath{Mathmode}},% formal - kein Link ins Glossar
%%%		user6       ={Textmode},%           informell - kein Link ins Glossar
		see         ={},%            Verweise ins Symbolverzeichnis und Glossar
		description ={\todoBeschreiben%
%%%			\SymbolAmRand{dummy}%
%%%			\TextAmRand  {dummy}%
		}
	}
	% dsgl. mit Absätzen in der Beschreibung
	\newVerweis         {\WikiDummy}{\glstext}{WikiDummy}
	\longnewglossaryentry{WikiDummy}{
		name            ={WikiDummy \addIdx   {WikiDummy}},
		text            ={WikiDummy},
		user1           ={},
		user2           ={},
%%%		symbol          ={\ensuremath{Mathmode}},% formal - kein Link ins Glossar
%%%		user6           ={Textmode},%           informell - kein Link ins Glossar
		see             ={},
	}{\todoBeschreiben%
%%%		\SymbolAmRand{dummy}%
%%%		\TextAmRand  {dummy}%
		\wikicite{bib:Wikipedia}{
		}
	}
\else\fi

% TODO ### Selbstreferenzen auflösen: \GloFt{text} und \gloFt{text})

% TODO ### Vorkommen prüfen und durch Makro ersetzen (???)

% TODO symbol, user6 und SymbolAmRand hinzufügen
% TODO Konkrete Werte für symbol und user6 einsetzen.

%%%	symbol      ={\ensuremath{Mathmode}},% ToDo=Mathmode
%%%	user6       ={Textmode},

%A === A === A === A === A === A === A === A === A === A === A === A === A === A

\newsynonym{\Abbildung}{Abbildung}{\Funktion}

\newVerweis     {\ableitbar} {\glstext}{ableitbar}
\newVerweis[e]  {\ableitbare}{\glstext}{ableitbar}
\newglossaryentry{ableitbar}{
	name        ={ableitbar \addIdx    {ableitbar}},
	text        ={ableitbar},
%%%	symbol      ={\ensuremath{Mathmode}},
%%%	user6       ={Textmode},
	see         ={Ableitungsrelation},
	description ={\todoPruefen%
%%%		\SymbolAmRand{ableitbar}%
		Wenn sich eine \Formel\ $\beta$ aus einer anderen \Formel\ $\alpha$ mittels \zulaessiger\ \Transformationen\ ableiten lässt, heißt $\beta$ \GloFt{ableitbar} aus $\alpha$.
		Sprechweise: $\alpha$ \GloFt{ableitbar}\synonym{\beweisbar} $\beta $.
		Eine oder beide \Formeln\ $\alpha$ \textbzw\ $\beta$ dürfen dabei durch \Formelmengen\ ersetzt werden.
	}
}

\newVerweis         {\Ableitung}  {\glstext}{Ableitung}
\newVerweis[en]     {\Ableitungen}{\glstext}{Ableitung}
\longnewglossaryentry{Ableitung}{
	name            ={Ableitung \addIdx     {Ableitung}},
	text            ={Ableitung},
	symbol          ={\MtsDerive},
	see             ={Ableitungsmenge,Ableitungsrelation,Konklusion,Logik,Praemisse,Schlussregel},
}{\todoPruefen%
	\SymbolAmRand{Ableitung}%
	\wikicite{bib:Ableitung}{
		Eine \wikiBoldFt{Ableitung}, \wikiBoldFt{Herleitung}, oder \wikiLinkFt{Deduktion} ist in der \wikiLinkFt{Logik} die Gewinnung von \wikiLinkFt{Aussagen} aus anderen Aussagen. Dabei werden \wikiLinkFt{Schlussregeln} auf \wikiLinkFt{Prämissen} angewandt, um zu \wikiLinkFt{Konklusionen} zu gelangen. Welche Schlussregeln dabei erlaubt sind, wird durch das verwendete \wikiLinkFt{Kalkül} bestimmt.

		Die Ableitung ist zusammen mit der \wikiLinkFt{semantischen Konklusion} einer der zwei logischen Methoden, um auf die Konklusion zu kommen.
	}
	Eine Ableitung ist für \ASBA\ eine \Aussage\ $A \MtsDerive B$ \textbzw\ allgemeiner $A \MtsDeriveR B$ mit $A,B \MtsSubsetEq \MtsSprache$, wobei \MtsSprache\ eine \Sprache\ ist.
	Dies entspricht einem \Element\ $(A,B)$ aus einer \Ableitungsrelation\ \MtsDerive\ \textbzw\ \MtsDeriveR\ (\textdh\ $(A,B) \in R_{\MtsIdxGraph}$ für eine \Ableitungsrelation\ $R$).
	Die semantische Aussage ist die, das die \Formeln\ aus $B$ aus den \Formeln\ aus $A$ abgeleitet werden können.
}

\newVerweis     {\Ableitungsmenge} {\glstext}{Ableitungsmenge}
\newcommand*    {\Ableitungsmengen}[1][]{\glstext[#1]{Ableitungsmenge}[n]}
\newglossaryentry{Ableitungsmenge}{
	name        ={Ableitungsmenge \addIdx    {Ableitungsmenge}},
	text        ={Ableitungsmenge},
%%%	symbol      ={\ensuremath{Mathmode}},% ToDo=Mathmode
%%%	user6       ={Textmode},
	description ={\todoPruefen%
%%%		\TextAmRand{Ableitungsmenge}%
		Eine \Menge\ von \Ableitungen, letztlich nichts anderes als eine \Ableitungsrelation.
	}
}

\newVerweis     {\Ableitungsrelation}  {\glstext}{Ableitungsrelation}
\newVerweis[en] {\Ableitungsrelationen}{\glstext}{Ableitungsrelation}
\newglossaryentry{Ableitungsrelation}{
	name        ={Ableitungsrelation \addIdx     {Ableitungsrelation}},
	text        ={Ableitungsrelation},
%%%	symbol      ={\ensuremath{Mathmode}},% ToDo=Mathmode
%%%	user6       ={Textmode},
	see         ={Ableitung},
	description ={\todoPruefen%
%%%		\SymbolAmRand{Ableitungsrelation}%
		Eine \binaere\ \Relation\ \MtsDerive\ aus \MtsAllDerive.
		Für $R \in \MtsAllDerive$ auch mit \MtsDeriveR\ bezeichnet.
	}
}

\newVerweis     {\Abtrennungsregel}{\glstext}{Abtrennungsregel}
\newglossaryentry{Abtrennungsregel}{
	name        ={Abtrennungsregel \addIdx   {Abtrennungsregel}},
	text        ={Abtrennungsregel},
%%%	symbol      ={\ensuremath{Mathmode}},% ToDo=Mathmode
%%%	user6       ={Textmode},
	see         ={TR},
	description ={\todoPruefen%
%%%		\SymbolAmRand{Abtrennungsregel}%
		Eine \Schlussregel.
	}
}

%%%\newVerweis     {\Aequivalenz}  {\glstext}{Aequivalenz}
%%%\newVerweis[en] {\Aequivalenzen}{\glstext}{Aequivalenz}
%%%\newglossaryentry{Aequivalenz}{
%%%	name        ={Äquivalenz \addIdx[
%%%		name    ={Äquivalenz}]            {Aequivalenz}},
%%%	text        ={Äquivalenz},
%%%	symbol      ={\ensuremath{Mathmode}},% ToDo=Mathmode
%%%	user6       ={Textmode},
%%%	see         ={MtsAequiv},
%%%	description ={\todoPruefen%
%%%		\SymbolAmRand{Aequivalenz}%
%%%		Eine \Gleichheitsrelation:
%%%		Zwei Objekte $A$ und $B$ sind \GloFt{äquivalent}\alternativi{ähnlich}, $A \MtsAequiv B$, wenn sie in den \interessierendenEigenschaften\ für \MtsAequiv\ übereinstimmen.
%%%	}
%%%}

\newVerweis        {\Aequivalenzrelation}  {\glstext}{Aequivalenzrelation}
\newVerweis[en]    {\Aequivalenzrelationen}{\glstext}{Aequivalenzrelation}
\longnewglossaryentry{Aequivalenzrelation}{
	name            ={Äquivalenzrelation \addIdx[
		name        ={Äquivalenzrelation}]           {Aequivalenzrelation}},
	text            ={Äquivalenzrelation},
%%%	symbol          ={\ensuremath{Mathmode}},% ToDo=Mathmode
%%%	user6           ={Textmode},
}{\todoPruefen%
%%%	\SymbolAmRand{Aequivalenzrelation}%
	Eine \GloFt{Äquivalenzrelation} ist eine \binaere\ \Relation\ auf einer \Menge\ $M$ mit folgenden Eigenschaften
	(dabei sei $\sim$ die \gloFt{Äquivalenzrelation}):
	\begin{align}
		&\text{\DefFt{reflexiv }}   &:&&\qquad  &a \sim a \\
		&\text{\DefFt{transitiv }}  &:&&\qquad((&a \sim b) \MtsAnd (b \sim c)) \MtsImp (a \sim c)\\
		&\text{\DefFt{symmetrisch }}&:&&\qquad (&a \sim b) \MtsImp (b \sim a)
		\formulatoleft
	\end{align}
	jeweils für alle \Elemente\ $a$, $b$ und $c$ aus $M$.
}

\newVerweis     {\Allquantor} {\glstext}{Allquantor}
\newglossaryentry{Allquantor}{
	name        ={Allquantor \addIdx    {Allquantor}},
	text        ={Allquantor},
%%%	symbol      ={\ensuremath{Mathmode}},% ToDo=Mathmode
%%%	user6       ={Textmode},
	description ={\todoPruefen%
%%%		\SymbolAmRand{Allquantor}%
		Man nennt den \Quantor\ \MtsForAll\ \textbzw\ \OjkForAll\ auch \GloFt{Allquantor}.
	}
}

\newVerweis     {\Alphabet} {\glstext}{Alphabet}
\newcommand*    {\Alphabets}[1][]{\glstext[#1]{Alphabet}[s]}
\newglossaryentry{Alphabet}{
	name        ={Alphabet \addIdx    {Alphabet}},
	text        ={Alphabet},
%%%	symbol      ={\ensuremath{Mathmode}},% ToDo=Mathmode
%%%	user6       ={Textmode},
	description ={\todoBeschreiben%
%%%		\SymbolAmRand{Alphabet}%
	}
}

\newVerweis     {\Anfangsglied}{\glstext} {Anfangsglied}
\newVerweis     {\AnfangsG}    {\glsuseri}{Anfangsglied}
\newglossaryentry{Anfangsglied}{
	name        ={Anfangsglied \addIdx   {Anfangsglied}},
	text        ={Anfangsglied},
	user1       ={Anfangs-},
	see         ={Endglied,Nachfolger,Vorgaenger,Zwischenglied},
	description ={\todoOk%
		Das \GloFt{Anfangsglied} einer \Kette\ ist ihr erstes (links) \Kettenglied.
	}
}

\newVerweis     {\Anfangsregel}{\glstext}{Anfangsregel}
\newglossaryentry{Anfangsregel}{
	name        ={Anfangsregel \addIdx   {Anfangsregel}},
	text        ={Anfangsregel},
	description ={\todoPruefen%
		Die \Schlussregel\ \glsAR\ um anfangen zu können.
	}
}

\newVerweis     {\ASBA}{\glstext} {ASBA}
\newglossaryentry{ASBA}{
	name        ={ASBA \addIdx   {ASBA}},
	text        ={ASBA},
	description ={\todoPruefen%
		ist ein Akronym für „\DefFt{A}xiome, \DefFt{S}ätze, \DefFt{B}eweise und \DefFt{A}uswertungen“.
		Es bezeichnet das \hier\ beschriebene Programmsystem, das zu eingegebenen \Axiomen, \Saetzen\ und \Beweisen\ letztere prüft, Auswertungen generiert und unter Zuhilfenahme gegebener \Ausgabeschemata\ eine Ausgabe im \LaTeX-Format in mathematisch üblicher Schreibweise mit \Formeln\ erstellt.
	}
}

% [#1= gleicher \Objektart], #2=Ein \Objekt, #3=sie kein echtes \Teilobjekt
\newcommand*{\atomarBeschreibung}[3][]{%
	#2\ ist \GloFt{\atomar}\synonym{\defTxt{\unzerlegbar}}, wenn #3\ #1\ enthält.
}
\newVerweis     {\atomar}  {\glstext}{atomar}
\newVerweis[e]  {\atomare} {\glstext}{atomar}
\newVerweis[en] {\atomaren}{\glstext}{atomar}
\newVerweis[es] {\atomares}{\glstext}{atomar}
\newglossaryentry{atomar}{
	name        ={atomar \addIdx     {atomar}},
	text        ={atomar},
	see         ={unzerlegbar,zerlegbar,zusammengesetzt},
	description ={\todoOk%
		\atomarBeschreibung[gleicher \Objektart]{Ein \Objekt\ (\Aussage, \Formel\ oder \Symbol)}{es kein \echtes\ \Teilobjekt}
	}
}

\dummyVerweis   {\Ausdruck}{\glstext}{Ausdruck}% ToDo=Ausdruck --> Formel?

\dummyVerweis   {\logischerAusdruck}   {\glstext}  {logischer Ausdruck}% ToDo=Ausdruck, logischer
\dummyVerweis   {\logischenAusdruecke} {\glsuseri} {logischen Ausdrücke}
\dummyVerweis[n]{\logischenAusdruecken}{\glsuseri} {logischen Ausdrücke}
\dummyVerweis    {\logischeAusdruecke} {\glsuserii}{logische  Ausdrücke}

\dummyVerweis   {\metasprachlicherAusdruck}   {\glstext} {metasprachlicher Ausdruck}% ToDo=Ausdruck, metasprachlicher
\dummyVerweis[n]{\metasprachlichenAusdruecken}{\glsuseri}{metasprachlichen Ausdrücke}

\newVerweis     {\Ausgabeschema}  {\glstext}{Ausgabeschema}
\newVerweis[ta] {\Ausgabeschemata}{\glstext}{Ausgabeschema}
\newglossaryentry{Ausgabeschema}{
	name        ={Ausgabeschema \addIdx     {Ausgabeschema}},
	text        ={Ausgabeschema},
%%%	symbol      ={\ensuremath{Mathmode}},% ToDo=Mathmode
%%%	user6       ={Textmode},
	description ={\todoGeprueft%
%%%		\SymbolAmRand{Ausgabeschema}%
		Ein \GloFt{Ausgabeschema} ist für \ASBA\ eine Beschreibung, wie ein bestimmtes mathematisches \Objekt\ ausgegeben werden soll.
		Dies kann \textzB\ ein Stück \LaTeX-Code mit entsprechenden Parametern sein.
	}
}

\newVerweis         {\Aussage}{\glstext}{Aussage}
\newVerweis[n]     {\Aussagen}{\glstext}{Aussage}
\longnewglossaryentry{Aussage}{
	name            ={Aussage \addIdx   {Aussage}},
	text            ={Aussage},
}{\todoOk%
	\wikicite{bib:Aussage}{
		Eine \wikiBoldFt{Aussage} im Sinn der \wikiLinkFt{aristotelischen Logik} ist ein sprachliches Gebilde, von dem es sinnvoll ist zu \wikiItalicFt{fragen}, ob es \wikiLinkFt{wahr} oder falsch ist (so genanntes Aristotelisches \wikiLinkFt{Zweiwertigkeitsprinzip}). Es ist nicht erforderlich, \wikiItalicFt{sagen} zu können, ob das Gebilde wahr oder falsch ist. Es genügt, dass die Frage nach Wahrheit („Zutreffen“) oder Falschheit („Nicht-Zutreffen“) sinnvoll ist, – was zum Beispiel bei Fragesätzen, Ausrufen und Wünschen nicht der Fall ist. Aussagen sind somit Sätze, die \wikiLinkFt{Sachverhalte} beschreiben und denen man einen \wikiLinkFt{Wahrheitswert} zuordnen kann.
	}
	Dies gilt natürlich auch, wenn \metasprachlicheSymbole\ verwendet werden, wovon wir im Folgenden reichlich Gebrauch machen.
	Da man \Relationen\ und \logischenAusdruecken\ ebenfalls einen \Wahrheitswert\ zuordnen kann%
	\footnote{%
		Zumindest prinzipiell nach Ersetzung von \Variablen\ durch konkrete Werte.
	},
	können wir sie auch als \gloFt{Aussagen} behandeln.
	Es handelt sich dann um \defTxt{\logischeA}, im Gegensatz zu \defTxt{\metasprachlichenAussagen}.
}

\newVerweis     {\atomareAussage} {\glstext}{atomareAussage}
\newVerweis[n]  {\atomareAussagen}{\glstext}{atomareAussage}
\newglossaryentry{atomareAussage}{
	name       =            {---, atomare \addIdx[
		name   =            {---, atomare},
		sort   =        {Aussage, atomare}] {atomareAussage}},
	sort       =        {Aussage, atomare},
	text       ={atomare Aussage},
	see        ={unzerlegbar,zerlegbar,zusammengesetzt},
	description={\todoOk%
		\atomarBeschreibung{Eine \Aussage}{sie keine \echteTeilaussage}
	}
}

\newVerweis      {\formaleAussage}  {\glstext} {formaleAussage}
\newVerweis[n]   {\formalenAussagen}{\glsuseri}{formaleAussage}
\newglossaryentry {formaleAussage}{
	name       =  {formaleAussage \addIdx      {formaleAussage}},
	name       =             {---, formale \addIdx[
		name   =             {---, formale},
		sort   =         {Aussage, formale}]   {formaleAussage}},
	sort       =         {Aussage, formale},
	text       ={formale  Aussage},
	user1      ={formalen Aussage},
	description={\todoOk%
		Eine \GloFt{formale Aussage} ist eine \Aussage\ in \Objektsprache.
	}
}

\newVerweis      {\metasprachlicheAussage} {\glstext}       {metasprachlicheAussage}
\newVerweis[n]   {\metasprachlicheAussagen}{\glstext}       {metasprachlicheAussage}
\newVerweis[n]  {\metasprachlichenAussagen}{\glsuseri}      {metasprachlicheAussage}
\newglossaryentry {metasprachlicheAussage}{
	name       =                     {---, metasprachliche \addIdx[
		name   =                     {---, metasprachliche},
		sort   =                 {Aussage, metasprachliche}]{metasprachlicheAussage}},
	sort       =                 {Aussage, metasprachliche},
	text       ={metasprachliche  Aussage},
	user1      ={metasprachlichen Aussage},
	description={\todoOk%
		Eine \GloFt{metasprachliche Aussage} ist eine \Aussage\ in \Metasprache.
	}
}

\newVerweis     {\logischeAussage} {\glstext} {logischeAussage}
\newVerweis[n]  {\logischeAussagen}{\glstext} {logischeAussage}
\newVerweis     {\logischeA}       {\glsuseri}{logischeAussage}
\newglossaryentry{logischeAussage}{
	name       =             {---, logische \addIdx[
		name   =             {---, logische},
		sort   =         {Aussage, logische}] {logischeAussage}},
	sort       =         {Aussage, logische},
	text       ={logische Aussage},
	user1      ={logische},
	description={\todoOk%
		\GloFt{Logische \Aussagen} sind \logischeAusdruecke, wozu auch Ergebnisse von \Relationen\ sowie Ergebnisse von \Funktionen\ mit \Wertebereich\ aus den \Wahrheitswerten\ gehören können.
	}
}

\newVerweis     {\parametrisierteAussage}{\glstext}        {parametrisierteAussage}
\newglossaryentry{parametrisierteAussage}{
	name       =                    {---, parametrisierte \addIdx[
		name   =                    {---, parametrisierte},
		sort   =                {Aussage, parametrisierte}]{parametrisierteAussage}},
	sort       =                {Aussage, parametrisierte},
	text       ={parametrisierte Aussage},
	user1      ={parametrisiert},
	description={\todoGeprueft
		Eine \Aussage\ heißt \GloFt{parametrisiert}, wenn sie mindestens einen \Parameter\ enthält.
	}
}

\newVerweis     {\zerlegbareAussage}{\glstext}   {zerlegbareAussage}
\newVerweis     {\zerlegbarA}       {\glsuseri}  {zerlegbareAussage}
\newglossaryentry{zerlegbareAussage}{
	name       =               {---, zerlegbare \addIdx[
		name   =               {---, zerlegbare},
		sort   =           {Aussage, zerlegbare}]{zerlegbareAussage}},
	sort       =           {Aussage, zerlegbare},
	text       ={zerlegbare Aussage},
	user1      ={zerlegbar},
	description={\todoOk%
		Eine \Aussage\ heißt \defTxt{\zerlegbar}, wenn sie mindestens eine \echteTeilaussage\ enthält.
	}
}

\newVerweis     {\Aussagedefinition}  {\glstext}{Aussagedefinition}
\newVerweis[en] {\Aussagedefinitionen}{\glstext}{Aussagedefinition}
\newglossaryentry{Aussagedefinition}{
	name        ={Aussagedefinition \addIdx     {Aussagedefinition}},
	text        ={Aussagedefinition},
	symbol      ={\ensuremath{\MtsDefEquiv}},
	see         ={Objektdefinition},
	description ={\todoOk%
		\SymbolAmRand{Aussagedefinition}%
		Eine \Metadefinition: Die formale Definition einer \Aussage\ mittels \MtsDefEquiv.
		Gewissermaßen ist $A$ nur eine andere Schreibweise für $B$.
	}
}

\newVerweis     {\Aussagenbereich}{\glstext} {Aussagenbereich}
\newglossaryentry{Aussagenbereich}{
	name        ={Aussagenbereich \addIdx    {Aussagenbereich}},
	text        ={Aussagenbereich},
	symbol      ={\ensuremath{\MtsAussagen}},
	description ={\todoOk%
		\SymbolAmRand{Aussagenbereich}%
		Der \GloFt{Aussagenbereich} \MtsAussagen\ ist der \Bereich\ aller \formalenAussagen, \textdh\ der \Aussagen\ in \Objektsprache.
		Es kann $\MtsAussagen \MtsSubsetEq \MtsUniversum$ gelten, muss es aber nicht.
	}
}

\newVerweis         {\Aussagenlogik}{\glstext} {Aussagenlogik}
\newVerweis         {\AussagenL}    {\glsuseri}{Aussagenlogik}
\longnewglossaryentry{Aussagenlogik}{
	name            ={Aussagenlogik \addIdx    {Aussagenlogik}},
	text            ={Aussagenlogik},
	user1           ={Aussagen-},
%%%	symbol      ={\ensuremath{Mathmode}},% ToDo=Mathmode
%%%	user6       ={Textmode},
	see             ={Aussage,Junktor,Logik,Praedikatenlogik,Wahrheitswert},
}{\todoPruefen%
%%%	\SymbolAmRand{Aussagenlogik}%
	\wikicite{bib:Aussagenlogik}{
		Die \wikiBoldFt{Aussagenlogik} ist ein Teilgebiet der \wikiLinkFt{Logik}, das sich mit Aussagen und deren Verknüpfung durch \wikiLinkFt{Junktoren} befasst, ausgehend von strukturlosen \wikiLinkFt{Elementaraussagen} (Atomen), denen ein \wikiLinkFt{Wahrheitswert} zugeordnet wird. In der \wikiItalicFt{klassischen Aussagenlogik} wird jeder Aussage genau einer der zwei Wahrheitswerte „wahr“ und „falsch“ zugeordnet. Der Wahrheitswert einer zusammengesetzten Aussage lässt sich ohne zusätzliche Informationen aus den Wahrheitswerten ihrer Teilaussagen bestimmen.
	}
}

\newVerweis     {\Auswertung}  {\glstext}{Auswertung}
\newVerweis[en] {\Auswertungen}{\glstext}{Auswertung}
\newglossaryentry{Auswertung}{
	name        ={Auswertung \addIdx     {Auswertung}},
	text        ={Auswertung},
%%%	symbol      ={\ensuremath{Mathmode}},% ToDo=Mathmode
%%%	user6       ={Textmode},
	description ={\todoOk%
%%%		\SymbolAmRand{Auswertung}%
		Eine \GloFt{Auswertung} ist für \ASBA\ eine statistische oder andere Auswertung, die bestimmten Elementen der Datei \textbzw\ Datenbank zugeordnet sind.
		\textZB\ können zu einem \Satz\ alle für einen \Beweis\ notwendigen \Axiome\ angegeben werden.
	}
}

\newVerweis     {\Axiom}  {\glstext}{Axiom}
\newVerweis[e]  {\Axiome} {\glstext}{Axiom}
\newVerweis[en] {\Axiomen}{\glstext}{Axiom}
\newglossaryentry{Axiom}{
	name        ={Axiom \addIdx     {Axiom}},
	text        ={Axiom},
%%%	symbol      ={\ensuremath{Mathmode}},% ToDo=Mathmode
%%%	user6       ={Textmode},
	see         ={MtsAxiom,MtsAxiomSet},
	description ={\todoOk%
%%%		\SymbolAmRand{Axiom}%
		Ein \GloFt{Axiom} ist eine \Aussage, die nicht aus anderen Aussagen abgeleitet werden kann.
		Es können wie bei \Saetzen\ \Praemissen\ und \Konklusionen\ vorhanden sein, aber keine \Beweise.
	}
}

\newVerweis     {\Axiomensystem} {\glstext}{Axiomensystem}
\newVerweis[e]  {\Axiomensysteme}{\glstext}{Axiomensystem}
\newglossaryentry{Axiomensystem}{
	name        ={Axiomensystem \addIdx    {Axiomensystem}},
	text        ={Axiomensysteme},
%%%	symbol      ={\ensuremath{Mathmode}},% ToDo=Mathmode
%%%	user6       ={Textmode},
	description ={\todoGeprueft%
%%%		\SymbolAmRand{Axiomensystem}%
		Eine \Menge\ von \Axiomen.
	}
}

%B === B === B === B === B === B === B === B === B === B === B === B === B === B

\newVerweis     {\Basisregel} {\glstext}{Basisregel}
\newVerweis[n]  {\Basisregeln}{\glstext}{Basisregel}
\newglossaryentry{Basisregel}{
	name        ={Basisregel \addIdx    {Basisregel}},
	text        ={Basisregel},
%%%	symbol      ={\ensuremath{Mathmode}},% ToDo=Mathmode
%%%	user6       ={Textmode},
	description ={\todoPruefen%
%%%		\SymbolAmRand{Basisregel}%
		Eine \Schlussregel, die nicht mehr auf andere zurückgeführt wird.
		Obwohl das auch auf die \Identitaetsregeln\ zutrifft, werden diese \hier\ aber nicht dazu gezählt.
	}
}

\newVerweis     {\Baustein} {\glstext}{Baustein}
\newVerweis[e]  {\Bausteine}{\glstext}{Baustein}
\newglossaryentry{Baustein}{
	name        ={Baustein \addIdx    {Baustein}},
	text        ={Baustein},
%%%	symbol      ={\ensuremath{Mathmode}},% ToDo=Mathmode
%%%	user6       ={Textmode},
	description ={\todoBeschreiben%
%%%		\SymbolAmRand{Baustein}%
	}
}

\newVerweis         {\Begriff}  {\glstext}{Begriff}
\newVerweis[e]      {\Begriffe} {\glstext}{Begriff}
\newVerweis[en]     {\Begriffen}{\glstext}{Begriff}
\newVerweis[s]      {\Begriffs} {\glstext}{Begriff}
\longnewglossaryentry{Begriff}{
	name            ={Begriff \addIdx     {Begriff}},
	text            ={Begriff},
%%%	symbol          ={\ensuremath{Mathmode}},% ToDo=Mathmode
%%%	user6           ={Textmode},
	see             ={Bezeichnung},
}{\todoOk%
%%%	\SymbolAmRand{Begriff}%
	\wikicite{bib:Begriff}{
		Mit dem Ausdruck \wikiBoldFt{Begriff} (\wikiLinkFt{mittelhochdeutsch} und \wikiLinkFt{frühneuhochdeutsch} \wikiItalicFt{begrif} oder \wikiItalicFt{begrifunge}) ist allgemein der \wikiLinkFt{Bedeutungsinhalt} einer \wikiLinkFt{Bezeichnung} angesprochen. Die Abgrenzung zwischen Begriffen und rein gedanklichen (mentalen) Einheiten erfolgt jedoch oft unscharf: Teilweise wird ein \wikiItalicFt{Begriff} als „mentale Informationseinheit“ beschrieben, (also genauso wie in der Kognitionswissenschaft das Konzept). Präziser ist die Abgrenzung des \wikiItalicFt{Begriffes} als \wikiItalicFt{Konzept, das sprachlich benannt ist}, oder geradezu als die \wikiItalicFt{Kombination aus einer sprachlichen Bezeichnung und dem entsprechenden Konzept}.
	}
}

\newVerweis     {\Beispielsymbol}{\glstext}{Beispielsymbol}
\newglossaryentry{Beispielsymbol}{
	name        ={Beispielsymbol \addIdx   {Beispielsymbol}},
	text        ={Beispielsymbol},
%%%	symbol      ={\ensuremath{Mathmode}},% ToDo=Mathmode
%%%	user6       ={Textmode},
	see         ={Symbol},
	description ={\todoBeschreiben%
%%%		\SymbolAmRand{Beispielsymbol}%
	}
}

\dummyVerweis{\Belegung}{\glstext}{Belegung}% ToDo=Belegung (einer Sprache)

\newVerweis         {\Benennung}  {\glstext}{Benennung}
\newVerweis[en]     {\Benennungen}{\glstext}{Benennung}
\longnewglossaryentry{Benennung}{
	name            ={Benennung \addIdx     {Benennung}},
	text            ={Benennung},
%%%	symbol          ={\ensuremath{Mathmode}},% ToDo=Mathmode
%%%	user6           ={Textmode},
	see             ={Bezeichnung},
}{\todoOk%
%%%	\SymbolAmRand{Benennung}%
	\wikicite{bib:Benennung}{
		Eine \wikiBoldFt{Benennung} ist die \wikiLinkFt{Bezeichnung} eines Gegenstandes durch ein \wikiLinkFt{Wort} oder mehrere Wörter. Die Benennung gilt in der Sprachwissenschaft und in der \wikiLinkFt{Terminologielehre} als die sprachliche Form, mit der \wikiLinkFt{Begriffe} ins Bewusstsein gerufen werden. Eine Benennung ist insofern die Versprachlichung einer Vorstellung. Der weiter gefasste Oberbegriff \wikiItalicFt{Bezeichnung} beinhaltet demgegenüber, neben der \wikiItalicFt{Benennung}, auch nichtsprachliches, wie Nummern, Notationen und Symbole. Bei einer \wikiLinkFt{fachsprachlichen} Benennung spricht man auch von einem \wikiLinkFt{Fachausdruck} oder Terminus. Benennungen kommen als Einwort- und als \wikiLinkFt{Mehrwortbenennungen}, auch Mehrworttermini genannt, vor.
	}
	Bei \GloFt{Benennungen} spielt im Gegensatz zu \Symbolketten\ die Schriftart üblicherweise keine Rolle.
}

\newVerweis     {\Bereich}  {\glstext} {Bereich}
\newVerweis[e]  {\Bereiche} {\glstext} {Bereich}
\newVerweis[en] {\Bereichen}{\glstext} {Bereich}
\newVerweis     {\BEREICH}  {\glsuseri}{Bereich}
\newglossaryentry{Bereich}{
	name        ={Bereich \addIdx     {Bereich}},
	text        ={Bereich},
	user1       ={BEREICH},
%%%	symbol      ={\ensuremath{Mathmode}},% ToDo=Mathmode
%%%	user6       ={Textmode},
	see         ={Element,Klasse,leererBereich,Menge},
	description ={\todoOk%
%%%		\SymbolAmRand{Bereich}%
		Ein \GloFt{Bereich} ist eine Zusammenfassung von \Aussagen\ und \Objekten.
		Für solche Zusammenfassungen brauchen wir nur wenige Eigenschaften, die explizit angegeben werden.
		Die in einem \gloFt{Bereich} zusammengefassten \Aussagen\ und \Objekte\ bezeichnen wir wie üblich als seine \DefFt{\Elemente}.
		\Klassen\ und \Mengen\ sind spezielle \gloFt{Bereiche}.%
		\footnote{In der Tat ist \MtsUniversum\ nur eine \Klasse\ und keine \Menge.}
	}
}

\newVerweis     {\leererBereich}{\glstext}{leererBereich}
\newglossaryentry{leererBereich}{
	name       =           {---, leere \addIdx[
		name   =           {---, leere},
		sort   =       {Bereich, leere}]  {leererBereich}},
	sort       =       {Bereich, leere},
	text       ={leerer Bereich},
%%%	symbol     ={\ensuremath{Mathmode}},% ToDo=Mathmode
%%%	user6      ={Textmode},
	description={\todoPruefen%
%%%		\SymbolAmRand{leererBereich}%
		\MtsEmptyset, die \leereMenge, ist auch der einzige \Bereich\ ohne \Elemente.
	}
}

\newVerweis     {\Bereichsoperation}  {\glstext}{Bereichsoperation}
\newVerweis[en] {\Bereichsoperationen}{\glstext}{Bereichsoperation}
\newglossaryentry{Bereichsoperation}{
	name        ={Bereichsoperation \addIdx     {Bereichsoperation}},
	text        ={Bereichsoperation},
%%%	symbol      ={\ensuremath{Mathmode}},% ToDo=Mathmode
%%%	user6       ={Textmode},
	description ={\todoOk%
%%%		\SymbolAmRand{Bereichsoperation}%
		Eine \GloFt{Bereichsoperation} ist eine \Operation\ auf \Bereichen.
		\Hier\ sind <es die \Operationen\ \MtsCup, \MtsCap\ und \MtsTimes.
	}
}

\newVerweis     {\Bereichsrelation}  {\glstext}{Bereichsrelation}
\newVerweis[en] {\Bereichsrelationen}{\glstext}{Bereichsrelation}
\newglossaryentry{Bereichsrelation}{
	name        ={Bereichsrelation \addIdx     {Bereichsrelation}},
	text        ={Bereichsrelation},
%%%	symbol      ={\ensuremath{Mathmode}},% ToDo=Mathmode
%%%	user6       ={Textmode},
	description ={\todoOk%
%%%		\SymbolAmRand{Bereichsrelation}%
		Eine \GloFt{Bereichsrelationen} ist eine \Relation\ zwischen \Bereichen.
		\Hier\ sind es die acht \Relationen\ \MtsSubset, \MtsSubsetEq, \MtsSupset, \MtsSupsetEq, \MtsSubsetN, \MtsSubsetEqN, \MtsSupsetN\ und \MtsSupsetEqN.
	}
}

\newVerweis     {\beschraenkt}  {\glstext}{beschraenkt}
\newVerweis[e]  {\beschraenkte} {\glstext}{beschraenkt}
\newVerweis[en] {\beschraenkten}{\glstext}{beschraenkt}
\newglossaryentry{beschraenkt}{
	name        ={beschränkt \addIdx[
		name    ={beschränkt}]            {beschraenkt}},
	text        ={beschränkt},
%%%	symbol      ={\ensuremath{Mathmode}},% ToDo=Mathmode
%%%	user6       ={Textmode},
	description ={\todoPruefen%
%%%		\SymbolAmRand{beschraenkt}%
		Eine \Schlussregel\ heißt \beschraenkt, wenn sie nur endlich viele Prämissen und Konklusionen hat.
	}
}

\newVerweis         {\Beweis}  {\glstext}{Beweis}
\newVerweis[e]      {\Beweise} {\glstext}{Beweis}
\newVerweis[es]     {\Beweises}{\glstext}{Beweis}
\newVerweis[en]     {\Beweisen}{\glstext}{Beweis}
\longnewglossaryentry{Beweis}{
	name            ={Beweis \addIdx     {Beweis}},
	text            ={Beweis},
%%%	symbol          ={\ensuremath{Mathmode}},% ToDo=Mathmode
%%%	user6           ={Textmode},
	see             ={Ableitung,Aussage,Axiom},
}{\todoOk%
%%%	\SymbolAmRand{Beweis}%
	\wikicite{bib:Beweis}{
		Ein \wikiBoldFt{Beweis} ist in der Mathematik die als fehlerfrei anerkannte Herleitung der Richtigkeit bzw. der Unrichtigkeit einer \wikiBoldFt{Aussage} aus einer Menge von \wikiLinkFt{Axiomen}, die als wahr vorausgesetzt werden, und anderen Aussagen, die bereits bewiesen sind. Um den Beweis klar vom gültigen Schluss zu unterscheiden, spricht man auch vom \wikiBoldFt{axiomatischen Beweis}.

		Umfangreichere Beweise von mathematischen Sätzen werden in der Regel in mehrere kleine Teilbeweise aufgeteilt, siehe dazu \wikiLinkFt{Satz} und \wikiLinkFt{Hilfssatz}.

		In der \wikiLinkFt{Beweistheorie}, einem Teilgebiet der \wikiLinkFt{mathematischen Logik}, werden Beweise formal als \wikiLinkFt{Ableitungen} aufgefasst und selbst als mathematische Objekte betrachtet, um etwa die Beweisbarkeit oder Unbeweisbarkeit von Sätzen aus gegebenen Axiomen selbst zu beweisen.
	}
	Ein \GloFt{Beweis} besteht aus einer \Folge\ von \Beweisschritten, die aus gegebenen \Praemissen\ \Konklusionen\ ableitet.
}
\newsynonym{\beweisbar}{beweisbar}{\ableitbar}

\newVerweis     {\Beweisschritt}  {\glstext}{Beweisschritt}
\newVerweis[e]  {\Beweisschritte} {\glstext}{Beweisschritt}
\newVerweis[en] {\Beweisschritten}{\glstext}{Beweisschritt}
\newglossaryentry{Beweisschritt}{
	name        ={Beweisschritt \addIdx     {Beweisschritt}},
	text        ={Beweisschritt},
%%%	symbol      ={\ensuremath{Mathmode}},% ToDo=Mathmode
%%%	user6       ={Textmode},
	see         ={MtsBeweisschritt,MtsBeweisschrittSet,MtsBeweisschrittTup},
	description ={\todoPruefen%
%%%		\SymbolAmRand{Beweisschritt}%
		Eine Vorschrift, wie aus vorgegebenen \Aussagen\ (den \Praemissen) weitere (die \Konklusionen) folgen.
	}
}

\newVerweis     {\Beweisschrittfolge} {\glstext}{Beweisschrittfolge}
\newVerweis[n]  {\Beweisschrittfolgen}{\glstext}{Beweisschrittfolge}
\newglossaryentry{Beweisschrittfolge}{
	name        ={Beweisschrittfolge \addIdx    {Beweisschrittfolge}},
	text        ={Beweisschrittfolge},
%%%	symbol      ={\ensuremath{Mathmode}},% ToDo=Mathmode
%%%	user6       ={Textmode},
	description ={\todoPruefen%
%%%		\SymbolAmRand{Beweisschrittfolge}%
		Eine Folge von \Beweisschritten.
	}
}

\newVerweis     {\Beweisschrittmenge} {\glstext}{Beweisschrittmenge}
\newVerweis[n]  {\Beweisschrittmengen}{\glstext}{Beweisschrittmenge}
\newglossaryentry{Beweisschrittmenge}{
	name        ={Beweisschrittmenge \addIdx    {Beweisschrittmenge}},
	text        ={Beweisschrittmenge},
%%%	symbol      ={\ensuremath{Mathmode}},% ToDo=Mathmode
%%%	user6       ={Textmode},
	description ={\todoPruefen%
%%%		\SymbolAmRand{Beweisschrittmenge}%
		Eine \Menge\ von \Beweisschritten, insbesondere die \Menge\ der Glieder einer \Beweisschrittfolge.
	}
}

\newVerweis         {\Bezeichnung}  {\glstext}{Bezeichnung}
\newVerweis[en]     {\Bezeichnungen}{\glstext}{Bezeichnung}
\longnewglossaryentry{Bezeichnung}{
	name            ={Bezeichnung \addIdx     {Bezeichnung}},
	text            ={Bezeichnung},
%%%	symbol          ={\ensuremath{Mathmode}},% ToDo=Mathmode
%%%	user6           ={Textmode},
	see             ={Begriff},
}{\todoOk%
%%%	\SymbolAmRand{Bezeichnung}%
	\wikicite{bib:Bezeichnung}{
		Eine \wikiBoldFt{Bezeichnung} ist die Repräsentation eines \wikiLinkFt{Begriffs} mit sprachlichen oder anderen Mitteln. Erfolgt diese Repräsentation mittels Wörtern, handelt es sich um eine \wikiLinkFt{Benennung}. Eine nichtsprachliche Bezeichnung kann durch ein \wikiLinkFt{Symbol} erfolgen.
	}
	Eine \GloFt{Bezeichnung} ist eine \Benennung\ oder eine \Symbolkette.
	\Benennungen\ sind \hier\ aber spezielle \Symbolketten.
	Eine \Symbolkette\ als \Bezeichnung\ wird normalerweise soweit möglich als \Benennung\ interpretiert, so dass die Schriftart dann, im Gegensatz zu \Symbolketten, keine Rolle mehr spielt.
}

\newVerweis     {\binaer}  {\glstext}{binaer}
\newVerweis[e]  {\binaere} {\glstext}{binaer}
\newVerweis[en] {\binaeren}{\glstext}{binaer}
\newglossaryentry{binaer}{
	name        ={binär \addIdx[
		name    ={binär}]            {binaer}},
	text        ={binär},
%%%	symbol      ={\ensuremath{Mathmode}},% ToDo=Mathmode
%%%	user6       ={Textmode},
	see         ={unaer},
	description ={\todoPruefen%
%%%		\SymbolAmRand{binaer}%
		Eine \Operation, \Funktion\ oder \Relation\ heißt \GloFt{binär}, wenn ihre \Stelligkeit\ gleich 2 ist.
	}
}

\newglossaryentry{Buchstabe}{% Platzhalter für Buchstabe
	name        ={Buchstabe \addIdx {Buchstabe}},
	text        ={Buchstabe},
%%%	symbol      ={\ensuremath{Mathmode}},% ToDo=Mathmode
%%%	user6       ={Textmode},
	description ={\todoOk%
%%%		\SymbolAmRand{Buchstabe}%
	}
}

\newVerweis     {\deutschenBuchstaben}{\glsuseri}{deutscherBuchstabe}
\newVerweis     {\deutscherBuchstabe} {\glstext} {deutscherBuchstabe}
\newglossaryentry{deutscherBuchstabe}{
	name       =                {---, deutscher \addIdx[
		name   =                {---, deutscher},
		sort   =          {Buchstabe, deutscher}]{deutscherBuchstabe}},
	sort       =          {Buchstabe, deutscher},
	text       ={deutscher Buchstabe},
	user1      ={deutschen Buchstaben},
%%%	symbol     ={\ensuremath{Mathmode}},% ToDo=Mathmode
%%%	user6      ={Textmode},
	see        ={deutschesWort,Textwort,griechischerBuchstabe,lateinischerBuchstabe,Textbuchstabe},
	description={\todoOk%
%%%		\SymbolAmRand{deutscherBuchstabe}%
		Ein Buchstabe des deutschen Alphabets.
	}
}

\newVerweis     {\griechischenBuchstaben}{\glsuseri}   {griechischerBuchstabe}
\newVerweis     {\griechischerBuchstabe} {\glstext}    {griechischerBuchstabe}
\newglossaryentry{griechischerBuchstabe}{
	name       =                   {---, griechischer \addIdx[
		name   =                   {---, griechischer},
		sort   =             {Buchstabe, griechischer}]{griechischerBuchstabe}},
	sort       =             {Buchstabe, griechischer},
	text       ={griechischer Buchstabe},
	user1      ={griechischen Buchstaben},
%%%	symbol     ={\ensuremath{Mathmode}},% ToDo=Mathmode
%%%	user6      ={Textmode},
	see        ={griechischesWort,Textwort,deutscherBuchstabe,lateinischerBuchstabe,Textbuchstabe},
	description={\todoOk%
%%%		\SymbolAmRand{griechischerBuchstabe}%
		Ein Buchstabe des griechischen Alphabets.
	}
}

\newVerweis     {\lateinischenBuchstaben}{\glsuseri}   {lateinischerBuchstabe}
\newVerweis     {\lateinischerBuchstabe} {\glstext}    {lateinischerBuchstabe}
\newglossaryentry{lateinischerBuchstabe}{
	name       =                   {---, lateinischer \addIdx[
		name   =                   {---, lateinischer},
		sort   =             {Buchstabe, lateinischer}]{lateinischerBuchstabe}},
	sort       =             {Buchstabe, lateinischer},
	text       ={lateinischer Buchstabe},
	user1      ={lateinischen Buchstaben},
%%%	symbol     ={\ensuremath{Mathmode}},% ToDo=Mathmode
%%%	user6      ={Textmode},
	see        ={lateinischesWort,Textwort,deutscherBuchstabe,griechischerBuchstabe,Textbuchstabe},
	description={\todoOk%
%%%		\SymbolAmRand{lateinischerBuchstabe}%
		Ein Buchstabe des lateinischen Alphabets.
	}
}

%D === D === D === D === D === D === D === D === D === D === D === D === D === D

\newVerweis         {\Darstellung}  {\glstext}{Darstellung}
\newVerweis[en]     {\Darstellungen}{\glstext}{Darstellung}
\longnewglossaryentry{Darstellung}{
	name            ={Darstellung \addIdx     {Darstellung}},
	text            ={Darstellung},
%%%	symbol          ={\ensuremath{Mathmode}},% ToDo=Mathmode
%%%	user6           ={Textmode},
}{\todoErgaenzen%
%%%	\SymbolAmRand{Darstellung}%
	\wikicite{bib:Darstellung}{
		Unter \wikiBoldFt{Darstellung} (zur semantischen Wurzel \wikiItalicFt{dar}- „öffentlich übergeben“, vergleiche Darbietung, \wikiLinkFt{Darlehen}, darreichen) versteht man die Umsetzung von \wikiLinkFt{Sachverhalten}, \wikiLinkFt{Ereignissen} oder abstrakten Konzepten mittels \wikiLinkFt{Zeichen}, performativer \wikiLinkFt{Handlungen} oder \wikiLinkFt{Modellen}. Historisch reicht die Darstellung von der \wikiLinkFt{mündlichen Überlieferung} über das \wikiLinkFt{Schauspiel} bis zur \wikiLinkFt{Computergrafik} und schließt zahlreiche Vermittlungsmethoden zwischen \wikiLinkFt{Text}, \wikiLinkFt{Bild} und künstlerischer \wikiLinkFt{Aufführung} ein.
	}
	Die \GloFt{Darstellung} mathematischer \Objekte\ geschieht auf mehreren Ebenen
}

\newVerweis     {\interneDarstellung}{\glstext} {interneDarstellung}
\newglossaryentry{interneDarstellung}{
	name       =                 {---, interne \addIdx[
		name   =                 {---, interne},
		sort   =         {Darstellung, interne}]{interneDarstellung}},
	sort       =         {Darstellung, interne},
	text       ={interne Darstellung},
%%%	symbol      ={\ensuremath{Mathmode}},% ToDo=Mathmode
%%%	user6       ={Textmode},
	description={\todoBeschreiben%
%%%		\SymbolAmRand{interneDarstellung}%
	}
}

\newVerweis     {\logischeDarstellung}{\glstext} {logischeDarstellung}
\newVerweis    {\logischenD}          {\glsuseri}{logischeDarstellung}
\newglossaryentry{logischeDarstellung}{
	name       =                 {---, logische \addIdx[
		name   =                 {---, logische},
		sort   =         {Darstellung, logische}]{logischeDarstellung}},
	sort       =         {Darstellung, logische},
	text       ={logische Darstellung},
	user1      ={logischen},
%%%	symbol     ={\ensuremath{Mathmode}},% ToDo=Mathmode
%%%	user6      ={Textmode},
	description={\todoBeschreiben%
%%%		\SymbolAmRand{logischeDarstellung}%
	}
}

\newVerweis     {\Darstellungsweise} {\glstext}{Darstellungsweise}
\newVerweis[n]  {\Darstellungsweisen}{\glstext}{Darstellungsweise}
\newglossaryentry{Darstellungsweise}{
	name        ={Darstellungsweise \addIdx    {Darstellungsweise}},
	text        ={Darstellungsweise},
%%%	symbol      ={\ensuremath{Mathmode}},% ToDo=Mathmode
%%%	user6       ={Textmode},
	description ={\todoPruefen%
%%%		\SymbolAmRand{Darstellungsweise}%
		Die Art der \Darstellung\ mathematischer \Objekte.
	}
}

\newVerweis     {\Definitionsbereich} {\glstext} {Definitionsbereich}
\newVerweis[e]  {\Definitionsbereiche}{\glstext} {Definitionsbereich}
\newVerweis     {\DefinitionsB}       {\glsuseri}{Definitionsbereich}
\newglossaryentry{Definitionsbereich}{
	name        ={Definitionsbereich \addIdx     {Definitionsbereich}},
	text        ={Definitionsbereich},
	user1       ={Definitions},
%%%	symbol      ={\ensuremath{Mathmode}},% ToDo=Mathmode
%%%	user6       ={Textmode},
	see         ={MtsDb,Quellbereich,Funktion},
	description ={\todoPruefen%
%%%		\SymbolAmRand{Definitionsbereich}%
		Für eine \Funktion\ \FunktionDef{f}{A}{B} ist $\MtsDb(f)A$ ihr \Definitionsbereich\ (domain).
	}
}

\newVerweis     {\Differenz} {\glstext}{Differenz}
\newglossaryentry{Differenz}{
	name        ={Differenz \addIdx    {Differenz}},
	text        ={Differenz},
%%%	symbol      ={\ensuremath{Mathmode}},% ToDo=Mathmode
%%%	user6       ={Textmode},
	description ={\todoErgaenzen%
%%%		\SymbolAmRand{Differenz}%
		Eine \Bereichsoperation:
	}
}

\newVerweis         {\Diskursuniversum} {\glstext}{Diskursuniversum}
\longnewglossaryentry{Diskursuniversum}{
	name            ={Diskursuniversum \addIdx    {Diskursuniversum}},
	text            ={Diskursuniversum},
	symbol          ={\ensuremath{\MtsUniversum}},
%%%	user6           ={Textmode},
	see             ={Aussage,Begriff,Logik},
}{\todoPruefen%
	\SymbolAmRand{Diskursuniversum}%
	\wikicite{bib:Diskursuniversum}{
		Unter einem \wikiBoldFt{Diskursuniversum} versteht man in der \wikiLinkFt{Logik} und \wikiLinkFt{Sprachphilosophie} die Gesamtheit der Gegenstände, auf die sich Aussagen wie „alle Gegenstände sind … “ (\wikiLinkFt{Allaussage}) oder „es gibt keine Gegenstände, die … sind“ (negative \wikiLinkFt{Existenzaussage}) beziehen. Solche Aussagen sind nur sinnvoll, wenn die Bedeutung von „Gegenstand“ auf einen bestimmten Bereich, das Diskursuniversum, eingeschränkt wird. Ausmaß und Art der Einschränkung hängen vom Inhalt und vom Zusammenhang der Aussagen ab. Es gibt daher nicht nur ein Diskursuniversum, sondern verschiedene Diskursuniversen.

		Der englische Ausdruck \wikiBoldFt{Universe of Discourse} wird auch in der deutschsprachigen Logik- und Informatikliteratur verwendet. Er geht auf \wikiLinkFt{Augustus De Morgan} (1847) zurück und bezeichnet den Bereich der Gegenstände (im weitesten Sinn), über die überhaupt geredet werden soll.

		Missverständnisse und Streit entstehen in der Logik wie im Alltag oft dadurch, dass Personen „aneinander vorbei“ von verschiedenen Dingen reden. Jemand behauptet z. B., dass es keine geflügelten Pferde gibt. Sein Widerpart weist dies mit dem Hinweis auf den \wikiLinkFt{Pegasus} zurück. Beide bewegen sich gedanklich in verschiedenen Welten. Ihr Streit lässt sich schlichten, wenn sie sich auf ein gemeinsames Diskursuniversum einigen, d. h. aushandeln, wovon die Rede (der \wikiLinkFt{Diskurs}) sein soll, ob nur von physisch existierenden Pferden oder auch von \wikiLinkFt{Fabelwesen}.

		Auch beim Gebrauch negativer (komplementärer) \wikiLinkFt{Begriffe} spielt das Diskursuniversum eine Rolle. Ausdrücke wie „Nichtschwimmer“, „Nichtfachmann“, „Nichtwähler“ können sinnvoll nur auf Personen angewandt werden. Die Nichtwähler bilden mit den Wählern zusammen das auf wahlberechtigte Personen eingeschränkte Diskursuniversum. Die Einschränkung geschieht beim Gebrauch solcher Begriffe automatisch. Wird die Automatik außer Betrieb gesetzt, indem man z. B. einen stillgelegten Schornstein als Nichtraucher bezeichnet, entsteht ein Wortspiel. Allgemein gilt für jeden Begriff: wird er mit dem zugehörigen negativen Begriff vereinigt (genauer: werden deren \wikiLinkFt{Extensionen} vereinigt), so bilden beide zusammen das Diskursuniversum oder den Bereich der Anwendungsfälle des positiv bestimmten Komplementärbegriffs:

		[eine Tabelle]

		In der \wikiLinkFt{Mengenlehre} entspricht dem Diskursuniversum die \wikiLinkFt{Grundmenge}, die Mengen entsprechen den Begriffen, die \wikiLinkFt{Komplemente} von Mengen der Negation von Begriffen. In der \wikiLinkFt{Prädikatenlogik} entspricht dem Diskursuniversum der Bereich der \wikiLinkFt{Definitionsmenge}, den die \wikiLinkFt{Gegenstandsvariable} einer \wikiLinkFt{quantifizierten} Aussage durchlaufen kann.

		Das \wikiItalicFt{Universe of Discourse} wird in der Logik zumeist abgekürzt mit \wikiItalicFt{U}, in der Informatik auch mit \wikiItalicFt{UoD}.

		Das \wikiItalicFt{U} ist in der Regel eine Teilmenge aller existierenden Objekte und insbesondere in der Prädikatenlogik der bei der Verwendung von \wikiLinkFt{Quantoren} festgelegte oder vorausgesetzte Objektbereich.
	}
	Das \GloFt{Diskursuniversum} \MtsUniversum\ ist der vorgegebene \Bereich\ aller \Objekte, die in \Aussagen\ einen \Parameter\ ersetzen dürfen.
}

\newVerweis     {\Durchschnitt} {\glstext}{Durchschnitt}
\newglossaryentry{Durchschnitt}{
	name        ={Durchschnitt \addIdx    {Durchschnitt}},
	text        ={Durchschnitt},
%%%	symbol      ={\ensuremath{Mathmode}},% ToDo=Mathmode
%%%	user6       ={Textmode},
	description ={\todoErgaenzen%
%%%		\SymbolAmRand{Durchschnitt}%
		Eine \Bereichsoperation:
	}
}

%E === E === E === E === E === E === E === E === E === E === E === E === E === E

\newVerweis     {\echt}  {\glstext}{echt}
\newVerweis[e]  {\echte} {\glstext}{echt}
\newVerweis[es] {\echtes}{\glstext}{echt}
\newglossaryentry{echt}{
	name        ={echt \addIdx     {echt}},
	text        ={echt},
	description ={\todoGeprueft%
%%%		\SymbolAmRand{echt}%
		Attribut für \Oberaussage, \Oberfolge, \Oberformel, \Oberobjekt, \Obermenge, \Obersprache, \Obersymbol, \Teilaussage, \Teilfolge,  \Teilformel, \Teilobjekt, \Teilmenge, \Teilsprache\ und \Teilsymbol.
	}
}

\newVerweis     {\Eigenschaft}  {\glstext}{Eigenschaft}
\newVerweis[en] {\Eigenschaften}{\glstext}{Eigenschaft}
\newglossaryentry{Eigenschaft}{
	name        ={Eigenschaft \addIdx   {Eigenschaft}},
	text        ={Eigenschaft},
%%%	symbol      ={\ensuremath{Mathmode}},% ToDo=Mathmode
%%%	user6       ={Textmode},
	description ={\todoPruefen% ToDo A(x) ist unpräzise
%%%		\SymbolAmRand{Eigenschaft}%
		Ist $x$ ein \Parameter\ einer \Aussage\ $A$, so ist die \Aussage\ \statement{$x$ hat die \Eigenschaft\ $A$} gleichbedeutend damit, das $A$ gilt.
		Wir schreiben etwas unpräzise auch $A(x)$, besonders dann, wenn auch $A(y)$ für $y \ne x$ von Interesse ist.
	}
}

% TODO ### ab hier weiter prüfen
\newVerweis      {\interessierendeEigenschaft}  {\glstext}      {interessierendeEigenschaft}
\newVerweis     {\interessierendenEigenschaft}  {\glsuseri}     {interessierendeEigenschaft}
\newVerweis[en] {\interessierendenEigenschaften}{\glsuseri}     {interessierendeEigenschaft}
\newglossaryentry {interessierendeEigenschaft}{
	name       =                 {Eigenschaft, interessierende \addIdx[
		name   =                 {Eigenschaft, interessierende},
		sort   =                 {Eigenschaft, interessierende}]{interessierendeEigenschaft}},
	sort       =                 {Eigenschaft, interessierende},
	text       ={interessierende  Eigenschaft},
	user1      ={interessierenden Eigenschaft},
%%%	symbol     ={\ensuremath{Mathmode}},% ToDo=Mathmode
%%%	user6      ={Textmode},
	description={\todoPruefen%
%%%		\SymbolAmRand{interessierendeEigenschaft}%
		Solche Eigenschaften von \Objekten, die im aktuellen Zusammenhang von Interesse sind, \textzB\ einen bestimmten Wert zu haben, \Element\ aus einer bestimmten \Menge\ zu sein, ein bestimmtes \Objekt\ zu bezeichnen, usw.
	}
}

\newVerweis         {\Element}  {\glstext}{Element}
\newVerweis[e]      {\Elemente} {\glstext}{Element}
\newVerweis[en]     {\Elementen}{\glstext}{Element}
\longnewglossaryentry{Element}{
	name            ={Element \addIdx     {Element}},
	text            ={Element},
%%%	symbol          ={\ensuremath{Mathmode}},% ToDo=Mathmode
%%%	user6           ={Textmode},
	see             ={Mengenlehre,Relation},
}{\todoOk%
%%%	\SymbolAmRand{Element}%
	\wikicite{bib:Element}{
		Ein \wikiBoldFt{Element} in der \wikiLinkFt{Mathematik} ist immer im Rahmen der \wikiLinkFt{Mengenlehre} oder \wikiLinkFt{Klassenlogik} zu verstehen. Die grundlegende \wikiLinkFt{Relation}, wenn $x$ ein Element ist und $M$ eine \wikiLinkFt{Menge} oder \wikiLinkFt{Klasse} ist, lautet:
		\begin{quote}
			„$x$ ist Element von $M$“ oder mit Hilfe des \wikiLinkFt{Elementzeichens} „$x \in M$“.
		\end{quote}
		Die Mengendefinition von \wikiLinkFt{Georg Cantor} beschreibt anschaulich, was unter einem Element im Zusammenhang mit einer Menge zu verstehen ist:
		\begin{quote}
			„Unter einer ‚Menge‘ verstehen wir jede Zusammenfassung $M$ von bestimmten wohlunterschiedenen Objekten $m$ unserer Anschauung oder unseres Denkens (welche die ‚Elemente‘ von $M$ genannt werden) zu einem Ganzen.“
		\end{quote}
		Diese anschauliche Mengenauffassung der \wikiLinkFt{naiven Mengenlehre} erwies sich als nicht widerspruchsfrei. Heute wird daher eine \wikiLinkFt{axiomatische} Mengenlehre benutzt, meist die \wikiLinkFt{Zermelo-Fraenkel-Mengenlehre}, teilweise auch eine allgemeinere \wikiLinkFt{Klassenlogik}.
	}
	\Hier\ sind \GloFt{Elemente} stets \Aussagen\ oder \Objekte\ und wir schreiben immer \enquote{\gloFt{Element} \ManFt{aus}} und lassen neben \Mengen\ und \Klassen\ auch \Bereiche\ zu.
}

\newVerweis     {\Elementoperation}  {\glstext}{Elementoperation}
\newVerweis[en] {\Elementoperationen}{\glstext}{Elementoperation}
\newglossaryentry{Elementoperation}{
	name        ={Elementoperation \addIdx     {Elementoperation}},
	text        ={Elementoperation},
%%%	symbol      ={\ensuremath{Mathmode}},% ToDo=Mathmode
%%%	user6       ={Textmode},
	description ={\todoBeschreiben%
%%%		\SymbolAmRand{Elementoperation}%
	}
}

\newVerweis     {\Elementrelation}  {\glstext}{Elementrelation}
\newVerweis[en] {\Elementrelationen}{\glstext}{Elementrelation}
\newglossaryentry{Elementrelation}{
	name        ={Elementrelation \addIdx     {Elementrelation}},
	text        ={Elementrelation},
%%%	symbol      ={\ensuremath{Mathmode}},% ToDo=Mathmode
%%%	user6       ={Textmode},
	see         ={Komponentenrelation},
	description ={\todoOk%
%%%		\SymbolAmRand{Elementrelation}%
		Eine \GloFt{Elementrelation} ist eine \Relation\ zwischen einem \Element\ und einem \Bereich.
		Hier sind es die vier \Relationen\ \MtsIn, \MtsNi, \MtsInN\ und \MtsNiN.
	}
}

\newVerweis     {\Endglied}  {\glstext}{Endglied}
\newVerweis[er] {\Endglieder}{\glstext}{Endglied}
\newglossaryentry{Endglied}{
	name        ={Endglied \addIdx     {Endglied}},
	text        ={Endglied},
%%%	symbol      ={\ensuremath{Mathmode}},% ToDo=Mathmode
%%%	user6       ={Textmode},
	see         ={Anfangsglied,Nachfolger,Vorgaenger,Zwischenglied},
	description ={\todoOk%
%%%		\SymbolAmRand{Endglied}%
		Das \GloFt{Endglied} einer \Kette\ ist ihr letztes (rechts) \Kettenglied.
	}
}

\newVerweis     {\Ergebnis}   {\glstext}{Ergebnis}
\newVerweis[se] {\Ergebnisse} {\glstext}{Ergebnis}
\newcommand*    {\Ergebnissen}[1][]{\glstext[#1]{Ergebnis}[sen]}
\newglossaryentry{Ergebnis}{
	name        ={Ergebnis \addIdx      {Ergebnis}},
	text        ={Ergebnis},
%%%	symbol      ={\ensuremath{Mathmode}},% ToDo=Mathmode
%%%	user6       ={Textmode},
	see         ={MtsErgebnis,MtsErgebnisSet,MtsErgebnisRel},
	description ={\todoPruefen%
%%%		\SymbolAmRand{Ergebnis}%
		Eine \Ableitung:
		Ein \Ergebnis\ eines \Beweises.
	}
}

\newVerweis     {\Ergebnismenge} {\glstext}{Ergebnismenge}
\newVerweis[n]  {\Ergebnismengen}{\glstext}{Ergebnismenge}
\newglossaryentry{Ergebnismenge}{
	name        ={Ergebnismenge \addIdx    {Ergebnismenge}},
	text        ={Ergebnismenge},
%%%	symbol      ={\ensuremath{Mathmode}},% ToDo=Mathmode
%%%	user6       ={Textmode},
	description ={\todoPruefen%
%%%		\SymbolAmRand{Ergebnismenge}%
		Eine \Ableitungsmenge:
		Die \Menge\ \MtsErgebnisSet\ der \Ergebnisse\ eines \Beweises.
	}
}

\newVerweis     {\Ersetzung}  {\glstext}{Ersetzung}
\newVerweis[en] {\Ersetzungen}{\glstext}{Ersetzung}
\newglossaryentry{Ersetzung}{
	name        ={Ersetzung \addIdx     {Ersetzung}},
	text        ={Ersetzung},
%%%	symbol      ={\ensuremath{Mathmode}},% ToDo=Mathmode
%%%	user6       ={Textmode},
	description ={\todoPruefen%
%%%		\SymbolAmRand{Ersetzung}%
		Eine \Funktion\ zur \Transformation\ einer \Formel\ mittels \Ersetzung\ in eine gleichwertige.
		Die \Ersetzung\ heißt \zulaessig, wenn sie vorgegebene Regeln erfüllt.
	}
}

\newVerweis     {\Ersetzungsmenge} {\glstext}{Ersetzungsmenge}
\newVerweis[n]  {\Ersetzungsmengen}{\glstext}{Ersetzungsmenge}
\newglossaryentry{Ersetzungsmenge}{
	name        ={Ersetzungsmenge \addIdx    {Ersetzungsmenge}},
	text        ={Ersetzungsmenge},
%%%	symbol      ={\ensuremath{Mathmode}},% ToDo=Mathmode
%%%	user6       ={Textmode},
	description ={\todoPruefen%
%%%		\SymbolAmRand{Ersetzungsmenge}%
		Eine \Menge\ von \Ersetzungen, meistens mit \MtsErsetzungSet\ bezeichnet.
	}
}

\newVerweis     {\Existenzquantor} {\glstext}{Existenzquantor}
\newglossaryentry{Existenzquantor}{
	name        ={Existenzquantor \addIdx    {Existenzquantor}},
	text        ={Existenzquantor},
	symbol      ={\ensuremath{\MtsExists \text{ \textbzw\ } \OjkExists}},
	description ={\todoPruefen%
		\SymbolAmRand{Existenzquantor}%
		Man nennt den \Quantor\ \MtsExists\ \textbzw\ \OjkExists\ auch \GloFt{Existenzquantor}.
	}
}

%F === F === F === F === F === F === F === F === F === F === F === F === F === F

\newVerweis         {\Fachbegriff}  {\glstext}{Fachbegriff}
\newVerweis[e]      {\Fachbegriffe} {\glstext}{Fachbegriff}
\newVerweis[en]     {\Fachbegriffen}{\glstext}{Fachbegriff}
\longnewglossaryentry{Fachbegriff}{
	name            ={Fachbegriff \addIdx     {Fachbegriff}},
	text            ={Fachbegriff},
%%%	symbol          ={\ensuremath{Mathmode}},% ToDo=Mathmode
%%%	user6           ={Textmode},
	see             ={Begriff,Fachgebiet},
}{\todoOk%
%%%	\SymbolAmRand{Fachbegriff}%
	\wikicite{bib:Terminus}{
		Ein \wikiBoldFt{Terminus} oder \wikiBoldFt{Fachbegriff} ist eine \wikiLinkFt{definierte} \wikiLinkFt{Benennung} für einen \wikiLinkFt{Begriff} innerhalb der \wikiLinkFt{Fachsprache} eines \wikiLinkFt{Fachgebietes}. Synonyme dazu sind auch \wikiBoldFt{Term} oder \wikiBoldFt{Terminus technicus} (lateinisch \wikiItalicFt{terminus technicus}; \wikiLinkFt{Genus} \wikiItalicFt{m.}; \wikiLinkFt{Pl.} \wikiItalicFt{Termini technici}, kurz \wikiItalicFt{Termini}). \wikiItalicFt{Terminus} kann allerdings neben der rein sprachlichen \wikiItalicFt{Benennung} auch den Bedeutungsinhalt, den \wikiItalicFt{Begriff} selbst, ansprechen.

		Eine vergleichbare Bezeichnung ist \wikiBoldFt{Fachwort}. Ein \wikiBoldFt{Fachausdruck} ist ein \wikiLinkFt{sprachlicher Ausdruck}, der in einer Fachsprache verwendet wird und dort eine spezielle Bedeutung besitzt. \wikiItalicFt{Fachausdruck} gilt gegenüber \wikiItalicFt{Fachwort} als ein geeigneteres Ersatzwort für Terminus. Denn ein Terminus kann nicht nur in der Form einer Einwortbenennung, sondern auch als \wikiLinkFt{Mehrwortbenennung} (auch \wikiItalicFt{Mehrwortterminus}) vorliegen.

		Die Menge aller Termini eines Fachgebietes (die Benennungen aller Begriffe) bildet die jeweilige fachspezifische \wikiLinkFt{Terminologie} (den \wikiLinkFt{Fachwortschatz}). Mit der Untersuchung und Aufstellung von Terminologien beschäftigt sich die \wikiLinkFt{Terminologielehre}. Wenn ein Fachwortschatz standardisiert oder normiert ist, spricht man auch von einem \wikiLinkFt{Thesaurus} oder \wikiLinkFt{kontrollierten Vokabular} und nennt die darin enthaltenen Termini \wikiLinkFt{Deskriptoren}.
	}
	Ein \GloFt{Fachbegriff} ist für \ASBA\ eine \Benennung\ für einen \Begriff\ aus einem \Fachgebiet.
	Insbesondere kann es auch ein spezielles \Symbol\ sein.
}

\newVerweis         {\Fachgebiet}  {\glstext}{Fachgebiet}
\newVerweis[s]      {\Fachgebiets} {\glstext}{Fachgebiet}
\newVerweis[e]      {\Fachgebiete} {\glstext}{Fachgebiet}
\newVerweis[en]     {\Fachgebieten}{\glstext}{Fachgebiet}
\longnewglossaryentry{Fachgebiet}{
	name            ={Fachgebiet \addIdx     {Fachgebiet}},
	text            ={Fachgebiet},
%%%	symbol          ={\ensuremath{Mathmode}},% ToDo=Mathmode
%%%	user6           ={Textmode},
}{\todoOk%
%%%	\SymbolAmRand{Fachgebiet}%
	\wikicite{bib:Fachgebiet}{
		\wikiBoldFt{Fachgebiet} (auch \wikiBoldFt{Fachbereich} oder \wikiBoldFt{Fachrichtung} oder \wikiBoldFt{Domäne}) ist das auf ein bestimmtes \wikiLinkFt{Wissensgebiet} begrenzte \wikiLinkFt{Wissen}.
	}
	Ein \GloFt{Fachgebiet} ist für \ASBA\ ein Teilgebiet der Mathematik mit einer zugehörigen Basis aus \Axiomen, \Saetzen, \Fachbegriffen\ und \Darstellungsweisen, \textzB\ \Logik\ und \Mengenlehre.

	Ein \GloFt{Fachgebiet} kann bei \ASBA\ sehr klein sein und im Extremfall kein einziges \Element\ enthalten.
	\emph{Umgebung} wäre vielleicht eine bessere \Bezeichnung, ist aber schon ein verbreiteter \Fachbegriff, so dass \hier\ die \Bezeichnung\ "`Fachgebiet"' verwendet wird.
}

\newVerweis         {\Folge} {\glstext}{Folge}
\newVerweis[n]      {\Folgen}{\glstext}{Folge}
\longnewglossaryentry{Folge}{
	name            ={Folge \addIdx    {Folge}},
	text            ={Folge},
%%%	symbol          ={\ensuremath{Mathmode}},% ToDo=Mathmode
%%%	user6           ={Textmode},
}{\todoGeprueft%
%%%	\SymbolAmRand{Folge}%
	\wikicite{bib:Folge}{
		Als \wikiBoldFt{Folge} oder \wikiBoldFt{Sequenz} wird in der \wikiLinkFt{Mathematik} eine Auflistung (\wikiLinkFt{Familie}) von endlich oder unendlich vielen fortlaufend nummerierten Objekten (beispielsweise Zahlen) bezeichnet. Dasselbe Objekt kann in einer Folge auch mehrfach auftreten. Das Objekt mit der Nummer $i$, man sagt auch: mit dem Index $i$, wird $i$-tes Glied oder $i$-te Komponente der Folge genannt. Endliche wie unendliche Folgen finden sich in allen Bereichen der Mathematik. Mit unendlichen Folgen, deren Glieder Zahlen sind, beschäftigt sich vor allem die \wikiLinkFt{Analysis}.

		Ist $n$ die Anzahl der Glieder einer endlichen Folge, so spricht man von einer Folge der Länge $n$, einer $n$-gliedrigen Folge oder von einem $n$-Tupel. Die Folge ohne Glieder, deren Index-Bereich also leer ist, wird leere Folge, 0-gliedrige Folge oder 0-Tupel genannt.
	}
	Ein \GloFt{Folge}\alternativi{Sequenz} $\vec{a}$ ist eine Aneinanderreihung ihrer \defTxt{\Komponenten} $a_i$, $i \in \MtsINo$, geschrieben $[a_1, a_2, \dots]$.
	Sind alle \Komponenten\ \Elemente\ aus einer \Menge\ $M$, so heißt $\vec{a}$ eine \GloFt{Folge} \DefFt{auf} $M$ oder \DefFt{von} \Elementen\ aus $M$.
	Hat die \GloFt{Folge} nur endlich viele \Komponenten, so heißt sie \DefFt{endlich} und die Anzahl $\MtsLen(\vec{a})$ ihrer \Komponenten\ ihre \DefFt{Länge}.
	Ist die Länge gleich $0$, so sprechen wir von der \defTxt{\leerenFolge} und bezeichnen sie mit \seqqt{$[]$}.
	Eine endliche \GloFt{Folge} der Länge $n$ heißt auch \DefFt{$n$-\Tupel} und die leere \GloFt{Folge} demnach \DefFt{$0$-\Tupel}.
}

%%%%\newVerweis {\endlicheFolge}{\glstext} {endlicheFolge} %%%% Angeblich schon definiert
\newglossaryentry{endlicheFolge}{
	name       =           {---, endliche \addIdx[
		name   =           {---, endliche},
		sort   =         {Folge, endliche}]{endlicheFolge}},
	sort       =         {Folge, endliche},
	text       ={endliche Folge},
%%%	symbol     ={\ensuremath{Mathmode}},% ToDo=Mathmode
%%%	user6      ={Textmode},
	see        ={MtsLen,Folge,Tupel},
	description={\todoPruefen%
%%%		\SymbolAmRand{endlicheFolge}%
		Eine \Folge\ heißt \GloFt{endlich}, wenn ihre Länge endlich ist, \textdh\ wenn sie nur endlich viele \Komponenten\ besitzt.
	}
}

\newVerweis      {\leereFolge} {\glstext} {leereFolge}
\newVerweis[n]   {\leereFolgen}{\glstext} {leereFolge}
\newVerweis     {\leerenFolge} {\glsuseri}{leereFolge}
\newglossaryentry {leereFolge}{
	name       =         {---, leere \addIdx[
		name   =         {---, leere},
		sort   =       {Folge, leere}]    {leereFolge}},
	sort       =       {Folge, leere},
	text       ={leere  Folge},
	user1      ={leeren Folge},
%%%	symbol     ={\ensuremath{Mathmode}},% ToDo=Mathmode
%%%	user6      ={Textmode},
	see        ={MtsLen,Folge,Tupel},
	description={\todoPruefen%
%%%		\SymbolAmRand{leereFolge}%
		Eine \Folge\ heißt \GloFt{leer}, wenn ihre Länge $0$ ist, \textdh\ wenn sie keine \Komponenten\ besitzt.
	}
}

\newVerweis     {\Folgenmenge}{\glstext}{Folgenmenge}
\newglossaryentry{Folgenmenge}{
	name        ={Folgenmenge \addIdx   {Folgenmenge}},
	text        ={Folgenmenge},
%%%	symbol      ={\ensuremath{Mathmode}},% ToDo=Mathmode
%%%	user6       ={Textmode},
	description ={\todoBeschreiben%
%%%		\SymbolAmRand{Folgenmenge}%
	}
}

\newVerweis     {\Folgenoperation}  {\glstext}{Folgenoperation}
\newVerweis[en] {\Folgenoperationen}{\glstext}{Folgenoperation}
\newglossaryentry{Folgenoperation}{
	name        ={Folgenoperation \addIdx     {Folgenoperation}},
	text        ={Folgenoperation},
	description ={\todoBeschreiben%
%%%		\SymbolAmRand{Folgenoperation}%
	}
}

\newVerweis     {\Folgenrelation}  {\glstext}{Folgenrelation}
\newVerweis[en] {\Folgenrelationen}{\glstext}{Folgenrelation}
\newglossaryentry{Folgenrelation}{
	name        ={Folgenrelation \addIdx     {Folgenrelation}},
	text        ={Folgenrelation},
%%%	symbol      ={\ensuremath{Mathmode}},% ToDo=Mathmode
%%%	user6       ={Textmode},
	description ={\todoBeschreiben%
%%%		\SymbolAmRand{Folgenrelation}%
	}
}

\newsynonym    {\Folgerung}            {Folgerung}{\Konklusion}
\newVerweis[en]{\Folgerungen}{\glstext}{Folgerung}

\newsynonym{\Folgerungsmenge}{Folgerungsmenge}{\Konklusionsmenge}

\newVerweis     {\Formationsregel} {\glstext}{Formationsregel}
\newVerweis[n]  {\Formationsregeln}{\glstext}{Formationsregel}
\newglossaryentry{Formationsregel}{
	name        ={Formationsregel \addIdx    {Formationsregel}},
	text        ={Formationsregel},
%%%	symbol      ={\ensuremath{Mathmode}},% ToDo=Mathmode
%%%	user6       ={Textmode},
	description ={\todoBeschreiben%
%%%		\SymbolAmRand{Formationsregel}%
	}
}

\newVerweis     {\Formel} {\glstext}{Formel}
\newVerweis[n]  {\Formeln}{\glstext}{Formel}
\newglossaryentry{Formel}{
	name        ={Formel \addIdx    {Formel}},
	text        ={Formel},
%%%	symbol      ={\ensuremath{Mathmode}},% ToDo=Mathmode
%%%	user6       ={Textmode},
	description ={\todoPruefen%
%%%		\SymbolAmRand{Formel}%
		Unter einer \GloFt{Formel} verstehen wir stets eine mathematische Formel.
		Diese kann aus einem einzigen \Symbol\ bestehen (\atomare\ \gloFt{Formel}), andererseits aber auch mehrdimensional sein, lässt sich dann aber mittels geeigneter Definitionen immer eindeutig als eine \Symbolkette\ schreiben.
		%%% besser: Formel = \Element\ aus einer Sprache?
	}
}

\newVerweis     {\allgemeingueltigeFormel}{\glstext}         {allgemeingueltigeFormel}
\newVerweis    {\allgemeingueltigenFormel}{\glsuseri}        {allgemeingueltigeFormel}
\newglossaryentry{allgemeingueltigeFormel}{
	name       =                     {---, allgemeingültige \addIdx[
		name   =                     {---, allgemeingültige},
		sort   =                  {Formel, allgemeingültige}]{allgemeingueltigeFormel}},
	sort       =                  {Formel, allgemeingültige},
	text       ={allgemeingültige  Formel},
%%%	symbol     ={\ensuremath{Mathmode}},% ToDo=Mathmode
%%%	user6      ={Textmode},
	user1      ={allgemeingültigen Formel},
	description={\todoPruefen%
%%%		\SymbolAmRand{allgemeingueltigeFormel}%
		Eine \Formel\ heißt \GloFt{allgemeingültig}, wenn sie aus den \Axiomen\ und \allgemeingueltigenSchlussregeln\ abgeleitet werden kann.
	}
}

%%%\newVerweis     {\atomareFormel} {\glstext}{atomareFormel}
%%%\newVerweis[n]  {\atomareFormeln}{\glstext}{atomareFormel}
\newglossaryentry{atomareFormel}{
	name       =           {---, atomare \addIdx[
		name   =           {---, atomare},
		sort   =        {Formel, atomare}] {atomareFormel}},
	sort       =        {Formel, atomare},
	text       ={atomare Formel},
%%%	symbol     ={\ensuremath{Mathmode}},% ToDo=Mathmode
%%%	user6      ={Textmode},
	see        ={unzerlegbar,zerlegbar,zusammengesetzt},
	description={\todoGeprueft%
%%%		\SymbolAmRand{atomareFormel}%
		\atomarBeschreibung{Eine \Formel}{sie keine \echteTeilformel}
	}
}

\newVerweis      {\aussagenlogischeFormel} {\glstext}        {aussagenlogischeFormel}
\newVerweis[n]   {\aussagenlogischeFormeln}{\glstext}        {aussagenlogischeFormel}
\newVerweis     {\aussagenlogischenFormel} {\glsuseri}       {aussagenlogischeFormel}
\newVerweis[n]  {\aussagenlogischenFormeln}{\glsuseri}       {aussagenlogischeFormel}
\newVerweis      {\aussagenlogischeF}      {\glsuserii}      {aussagenlogischeFormel}
\newglossaryentry {aussagenlogischeFormel}{
	name       =                     {---, aussagenlogische \addIdx[
		name   =                     {---, aussagenlogische},
		sort   =                  {Formel, aussagenlogische}]{aussagenlogischeFormel}},
	sort       =                  {Formel, aussagenlogische},
	text       ={aussagenlogische  Formel},
	user1      ={aussagenlogischen Formel},
	user2      ={aussagenlogische},
%%%	symbol     ={\ensuremath{Mathmode}},% ToDo=Mathmode
%%%	user6      ={Textmode},
	description={\todoPruefen%
%%%		\SymbolAmRand{aussagenlogischeFormel}%
		Eine \Formel\ heißt \GloFt{aussagenlogisch}, wenn sie ein \Element\ aus \OjkFor\ ist.
	}
}

\newVerweis     {\praedikatenlogischeFormel} {\glstext}           {praedikatenlogischeFormel}
\newVerweis[n]  {\praedikatenlogischeFormeln}{\glstext}           {praedikatenlogischeFormel}
\newglossaryentry{praedikatenlogischeFormel}{
	name       =                       {---, praedikatenlogische \addIdx[
		name   =                       {---, praedikatenlogische},
		sort   =                    {Formel, praedikatenlogische}]{praedikatenlogischeFormel}},
	sort       =                    {Formel, praedikatenlogische},
	text       = {prädikatenlogische Formel},
%%%	symbol     ={\ensuremath{Mathmode}},% ToDo=Mathmode
%%%	user6      ={Textmode},
	description={\todoPruefen% ToDo=\OjkFor durch korrekte Menge ersetzen
%%%		\SymbolAmRand{praedikatenlogischeFormel}%
		Eine \Formel\ heißt \GloFt{prädikatenlogisch}, wenn sie ein \Element\ aus \OjkFor\ ist.
	}
}

\newVerweis     {\Formelmenge} {\glstext}{Formelmenge}
\newVerweis[n]  {\Formelmengen}{\glstext}{Formelmenge}
\newglossaryentry{Formelmenge}{
	name        ={Formelmenge \addIdx    {Formelmenge}},
	text        ={Formelmenge},
%%%	symbol      ={\ensuremath{Mathmode}},% ToDo=Mathmode
%%%	user6       ={Textmode},
	description ={\todoPruefen%
%%%		\SymbolAmRand{Formelmenge}%
		Eine \Menge\ von \Formeln, oft mit \MtsSprache\ bezeichnet.
		Man nennt \MtsSprache\ auch eine \Sprache\ und ihre \Elemente\ \Woerter, insbesondere dann, wenn es eindeutige Regeln zur Konstruktion von \MtsSprache\ gibt.
		Wir bevorzugen „\Formel“ und „\Formelmenge“.
	}
}

\newVerweis      {\MtsFktSep}    {\glsuseri} {Funktion}
\newVerweis      {\MtsFktArrow}  {\glsuserii}{Funktion}
\newVerweis         {\Funktion}  {\glstext}  {Funktion}
\newVerweis[en]     {\Funktionen}{\glstext}  {Funktion}
\longnewglossaryentry{Funktion}{
	name            ={Funktion \addIdx       {Funktion}},
	text            ={Funktion},
	user1           ={:},
	user2           ={\ensuremath{\RawMtsFktArrow}},
%%%	symbol          ={\ensuremath{Mathmode}},% ToDo=Mathmode
%%%	user6           ={Textmode},
	see             ={Abbildung,Element,Menge,Objekt,Relation},
}{\todoPruefen%
%%%	\SymbolAmRand{Funktion}%
	\wikicite{bib:Funktion}{
		In der \wikiLinkFt{Mathematik} ist eine \wikiBoldFt{Funktion} (lateinisch \wikiItalicFt{functio}) oder \wikiBoldFt{Abbildung} eine Beziehung (\wikiLinkFt{Relation}) zwischen zwei \wikiLinkFt{Mengen}, die jedem Element der einen Menge (Funktionsargument, unabhängige Variable, $x$-Wert) genau ein Element der anderen Menge (Funktionswert, abhängige Variable, $y$-Wert) zuordnet. Der Funktionsbegriff wird in der Literatur unterschiedlich definiert, jedoch geht man generell von der Vorstellung aus, dass Funktionen \wikiLinkFt{mathematischen Objekten} mathematische Objekte zuordnen, zum Beispiel jeder reellen Zahl deren Quadrat.  Das Konzept der Funktion oder Abbildung nimmt in der modernen Mathematik eine zentrale Stellung ein; es enthält als Spezialfälle unter anderem \wikiLinkFt{parametrische Kurven}, Skalar- und \wikiLinkFt{Vektorfelder}, \wikiLinkFt{Transformationen}, \wikiLinkFt{Operationen}, \wikiLinkFt{Operatoren} und vieles mehr.
	}
	Eine \GloFt{$n$-\stellige\ Funktion} $f$ von einer \Menge\ $A = A_1 \MtsTimes \dots \MtsTimes A_n$, dem \Definitionsbereich, in eine \Menge\ $B$, den \Zielbereich, ist eine ($n$+1)-\stellige\ \Relation\ $(G,A_1,\dots,A_n,B)$ derart, dass es für jedes $\vec{a} = (a_1,\dots,a_n)$ mit $a_i \in A_i$ genau ein $b \in B$ gibt mit $(a_1,\dots,a_n,b) \in f$.
	Dieses $b$ wird auch mit \seqqt{$f(a_1,\dots,a_n)$} , \seqqt{$f a_1 \dots a_n$} , \seqqt{$f(\vec{a})$} oder \seqqt{$f\vec{a}$} bezeichnet.
	\\Schreibweise: \seqqt{\FunktionDef{f}{A}{B}} \textbzw\ \seqqt{$\FunktionDef{f}{A_1 \MtsTimes \dots \MtsTimes A_n}{B}$}
}

\newVerweis     {\Funktionssymbol}  {\glstext}{Funktionssymbol}
\newVerweis[e]  {\Funktionssymbole} {\glstext}{Funktionssymbol}
\newVerweis[en] {\Funktionssymbolen}{\glstext}{Funktionssymbol}
\newglossaryentry{Funktionssymbol}{
	name        ={Funktionssymbol \addIdx     {Funktionssymbol}},
	text        ={Funktionssymbol},
%%%	symbol      ={\ensuremath{Mathmode}},% ToDo=Mathmode
%%%	user6       ={Textmode},
	description ={\todoPruefen%
%%%		\SymbolAmRand{Funktionssymbol}%
		Ein \Symbol\ für eine \Funktion.
	}
}

\newVerweis     {\Funktionswert} {\glstext}{Funktionswert}
\newVerweis[e]  {\Funktionswerte}{\glstext}{Funktionswert}
\newglossaryentry{Funktionswert}{
	name        ={Funktionswert \addIdx    {Funktionswert}},
	text        ={Funktionswert},
%%%	symbol      ={\ensuremath{Mathmode}},% ToDo=Mathmode
%%%	user6       ={Textmode},
	description ={\todoPruefen%
%%%		\SymbolAmRand{Funktionswert}%
		einer \Funktion.
	}
}

%G === G === G === G === G === G === G === G === G === G === G === G === G === G

\newVerweis     {\Gleichheit}{\glstext}{Gleichheit}
\newglossaryentry{Gleichheit}{
	name        ={Gleichheit \addIdx   {Gleichheit}},
	text        ={Gleichheit},
%%%	symbol      ={\ensuremath{Mathmode}},% ToDo=Mathmode
%%%	user6       ={Textmode},
	description ={\todoPruefen%
%%%		\SymbolAmRand{Gleichheit}%
		Eine \Gleichheitsrelation:
		Zwei Objekte $A$ und $B$ sind \DefFt{gleich} (dasselbe; identisch), $A \MtsEq B$, wenn sie in den \interessierendenEigenschaften\ für \MtsEq\ übereinstimmen.
	}
}

\newVerweis     {\Gleichheitsrelation}  {\glstext}{Gleichheitsrelation}
\newVerweis[en] {\Gleichheitsrelationen}{\glstext}{Gleichheitsrelation}
\newglossaryentry{Gleichheitsrelation}{
	name        ={Gleichheitsrelation \addIdx     {Gleichheitsrelation}},
	text        ={Gleichheitsrelation},
%%%	symbol      ={\ensuremath{Mathmode}},% ToDo=Mathmode
%%%	user6       ={Textmode},
	description ={\todoPruefen%
%%%		\SymbolAmRand{Gleichheitsrelation}%
%%%		Eine mit \Gleichheit\ verwandte \Relation: \MtsEq, \MtsEqN, \MtsAequiv\ und \MtsAequivN.
		Eine mit \Gleichheit\ verwandte \Relation: \MtsEq\ und \MtsEqN.
	}
}

%%%\newVerweis[er]{\Glieder}{\glstext}{Glied}
%%%\newsynonym    {\Glied}{Glied}{\Kettenglied}

\newVerweis     {\Gliederungszeichen}{\glstext}{Gliederungszeichen}
\newglossaryentry{Gliederungszeichen}{
	name        ={Gliederungszeichen \addIdx   {Gliederungszeichen}},
	text        ={Gliederungszeichen},
%%%	symbol      ={\ensuremath{Mathmode}},% ToDo=Mathmode
%%%	user6       ={Textmode},
	description ={\todoBeschreiben%
%%%		\SymbolAmRand{Gliederungszeichen}%
	}
}

\newVerweis     {\Graph}  {\glstext}{Graph}
\newVerweis[en] {\Graphen}{\glstext}{Graph}
\newglossaryentry{Graph}{
	name        ={Graph \addIdx     {Graph}},
	text        ={Graph},
%%%	symbol      ={\ensuremath{Mathmode}},% ToDo=Mathmode
%%%	user6       ={Textmode},
	see      ={MtsGraph},
	description ={\todoPruefen%
%%%		\SymbolAmRand{Graph}%
		einer \Funktion\ oder \Relation.
	}
}

%I === I === I === I === I === I === I === I === I === I === I === I === I === I

\newVerweis     {\Identitaetsregel} {\glstext}{Identitaetsregel}
\newVerweis[n]  {\Identitaetsregeln}{\glstext}{Identitaetsregel}
\newglossaryentry{Identitaetsregel}{
	name        ={Identitätsregel \addIdx[
		name    ={Identitätsregel}]           {Identitaetsregel}},
	text        ={Identitätsregel},
%%%	symbol      ={\ensuremath{Mathmode}},% ToDo=Mathmode
%%%	user6       ={Textmode},
	description ={\todoPruefen%
%%%		\SymbolAmRand{Identitaetsregel}%
		Eigentlich eine \Basisregel\ zur Identität.
		Da die \Identitaetsregeln\ nur zur Rechtfertigung der \Ersetzung\ verwendet werden, werden sie \hier\ nicht zu den \Basisregeln\ gezählt.
	}
}

%J === J === J === J === J === J === J === J === J === J === J === J === J === J

\newVerweis         {\Junktor}  {\glstext}{Junktor}
\newVerweis[en]     {\Junktoren}{\glstext}{Junktor}
\longnewglossaryentry{Junktor}{
	name            ={Junktor \addIdx     {Junktor}},
	text            ={Junktor},
%%%	symbol          ={\ensuremath{Mathmode}},% ToDo=Mathmode
%%%	user6           ={Textmode},
	see             ={Metajunktor},
}{\todoPruefen%
%%%	\SymbolAmRand{Junktor}%
	\wikicite{bib:Junktor}{
		Ein \wikiBoldFt{Junktor} (von \wikiLinkFt{lat.} \wikiItalicFt{iungere} „verknüpfen, verbinden“) ist eine \wikiLinkFt{logische Verknüpfung} zwischen Aussagen innerhalb der \wikiLinkFt{Aussagenlogik}, also ein logischer \wikiLinkFt{Operator}. Junktoren werden auch Konnektive, Konnektoren, Satzoperatoren, Satzverknüpfer, Satzverknüpfungen, Aussagenverknüpfer, logische Bindewörter, Verknüpfungszeichen oder Funktoren genannt und als \wikiLinkFt{logische Partikel} klassifiziert.

		Sprachlich wird zwischen der jeweiligen Verknüpfung selbst (zum Beispiel der \wikiLinkFt{Konjunktion}) und dem sie bezeichnenden Wort beziehungsweise Sprachzeichen (zum Beispiel dem Wort „und“ beziehungsweise dem Zeichen „\OjkAnd“) oft nicht unterschieden.
	}
	Ein \GloFt{Junktor} ist eine \aussagenlogischeOperation\ oder -\aRelation.
	Da die Werte einer aussagenlogischen \Operation\ \Wahrheitswerte\ sind, kann man einen \Junktor\ auch stets als \Relation\ verstehen.
}

\newVerweis     {\binaererJunktor}  {\glstext} {binaererJunktor}
\newVerweis[en] {\binaerenJunktoren}{\glsuseri}{binaererJunktor}
\newglossaryentry{binaererJunktor}{
	name        =            {---, binärer \addIdx[
		name    =            {---, binärer},
		sort    =        {Junktor, binärer}]   {binaererJunktor}},
	sort        =        {Junktor, binärer},
	text        ={binärer Junktor},
%%%	symbol      ={\ensuremath{Mathmode}},% ToDo=Mathmode
%%%	user6       ={Textmode},
	user1       ={binären Junktor},
	description ={\todoBeschreiben%
%%%		\SymbolAmRand{binaererJunktor}%
	}
}

\newVerweis     {\unaererJunktor}  {\glstext} {unaererJunktor}
\newVerweis[en] {\unaerenJunktoren}{\glsuseri}{unaererJunktor}
\newglossaryentry{unaererJunktor}{
	name        =           {---, unärer \addIdx[
		name    =           {---, unärer},
		sort    =       {Junktor, unärer}]    {unaererJunktor}},
	sort        =       {Junktor, unärer},
	text        ={unärer Junktor},
%%%	symbol      ={\ensuremath{Mathmode}},% ToDo=Mathmode
%%%	user6       ={Textmode},
	user1       ={unären Junktor},
	description ={\todoBeschreiben%
%%%		\SymbolAmRand{unaererJunktor}%
	}
}

\newVerweis     {\Junktorsymbol} {\glstext}{Junktorsymbol}
\newVerweis[e]  {\Junktorsymbole}{\glstext}{Junktorsymbol}
\newglossaryentry{Junktorsymbol}{
	name        ={Junktorsymbol \addIdx    {Junktorsymbol}},
	text        ={Junktorsymbol},
%%%	symbol      ={\ensuremath{Mathmode}},% ToDo=Mathmode
%%%	user6       ={Textmode},
	description ={\todoPruefen%
%%%		\SymbolAmRand{Junktorsymbol}%
		Ein \Symbol\ für einen \Junktor.
	}
}

%K === K === K === K === K === K === K === K === K === K === K === K === K === K

\newVerweis         {\Kalkuel}{\glstext}{Kalkuel}
\longnewglossaryentry{Kalkuel}{
	name            ={Kalkuel \addIdx   {Kalkuel}},
	text            ={Kalkül},
%%%	symbol          ={\ensuremath{Mathmode}},% ToDo=Mathmode
%%%	user6           ={Textmode},
	see             ={Axiom,Logik},
}{\todoErgaenzen%
%%%	\SymbolAmRand{Kalkuel}%
	\wikicite{bib:Kalkuel}{
		Als der oder das \wikiBoldFt{Kalkül} (französisch \wikiItalicFt{calcul} „Rechnung“; von \wikiLinkFt{lateinisch} \wikiItalicFt{calculus} „\wikiLinkFt{Rechenstein}“, „\wikiLinkFt{Spielstein}“) versteht man in den formalen Wissenschaften wie \wikiLinkFt{Logik} und \wikiLinkFt{Mathematik} ein System von Regeln, mit denen sich aus gegebenen Aussagen (\wikiLinkFt{Axiomen}) weitere Aussagen ableiten lassen. Kalküle, auf eine Logik selbst angewandt, werden auch Logikkalküle genannt.
	}
}

\newVerweis     {\Kette} {\glstext}{Kette}
\newVerweis[n]  {\Ketten}{\glstext}{Kette}
\newglossaryentry{Kette}{
	name        ={Kette \addIdx    {Kette}},
	text        ={Kette},
%%%	symbol      ={\ensuremath{Mathmode}},% ToDo=Mathmode
%%%	user6       ={Textmode},
	see         ={Symbolkette,Zeichenkette},
	description ={\todoOk%
%%%		\SymbolAmRand{Kette}%
		Eine \GloFt{Kette} ist eine lineare Aneinanderreihung von endlich vielen, nicht notwendig verschiedenen \defGlo{\Kettengliedern}.
		In der Darstellung werden die \Kettenglieder\ \hier\ von links (Anfang) nach rechts (Ende) geschrieben.
		Wenn sie kein \Kettenglied\ hat, ist sie \leerK, ansonsten \DefFt{nicht leer}.
		Sie hat folgende Eigenschaften:
		\begin{enumerate}
			\item Wenn nichts anderes gesagt wird, hat sie mindestens ein \Kettenglied.
			\item Jedes \Kettenglied\ hat höchstens einen \defGlo{\Vorgaenger} (Richtung Anfang) und höchstens einen \defGlo{\Nachfolger} (Richtung Ende).
			\item Eine nicht leere \Kette\ hat genau ein \defGlo{\AnfangsG} und ein \defGlo{\Endglied}.
			\item \AnfangsG\ und \Endglied\ dürfen übereinstimmen.
			Die \gloFt{Kette} hat dann nur ein \Kettenglied.
			\item Ein \Anfangsglied\ hat keinen \Vorgaenger\ und höchstens einen \Nachfolger.
			\item Ein \Endglied\     hat keinen \Nachfolger\ und höchstens einen \Vorgaenger.
			\item \defGlo{\Zwischenglieder} sind die \Kettenglieder, die weder \AnfangsG\ noch \Endglied\ sind.
			\item Jedes \Zwischenglied\ hat genau einen \Vorgaenger\ und einen \Nachfolger.
			\item Zwei \Ketten\ sind gleich, wenn sie die gleichen \Kettenglieder\ in der gleichen Reihenfolge besitzen.
		\end{enumerate}
		Die \Kette\ ohne \Kettenglieder\ ist die \defGlo{\leereKette}.
		Wegen 9 gibt es genau eine \leereKette.
	}
}

\newVerweis     {\leereKette}{\glstext} {leereKette}
\newVerweis     {\leerK}     {\glsuseri}{leereKette}
\newglossaryentry{leereKette}{
	name       =        {---, leere \addIdx[
		name   =        {---, leere},
		sort   =      {Kette, leere}]   {leereKette}},
	sort       =      {Kette, leere},
	text       ={leere Kette},
	user1      ={leer},
%%%	symbol     ={\ensuremath{Mathmode}},% ToDo=Mathmode
%%%	user6      ={Textmode},
	description={\todoOk%
%%%		\SymbolAmRand{leereKette}%
		Eine \GloFt{leere Kette} ist eine \Kette\ ohne \Kettenglieder.
		Es gibt genau eine \gloFt{leere Kette}.
	}
}

\newVerweis     {\Kettenglied}   {\glstext}{Kettenglied}
\newVerweis[er] {\Kettenglieder} {\glstext}{Kettenglied}
\newVerweis[ern]{\Kettengliedern}{\glstext}{Kettenglied}
\newVerweis[s]  {\Kettenglieds}  {\glstext}{Kettenglied}
\newglossaryentry{Kettenglied}{
	name        ={Kettenglied \addIdx      {Kettenglied}},
	text        ={Kettenglied},
%%%	symbol      ={\ensuremath{Mathmode}},% ToDo=Mathmode
%%%	user6       ={Textmode},
	description ={\todoOk%
%%%		\SymbolAmRand{Kettenglied}%
		Eine \GloFt{Kettenglied}\synonym{Glied} ist Teil einer \Kette\ und kann jedes nicht weiter \zerlegbare\ \Element\ sein.
	}
}

\newVerweis     {\Klammerung}{\glstext}{Klammerung}
\newglossaryentry{Klammerung}{
	name        ={Klammerung \addIdx   {Klammerung}},
	text        ={Klammerung},
%%%	symbol      ={\ensuremath{Mathmode}},% ToDo=Mathmode
%%%	user6       ={Textmode},
	description ={\todoBeschreiben%
%%%		\SymbolAmRand{Klammerung}%
	}
}

\newVerweis         {\Klasse} {\glstext}{Klasse}
\newVerweis[n]      {\Klassen}{\glstext}{Klasse}
\longnewglossaryentry{Klasse}{
	name            ={Klasse \addIdx    {Klasse}},
	text            ={Klasse},
%%%	symbol          ={\ensuremath{Mathmode}},% ToDo=Mathmode
%%%	user6           ={Textmode},
	see             ={Menge,Mengenlehre}
}{\todoOk%
%%%	\SymbolAmRand{Klasse}%
	\wikicite{bib:Klasse}{
		Als \wikiBoldFt{Klasse} gilt in der \wikiLinkFt{Mathematik}, \wikiLinkFt{Klassenlogik} und \wikiLinkFt{Mengenlehre} eine Zusammenfassung beliebiger Objekte, definiert durch eine logische Eigenschaft, die alle Objekte der Klasse erfüllen. Vom Klassenbegriff ist der Mengenbegriff zu unterscheiden. Nicht alle Klassen sind automatisch auch Mengen, weil Mengen zusätzliche Bedingungen erfüllen müssen. Mengen sind aber stets Klassen und werden daher auch in der Praxis in Klassenschreibweise notiert.
	}
	Eine \GloFt{Klasse} ist eine \Bereich, deren \Elemente\ genau die \Objekte\ mit einer bestimmten \Eigenschaft\ sind.

	Schreibweise: \MengeDef{x}{Eigenschaft(x)}
	-- Jede \Menge\ ist auch eine \gloFt{Klasse} und jede \gloFt{Klasse} ein \Bereich.
}

\newVerweis     {\leereKlasse}{\glstext}{leereKlasse}
\newglossaryentry{leereKlasse}{
	name        =         {---, leere \addIdx[
		name    =         {---, leere},
		sort    =      {Klasse, leere}] {leereKlasse}},
	sort        =      {Klasse, leere},
	text        ={leere Klasse},
%%%	symbol      ={\ensuremath{Mathmode}},% ToDo=Mathmode
%%%	user6       ={Textmode},
	description ={\todoPruefen%
%%%		\SymbolAmRand{leereKlasse}%
		\MtsEmptyset, die \leereMenge, ist auch die einzige \Klasse\ ohne \Elemente.
	}
}

\newVerweis         {\Klassenlogik} {\glstext}{Klassenlogik}
\longnewglossaryentry{Klassenlogik}{
	name            ={Klassenlogik \addIdx    {Klassenlogik}},
	text            ={Klassenlogik},
%%%	symbol          ={\ensuremath{Mathmode}},% ToDo=Mathmode
%%%	user6           ={Textmode},
	see             ={Klasse,Logik},
}{\todoErgaenzen%
%%%	\SymbolAmRand{Klassenlogik}%
	\wikicite{bib:Klassenlogik}{
		Die \wikiBoldFt{Klassenlogik} ist im weiteren Sinn eine \wikiLinkFt{Logik}, deren Objekte als Klassen bezeichnet werden. Im engeren Sinn spricht man von einer Klassenlogik nur dann, wenn \wikiLinkFt{Klassen} durch eine Eigenschaft ihrer Elemente beschrieben werden. Diese Klassenlogik ist daher eine Verallgemeinerung der \wikiLinkFt{Mengenlehre}, die nur eine eingeschränkte Klassenbildung erlaubt.
	}
}

\newVerweis     {\Komponente} {\glstext}{Komponente}
\newVerweis[n]  {\Komponenten}{\glstext}{Komponente}
\newglossaryentry{Komponente}{
	name        ={Komponente \addIdx    {Komponente}},
	text        ={Komponente},
%%%	symbol      ={\ensuremath{Mathmode}},% ToDo=Mathmode
%%%	user6       ={Textmode},
	see         ={Folge,Tupel},
	description ={\todoPruefen%
%%%		\SymbolAmRand{Komponente}%
		Die \Komponenten\ einer \Folge\ $\vec{a} = (a_1, a_2, \dots)$ sind die $a_i$.
		$a_i$ heißt die \GloFt{$i$-te \Komponente} von $\vec{a}$.
	}
}

\newVerweis     {\Komponentenmenge}  {\glstext}{Komponentenmenge}
\newglossaryentry{Komponentenmenge}{
	name        ={Komponentenmenge \addIdx     {Komponentenmenge}},
	text        ={Komponentenmenge},
%%%	symbol      ={\ensuremath{Mathmode}},% ToDo=Mathmode
%%%	user6       ={Textmode},
	see         ={Menge},
	description ={\todoPruefen%
%%%		\SymbolAmRand{Komponentenmenge}%
		$\MtsSet(\vec{a}) \MtsDefEq \RawMengeDef{a}{a \MtsSeqIn \vec{a}}$ ist die \GloFt{Komponentenmenge} einer \Folge\ \textbzw\ eines \Tupels\ $\vec{a}$.
	}
}

\newVerweis     {\Komponentenrelation}  {\glstext}{Komponentenrelation}
\newVerweis[en] {\Komponentenrelationen}{\glstext}{Komponentenrelation}
\newglossaryentry{Komponentenrelation}{
	name        ={Komponentenrelation \addIdx     {Komponentenrelation}},
	text        ={Komponentenrelation},
%%%	symbol      ={\ensuremath{Mathmode}},% ToDo=Mathmode
%%%	user6       ={Textmode},
	see         ={Elementrelation},
	description ={\todoPruefen% ToDo Was heißt (möglichen)?
%%%		\SymbolAmRand{Komponentenrelation}%
		Eine \GloFt{Komponentenrelation} ist eine Relation zwischen einer (möglichen) \Komponente\ und einer \Folge: \MtsSeqIn, \MtsSeqNi, \MtsSeqInN\ und \MtsSeqNiN
	}
}

\newVerweis     {\Konklusion}  {\glstext}{Konklusion}
\newVerweis[en] {\Konklusionen}{\glstext}{Konklusion}
\newglossaryentry{Konklusion}{
	name        ={Konklusion \addIdx     {Konklusion}},
	text        ={Konklusion},
%%%	symbol      ={\ensuremath{Mathmode}},% ToDo=Mathmode
%%%	user6       ={Textmode},
	see         ={Schlussregel},
	description ={\todoPruefen%
%%%		\SymbolAmRand{Konklusion}%
		Eine \Ableitung:
		Die \Konklusionen\ einer \Schlussregel\ $\frac{\MtsPraemisseSet}{\MtsKonklusionSet}$ \textbzw\ $\frac{\MtsPraemisseSet}{\MtsKonklusionSet}$ sind die \Elemente\ aus \MtsKonklusionSet\ \textbzw\ \MtsKonklusionRel.
		Die \Konklusionen\ werden normalerweise mit $\MtsKonklusion_i$ bezeichnet.
	}
}

\newVerweis     {\Konklusionsmenge} {\glstext}{Konklusionsmenge}
\newVerweis[n]  {\Konklusionsmengen}{\glstext}{Konklusionsmenge}
\newglossaryentry{Konklusionsmenge}{
	name        ={Konklusionsmenge \addIdx    {Konklusionsmenge}},
	text        ={Konklusionsmenge},
%%%	symbol      ={\ensuremath{Mathmode}},% ToDo=Mathmode
%%%	user6       ={Textmode},
	description ={\todoPruefen%
%%%		\SymbolAmRand{Konklusionsmenge}%
		Eine \Ableitungsmenge:
		Die \Menge\ \MtsKonklusionSet\ der \Konklusionen\ einer \Schlussregel\ \textbzw\ eines \Beweises.
	}
}

\newVerweis         {\Konstante} {\glstext}{Konstante}
\newVerweis[n]     {\Konstanten}{\glstext}{Konstante}
\longnewglossaryentry{Konstante}{
	name            ={Konstante \addIdx    {Konstante}},
	text            ={Konstante},
%%%	symbol          ={\ensuremath{Mathmode}},% ToDo=Mathmode
%%%	user6           ={Textmode},
	see             ={Symbol,Variable},
}{\todoPruefen%
%%%	\SymbolAmRand{Konstante}%
	\wikicite{bib:Konstante}{
		Allgemein ist eine \wikiBoldFt{Konstante} (von \wikiLinkFt{lateinisch} \wikiItalicFt{constans} „feststehend“) ein Zeichen beziehungsweise ein Sprachausdruck mit einer „genau bestimmte[n]Bedeutung, die im Laufe der Überlegungen unverändert bleibt“. Die Konstante ist damit ein Gegenbegriff zur \wikiLinkFt{Variablen}.
	}
}

\newVerweis      {\aussagenlogischeKonstante} {\glstext}        {aussagenlogischeKonstante}
\newVerweis     {\aussagenlogischenKonstante} {\glsuseri}       {aussagenlogischeKonstante}
\newVerweis[n]  {\aussagenlogischenKonstanten}{\glsuseri}       {aussagenlogischeKonstante}
\newglossaryentry {aussagenlogischeKonstante}{
	name       =                        {---, aussagenlogische \addIdx[
		name   =                        {---, aussagenlogische},
		sort   =                  {Konstante, aussagenlogische}]{aussagenlogischeKonstante}},
	sort       =                  {Konstante, aussagenlogische},
	text       ={aussagenlogische  Konstante},
%%%	symbol     ={\ensuremath{Mathmode}},% ToDo=Mathmode
%%%	user6      ={Textmode},
	user1      ={aussagenlogischen Konstante},
	description={\todoPruefen%
%%%		\SymbolAmRand{aussagenlogischeKonstante}%
		Eine \Konstante\ heißt \GloFt{aussagenlogisch}, wenn sie ein \Element\ aus \OjkCon\ ist.
	}
}

\newVerweis     {\Kontraposition}{\glstext}{Kontraposition}
\newglossaryentry{Kontraposition}{
	name        ={Kontraposition \addIdx   {Kontraposition}},
	text        ={Kontraposition},
%%%	symbol      ={\ensuremath{Mathmode}},% ToDo=Mathmode
%%%	user6       ={Textmode},
	description ={\todoPruefen%
%%%		\SymbolAmRand{Kontraposition}%
		Die allgemeingültige \Aussage: $ (\alpha \OjkImp \beta) \OjkImp (\OjkNot\beta \OjkImp \OjkNot\alpha) $.
	}
}

%%%\newVerweis     {\Kontravalenz}{\glstext}{Kontravalenz}
%%%\newglossaryentry{Kontravalenz}{
%%%	name        ={Kontravalenz \addIdx   {Kontravalenz}},
%%%	text        ={Kontravalenz},
%%%	symbol      ={\ensuremath{Mathmode}},% ToDo=Mathmode
%%%	user6       ={Textmode},
%%%	description ={\todoPruefen%
%%%		\SymbolAmRand{Kontravalenz}%
%%%		Eine \Gleichheitsrelation:
%%%		Zwei Objekte $A$ und $B$ sind \DefFt{nicht äquivalent} (nicht ähnlich), $A \MtsAequivN B$, wenn sie in mindestens einer \interessierendenEigenschaft\ für \MtsAequiv\ nicht übereinstimmen.
%%%	}
%%%}

%L === L === L === L === L === L === L === L === L === L === L === L === L === L

\newVerweis         {\Logik}  {\glstext}{Logik}
\newVerweis[en]     {\Logiken}{\glstext}{Logik}
\longnewglossaryentry{Logik}{
	name            ={Logik \addIdx     {Logik}},
	text            ={Logik},
%%%	symbol          ={\ensuremath{Mathmode}},% ToDo=Mathmode
%%%	user6           ={Textmode},
	see             ={atomar,Aussage,Aussagenlogik,Praedikatenlogik,Schlussregel},
}{\todoGeprueft%
%%%	\SymbolAmRand{Logik}%
	\wikicite{bib:Logik}{
		Mit \wikiBoldFt{Logik} (von \wikiLinkFt{altgriechisch} [\textdots]‚denkende Kunst‘, ‚Vorgehensweise‘) oder auch \wikiBoldFt{Folgerichtigkeit} wird im Allgemeinen das \wikiLinkFt{vernünftige Schlussfolgern} und im Besonderen dessen Lehre – die \wikiBoldFt{Schlussfolgerungslehre} oder auch \wikiBoldFt{Denklehre} – bezeichnet. In der Logik wird die Struktur von \wikiLinkFt{Argumenten} im Hinblick auf ihre \wikiLinkFt{Gültigkeit} untersucht, unabhängig vom Inhalt der \wikiLinkFt{Aussagen}. Bereits in diesem Sinne spricht man auch von „formaler“ Logik. Traditionell ist die Logik ein Teil der \wikiLinkFt{Philosophie}. Ursprünglich hat sich die traditionelle Logik in Nachbarschaft zur \wikiLinkFt{Rhetorik} entwickelt. Seit dem 20. Jahrhundert versteht man unter Logik überwiegend {symbolische Logik}, die auch als grundlegende \wikiLinkFt{Strukturwissenschaft}, z. B. innerhalb der \wikiLinkFt{Mathematik} und der \wikiLinkFt{theoretischen Informatik}, behandelt wird.

		Die moderne symbolische Logik verwendet statt der \wikiLinkFt{natürlichen Sprache} eine \wikiLinkFt{künstliche Sprache} (Ein Satz wie \wikiItalicFt{Der Apfel ist rot} wird z. B. in der \wikiLinkFt{Prädikatenlogik} als $f(a)$ formalisiert, wobei $a$ für \wikiItalicFt{Der Apfel} und $f$ für \wikiItalicFt{ist rot} steht) und verwendet streng \wikiLinkFt{definierte Schlussregeln}. Ein einfaches Beispiel für ein solches \wikiLinkFt{formales System} ist die \wikiLinkFt{Aussagenlogik} (dabei werden sogenannte \wikiLinkFt{atomare Aussagen} durch Buchstaben ersetzt). Die symbolische Logik nennt man auch \wikiLinkFt{mathematische Logik} oder formale Logik im engeren Sinn.
	}
}

\newVerweis         {\mathematischeLogik}{\glstext}       {mathematischeLogik}
\longnewglossaryentry{mathematischeLogik}{
	name            =                {---, mathematische \addIdx[
		name        =                {---, mathematische},
		sort        =              {Logik, mathematische}]{mathematischeLogik}},
	sort            =              {Logik, mathematische},
%%%	symbol          ={\ensuremath{Mathmode}},% ToDo=Mathmode
%%%	user6           ={Textmode},
	text            ={mathematische Logik},
	see             ={Mengenlehre,Fachgebiet},
}{\todoGeprueft%
%%%	\SymbolAmRand{mathematischeLogik}%
	\wikicite{bib:mathematischeLogik}{
		Die \wikiBoldFt{mathematische Logik}, auch \wikiBoldFt{symbolische Logik}, (alternativer Sprachgebrauch auch \wikiItalicFt{Logistik}), ist ein Teilgebiet der \wikiLinkFt{Mathematik}, insbesondere als Methode der \wikiLinkFt{Metamathematik} und eine Anwendung der modernen \wikiLinkFt{formalen Logik}. Oft wird sie wiederum in die Teilgebiete \wikiLinkFt{Modelltheorie}, \wikiLinkFt{Beweistheorie}, \wikiLinkFt{Mengenlehre} und \wikiLinkFt{Rekursionstheorie} aufgeteilt. Forschung im Bereich der mathematischen Logik hat zum Studium der \wikiLinkFt{Grundlagen der Mathematik} beigetragen und wurde auch durch dieses motiviert. Infolgedessen wurde sie auch unter dem Begriff \wikiItalicFt{Metamathematik} bekannt.

		Ein Aspekt der Untersuchungen der mathematischen Logik ist das Studium der Ausdrucksstärke von formalen Logiken und formalen \wikiLinkFt{Beweissystemen}. Eine Möglichkeit, die \wikiLinkFt{Komplexität} solcher Systeme zu messen, besteht darin, festzustellen, was damit bewiesen oder definiert werden kann.

		Früher wurde die mathematische Logik auch \wikiItalicFt{symbolische Logik} (als Gegensatz zur \wikiLinkFt{philosophischen Logik}) genannt, wobei jener Name mittlerweile nur noch für gewisse Aspekte der \wikiLinkFt{Beweistheorie} verwendet wird.
	}
}

%M === M === M === M === M === M === M === M === M === M === M === M === M === M

\newVerweis      {\MtsSetSep}{\glsuseri}{Menge}
\newVerweis         {\Menge} {\glstext} {Menge}
\newVerweis[n]      {\Mengen}{\glstext} {Menge}
\longnewglossaryentry{Menge}{
	name            ={Menge \addIdx     {Menge}},
	text            ={Menge},
	user1           ={\ensuremath{\RawMtsSetSep}},
%%%	symbol          ={\ensuremath{Mathmode}},% ToDo=Mathmode
%%%	user6           ={Textmode},
	see             ={Bereich,Element,Folge,leereMenge,Mengenlehre,Tupel},
}{\todoGeprueft%
%%%	\SymbolAmRand{Menge}%
	\wikicite{bib:Menge}{
		Eine \wikiBoldFt{Menge} ist ein Verbund, eine Zusammenfassung von einzelnen \wikiLinkFt{Elementen}. Die \wikiItalicFt{Menge} ist eines der wichtigsten und grundlegenden Konzepte der Mathematik, mit ihrer Betrachtung beschäftigt sich die \wikiLinkFt{Mengenlehre}.

		Bei der Beschreibung einer Menge geht es ausschließlich um die Frage, welche Elemente in ihr enthalten sind. Es wird nicht danach gefragt, ob ein Element mehrmals enthalten ist oder ob es eine Reihenfolge unter den Elementen gibt. Eine Menge muss kein Element enthalten – es gibt genau eine Menge ohne Elemente, die „\wikiLinkFt{leere Menge}“. In der Mathematik sind die Elemente einer Menge häufig Zahlen, Punkte eines \wikiLinkFt{Raumes} oder ihrerseits Mengen. Das Konzept ist jedoch auf beliebige Objekte anwendbar: z. B. in der \wikiLinkFt{Statistik} auf Stichproben, in der Medizin auf Patientenakten, am Marktstand auf eine Tüte mit Früchten.

		Ist die Reihenfolge der Elemente von Bedeutung, dann spricht man von einer endlichen oder unendlichen \wikiLinkFt{Folge}, wenn sich die Folgenglieder mit den natürlichen Zahlen aufzählen lassen (das erste, das zweite, usw.). Endliche Folgen heißen auch \wikiLinkFt{Tupel}. In einem Tupel oder einer Folge können Elemente auch mehrfach vorkommen. Ein Gebilde, das wie eine Menge Elemente enthält, wobei es zusätzlich auf die Anzahl der Exemplare jedes Elements ankommt, jedoch nicht auf die Reihenfolge, heißt \wikiLinkFt{Multimenge}.
	}
	Eine \GloFt{Menge} ist eine \Klasse\ mit zusätzlichen Eigenschaften.
}

\newVerweis     {\leereMenge}{\glstext}{leereMenge}
\newglossaryentry{leereMenge}{
	name       =        {---, leere \addIdx[
		name   =        {---, leere},
		sort   =      {Menge, leere}]  {leereMenge}},
	sort       =      {Menge, leere},
	text       ={leere Menge},
%%%	symbol     ={\ensuremath{Mathmode}},% ToDo=Mathmode
%%%	user6      ={Textmode},
	description={\todoPruefen%
%%%		\SymbolAmRand{leereMenge}%
		\MtsEmptyset, die \GloFt{leere Menge}, ist die einzige \Menge\ ohne \Elemente.
		Sie wird auch mit \seqqt{$\{\}$} bezeichnet.
	}
}

\newVerweis         {\Mengenlehre}{\glstext}{Mengenlehre}
\longnewglossaryentry{Mengenlehre}{
	name            ={Mengenlehre \addIdx   {Mengenlehre}},
	text            ={Mengenlehre},
%%%	symbol          ={\ensuremath{Mathmode}},% ToDo=Mathmode
%%%	user6           ={Textmode},
	see             ={Axiom,Fachgebiet,Menge,Objekt},
}{\todoPruefen%
%%%	\SymbolAmRand{Mengenlehre}%
	\wikicite{bib:Mengenlehre}{
		Die \wikiBoldFt{Mengenlehre} ist ein grundlegendes \wikiLinkFt{Teilgebiet der Mathematik}, das sich mit der Untersuchung von \wikiLinkFt{Mengen}, also von Zusammenfassungen von \wikiLinkFt{Objekten}, beschäftigt. Die gesamte Mathematik, wie sie heute üblicherweise gelehrt wird, ist in der Sprache der Mengenlehre formuliert und baut auf den \wikiLinkFt{Axiomen der Mengenlehre} auf. Die meisten mathematischen Objekte, die in Teilbereichen wie \wikiLinkFt{Algebra}, \wikiLinkFt{Analysis}, \wikiLinkFt{Geometrie}, \wikiLinkFt{Stochastik} oder \wikiLinkFt{Topologie} behandelt werden, um nur einige wenige zu nennen, lassen sich als Mengen definieren. Gemessen daran ist die Mengenlehre eine recht junge Wissenschaft; erst nach der Überwindung der \wikiLinkFt{Grundlagenkrise der Mathematik} im frühen 20. Jahrhundert konnte die Mengenlehre ihren heutigen, zentralen und grundlegenden Platz in der Mathematik einnehmen.
	}
}

\newVerweis     {\Mengenoperation}  {\glstext}{Mengenoperation}
\newVerweis[en] {\Mengenoperationen}{\glstext}{Mengenoperation}
\newglossaryentry{Mengenoperation}{
	name        ={Mengenoperation \addIdx     {Mengenoperation}},
	text        ={Mengenoperation},
%%%	symbol      ={\ensuremath{Mathmode}},% ToDo=Mathmode
%%%	user6       ={Textmode},
	description ={\todoBeschreiben%
%%%		\SymbolAmRand{Mengenoperation}%
	}
}

\newsynonym{\Mengenprodukt}{Mengenprodukt}{\kartesischesProdukt}

\newVerweis     {\Mengenrelation}  {\glstext}{Mengenrelation}
\newVerweis[en] {\Mengenrelationen}{\glstext}{Mengenrelation}
\newglossaryentry{Mengenrelation}{
	name        ={Mengenrelation \addIdx     {Mengenrelation}},
	text        ={Mengenrelation},
%%%	symbol      ={\ensuremath{Mathmode}},% ToDo=Mathmode
%%%	user6       ={Textmode},
	description ={\todoBeschreiben%
%%%		\SymbolAmRand{Mengenrelation}%
	}
}

\newVerweis     {\Metadefinition}  {\glstext}{Metadefinition}
\newVerweis[en] {\Metadefinitionen}{\glstext}{Metadefinition}
\newglossaryentry{Metadefinition}{
	name        ={Metadefinition \addIdx     {Metadefinition}},
	text        ={Metadefinition},
%%%	symbol      ={\ensuremath{Mathmode}},% ToDo=Mathmode
%%%	user6       ={Textmode},
	description ={\todoPruefen%
%%%		\SymbolAmRand{Metadefinition}%
		Eine \Metaoperation: Die formale Definition einer \Aussage\ (\Aussagedefinition) \textbzw\ eines \Objekts\ (\Objektdefinition).
	}
}

\newVerweis     {\Metaformel} {\glstext}{Metaformel}
\newVerweis[n]  {\Metaformeln}{\glstext}{Metaformel}
\newglossaryentry{Metaformel}{
	name        ={Metaformel \addIdx    {Metaformel}},
	text        ={Metaformel},
%%%	symbol      ={\ensuremath{Mathmode}},% ToDo=Mathmode
%%%	user6       ={Textmode},
	description ={\todoPruefen%
%%%		\SymbolAmRand{Metaformel}%
		Eine \Formel\ der \formalenMetasprache.
	}
}

\newVerweis     {\Metajunktor}  {\glstext}{Metajunktor}
\newVerweis[en] {\Metajunktoren}{\glstext}{Metajunktor}
\newglossaryentry{Metajunktor}{
	name        ={Metajunktor \addIdx     {Metajunktor}},
	text        ={Metajunktor},
%%%	symbol      ={\ensuremath{Mathmode}},% ToDo=Mathmode
%%%	user6       ={Textmode},
	see         ={Junktor},
	description ={\todoBeschreiben%
%%%		\SymbolAmRand{Metajunktor}%
	}
}

\newVerweis     {\Metaoperation}  {\glstext} {Metaoperation}
\newVerweis[en] {\Metaoperationen}{\glstext} {Metaoperation}
\newVerweis[en]    {\Moperationen}{\glsuseri}{Metaoperation}
\newglossaryentry{Metaoperation}{
	name        ={Metaoperation \addIdx      {Metaoperation}},
	text        ={Metaoperation},
	user1       =    {operation},
%%%	symbol      ={\ensuremath{Mathmode}},% ToDo=Mathmode
%%%	user6       ={Textmode},
	see         ={Objektoperation},
	description ={\todoPruefen%
%%%		\SymbolAmRand{Metaoperation}%
		Eine \Operation\ der \Metasprache: \MtsAnd, \MtsOr\ oder \MtsUnd.
	}
}

\newVerweis     {\Metarelation}  {\glstext} {Metarelation}
\newVerweis[en] {\Metarelationen}{\glstext} {Metarelation}
\newVerweis[en]    {\Mrelationen}{\glsuseri}{Metarelation}
\newglossaryentry{Metarelation}{
	name        ={Metarelation \addIdx      {Metarelation}},
	text        ={Metarelation},
	user1       =    {relation},
%%%	symbol      ={\ensuremath{Mathmode}},% ToDo=Mathmode
%%%	user6       ={Textmode},
	see         ={Objektrelation},
	description ={\todoPruefen%
%%%		\SymbolAmRand{Metarelation}%
		Eine \Relation\ der \Metasprache: \MtsImp, \MtsRep\ oder \MtsEquiv.
	}
}

\newVerweis     {\Metasprache} {\glstext}{Metasprache}
\newVerweis[n]  {\Metasprachen}{\glstext}{Metasprache}
\newglossaryentry{Metasprache}{
	name        ={Metasprache \addIdx    {Metasprache}},
	text        ={Metasprache},
%%%	symbol      ={\ensuremath{Mathmode}},% ToDo=Mathmode
%%%	user6       ={Textmode},
	see         ={Objektsprache},
	description ={\todoOk%
%%%		\SymbolAmRand{Metasprache}%
		Die \Sprache, in der \Aussagen\ über eine andere \Sprache\ getroffen werden können.
		\Hier\ ist dies immer die normale Umgangssprache.
		Ihre \Syntax\ ist gegeben, \textbzgl\ der \Semantik\ bemühen wir uns um exakte Definitionen der \Begriffe\ und \Bezeichnungen.
	}
}

\dummyVerweis   {\Modell}{\glstext}{Modell}% ToDo=Modell

\newVerweis      {\formaleMetasprache}{\glstext}  {formaleMetasprache}
\newVerweis     {\formalenMetasprache}{\glsuseri} {formaleMetasprache}
\newVerweis     {\formalenM}          {\glsuserii}{formaleMetasprache}
\newglossaryentry {formaleMetasprache}{
	name       =                 {---, formale \addIdx[
		name   =                 {---, formale},
		sort   =         {Metasprache, formale}]  {formaleMetasprache}},
	sort       =         {Metasprache, formale},
	text       ={formale  Metasprache},
	user1      ={formalen Metasprache},
	user2      ={formalen},
%%%	symbol     ={\ensuremath{Mathmode}},% ToDo=Mathmode
%%%	user6      ={Textmode},
	description={\todoOk%
%%%		\SymbolAmRand{formaleMetasprache}%
		Die \Metasprache, deren Ausdrucksmittel nur \atomare\ \Aussagen\ und definierte \Metasymbole\ sind.
		\Hier\ ist ihre Syntax und Semantik passend für \ASBA\ definiert, in der Regel parallel zur \Praedikatenlogik.
	}
}

\newVerweis     {\Metasymbol} {\glstext}{Metasymbol}
\newVerweis[e]  {\Metasymbole}{\glstext}{Metasymbol}
\newglossaryentry{Metasymbol}{
	name        ={Metasymbol \addIdx    {Metasymbol}},
	text        ={Metasymbol},
%%%	symbol      ={\ensuremath{Mathmode}},% ToDo=Mathmode
%%%	user6       ={Textmode},
	see         ={Objektsymbol},
	description ={\todoPruefen%
%%%		\SymbolAmRand{Metasymbol}%
		Ein \Symbol\ der \formalenMetasprache.
	}
}

\newVerweis     {\Metavariable} {\glstext} {Metavariable}
\newVerweis[n]     {\Mvariablen}{\glsuseri}{Metavariable}
\newglossaryentry{Metavariable}{
	name        ={Metavariable \addIdx     {Metavariable}},
	text        ={Metavariable},
	user1       =    {variable},
%%%	symbol      ={\ensuremath{Mathmode}},% ToDo=Mathmode
%%%	user6       ={Textmode},
	description ={\todoPruefen%
%%%		\SymbolAmRand{Metavariable}%
		Eine \Variable\ der \formalenMetasprache.
	}
}

\newVerweis     {\Monotonieregel}{\glstext}{Monotonieregel}
\newglossaryentry{Monotonieregel}{
	name        ={Monotonieregel \addIdx   {Monotonieregel}},
	text        ={Monotonieregel},
%%%	symbol      ={\ensuremath{Mathmode}},% ToDo=Mathmode
%%%	user6       ={Textmode},
	see         ={MR},
	description ={\todoPruefen%
%%%		\SymbolAmRand{Monotonieregel}%
		Eine \Schlussregel.
	}
}

%N === N === N === N === N === N === N === N === N === N === N === N === N === N

\newVerweis     {\Nachfolger}{\glstext}{Nachfolger}
\newglossaryentry{Nachfolger}{
	name        ={Nachfolger \addIdx   {Nachfolger}},
	text        ={Nachfolger},
%%%	symbol      ={\ensuremath{Mathmode}},% ToDo=Mathmode
%%%	user6       ={Textmode},
	see         ={Anfangsglied,Endglied,Vorgaenger,Zwischenglied},
	description ={\todoOk%
%%%		\SymbolAmRand{Nachfolger}%
		Der \GloFt{Nachfolger} eines \Kettenglieds\ in einer \Kette\ ist das nachfolgende (rechts stehende) \Kettenglied\ in der \Kette.
	}
}

\newVerweis     {\Negation}  {\glstext}{Negation}
\newVerweis[en] {\Negationen}{\glstext}{Negation}
\newglossaryentry{Negation}{
	name        ={Negation \addIdx     {Negation}},
	text        ={Negation},
%%%	symbol      ={\ensuremath{Mathmode}},% ToDo=Mathmode
%%%	user6       ={Textmode},
	description ={\todoPruefen%
%%%		\SymbolAmRand{Negation}%
		Die \GloFt{Negation} \emph{von} einer \binaeren\ \Relation\ $(G,A,B)$ ist die \Relation\ $(H,A,B)$ mit $H = (A \MtsTimes B) \MtsSetminus G\}$.
		Üblicherweise wird das zugehörige \Relationssymbol\ mit einem schrägen oder vertikalen Strich durchgestrichen.
		Die \gloFt{Negation} der \Umkehrrelation\ einer \Relation\ ist gleich der \Umkehrrelation\ ihrer \gloFt{Negation}.
	}
}

%O === O === O === O === O === O === O === O === O === O === O === O === O === O

\newVerweis     {\Oberaussage} {\glstext}{Oberaussage}
\newVerweis[n]  {\Oberaussagen}{\glstext}{Oberaussage}
\newglossaryentry{Oberaussage}{
	name        ={Oberaussage \addIdx    {Oberaussage}},
	text        ={Oberaussage},
%%%	symbol      ={\ensuremath{Mathmode}},% ToDo=Mathmode
%%%	user6       ={Textmode},
	description ={\todoOk%
%%%		\SymbolAmRand{Oberaussage}%
		Eine \Aussage\ $A$ ist genau dann eine \GloFt{Oberaussage} einer \Aussage\ $B$, wenn $B$ eine \Teilaussage\ von $A$ ist.
	}
}

\newVerweis      {\echteOberaussage}{\glstext} {echteOberaussage}
\newVerweis     {\echtenOberaussage}{\glsuseri}{echteOberaussage}
\newglossaryentry {echteOberaussage}{
	name       =               {---, echte \addIdx[
		name   =               {---, echte},
		sort   =       {Oberaussage, echte}]   {echteOberaussage}},
	sort       =       {Oberaussage, echte},
	text       ={echte  Oberaussage},
	user1      ={echten Oberaussage},
%%%	symbol     ={\ensuremath{Mathmode}},% ToDo=Mathmode
%%%	user6      ={Textmode},
	description={\todoOk%
%%%		\SymbolAmRand{echteOberaussage}%
		Eine \Aussage\ $A$ ist genau dann eine \GloFt{echte Oberaussage} einer \Aussage\ $B$, wenn $B$ eine \echteTeilaussage\ von $A$ ist.
	}
}

\newVerweis     {\Oberbereich} {\glstext} {Oberbereich}
\newVerweis[n]  {\Oberbereichn}{\glstext} {Oberbereich}
\newglossaryentry{Oberbereich}{
	name        ={Oberbereich \addIdx     {Oberbereich}},
	text        ={Oberbereich},
%%%	symbol      ={\ensuremath{Mathmode}},% ToDo=Mathmode
%%%	user6       ={Textmode},
	see         ={Teilbereich},
	description ={\todoOk%
%%%		\SymbolAmRand{Oberbereich}%
		Ein \Bereich\ $A$ ist ist genau dann ein \GloFt{Oberbereich} von einem \Bereich\ $B$, wenn $A \MtsSupsetEq B$ ist.
	}
}

\newVerweis     {\echterOberbereich}{\glstext}  {echterOberbereich}
\newglossaryentry{echterOberbereich}{
	name       =               {---, echter \addIdx[
		name   =               {---, echter},
		sort   =       {Oberbereich, echter}]   {echterOberbereich}},
	sort       =       {Oberbereich, echter},
	text       ={echter Oberbereich},
%%%	symbol     ={\ensuremath{Mathmode}},% ToDo=Mathmode
%%%	user6      ={Textmode},
	see        ={echterTeilbereich},
	description={\todoOk%
%%%		\SymbolAmRand{echterOberbereich}%
		Ein \Bereich\ $A$ ist ist genau dann ein \GloFt{echter Oberbereich} von einem \Bereich\ $B$, wenn $A \MtsSupset B$ ist.
	}
}

\newVerweis     {\Oberfolge} {\glstext}{Oberfolge}
\newVerweis[n]  {\Oberfolgen}{\glstext}{Oberfolge}
\newglossaryentry{Oberfolge}{
	name        ={Oberfolge \addIdx    {Oberfolge}},
	text        ={Oberfolge},
%%%	symbol      ={\ensuremath{Mathmode}},% ToDo=Mathmode
%%%	user6       ={Textmode},
	description ={\todoPruefen%
%%%		\SymbolAmRand{Oberfolge}%
		Eine \Folge\ $A$ ist genau dann eine \GloFt{Oberfolge} einer \Folge\ $B$, wenn $B$ eine \Teilfolge\ von $A$ ist.
	}
}

\newVerweis      {\echteOberfolge}{\glstext} {echteOberfolge}
\newVerweis     {\echtenOberfolge}{\glsuseri}{echteOberfolge}
\newglossaryentry {echteOberfolge}{
	name       =              {---, echte \addIdx[
		name   =              {---, echte},
		sort   =       {Oberfolge, echte}]   {echteOberfolge}},
	sort       =       {Oberfolge, echte},
	text       ={echte  Oberfolge},
	user1      ={echten Oberfolge},
%%%	symbol     ={\ensuremath{Mathmode}},% ToDo=Mathmode
%%%	user6      ={Textmode},
	description={\todoPruefen%
%%%		\SymbolAmRand{echteOberfolge}%
		Eine \Folge\ $A$ ist genau dann eine \GloFt{echte Oberfolge} einer \Folge\ $B$, wenn $B$ eine \echteTeilfolge\ von $A$ ist.
	}
}

\newVerweis     {\Oberformel} {\glstext}{Oberformel}
\newVerweis[n]  {\Oberformeln}{\glstext}{Oberformel}
\newglossaryentry{Oberformel}{
	name        ={Oberformel \addIdx    {Oberformel}},
	text        ={Oberformel},
%%%	symbol      ={\ensuremath{Mathmode}},% ToDo=Mathmode
%%%	user6       ={Textmode},
	description ={\todoPruefen%
%%%		\SymbolAmRand{Oberformel}%
		Eine \Formel\ $A$ ist genau dann eine \GloFt{Oberformel} einer \Formel\ $B$, wenn $B$ eine \Teilformel\ von $A$ ist.
	}
}

\newVerweis      {\echteOberformel}{\glstext} {echteOberformel}
\newVerweis     {\echtenOberformel}{\glsuseri}{echteOberformel}
\newglossaryentry {echteOberformel}{
	name       =              {---, echte \addIdx[
		name   =              {---, echte},
		sort   =       {Oberformel, echte}]   {echteOberformel}},
	sort       =       {Oberformel, echte},
	text       ={echte  Oberformel},
	user1      ={echten Oberformel},
%%%	symbol     ={\ensuremath{Mathmode}},% ToDo=Mathmode
%%%	user6      ={Textmode},
	description={\todoPruefen%
%%%		\SymbolAmRand{echteOberformel}%
		Eine \Formel\ $A$ ist genau dann eine \GloFt{echte Oberformel} einer \Formel\ $B$, wenn $B$ eine \echteTeilformel\ von $A$ ist.
	}
}

\newVerweis     {\Obermenge} {\glstext}{Obermenge}
\newVerweis[n]  {\Obermengen}{\glstext}{Obermenge}
\newglossaryentry{Obermenge}{
	name        ={Obermenge \addIdx    {Obermenge}},
	text        ={Obermenge},
%%%	symbol      ={\ensuremath{Mathmode}},% ToDo=Mathmode
%%%	user6       ={Textmode},
	see         ={Oberbereich,Teilmenge},
	description ={\todoOk%
%%%		\SymbolAmRand{Obermenge}%
		Eine \Menge\ $A$ ist genau dann eine \GloFt{Obermenge} von einer \Menge\ $B$, wenn $A \MtsSupsetEq B$ ist.
	}
}

\newVerweis      {\echteObermenge}{\glstext}  {echteObermenge}
\newVerweis     {\echtenObermenge}{\glsuseri} {echteObermenge}
\newVerweis      {\echteOM}       {\glsuserii}{echteObermenge}
\newglossaryentry {echteObermenge}{
	name       =             {---, echte \addIdx[
		name   =             {---, echte},
		sort   =       {Obermenge, echte}]    {echteObermenge}},
	sort       =       {Obermenge, echte},
	text       ={echte  Obermenge},
	user1      ={echten Obermenge},
	user2      ={echte},
%%%	symbol     ={\ensuremath{Mathmode}},% ToDo=Mathmode
%%%	user6      ={Textmode},
	see        ={echterOberbereich,echteTeilmenge},
	description={\todoOk%
%%%		\SymbolAmRand{echteObermenge}%
		Eine \Menge\ $A$ ist genau dann eine \GloFt{echte Obermenge} von einer \Menge\ $B$, wenn $A \MtsSupset B$ ist.
	}
}

\newVerweis     {\Oberobjekt} {\glstext}{Oberobjekt}
\newVerweis[e]  {\Oberobjekte}{\glstext}{Oberobjekt}
\newglossaryentry{Oberobjekt}{
	name        ={Oberobjekt \addIdx    {Oberobjekt}},
	text        ={Oberobjekt},
%%%	symbol      ={\ensuremath{Mathmode}},% ToDo=Mathmode
%%%	user6       ={Textmode},
	description ={\todoPruefen%
%%%		\SymbolAmRand{Oberobjekt}%
		Eine \Objekt\ $A$ ist genau dann ein \GloFt{Oberobjekt} eines \Objekts\ $B$, wenn $B$ ein \Teilobjekt\ von $A$ ist.
	}
}

\newVerweis     {\echtesOberobjekt}{\glstext} {echtesOberobjekt}
\newVerweis     {\echtenOberobjekt}{\glsuseri}{echtesOberobjekt}
\newglossaryentry{echtesOberobjekt}{
	name       =              {---, echtes \addIdx[
		name   =              {---, echtes},
		sort   =       {Oberobjekt, echtes}]  {echtesOberobjekt}},
	sort       =       {Oberobjekt, echtes},
	text       ={echtes Oberobjekt},
	user1      ={echten Oberobjekt},
%%%	symbol     ={\ensuremath{Mathmode}},% ToDo=Mathmode
%%%	user6      ={Textmode},
	description={\todoPruefen%
%%%		\SymbolAmRand{echtesOberobjekt}%
		Ein \Objekt\ $A$ ist genau dann ein \GloFt{echtes Oberobjekt} eines \Objekts\ $B$, wenn $B$ ein \echtesTeilobjekt\ von $A$ ist.
	}
}

\newVerweis     {\Obersprache} {\glstext} {Obersprache}
\newVerweis[e]  {\Obersprachee}{\glstext} {Obersprache}
\newglossaryentry{Obersprache}{
	name        ={Obersprache \addIdx     {Obersprache}},
	text        ={Obersprache},
	user1       =    {sprache},
%%%	symbol      ={\ensuremath{Mathmode}},% ToDo=Mathmode
%%%	user6       ={Textmode},
	description ={\todoPruefen%
%%%		\SymbolAmRand{Obersprache}%
		Eine \Sprache\ $A$ ist genau dann eine \GloFt{Obersprache} einer \Sprache\ $B$, wenn $B$ eine \Teilsprache\ von $A$ ist.
	}
}

\newVerweis     {\echteObersprache}{\glstext}  {echteObersprache}
\newglossaryentry{echteObersprache}{
	name        =               {---, echte \addIdx[
		name    =               {---, echte},
		sort    =       {Obersprache, echte}]  {echteObersprache}},
	sort        =       {Obersprache, echte},
	text        ={echte Obersprache},
	user1       ={echten Obersprache},
	user2       =           {sprache},
%%%	symbol      ={\ensuremath{Mathmode}},% ToDo=Mathmode
%%%	user6       ={Textmode},
	description={\todoPruefen%
%%%		\SymbolAmRand{echteObersprache}%
		Eine \Sprache\ $A$ ist genau dann eine \GloFt{echte Obersprache} einer \Sprache\ $B$, wenn $B$ eine \echteTeilsprache\ von $A$ ist.
	}
}

\newVerweis     {\Obersymbol} {\glstext}{Obersymbol}
\newVerweis[e]  {\Obersymbole}{\glstext}{Obersymbol}
\newglossaryentry{Obersymbol}{
	name        ={Obersymbol \addIdx    {Obersymbol}},
	text        ={Obersymbol},
%%%	symbol      ={\ensuremath{Mathmode}},% ToDo=Mathmode
%%%	user6       ={Textmode},
	description ={\todoErgaenzen%
%%%		\SymbolAmRand{Obersymbol}%
		Ein \Symbol\ $A$ ist genau dann ein \GloFt{Obersymbol} eines \Symbols\ $B$, wenn $B$ ein \Teilsymbol\ von $A$ ist.
	}
}

\newVerweis     {\echtesObersymbol}{\glstext} {echtesObersymbol}
\newVerweis     {\echtenObersymbol}{\glsuseri}{echtesObersymbol}
\newglossaryentry{echtesObersymbol}{
	name       =              {---, echtes \addIdx[
		name   =              {---, echtes},
		sort   =       {Obersymbol, echtes}]  {echtesObersymbol}},
	sort       =       {Obersymbol, echtes},
	text       ={echtes Obersymbol},
	user1      ={echten Obersymbol},
%%%	symbol     ={\ensuremath{Mathmode}},% ToDo=Mathmode
%%%	user6      ={Textmode},
	description={\todoErgaenzen%
%%%		\SymbolAmRand{echtesObersymbol}%
		Ein \Symbol\ $A$ ist genau dann ein \GloFt{echtes Obersymbol} eines \Symbols\ $B$, wenn $B$ ein \echtesTeilsymbol\ von $A$ ist.
	}
}

\newVerweis         {\Objekt}  {\glstext}{Objekt}
\newVerweis[e]      {\Objekte} {\glstext}{Objekt}
\newVerweis[s]      {\Objekts} {\glstext}{Objekt}
\newVerweis[en]     {\Objekten}{\glstext}{Objekt}
\longnewglossaryentry{Objekt}{
	name            ={Objekt \addIdx     {Objekt}},
	text            ={Objekt},
%%%	symbol          ={\ensuremath{Mathmode}},% ToDo=Mathmode
%%%	user6           ={Textmode},
}{\todoOk%
%%%	\SymbolAmRand{Objekt}%
	\wikicite{bib:mathematischesObjekt}{
		Als \wikiBoldFt{mathematische Objekte} werden die \wikiLinkFt{abstrakten} \wikiLinkFt{Objekte} bezeichnet, die in den verschiedenen \wikiLinkFt{Teilgebieten der Mathematik} beschrieben und untersucht werden. \wikiLinkFt{Grundlegende} Beispiele sind \wikiLinkFt{Zahlen}, \wikiLinkFt{Mengen} und \wikiLinkFt{geometrische Körper}, weiterführend sind beispielsweise \wikiLinkFt{Graphen}, \wikiLinkFt{Integrale} und \wikiLinkFt{Kohomologien}. Die Fragen zur Existenz und zu der Natur von mathematischen Objekten sind zentral in der \wikiLinkFt{Philosophie der Mathematik}. Die zeitgenössische Mathematik hingegen klammert diese Fragestellungen aus und beschäftigt sich \wikiLinkFt{innerstrukturell} mit ihnen. Dies schließt Bereiche wie \wikiLinkFt{Mengenlehre}, \wikiLinkFt{Prädikatenlogik}, \wikiLinkFt{Modelltheorie} und \wikiLinkFt{Kategorientheorie} mit ein, in denen die (sonst übergeordneten) mathematischen Strukturen wie \wikiLinkFt{Axiome}, \wikiLinkFt{Schlussregeln} und \wikiLinkFt{Beweise} erforscht werden, die damit selbst zu mathematischen Objekten werden. Die Ansichten darüber, was mathematische Objekte sind, haben sich im Lauf der \wikiLinkFt{Geschichte der Mathematik} stark gewandelt.
	}
	Ein \GloFt{Objekt} ist \hier\ immer ein \Element\ aus \MtsUniversum.
}

\newVerweis     {\formalesObjekt} {\glstext} {formalesObjekt}
\newVerweis     {\formalenObjekte}{\glsuseri}{formalesObjekt}
\newglossaryentry{formalesObjekt}{
	name       = {formalesObjekt \addIdx     {formalesObjekt}},
	name       =            {---, formales \addIdx[
		name   =            {---, formales},
		sort   =         {Objekt, formales}] {formalesObjekt}},
	sort       =         {Objekt, formales},
	text       ={formales Objekt},
	user1      ={formalen Objekte},
%%%	symbol     ={\ensuremath{Mathmode}},% ToDo=Mathmode
%%%	user6      ={Textmode},
	description={\todoOk%
%%%		\SymbolAmRand{formalesObjekt}%
		Ein \GloFt{formales Objekt} ist ein \Objekt, das in \Aussagen\ in \Objektsprache\ einen \Parameter\ ersetzen darf.
		Es ist notwendigerweise in \Objektsprache\ geschrieben.
	}
}

\newVerweis     {\metasprachlichesObjekt} {\glstext}        {metasprachlichesObjekt}
\newVerweis      {\metasprachlicheObjekte}{\glspl}          {metasprachlichesObjekt}
\newglossaryentry{metasprachlichesObjekt}{
	name       = {metasprachlichesObjekt \addIdx            {metasprachlichesObjekt}},
	name       =                    {---, metasprachliches \addIdx[
		name   =                    {---, metasprachliches},
		sort   =                 {Objekt, metasprachliches}]{metasprachlichesObjekt}},
	sort       =                 {Objekt, metasprachliches},
	text       ={metasprachliches Objekt},
	plural     ={metasprachliche  Objekte},
%%%	symbol     ={\ensuremath{Mathmode}},% ToDo=Mathmode
%%%	user6      ={Textmode},
	description={\todoPruefen%
%%%		\SymbolAmRand{metasprachlichesObjekt}%
		Ein \GloFt{metasprachliches Objekt} ist ein \Objekt\ in \Metasprache.
	}
}

\newVerweis     {\Objektart}  {\glstext}{Objektart}
\newVerweis[en] {\Objektarten}{\glstext}{Objektart}
\newglossaryentry{Objektart}{
	name        ={Objektart \addIdx     {Objektart}},
	text        ={Objektart},
%%%	symbol      ={\ensuremath{Mathmode}},% ToDo=Mathmode
%%%	user6       ={Textmode},
	description ={\todoBeschreiben%
%%%		\SymbolAmRand{Objektart}%
	}
}

\newVerweis     {\Objektbereich}{\glstext} {Objektbereich}
\newglossaryentry{Objektbereich}{
	name        ={Objektbereich \addIdx    {Objektbereich}},
	text        ={Objektbereich},
%%%	symbol      ={\ensuremath{Mathmode}},% ToDo=Mathmode
%%%	user6       ={Textmode},
	description ={\todoOk%
%%%		\SymbolAmRand{Objektbereich}%
		Der \GloFt{Objektbereich} \MtsObjekte\ ist der \Bereich\ aller \formalenObjekte, \textdh\ der \Objekte, die in \Aussagen\ in \Objektsprache\ einen \Parameter\ ersetzen dürfen.
		Diese Objekte sind notwendigerweise auch in \Objektsprache\ geschrieben und offensichtlich ist $\MtsObjekte \MtsSubsetEq \MtsUniversum$.
	}
}

\newVerweis     {\Objektdefinition}  {\glstext}{Objektdefinition}
\newVerweis[en] {\Objektdefinitionen}{\glstext}{Objektdefinition}
\newglossaryentry{Objektdefinition}{
	name        ={Objektdefinition \addIdx     {Objektdefinition}},
	text        ={Objektdefinition},
%%%	symbol      ={\ensuremath{Mathmode}},% ToDo=Mathmode
%%%	user6       ={Textmode},
	see         ={Aussagedefinition},
	description ={\todoOk%
%%%		\SymbolAmRand{Objektdefinition}%
		Eine \Metadefinition: Die formale Definition eines \Objekts.
		\ifmarginparFlg\newline\else\fi
		\seqqt{$A \MtsDefEq B$} steht für \standsfor{$A$ ist \DefFt{definitionsgemäß gleich} $B$} für \Objekte\ $A$ und $B$.
		Gewissermaßen ist $A$ nur eine andere Schreibweise für $B$.
	}
}

\newVerweis     {\Objektformel} {\glstext}{Objektformel}
\newVerweis[n]  {\Objektformeln}{\glstext}{Objektformel}
\newglossaryentry{Objektformel}{
	name        ={Objektformel \addIdx    {Objektformel}},
	text        ={Objektformel},
%%%	symbol      ={\ensuremath{Mathmode}},% ToDo=Mathmode
%%%	user6       ={Textmode},
	description ={\todoPruefen%
%%%		\SymbolAmRand{Objektformel}%
		Eine \Formel\ der \Objektsprache.
	}
}

\newVerweis     {\Objektkonstante} {\glstext}{Objektkonstante}
\newVerweis[n]  {\Objektkonstanten}{\glstext}{Objektkonstante}
\newglossaryentry{Objektkonstante}{
	name        ={Objektkonstante \addIdx    {Objektkonstante}},
	text        ={Objektkonstante},
%%%	symbol      ={\ensuremath{Mathmode}},% ToDo=Mathmode
%%%	user6       ={Textmode},
	description ={\todoPruefen%
%%%		\SymbolAmRand{Objektkonstante}%
		Eine \Konstante\ der \Objektsprache.
	}
}

\newVerweis     {\Objektoperation}  {\glstext} {Objektoperation}
\newVerweis[en] {\Objektoperationen}{\glstext} {Objektoperation}
\newVerweis[en]      {\Ooperationen}{\glsuseri}{Objektoperation}
\newglossaryentry{Objektoperation}{
	name        ={Objektoperation \addIdx      {Objektoperation}},
	text        ={Objektoperation},
	user1       =      {operation},
%%%	symbol      ={\ensuremath{Mathmode}},% ToDo=Mathmode
%%%	user6       ={Textmode},
	see         ={Metaoperation},
	description ={\todoPruefen%
%%%		\SymbolAmRand{Objektoperation}%
		Eine \Operation\ der \Objektsprache: \OjkAnd, \OjkOr.
	}
}

\newVerweis     {\Objektrelation}  {\glstext} {Objektrelation}
\newVerweis[en] {\Objektrelationen}{\glstext} {Objektrelation}
\newVerweis[en]      {\Orelationen}{\glsuseri}{Objektrelation}
\newglossaryentry{Objektrelation}{
	name        ={Objektrelation \addIdx      {Objektrelation}},
	text        ={Objektrelation},
	user1       =      {relation},
%%%	symbol      ={\ensuremath{Mathmode}},% ToDo=Mathmode
%%%	user6       ={Textmode},
	see         ={Metarelation},
	description ={\todoPruefen%
%%%		\SymbolAmRand{Objektrelation}%
		Eine \Relation\ der \Objektsprache: \OjkImp, \OjkRep\ oder \OjkEquiv.
	}
}

\newVerweis     {\Objektsprache} {\glstext}{Objektsprache}
\newVerweis[n]  {\Objektsprachen}{\glstext}{Objektsprache}
\newglossaryentry{Objektsprache}{
	name        ={Objektsprache \addIdx    {Objektsprache}},
	text        ={Objektsprache},
%%%	symbol      ={\ensuremath{Mathmode}},% ToDo=Mathmode
%%%	user6       ={Textmode},
	description ={\todoOk%
%%%		\SymbolAmRand{Objektsprache}%
		Die \Sprache, über die mittels einer (\formalenM) \Metasprache\ "`geredet"' wird.
		Unser \Objekt, mit dem mathematische \Beweise\ formuliert werden sollen, ist die \Logik.
		Demnach sind die Ausdrucksmittel der \Objektsprache\ die der \Logik.
		Wir verwenden \hier\ die \Praedikatenlogik\ oder, als \echteTeilsprache, die \Aussagenlogik.
	}
}

\newVerweis     {\Objektsymbol} {\glstext}{Objektsymbol}
\newVerweis[e]  {\Objektsymbole}{\glstext}{Objektsymbol}
\newglossaryentry{Objektsymbol}{
	name        ={Objektsymbol \addIdx    {Objektsymbol}},
	text        ={Objektsymbol},
%%%	symbol      ={\ensuremath{Mathmode}},% ToDo=Mathmode
%%%	user6       ={Textmode},
	see         ={Metasymbol},
	description ={\todoPruefen%
%%%		\SymbolAmRand{Objektsymbol}%
		Ein \Symbol\ der \Objektsprache.
	}
}

\newVerweis     {\Operation}  {\glstext}{Operation}
\newVerweis[en] {\Operationen}{\glstext}{Operation}
\newglossaryentry{Operation}{
	name        ={Operation \addIdx     {Operation}},
	text        ={Operation},
%%%	symbol      ={\ensuremath{Mathmode}},% ToDo=Mathmode
%%%	user6       ={Textmode},
	description ={\todoPruefen%
%%%		\SymbolAmRand{Operation}%
		Eine \GloFt{Operation} ist eine --- meistens \binaere, \textdh\ zweiwertige --- \Funktion\ $M^n \MtsFktArrow M$ mit $n \MtsIn \MtsINo$.
		Für eine \binaere, \textdh\ $n = 2$, \gloFt{Operation} $\FunktionDef{\BspOpB}{M \MtsTimes M}{M}$ schreibt man meistens $x \BspOpB y$ statt $\BspOpB(x,y)$.
		Für $n = 0$ kann man die \gloFt{Operation} mit einer \Konstanten\ identifizieren.
	}
}

\newVerweis      {\aussagenlogischeOperation}  {\glstext}       {aussagenlogischeOperation}
\newVerweis[en]  {\aussagenlogischeOperationen}{\glstext}       {aussagenlogischeOperation}
\newVerweis[en] {\aussagenlogischenOperationen}{\glsuseri}      {aussagenlogischeOperation}
\newVerweis[en]                 {\aOperationen}{\glsuserii}     {aussagenlogischeOperation}
\newglossaryentry {aussagenlogischeOperation}{
	name       =                        {---, aussagenlogische \addIdx[
		name   =                        {---, aussagenlogische},
		sort   =                  {Operation, aussagenlogische}]{aussagenlogischeOperation}},
	sort       =                  {Operation, aussagenlogische},
	text       ={aussagenlogische  Operation},
	user1      ={aussagenlogischen Operation},
	user2      =                  {Operation},
%%%	symbol     ={\ensuremath{Mathmode}},% ToDo=Mathmode
%%%	user6      ={Textmode},
	description={\todoErgaenzen%
%%%		\SymbolAmRand{aussagenlogischeOperation}%
		Die \GloFt{aussagenlogischen Operationen} sind ...
	}
}

\newVerweis     {\Operationssymbol} {\glstext}{Operationssymbol}
\newVerweis[e]  {\Operationssymbole}{\glstext}{Operationssymbol}
\newglossaryentry{Operationssymbol}{
	name        ={Operationssymbol \addIdx    {Operationssymbol}},
	text        ={Operationssymbol},
%%%	symbol      ={\ensuremath{Mathmode}},% ToDo=Mathmode
%%%	user6       ={Textmode},
	description ={\todoPruefen%
%%%		\SymbolAmRand{Operationssymbol}%
		Ein \Symbol\ für eine \Operation.
	}
}

\newVerweis         {\Ordnungsrelation}  {\glstext}{Ordnungrelation}
\newVerweis[en]     {\Ordnungsrelationen}{\glstext}{Ordnungrelation}
\longnewglossaryentry{Ordnungsrelation}{
	name            ={Ordnungsrelation \addIdx[
		name        ={Ordnungsrelation}]           {Ordnungsrelation}},
	text            ={Ordnungsrelation},
%%%	symbol          ={\ensuremath{Mathmode}},% ToDo=Mathmode
%%%	user6           ={Textmode},
}{\todoPruefen%
%%%	\SymbolAmRand{Ordnungsrelation}%
	Eine \GloFt{Ordnungsrelation} ist ein \binaere\ \Relation\ auf einer \Menge\ $M$ mit der folgenden Eigenschaft
	(dabei sei $\preceq$ die \gloFt{Ordnungsrelation}):
	\begin{align}
		&\text{\DefFt{transitiv }}:\qquad ((a \preceq b) \MtsAnd (b \preceq c)) \MtsImp (a \preceq c) \formulatoleft
	\end{align}
	jeweils für alle \Elemente\ $a$, $b$ und $c$ aus $M$.
}

%P === P === P === P === P === P === P === P === P === P === P === P === P === P

\newVerweis     {\geordnetesPaar} {\glstext}  {geordnetesPaar}
\newVerweis[e]  {\geordnetenPaare}{\glsuseri} {geordnetesPaar}
\newglossaryentry{geordnetesPaar}{
	name       =           {Paar, geordnetes \addIdx[
		name   =           {Paar, geordnetes}]{geordnetesPaar}},
	text       ={geordnetes Paar},
	user1      ={geordneten Paar},
%%%	symbol     ={\ensuremath{Mathmode}},% ToDo=Mathmode
%%%	user6      ={Textmode},
	description={\todoBeschreiben%
%%%		\SymbolAmRand{geordnetesPaar}%
	}
}

\newVerweis     {\Parameter} {\glstext}{Parameter}
\newVerweis[n]  {\Parametern}{\glstext}{Parameter}
\newVerweis[s]  {\Parameters}{\glstext}{Parameter}
\newglossaryentry{Parameter}{
	name        ={Parameter \addIdx    {Potenzmenge}},
	text        ={Parameter},
%%%	symbol      ={\ensuremath{Mathmode}},% ToDo=Mathmode
%%%	user6       ={Textmode},
	see         ={Aussage,Variable},
	description ={\todoGeprueft%
%%%		\SymbolAmRand{Parameter}%
		Die \GloFt{Parameter} einer \Aussage\ sind deren \freieVariablen.
	}
}

\newVerweis      {\PolnischeNotation}  {\glstext}  {PolnischeNotation}
\newVerweis[en]  {\PolnischeNotationen}{\glstext}  {PolnischeNotation}
\newVerweis      {\PolnischenNotation} {\glsuseri} {PolnischeNotation}
\newVerweis      {\PolnischerNotation} {\glsuserii}{PolnischeNotation}
\newglossaryentry{PolnischeNotation}{
	name        =           {Notation, Polnische \addIdx[
		name    =           {Notation, Polnische},
		text    ={Polnische  Notation}]            {PolnischeNotation}},
	text        ={Polnische  Notation},
	user1       ={Polnischen Notation},
	user2       ={Polnischer Notation},
%%%	symbol      ={\ensuremath{Mathmode}},% ToDo=Mathmode
%%%	user6       ={Textmode},
	description ={\todoPruefen%
%%%		\SymbolAmRand{PolnischeNotation}%
		Bei der \GloFt{Polnischen Notation} stehen die Argumente von \Relationen\ und \Funktionen\ stets rechts von den \RelationsS- und \Funktionssymbolen.
		Dadurch kann auf \Gliederungszeichen\ wie Klammern und Kommata verzichtet werden.
		Noch einfacher für Computer ist die \GloFt{umgekehrte Polnische Notation}, bei der die Argumente immer links stehen.
	}
}

\newVerweis     {\Potenzmenge} {\glstext}{Potenzmenge}
\newVerweis[n]  {\Potenzmengen}{\glstext}{Potenzmenge}
\newglossaryentry{Potenzmenge}{
	name        ={Potenzmenge \addIdx    {Potenzmenge}},
	text        ={Potenzmenge},
%%%	symbol      ={\ensuremath{Mathmode}},% ToDo=Mathmode
%%%	user6       ={Textmode},
	description ={\todoPruefen%
%%%		\SymbolAmRand{Potenzmenge}%
		Die \Potenzmenge\ $\MtsPot(M)$ einer \Menge\ $M$ ist die \Menge\ ihrer \Teilmengen.
	}
}

\newVerweis     {\Praedikat} {\glstext}{Praedikat}
\newVerweis[e]  {\Praedikate}{\glstext}{Praedikat}
\newVerweis[s]  {\Praedikats}{\glstext}{Praedikat}
\newglossaryentry{Praedikat}{
	name        ={Prädikat \addIdx[
		name    ={Prädikat}]           {Praedikat}},
	text        ={Prädikat},
%%%	symbol      ={\ensuremath{Mathmode}},% ToDo=Mathmode
%%%	user6       ={Textmode},
	description ={\todoPruefen%
%%%		\SymbolAmRand{Praedikat}%
		Ein Element der \Praedikatenlogik.
		--- \textZB\ kann man eine Gruppe als ein zwei\stelliges\ \Praedikat\ $\Preft{Gruppe}(G,+)$ definieren, in dem $G$ eine \Menge\ und $+$ eine \Operation, \textdh\ eine \binaere\ (zwei\stellige) \Funktion\ $ +: G \MtsTimes G \rightarrow G $ ist, so dass die Gruppenaxiome erfüllt sind.
	}
}

\newVerweis         {\Praedikatenlogik}{\glstext}{Praedikatenlogik}
\longnewglossaryentry{Praedikatenlogik}{
	name            ={Prädikatenlogik \addIdx[
		name        ={Prädikatenlogik}]          {Praedikatenlogik}},
	text            ={Prädikatenlogik},
%%%	symbol          ={\ensuremath{Mathmode}},% ToDo=Mathmode
%%%	user6           ={Textmode},
	see             ={Aussagenlogik,Logik},
}{\todoPruefen%
%%%	\SymbolAmRand{Praedikatenlogik}%
	\wikicite{bib:Praedikatenlogik}{
		Die \wikiBoldFt{Prädikatenlogiken} (auch \wikiBoldFt{Quantorenlogiken}) bilden eine Familie \wikiLinkFt{logischer} Systeme, die es erlauben, einen weiten und in der Praxis vieler Wissenschaften und deren Anwendungen wichtigen Bereich von Argumenten zu formalisieren und auf ihre Gültigkeit zu überprüfen. Auf Grund dieser Eigenschaft spielt die Prädikatenlogik eine große Rolle in der \wikiLinkFt{Logik} sowie in \wikiLinkFt{Mathematik}, \wikiLinkFt{Informatik}, \wikiLinkFt{Linguistik} und \wikiLinkFt{Philosophie}.
	}
}

\newVerweis     {\Praemisse} {\glstext}{Praemisse}
\newVerweis[n]  {\Praemissen}{\glstext}{Praemisse}
\newglossaryentry{Praemisse}{
	name        ={Prämisse \addIdx     {Praemisse}},
	text        ={Prämisse},
%%%	symbol      ={\ensuremath{Mathmode}},% ToDo=Mathmode
%%%	user6       ={Textmode},
	see         ={Schlussregel},
	description ={\todoPruefen%
%%%		\SymbolAmRand{Praemisse}%
		Eine \Ableitung:
		Die \Praemissen\ einer \Schlussregel\ $\frac{\MtsPraemisseSet}{\MtsKonklusionSet}$ \textbzw\ $\frac{\MtsPraemisseSet}{\MtsKonklusionSet}$ sind die \Elemente\ aus \MtsPraemisseSet\ \textbzw\ \MtsPraemisseRel.
		Die \Praemissen\ werden normalerweise mit $\MtsPraemisse_i$ bezeichnet.
	}
}

\newVerweis     {\Praemissenmenge} {\glstext} {Praemissenmenge}
\newVerweis[n]  {\Praemissenmengen}{\glstext} {Praemissenmenge}
%%%\newVerweis     {\PraemissenM}     {\glsuseri}{Praemissenmenge}
\newglossaryentry{Praemissenmenge}{
	name        = {Prämissenmenge \addIdx     {Praemissenmenge}},
	text        = {Prämissenmenge},
	user1       = {Prämissen},
%%%	symbol      ={\ensuremath{Mathmode}},% ToDo=Mathmode
%%%	user6       ={Textmode},
	description ={\todoPruefen%
%%%		\SymbolAmRand{Praemissenmenge}%
		Eine \Ableitungsmenge:
		Die \Menge\ \MtsPraemisseSet\ der \Praemissen\ einer \Schlussregel\ \textbzw\ eines \Beweises.
	}
}

\newVerweis         {\kartesischesProdukt}{\glstext}      {kartesischesProdukt}
\newVerweis          {\kartesischeProdukt}{\glsuseri}     {kartesischesProdukt}
\longnewglossaryentry{kartesischesProdukt}{
	name            =             {Produkt, kartesisches \addIdx[
		name        =             {Produkt, kartesisches}]{kartesischesProdukt}},
	text            ={kartesisches Produkt},
	user1           ={kartesische  Produkt},
%%%	symbol          ={\ensuremath{Mathmode}},% ToDo=Mathmode
%%%	user6           ={Textmode},
}{\todoPruefen%
%%%	\SymbolAmRand{kartesischesProdukt}%
	\wikicite{bib:kartesischesProdukt}{
		Das \wikiBoldFt{kartesische Produkt} oder \wikiBoldFt{Mengenprodukt} ist in der Mengenlehre eine grundlegende Konstruktion, aus gegebenen Mengen eine neue Menge zu erzeugen. [\textdots] Das kartesische Produkt zweier Mengen ist die Menge aller geordneten Paare von Elementen der beiden Mengen, wobei die erste Komponente ein Element der ersten Menge und die zweite Komponente ein Element der zweiten Menge ist. Allgemeiner besteht das kartesische Produkt mehrerer Mengen aus der Menge aller Tupel von Elementen der Mengen, wobei die Reihenfolge der Mengen und damit der entsprechenden Elemente fest vorgegeben ist. Die Ergebnismenge des kartesischen Produkts wird auch \wikiBoldFt{Produktmenge}, \wikiBoldFt{Kreuzmenge} oder \wikiBoldFt{Verbindungsmenge} genannt. [\textdots]
	}
}

%Q === Q === Q === Q === Q === Q === Q === Q === Q === Q === Q === Q === Q === Q

\newVerweis[en]     {\Quantoren}{\glstext}{Quantor}
\newVerweis         {\Quantor}  {\glstext}{Quantor}
\longnewglossaryentry{Quantor}{
	name            ={Quantor \addIdx     {Quantor}},
	text            ={Quantor},
%%%	symbol          ={\ensuremath{Mathmode}},% ToDo=Mathmode
%%%	user6           ={Textmode},
	see             ={Allquantor,Existenzquantor,Junktor,Praedikatenlogik},
}{\todoErgaenzen%
%%%	\SymbolAmRand{Quantor}%
	\wikicite{bib:Quantor}{
		Ein \wikiBoldFt{Quantor} oder \wikiBoldFt{Quantifikator}, die Re-Latinisierung des von \wikiLinkFt{C. S. Peirce} eingeführten Ausdrucks „quantifier“, ist ein \wikiLinkFt{Operator} der \wikiLinkFt{Prädikatenlogik}. Neben den \wikiLinkFt{Junktoren} sind die Quantoren Grundzeichen der Prädikatenlogik. Allen Quantoren gemeinsam ist, dass sie \wikiLinkFt{Variablen} \wikiLinkFt{binden}.

		Die beiden gebräuchlichsten Quantoren sind der \wikiItalicFt{Existenzquantor} (in natürlicher Sprache zum Beispiel als „mindestens ein“ ausgedrückt) und der \wikiItalicFt{Allquantor} (in natürlicher Sprache zum Beispiel als „alle“ oder „jede/r/s“ ausgedrückt). Andere Arten von Quantoren sind \wikiItalicFt{Anzahlquantoren} wie „ein“ oder „zwei“, die sich auf Existenz- beziehungsweise Allquantor zurückführen lassen, und Quantoren wie „manche“, „einige“ oder „viele“, die auf Grund ihrer Unbestimmtheit in der \wikiLinkFt{klassischen Logik} nicht verwendet werden.
	}
}

\newVerweis     {\logischerQuantor} {\glstext} {logischerQuantor}
\newglossaryentry{logischerQuantor}{
	name       =              {---, logischer \addIdx[
		name   =              {---, logischer},
		sort   =          {Quantor, logischer}]{logischerQuantor}},
	sort       =          {Quantor, logischer},
	text       ={logischer Quantor},
%%%	symbol     ={\ensuremath{Mathmode}},% ToDo=Mathmode
%%%	user6      ={Textmode},
	description={\todoBeschreiben%
%%%		\SymbolAmRand{logischerQuantor}%
	}
}

\newVerweis     {\metasprachlicherQuantor} {\glstext}        {metasprachlicherQuantor}
\newglossaryentry{metasprachlicherQuantor}{
	name       =                     {---, metasprachlicher \addIdx[
		name   =                     {---, metasprachlicher},
		sort   =                 {Quantor, metasprachlicher}]{metasprachlicherQuantor}},
	sort       =                 {Quantor, metasprachlicher},
	text       ={metasprachlicher Quantor},
%%%	symbol     ={\ensuremath{Mathmode}},% ToDo=Mathmode
%%%	user6      ={Textmode},
	description={\todoBeschreiben%
%%%		\SymbolAmRand{metasprachlicherQuantor}%
	}
}

\newVerweis     {\Quellbereich} {\glstext} {Quellbereich}
\newVerweis[e]  {\Quellbereiche}{\glstext} {Quellbereich}
\newVerweis     {\QuellB}       {\glsuseri}{Quellbereich}
\newglossaryentry{Quellbereich}{
	name        ={Quellbereich \addIdx     {Quellbereich}},
	text        ={Quellbereich},
	user1       ={Quell},
%%%	symbol      ={\ensuremath{Mathmode}},% ToDo=Mathmode
%%%	user6       ={Textmode},
	see         ={Definitionsbereich,Menge},
	description ={\todoPruefen%
%%%		\SymbolAmRand{Quellbereich}%
		Für die \Funktion \FunktionDef{f}{A}{B} ist die \Menge\ $\MtsQb(f) \MtsDefEq \RawMengeDef{a \in A}{f(a) \text{ existiert}}$ ihr \Quellbereich%
		\footnote{%
			Der \GloFt{Quellbereich} $\MtsQb(f)$ unterscheidet sich nur bei \DefFt{partiellen} \Funktionen\ vom \Definitionsbereich\ $\MtsDb(f)$, \textdh\ solchen \Funktionen, für die $f(a)$ nicht für alle $a \MtsIn A$ definiert ist.
		}
		(source).
	}
}

%R === R === R === R === R === R === R === R === R === R === R === R === R === R

\newVerweis         {\Relation}  {\glstext}{Relation}
\newVerweis[en]     {\Relationen}{\glstext}{Relation}
\longnewglossaryentry{Relation}{
	name            ={Relation \addIdx     {Relation}},
	text            ={Relation},
%%%	symbol          ={\ensuremath{Mathmode}},% ToDo=Mathmode
%%%	user6           ={Textmode},
	see             ={Aequivalenzrelation,Begriff,Menge,Objekt,Ordnungsrelation},
}{\todoPruefen%
%%%	\SymbolAmRand{Relation}%
	\wikicite{bib:Relation}{
		Eine \wikiBoldFt{Relation} (\wikiLinkFt{lateinisch} \wikiItalicFt{relatio} „Beziehung“, „Verhältnis“) ist allgemein eine Beziehung, die zwischen Dingen bestehen kann. Relationen im Sinne der \wikiLinkFt{Mathematik} sind ausschließlich diejenigen Beziehungen, bei denen stets klar ist, ob sie bestehen oder nicht; Objekte können also nicht „bis zu einem gewissen Grade“ in einer Relation zueinander stehen. Damit ist eine einfache \wikiLinkFt{mengentheoretische} Definition des Begriffs möglich: Eine Relation $R$ ist eine Menge von $n$-\wikiLinkFt{Tupeln}. In der Relation $R$ zueinander stehende Dinge bilden $n$-Tupel, die Element von $R$ sind.

		Wird nicht ausdrücklich etwas anderes angegeben, versteht man unter einer Relation gemeinhin eine zweistellige oder binäre Relation. Bei einer solchen Beziehung bilden dann jeweils zwei Elemente $a$ und $b$ ein \wikiLinkFt{geordnetes Paar} $(a,b)$. Stammen dabei $a$ und $b$ aus verschiedenen Grundmengen $A$ und $B$, so heißt die Relation \wikiItalicFt{heterogen} oder „Relation \wikiItalicFt{zwischen} den Mengen $A$ und $B$.“ Stimmen die Grundmengen überein ($A = B$), dann heißt die Relation \wikiItalicFt{homogen} oder „Relation \wikiItalicFt{in} bzw. \wikiItalicFt{auf} der Menge $A$.“

		Wichtige Spezialfälle, zum Beispiel \wikiLinkFt{Äquivalenzrelationen} und \wikiLinkFt{Ordnungsrelationen}, sind Relationen \wikiItalicFt{auf} einer Menge.

		Heute sehen manche Autoren den Begriff Relation nicht unbedingt als auf Mengen beschränkt an, sondern lassen jede aus geordneten Paaren bestehende \wikiLinkFt{Klasse} als Relation gelten.
	}
	Eine \DefFt{$n$-\stellige} \GloFt{Relation} $R$ ist ein (1+$n$)-\Tupel\ $(G,A_1,\dots,A_n)$ mit $G \MtsSubsetEq A_1 \MtsTimes \dots \MtsTimes A_n)$.
}

\newVerweis      {\aussagenlogischeRelation}  {\glstext}       {aussagenlogischeRelation}
\newVerweis[en]  {\aussagenlogischeRelationen}{\glstext}       {aussagenlogischeRelation}
\newVerweis[en] {\aussagenlogischenRelationen}{\glsuseri}      {aussagenlogischeRelation}
\newVerweis                     {\aRelation}  {\glsuserii}     {aussagenlogischeRelation}
\newVerweis[en]         {\aRelationen}{\glsuserii}{aussagenlogischeRelation}
\newglossaryentry {aussagenlogischeRelation}{
	name       =                       {---, aussagenlogische \addIdx[
		name   =                       {---, aussagenlogische},
		sort   =                  {Relation, aussagenlogische}]{aussagenlogischeRelation}},
	sort       =                  {Relation, aussagenlogische},
	text       ={aussagenlogische  Relation},
	user1      ={aussagenlogischen Relation},
	user2      =                  {Relation},
%%%	symbol     ={\ensuremath{Mathmode}},% ToDo=Mathmode
%%%	user6      ={Textmode},
	description={\todoErgaenzen%
%%%		\SymbolAmRand{aussagenlogischeRelation}%
		Die \GloFt{aussagenlogischen} \Relationen\ sind ...
	}
}

\newVerweis     {\Relationssymbol} {\glstext} {Relationssymbol}
\newVerweis[e]  {\Relationssymbole}{\glstext} {Relationssymbol}
\newVerweis     {\RelationsS}      {\glsuseri}{Relationssymbol}
\newglossaryentry{Relationssymbol}{
	name        ={Relationssymbol \addIdx     {Relationssymbol}},
	text        ={Relationssymbol},
	user1       ={Relations},
%%%	symbol      ={\ensuremath{Mathmode}},% ToDo=Mathmode
%%%	user6       ={Textmode},
	description ={\todoPruefen%
%%%		\SymbolAmRand{Relationssymbol}%
		Ein \Symbol\ für eine \Relation.
	}
}

%S === S === S === S === S === S === S === S === S === S === S === S === S === S

\newVerweis     {\Satz}   {\glstext}{Satz}
\newVerweis[es] {\Satzes} {\glstext}{Satz}
\newVerweis     {\Saetze} {\glspl}  {Satz}
\newVerweis[n]  {\Saetzen}{\glspl}  {Satz}
\newglossaryentry{Satz}{
	name        ={Satz \addIdx      {Satz}},
	text        ={Satz},
	plural      ={Sätze},
%%%	symbol      ={\ensuremath{Mathmode}},% ToDo=Mathmode
%%%	user6       ={Textmode},
	description ={\todoOk%
%%%		\SymbolAmRand{Satz}%
		Ein \GloFt{Satz} ist eine \Aussage, bestehend aus einer Anzahl von \Praemissen\ und \Konklusionen\ und einem \Beweis, der die \Konklusionen\ aus den \Praemissen\ ableitet.
	}
}

\newVerweis     {\formalerSatz} {\glstext} {formalerSatz}
\newVerweis     {\formalenSatz} {\glsuseri}{formalerSatz}
\newglossaryentry{formalerSatz}{
	name       =          {---, formaler \addIdx[
		name   =          {---, formaler},
		sort   =         {Satz, formaler}] {formalerSatz}},
	sort       =         {Satz, formaler},
	text       ={formaler Satz},
	user1      ={formalen Satz},
%%%	symbol     ={\ensuremath{Mathmode}},% ToDo=Mathmode
%%%	user6      ={Textmode},
	see        ={FS},
	description={\todoPruefen%
%%%		\SymbolAmRand{formalerSatz}%
		Formale \Darstellung\ eines mathematischen \Satzes.
	}
}

\newVerweis         {\Schlussregel} {\glstext}{Schlussregel}
\newVerweis[n]      {\Schlussregeln}{\glstext}{Schlussregel}
\longnewglossaryentry{Schlussregel}{
	name            ={Schlussregel \addIdx    {Schlussregel}},
	text            ={Schlussregel},
%%%	symbol          ={\ensuremath{Mathmode}},% ToDo=Mathmode
%%%	user6           ={Textmode},
	see             ={MtsSchlussregel,MtsSchlussregelSet,Kalkuel},
}{\todoPruefen%
%%%	\SymbolAmRand{Schlussregel}%
	\wikicite{bib:Schlussregel}{
		Eine \wikiBoldFt{Schlussregel} (oder \wikiItalicFt{Inferenzregel}) bezeichnet eine Transformationsregel (Umformungsregel) in einem \wikiLinkFt{Kalkül} der \wikiLinkFt{formalen Logik}, d. h. eine \wikiLinkFt{syntaktische} Regel, nach der es erlaubt ist, von bestehenden Ausdrücken einer formalen Sprache zu neuen Ausdrücken überzugehen. Dieser regelgeleitete Übergang stellt eine \wikiLinkFt{Schlussfolgerung} dar.
	}
	Eine \Schlussregel\ $\frac{\MtsPraemisseSet}{\MtsKonklusionSet}$ entspricht der \Aussage:
	\begin{quote}
		Wenn alle \Praemissen\ $\MtsPraemisse \MtsIn \MtsPraemisseSet$ zutreffen, dann auch alle \Konklusionen\ $\MtsKonklusion \MtsIn \MtsKonklusionSet$.
	\end{quote}
	Wenn diese \Aussage\ zutrifft, kann die Schlussregel zur \zulaessigen\ \Transformation\ von \Formeln\ dienen.
}

\newVerweis     {\allgemeingueltigeSchlussregel} {\glstext}        {allgemeingueltigeSchlussregel}
\newVerweis[n]  {\allgemeingueltigeSchlussregeln}{\glstext}        {allgemeingueltigeSchlussregel}
\newVerweis    {\allgemeingueltigenSchlussregel} {\glsuseri}       {allgemeingueltigeSchlussregel}
\newVerweis[n] {\allgemeingueltigenSchlussregeln}{\glsuseri}       {allgemeingueltigeSchlussregel}
\newglossaryentry{allgemeingueltigeSchlussregel}{
	name       =                           {---, allgemeingültige \addIdx[
		name   =                           {---, allgemeingültige},
		sort   =                  {Schlussregel, allgemeingültige}]{allgemeingueltigeSchlussregel}},
	sort       =                  {Schlussregel, allgemeingültige},
	text       ={allgemeingültige  Schlussregel},
	user1      ={allgemeingültigen Schlussregel},
%%%	symbol     ={\ensuremath{Mathmode}},% ToDo=Mathmode
%%%	user6      ={Textmode},
	description={\todoPruefen%
%%%		\SymbolAmRand{allgemeingueltigeSchlussregel}%
		Eine \Schlussregel\ heißt \GloFt{allgemeingültig}, wenn sie aus den \Basisregeln\ und schon bekannten \allgemeingueltigenSchlussregeln\ abgeleitet werden kann.
	}
}

\newVerweis     {\Schlussregelmenge} {\glstext}{Schlussregelmenge}
\newcommand*    {\Schlussregelmengen}[1][]{\glstext[#1]{Schlussregelmenge}n[]}
\newglossaryentry{Schlussregelmenge}{
	name        ={Schlussregelmenge \addIdx    {Schlussregelmenge}},
	text        ={Schlussregelmenge},
%%%	symbol      ={\ensuremath{Mathmode}},% ToDo=Mathmode
%%%	user6       ={Textmode},
	see         ={MtsSchlussregelSet},
	description ={\todoPruefen%
%%%		\SymbolAmRand{Schlussregelmenge}%
		Eine \Menge\ von \Schlussregeln, meistens mit \MtsSchlussregelSet\ bezeichnet.
	}
}

\newVerweis     {\Schnittregel}{\glstext}{Schnittregel}
\newglossaryentry{Schnittregel}{
	name        ={Schnittregel \addIdx   {Schnittregel}},
	text        ={Schnittregel},
	see         ={SR},
	description ={\todoPruefen%
%%%		\SymbolAmRand{Schnittregel}%
		Eine \allgemeingueltigeSchlussregel.
	}
}

\newVerweis         {\Semantik} {\glstext}{Semantik}
\longnewglossaryentry{Semantik}{
	name            ={Semantik \addIdx    {Semantik}},
	text            ={Semantik},
%%%	symbol          ={\ensuremath{Mathmode}},% ToDo=Mathmode
%%%	user6           ={Textmode},
}{\todoGeprueft%
%%%	\SymbolAmRand{Semantik}%
	\wikicite{bib:Wikipedia}{
		\wikiBoldFt{Semantik} [\textdots], auch \wikiBoldFt{Bedeutungslehre}, nennt man die Theorie oder Wissenschaft von der Bedeutung der Zeichen. \wikiItalicFt{Zeichen} können hierbei beliebige \wikiLinkFt{Symbole} sein, insbesondere aber auch \wikiLinkFt{Sätze}, Satzteile, \wikiLinkFt{Wörter} oder \wikiLinkFt{Wortteile}.
	}
	In der \formalenMetasprache\ und der \Objektsprache\ sind die Zeichen die \Symbole\ und \Formeln.
}

\newVerweis         {\Signatur}{\glstext}{Signatur}
\longnewglossaryentry{Signatur}{
	name            ={Signatur \addIdx   {Signatur}},
	text            ={Signatur},
%%%	symbol          ={\ensuremath{Mathmode}},% ToDo=Mathmode
%%%	user6           ={Textmode},
	see             ={Abbildung,Logik,Praedikatenlogik,Sprache,Stelligkeit,Symbol},
}{\todoPruefen%
%%%	\SymbolAmRand{Signatur}%
	\wikicite{bib:Signatur}{
		In der \wikiLinkFt{mathematischen Logik} besteht eine \wikiBoldFt{Signatur} aus der \wikiLinkFt{Menge} der \wikiLinkFt{Symbole}, die in der betrachteten \wikiLinkFt{Sprache} zu den üblichen, rein logischen Symbolen hinzukommt, und einer \wikiLinkFt{Abbildung}, die jedem Symbol der Signatur eine \wikiLinkFt{Stelligkeit} eindeutig zuordnet. Während die logischen Symbole wie  $\forall ,\exists ,\land ,\lor ,\rightarrow ,\leftrightarrow ,\neg$ stets als „für alle“, „es gibt ein“, „und“, „oder“, „folgt“, „äquivalent zu“ bzw. „nicht“ interpretiert werden, können durch die semantische \wikiLinkFt{Interpretation} der Symbole der Signatur verschiedene \wikiLinkFt{Strukturen} (insbesondere Modelle von Aussagen der Logik) unterschieden werden. Die Signatur ist der spezifische Teil einer \wikiLinkFt{elementaren Sprache}.

		Beispielsweise lässt sich die gesamte \wikiLinkFt{Zermelo-Fraenkel-Mengenlehre} in der Sprache der \wikiLinkFt{Prädikatenlogik erster Stufe} und dem einzigen Symbol \MtsIn (neben den rein logischen Symbolen) formulieren; in diesem Fall ist die Symbolmenge der Signatur gleich $\{\MtsIn\}$.
	}
}

\newVerweis      {\BoolescheSignatur}{\glstext}  {BoolescheSignatur}
\newVerweis     {\BooleschenSignatur}{\glsuseri} {BoolescheSignatur}
\newglossaryentry {BoolescheSignatur}{
	name       =                {---, Boolesche \addIdx[
		name   =                {---, Boolesche},
		sort   =           {Signatur, Boolesche}]{BoolescheSignatur}},
	sort       =           {Signatur, Boolesche},
	text       ={Boolesche  Signatur},
	user1      ={Booleschen Signatur},
%%%	symbo      ={\ensuremath{Mathmode}},% ToDo=Mathmode
%%%	user6      ={Textmode},
	description={\todoPruefen%
%%%		\SymbolAmRand{BoolescheSignatur}%
		Die \logischeSignatur\ $\{\OjkNot, \OjkAnd, \OjkOr\}$.
	}
}

\newVerweis      {\logischeSignatur}  {\glstext} {logischeSignatur}
\newVerweis[en]  {\logischeSignaturen}{\glstext} {logischeSignatur}
\newVerweis     {\logischenSignatur}  {\glsuseri}{logischeSignatur}
\newglossaryentry {logischeSignatur}{
	name       =               {---, logische \addIdx[
		name   =               {---, logische},
		sort   =          {Signatur, logische}]  {logischeSignatur}},
	sort       =          {Signatur, logische},
	text       ={logische  Signatur},
	user1      ={logischen Signatur},
%%%	symbol     ={\ensuremath{Mathmode}},% ToDo=Mathmode
%%%	user6      ={Textmode},
	description={\todoPruefen%
%%%		\SymbolAmRand{logischeSignatur}%
		Abweichend von der Definition von \Signatur\ in \Wikipedia\ ist eine \GloFt{logische Signatur} eine \Teilmenge\ von \OjkJun, ausreichend um damit und mit \OjkVar\ und Klammerung alle anderen \Elemente\ aus \OjkJun\ zu definieren.
	}
}

\newVerweis     {\Sprache} {\glstext}{Sprache}
\newVerweis[n]  {\Sprachen}{\glstext}{Sprache}
\newglossaryentry{Sprache}{
	name        ={Sprache \addIdx    {Sprache}},
	text        ={Sprache},
%%%	symbol      ={\ensuremath{Mathmode}},% ToDo=Mathmode
%%%	user6       ={Textmode},
	description ={\todoPruefen%
%%%		\SymbolAmRand{Sprache}%
		--- Siehe \Formelmenge.
	}
}

\newVerweis      {\aussagenlogischeSprache}{\glstext}         {aussagenlogischeSprache}
\newVerweis     {\aussagenlogischenSprache}{\glsuseri}        {aussagenlogischeSprache}
\newglossaryentry {aussagenlogischeSprache}{
	name       =                      {---, aussagenlogische \addIdx[
		name   =                      {---, aussagenlogische},
		sort   =                  {Sprache, aussagenlogische}]{aussagenlogischeSprache}},
	sort       =                  {Sprache, aussagenlogische},
	text       ={aussagenlogische  Sprache},
	user1      ={aussagenlogischen Sprache},
%%%	symbol     ={\ensuremath{Mathmode}},% ToDo=Mathmode
%%%	user6      ={Textmode},
	description={\todoBeschreiben%
%%%		\SymbolAmRand{aussagenlogischeSprache}%
	}
}

\newVerweis     {\Sprachebene} {\glstext}{Sprachebene}
\newVerweis[n]  {\Sprachebenen}{\glstext}{Sprachebene}
\newglossaryentry{Sprachebene}{
	name        ={Sprachebene \addIdx    {Sprachebene}},
	text        ={Sprachebene},
%%%	symbol      ={\ensuremath{Mathmode}},% ToDo=Mathmode
%%%	user6       ={Textmode},
	description ={\todoOk%
%%%		\SymbolAmRand{Sprachebene}%
		Wir unterscheiden \hier\ drei \GloFt{Sprachebenen}: Die obere Ebene mit der \Metasprache, die mittlere mit der \formalenMetasprache\ und die untere mit der \Objektsprache.
		Mit einer \Sprache\ einer höheren Ebene kann man \textua\ \Aussagen\ über \Sprachen\ mit niedrigere Ebene treffen.
	}
}

\newVerweis     {\stellig}  {\glstext}{stellig}
\newVerweis[e]  {\stellige} {\glstext}{stellig}
\newVerweis[es] {\stelliges}{\glstext}{stellig}
\newVerweis[er] {\stelliger}{\glstext}{stellig}
\newglossaryentry{stellig}{
	name        ={$n$-stellig \addIdx[
		name    ={$n$-stellig},
		sort    ={stellig}]           {stellig}},
	sort        ={stellig},
	text        ={stellig},
%%%	symbol      ={\ensuremath{Mathmode}},% ToDo=Mathmode
%%%	user6       ={Textmode},
	see         ={MtsStelF,MtsStelR},
	description ={\todoPruefen%
%%%		\SymbolAmRand{stellig}%
		Eine \Funktion, \Relation\ oder ein \Praedikat\ mit der \Stelligkeit\ $n \MtsIn \MtsINo$ nennt man \GloFt{$n$-stellig}.
	}
}

\newVerweis     {\Stelligkeit}  {\glstext}{Stelligkeit}
\newVerweis[en] {\Stelligkeiten}{\glstext}{Stelligkeit}
\newglossaryentry{Stelligkeit}{
	name        ={Stelligkeit \addIdx     {Stelligkeit}},
	text        ={Stelligkeit},
%%%	symbol      ={\ensuremath{Mathmode}},% ToDo=Mathmode
%%%	user6       ={Textmode},
	see         ={MtsStelF,MtsStelR},
	description ={\todoPruefen%
%%%		\SymbolAmRand{Stelligkeit}%
		einer \Funktion, \Relation\ oder eines \Praedikats.
	}
}

\newVerweis         {\Symbol}  {\glstext} {Symbol}
\newVerweis[e]      {\Symbole} {\glstext} {Symbol}
\newVerweis[s]      {\Symbols} {\glstext} {Symbol}
\newVerweis[en]     {\Symbolen}{\glstext} {Symbol}
\longnewglossaryentry{Symbol}{
	name            ={Symbol \addIdx      {Symbol}},
	text            ={Symbol},
%%%	symbol          ={\ensuremath{Mathmode}},% ToDo=Mathmode
%%%	user6           ={Textmode},
	see             ={Beispielsymbol,Metasymbol,Obersymbol,Objektsymbol,Teilsymbol},
}{\todoPruefen%
%%%	\SymbolAmRand{Symbol}%
	Ein \GloFt{Symbol} ist eine \Kette\ von \atomarenSymbolen.
	Aus dem Zusammenhang muss hervorgehen, welche \atomarenSymbole\ gemeint sind.
	Die Darstellung kann auch mehrdimensional sein, \textzB\ mit Indizes.
	Prinzipiell kann eine mehrdimensionale Darstellung eines \gloFt{Symbols} aber auch eindimensional erfolgen.

	Ist ein \gloFt{Symbol} Teil eines anderen \gloFt{Symbols} (im Sinne der Darstellung), aber kein \Teilsymbol\ davon, so wird es (in diesem Zusammenhang) nicht als \gloFt{Symbol} interpretiert.
}


\newVerweis     {\aussagenlogischesSymbol}  {\glstext}       {aussagenlogischesSymbol}
\newVerweis[en] {\aussagenlogischenSymbolen}{\glsuseri}      {aussagenlogischesSymbol}
\newglossaryentry{aussagenlogischesSymbol}{
	name       =                       {---, aussagenlogisches \addIdx[
		name   =                       {---, aussagenlogisches},
		sort   =                  {Symbol, aussagenlogische}]{aussagenlogischesSymbol}},
	sort       =                  {Symbol, aussagenlogische},
	text       ={aussagenlogisches Symbol},
	user1      ={aussagenlogischen Symbol},
%%%	symbol     ={\ensuremath{Mathmode}},% ToDo=Mathmode
%%%	user6      ={Textmode},
	description={\todoErgaenzen%
%%%		\SymbolAmRand{aussagenlogischesSymbol}%
		Die \GloFt{aussagenlogischen} \Symbole\ sind ...
	}
}

\newVerweis     {\atomaresSymbol}  {\glstext}  {atomaresSymbol}
\newVerweis      {\atomareSymbol}  {\glsuseri} {atomaresSymbol}
\newVerweis[e]   {\atomareSymbole} {\glsuseri} {atomaresSymbol}
\newVerweis     {\atomarenSymbol}  {\glsuserii}{atomaresSymbol}
\newVerweis[e]  {\atomarenSymbole} {\glsuserii}{atomaresSymbol}
\newVerweis[en] {\atomarenSymbolen}{\glsuserii}{atomaresSymbol}
\newglossaryentry{atomaresSymbol}{
	name       =            {---, atomares \addIdx[
		name   =            {---, atomares},
		sort   =         {Symbol, atomares}]   {atomaresSymbol}},
	sort       =         {Symbol, atomares},
	text       ={atomares Symbol},
	user1      ={atomare  Symbol},
	user2      ={atomaren Symbol},
	user6      ={\{Textwort|typographisches Symbol\}},
	see        ={unzerlegbar,zerlegbar,zusammengesetzt},
	description={\todoOk%
		\TextAmRand{atomaresSymbol}%
		Ein \GloFt{atomares Symbol} ist ein \Textwort\ oder ein \typographischesSymbol.
		Aus dem Zusammenhang muss hervorgehen, welche \Textworte\ als \gloFt{\atomareSymbole} anzusehen sind.\footnote{%
			\textZB\ könnte "`sincos"' als ein \gloFt{atomares Symbol} oder als Aneinanderreihung der zwei \gloFt{atomaren Symbole} "`sin"' und "`cos"' verstanden werden.
		}
	}
}

\newVerweis     {\metasprachlichesSymbol} {\glstext}         {metasprachlichesSymbol}
\newVerweis      {\metasprachlicheSymbole}{\glspl}           {metasprachlichesSymbol}
\newglossaryentry{metasprachlichesSymbol}{
	name       =                     {---, metasprachliches \addIdx[
		name   =                     {---, metasprachliches},
		sort   =                  {Symbol, metasprachliches}]{metasprachlichesSymbol}},
	sort       =                  {Symbol, metasprachliches},
	text       ={metasprachliches Symbol},
	plural     ={metasprachliche  Symbole},
%%%	symbol     ={\ensuremath{Mathmode}},% ToDo=Mathmode
%%%	user6      ={Textmode},
	description={\todoBeschreiben%
%%%		\SymbolAmRand{metasprachlichesSymbol}%
	}
}

\newVerweis      {\typographischeSymbole}{\glspl}         {typographischesSymbol}
\newVerweis      {\TypographischeSymbole}{\Glspl}         {typographischesSymbol}
\newVerweis     {\typographischesSymbol} {\glstext}       {typographischesSymbol}
\newglossaryentry{typographischesSymbol}{
	name       =                   {---, typographisches \addIdx[
		name   =                   {---, typographisches},
		sort   =                {Symbol, typographisches}]{typographischesSymbol}},
	sort       =                {Symbol, typographisches},
	text       ={typographisches Symbol},
	plural     ={typographische  Symbole},
%%%	symbol     ={\ensuremath{Mathmode}},% ToDo=Mathmode
%%%	user6      ={Textmode},
	see        ={typographischesZeichen},
	description={\todoOk%
%%%		\SymbolAmRand{typographischesSymbol}%
		Ein sichtbares Zeichen aus den verwendeten Alphabeten (und damit kein Leerzeichen, Tab \textoae), aber kein Buchstabe.
	}
}

\newVerweis     {\zerlegbaresSymbol}{\glstext}    {zerlegbaresSymbol}
\newglossaryentry{zerlegbaresSymbol}{
	name       =               {---, zerlegbares \addIdx[
		name   =               {---, zerlegbares},
		sort   =            {Symbol, zerlegbares}]{zerlegbaresSymbol}},
	sort       =            {Symbol, zerlegbares},
	text       ={zerlegbares Symbol},
%%%	symbol     ={\ensuremath{Mathmode}},% ToDo=Mathmode
%%%	user6      ={Textmode},
	description={\todoBeschreiben%
%%%		\SymbolAmRand{zerlegbaresSymbol}%
		Ist ein \Symbol\ eine \Kette\ aus mehr als einem \Symbol, so ist es \DefFt{\zerlegbar} \OptFt{in diese \Symbole}.
		Aus \zerlegbar\ folgt für \Symbole\ auch \zusammengesetzt, aber nicht immer umgekehrt.
	}
}

\newVerweis     {\zusammengesetztesSymbol} {\glstext}         {zusammengesetztesSymbol}
\newVerweis      {\zusammengesetzteSymbole}{\glspl}           {zusammengesetztesSymbol}
\newglossaryentry{zusammengesetztesSymbol}{
	name       =                     {---, zusammengesetztes \addIdx[
		name   =                     {---, zusammengesetztes},
		sort   =                  {Symbol, zusammengesetztes}]{zusammengesetztesSymbol}},
	sort       =                  {Symbol, zusammengesetztes},
	text       ={zusammengesetztes Symbol},
	plural     ={zusammengesetzte  Symbole},
%%%	symbol     ={\ensuremath{Mathmode}},% ToDo=Mathmode
%%%	user6      ={Textmode},
	description={\todoBeschreiben%
%%%		\SymbolAmRand{zusammengesetztesSymbol}%
		Ist ein \Symbol\ ein \Textwort\ aus mehr als einem \Textbuchstaben, so ist es \DefFt{\zusammengesetzt} \OptFt{aus diesen \Textbuchstaben}.
		Aus \zerlegbar\ folgt für \Symbole\ auch \zusammengesetzt, aber nicht immer umgekehrt.
	}
}

\newVerweis     {\Symbolkette} {\glstext}{Symbolkette}
\newVerweis[n]  {\Symbolketten}{\glstext}{Symbolkette}
\newglossaryentry{Symbolkette}{
	name        ={Symbolkette \addIdx    {Symbolkette}},
	text        ={Symbolkette},
%%%	symbol      ={\ensuremath{Mathmode}},% ToDo=Mathmode
%%%	user6       ={Textmode},
	see         ={Zeichenkette},
	description ={\todoOk%
%%%		\SymbolAmRand{Symbolkette}%
		Eine \GloFt{Symbolkette} ist eine \Kette\ von \Symbolen.
		Zur Strukturierung und insbesondere zur Trennung von \Symbolen, wenn die Bedeutung sonst mehrdeutig ist, kann bei der Darstellung an verschiednen Stellen Zwischenraum (\textzB\ ein oder mehrere Leerzeichen) eingeschoben werden.
		Logisch gehört der Zwischenraum jedoch nicht zur \gloFt{\Symbolkette}.
	}
}

\newVerweis         {\Syntax} {\glstext}{Syntax}
\longnewglossaryentry{Syntax}{
	name            ={Syntax \addIdx    {Syntax}},
	text            ={Syntax},
%%%	symbol          ={\ensuremath{Mathmode}},% ToDo=Mathmode
%%%	user6           ={Textmode},
	see             ={Semantik,Sprache},
}{\todoGeprueft%
%%%	\SymbolAmRand{Syntax}%
	\wikicite{bib:Wikipedia}{
		Unter \wikiBoldFt{Syntax} [\textdots] versteht man allgemein ein Regelsystem zur Kombination elementarer Zeichen zu zusammengesetzten Zeichen in natürlichen oder künstlichen Zeichensystemen. Die Zusammenfügungsregeln der Syntax stehen hierbei den Interpretationsregeln der \wikiLinkFt{Semantik} gegenüber.
	}
	Wir nennen in der \formalenMetasprache\ und der \Objektsprache\ die elementaren Zeichen \Symbole\ und die zusammengesetzten Zeichen \Formeln.
}

%T === T === T === T === T === T === T === T === T === T === T === T === T === T

\newVerweis     {\Teilaussage} {\glstext} {Teilaussage}
\newVerweis        {\Taussage} {\glsuseri}{Teilaussage}
\newVerweis[n]  {\Teilaussagen}{\glstext} {Teilaussage}
\newglossaryentry{Teilaussage}{
	name        ={Teilaussage \addIdx     {Teilaussage}},
	text        ={Teilaussage},
	user1       =    {aussage},
%%%	symbol      ={\ensuremath{Mathmode}},% ToDo=Mathmode
%%%	user6       ={Textmode},
	description ={\todoOk%
%%%		\SymbolAmRand{Teilaussage}%
		Eine \Aussage\ $A$ heißt \GloFt{Teilaussage}\synonym{\defTxt{\Unteraussage}} \OptFt{von} einer \Aussage\ $B$, wenn sie Teil von $B$ ist und man sie ohne Bedeutungsänderung von $B$ dort klammern könnte.
	}
}

\newVerweis      {\echteTeilaussage}{\glstext}  {echteTeilaussage}
\newVerweis     {\echtenTeilaussage}{\glsuseri} {echteTeilaussage}
\newVerweis             {\eTaussage}{\glsuserii}{echteTeilaussage}
\newglossaryentry {echteTeilaussage}{
	name       =               {---, echte \addIdx[
		name   =               {---, echte},
		sort   =       {Teilaussage, echte}]    {echteTeilaussage}},
	sort       =       {Teilaussage, echte},
	text       ={echte  Teilaussage},
	user1      ={echten Teilaussage},
	user2      =           {aussage},
%%%	symbol     ={\ensuremath{Mathmode}},% ToDo=Mathmode
%%%	user6      ={Textmode},
	description={\todoOk%
%%%		\SymbolAmRand{echteTeilaussage}%
		Eine \Teilaussage\ $A$ einer \Aussage\ $B$ heißt \GloFt{echte Teilaussage} von $B$, wenn $A$ verschieden von $B$ ist.
	}
}

\newVerweis     {\Teilbereich} {\glstext} {Teilbereich}
\newVerweis[n]  {\Teilbereichn}{\glstext} {Teilbereich}
\newglossaryentry{Teilbereich}{
	name        ={Teilbereich \addIdx     {Teilbereich}},
	text        ={Teilbereich},
%%%	symbol      ={\ensuremath{Mathmode}},% ToDo=Mathmode
%%%	user6       ={Textmode},
	see         ={Oberbereich},
	description ={\todoOk%
%%%		\SymbolAmRand{Teilbereich}%
		Ein \Bereich\ $A$ ist genau dann ein \GloFt{Teilbereich} von einem \Bereich\ $B$, wenn $A \MtsSubsetEq B$ ist.
	}
}

\newVerweis     {\echterTeilbereich}{\glstext}  {echterTeilbereich}
\newVerweis     {\echtenTeilbereich}{\glsuseri} {echterTeilbereich}
\newVerweis             {\eTbereich}{\glsuserii}{echterTeilbereich}
\newglossaryentry{echterTeilbereich}{
	name       =               {---, echter \addIdx[
		name   =               {---, echter},
		sort   =       {Teilbereich, echter}]   {echterTeilbereich}},
	sort       =       {Teilbereich, echter},
	text       ={echter Teilbereich},
%%%	symbol     ={\ensuremath{Mathmode}},% ToDo=Mathmode
%%%	user6      ={Textmode},
	see        ={echterOberbereich},
	description={\todoOk%
%%%		\SymbolAmRand{echterTeilbereich}%
		Ein \Bereich\ $A$ ist genau dann ein \GloFt{echter Teilbereich} von einem \Bereich\ $B$, wenn $A \MtsSubset B$ ist.
	}
}

\newVerweis     {\Teilfolge} {\glstext} {Teilfolge}
\newVerweis        {\Tfolge} {\glsuseri}{Teilfolge}
\newVerweis[n]  {\Teilfolgen}{\glstext} {Teilfolge}
\newglossaryentry{Teilfolge}{
	name        ={Teilfolge \addIdx     {Teilfolge}},
	text        ={Teilfolge},
	user1       =    {folge},
%%%	symbol      ={\ensuremath{Mathmode}},% ToDo=Mathmode
%%%	user6       ={Textmode},
	description ={\todoBeschreiben%
%%%		\SymbolAmRand{Teilfolge}%
	}
}

\newVerweis      {\echteTeilfolge}{\glstext}  {echteTeilfolge}
\newVerweis     {\echtenTeilfolge}{\glsuseri} {echteTeilfolge}
\newVerweis             {\eTfolge}{\glsuserii}{echteTeilfolge}
\newglossaryentry {echteTeilfolge}{
	name       =             {---, echte \addIdx[
		name   =             {---, echte},
		sort   =       {Teilfolge, echte}]    {echteTeilfolge}},
	sort       =       {Teilfolge, echte},
	text       ={echte  Teilfolge},
	user1      ={echten Teilfolge},
	user2      =           {folge},
%%%	symbol     ={\ensuremath{Mathmode}},% ToDo=Mathmode
%%%	user6      ={Textmode},
	description={\todoBeschreiben%
%%%		\SymbolAmRand{echteTeilfolge}%
	}
}

\newVerweis     {\Teilformel} {\glstext} {Teilformel}
\newVerweis[n]  {\Teilformeln}{\glstext} {Teilformel}
\newVerweis        {\Tformel} {\glsuseri}{Teilformel}
\newglossaryentry{Teilformel}{
	name        ={Teilformel \addIdx     {Teilformel}},
	text        ={Teilformel},
	user1       =    {formel},
%%%	symbol      ={\ensuremath{Mathmode}},% ToDo=Mathmode
%%%	user6       ={Textmode},
	description ={\todoBeschreiben%
%%%		\SymbolAmRand{Teilformel}%
	}
}

\newVerweis      {\echteTeilformel}{\glstext}  {echteTeilformel}
\newVerweis     {\echtenTeilformel}{\glsuseri} {echteTeilformel}
\newVerweis             {\eTformel}{\glsuserii}{echteTeilformel}
\newglossaryentry {echteTeilformel}{
	name       =              {---, echte \addIdx[
		name   =              {---, echte},
		sort   =       {Teilformel, echte}]    {echteTeilformel}},
	sort       =       {Teilformel, echte},
	text       ={echte  Teilformel},
	user1      ={echten Teilformel},
	user2      =           {formel},
%%%	symbol     ={\ensuremath{Mathmode}},% ToDo=Mathmode
%%%	user6      ={Textmode},
	description={\todoBeschreiben%
%%%		\SymbolAmRand{echteTeilformel}%
	}
}

\newVerweis     {\Teilkette} {\glstext} {Teilkette}
%%%\newVerweis        {\Tkette} {\glsuseri}{Teilkette}
%%%\newVerweis[n]  {\Teilketten}{\glstext} {Teilkette}
\newglossaryentry{Teilkette}{
	name        ={Teilkette \addIdx     {Teilkette}},
	text        ={Teilkette},
	user1       =    {kette},
%%%	symbol      ={\ensuremath{Mathmode}},% ToDo=Mathmode
%%%	user6       ={Textmode},
	see         ={},% ToDo=Oberkette
	description ={\todoOk%
%%%		\SymbolAmRand{Teilkette}%
		Eine \Kette\ $A$ ist ist genau dann eine \GloFt{Teilkette} von einer \Kette\ $B$, wenn $A ein zusammenhängender Teil von B$ ist.
	}
}

\newVerweis      {\echteTeilkette}{\glstext}  {echteTeilkette}
%%%\newVerweis     {\echtenTeilkette}{\glsuseri} {echteTeilkette}
\newVerweis             {\eTkette}{\glsuserii}{echteTeilkette}
\newglossaryentry {echteTeilkette}{
	name       =             {---, echte \addIdx[
		name   =             {---, echte},
		sort   =       {Teilkette, echte}]    {echteTeilkette}},
	sort       =       {Teilkette, echte},
	text       ={echte  Teilkette},
	user1      ={echten Teilkette},
	user2      =           {kette},
%%%	symbol     ={\ensuremath{Mathmode}},% ToDo=Mathmode
%%%	user6      ={Textmode},
	see        ={},% ToDo=echteOberkette
	description={\todoOk%
%%%		\SymbolAmRand{echteTeilkette}%
		Eine \Kette\ $A$ ist ist genau dann eine \GloFt{echte Teilkette} von einer \Kette\ $B$, wenn $A$ \Teilkette\ von $B$, aber ungleich $B$ ist.
	}
}

\newVerweis     {\Teilmenge} {\glstext} {Teilmenge}
\newVerweis        {\Tmenge} {\glsuseri}{Teilmenge}
\newVerweis[n]  {\Teilmengen}{\glstext} {Teilmenge}
\newglossaryentry{Teilmenge}{
	name        ={Teilmenge \addIdx     {Teilmenge}},
	text        ={Teilmenge},
	user1       =    {menge},
%%%	symbol      ={\ensuremath{Mathmode}},% ToDo=Mathmode
%%%	user6       ={Textmode},
	see         ={Obermenge,Teilbereich},
	description ={\todoOk%
%%%		\SymbolAmRand{Teilmenge}%
		Eine \Menge\ $A$ ist ist genau dann eine \GloFt{Teilmenge} von einer \Menge\ $B$, wenn $A \MtsSubsetEq B$ ist.
	}
}

\newVerweis      {\echteTeilmenge}{\glstext}  {echteTeilmenge}
\newVerweis     {\echtenTeilmenge}{\glsuseri} {echteTeilmenge}
\newVerweis             {\eTmenge}{\glsuserii}{echteTeilmenge}
\newglossaryentry {echteTeilmenge}{
	name       =             {---, echte \addIdx[
		name   =             {---, echte},
		sort   =       {Teilmenge, echte}]    {echteTeilmenge}},
	sort       =       {Teilmenge, echte},
	text       ={echte  Teilmenge},
	user1      ={echten Teilmenge},
	user2      =           {menge},
%%%	symbol     ={\ensuremath{Mathmode}},% ToDo=Mathmode
%%%	user6      ={Textmode},
	see        ={echteObermenge,echterTeilbereich},
	description={\todoOk%
%%%		\SymbolAmRand{echteTeilmenge}%
		Eine \Menge\ $A$ ist ist genau dann eine \GloFt{echte Teilmenge} von einer \Menge\ $B$, wenn $A \MtsSubset B$ ist.
	}
}

\newVerweis     {\Teilobjekt} {\glstext} {Teilobjekt}
\newVerweis        {\Tobjekt} {\glsuseri}{Teilobjekt}
\newVerweis[e]  {\Teilobjekte}{\glstext} {Teilobjekt}
\newglossaryentry{Teilobjekt}{
	name        ={Teilobjekt \addIdx     {Teilobjekt}},
	text        ={Teilobjekt},
	user1       =    {objekt},
%%%	symbol      ={\ensuremath{Mathmode}},% ToDo=Mathmode
%%%	user6       ={Textmode},
	description ={\todoBeschreiben%
%%%		\SymbolAmRand{Teilobjekt}%
	}
}

\newVerweis     {\echtesTeilobjekt}{\glstext}  {echtesTeilobjekt}
\newVerweis     {\echtenTeilobjekt}{\glsuseri} {echtesTeilobjekt}
\newVerweis             {\eTobjekt}{\glsuserii}{echtesTeilobjekt}
\newglossaryentry{echtesTeilobjekt}{
	name        =              {---, echtes \addIdx[
		name    =              {---, echtes},
		sort    =       {Teilobjekt, echtes}]  {echtesTeilobjekt}},
	sort        =       {Teilobjekt, echtes},
	text        ={echtes Teilobjekt},
	user1       ={echten Teilobjekt},
	user2       =           {objekt},
%%%	symbol      ={\ensuremath{Mathmode}},% ToDo=Mathmode
%%%	user6       ={Textmode},
	description ={\todoBeschreiben%
%%%		\SymbolAmRand{echtesTeilobjekt}%
	}
}

\newVerweis     {\Teilsprache} {\glstext} {Teilsprache}
\newglossaryentry{Teilsprache}{
	name        ={Teilsprache \addIdx     {Teilsprache}},
	text        ={Teilsprache},
%%%	symbol      ={\ensuremath{Mathmode}},% ToDo=Mathmode
%%%	user6       ={Textmode},
	description ={\todoBeschreiben%
%%%		\SymbolAmRand{Teilsprache}%
	}
}

\newVerweis     {\echteTeilsprache}{\glstext}  {echteTeilsprache}
\newglossaryentry{echteTeilsprache}{
	name        =               {---, echte \addIdx[
		name    =               {---, echte},
		sort    =       {Teilsprache, echte}]  {echteTeilsprache}},
	sort        =       {Teilsprache, echte},
	text        ={echte Teilsprache},
%%%	symbol      ={\ensuremath{Mathmode}},% ToDo=Mathmode
%%%	user6       ={Textmode},
	description ={\todoBeschreiben%
%%%		\SymbolAmRand{echteTeilsprache}%
	}
}

\newVerweis     {\Teilsymbol} {\glstext} {Teilsymbol}
\newVerweis        {\Tsymbol} {\glsuseri}{Teilsymbol}
\newVerweis[e]  {\Teilsymbole}{\glstext} {Teilsymbol}
\newglossaryentry{Teilsymbol}{
	name        ={Teilsymbol \addIdx     {Teilsymbol}},
	text        ={Teilsymbol},
	user1       =    {symbol},
%%%	symbol      ={\ensuremath{Mathmode}},% ToDo=Mathmode
%%%	user6       ={Textmode},
	description ={\todoErgaenzen%
%%%		\SymbolAmRand{Teilsymbol}%
		Ein \Symbol\ $A$ sei eine \Kette\ von \atomarenSymbolen\ und $B$ eine \Teilkette\ davon.
		Wenn (in dem Zusammenhang) $B$ auch ein \Symbol\ ist, dann ist $B$ ein \GloFt{Teilsymbol} \OptFt{von} $A$.
		$A$ ist stets sein eigenes Teilsymbol.
%%%%		Sprechweise: $B aus A$
	}
}

\newVerweis     {\echtesTeilsymbol} {\glstext}  {echtesTeilsymbol}
\newVerweis     {\echtenTeilsymbol} {\glsuseri} {echtesTeilsymbol}
\newVerweis[s]  {\echtenTeilsymbols}{\glsuseri} {echtesTeilsymbol}
\newVerweis             {\eTsymbol} {\glsuserii}{echtesTeilsymbol}
\newglossaryentry{echtesTeilsymbol}{
	name       =              {---, echtes \addIdx[
		name   =              {---, echtes},
		sort   =       {Teilsymbol, echtes}]    {echtesTeilsymbol}},
	sort       =       {Teilsymbol, echtes},
	text       ={echtes Teilsymbol},
	user1      ={echten Teilsymbol},
	user2      =           {symbol},
%%%	symbol     ={\ensuremath{Mathmode}},% ToDo=Mathmode
%%%	user6      ={Textmode},
	see        ={Obersymbol},
	description={\todoBeschreiben%
%%%		\SymbolAmRand{echtesTeilsymbol}%
		Ein \Teilsymbol\ $A$ eines \Symbols\ $B$ welches nicht mit $B$ übereinstimmt, ist ein \GloFt{echtes Teilsymbol} \OptFt{von} $B$.
%%%%		Sprechweise: $A echt aus B$
	}
}

\newVerweis     {\Textbuchstabe} {\glstext}{Textbuchstabe}
\newVerweis[n]  {\Textbuchstaben}{\glstext}{Textbuchstabe}
\newglossaryentry{Textbuchstabe}{
	name        ={Textbuchstabe \addIdx    {Textbuchstabe}},
	text        ={Textbuchstabe},
%%%	symbol      ={\ensuremath{Mathmode}},% ToDo=Mathmode
%%%	user6       ={Textmode},
	see         ={deutscherBuchstabe,lateinischerBuchstabe,griechischerBuchstabe,Textwort},
	description ={\todoOk%
%%%		\SymbolAmRand{Textbuchstabe}%
		\footnote{\glqq Buchstabe\grqq\ allein ist nicht genau genug.}
		Ein Buchstabe aus einem der verwendeten\footnote{%
			Wir verwenden hier nur das lateinische, das deutsche und das griechische Alphabet.
			Denkbar wären auch weitere nationale Erweiterungen und \textzB\ das kyrillische Alphabet.
		} Alphabete.
		\begin{itemize}
			\item Klein- und Großbuchstaben gelten als verschieden.
			\item \GloFt{Textbuchstaben} in verschiedenen Schriftarten gelten als verschieden
			\item \gloFt{Textbuchstaben} in verschiedenen Textauszeichnungen gelten nicht als verschieden.
			Die unterschiedlichen Textauszeichnungen können aber eine Bedeutung haben.
		\end{itemize}
	}
}

\newVerweis     {\Textwort} {\glstext}{Textwort}
\newVerweis[e]  {\Textworte}{\glstext}{Textwort}
\newglossaryentry{Textwort}{
	name        ={Textwort \addIdx   {Textwort}},
	text        ={Textwort},
%%%	symbol      ={\ensuremath{Mathmode}},% ToDo=Mathmode
%%%	user6       ={Textmode},
	see         ={deutschesWort,lateinischesWort,griechischesWort,Textbuchstabe},
	description ={\todoOk%
%%%		\SymbolAmRand{Textwort}%
		\footnote{%
			\glqq Wort\grqq\ allein ist nicht genau genug.
			Man denke an gemischte Alphabete, die Verwendung von Bindestrichen und die in Programmiersprachen übliche Verwendung von Unterstrichen.
		} Eine \Kette\ von \Textbuchstaben, alle aus demselben Alphabet und in derselben Schriftart.
		Normalerweise hat ein \gloFt{Textwort} als Ganzes dieselbe Textauszeichnung.
	}
}

\newVerweis     {\Traegermenge} {\glstext}{Traegermenge}
\newVerweis[n]  {\Traegermengen}{\glstext}{Traegermenge}
\newglossaryentry{Traegermenge}{
	name        ={Trägermenge \addIdx[
		name    ={Trägermenge}]           {Traegermenge}},
	text        ={Trägermenge},
%%%	symbol      ={\ensuremath{Mathmode}},% ToDo=Mathmode
%%%	user6       ={Textmode},
	see         ={MtsTraeger},
	description ={\todoPruefen%
%%%		\SymbolAmRand{Traegermenge}%
		einer \Relation.
	}
}

\newVerweis     {\Transformation}  {\glstext}{Transformation}
\newVerweis[en] {\Transformationen}{\glstext}{Transformation}
\newglossaryentry{Transformation}{
	name        ={Transformation \addIdx     {Transformation}},
	text        ={Transformation},
%%%	symbol      ={\ensuremath{Mathmode}},% ToDo=Mathmode
%%%	user6       ={Textmode},
	see         ={MtsTransformation,MtsTransformationTup,zulaessigeTransformation},
	description ={\todoPruefen%
%%%		\SymbolAmRand{Transformation}%
		Eine Umformung oder Erzeugung einer \Formel\ aus einer vorgegebenen \Menge\ von \Formeln, \textdh\ die Anwendung einer \Schlussregel.
	}
}

\newVerweis      {\zulaessigeTransformation}  {\glstext}  {zulaessigeTransformation}
\newVerweis[en]  {\zulaessigeTransformationen}{\glstext}  {zulaessigeTransformation}
\newVerweis     {\zulaessigenTransformation}  {\glsuseri} {zulaessigeTransformation}
\newVerweis[en] {\zulaessigenTransformationen}{\glsuseri} {zulaessigeTransformation}
\newVerweis[en] {\zulaessigerTransformationen}{\glsuserii}{zulaessigeTransformation}
\newglossaryentry{zulaessigeTransformation}{
	name        =                      {---, zulässige \addIdx[
		name    =                      {---, zulässige},
		sort    =           {Transformation, zulässige}]  {zulaessigeTransformation}},
	sort        =           {Transformation, zulässige},
	text        ={zulässige  Transformation},
	user1       ={zulässigen Transformation},
	user2       ={zulässiger Transformation},
%%%	symbol      ={\ensuremath{Mathmode}},% ToDo=Mathmode
%%%	user6       ={Textmode},
	description ={\todoPruefen%
%%%		\SymbolAmRand{zulaessigeTransformation}%
		Eine \Transformation\ heißt \GloFt{zulässig}, wenn sie \Element\ aus einer vorgegebenen \Menge\ von \Transformationen\ oder eine daraus zulässigerweise abgeleitete \Transformation\ ist.
	}
}

\newVerweis     {\Transformationsfolge} {\glstext}{Transformationsfolge}
\newVerweis[n]  {\Transformationsfolgen}{\glstext}{Transformationsfolge}
\newglossaryentry{Transformationsfolge}{
	name        ={Transformationsfolge \addIdx    {Transformationsfolge}},
	text        ={Transformationsfolge},
%%%	symbol      ={\ensuremath{Mathmode}},% ToDo=Mathmode
%%%	user6       ={Textmode},
	see         ={MtsTransformation,MtsTransformationTup,Transformation},
	description ={\todoPruefen%
%%%		\SymbolAmRand{Transformationsfolge}%
		Eine Folge von \Transformationen.
	}
}

\newVerweis     {\Transformationsregel} {\glstext}{Transformationsregel}
\newVerweis[n]  {\Transformationsregeln}{\glstext}{Transformationsregel}
\newglossaryentry{Transformationsregel}{
	name        ={Transformationsregel \addIdx    {Transformationsregel}},
	text        ={Transformationsregel},
%%%	symbol      ={\ensuremath{Mathmode}},% ToDo=Mathmode
%%%	user6       ={Textmode},
	description ={\todoBeschreiben%
%%%		\SymbolAmRand{Transformationsregel}%
	}
}

\newVerweis         {\Tupel} {\glstext}{Tupel}
\newVerweis[s]      {\Tupels}{\glstext}{Tupel}
\longnewglossaryentry{Tupel}{
	name            ={Tupel \addIdx    {Tupel}},
	text            ={Tupel},
%%%	symbol          ={\ensuremath{Mathmode}},% ToDo=Mathmode
%%%	user6           ={Textmode},
	see             ={Folge,Komponente,Menge,Objekt,Symbolkette,Zeichenkette},
}{\todoPruefen%
%%%	\SymbolAmRand{Tupel}%
	\wikicite{bib:Tupel}{
		\wikiBoldFt{Tupel} (abgetrennt von \wikiLinkFt{mittellat.} \wikiItalicFt{quintuplus} ‚fünffach‘, \wikiItalicFt{septuplus} ‚siebenfach‘, \wikiItalicFt{centuplus} ‚hundertfach‘ etc.) sind in der \wikiLinkFt{Mathematik} neben \wikiLinkFt{Mengen} eine wichtige Art und Weise, \wikiLinkFt{mathematische Objekte} zusammenzufassen. Ein Tupel besteht aus einer \wikiLinkFt{Liste} endlich vieler, nicht notwendigerweise voneinander verschiedener Objekte. Dabei spielt, im Gegensatz zu Mengen, die Reihenfolge der Objekte eine Rolle. Es gibt verschiedene Möglichkeiten, Tupel formal als Mengen darzustellen. Tupel finden in vielen Bereichen der Mathematik Verwendung, zum Beispiel als \wikiLinkFt{Koordinaten} von Punkten oder als \wikiLinkFt{Vektoren} in mehrdimensionalen \wikiLinkFt{Vektorräumen}.

		Von Tupeln unabhängig von ihrer Länge ist selten die Rede. Vielmehr verwendet man das Wort \wikiBoldFt{$n$-Tupel} und die im nächsten Abschnitt genannten Spezialfälle davon dann, wenn sich aus dem Zusammenhang die Länge als feste Zahl oder als benannte Konstante wie $n$ ergibt. Betrachtet man dagegen viele endliche Folgen unterschiedlicher Längen von Elementen einer Grundmenge, spricht man von endlichen Folgen oder definiert einen neuen Begriff, der oft mit „Kette“ zusammengesetzt ist, z. B. \wikiLinkFt{Zeichenkette}, \wikiLinkFt{Additionskette}.
	}
	Ein \GloFt{$n$-Tupel}\alternativi{Vektor} $\vec{a}$ ist eine endliche \Folge\alternativi{Sequenz} $(a_1, \dots, a_n)$ \DefFt{von} seinen \DefFt{Komponenten} $a_i$.
	Sind alle Komponenten \Elemente\ aus derselben \Menge\ $M$, so heißt $\vec{a}$ ein $n$-\Tupel\ \DefFt{auf} $M$.
}

\newVerweis     {\Tupelmenge} {\glstext}{Tupelmenge}
\newVerweis[n]  {\Tupelmengen}{\glstext}{Tupelmenge}
\newglossaryentry{Tupelmenge}{
	name        ={Tupelmenge \addIdx    {Tupelmenge}},
	text        ={Tupelmenge},
%%%	symbol      ={\ensuremath{Mathmode}},% ToDo=Mathmode
%%%	user6       ={Textmode},
	description ={\todoPruefen%
%%%		\SymbolAmRand{Tupelmenge}%
		Die \Tupelmenge\ $\MtsTup(M)$ einer \Menge\ $M$ ist die \Menge\ aller $n$-Tupel aus $M^n$ für alle $n \in \MtsINo$.
	}
}

%U === U === U === U === U === U === U === U === U === U === U === U === U === U

\newVerweis     {\Umkehrrelation}  {\glstext}{Umkehrrelation}
\newVerweis[en] {\Umkehrrelationen}{\glstext}{Umkehrrelation}
\newglossaryentry{Umkehrrelation}{
	name        ={Umkehrrelation \addIdx     {Umkehrrelation}},
	text        ={Umkehrrelation},
%%%	symbol      ={\ensuremath{Mathmode}},% ToDo=Mathmode
%%%	user6       ={Textmode},
	see         ={Menge},
	description ={\todoPruefen%
%%%		\SymbolAmRand{Umkehrrelation}%
		Die \GloFt{Umkehrrelation}\alternativiii{konverse Relation}{Konverse }{inverse Relation } \emph{von} einer \binaeren\ \Relation\ $(G,A,B)$ ist die \Relation\ $(H,B,A)$ mit $H = \RawMengeDef{(b,a)}{(a,b) \in G}$.
		Üblicherweise wird das zugehörige \Relationssymbol\ gespiegelt.
		Die \gloFt{Umkehrrelation} der \Negation\ einer \Relation\ ist gleich der \Negation\ ihrer \gloFt{Umkehrrelation}.
	}
}

\newVerweis     {\unaer}  {\glstext}{unaer}
\newVerweis[e]  {\unaere} {\glstext}{unaer}
\newVerweis[en] {\unaeren}{\glstext}{unaer}
\newVerweis[er] {\unaerer}{\glstext}{unaer}
\newglossaryentry{unaer}{
	name        ={unär \addIdx[
		name    ={unär}]            {unaer}},
	text        ={unär},
%%%	symbol      ={\ensuremath{Mathmode}},% ToDo=Mathmode
%%%	user6       ={Textmode},
	see         ={binaer},
	description ={\todoPruefen%
%%%		\SymbolAmRand{unaer}%
		Eine \Operation, \Funktion\ oder \Relation\ heißt \GloFt{unär}, wenn ihre \Stelligkeit\ gleich 1 ist.
	}
}

\newVerweis     {\Ungleichheit}{\glstext}{Ungleichheit}
\newglossaryentry{Ungleichheit}{
	name        ={Ungleichheit \addIdx   {Ungleichheit}},
	text        ={Ungleichheit},
%%%	symbol      ={\ensuremath{Mathmode}},% ToDo=Mathmode
%%%	user6       ={Textmode},
	description ={\todoPruefen%
%%%		\SymbolAmRand{Ungleichheit}%
		Eine \Gleichheitsrelation:
		Zwei Objekte $A$ und $B$ sind \DefFt{nicht gleich}\alternativii{nicht dasselbe}{nicht identisch} $A \MtsEqN B$, wenn sie in mindestens einer \interessierendenEigenschaft\ für \MtsEq\ nicht übereinstimmen.
	}
}

\newsynonym{\Unteraussage}{Unteraussage}{\Teilaussage}
\newsynonym{\Unterformel} {Unterformel} {\Teilformel}
\newsynonym{\Untermenge}  {Untermenge}  {\Teilmenge}
\newsynonym{\Unterobjekt} {Unterobjekt} {\Teilobjekt}
\newsynonym{\Untersymbol} {Untersymbol} {\Teilsymbol}

\newsynonym{\unzerlegbar}{unzerlegbar}{\atomar}

%V === V === V === V === V === V === V === V === V === V === V === V === V === V

\newVerweis         {\Variable} {\glstext}{Variable}
\newVerweis[n]      {\Variablen}{\glstext}{Variable}
\longnewglossaryentry{Variable}{
	name            ={Variable \addIdx    {Variable}},
	text            ={Variable},
	see             ={Konstante},
}{\todoGeprueft%
%%%	\SymbolAmRand{Variable}%
	\wikicite{bib:Variable}{
		Eine \wikiBoldFt{Variable} ist ein Name für eine Leerstelle in einem logischen oder mathematischen Ausdruck. Der Begriff leitet sich vom lateinischen \wikiLinkFt{Adjektiv} \wikiItalicFt{variabilis} (veränderlich) ab. Gleichwertig werden auch die Begriffe \wikiItalicFt{Platzhalter} oder \wikiItalicFt{Veränderliche} benutzt. Als „Variable“ dienten früher Wörter oder Symbole, heute verwendet man zur \wikiLinkFt{mathematischen Notation} in der Regel Buchstaben als Zeichen. Wird anstelle der Variablen ein konkretes Objekt eingesetzt, so ist darauf zu achten, dass überall dort, wo die Variable auftritt, auch dasselbe Objekt benutzt wird.
	}
}

\newVerweis      {\aussagenlogischeVariable} {\glstext}         {aussagenlogischeVariable}
\newVerweis[n]  {\aussagenlogischenVariablen}{\glsuseri}        {aussagenlogischeVariable}
\newVerweis     {\aussagenlogischenV}        {\glsuserii}       {aussagenlogischeVariable}
\newglossaryentry {aussagenlogischeVariable}{
	name       =                       {---, aussagenlogische \addIdx[
		name   =                       {---, aussagenlogische},
		sort   =                  {Variable, aussagenlogische}] {aussagenlogischeVariable}},
	sort       =                  {Variable, aussagenlogische},
	text       ={aussagenlogische  Variable},
	user1      ={aussagenlogischen Variable},
	user2      ={aussagenlogischen},
%%%	symbol     ={\ensuremath{Mathmode}},% ToDo=Mathmode
%%%	user6      ={Textmode},
	description={\todoPruefen%
%%%		\SymbolAmRand{aussagenlogischeVariable}%
		Die \GloFt{aussagenlogischen} \Variablen\ sind die \Elemente\ aus \OjkVar.
	}
}

\newVerweis      {\freieVariable} {\glstext}  {freieVariable}
\newVerweis[n]   {\freieVariablen}{\glstext}  {freieVariable}
\newVerweis[n]  {\freienVariablen}{\glsuseri} {freieVariable}
\newVerweis       {\freiV}        {\glsuserii}{freieVariable}
\newVerweis[e]   {\freieV}        {\glsuserii}{freieVariable}
\newglossaryentry {freieVariable}{
	name       =            {---, freie \addIdx[
		name   =            {---, freie},
		sort   =       {Variable, freie}]{freieVariable}},
	sort       =       {Variable, freie},
	text       ={freie  Variable},
	user1      ={freien Variable},
	user2      ={frei},
%%%	symbol     ={\ensuremath{Mathmode}},% ToDo=Mathmode
%%%	user6      ={Textmode},
	description={\todoGeprueft%
%%%		\SymbolAmRand{freieVariable}%
		Eine \Variable\ heißt \GloFt{frei}, wenn sie nicht \gebundenV\ ist.
	}
}

\newVerweis      {\gebundeneVariable} {\glstext}  {gebundeneVariable}
\newVerweis[n]   {\gebundeneVariablen}{\glstext}  {gebundeneVariable}
\newVerweis[n]  {\gebundenenVariablen}{\glsuseri} {gebundeneVariable}
\newVerweis       {\gebundenV}        {\glsuserii}{gebundeneVariable}
\newVerweis[e]   {\gebundeneV}        {\glsuserii}{gebundeneVariable}
\newglossaryentry {gebundeneVariable}{
	name       =            {---, gebundene \addIdx[
		name   =            {---, gebundene},
		sort   =       {Variable, gebundene}]{gebundeneVariable}},
	sort       =       {Variable, gebundene},
	text       ={gebundene  Variable},
	user1      ={gebundenen Variable},
	user2      ={gebunden},
%%%	symbol     ={\ensuremath{Mathmode}},% ToDo=Mathmode
%%%	user6      ={Textmode},
	description={\todoGeprueft%
%%%		\SymbolAmRand{gebundeneVariable}%
		Eine \Variable\ heißt durch einen bestimmten \Quantor\ \GloFt{gebunden}, wenn sie die zum \Quantor\ gehörige \Variable\ ist und im zugehörigen \Ausdruck\ auch \freiV\ vorkommt.
	}
}

\newVerweis      {\logischeVariable} {\glstext} {logischeVariable}
\newVerweis[n]   {\logischeVariablen}{\glstext} {logischeVariable}
\newVerweis      {\logischeV}        {\glsuseri}{logischeVariable}
\newglossaryentry {logischeVariable}{
	name       =               {---, logische \addIdx[
		name   =               {---, logische},
		sort   =          {Variable, logische}] {logischeVariable}},
	sort       =          {Variable, logische},
	text       ={logische  Variable},
	user1      ={logische},
%%%	symbol     ={\ensuremath{Mathmode}},% ToDo=Mathmode
%%%	user6      ={Textmode},
	description={\todoPruefen%
%%%		\SymbolAmRand{logischeVariable}%
		Die \GloFt{logischen} \Variablen\ entsprechen den \aussagenlogischenV.
	}
}

\newVerweis     {\metasprachlicheVariable}{\glstext}        {metasprachlicheVariable}
\newVerweis     {\metasprachlicheV}       {\glsuseri}       {metasprachlicheVariable}
\newglossaryentry{metasprachlicheVariable}{
	name       =                     {---, metasprachliche \addIdx[
		name   =                     {---, metasprachliche},
		sort   =                {Variable, metasprachliche}]{metasprachlicheVariable}},
	sort       =                {Variable, metasprachliche},
	text       ={metasprachliche Variable},
	user1      ={metasprachliche},
%%%	symbol     ={\ensuremath{Mathmode}},% ToDo=Mathmode
%%%	user6      ={Textmode},
	description={\todoErgaenzen%
%%%		\SymbolAmRand{metasprachlicheVariable}%
		Die \GloFt{metasprachlichen} \Variablen\ sind die \Elemente\ aus ...
	}
}

\newVerweis     {\Vereinigung} {\glstext}{Vereinigung}
\newglossaryentry{Vereinigung}{
	name        ={Vereinigung \addIdx    {Vereinigung}},
	text        ={Vereinigung},
%%%	symbol      ={\ensuremath{Mathmode}},% ToDo=Mathmode
%%%	user6       ={Textmode},
	description ={\todoErgaenzen%
%%%		\SymbolAmRand{Vereinigung}%
		Eine \Bereichsoperation:
	}
}

\newVerweis     {\vergleichbar} {\glstext}{vergleichbar}
\newVerweis     {\Vergleichbar} {\Glstext}{vergleichbar}
\newVerweis[e]  {\vergleichbare}{\glstext}{vergleichbar}
\newglossaryentry{vergleichbar}{
	name        ={vergleichbar \addIdx    {vergleichbar}},
	text        ={vergleichbar},
%%%	symbol      ={\ensuremath{Mathmode}},% ToDo=Mathmode
%%%	user6       ={Textmode},
	description ={\todoPruefen% ToDo=Wert und Ergebnis definieren
%%%		\SymbolAmRand{vergleichbar}%
		Zwei \Objekte\ $A$ und $B$ sind \vergleichbar, wenn beide von derselben \Objektart\ sind, \textdh\ wenn beide \textzB\ jeweils Mengen, \Symbolketten, Zahlen, \textusw\ sind.
		Dabei muss bei \Formeln\ zwischen der \Formel\ an sich und ihrem \emph{Wert} oder \emph{Ergebnis} unterschieden werden.
	}
}

\newVerweis     {\Verkettung} {\glstext}{Verkettung}
\newglossaryentry{Verkettung}{
	name        ={Verkettung \addIdx    {Verkettung}},
	text        ={Verkettung},
%%%	symbol      ={\ensuremath{Mathmode}},% ToDo=Mathmode
%%%	user6       ={Textmode},
	description ={\todoBeschreiben%
%%%		\SymbolAmRand{Verkettung}%
	}
}

\newVerweis     {\Vertauschung}  {\glstext}{Vertauschung}
\newVerweis[en] {\Vertauschungen}{\glstext}{Vertauschung}
\newglossaryentry{Vertauschung}{
	name        ={Vertauschung \addIdx     {Vertauschung}},
	text        ={Vertauschung},
%%%	symbol      ={\ensuremath{Mathmode}},% ToDo=Mathmode
%%%	user6       ={Textmode},
	description ={\todoPruefen%
%%%		\SymbolAmRand{Vertauschung}%
		Die \GloFt{Vertauschung} von zwei unabhängigen Teil-\Formeln\ ($\alpha$ und $\beta$) in einer anderen \Formel\ ($\gamma$)
		\\--- Formal: $\gamma(\alpha \MtsSwap \beta)$.
		Die \gloFt{Vertauschung} ist eine spezielle Form der \Ersetzung.
	}
}

\newVerweis[en]{\Voraussetzungen}{\glstext}{Voraussetzung}
\newsynonym    {\Voraussetzung}            {Voraussetzung}{\Praemisse}

\newVerweis     {\Vorgaenger}{\glstext}{Vorgaenger}
\newglossaryentry{Vorgaenger}{
	name        ={Vorgänger \addIdx    {Vorgaenger}},
	text        ={Vorgänger},
%%%	symbol      ={\ensuremath{Mathmode}},% ToDo=Mathmode
%%%	user6       ={Textmode},
	see         ={Anfangsglied,Endglied,Nachfolger,Zwischenglied},
	description ={\todoOk%
%%%		\SymbolAmRand{Vorgaenger}%
		Der \GloFt{Vorgänger} eines \Kettenglieds\ in einer \Kette\ ist das vorhergehende (links stehende) \Kettenglied\ in der \Kette.
	}
}

%W === W === W === W === W === W === W === W === W === W === W === W === W === W

\newVerweis         {\Wahrheitswert}  {\glstext}{Wahrheitswert}
\newVerweis[e]      {\Wahrheitswerte} {\glstext}{Wahrheitswert}
\newVerweis[en]     {\Wahrheitswerten}{\glstext}{Wahrheitswert}
\longnewglossaryentry{Wahrheitswert}{
	name            ={Wahrheitswert \addIdx     {Wahrheitswert}},
	text            ={Wahrheitswert},
%%%	symbol          ={\ensuremath{Mathmode}},% ToDo=Mathmode
%%%	user6           ={Textmode},
	see             ={atomar,Aussage,Element,Junktor,Logik,Satz,Teilaussage},
}{\todoOk%
%%%	\SymbolAmRand{Wahrheitswert}%
	\wikicite{bib:Wahrheitswert}{
		Ein \wikiBoldFt{Wahrheitswert} ist in \wikiLinkFt{Logik} und \wikiLinkFt{Mathematik} ein \wikiItalicFt{logischer Wert}, den eine Aussage in Bezug auf Wahrheit annehmen kann.

		In der zweiwertigen \wikiLinkFt{klassischen Logik} kann eine Aussage nur entweder \wikiItalicFt{wahr} oder \wikiItalicFt{falsch} sein, die Menge der Wahrheitswerte $\{W, F\}$ hat so zwei Elemente. In \wikiLinkFt{mehrwertigen Logiken} enthält die \wikiLinkFt{Wahrheitswertemenge} mehr als zwei Elemente, z. B. in einer \wikiLinkFt{dreiwertigen Logik} oder einer \wikiLinkFt{Fuzzy-Logik}, die damit zu den \wikiLinkFt{nichtklassischen} zählen. Hier wird dann auch neben Wahrheitswerten von \wikiItalicFt{Quasiwahrheitswerten}, \wikiItalicFt{Pseudowahrheitswerten} oder \wikiItalicFt{Geltungswerten} gesprochen.

		Die Abbildung der Menge von Aussagen einer (meist formalen) Sprache auf die Wahrheitswertemenge wird \wikiLinkFt{Wahrheitswertzuordnung}  genannt und ist eine aussagenlogisch spezifische \wikiLinkFt{Bewertungsfunktion}. In der klassischen Logik kann auch explizit die Klasse aller wahren Aussagen beziehungsweise die Klasse aller falschen Aussagen definiert werden. Die Abbildung von Wahrheitswerten der (\wikiLinkFt{atomaren}) Teilaussagen einer zusammengesetzten Aussage auf die Wahrheitswertemenge heißt \wikiLinkFt{Wahrheitswertefunktion} oder Wahrheitsfunktion. Die Wertetabelle dieser \wikiLinkFt{Funktion} im mathematischen Sinn wird auch als \wikiLinkFt{Wahrheitstafel} bezeichnet und häufig dazu verwendet, die Bedeutung wahrheitsfunktionaler \wikiLinkFt{Junktoren} anzugeben.
	}
	\nurImGlossar{
		Wir verwenden nur die beiden \GloFt{Wahrheitswerte} der zweiwertigen klassischen \Logik, die wir (in der \Metasprache) mit \chrqt{\TxtTrue} und \chrqt{\TxtFalse} bezeichnen.
		In der \formalenMetasprache\ hingegen verwenden wir \chrqt{\MtsTrue} und \chrqt{\MtsFalse} und in der \Objektsprache\ \chrqt{\OjkTrue} und \chrqt{\OjkFalse}.
		In der Literatur findet man auch einfach \chrqt{$1$} und \chrqt{$0$}.

		Ist statt Wahrheit nur Beweisbarkeit von Interesse, so gelangt man zum Intuitionismus, in dem der Satz vom ausgeschlossenen Dritten\citenote{bib:TertiumNonDatur} nicht gilt.

		\wikiciteChapter{bib:Intuitionismus}{1}{
			Die Wahrheit eines mathematischen Satzes wird im Intuitionismus bezogen auf die Möglichkeit, einen entsprechenden Beweis zu formulieren. Wahrheit entsteht also erst durch die Verifizierung. Wahre Sätze oder von ihnen beschriebene Objekte haben keine Existenz unabhängig von tatsächlichen Denkprozessen. Dies steht im Kontrast unter anderem zum sog. \wikiLinkFt{Platonismus} in der Philosophie der Mathematik.
		}
	}
	\nichtImGlossar{
		Ein \GloFt{Wahrheitswert} ist ein \Wert, den eine \Aussage\ in Bezug auf Wahrheit annehmen kann.
		Für die \Darstellung\ der \Wahrheitswerte\ abhängig von der \Sprachebene\ und dem logischen Wert der Aussage definieren wir:
		\begin{table}[H]
			\centering
			\begin{threeparttable}
				\setlength\extrarowheight{3pt}
				\begin{tabularx}{10cm}{l@{\extracolsep{\fill}}|cc|c|}
					& \multicolumn{2}{c|}{ Aussagewert } & \\
					\TabFt{\Sprachebene} & \TabFt{wahr} & \TabFt{falsch} & \TabFt{Symbolart} \\
					\hline
					\Metasprache          & \TxtTrue & \TxtFalse & normaler Text \\
					\formaleMetasprache~~ & \MtsTrue & \MtsFalse & \Metasymbol   \\
					\Objektsprache        & \OjkTrue & \OjkFalse & \Objektsymbol \\
					\hline
				\end{tabularx}
			\end{threeparttable}
			\caption{\Darstellung\ der \Wahrheitswerte}
			\label{tab:Wahrheitswerte}% Erst nach '\caption'!
		\end{table}
		Wir schließen nicht aus, dass es weitere \gloFt{Wahrheitswerte} gibt.
	}
}

\newVerweis     {\aussagenlogischerWahrheitswert}{\glstext}          {aussagenlogischerWahrheitswert}
\newglossaryentry{aussagenlogischerWahrheitswert}{
	name       =                            {---, aussagenlogischer \addIdx[
		name   =                            {---, aussagenlogischer},
		sort   =                  {Wahrheitswert, aussagenlogischer}]{aussagenlogischerWahrheitswert}},
	sort       =                  {Wahrheitswert, aussagenlogischer},
	text       ={aussagenlogischer Wahrheitswert},
%%%	symbol     ={\ensuremath{Mathmode}},% ToDo=Mathmode
%%%	user6      ={Textmode},
	description={\todoGeprueft%
%%%		\SymbolAmRand{aussagenlogischerWahrheitswert}%
		Es gib nur die beiden \GloFt{aussagenlogischen \Wahrheitswerte} \OjkTrue\ und \OjkFalse.
	}
}

\newVerweis     {\metasprachlicherWahrheitswert}{\glstext}         {metasprachlicherWahrheitswert}
\newVerweis      {\metasprachlicheWahrheitswert}{\glsuseri}        {metasprachlicherWahrheitswert}
\newglossaryentry{metasprachlicherWahrheitswert}{
	name       =                           {---, metasprachlicher \addIdx[
		name   =                           {---, metasprachlicher},
		sort   =                 {Wahrheitswert, metasprachlicher}]{metasprachlicherWahrheitswert}},
	sort       =                 {Wahrheitswert, metasprachlicher},
	text       ={metasprachlicher Wahrheitswert},
	user1      ={metasprachliche  Wahrheitswert},
%%%	symbol     ={\ensuremath{Mathmode}},% ToDo=Mathmode
%%%	user6      ={Textmode},
	description={\todoPruefen%
%%%		\SymbolAmRand{metasprachlicherWahrheitswert}%
		Es gib die beiden \GloFt{metasprachlichen \Wahrheitswerte} in Textform (\TxtTrue, \TxtFalse) und in der \formalenMetasprache\ (\MtsTrue, \MtsFalse).
	}
}

\newVerweis     {\Wert}  {\glstext}{Wert}
\newVerweis[en] {\Werten}{\glstext}{Wert}
\newglossaryentry{Wert}{
	name        ={Wert \addIdx     {Wert}},
	text        ={Wert},
%%%	symbol      ={\ensuremath{Mathmode}},% ToDo=Mathmode
%%%	user6       ={Textmode},
	description ={\todoGeprueft%
%%%		\SymbolAmRand{Wert}%
		Der \GloFt{Wert} einer \Formel\ ergibt sich rekursiv aus der \Belegung\ der \Symbole, aus denen die \Formel\ besteht.
		Beispielsweise hat die \Formel\ \seqqt{a+b=c} mit der \Belegung\ von \chrqt{$a$}, \chrqt{$b$}, \chrqt{$c$}, \chrqt{$+$} und \chrqt{$=$} durch die Zahlen Eins, Zwei und Drei, den Additionsoperator und die Gleichheit den \gloFt{Wert} \glqq\TxtTrue\grqq.%
		\footnote{Genau genommen \MtsTrue, was wiederum standardmäßig die \Belegung\ \TxtTrue\ hat.}
		Belegt man bei sonst gleicher Belegung \chrqt{$c$} mit Vier, so ist der \gloFt{Wert} hingegen \glqq\TxtFalse\grqq.
	}
}

\newVerweis     {\logischerWert}{\glstext}  {logischerWert}
\newVerweis      {\logischeWert}{\glsuseri} {logischerWert}
\newglossaryentry{logischerWert}{
	name       =           {---, logischer \addIdx[
		name   =           {---, logischer},
		sort   =          {Wert, logischer}]{logischerWert}},
	sort       =          {Wert, logischer},
	text       ={logischer Wert},
	user1      ={logische  Wert},
%%%	symbol     ={\ensuremath{Mathmode}},% ToDo=Mathmode
%%%	user6      ={Textmode},
	description={\todoGeprueft%
%%%		\SymbolAmRand{logischerWert}%
		Synonym zu \Wahrheitswert.
	}
}

\newVerweis     {\Wertebereich} {\glstext}{Wertebereich}
\newVerweis[e]  {\Wertebereiche}{\glstext}{Wertebereich}
\newglossaryentry{Wertebereich}{
	name        ={Wertebereich \addIdx    {Wertebereich}},
	text        ={Wertebereich},
%%%	symbol      ={\ensuremath{Mathmode}},% ToDo=Mathmode
%%%	user6       ={Textmode},
	see         ={MtsWb,Zielbereich,Funktion},
	description ={\todoPruefen%
%%%		\SymbolAmRand{Wertebereich}%
		einer \Funktion.
	}
}

\newVerweis         {\Wikipedia}{\glstext}{Wikipedia}
\longnewglossaryentry{Wikipedia}{
	name            ={Wikipedia \addIdx   {Wikipedia}},
	text            ={Wikipedia},
%%%	symbol          ={\ensuremath{Mathmode}},% ToDo=Mathmode
%%%	user6           ={Textmode},
}{\todoPruefen%
%%%	\SymbolAmRand{Wikipedia}%
	\wikicite{bib:Wikipedia}{
		Wikipedia ist ein Projekt zum Aufbau einer [Internet-\nobreak]Enzyklopädie aus freien Inhalten.
	}
}

\newVerweis     {\Wort}   {\glstext}{Wort}
\newVerweis[e]  {\Worte}  {\glstext}{Wort}
\newVerweis     {\Woerter}{\glspl}  {Wort}
\newglossaryentry{Wort}{
	name        ={Wort \addIdx      {Wort}},
	text        ={Wort},
	plural      ={Wörter},
	see         ={Formelmenge},
%%%	symbol      ={\ensuremath{Mathmode}},% ToDo=Mathmode
%%%	user6       ={Textmode},
	description ={\todoPruefen%
%%%		\SymbolAmRand{Wort}%
		Synonym: \Formel\ ---
		Ein \Element\ aus einer \Sprache.
	}
}

\newVerweis     {\deutschesWort}{\glstext}  {deutschesWort}
\newglossaryentry{deutschesWort}{
	name       =           {---, deutsches},% \addIdx[
	name       =           {---, deutsches \addIdx[
		name   =           {---, deutsches},
		sort   =          {Wort, deutsches}]{deutschesWort}},
	sort       =          {Wort, deutsches},
	text       ={deutsches Wort},
%%%	symbol     ={\ensuremath{Mathmode}},% ToDo=Mathmode
%%%	user6      ={Textmode},
	see        ={deutscherBuchstabe,griechischesWort,lateinischesWort,Textwort},
	description={\todoOk%
%%%		\SymbolAmRand{deutschesWort}%
		Eine \Kette\ von \deutschenBuchstaben\ in derselben Schriftart.
	}
}

\newVerweis     {\griechischesWort}{\glstext}     {griechischesWort}
\newglossaryentry{griechischesWort}{
	name       =              {---, griechisches},% \addIdx[
	name       =              {---, griechisches \addIdx[
		name   =              {---, griechisches},
		sort   =             {Wort, griechisches}]{griechischesWort}},
	sort       =             {Wort, griechisches},
	text       ={griechisches Wort},
%%%	symbol     ={\ensuremath{Mathmode}},% ToDo=Mathmode
%%%	user6      ={Textmode},
	see        ={griechischerBuchstabe,deutschesWort,lateinischesWort,Textwort},
	description={\todoOk%
%%%		\SymbolAmRand{griechischesWort}%
		Eine \Kette\ von \griechischenBuchstaben\ in derselben Schriftart.
	}
}

\newVerweis     {\lateinischesWort}{\glstext}     {lateinischesWort}
\newglossaryentry{lateinischesWort}{
	name       =              {---, lateinisches \addIdx[
		name   =              {---, lateinisches},
		sort   =             {Wort, lateinisches}]{lateinischesWort}},
	sort       =             {Wort, lateinisches},
	text       ={lateinisches Wort},
%%%	symbol     ={\ensuremath{Mathmode}},% ToDo=Mathmode
%%%	user6      ={Textmode},
	see        ={lateinischerBuchstabe,deutschesWort,griechischesWort,Textwort},
	description={\todoOk%
%%%		\SymbolAmRand{lateinischesWort}%
		Eine \Kette\ von \lateinischenBuchstaben\ in derselben Schriftart.
	}
}

%Z === Z === Z === Z === Z === Z === Z === Z === Z === Z === Z === Z === Z === Z

\newVerweis         {\natuerlicheZahl}  {\glstext} {natuerlicheZahl}
\newVerweis        {\natuerlichenZahl}  {\glsuseri}{natuerlicheZahl}
\newVerweis[en]    {\natuerlichenZahlen}{\glsuseri}{natuerlicheZahl}
\longnewglossaryentry{natuerlicheZahl}{
	name           =            {Zahl, natürliche \addIdx[
		name       =            {Zahl, natürliche}]{natuerlicheZahl}},
	text           ={natürliche  Zahl},
	user1          ={natürlichen Zahl},
%%%	symbol         ={\ensuremath{Mathmode}},% ToDo=Mathmode
%%%	user6          ={Textmode},
}{\todoBeschreiben%
%%%	\SymbolAmRand{natuerlicheZahl}%
	Eine verbreitete Version für die Definition der \Menge\ \MtsINo\ der \GloFt{natürlichen Zahlen} ist folgende:
	\begin{itemize}
		\item [] $\emptyset \MtsIn \MtsINo$
		\item [] $n \MtsIn \MtsINo \MtsImp n \MtsCup \{n\} \MtsIn \MtsINo$
		\item [] Nur die so definierten \Elemente\ sind \Elemente\ aus \MtsINo.
	\end{itemize}
	Man nennt $n \MtsCup \{n\}$ auch den \DefFt{Nachfolger} von $n$ und es gilt:
	\begin{itemize}
		\item [] $n \MtsSubset \MtsINo$               , für $n   \MtsIn \MtsINo$
		\item [] $n  <  m \MtsEquiv n \MtsSubset   m$ , für $n,m \MtsIn \MtsINo$
		\item [] $n \le m \MtsEquiv n \MtsSubsetEq m$ , für $n,m \MtsIn \MtsINo$
	\end{itemize}
}

\newVerweis     {\Zeichen}{\glstext}{Zeichen}
\newglossaryentry{Zeichen}{
	name        ={Zeichen \addIdx   {Zeichen}},
	text        ={Zeichen},
%%%	symbol      ={\ensuremath{Mathmode}},% ToDo=Mathmode
%%%	user6       ={Textmode},
	description ={\todoOk%
%%%		\SymbolAmRand{Zeichen}%
		Ein \typographischesZeichen\ oder ein Leerzeichen.
	}
}

\newVerweis     {\typographischesZeichen}{\glstext}        {typographischesZeichen}
\newVerweis      {\typographischeZeichen}{\glsuseri}       {typographischesZeichen}
\newVerweis      {\TypographischeZeichen}{\Glsuseri}       {typographischesZeichen}
\newVerweis     {\typographischenZeichen}{\glsuserii}      {typographischesZeichen}
\newglossaryentry{typographischesZeichen}{
	name       =                    {---, typographisches \addIdx[
		name   =                    {---, typographisches},
		sort   =                {Zeichen, typographisches}]{typographischesZeichen}},
	sort       =                {Zeichen, typographisches},
	text       ={typographisches Zeichen},
	user1      ={typographische  Zeichen},
	user2      ={typographischen Zeichen},
%%%	symbol     ={\ensuremath{Mathmode}},% ToDo=Mathmode
%%%	user6      ={Textmode},
	see        ={typographischesSymbol},
	description={\todoOk%
%%%		\SymbolAmRand{typographischesZeichen}%
		Ein \Textbuchstabe\ oder ein \typographischesSymbol.
	}
}

\newVerweis     {\Zeichenkette} {\glstext}{Zeichenkette}
\newVerweis[n]  {\Zeichenketten}{\glstext}{Zeichenkette}
\newglossaryentry{Zeichenkette}{
	name        ={Zeichenkette \addIdx    {Zeichenkette}},
	text        ={Zeichenkette},
%%%	symbol      ={\ensuremath{Mathmode}},% ToDo=Mathmode
%%%	user6       ={Textmode},
	see         ={Symbolkette},
	description ={\todoOk%
%%%		\SymbolAmRand{Zeichenkette}%
		Eine \Kette\ von \Zeichen.
		Im Gegensatz zur \Symbolkette\ darf sie auch leer sein und die Leerzeichen gehören logisch dazu.
	}
}

\newVerweis     {\zerlegbar}  {\glstext}{zerlegbar}
\newVerweis[e]  {\zerlegbare} {\glstext}{zerlegbar}
\newVerweis[e]  {\Zerlegbare} {\Glstext}{zerlegbar}
\newVerweis[en] {\zerlegbaren}{\glstext}{zerlegbar}
\newVerweis[es] {\zerlegbares}{\glstext}{zerlegbar}
\newglossaryentry{zerlegbar}{
	name        ={zerlegbar \addIdx     {zerlegbar}},
	text        ={zerlegbar},
%%%	symbol      ={\ensuremath{Mathmode}},% ToDo=Mathmode
%%%	user6       ={Textmode},
	see         ={atomar},
	description ={\todoPruefen%
%%%		\SymbolAmRand{zerlegbar}%
		Eine \Aussage, \Formel, \Folge, \Kette\ oder \Symbol, die eine \echteTeilaussage,  -\eTfolge, -\eTformel, -\eTkette\ \textbzw -\eTsymbol\ enthält, heißt \GloFt{zerlegbar}.
	}
}

\newVerweis     {\Ziel} {\glstext}{Ziel}
\newVerweis[e]  {\Ziele}{\glstext}{Ziel}
\newglossaryentry{Ziel}{
	name        ={Ziel \addIdx    {Ziel}},
	text        ={Ziel},
%%%	symbol      ={\ensuremath{Mathmode}},% ToDo=Mathmode
%%%	user6       ={Textmode},
	description ={\todoPruefen%
%%%		\SymbolAmRand{Ziel}%
		Ein \GloFt{Ziel} ist \hier\ eine Anforderungen an \ASBA.
	}
}

\newVerweis     {\Zielbereich} {\glstext}{Zielbereich}
\newVerweis[e]  {\Zielbereiche}{\glstext}{Zielbereich}
\newglossaryentry{Zielbereich}{
	name        ={Zielbereich \addIdx    {Zielbereich}},
	text        ={Zielbereich},
%%%	symbol      ={\ensuremath{Mathmode}},% ToDo=Mathmode
%%%	user6       ={Textmode},
	see         ={MtsZb,Wertebereich,Funktion},
	description ={\todoPruefen%
%%%		\SymbolAmRand{Zielbereich}%
		einer \Funktion.
	}
}

\newVerweis     {\zulaessig}  {\glstext}{zulaessig}
\newVerweis[e]  {\zulaessige} {\glstext}{zulaessig}
\newVerweis[en] {\zulaessigen}{\glstext}{zulaessig}
\newVerweis[er] {\zulaessiger}{\glstext}{zulaessig}
\newglossaryentry{zulaessig}{
	name        ={zulässig \addIdx[
		name    ={zulässig}]            {zulaessig}},% wegen des 'ä'
	text        ={zulässig},
%%%	symbol      ={\ensuremath{Mathmode}},% ToDo=Mathmode
%%%	user6       ={Textmode},
	see         ={Formel,Transformation,Ersetzung},
	description ={\todoPruefen%
%%%		\SymbolAmRand{zulaessig}%
		Eine Eigenschaft von \Formeln, \Transformationen\ und \Ersetzungen.
	}
}

\newVerweis     {\zusammengesetzt}  {\glstext}{zusammengesetzt}
\newVerweis[en] {\zusammengesetzten}{\glstext}{zusammengesetzt}
\newglossaryentry{zusammengesetzt}{
	name        ={zusammengesetzt \addIdx     {zusammengesetzt}},
	text        ={zusammengesetzt},
%%%	symbol      ={\ensuremath{Mathmode}},% ToDo=Mathmode
%%%	user6       ={Textmode},
	see         ={zerlegbar},
	description ={\todoOk%
%%%		\SymbolAmRand{zusammengesetzt}%
		Eine Eigenschaft von \Symbolen.
		Ein \Symbol\ kann \unzerlegbar, aber \gloFt{zusammengesetzt} sein.
	}
}

\newVerweis     {\Zwischenglied}{\glstext}{Zwischenglied}
\newVerweis[er] {\Zwischenglieder}{\glstext}{Zwischenglied}
\newglossaryentry{Zwischenglied}{
	name        ={Zwischenglied \addIdx   {Zwischenglied}},
	text        ={Zwischenglied},
%%%	symbol      ={\ensuremath{Mathmode}},% ToDo=Mathmode
%%%	user6       ={Textmode},
	see         ={Anfangsglied,Endglied,Nachfolger,Vorgaenger,Zwischenglied},
	description ={\todoOk%
%%%		\SymbolAmRand{Zwischenglied}%
		Eine \GloFt{Zwischenglied} einer \Kette\ ist jedes ihrer \Kettenglieder, das weder \AnfangsG\ noch \Endglied\ \OptFt{der \Kette} ist.
	}
}
