%%############################################################################%%
%%                                                                            %%
%% Datei ASBA.tex zur Erzeugung des Projektdokuments von ASBA.                %%
%%                                                                            %%
%% Copyright (C) 2017  Winfried Teschers                                      %%
%%                                                                            %%
%% This program is free software: you can redistribute it and/or modify       %%
%% it under the terms of the GNU Affero General Public License as published   %%
%% by the Free Software Foundation, either version 3 of the License, or       %%
%% (at your option) any later version.                                        %%
%%                                                                            %%
%% This program is distributed in the hope that it will be useful,            %%
%% but WITHOUT ANY WARRANTY; without even the implied warranty of             %%
%% MERCHANTABILITY or FITNESS FOR A PARTICULAR PURPOSE.  See the              %%
%% GNU Affero General Public License for more details.                        %%
%%                                                                            %%
%% You should have received a copy of the GNU Affero General Public License   %%
%% along with this program.  If not, see <http://www.gnu.org/licenses/>.      %%
%%                                                                            %%
%% Dr. Winfried Teschers                                                      %%
%% Anton-Günther-Str. 26c                                                     %%
%% 91083 Baiersdorf                                                           %%
%% Germany                                                                    %%
%%                                                                            %%
%% e-mail: winfried.teschers@t-online.de                                      %%
%%                                                                            %%
%%############################################################################%%

% !TeX root = ASBA.tex
% !TeX encoding = UTF-8
% !TeX spellcheck = de_DE

%%############################################################################%%
%%                                                                            %%
%% Datei:  ASBA-Vorspann.tex                                                  %%
%% Inhalt: Vorspann für ASBA                                                  %%
%%                                                                            %%
%% Copyright (C) 2017  Winfried Teschers                                      %%
%%                                                                            %%
%% This program is free software: you can redistribute it and/or modify       %%
%% it under the terms of the GNU Affero General Public License as published   %%
%% by the Free Software Foundation, either version 3 of the License, or       %%
%% (at your option) any later version.                                        %%
%%                                                                            %%
%% This program is distributed in the hope that it will be useful,            %%
%% but WITHOUT ANY WARRANTY; without even the implied warranty of             %%
%% MERCHANTABILITY or FITNESS FOR A PARTICULAR PURPOSE.  See the              %%
%% GNU Affero General Public License for more details.                        %%
%%                                                                            %%
%% You should have received a copy of the GNU Affero General Public License   %%
%% along with this program.  If not, see <http://www.gnu.org/licenses/>.      %%
%%                                                                            %%
%% Dr. Winfried Teschers                                                      %%
%% Anton-Günther-Straße 26c                                                   %%
%% 91083 Baiersdorf                                                           %%
%% Germany                                                                    %%
%%                                                                            %%
%% e-mail: winfried.teschers@t-online.de                                      %%
%%                                                                            %%
%%############################################################################%%

% !TeX root = ASBA.tex
% !TeX encoding = UTF-8
% !TeX spellcheck = de_DE

% Glossareinträge werden in "ASBA-Vorspann-Glossar" definiert.
% Elemente, die nur in "ASBA-Mathematik.tex" verwendet werden, werden in "ASBA-Vorspann-Mathematik.tex" definiert.

\documentclass[english, ngerman, parskip=half, headsepline, footsepline, fleqn, notitlepage]{scrreprt}

% Pakete #######################################################################

% Pakete aus "LaTeX 2e Befehlsübersicht" - <...> für Parameter -----------------
\usepackage[utf8]{inputenc}% Für direkte Eingabe von Umlauten mit [utf8]
\usepackage[T1]{fontenc}%    Darstellung von Umlauten mit [T1]
\usepackage[english,ngerman]{babel}% Neue deutsche Rechtschreibung mit [ngerman]
%\usepackage{multicol}%       Verwende <n> Spalten: \begin{multicols}{<n>}
\usepackage[fleqn]{amsmath}% Erweiterung für LaTeX-Mathe; fleqn=einrücken
\usepackage{amssymb}%        Zusätzliche Mathesymbole (bsp. IR)
\usepackage{graphicx}%       Bilder einbinden:\includegraphics[width=<x>]{build}
%\usepackage{url}%            URLs einfügen: \url{http://...}
\usepackage{textcomp}%       Zusätzliche Symbole z.B. \textmu
%\usepackage{upgreek}%        Aufrechte griechische Symbole
%\usepackage{wrapfig}%        Textumflossene Abbildungen
%\usepackage{subcaption}%     Teilabbildungen in \begin{figure} mit
%                            \begin{subfigure}[b]{<Weite>} und \hfill
%\usepackage{pdfpages}%       \includepdf[<pages=1-2>]{<Anhang.pdf>}
% weiter empfohlen ---------------------
\usepackage{lmodern}%        Ersetzt "Computer Modern" durch "Latin Modern"
\usepackage{mathtools}%      Mathematical tools to use with asmmath

% allgemein --------------------------------------------------------------------
\usepackage{scrlayer-scrpage}
\usepackage{geometry}%  Flexible and complete interface to document dimensions.
\usepackage{microtype}% Subliminal refinements towards typographical perfection.
\usepackage{pict2e}%    New implementation of picture commands.
\usepackage[autostyle]{csquotes}% Contex sensitive quotation facilities

% mathematische Pakete ---------------------------------------------------------
\usepackage{amsfonts}%       TeX fonts from the American Mathematical Society.
\usepackage{amsopn}%         Provides a command \DeclareMathOperator.
\usepackage{mathabx}%        Three series of mathematical symbols.
\usepackage{mathpazo}%       Fonts to typeset mathematics to match palatino.
%\usepackage{stix}%           OpenType Unicode maths fonts.
%%                            führt zu: To many symbol fonts declared
%\usepackage{cancel}%         Place lines through mathematical formulae.

% Tabellen ---------------------------------------------------------------------
\usepackage{float}%          An Improved Environment for Floats
\usepackage[table]{xcolor}%  Driver-independent color extensions - vor 'color'?
%\usepackage{ctable}%   Flexible typesetting of table and figure using key/value
% Das Paket 'ctable' fasst (mit anderer Syntax) die Eigenschaften der Pakete
\usepackage{array}%
\usepackage{tabularx}%       Erweiterung von tabular*
\usepackage{booktabs}%       Nicer layout of tables
% zusammen und lädt zusätzlich noch die Pakete
\usepackage{rotating}%       Rotating tools, including rotated full page floats
\usepackage{xspace}%         Behandelt Zwischenraum nach Makros
\usepackage{color}%          LaTeX support for color
\usepackage{xkeyval}%        Extension of the keyval package
% Ende der von 'ctable' geladenen Pakete
\usepackage{threeparttable}% Tables with captions and notes all the same width.
\usepackage{multirow}%       Create tabular cell spanning multiple rows.
\usepackage{diagbox}%        Table heads with diagonal lines.
\usepackage{arydshln}%       Draw dash-lines in array/tabular.
\usepackage{caption}%        Customizing captions in floating environments.

%% Nur für Entwicklung und Test ------------------------------------------------
\usepackage{srcltx}%            Klick im Viewer öffnet Editor und umgekehrt
\setlength{\overfullrule}{5pt}% Breite des schwarzen Balkens rechts von Overfull
%\usepackage{showidx}%           Index auf Seitenrand; nicht mit 'splitidx'\marginparwidth
%\setlength{\marginparwidth}{2cm}% Breite des Seitenrandes

% Verzeichnisse ----------------------------------------------------------------
%TODO 'minitoc' anwenden
\usepackage[germanb]{minitoc}%     Unterverzeichnisse
\usepackage[protected]{splitidx}%  mehrere Indizes - Nur statt 'makeidx'
%\usepackage{makeidx}%             Indexing - Entweder 'makeidx' oder 'splitidx'
%\usepackage{hvindex}%             Support for 'makeidx' - after 'babel

\usepackage{varioref}%             \vref...
\usepackage[colorlinks,destlabel,hyperindex,linkcolor=blue,pagebackref,unicode,
	pdftitle   ={ASBA - Axiome, Sätze, Beweise und Auswertungen},
	pdfauthor  ={Dr. Winfried Teschers},
	pdfsubject ={Projektdokument},
	pdfkeywords={Mathematik, automatisches Beweisen, Beweisunterstützung}
]{hyperref}
% Paket ' hyperref': Extensive support for hypertext in LaTeX
% colorlinks  - colors the links instead of using boxes
% destlabel   - verwendet key von \label{key} nach Start von Kapiteln u.a.
% hyperindex  - make items in the index by hyperlinked back to the text
% linkcolor   - Color for normal internal links
% pagebackref - inserts extra ‘back’ links into the bibliography for each entry
% unicode     - Unicode encoded PDF strings
% pdftitle    - Title    im PDF-Dokument
% pdfauthor   - Author   im PDF-Dokument
% pdfsubject  - Subject  im PDF-Dokument
% pdfkeywords - Keywords im PDF-Dokument
%TODO \autoref u.a. von 'hyperref' nutzen
%%%\addto\extrasngerman{% siehe manual for 'hyperref' Seite 17
%%%	\def\subsectionautorefname{\subsectionname}%
%%%	\def\subsubsectionautorefname{\subsubsectionname}%
%%%}

\usepackage[nopostdot,xindy]{glossaries}
%%%\usepackage[toc,index,nohypertypes={index},nopostdot,xindy]{glossaries}
% Paket 'glossaries': Create glossaries and lists of acronyms
% - nach 'hyperref'
% - läd 'glossaries-german'
% Optionen:         ('glossaries' erfordert 1. zusätzlichen Lauf von pdflatex)
% toc          - in den Inhalt (erfordert 2. zusätzlichen Lauf von pdflatex)
% index
% nohypertypes - ={index} -
% nopostdot    - Keinen zusätzlichen Punkt am Ende der Einträge.
% xindy        - Verwendung von 'xindy' für die Sortierung
% lädt {glossaries-german}% German language module for glossaries package
%\GlsSetQuote{+}% nur für 'makeindex' - siehe 'User Manual for glossaries' S. 31

% Einstellung von globalen Werten und Makro-Redefinitionen #####################

\geometry{textwidth=170mm,textheight=256mm,twoside}% optional Option 'showframe'
%TODO ausprobieren
%%%\geometry{showframe}

% Kopfzeilen ===================================================================
\newcommand*{\texthead}[1]{\textnormal{\textsf{\textbf{#1}}}}% Schriftart
\newcommand*{\Lehead}  [1]{\lehead{\texthead{#1}}}
\newcommand*{\Cehead}  [1]{\cehead{\texthead{#1}}}
\newcommand*{\Rehead}  [1]{\rehead{\texthead{#1}}}
\newcommand*{\Lohead}  [1]{\lohead{\texthead{#1}}}
\newcommand*{\Cohead}  [1]{\cohead{\texthead{#1}}}
\newcommand*{\Rohead}  [1]{\rohead{\texthead{#1}}}
\newcommand*{\Ohead}   [1]{\ohead {\texthead{#1}}}
\newcommand*{\Chead}   [1]{\chead {\texthead{#1}}}
\newcommand*{\Ihead}   [1]{\ihead {\texthead{#1}}}
\newcommand*{\Ofoot}   [1]{\ofoot {\textnormal{\textbf{#1}}}}
\newcommand*{\Cfoot}   [1]{\cfoot {\textnormal{#1}}}
\newcommand*{\Ifoot}   [1]{\ifoot {\textnormal{#1}}}
\newcommand*{\Pagestyle}{\pagestyle{scrheadings}}
\newcommand*{\Thispagestyle}{\thispagestyle{scrheadings}}

% Kopfzeilen mit 'scrlayer-scrpage'
%         \Lehead \Cehead \Rehead | \Lohead \Cohead \Rohead
% \Ohead: \Lehead                                   \Rohead
% \Chead:         \Cehead                   \Cohead
% \Ihead:                 \Rehead   \Lohead
% ASBA <Chapter-Überschrift> \Chaptername~\thechapter
%                            \sectionname~\thesection <Section-Überschrift> ASBA
%Initialisierung
\Ohead{ASBA}%          bleibt unverändert
\Chead{\contentsname}% wird laufend verändert
\Ihead{}%              wird laufend verändert

% Kapitel ======================================================================
\newcommand*{\Chaptername}{\chaptername}% wird mit 'Anhang' überschrieben
\newcommand*{\beginchapter}[2][\Chaptername~\thechapter]{% direkt nach \chapter
	\Chead{#2}%                         Kopfzeile Mitte = <Kapitelname>
	\Ihead{#1}%                         Kopfzeile Innen = Kapitel/Anhang <Nr.>
	\Pagestyle%                         veränderte Kopfzeile aktivieren
	\Thispagestyle%                     ... auch für diese Seite, da ...
}%                                      '\chapter' Kopf-/Fußzeilen deaktiviert
\newcommand*{\Endchapter}{% am Ende eines Kapitels
	\Thispagestyle% sicherheitshalber Kopfzeile für diese Seite aktivieren
}

% Abschnitte ===================================================================
\newcommand*{\beginsection}[1]{%        direkt nach \section
	\Cohead{#1}%                        oben rechts mittig = <Abschnittsname>
	\Lohead{\sectionname~\thesection}%  oben rechts innen = <Abschnittsnummer>
	\Pagestyle%                         veränderte Kopfzeile aktivieren
}

% Fußzeilen ====================================================================
\Ofoot{\thepage}
\Cfoot{Winfried Teschers}
\Ifoot{\today}
\Pagestyle%                                       aktiviert Kopf- und Fußzeilen

% Fußnoten ---------------------------------------------------------------------
\deffootnote[10pt]% Markenbreite
{10pt}% Einzug - für Blocksatz: Markenbreite
{0pt}% Absatzeinzug für Folgeabsätze
{\makebox[9pt][r]{\textsuperscript{\thefootnotemark)} }}% Zeichen; < Markenbreite
\deffootnotemark {\textsuperscript{\thefootnotemark)}}
\newcommand*{\Tnote}[1]{\tnote{#1)}}

% Vordefinierte Werte ändern ===================================================
\setcounter{tocdepth}{3}%    Tiefe des Inhaltsverzeichnisses: 2 => subsection
\setcounter{secnumdepth}{3}% Nummerierung:                    3 => subsubsection
\setlength\extrarowheight{1pt}% Tabellenzellenhöhe vergrößern
\captionsetup{labelfont=bf}%    Tabellenbeschriftung in bf = bold font

% Empfehlung aus: Herbert Voß, LaTeX Referenz, 3. Auflage, Berlin 2014; S. 37f
\renewcommand{\floatpagefraction}{0.7}% Empfehlung: 0.5-0.8 Voreinstellung: 0.9
\renewcommand{\textfraction}{0.15}%                 0.1-0.3                 0.05
\renewcommand{\topfraction}{0.8}%                   0.5-0.85                0.9
\renewcommand{\bottomfraction}{0.5}%                0.2-0.5                 0.9
\setcounter{topnumber}{3}%                                                  2
\setcounter{totalnumber}{15}%                                               3

% Neue Elemente ----------------------------------------------------------------
\newcounter{Enumi}% für unterbrochene Listennummerierung
\newcounter{Enumii}% für unterbrochene Listennummerierung
\newcounter{Enumiii}% für unterbrochene Listennummerierung

% Bildelemente #################################################################

\newcommand*{\textbild}[1]{\textbf{\textsf{#1}}}% Textauszeichnungen für Text im Bild
\newcommand*{\Datei}[4][0.5]{% #2 x #3 = (-#2/2,-#3/2),(#2/2,#3/2)
	% [Eck-Radius], Breite, Höhe, Name
	\put(0,0){\oval[#1](#2,#3)}
	\put(0,0){\makebox(0,0){\textbild{#4}}}
}
\newcommand*{\Datenbank}[5]{% 2(#1) x 2(#2+#3) = (-#1,-#2-#3),(+#1,+#2+#3)
	% Halbmesser x, Halbmesser y, halbe Höhe, Name - Ursprung in der Mitte
	\put(0.0,-#3){
		\qbezier(-#1,0.0)(-#1,-#2)(0.0,-#2)
		\qbezier(+#1,0.0)(+#1,-#2)(0.0,-#2)
	}
	\put(0,0){\Line(-#1,-#3)(-#1, #3)}
	\put(0,0){\Line( #1,-#3)( #1, #3)}
	\put(0.0,#3){
		\qbezier(-#1,0.0)(-#1,-#2)(0.0,-#2)
		\qbezier( #1,0.0)( #1,-#2)(0.0,-#2)
		\qbezier(-#1,0.0)(-#1, #2)(0.0, #2)
		\qbezier( #1,0.0)( #1, #2)(0.0, #2)
	}
	\makebox(0,0){\textbild{#4}}
	\makebox(0,-#3){\textbild{#5}}
}
% '\Männchen' (mit 'ä') führt zu Fehler
\newcommand*{\Maennchen}{% 1x2 = (-0.5,-1.7),(+0.5,+0.3) Ursprung im Kopf
	\put(0,0){\circle{0.6}}
	\Line(0.0,-0.3)(0.0,-1.2)
	\polyline(-0.5,-0.3)(0.0,-0.6)(0.5,-0.3)
	\polyline(-0.5,-1.7)(0.0,-1.2)(0.5,-1.7)
}
\newcommand*{\Marker}[2][0.5]{% 2x#1 x 2x#1 - Kreis mit Text
	{
		\linethickness{0.5pt}
		\color{white}
		\put(0,0){\circle*{#1}}
		\color{black}
		\put(0,0){\circle{#1}}
		\put(0,0){\makebox(0,0){\small\textbild{#2}}}
	}
}
\newcommand*{\marker}[2][0.5]{% 2x#1 x 2x#1 - Kreis mit Text - grau
	{
		\linethickness{0.5pt}
		\color{white}
		\put(0,0){\circle*{#1}}
		\color{gray}
		\put(0,0){\circle{#1}}
		\put(0,0){\makebox(0,0){\small\textbild{#2}}}
	}
}
\newcommand*{\Papier}[3]{% 3x#1+#2 = (0.0,-#2),(2.1,#1) Ursprung links unten
	% Länge (Höhe), Länge Abschluss, Name
	\polyline(0.0,-0.01)(+0.0,+#1)(+2.8,+#1)(+2.8,-0.01)
	\qbezier(+2.1,+#2)(+2.6,+#2)(+2.8,0.0)
	\qbezier(+2.1,+#2)(+1.6,+#2)(+1.4,0.0)
	\qbezier(+0.7,-#2)(+1.2,-#2)(+1.4,0.0)
	\qbezier(+0.7,-#2)(+0.2,-#2)(+0.0,0.0)
	\put(0,0){\makebox(+2.8,+#1){\textbild{#3}}}
}
\newcommand*{\Terminal}[1]{% 2x2 =(-1.0,-1.4),(+1.0,+0.6)Ursprung im Monitor
	% Bildschirm
	%		\put(0,0){\polygon(-1.0,-0.6)(+1.0,-0.6)(+1.0,+0.6)(-1.0,+0.6)}
	\put(-1.0,-0.6){\framebox(2,1.2){#1}}
	\put(0,0){\oval[0.1](1.65,0.85)}
	% Hals
	\put(0,0){\Line(-0.2,-0.6)(-0.2,-1.0)}
	\put(0,0){\Line(+0.2,-0.6)(+0.2,-1.0)}
	% Tastatur
	\multiput(-1.0,-1.0)(+0.0,-0.133){4}{\line(1,0){2.0}}
	\multiput(-1.0,-1,0)(+0.2,+0.0){11}{\line(0,-1){0.4}}
}
\newcommand*{\Wolke}[1]{% 3.0x1.5 = (-1.5,-1.0),(+1.5,+0.5)
	% unterer Bogen
	\qbezier(-1.5,+0.0)(-1.5,-1.0)(-0.0,-1.0)
	\qbezier(+1.5,+0.0)(+1.5,-1.0)(+0.0,-1.0)
	% oberer Bogen rechts
	\qbezier(+1.5,+0.0)(+1.5,+0.5)(+0.8,+0.5)
	\qbezier(+0.4,+0.4)(+0.4,+0.5)(+0.8,+0.5)
	% oberer Bogen Mitte
	\qbezier(+0.5,+0.2)(+0.5,+0.5)(+0.0,+0.5)
	\qbezier(-0.4,+0.4)(-0.4,+0.5)(-0.0,+0.5)
	% oberer Bogen links
	\qbezier(-0.3,+0.2)(-0.3,+0.5)(-0.8,+0.5)
	\qbezier(-1.5,+0.0)(-1.5,+0.5)(-0.8,+0.5)
	\put(-1.5,-1.0){\makebox(3.0,1.5){\textbild{#1}}}
}

% sonstige nützliche Kommandos #################################################

% Im Parameter von '\turl' muss vor jedem Zeichen aus "{}#&%$" ein '\' stehen
% und '\' und '~' durch '\textbackslash' bzw. '\textasctilde' ersetzt werden.
\newcommand*{\tourl}[1]{$\rightarrow$~\url{#1}}
\newcommand*{\formulatoleft}{&&&&&&&&&&}%  Um Formeln nach links zu komprimieren
\newcommand*{\formulaspace} {&&&&}      %  Für Platz zwischen den Formeln
\newcommand*{\todo}[1]{\textbf{>~>~>~#1~<~<~<}}% für TODOs
% Quotierung von Zeichen [chr], Zeichenfolgen [seq] und Zeichenketten [str]
% Parameter im Textmodus
%%%\newcommand*{\chrqt}[1]{\ensuremath{\langle}#1\ensuremath{\rangle}}
\newcommand*{\chrqt}[1]{\ensuremath{\langle\text{#1}\rangle}}% Zeichen/Symbol
\newcommand*{\seqqt}[1]{\ensuremath{\langle\!\langle\text{#1}\rangle\!\rangle}}% Zeichenfolge/Formel
\newcommand*{\charf}[1]{\textbf{\texttt{#1}}}%Schriftart für Zeichen
\newcommand*{\strqt}[1]       {``\charf{#1}''}%              Zeichenkette

% Strukturbezeichnungen ergänzen und Verweise auf Strukturen vereinfachen
\newcommand*{\sectionname}       {Abschnitt}
\newcommand*{\sectionnames}      {Abschnitte}
\newcommand*{\subsectionname}    {Unterabschnitt}
\newcommand*{\subsectionnames}   {Unterabschnitte}
\newcommand*{\subsubsectionname} {Paragraph}
\newcommand*{\subsubsectionnames}{Paragraphen}

%%%% Nominativ
%%%\newcommand*{\nomcha}        [1]{dieses       \chaptername~\ref{#1}}
%%%\newcommand*{\nomsec}        [1]{dieser       \sectionname~\ref{#1}}
%%%\newcommand*{\nomsub}        [1]{dieser    \subsectionname~\ref{#1}}
%%%\newcommand*{\nomsubsub}     [1]{dieser \subsubsectionname~\ref{#1}}

% Dativ
\newcommand*{\datcha}        [1]{diesem       \chaptername~\ref{#1}}
\newcommand*{\datsec}        [1]{diesem       \sectionname~\ref{#1}}
\newcommand*{\datsub}        [1]{diesem    \subsectionname~\ref{#1}}
\newcommand*{\datsubsub}     [1]{diesem \subsubsectionname~\ref{#1}}

% Akkusativ
\newcommand*{\akkcha}        [1]{dieses       \chaptername~\ref{#1}}
\newcommand*{\akksec}        [1]{diesen       \sectionname~\ref{#1}}
\newcommand*{\akksub}        [1]{diesen    \subsectionname~\ref{#1}}
\newcommand*{\akksubsub}     [1]{diesen \subsubsectionname~\ref{#1}}

% Referenzen

\newcommand*{\vreffig}       [1]             {\figurename~\vref{#1}}
\newcommand*{\vreftab}       [1]              {\tablename~\vref{#1}}
\newcommand*{\vrefcha}       [1]            {\chaptername~\vref{#1}}
\newcommand*{\vrefsec}       [1]            {\sectionname~\vref{#1}}
\newcommand*{\vrefsub}       [1]         {\subsectionname~\vref{#1}}
\newcommand*{\vrefsubsub}    [1]      {\subsubsectionname~\vref{#1}}
\newcommand*{\vrefdef}       [1]              {Definition~\vref{#1}}
\newcommand*{\vreffor}    [1]{\eqref{#1} auf \pagename~\pageref{#1}}
\newcommand*{\vrefziel}      [1]                    {Ziel~\vref{#1}}

\newcommand*{\vrefausfig}    [1]{aus der   \vreffig            {#1}}
\newcommand*{\vrefaustab}    [1]{aus der   \vreftab            {#1}}
\newcommand*{\vrefauscha}    [1]{aus       \vrefcha            {#1}}
\newcommand*{\vrefaussec}    [1]{aus       \vrefsec            {#1}}
\newcommand*{\vrefaussub}    [1]{aus       \vrefsub            {#1}}
\newcommand*{\vrefaussubsub} [1]{aus       \vrefsubsub         {#1}}

\newcommand*{\vrefinfig}     [1]{in  der   \vreffig            {#1}}
\newcommand*{\vrefintab}     [1]{in  der   \vreftab            {#1}}
\newcommand*{\vrefincha}     [1]{im        \vrefcha            {#1}}
\newcommand*{\vrefinsec}     [1]{im        \vrefsec            {#1}}
\newcommand*{\vrefinsub}     [1]{im        \vrefsub            {#1}}
\newcommand*{\vrefinsubsub}  [1]{im        \vrefsubsub         {#1}}

\newcommand*{\vrefInfig}     [1]{In  der   \vreffig            {#1}}
\newcommand*{\vrefIntab}     [1]{In  der   \vreftab            {#1}}
\newcommand*{\vrefIncha}     [1]{Im        \vrefcha            {#1}}
\newcommand*{\vrefInsec}     [1]{Im        \vrefsec            {#1}}
\newcommand*{\vrefInsub}     [1]{Im        \vrefsub            {#1}}
\newcommand*{\vrefInsubsub}  [1]{Im        \vrefsubsub         {#1}}

\newcommand*{\vrefvonfig}    [1]{von       \vreffig            {#1}}
\newcommand*{\vrefvontab}    [1]{von       \vreftab            {#1}}
\newcommand*{\vrefvoncha}    [1]{von       \vrefcha            {#1}}
\newcommand*{\vrefvonsec}    [1]{von       \vrefsec            {#1}}
\newcommand*{\vrefvonsub}    [1]{von       \vrefsub            {#1}}
\newcommand*{\vrefvonsubsub} [1]{von       \vrefsubsub         {#1}}

%%%\newcommand*{\vrefdfig}      [1]{die       \vreffig            {#1}}
%%%\newcommand*{\vrefdtab}      [1]{die       \vreftab            {#1}}
%%%\newcommand*{\vrefdcha}      [1]{das       \vrefcha            {#1}}
%%%\newcommand*{\vrefdsec}      [1]{der       \vrefsec            {#1}}
%%%\newcommand*{\vrefdsub}      [1]{der       \vrefsub            {#1}}
%%%\newcommand*{\vrefdsubsub}   [1]{der       \vrefsubsub         {#1}}
\newcommand*{\vrefDfig}      [1]{Die       \vreffig            {#1}}
\newcommand*{\vrefDtab}      [1]{Die       \vreftab            {#1}}
\newcommand*{\vrefDcha}      [1]{Das       \vrefcha            {#1}}
\newcommand*{\vrefDsec}      [1]{Der       \vrefsec            {#1}}
\newcommand*{\vrefDsub}      [1]{Der       \vrefsub            {#1}}
\newcommand*{\vrefDsubsub}   [1]{Der       \vrefsubsub         {#1}}

\newcommand*{\vrefseefig}    [1]{siehe     \vreffig            {#1}}
\newcommand*{\vrefseetab}    [1]{siehe     \vreftab            {#1}}
\newcommand*{\vrefseecha}    [1]{siehe     \vrefcha            {#1}}
\newcommand*{\vrefseesec}    [1]{siehe     \vrefsec            {#1}}
\newcommand*{\vrefseesub}    [1]{siehe     \vrefsub            {#1}}
\newcommand*{\vrefseesubsub} [1]{siehe     \vrefsubsub         {#1}}
\newcommand*{\vrefseedef}    [1]{siehe     \vrefdef            {#1}}
\newcommand*{\vrefseefor}    [1]{siehe     \vreffor            {#1}}
\newcommand*{\vrefseeziel}   [1]{siehe     \vrefziel           {#1}}

\newcommand*{\vrefSeefig}    [1]{Siehe     \vreffig            {#1}}
\newcommand*{\vrefSeetab}    [1]{Siehe     \vreftab            {#1}}
\newcommand*{\vrefSeecha}    [1]{Siehe     \vrefcha            {#1}}
\newcommand*{\vrefSeesec}    [1]{Siehe     \vrefsec            {#1}}
\newcommand*{\vrefSeesub}    [1]{Siehe     \vrefsub            {#1}}
\newcommand*{\vrefSeesubsub} [1]{Siehe     \vrefsubsub         {#1}}
\newcommand*{\vrefSeedef}    [1]{Siehe     \vrefdef            {#1}}
\newcommand*{\vrefSeefor}    [1]{Siehe     \vreffor            {#1}}
\newcommand*{\vrefSeeziel}   [1]{Siehe     \vrefziel           {#1}}

\newcommand*{\vrefnotefig}   [1]{\footnote{\vrefseefig         {#1}}}
\newcommand*{\vrefnotetab}   [1]{\footnote{\vrefseetab         {#1}}}
\newcommand*{\vrefnotecha}   [1]{\footnote{\vrefseecha         {#1}}}
\newcommand*{\vrefnotesec}   [1]{\footnote{\vrefseesec         {#1}}}
\newcommand*{\vrefnotesub}   [1]{\footnote{\vrefseesub         {#1}}}
\newcommand*{\vrefnotesubsub}[1]{\footnote{\vrefseesubsub      {#1}}}
\newcommand*{\vrefnotedef}   [1]{\footnote{\vrefseedef         {#1}}}
\newcommand*{\vrefnotefor}   [1]{\footnote{\vrefseefor         {#1}}}
\newcommand*{\vrefnoteziel}  [1]{\footnote{\vrefseeziel        {#1}}}

\newcommand*{\citesee}       [1]{\seename~\cite                {#1}}
\newcommand*{\citenote}      [1]{\footnote{\citesee            {#1}}}
\newcommand*{\alternativ}   [1]{\footnote{alternativ: \defn{#1}}}
\newcommand*{\alternativen} [2]{\footnote{alternativ: \defn{#1} oder \defn{#2}}}

% Abkürzungen mit Punkten; zur Unterscheidung vom Satzende
\newcommand*{\textbzgl}{bzgl.\@}
\newcommand*{\textbzw} {bzw.\@}
\newcommand*{\textdh}  {d.\@\,h.\@}
\newcommand*{\textDh}  {D.\@\,h.\@}
\newcommand*{\textevtl}{evtl.\@}
\newcommand*{\textggf} {ggf.\@}
\newcommand*{\textGgf} {Ggf.\@}
\newcommand*{\textiAlg}{i.\@\,Alg.\@}
\newcommand*{\textIAlg}{I.\@\,Alg.\@}
\newcommand*{\textua}  {u.\@\,a.\@}
\newcommand*{\textUa}  {U.\@\,a.\@}
\newcommand*{\textusw} {usw.\@}
\newcommand*{\textzB}  {z.\@\,B.\@}
\newcommand*{\textZB}  {Z.\@\,B.\@}
% Weitere Abkürzungen
\newcommand*{\textdots}{…}

% Verzeichnisse ################################################################

% Indices und Symbole
\makeindex
\newindex[Symbolverzeichnis]{sym}
\newindex[Index]{idx}

% Glossareinträge
\makeglossaries
\setacronymstyle{long-sc-short}

% Hervorhebung von neu definierten Begriffen
\newcommand*{\defn}      [1]{\textbf{#1}}
\newcommand*{\textdef}   [1]{\textbf{\textit{\hyperpage{#1}}}}% ein 'Font-Kommando'
% Die folgenden #1 müssen mit einem Makro enden, das genau einen optionalen
% Parameter hat, der ein 'Font-Kommando' wie z.B. '\textdef'sein muss.
% Das sind i.Alg. die vor '\...newglossaryentry' definierten Makros.
\newcommand*{\definition}   [1]  {\defn{#1[textdef]}}
\newcommand*{\undefinition} [1]  {\defn{#1[textdef]}\;}%unär: folgender  Abstand
\newcommand*{\bindefinition}[1]{\;\defn{#1[textdef]}\;}%binär:umgebender Abstand


% Titelseite ###################################################################

\titlehead{
	{\Large Dr. Winfried Teschers}\\
	Anton-Günther-Straße 26c\\91083 Baiersdorf\\
	{\footnotesize winfried.teschers@t-online.de}
}
\subject{Projektdokument}
\title{{\Huge \ASBA}\\Axiome, Sätze, Beweise und Auswertungen}
\subtitle{Projekt zur maschinellen Überprüfung von mathematischen Beweisen und deren Ausgabe in lesbarer Form}
\author{Winfried Teschers}
\date{\today}
\publishers{\vspace{1cm}\normalsize
	Es wird ein Programmsystem beschrieben, das zu eingegebenen Axiomen, Sätzen, und Beweisen letztere prüft, Auswertungen generiert und zu gegebenen Ausgabeschemata eine Ausgabe der Elemente in üblicher Formelschreibweise im \LaTeX-Format erstellt.
}

% Dokument #####################################################################

\begin{document}
	\maketitle

	~\vfill Copyright \copyright{} 2017 Winfried Teschers\bigskip

	Permission is granted to copy, distribute and/or modify this document under the terms of the GNU Free Documentation License, Version~1.3 or any later version published by the Free Software Foundation; with no Invariant Sections, no Front-Cover Texts, and no Back-Cover Texts.
	You should have received a copy of the GNU Free Documentation License along with this document.
	If not, see \url{http://www.gnu.org/licenses/}.

	\beforechapter
	%chapter{Inhaltsverzeichnis}% ##############################################
	\likechapter{\contentsname}
	\label{cha:Inhaltsverzeichnis}
	\tableofcontents
	\Endchapter

	%%############################################################################%%
%%                                                                            %%
%% Datei:  ASBA-Analyse.tex                                                   %%
%% Inhalt: Kapitel "Analyse"                                                  %%
%%                                                                            %%
%% Copyright (C) 2017  Winfried Teschers                                      %%
%%                                                                            %%
%% This program is free software: you can redistribute it and/or modify       %%
%% it under the terms of the GNU Affero General Public License as published   %%
%% by the Free Software Foundation, either version 3 of the License, or       %%
%% (at your option) any later version.                                        %%
%%                                                                            %%
%% This program is distributed in the hope that it will be useful,            %%
%% but WITHOUT ANY WARRANTY; without even the implied warranty of             %%
%% MERCHANTABILITY or FITNESS FOR A PARTICULAR PURPOSE.  See the              %%
%% GNU Affero General Public License for more details.                        %%
%%                                                                            %%
%% You should have received a copy of the GNU Affero General Public License   %%
%% along with this program.  If not, see <http://www.gnu.org/licenses/>.      %%
%%                                                                            %%
%% Dr. Winfried Teschers                                                      %%
%% Anton-Günther-Straße 26c                                                   %%
%% 91083 Baiersdorf                                                           %%
%% Germany                                                                    %%
%%                                                                            %%
%% e-mail: winfried.teschers@t-online.de                                      %%
%%                                                                            %%
%%############################################################################%%

% !TeX root = ASBA.tex
% !TeX encoding = UTF-8
% !TeX spellcheck = de_DE

\chapter     {Analyse}% ########################################################
\beginchapter{Analyse}
\label   {cha:Analyse}

In der Mathematik gibt es eine unüberschaubare Menge an \Axiomen, \Saetzen, \Beweisen, \Fachbegriffen\ und \Fachgebieten.
Zu den meisten \Fachgebieten\ gibt es noch ungelöste Probleme.

Es fehlt ein System, das einen Überblick bietet und die Möglichkeit, \Beweise\ automatisch zu überprüfen.
Außerdem sollte all dies in üblicher mathematischer Schreibweise ein- und ausgegeben werden können.
In diesem Dokument werden die Grundlagen für das zu entwickelnde Programmsystem \defTxt{\ASBA} (ein Akronym für "`\textbf{A}xiome, \textbf{S}ätze, \textbf{B}eweise und \textbf{A}uswertungen"') behandelt.

Ein Programmsystem mit ähnlicher Aufgabenstellung findet sich im GitHub Projekt \emph{Hilbert~II} (\cite{bib:HilbertII, bib:qedeq}).
Einige Ideen sind von dort übernommen worden.

\section     {Fragen}% =========================================================
\beginsection{Fragen}
\label   {sec:Fragen}

Einige der Fragen, die in diesem Zusammenhang auftauchen,
werden nun formuliert:
\begin{enumerate}
	%
	\item \label{Frage:Grundlagen} \defFt{Grundlagen}:
	Was sind die Grundlagen?
	\textZB\ welche \Logik\ und welche \Mengenlehre.
	%
	\item \label{Frage:Basis} \defFt{Basis}:
	Welche wichtigen \Axiome, \Saetze, \Beweise, \Fachbegriffe\ und \Fachgebiete\ gibt es?
	Welche davon sind Standard?
	%
	\item \label{Frage:Axiome} \defFt{\Axiome}:
	Welche \Axiome\ werden bei einem \Satz\ oder \Beweis\ vorausgesetzt?
	Allgemein anerkannte oder auch strittige, wie \textzB\ den \emph{\Satz\ vom ausgeschlossenen Dritten} (\emph{tertium non datur}) oder das \emph{Auswahlaxiom}.
	%
	\item \label{Frage:Beweis} \defFt{\Beweis}:
	Ist ein \Beweis\ fehlerfrei?
	%
	\item \label{Frage:Konstruktion} \defFt{Konstruktion}:
	Gibt es einen konstruktiven \Beweis?
	%
	\item \label{Frage:Vergleiche} \defFt{Vergleiche}:
	Welcher \Beweis\ ist besser?
	Nach welchem Kriterium?
	\textZB\ elegant, kurz, einsichtig oder wenige \Axiome.
	Was heißt eigentlich \emph{elegant}?
	%
	\item \label{Frage:Definitionen} \defFt{Definitionen}:
	Was ist mit einem \Fachbegriff\ jeweils genau gemeint?
	\textZB\ \emph{Stetigkeit}, \emph{Integral} und \emph{Analysis}.
	%
	\item \label{Frage:Abhaengigkeiten} \defFt{Abhängigkeiten}:
	Wie heißt ein \Fachbegriff\ in einer anderen Sprache?
	Ist wirklich dasselbe gemeint?
	Was ist mit \Fachbegriffen\ in verschiedenen \Fachgebieten?
	%
	\item \label{Frage:Ueberblick} \defFt{Überblick}:
	Ist ein \Axiom, \Satz, \Beweis\ oder \Fachbegriff\ schon einmal --- \textggf\ abweichend --- definiert, formuliert oder bewiesen worden?
	%
	\item \label{Frage:Darstellung} \defFt{\Darstellung}:
	Wie kann man einen \Satz\ und den zugehörigen \Beweis\ --- \textggf\ auch spezifisch für ein \Fachgebiet\ --- darstellen?
	%
	\item \label{Frage:Forschung} \defFt{Forschung}:
	Welche Probleme gibt es noch zu erforschen.
	%
\end{enumerate}

\section     {Eigenschaften}% ==================================================
\beginsection{Eigenschaften}
\label   {sec:Eigenschaften}

\ASBA\ soll ausgehend von den Fragen in \vrefsec{sec:Fragen} entwickelt werden, und die folgenden Eigenschaften haben:
\begin{enumerate}
	%
	\item \label{Eigenschaft:Daten} \defFt{Daten}:
	\Axiome, \Saetze, \Beweise, \Fachbegriffe\ und \Fachgebiete\ können in formaler Form gespeichert werden --- auch (noch) nicht oder unvollständig bewiesene \Saetze.
	Dabei soll die übliche mathematische Schreibweise verwendet werden können.
	%
	\item \label{Eigenschaft:Definitionen} \defFt{Definitionen}:
	Es können \Fachbegriffe\ für \Axiome, \Saetze, \Beweise\ und \Fachgebiete\ --- letztere mit eigenen \Axiomen, \Saetzen, \Beweisen, \Fachbegriffen\ und über- oder untergeordneten \Fachgebieten\ --- definiert werden.
	Die Definitionen dürfen wiederum an dieser Stelle schon bekannte \Fachbegriffe\ und \Fachgebiete\ verwenden.
	%
	\item \label{Eigenschaft:Pruefung} \defFt{Prüfung}:
	Vorhandene \Beweise\ können automatisch geprüft werden.
	%
	\item \label{Eigenschaft:Ausgaben} \defFt{Ausgaben}:
	Die \Axiome, \Saetze\ und \Beweise\ können in üblicher Schreibweise --- abhängig von Sprache und \Fachgebiet\ --- ausgegeben werden.
	%
	\item \label{Eigenschaft:Auswertungen} \defFt{Auswertungen}:
	Zusätzlich zur Ausgabe der gespeicherten Daten sind verschiedene Auswertungen möglich, unter anderem für die meisten der unter \vrefsec{sec:Fragen} behandelten Fragen.
	%
	\setcounter{Enumi}{\value{enumi}}% Nummerierung wird fortgesetzt.
\end{enumerate}
%
Damit \ASBA\ nicht umsonst erstellt wird und möglichst breite Verwendung findet, werden noch zwei Punkte angefügt:
\begin{enumerate}
	\setcounter{enumi}{\value{Enumi}}% Nummerierung wird fortgesetzt.
	%
	\item \label{Eigenschaft:Lizenz} \defFt{Lizenz}:
	Die Software ist \emph{Open Source}.
	%
	\item \label{Eigenschaft:Akzeptanz} \defFt{Akzeptanz}:
	\ASBA\ wird von Mathematikern akzeptiert und verwendet.
\end{enumerate}
%
\vreftab{tab:Fragen2Eigenschaften} zeigt, wie sich die Eigenschaften zu den Fragen \vrefinsec{sec:Fragen} verhalten.
Mit einem X werden die Spalten einer Zeile markiert, deren zugehörige Eigenschaften zur Beantwortung der entsprechenden Frage beitragen sollen.
Idealerweise sollte die Erfüllung aller angegebenen Eigenschaften alle gestellten Fragen beantworten, was allerdings illusorisch ist.
%
% Abstände für die nächsten drei Tabellen
\newcommand*{\vsL}{\hspace{-1.0cm}}  % für 1-stellige Zahlen
\newcommand*{\vsl}{\hspace{-6pt}\vsL}% für 2-stellige Zahlen
\newcommand*{\vsc}{\hspace{6pt}}     % für die gedrehten Überschriften
%
\begin{table}[H]
	\begin{tabularx}{\linewidth}
		{@{\hspace{.5cm}}rl@{\extracolsep{\fill}}|*{7}{c}@{\hspace{1cm}}|}
		\multicolumn{2}{l|}{\diagbox[height=3.0cm,width=4.5cm]%
			{\textbf{Frage}\\~}{\\\textbf{Eigenschaft}}}
		&\rotatebox{90}{%
			\mbox{\vsL\ref{Eigenschaft:Daten}        \vsc Daten        }}
		&\rotatebox{90}{%
			\mbox{\vsL\ref{Eigenschaft:Definitionen} \vsc Definitionen }}
		&\rotatebox{90}{%
			\mbox{\vsL\ref{Eigenschaft:Pruefung}     \vsc Prüfung      }}
		&\rotatebox{90}{%
			\mbox{\vsL\ref{Eigenschaft:Ausgaben}     \vsc Ausgaben     }}
		&\rotatebox{90}{%
			\mbox{\vsL\ref{Eigenschaft:Auswertungen} \vsc Auswertungen }}
		&\rotatebox{90}{%
			\mbox{\vsL\ref{Eigenschaft:Lizenz}       \vsc Lizenz       }}
		&\rotatebox{90}{%
			\mbox{\vsL\ref{Eigenschaft:Akzeptanz}    \vsc Akzeptanz    }}
		\\\hline
		\ref{Frage:Grundlagen}      & Grundlagen
		& X & X & - & X & X & - & - \\
		\ref{Frage:Basis}           & Basis
		& X & X & - & X & X & - & - \\
		\ref{Frage:Axiome}          & \Axiome
		& X & X & - & X & X & - & - \\
		\hdashline[2pt/2pt]
		\ref{Frage:Beweis}          & \Beweis
		& X & - & X & X & - & - & - \\
		\ref{Frage:Konstruktion}    & Konstruktion
		& X & - & - & X & - & - & - \\
		\ref{Frage:Vergleiche}      & Vergleiche
		& X & - & - & - & X & - & - \\
		\hdashline[2pt/2pt]
		\ref{Frage:Definitionen}    & Definitionen
		& X & X & - & X & - & - & - \\
		\ref{Frage:Abhaengigkeiten} & Abhängigkeiten
		& X & - & - & X & - & - & - \\
		\ref{Frage:Ueberblick}      & Überblick
		& X & - & - & - & X & - & - \\
		\hdashline[2pt/2pt]
		\ref{Frage:Darstellung}     & \Darstellung
		& - & X & - & X & - & - & - \\
		\ref{Frage:Forschung}       & Forschung
		& X & - & - & - & X & - & - \\
		\hline
	\end{tabularx}
	\caption{%
		Fragen (\ref{sec:Fragen}) $\to$ Eigenschaften (\ref{sec:Eigenschaften})
	}
	\label{tab:Fragen2Eigenschaften}% Erst nach '\caption'!
\end{table}

\section[Ziele]{\Ziele}% =======================================================
\beginsection  {\Ziele}
\label      {sec:Ziele}

Um die Eigenschaften von \vrefsec{sec:Eigenschaften} zu erreichen, werden für \ASBA\ die folgenden \Ziele%
\footnote{%
	Es sind eigentlich Anforderungen.
	Diese \Bezeichnung\ wird aber schon \vrefincha{cha:Design} verwendet.
}
gesetzt:
\begin{enumerate}
	%
	\item \label{Ziel:Daten} \defFt{Daten}:
	Die verteilte Datenbank von \ASBA\ enthält möglichst viele wichtige \Axiome, \Saetze, \Beweise, \Fachbegriffe, \Fachgebiete\ und \Ausgabeschemata%
	\footnote{%
		Um den Punkt~\ref{Eigenschaft:Ausgaben} \vrefvonsec{sec:Eigenschaften} erfüllen zu können, werden noch fachgebietsspezifische \Ausgabeschemata\ benötigt, welche die Art der Ausgaben beschreiben.
	}.
	%
	\item \label{Ziel:Form} \defFt{Form}:
	Die Daten liegen in formaler, geprüfter Form vor.
	%
	\item \label{Ziel:Eingaben} \defFt{Eingaben}:
	Die Eingabe von Daten erfolgt in einer formalen \Syntax\ unter Verwendung der üblichen mathematischen Schreibweise.
	%
	\item \label{Ziel:Pruefung} \defFt{Prüfung}:
	\Beweise\ können automatisch geprüft\footnote{%
		An dieser Stelle soll \ASBA\ soll keine \Beweise\ finden --- das ist \Ziel\ von Punkt \ref{Ziel:Beweisunterstuetzung}, sondern nur vorhandene prüfen.
	}
	werden.
	%
	\item \label{Ziel:Ausgaben} \defFt{Ausgaben}:
	Die Ausgabe kann in einer eindeutigen, formalen \Syntax\ gemäß vorhandener \Ausgabeschemata\ erfolgen.
	%
	\item \label{Ziel:Auswertungen} \defFt{Auswertungen}:
	Zusätzlich zur Ausgabe der Daten sind verschiedene Auswertungen möglich.
	Insbesondere kann zu jedem \Beweis\ angegeben werden, wie lang er ist und welche \Axiome\ und \Saetze%
	\footnote{%
		\Saetze, die quasi als \Axiome\ verwendet werden.
	}
	er benötigt.
	%
	\item \label{Ziel:Anpassbarkeit} \defFt{Anpassbarkeit}:
	\Fachbegriffe\ und die \Darstellung\ bei der Ausgabe können mit Hilfe von --- gegebenenfalls unbenannten --- untergeordneten \Fachgebieten\ angepasst werden.
	%
	\item \label{Ziel:Individualitaet} \defFt{Individualität}:
	\Axiome\ und \Saetze\ können für jeden \Beweis\ individuell vorausgesetzt werden.
	Dabei sind fachgebietsspezifische \Fachbegriffe\ erlaubt.
	%
	\item \label{Ziel:Internet} \defFt{Internet}:
	Die Daten können auf mehrere Dateien verteilt sein.
	Ein Teil davon --- oder sogar alle --- können im Internet liegen.
	%
	\item \label{Ziel:Kommunikation} \defFt{Kommunikation}:
	Die Kommunikation mit \ASBA\ kann mit den \Fachbegriffen\ der einzelnen \Fachgebiete\ erfolgen.
	%
	\item \label{Ziel:Zugriff} \defFt{Zugriff}:
	Der Zugriff auf \ASBA\ kann lokal und über das Internet erfolgen.
	%
	\item \label{Ziel:Unabhaengigkeit} \defFt{Unabhängigkeit}:
	\ASBA\ kann online und offline arbeiten.
	%
	\item \label{Ziel:Rekursion} \defFt{Rekursion}:
	Es kann rekursiv über alle verwendeten Dateien --- auch solchen, die im Internet liegen --- ausgewertet werden.
	%
	\item \label{Ziel:Bedienbarkeit} \defFt{Bedienbarkeit}:
	\ASBA\ ist einfach zu bedienen.
	%
	\item \label{Ziel:Lizenz} \defFt{Lizenz}:
	Die Software ist \emph{Open Source}.
	%
	\item \label{Ziel:Zwischenspeicher} \defFt{Zwischenspeicher}:
	Wichtige Auswertungen können an vorhandenen Dateien angehängt oder separat in eigenen Dateien gespeichert werden.
	%
	\item \label{Ziel:Beweisunterstuetzung} \defFt{Beweisunterstützung}:
	\ASBA\ hilft bei der Erstellung von \Beweisen.
	%
\end{enumerate}
%
Punkt \ref{Ziel:Zwischenspeicher} wurde noch angefügt, damit \ASBA\ effizient arbeiten kann und um die Akzeptanz zu erhöhen.
Um letzteres zu erreichen, dafür ist auch Punkt \ref{Ziel:Beweisunterstuetzung} nützlich.
Es bietet sich ja auch an, die Fähigkeiten, die \ASBA\ mit der Prüfung von Beweisen haben wird, auch auf die Erstellung von Beweisen anzuwenden.
Die Reihenfolge der \Ziele\ stellt noch keine Priorisierung fest.

\vrefDtab{tab:Eigenschaften2Ziele} zeigt wieder, wie sich die Ziele zu den Eigenschaften \vrefinsec{sec:Eigenschaften} verhalten.
Mit einem X werden wieder die Spalten einer Zeile markiert, deren zugehörige Ziele zur Sicherstellung der entsprechenden Eigenschaft beitragen sollen.
Idealerweise sollte durch Erreichen aller aufgestellten Ziele \ASBA\ alle angegebenen Eigenschaften aufweisen, was wahrscheinlich ebenfalls illusorisch ist.
%
\begin{table}[H]
	\begin{tabularx}{\linewidth}
		{@{\hspace{.2cm}}rl@{\extracolsep{\fill}}|*{17}{c}@{\hspace{0.2cm}}|}
		\multicolumn{2}{l|}{\diagbox[height=3.0cm,width=3.6cm]%
			{\textbf{Eigenschaft}\\~}{\\\\\textbf{Ziel}}}
		&\rotatebox{90}{%
			\mbox{\vsL\ref{Ziel:Daten}                \vsc Daten              }}
		&\rotatebox{90}{%
			\mbox{\vsL\ref{Ziel:Form}                 \vsc Form               }}
		&\rotatebox{90}{%
			\mbox{\vsL\ref{Ziel:Eingaben}             \vsc Eingaben           }}
		&\rotatebox{90}{%
			\mbox{\vsL\ref{Ziel:Pruefung}             \vsc Prüfung            }}
		&\rotatebox{90}{%
			\mbox{\vsL\ref{Ziel:Ausgaben}             \vsc Ausgaben           }}
		&\rotatebox{90}{%
			\mbox{\vsL\ref{Ziel:Auswertungen}         \vsc Auswertungen       }}
		&\rotatebox{90}{%
			\mbox{\vsL\ref{Ziel:Anpassbarkeit}        \vsc Anpassbarkeit      }}
		&\rotatebox{90}{%
			\mbox{\vsL\ref{Ziel:Individualitaet}      \vsc Individualität     }}
		&\rotatebox{90}{%
			\mbox{\vsL\ref{Ziel:Internet}             \vsc Internet           }}
		&\rotatebox{90}{%
			\mbox{\vsl\ref{Ziel:Kommunikation}        \vsc Kommunikation      }}
		&\rotatebox{90}{%
			\mbox{\vsl\ref{Ziel:Zugriff}              \vsc Zugriff            }}
		&\rotatebox{90}{%
			\mbox{\vsl\ref{Ziel:Unabhaengigkeit}      \vsc Unabhängigkeit     }}
		&\rotatebox{90}{%
			\mbox{\vsl\ref{Ziel:Rekursion}            \vsc Rekursion          }}
		&\rotatebox{90}{%
			\mbox{\vsl\ref{Ziel:Bedienbarkeit}        \vsc Bedienbarkeit      }}
		&\rotatebox{90}{%
			\mbox{\vsl\ref{Ziel:Lizenz}               \vsc Lizenz             }}
		&\rotatebox{90}{%
			\mbox{\vsl\ref{Ziel:Zwischenspeicher}     \vsc Zwischenspeicher   }}
		&\rotatebox{90}{%
			\mbox{\vsl\ref{Ziel:Beweisunterstuetzung} \vsc Beweisunterstützung}}
		\\\hline
		\ref{Eigenschaft:Daten}         & Daten%
		& X & X & X & - & - & - & - & - & - & - & - & - & - & - & - & - & - \\
		\ref{Eigenschaft:Definitionen}  & Definitionen%
		& X & - & X & - & - & - & - & - & - & - & - & - & - & - & - & - & - \\
		\ref{Eigenschaft:Pruefung}      & Prüfung
		& - & - & - & X & - & - & - & - & - & - & - & - & - & - & - & - & - \\
		\hdashline[2pt/2pt]
		\ref{Eigenschaft:Ausgaben}      & Ausgaben%
		& - & - & - & - & X & - & - & - & - & - & - & - & - & - & - & - & - \\
		\ref{Eigenschaft:Auswertungen}  & Auswertungen%
		& - & - & - & - & - & X & - & - & - & - & - & - & - & - & - & - & - \\
		\ref{Eigenschaft:Lizenz}        & Lizenz%
		& - & - & - & - & - & - & - & - & - & - & - & - & - & - & X & - & - \\
		\hdashline[2pt/2pt]
		\ref{Eigenschaft:Akzeptanz}     & Akzeptanz%
		& X & X & X & X & X & X & X & X & X & X & X & X & X & X & X & X & X \\
		\hline
	\end{tabularx}
	\caption{%
		Eigenschaften (\ref{sec:Eigenschaften}) $\to$ Ziele (\ref{sec:Ziele})
	}
	\label{tab:Eigenschaften2Ziele}% Erst nach '\caption'!
\end{table}

\section     {Zusammenfassung}% ================================================
\beginsection{Zusammenfassung}
\label   {sec:Zusammenfassung}

\begin{table}[H]
	\begin{tabularx}{\linewidth}
		{@{\hspace{.2cm}}rl@{\extracolsep{\fill}}|*{17}{c}@{\hspace{0.2cm}}|}
		\multicolumn{2}{l|}{\diagbox[height=3.0cm,width=4.0cm]%
			{\textbf{Frage}\\~}{\\\\\textbf{Ziel}}}
		&\rotatebox{90}{%
			\mbox{\vsL\ref{Ziel:Daten}                \vsc Daten              }}
		&\rotatebox{90}{%
			\mbox{\vsL\ref{Ziel:Form}                 \vsc Form               }}
		&\rotatebox{90}{%
			\mbox{\vsL\ref{Ziel:Eingaben}             \vsc Eingaben           }}
		&\rotatebox{90}{%
			\mbox{\vsL\ref{Ziel:Pruefung}             \vsc Prüfung            }}
		&\rotatebox{90}{%
			\mbox{\vsL\ref{Ziel:Ausgaben}             \vsc Ausgaben           }}
		&\rotatebox{90}{%
			\mbox{\vsL\ref{Ziel:Auswertungen}         \vsc Auswertungen       }}
		&\rotatebox{90}{%
			\mbox{\vsL\ref{Ziel:Anpassbarkeit}        \vsc Anpassbarkeit      }}
		&\rotatebox{90}{%
			\mbox{\vsL\ref{Ziel:Individualitaet}      \vsc Individualität     }}
		&\rotatebox{90}{%
			\mbox{\vsL\ref{Ziel:Internet}             \vsc Internet           }}
		&\rotatebox{90}{%
			\mbox{\vsl\ref{Ziel:Kommunikation}        \vsc Kommunikation      }}
		&\rotatebox{90}{%
			\mbox{\vsl\ref{Ziel:Zugriff}              \vsc Zugriff            }}
		&\rotatebox{90}{%
			\mbox{\vsl\ref{Ziel:Unabhaengigkeit}      \vsc Unabhängigkeit     }}
		&\rotatebox{90}{%
			\mbox{\vsl\ref{Ziel:Rekursion}            \vsc Rekursion          }}
		&\rotatebox{90}{%
			\mbox{\vsl\ref{Ziel:Bedienbarkeit}        \vsc Bedienbarkeit      }}
		&\rotatebox{90}{%
			\mbox{\vsl\ref{Ziel:Lizenz}               \vsc Lizenz             }}
		&\rotatebox{90}{%
			\mbox{\vsl\ref{Ziel:Zwischenspeicher}     \vsc Zwischenspeicher   }}
		&\rotatebox{90}{%
			\mbox{\vsl\ref{Ziel:Beweisunterstuetzung} \vsc Beweisunterstützung}}
		\\\hline
		\ref{Frage:Grundlagen}      & Grundlagen%
		& X & X & X & - & X & X & x & - & - & - & - & - & - & - & - & - & - \\
		\ref{Frage:Basis}           & Basis%
		& X & X & X & - & X & X & x & x & - & - & - & - & - & - & - & - & - \\
		\ref{Frage:Axiome}          & \Axiome%
		& X & X & X & - & X & X & x & - & - & - & - & - & - & - & - & - & - \\
		\hdashline[2pt/2pt]
		\ref{Frage:Beweis}          & \Beweis%
		& X & X & X & X & X & - & - & x & - & - & - & - & - & - & - & - & - \\
		\ref{Frage:Konstruktion}    & Konstruktion%
		& X & X & X & - & X & - & - & x & - & - & - & - & - & - & - & - & - \\
		\ref{Frage:Vergleiche}      & Vergleiche%
		& X & X & X & - & - & X & - & x & - & - & - & - & - & - & - & - & - \\
		\hdashline[2pt/2pt]
		\ref{Frage:Definitionen}    & Definitionen%
		& X & X & X & - & X & - & x & - & - & - & - & - & - & - & - & - & - \\
		\ref{Frage:Abhaengigkeiten} & Abhängigkeiten%
		& X & X & X & - & X & - & x & - & - & - & - & - & - & - & - & - & - \\
		\ref{Frage:Ueberblick}      & Überblick%
		& X & X & X & - & - & X & x & - & - & - & - & - & - & - & - & - & - \\
		\hdashline[2pt/2pt]
		\ref{Frage:Darstellung}     & \Darstellung%
		& X & - & X & - & X & - & x & - & - & - & - & - & - & - & - & - & - \\
		\ref{Frage:Forschung}       & Forschung%
		& X & X & X & - & - & X & x & - & - & - & - & - & - & - & - & - & - \\
		\hline
		\multicolumn{19}{l|}{Die nächsten beiden Punkte
			sind Eigenschaften aus \vrefsec{sec:Eigenschaften}:}\\
		\hline
		\ref{Eigenschaft:Lizenz}    & Lizenz%
		& - & - & - & - & - & - & - & - & - & - & - & - & - & - & X & - & - \\
		\ref{Eigenschaft:Akzeptanz} & Akzeptanz%
		& X & X & X & X & X & X & X & X & X & X & X & X & X & X & X & X & X \\
		\hline
	\end{tabularx}
	\caption{Fragen (\ref{sec:Fragen}) $\to$ Ziele (\ref{sec:Ziele})}
	\label{tab:Fragen2Ziele}% Erst nach '\caption'!
\end{table}
%
\vrefDtab{tab:Fragen2Ziele} ist eine Kombination der Tabellen~ \ref{tab:Fragen2Eigenschaften} und~\ref{tab:Eigenschaften2Ziele} und zeigt, wie sich die Ziele \vrefinsec{sec:Ziele} zu den Fragen \vrefinsec{sec:Fragen} verhalten.
Auch hier werden mit einem X die Spalten einer Zeile markiert, deren zugehörige Ziele für die Beantwortung der entsprechenden Frage nötig sind.
Mit einem kleinen x werden sie markiert, wenn sie zur Beantwortung der Fragen nicht nötig, aber von Interesse sind.
Idealerweise sollte das Erreichen aller aufgestellten Ziele alle gestellten Fragen beantworten, was natürlich auch illusorisch ist.

\clearpage

\section[Die Umgebung von \glsentrytext{ASBA}]{Die Umgebung von \ASBA}%
\beginsection                                 {Die Umgebung von \ASBA}
\label                                        {sec:Umgebung}

\vrefInfig{fig:Umgebung} wird beschrieben, welche Interaktionen \ASBA\ mit der Umgebung hat, \textdh\ welche Ein- und Ausgaben existieren und woher sie kommen \textbzw\ wohin sie gehen.

\begin{figure}[H]
	\setlength\unitlength{1cm}
	\begin{picture}(17.0,9.5)(-8.4,-4.5)
		% Hilfsgitter während der Bildbearbeitung
		%\color{lightgray}
		%\multiput(-8.4,-4.5)(+0.0,1.0){10}{\line(1,0){17.0}}
		%\multiput(-7.9,-4.5)(+1.0,0.0){17}{\line(0,1){ 9.5}}
		\linethickness{1.5pt}
		% Hintergrund (grau) ===============================================
		\color{gray}
		% rechts: externes ASBA mit Pfeilen --------------------------------
		\put(+3.00,+0.50){\framebox(2.40,1.60){\huge\textbild{\ASBA}}}
		\put(+3.00,+0.50){\makebox(2.40,1.50)[t]{\textbild{externes}}}
		\put(+4.00,+2.12){\vector(-1,+4){0.35}}% <--- externes ASBA
		\put(+4.00,+3.55){\vector(+1,-4){0.36}}% ---> externes ASBA
		\put(+3.81,+2.82){\marker{a}}
		% rechts oben: externe Datenbank mit Pfeilen -----------------------
		\put(+7.30,+3.80){\Datenbank{1.20}{0.40}{0.80}{\small externe}{\small Datenbank}}
		\put(+5.41,+2.10){\vector(+1,+2){0.70}}% <--- externes ASBA
		\put(+6.14,+2.89){\vector(-1,-2){0.72}}% ---> externes ASBA
		\put(+5.60,+2.48){\marker{b}}
		% Verbindung Auswertungen ---> Männchen ----------------------------
		\put(+5.60,-3.30){\vector(-1,0){4.05}}% Auswertungen ---> Männchen
		\put(+3.50,-3.30){\marker{c}}
		% Verbindung Männchen <---> Terminal -------------------------------
		\put(-2.00,-3.00){\vector(+1,0){2.45}}% Männchen <--- Terminal
		\put(+0.40,-3.30){\vector(-1,0){2.40}}% Männchen ---> Terminal
		\put(-0.75,-3.15){\marker{d}}
		% Verbindung Terminal <---> Datei ----------------------------------
		\put(-5.50,-1.50){\vector(+3,-2){1.50}}% Terminal <--- Datei
		\put(-4.01,-2.80){\vector(-3,+2){1.60}}% Terminal ---> Datei
		\put(-5.00,-2.10){\marker{e}}
		% Vordergrund (schwarz) ============================================
		\color{black}
		% rechts oben: Wolke mit Pfeilen -----------------------------------
		\put(+3.40,+4.50){\Wolke{Internet}}
		\put(+2.00,+4.04){\vector(-1,-3){0.82}}% ---> ASBA
		\put(+1.53,+1.53){\vector(+1,+3){0.75}}% <--- ASBA
		\put(+1.55,+2.70){\Marker{1}}
		% links oben: Datenbank mit Pfeilen --------------------------------
		\put(-7.00,+3.50){\Datenbank{1.50}{0.50}{1.00}{\large \ASBA}{\large Datenbank}}
		\put(-5.50,+3.75){\vector(+7,-4){3.95}}% ---> ASBA
		\put(-1.51,+1.10){\vector(-7,+4){4.00}}% <--- ASBA
		\put(-3.70,+2.40){\Marker{2}}
		% links Mitte: Datei mit Pfeilen -----------------------------------
		\put(-7.00,-1.00){\Datei{3.00}{2.00}{\ASBA}{Datei}}
		\put(-5.50,-0.80){\vector(+4,+1){3.95}}% ---> ASBA
		\put(-1.51,-0.10){\vector(-4,-1){4.00}}% <--- ASBA
		\put(-3.70,-0.45){\Marker{3}}
		%links unten: Rechner mit Pfeilen ----------------------------------
		\put(-3.00,-3.10){\Terminal{Terminal}}
		\put(-2.50,-2.48){\vector(+1,+2){0.98}}% ---> ASBA
		\put(-1.09,-0.52){\vector(-1,-2){0.98}}% <--- ASBA
		\put(-2.00,-1.50){\Marker{4}}
		% Mitte unten: Männchen mit Pfeilen --------------------------------
		\put(+1.00,-2.80){\Maennchen}
		\put(+0.85,-2.50){\vector(0,+1){2.00}}% ---> ASBA
		\put(+1.15,-0.51){\vector(0,-1){2.00}}% <--- ASBA
		\put(+0.80,-1.52){\Marker{5}}
		% rechts unten: Papier mit Pfeil -----------------------------------
		\put(+5.60,-4.20){\Papier{+2.00}{+0.30}{\ASBA}{Ausgabe}}
		\put(+1.51,-0.55){\vector(+2,-1){+4.10}}% <--- ASBA
		\put(+3.25,-1.55){\Marker{6}}
		% Mitte: ASBA ------------------------------------------------------
		\linethickness{3pt}
		\put(-1.5,-0.5){\framebox(3.0,2.0){\Huge\textbild{\ASBA}}}
	\end{picture}
	\caption{Die Umgebung von \ASBA}
	\label{fig:Umgebung}% Erst nach '\caption'!
\end{figure}

In den \vrefinfig{fig:Umgebung} abgebildeten Datenflüssen (1) bis (6) und (a) bis (e) werden die folgenden Daten übertragen:
\begin{itemize}
	\newcommand*{\vonnach}  [2]{#1 $\rightarrow$ #2}
	\newcommand*{\nachvon}  [2]{\vonnach{#2}{#1}}
	\newcommand*{\hinundher}[2]{#1 $\leftrightarrow$ #2}
	%
	\item[(1)]\label{dat:Internet}
	\begin{description}
		\item[\vonnach{\ASBA}{Internet}]\label{dat:ausInternet}
		Inhalte der Datenbank.
		\item[\nachvon{\ASBA}{Internet}]\label{dat:inInternet}
		Inhalte der externen Datenbank.
	\end{description}
	%
	\item[(2)]\label{dat:Datenbank}
	\begin{description}
		\item[\vonnach{Datenbank}{\ASBA}]\label{dat:ausDatenbank}
		Inhalte der Datenbank und Antworten auf Datenbankanweisungen.
		\item[\nachvon{Datenbank}{\ASBA}]\label{dat:inDatenbank}
		Inhalte der Datei, der externen Datenbank und Datenbankanweisungen.
	\end{description}
	%
	\item[(3)]\label{dat:Datei}
	\begin{description}
		\item[\vonnach{Datei}{\ASBA}]\label{dat:ausDatei}
		Inhalte der Datei.
		\item[\nachvon{Datei}{\ASBA}]\label{dat:inDatei}
		Die Datei wird um zusätzliche Auswertungen ergänzt, \textzB\ ob die \Beweise\ korrekt sind, welche \Axiome\ und \Saetze\ --- auch externe aus dem Internet --- verwendet wurden, Länge des \Beweises\ usw.
	\end{description}
	%
	\item[(4)]\label{dat:Terminal}
	\begin{description}
		\item[\vonnach{Terminal}{\ASBA}]\label{dat:ausTerminal}
		Anweisungen, Daten und Batchprogramme.
		\item[\nachvon{Terminal}{\ASBA}]\label{dat:inTerminal}
		Antworten auf Anweisungen, Auswertungen usw.
	\end{description}
	Außerdem interaktive Ein- und Ausgabe durch einen Anwender, wie in (5) beschrieben.
	%
	\item[(5)]\label{dat:Anwender}
	\begin{description}
		\item[\hinundher{Anwender}{\ASBA}]\label{dat:mitAnwender}
		Interaktive Ein- und Ausgaben durch einen Anwender mit Komponenten von (3), (4) und (6).
		--- Die Kommunikation läuft \textiAlg\ über ein Terminal.
	\end{description}
	%
	\item[(6)]\label{dat:Ausgabe}
	\begin{description}
		\item[\nachvon{Ausgabe}{\ASBA}]\label{dat:inAusgabe}
		Inhalte von Datei und Datenbank in lesbarer Form, \textua\ mit Hilfe von \Ausgabeschemata\ auch mit \Formeln.
		Die Ausgabe kann auch in eine Datei erfolgen,
		\textzB\ im \LaTeX-Format.
	\end{description}
	%
	\item[(a)]\label{dat:extInternet}
	\begin{description}
		\item[\vonnach{Internet}{externes \ASBA}]\label{dat:ausextInternet}
		Inhalte der Datenbank.
		\item[\nachvon{Internet}{externes \ASBA}]\label{dat:inextInternet}
		Inhalte der externen Datenbank.
	\end{description}
	%
	\item[(b)]\label{dat:extDatenbank}
	\begin{description}
		\item[\vonnach{externe Datenbank}{externes \ASBA}]
		\label{dat:ausextDatenbank} Inhalte der externen Datenbank.
		\item[\nachvon{externe Datenbank}{externes \ASBA}]
		\label{dat:inextDatenbank} Inhalte der Datenbank.
	\end{description}
	%
	\item[(c)]\label{dat:AusgabeAnwender}
	\begin{description}
		\item[\vonnach{Ausgabe}{Anwender}]\label{dat:Ausgabe2Anwender}
		Alle Daten der Ausgabe.
	\end{description}
	%
	\item[(d)] \label{dat:AnwenderTerminal}
	\begin{description}
		\item[\hinundher{Anwender}{Terminal}]\label{dat:Anwender22Terminal}
		Interaktive Ein- und Ausgabe durch einen Anwender, wie in (5) beschrieben.
	\end{description}
	%
	\item[(e)] \label{dat:TerminalDatei}
	\begin{description}
		\item[\hinundher{Terminal}{Datei}]\label{dat:Terminal22Datei}
		Erstellen und Bearbeiten der Datei durch einen Anwender.
		--- siehe (d)
	\end{description}
	%
\end{itemize}
Die Datenflüsse (a) bis (e) erfolgen außerhalb von \ASBA\ und werden nicht weiter behandelt.

Die Datenbank und die Datei enthalten im Prinzip die gleichen Daten, wobei sie in der Datei im Textformat in lesbarer Form und in der Datenbank in einem internen Format vorliegen.
Zudem enthält die Datenbank \textiAlg\ sehr viel mehr Daten. Es handelt sich dabei jeweils um die folgenden Daten:
\begin{description}
	\item[\Axiome]         \label{Daten:Axiom}         \AxiomDescription
	\item[\Saetze]         \label{Daten:Satz}          \SatzDescription
	\item[\Beweise]        \label{Daten:Beweis}        \BeweisDescription
	\item[\Fachbegriffe]   \label{Daten:Fachbegriff}   \FachbegriffDescription
	\item[\Fachgebiete]    \label{Daten:Fachgebiet}    \FachgebietDescription
	\item[\Ausgabeschemata]\label{Daten:Ausgabeschema} \AusgabeschemaDescription
	\item[\Auswertungen]   \label{Daten:Auswertung}    \AuswertungDescription
\end{description}
Alle Daten können interne und externe Verweise enthalten.

\color{gray}%%% Anfang grauer Text
\section[Basis von Beweisen]{Basis von \Beweisen}% =============================
\beginsection               {Basis von \Beweisen}
\label                             {sec:BeweisBasis}

Da ein Computerprogramm erstellt werden soll, muss die Grundstruktur des Vorgehens bei \Beweisen\ definiert werden.%
\footnote{\seename~\cite{bib:Kalkuel}}

\begin{description}
	%
	\item[Die \logischeDarstellung] von mathematischen \Aussagen, wozu auch \Axiome\ und \Saetze\ gehören, erfolgt, da es sich immer um \Formeln\ handelt, an besten mit \Symbolfolgen%
	\footnote{%
		Die \interneDarstellung\ der \Symbolfolgen\ kann zur Optimierung von \ASBA\ von der \logischenD\ abweichen.
	},
	\textdh\ Folgen von Zeichen und Symbolen, in denen Zwischenraum --- insbesondere Leerstellen --- nicht zählen.
	Mehrdimensionale \Formeln, wie \textzB\ Matrizen, Baumstrukturen, Funktionsschemata und anderes, können auch als (eindimensionale) Symbolfolgen dargestellt werden.%
	\footnote{%
		\textZB\ könnte man eine 2$\times$2-Matrix
		$\begin{bmatrix} a & b \\ c & d \end{bmatrix}$
		auch darstellen als Folge von Zeilen: \seqqt{$[(a,b),(c,d)]$}, oder noch einfacher: \seqqt{$[a,b;c,d]$}.
		In \ASBA\ wird die \LaTeX-Syntax verwendet.
		\\Damit wird die soeben angegebene Matrix codiert durch \seqqt{\$\textbackslash begin\{bmatrix\}a\&b\textbackslash\textbackslash c\&d\textbackslash end\{bmatrix\}\$}.
	}
	\Beweise\ sind letztendlich nichts anderes, als erlaubte \Transformationen\ dieser \Symbolfolgen.
	%
	\item[\Bausteine] sind Grundelemente, auch \defFt{Zeichen} oder \defFt{(Satz-)Buchstaben} genannt, aus denen die Symbolfolgen bestehen dürfen, und müssen definiert werden.
	%
	\item[\Formationsregeln] dienen zur Festlegung, wie man aus den Bausteinen Ausdrücke erzeugen kann, und müssen ebenfalls definiert werden.
	%
	\item[\Saetze] lassen sich als eine \Menge\ von \Formeln, den \Praemissen, wozu auch \Axiome\ und andere \Saetze\ gehören können, einer weiteren \Menge\ von \Formeln\ (\Symbolfolgen), den \Konklusionen, und der Angabe eines \Beweises\ darstellen.
	%
	\item[\Beweise] zu gegebenen \Praemissen\ und \Konklusionen\ lassen sich als \Folge\ von \Transformationen, beginnend mit den \Praemissen\ und endend mit den \Konklusionen, darstellen.
	%
	\item[\Transformationsregeln] definieren, welche \Transformationen\ mit gegebenen \Formelmengen\ zulässig sind.%
	\footnote{\seename~\cite{bib:Rautenberg,bib:Schlussregel,bib:NatuerlichesSchliessen}}
	%
\end{description}
\color{black}%%% Ende grauer Text

\Endchapter

	%%############################################################################%%
%%                                                                            %%
%% Datei:  ASBA-Mathematik.tex                                                %%
%% Inhalt: Kapitel "Mathematische Grundlagen"                                 %%
%%                                                                            %%
%% Copyright (C) 2017  Winfried Teschers                                      %%
%%                                                                            %%
%% This program is free software: you can redistribute it and/or modify       %%
%% it under the terms of the GNU Affero General Public License as published   %%
%% by the Free Software Foundation, either version 3 of the License, or       %%
%% (at your option) any later version.                                        %%
%%                                                                            %%
%% This program is distributed in the hope that it will be useful,            %%
%% but WITHOUT ANY WARRANTY; without even the implied warranty of             %%
%% MERCHANTABILITY or FITNESS FOR A PARTICULAR PURPOSE.  See the              %%
%% GNU Affero General Public License for more details.                        %%
%%                                                                            %%
%% You should have received a copy of the GNU Affero General Public License   %%
%% along with this program.  If not, see <http://www.gnu.org/licenses/>.      %%
%%                                                                            %%
%% Dr. Winfried Teschers                                                      %%
%% Anton-Günther-Straße 26c                                                   %%
%% 91083 Baiersdorf                                                           %%
%% Germany                                                                    %%
%%                                                                            %%
%% e-mail: winfried.teschers@t-online.de                                      %%
%%                                                                            %%
%%############################################################################%%

% !TeX root = ASBA.tex
% !TeX encoding = UTF-8
% !TeX spellcheck = de_DE

\chapter     {Mathematische Grundlagen}% #######################################
\beginchapter{Mathematische Grundlagen}
\label                 {cha-Grundlagen}

Die mathematischen Grundlagen werden einerseits gebraucht, um die erlaubten \Beweisschritte\vrefnotesec{sub-Beweisschritte} zu definieren, andererseits dienen sie auch zum Testen von \ASBA.
Daher werden sie in \datcha{cha-Grundlagen} ausführlicher behandelt, als für die Erstellung von \ASBA\ erforderlich ist.
Alle hier aufgeführten \Axiome, \Saetze\ und \Beweise\ sollen dazu kodiert und die \Beweise\ dann von \ASBA\ verifiziert werden.

\section     {Metasprache}% ====================================================
\beginsection{Metasprache}
\label   {sec-Metasprache}

Wenn man über eine Sprache, die sogenannte \defFt{Objektsprache}, spricht, braucht man eine zweite Sprache, die sogenannte \defTxt{\Metasprache}, in der \Aussagen\ über erstere getroffen werden können.%
\footnote{%
	Die beiden Sprachen können auch übereinstimmen, \textzB\ wenn man über die natürliche Sprache spricht.
}
Wenn die \Objektsprache\ die der Mathematik ist, wählt man üblicherweise die natürliche Sprache als \Metasprache.
Leider ist diese oft ungenau, nicht immer eindeutig und abhängig vom Zusammenhang, in dem sie gesprochen wird.%
\footnote{%
	Man betrachte die beiden \Aussagen\ \statement{Studenten und Rentner zahlen die Hälfte.} und \statement{Studenten oder Rentner zahlen die Hälfte.}, die beide das gleiche meinen.
	--- Entnommen aus \cite{bib:Rautenberg} \sectionname~1.2 Bemerkung 1.

	Ein weiteres Problem ist, dass man unauflösbare Widersprüche formulieren kann, \textzB\ \statement{Der Barbier ist der Mann im Ort, der genau die Männer im Ort rasiert, die sich nicht selbst rasieren.}.
	Und der Barbier?
	Wenn er sich selbst rasiert, dann rasiert er sich nicht selbst, und wenn er sich nicht selbst rasiert, dann rasiert er sich selbst.
	Was denn nun?
	--- Quelle unbekannt) --
	Das Problem ist verwandt mit dem Problem der \statement{Menge aller Mengen, die sich nicht selbst enthalten}.
}
Um diese Probleme in den Griff zu bekommen, kann die \Metasprache\ teilweise formalisiert werden.
Durch diese Formalisierung erinnert sie dann schon an mathematische \Formeln.
Die Sprachebenen sollten aber sorgfältig unterschieden werden.

Wir unterscheiden hier:
\begin{description}
	\item[\Metasprache] Die normale Umgangssprache.
	\item[\formalisierteMetasprache] Die Verwendung von \Metaoperationen, \Mrelationen\ und \Mvariablen.
	Dies umfasst die meisten der auftretenden \Formeln, die wir dann konsequenterweise als \defFt{Metaformeln} bezeichnen.
	\item[Objektsprache] Unser Objekt ist die Mathematik, genauer mathematische \Formeln.
	Dies werden \Formeln\ der \Aussagen- und \Praedikatenlogik\ sein.
\end{description}

\subsection[Aussagen und Metaoperationen]{\Aussagen\ und \Metaoperationen}% ----
\label  {sub-AussagenUndMetaoperationen}

Beispiele für \defTxt{\Aussagen} in \Metasprache\ sind
(a) \statement{Morgen scheint die Sonne.},
(b) \statement{Ich bin 1,83\,m groß.},
(c) \statement{Ich habe ein rotes Auto und das kann 200\,km/h schnell fahren.}, usw.
Wie Beispiel (c) zeigt, kann eine \Aussage\ auch aus anderen \Aussagen\ zusammengesetzt sein.
In diesem Fall bezeichnen wir sie als \defTxt{\zerlegbar}, ansonsten als \defTxt{\unzerlegbar} oder auch \defTxt{\atomar}.
-- Wir betrachten auch Relationen einschließlich ihrer Operanden als \Aussagen.%
\footnote{%
	Wird statt des Symbols der Name der zugehörigen Relation verwendet, ist dies unmittelbar einleuchtend.
	So wird \textzB\ aus der \Formel\ \seqqt{$A<B$} die \Aussage\ \statement{$A$ ist kleiner als $B$}.
}

Während die Beispiele (a) und (b) \unzerlegbare\ (\atomare) \Aussagen\ sind, ist Beispiel (c) \zerlegbar.
Für alle drei \Aussagen\ lässt sich feststellen, ob sie richtig sind oder nicht;
für (a) allerdings nur im Nachhinein und für den zweiten Teil von (c) nur weil klar ist, worauf sich "`das"' bezieht.
Natürlich muss auch der Zusammenhang, in dem eine \Aussage\ formuliert wird, bekannt sein.
\textZB\ ist die Bedeutung von "`Ich"' nur dann bekannt, wenn man weiss, von wem die \Aussage\ ist.
Auf eine exakte Definition von \Aussage\ wird verzichtet, weil das intuitive Verständnis hier ausreicht.

\Zerlegbare\ \Aussagen\ wie (c) können zum Teil formalisiert werden.
Dies wird mit den folgenden Definitionen erreicht:%
\footnote{%
	Damit es nicht zu Verwechslungen führt, verwenden wir für die metasprachliche Negation nicht das logische Symbol \chrqt{\FrmNot}.
	Wegen \eqref{def-relback} \pagename~\pageref{def-relback} ist die Definition von \chrqt{\MtsRep} überflüssig, wird wegen der angegebenen Sprechweise aber dennoch angegeben.
}
\begin{align}
	%
	&    \defSymUna{\MtsNot}   A & \MtsDefEquiv \qquad &
	\text{$A$ \defFt{gilt nicht}.}
	\\
	%
	& A \defSymBin{\MtsImp}   B & \MtsDefEquiv \qquad &
	\text{\defFt{Wenn} $A$ gilt \defFt{dann} gilt auch $B$.}
	\\
	& A \defSymBin{\MtsRep}   B & \MtsDefEquiv \qquad &
	\text{$A$ gilt \defFt{sofern} $B$ gilt.}
	\\
	& A \defSymBin{\MtsEquiv} B & \MtsDefEquiv \qquad &
	\text{$A$ gilt \defFt{genau dann wenn} $B$ gilt.}
	\\
	& A \defSymBin{\MtsAnd}   B & \MtsDefEquiv \qquad &
	\text{$A$ \defFt{und}  $B$.}
	\\
	& A \defSymBin{\MtsOr}    B & \MtsDefEquiv \qquad &
	\text{$A$ \defFt{oder} $B$.}
	\formulatoleft
\end{align}

Offensichtlich sind das alles ebenfalls \Aussagen, jetzt aber teilweise formalisiert.
(c) lässt sich dann ausdrücken als \statement{\statement{Ich habe ein rotes Auto} \MtsAnd\ \statement{das kann 200\,km/h schnell fahren.}}.
\seqqt{$A \defSymBin{\MtsRep} B$} ist nur eine andere Schreibweise für \seqqt{$B \MtsImp A$}.
-- Ein Symbol für "`nicht"' wird hier nicht gebraucht.

Wir nennen \MtsAnd\ und \MtsOr\ \defTxt{\Metaoperationen} und \MtsImp, \MtsRep\ und \MtsEquiv\ \defTxt{\Metarelationen}%
\footnote{%
	Man könnte \Metaoperationen\ und \Metarelationen\ auch als \defFt{Metajunktoren} bezeichnen. Zur Unterscheidung von \Operationen\ und \Relationen\ vergleiche aber auch die Fußnote~\ref{def-Junktor} auf Seite~\pageref{def-Junktor}.
}.
Die damit gebildeten \Aussagen\ können natürlich auch geklammert werden, um die Reihenfolge der Auswertung eindeutig zu machen.
Für den Fall fehlender Klammern sind ihre Prioritäten \vrefintab{tab-Prioritaeten} angegeben.

Um Verwechslungen mit den \Junktoren\ zu vermeiden, verwenden wir für die metasprachlichen \Operationen\ "`und"' und "`oder"' die Symbole \chrqt{\MtsAnd} und \chrqt{\MtsOr}.
$A$ und $B$ können als Operanden von \chrqt{\MtsEquiv}, \chrqt{\MtsAnd} und \chrqt{\MtsOr} vertauscht werden, ohne das Ergebnis zu ändern.%
\footnote{%
	\textDh\ die \Operationen\ \chrqt{\MtsEquiv}, \chrqt{\MtsAnd} und \chrqt{\MtsOr} sind \emph{kommutativ}.
}
Wird in einer (Teil"~)\Aussage\ nur eine der \Operationen\ \MtsAnd\ oder \MtsOr\ verwendet, können die Klammern dort weggelassen und die Operationen in beliebiger Reihenfolge ausgewertet werden, wiederum ohne das Ergebnis zu ändern.%
\footnote{%
	\textDh\ die \Operationen\ \MtsAnd\ und \MtsOr\ sind auch \emph{assoziativ}.
	Bei den den logischen \Operationen\ \FrmAnd\ und \FrmOr\ müssen Kommutativität und Assoziativität durch \Axiome\ gefordert werden.
	Die Kommutativität von \MtsEquiv\ kann abgeleitet werden.
}
Zusammengefasst ist die Reihenfolge der \Operationen\ und der Auswertung dort beliebig.

\subsection[Mit Gleichheit verwandte Relationen]{Mit \Gleichheit\ verwandte \Relationen}
\label     {sub-Gleichheit}

\subsubsection[Vergleichbar]{\Vergleichbar}% - - - - - - - - - - - - - - - - - -
\label {subsub-Vergleichbar}

Zwei \Objekte\ $A$ und $B$ sind \defTxt{\vergleichbar}, wenn beide von derselben Art sind, \textdh\ wenn \textzB\ jeweils beide Mengen, \Zeichenfolgen, Zahlen, \textusw\ sind.
Dabei muss bei \Formeln\ zwischen der \Formel\ an sich und dem Ergebnis der \Formel\ unterschieden werden. Siehe Beispiel (a).

Intuitiv scheint klar zu sein, was damit  gemeint ist.
Wenn aber entschieden werden muss, ob \textzB\ (a) "`1+1"' gleich "`2"' oder (b) "`1+1"' gleich "`1 + 1"' ist, muss man erst entscheiden, von welcher Art die beiden zu vergleichenden Ausdrücke sind, \textdh\ \emph{wie} verglichen wird.
Wenn sie als jeweiliges Ergebnis der beiden \Formeln, \textdh\ als \Objekt, verglichen werden, dann ist (a) richtig.
Wenn sie als \Formeln, \textdh\ als \Zeichenfolgen, verglichen werden, ist (a) falsch.
Wenn die Ausdrücke in (b) als \Zeichenfolgen\ verglichen werden, dann ist (b) richtig.
Wenn sie als \Zeichenketten\ verglichen werden, ist (b) falsch.

Die folgende Tabelle fasst dass zusammen:

\begin{center}
	\begin{tabular}{|c|c|c|c|}
		\hline
		$        A $  &        $B$        & Art    & $A$ gleich $B$ \\
		\hline
		$       1+1$  &        $2$        & \Objekt       & richtig \\
		\seqqt{$1+1$} & \seqqt{$2$}       & \Formel       & falsch  \\
		\seqqt{$1+1$} & \seqqt{$1\;+\;1$} & \Zeichenfolge & richtig \\
		\strqt{1+1}   & \strqt{1 + 1}     & \Zeichenkette & falsch  \\
		\hline
	\end{tabular}
\end{center}

\subsubsection{Vergleiche}%- - - - - - - - - - - - - - - - - - - - - - - - - - -
\label {subsub-Vergleiche}

$A$ und $B$ seien \Objekte.
Dann definieren wir:

\begin{description}
	%
	\item[$\defSym{\MtsEq}$] \defTxt{\Gleichheit} \label{def-Gleichheit}
	\seqqt{$A \MtsEq B$} heißt, dass $A$ und $B$ in den \interessierendenEigenschaften\ für \MtsEq\ übereinstimmen.%
	\footnote{%
		\textZB\ sind zwei \Junktoren\ üblicherweise dann gleich, wenn sie stets denselben \emph{\Wahrheitswert} liefern.
		Ihre Bezeichnungen oder \Symbole\ können dabei durchaus verschieden sein, interessieren bei der Feststellung der \Gleichheit\ aber nicht.
		\textZB\ bezeichnen \chrqt{\MtsAnd} und \chrqt{\MtsUnd} dieselbe \Operation, haben aber verschiedene Priorität. --- \vrefseetab{tab-Prioritaeten}
	}
	Sprechweisen: \standsfor{$A$ ist \emph{dasselbe} wie $B$} oder \standsfor{$A$ ist \emph{identisch} zu $B$}
	--- Inwieweit die Begriffe \emph{Gleichheit} und \emph{Identität} korrelieren, wird hier nicht erörtert.\citenote{bib:Identitaet}
	%
	\item[$\defSym{\MtsEqN}$] \defTxt{\Ungleichheit} \label{def-Ungleichheit}
	\seqqt{$A \MtsEqN B$} heißt, dass $A$ und $B$ in mindestens einer \interessierendenEigenschaft\ für \MtsEq\ nicht übereinstimmen.
	Sprechweisen: \standsfor{$A$ ist \emph{nicht dasselbe} wie $B$} (aber vielleicht das gleiche; siehe \MtsEquiv) oder \standsfor{$A$ ist \emph{nicht identisch} zu $B$}.
	%
	\item[$\defSym{\MtsAequiv}$] \defTxt{\Aequivalenz} \label{def-Aequivalenz}
	\seqqt{$A \MtsAequiv B$} heißt, dass $A$ und $B$ in den \interessierendenEigenschaften\ für \MtsAequiv\ übereinstimmen.
	Sprechweisen: \standsfor{$A$ ist \emph{das gleiche} wie $B$} (aber nicht unbedingt dasselbe; siehe \MtsEq) oder \standsfor{$A$ ist \emph{so wie} $B$}.
	--- Es kann auch verschiedene Äquivalenzen geben, für die dann verschiedene Bezeichnungen verwendet werden.
	%
	\item[$\defSym{\MtsAequivN}$] \defTxt{\Kontravalenz} \label{def-Kontravalenz}
	\seqqt{$A \MtsAequivN B$} heißt, dass $A$ und $B$ in mindestens einer \interessierendenEigenschaft\ für \MtsAequivN\ nicht übereinstimmen.
	Sprechweisen: \standsfor{$A$ ist \emph{nicht das gleiche} wie $B$} oder \standsfor{$A$ ist \emph{nicht so wie} $B$}.
	%
\end{description}

\MtsEq, \MtsEqN, \MtsAequiv\ und \MtsAequivN\ bezeichnen wir als  \defTxt{\Gleichheitsrelationen}.
\Gleichheit\ und \Aequivalenz\ sind \defTxt{\Aequivalenzrelationen}, \textdh\ sie sind \emph{reflexiv} ($a \sim a$), \emph{transitiv} ($((a \sim b) \MtsAnd (b \sim c)) \MtsImp (a \sim c)$) und \emph{symmetrisch} ($(a \sim b) \MtsImp (b \sim a)$)
-- jeweils für alle zulässigen Objekte $a$, $b$ und $c$.

Jede \interessierendeEigenschaft\ für \MtsAequiv\ oder eine andere \Aequivalenz\ muss auch eine für \MtsEq\ sein.
Daraus folgt insbesondere, dass mit $(A \MtsEq B)$ auch $(A \MtsAequiv B)$ und mit $(A \MtsAequivN B)$ auch $(A \MtsEqN B)$ gilt.

\subsubsection[Definitionen]{\Definitionen}% - - - - - - - - - - - - - - - - - -
\label {subsub-Definitionen}

Seien $A$ und $B$ \Aussagen\ \textbzw\ \Objekte%
\footnote{%
	Die Anforderungen an $A$ und $B$ sind intuitiv klar.
	Insbesondere darf $B$ nicht von einem bisher undefinierten Teil von $A$ abhängig sein.
}.
\begin{description}
	%
	\item[$\defSym{\MtsDefEquiv}$] \defTxt{\Metadefinition} \label{def-Metadefinition}
	\seqqt{$A \MtsDefEquiv B$} heißt, dass die \Aussage\ $A$ \emph{definitionsgemäß gleich} der \Aussage\ $B$ ist.
	Gewissermaßen ist $A$ nur eine andere Schreibweise für $B$.
	\standsfor{$A$ \emph{steht für} $B$}; $A$ und $B$ können sich gegenseitig ersetzten.
	%
	\item[$\defSym{\MtsDefEq}$] \defTxt{\Definition} \label{def-Definition}
	\seqqt{$A \MtsDefEq B$} heißt, dass das \Objekt\ $A$ \emph{definitionsgemäß gleich} dem \Objekt\ $B$ ist.
	Gewissermaßen ist $A$ nur eine andere Schreibweise für $B$.
	\standsfor{$A$ \emph{steht für} $B$}; $A$ und $B$ können sich gegenseitig ersetzten.%
	\footnote{%
		Nach den Definitionen von \MtsDefEquiv\ und \MtsDefEq\ sind zwei Ausdrücke $P$ und $Q$ schon dann gleich, wenn nach der Ersetzung aller Vorkommen von $A$ durch $B$ sowohl in $P$ als auch in $Q$ die resultierenden Ausdrücke $\overline{P}$ und $\overline{Q}$ gleich sind.
	}
\end{description}
Man beachte, dass \MtsDefEquiv\ und \MtsDefEq\ verschiedene Sprachebenen sind.

\section     {Notationen}% =====================================================
\beginsection{Notationen}
\label   {sec-Notationen}

\begin{itemize}
	\item Die in \datsec{sec-Notationen} aufgeführten Notationen werden in \datcha{cha-Grundlagen} verwendet, ohne nochmals erläutert zu werden. Abweichungen davon müssen gesondert angegeben werden.
	%
	\item Sätze mit "`wir"' bestimmen Notationen, die \textevtl\ nur für dieses Dokument gelten.
	Bei allgemein bekannten Notationen wird "`wir"' nicht verwendet.
	Die Verwendung von "`wir"' ist allerdings möglicherweise nicht konsistent und soll nur als Hinweis dienen.
	%
	\item Allgemein bekannte Notationen werden nicht alle erklärt, jedoch solche, die in der Literatur unterschiedlich verwendet werden.
	Oft findet sich aber noch eine Definition im Symbolverzeichnis \pagename~\pageref{dic-Symbolverzeichnis} oder Glossar \pagename~\pageref{dic-Glossar}.
	%
	\item Werden Begriffe definiert, so werden sie \textcolor{blue}{\defFt{in dieser Schriftart}} hervorgehoben und bei Verwendung mit einem Link ins Glossar versehen.
	Ähnlich für \Symbole\, nur dass deren Schriftart vom \Symbol\ abhängt.
\end{itemize}
%
Im Vorgriff auf \vrefsubsub{subsub-Definitionen} stehen \seqqt{$A \defSym{\MtsDefEquiv} B$} und \seqqt{$A \defSym{\MtsDefEq} B$} für \standsfor{$A$ \emph{ist definitionsgemäß gleich} $B$}, \seqqt{$A \defSym{\MtsAnd} B$} für \standsfor{$A$ \emph{und} $B$} und \seqqt{$A \defSym{\MtsOr} B$} für \standsfor{$A$ \emph{oder} $B$}.
Damit definieren wir für Elemente $a$ und Mengen $A$ und $B$%
\footnote{%
	In der Literatur wird \chrqt{\MtsSubset} oft in der Bedeutung von \chrqt{\MtsSubsetEq} verwendet.
	Wir verwenden \chrqt{\MtsSubset} jedoch nur, wenn wir explizit \Ungleichheit\ verlangen.
}
\begin{align}
	&   \defSym{\MtsIN}           & \MtsDefEq    \quad &
	\text{die Menge der \defFt{natürlichen Zahlen}  ohne           $0$}
	\label{def-MtsIN}   \\
	&   \defSym{\MtsINo}          & \MtsDefEq    \quad &
	\text{die Menge der \defFt{natürlichen Zahlen} (einschließlich $0$)}
	\label{def-MtsINo}  \\
	%
	& a \defSymBin{\MtsIn}       A & \MtsDefEquiv \quad &
	\text{$a$ ist         \defFt{Element aus} $A$}
	\label{def-MtsIn} \\
	& A \defSymBin{\MtsSubset}   B & \MtsDefEquiv \quad &
	\text{$A$ ist \defFt{echte Teilmenge von} $B$}
	\label{def-MtsSub}   \\
	& A \defSymBin{\MtsSubsetEq} B & \MtsDefEquiv \quad &
	\text{$A$ ist  \defFt     {Teilmenge von} $B$}
	\label{def-MtsSubeq} \formulatoleft
\end{align}
\defSymBin{\MtsNi}, \defSymBin{\MtsSupset} und \defSymBin{\MtsSupsetEq} sind die \Umkehrrelationen\ zu \MtsIn\, \MtsSubset\ und \MtsSubsetEq\ und wir sprechen von \defFt{Obermengen}.
%Schließlich sind \defSymBin{\MtsInN}, \defSymBin{\MtsSubsetN}, \defSymBin{\MtsSubsetEqN}, \defSymBin{\MtsNiN}, \defSymBin{\MtsSupsetN} und \defSymBin{\MtsSupsetEqN} noch die zugehörigen \Negationen.

Wenn wir von einer \defFt{natürlichen Zahl} sprechen, meinen wir immer ein Element aus \MtsINo.

%TODO G-Eintrag natürliche Zahl, Untermenge, Obermenge
%TODO G-Einträge für \times, \cup, \cap, \setminus, \MtsInN, \MtsNiN, \MtsSubsetEqN, \MtsSupsetEqN

\subsection{Bezeichnungen}% ----------------------------------------------------
\label {sub-Bezeichnungen}

\begin{description}

	% ----- Symbol -------------------------------------------------------------
	\item [\Symbole] umfassen neben speziellen \Symbolen\ auch Buchstaben, Ziffern und Sonderzeichen.
	\Symbole, für die es kein eigenes typographisches Zeichen gibt, können auch durch Aufeinanderfolge mehrerer typographischer Zeichen, \textiAlg\ lateinische Buchstaben, dargestellt werden.
	Wir nennen sie dann \defFt{zusammengesetzte Symbole}, im Gegensatz zu den \defFt{einfachen Symbolen}.
	Charakteristisch für ein Symbol ist, dass es ohne Bedeutungsverlust nicht zerlegt werden kann.
	Einzelne Symbole werden \chrqt{so} quotiert, \textzB\ \chrqt{\MtsINo}%
	\footnote{%
		Man kann \chrqt{\MtsINo} auch als als Aufeinanderfolge der beiden Symbole \chrqt{\MtsIN} und \chrqt{${}_0$} betrachten.
		Welche Interpretation richtig ist, ist nicht immer wichtig und ergibt sich bei Bedarf aus dem Zusammenhang.
	}
	für die Menge der natürlichen Zahlen einschließlich 0 und \chrqt{$\sin$} für die Sinusfunktion.
	--- Die Quotierung ist kein Bestandteil des \Symbols!

	Wird für bestimmte \Objekte\ ein \Symbol\ verwendet, so nennen wir dies ein \defFt{Objektsymbol}.
	Ist das Objekt eine Funktion, Operation, Relation \textusw, so nennen wir das Symbol ein \defFt{Funktionssymbol}, \defFt{Operationssymbol}, \defFt{Relationssymbol} usw.

	% ----- Zeichenkette -------------------------------------------------------
	\item [\Zeichenketten] sind Folgen von einfachen \Symbolen, in denen im Prinzip auch Leerstellen und andere nicht druckbare Zeichen zulässig sind.%
	\footnote{%
		Da beim Ausdruck optisch nicht entschieden werden kann, ob ein Zwischenraum (white space) aus einem Tabulator oder \textevtl\ mehreren Leerzeichen besteht, verwenden wir nur einzelne Leerzeichen als Zwischenraumzeichen und vermeiden Zeilenumbrüche.
	}
	Damit Leerstellen in \Zeichenketten\ leicht bestimmt und sogar gezählt werden können,
	werden \Zeichenketten\ stets \strqt{in dieser} Schriftart und Quotierung dargestellt.
	--- Die Quotierung ist kein Bestandteil der \Zeichenkette!

	% ----- Zeichenfolge -------------------------------------------------------
	\item [\Zeichenfolgen] sind ähnlich wie \Zeichenketten, außer das sie neben einfachen auch zusammengesetzte \Symbole\ enthalten können und Leerzeichen und andere Zwischenraumzeichen nicht zählen.
	Letztere dienen nur der optischen Trennung der \Symbole\ und der besseren Lesbarkeit.
	\Zeichenfolgen\ werden stets \seqqt{in dieser} Quotierung dargestellt.
	--- Die Quotierung ist kein Bestandteil der \Zeichenfolge!

	% ----- Formel -------------------------------------------------------------
	\item [\Formeln] \label{def-Formel} sind in diesem Dokument immer nach vorgegebenen Regeln aufgebaute \Zeichenfolgen%
	\footnote{%
		Es kann verschiedene Arten von \Formeln\ geben, \textzB\ aussagenlogische, prädikatenlogische und solche, die ein Taschenrechner auswerten kann.
	}.
	Daher werden sie wie \Zeichenfolgen\ quotiert.
	--- Die Quotierung ist kein Bestandteil der \Zeichenfolge!

	Man kann eine \Formel\ auch dadurch charakterisieren, dass sie ein Element einer vorgegebenen Menge \MtsSprache\ von \Zeichenfolgen\ ist.%
	\footnote{%
		Die \Formel\ wird dann auch \defTxt{\Wort} der \defTxt{\Sprache} \MtsSprache\ genannt - besonders, wenn die Elemente aus \MtsSprache\ \Zeichenketten\ statt \Zeichenfolgen\ sind.
		Wir bleiben der Klarheit willen bei "`\Formel"'.
	}
	Das ist dann so ziemlich die einfachste Regel.

	Wenn eine \Zeichenfolge\ nicht korrekt nach den vorgegebenen Regeln aufgebaut ist \textbzw\ kein Element der vorgegebenen Menge \MtsSprache\ ist, werden wir sie \emph{nicht} als \Formel\ bezeichnen, auch nicht als "`fehlerhafte Formel"' oder ähnlich.
	Sie ist dann einfach keine \Formel.

	% ----- Objekt -------------------------------------------------------------
	\item [\Objekte] sind \textzB\ \Symbole, \Zeichenketten, \Zeichenfolgen\ und \Formeln, oder auch \Aussagen, Mengen, Zahlen, \textusw\ --- ganz allgemein reale oder gedachte Dinge an sich.
	Eine \Formel, die nicht quotiert ist, steht für den Wert dieser \Formel, der dann wieder ein \Objekt\ ist.
	Entsprechend steht ein \Symbol, das nicht quotiert ist, für das dadurch bezeichnete \Objekt.
	\textZB bezeichnet das \Symbol\ \chrqt{\MtsIN} die Menge \MtsIN der natürlichen Zahlen ohne 0.

\end{description}

\subsection{Quotierung}% -------------------------------------------------------
\label {sub-Quotierung}

Zur Verdeutlichung der soeben definierten Quotierungen ein Beispiel:\footnote{%
	Was zusammengesetzte \Symbole\ sind, muss jeweils definiert werden  \textbzw\ ergibt sich aus dem Zusammenhang.
}

\begin{tabular}{llll}
	&        $\sin$  & Ein \Objekt
	& die Sinusfunktion
	\\
	& \chrqt{$\sin$} & Ein \Symbol\ (Bezeichnung)
	& für das \Objekt
	\\
	& \seqqt{$\sin$} & Eine \Zeichenfolge\ (\Formel)
	& aus dem zusammengesetzten \Symbol\ \chrqt{$\sin$}
	\\
	& \seqqt {$sin$} & Eine \Zeichenfolge\ (\Formel)
	& aus den einfachen \Symbolen\ \chrqt{$s$}, \chrqt{$i$} und \chrqt{$n$}
	\\
	& \strqt  {sin}  & Eine \Zeichenkette
	& aus den einfachen \Symbolen\ \chrqt{\charf{s}}, \chrqt{\charf{i}} und \chrqt{\charf{n}}
\end{tabular}

Die Bezeichnung eines \Objekts\ kann auch aus mehreren Symbolen bestehen, \textdh\ einer \Zeichenfolge\ oder sogar einer ganzen \Formel; \textzB\ ist die Bezeichnung für das indizierte \Objekt\ $a_i$ gleich \seqqt{$a_i$}.

\subsection{Weitere Bezeichnungen}% --------------------------------------------
\label  {sub-weitereBezeichnungen}

\begin{description}

%TODO überarbeiten, Dopplungen meiden, mit Glossar abgleichen; eindeutige Bezeichnungen
	% ----- Folge --------------------------------------------------------------
	%TODO Folge ins Glossar

	% ----- Tupel --------------------------------------------------------------
	\item [\Tupel] Ein \defFt{$n$-\Tupel} ist eine endliche Folge $\vec{a} = (a_1, \dots, a_n)$ mit folgenden Eigenschaften:
	\begin{itemize}
		\item $n$, die \defFt{Länge}, \textdh\ die Anzahl der \defFt{Komponenten} aus $\vec{a}$, ist eine natürliche Zahl.

		$\defSymUna{\MtsLen} \vec{a} \MtsDefEq \defSym{\MtsLen}(\vec{a}) \MtsDefEq n$
		%
		\item Die $a_i$ für $1 \le i \le n$ sind Elemente meist vorgegebener Mengen.
		%
		\item $\defSymUna{\MtsSet} \vec{a} \MtsDefEq \defSym{\MtsSet}(\vec{a}) \MtsDefEq$ die Menge aller Komponenten $a_i$ aus $\vec{a}$.
	\end{itemize}
	Für $n=0$ ist $\vec{a} = ()$, das \defFt{leere \Tupel} oder \defFt{$0$-\Tupel}.

	Wo immer $\vec{a}$ und $a_i$ mit $i \in \MtsINo$ gemeinsam vorkommen, ist $a_i$ die $i$-te Komponente aus $\vec{a}$.

	% ----- Relation -----------------------------------------------------------
	\item [\Relation] Eine \defFt{$n$-stellige \Relation}\citenote{bib:RelationMehrstellig} $R$ ist ein (1+$n$)-\Tupel\ $(G,A_1,\dots,A_n$) mit folgenden Eigenschaften:
	\begin{itemize}
		\item $n$, die \defFt{relationale \Stelligkeit}, ist eine natürliche Zahl.

		$\MtsStelR R \MtsDefEq \MtsStelR(R) \MtsDefEq n$
		%
		\item Die $A_i$ für $1 \le i \le n$ sind Mengen, die \defTxt{\Traegermengen} (carrier) von $R$.

		$\MtsTraeger_i R \MtsDefEq \MtsTraeger_i(R) \MtsDefEq A_i$
		%
		\item $G$, der \defTxt{\Graph} von $R$, ist eine Teilmenge des kartesischen Produkts $A_1 \times \dots \times A_n$.

		$\MtsGraph R \MtsDefEq \MtsGraph(R) \MtsDefEq G \quad$ (oft einfach mit $R$ bezeichnet)
		%
		\item $R(a_1,\dots\,a_n) \MtsDefEquiv (a_1,\dots\,a_n) \in G$
	\end{itemize}
	Für $n=0$ ist $G \MtsSubsetEq \{()\}$%
	\footnote{%
		Das kartesische Produkt enthält nur noch das $0$-\Tupel\ $()$.
	},
	\textdh\ $R()$ ist entweder \TxtTrue\ (\MtsTrue) oder \TxtFalse\ (\MtsFalse).
	\\Für $n=1$ ist $G \MtsSubsetEq A_1$, \textdh\ $R$ kann als Teilmenge von $A_1$ aufgefasst werden.
	\\Für $n=2$ heißt die Relation \defFt{binär} und man schreibt \seqqt{$x R y$} statt \seqqt{$R(x,y)$} \textbzw\ \seqqt{$(x,y) \in R$}.

	Ist $R=(G,M,\dots,M)$, so heißt $R$ eine $n$-stellige Relation \defFt{auf}\alternativ{in} $M$.

	Ist $|G|$ endlich, so nennen wir auch $R$ \defFt{endlich}.

	% ----- Umkehrrelation -----------------------------------------------------
	\item [\Umkehrrelation] Die \defTxt{\Umkehrrelation} einer binären Relation $(G,A,B)$ ist die Relation $(G',B,A)$ mit $G' \MtsDefEq \{(b,a)\mid(a,b) \in G\}$.
	Üblicherweise wird das zugehörige Relationssymbol gespiegelt.

	% ----- Funktion -----------------------------------------------------------
	\item [\Funktion] Eine \defFt{$n$-stellige \Funktion}\citenote{bib:Funktion} ist ein (1+$n$+1)-\Tupel\ $f = (G,A_1,\dots,A_n,B)$ mit folgenden Eigenschaften:
	\begin{itemize}
		\item $n$, die \defTxt{\Stelligkeit}%
		\footnote{%
			Die Werte der Stelligkeit als Relation und als Funktion sind verschieden, \textdh\ es gilt stets: $\MtsStelR(f) = \MtsStelF(f) + 1$.
		},
		ist eine natürliche Zahl.

		$\MtsStelF f \MtsDefEq \MtsStelF(f) \MtsDefEq n$

		\item $f$ ist eine ($n$+1)-stellige Relation.

		\item Zu jedem $n$-\Tupel\ $\vec{a} = (a_1,\dots,a_n)$ mit $a_i \in A_i$ für $1 \le i \le n$ gibt es genau ein $b \in B$ mit $(a_1,\dots,a_n,b) \in G$, den \defTxt{\Funktionswert} von $\vec{a}$.

		$f\vec{a} \MtsDefEq f a_1 \dots a_n \MtsDefEq f(\vec{a}) \MtsDefEq f(a_1,\dots,a_n) \MtsDefEq b$
		\footnote{%
			$f(a_1,\dots,a_n)$ und $f(a_1,\dots,a_n,b)$ sind wohl zu unterscheiden.
			Ersteres ist ein Funktionsaufruf mit einem Funktionswert, letzteres eine Relation mit einem Wahrheitswert.
		}

		\item $A = A_1 \times \dots \times A_n$ ist der \defTxt{\Definitionsbereich} (domain) von $f$.

		$\MtsDb f \MtsDefEq \MtsDb(f) \MtsDefEq A_1 \times \dots \times A_n$

		\item $B$ ist der \defTxt{\Zielbereich} (target) von $f$

		$\MtsZb f \MtsDefEq \MtsZb(f)$
	\end{itemize}
	Für $n = 0$ ist $G = ((),b)$ für ein $b \in B$ und somit $f() = b$. $f$ kann damit auch als Konstante $b$ aufgefasst werden.%
	\footnote{%
		Bei der Schreibweise ohne Klammern steht da statt \seqqt{$f()$} nur noch \seqqt{$f$} und statt \seqqt{$f()=b$}, insgesamt also nur noch \seqqt{$f=b$}.
	}

	Man sagt: $f$ ist eine $n$-stellige \Funktion\ von $A_1 \times \dots \times A_n$ \defFt{nach}\alternativ{in} $B$ (Schreibweise: $f : A_1 \times \dots \times A_n \rightarrow B$) oder, im Fall $n=1$, $f$ ist eine Funktion von $A$ nach $B$ (Schreibweise: $f : A \rightarrow B$).
	Mit $A \MtsDefEq A_1 \times \dots \times A_n$ kann für $n > 0$ jede Funktion als $1$-stellig aufgefasst werden.

	% ----- Operation ----------------------------------------------------------
	\item [\Operationen] in oder auf einer Menge $M$ sind $n$-stellige Funktionen $\MtsMn \rightarrow M$.
	Für eine \defFt{binäre}, \textdh\ 2-stellige \Operation\ \BspOpB\ schreibt man \textiAlg\ \seqqt{$x \BspOpB y$} statt \seqqt{$\BspOpB(x,y)$}.
	Wenn nicht anders angegeben, sind \Operationen\ stets binär.
	0-stellige \Operationen\ können wieder als Konstante aufgefasst werden.

	Um Missverständnisse zu vermeiden, werden wir den Begriff "`Operator"' nicht verwenden.

	% ----- Junktor ------------------------------------------------------------
	\item [\Junktoren] sind aussagenlogische \Relationen\ und \Operationen.%
	\footnote{\label{def-Junktor}%
		Ein $n$-stelliger \Junktor\ $J$ sei eine \Operation\ und somit eine \Funktion.
		Wegen $M = \{\MtsTrue,\MtsFalse\}$ kann er auch als eine $n$-stellige \Relation\ $J'$ aufgefasst werden:
		$J' \MtsDefEq \{\vec{a} \in \MtsMn \mid J(\vec{a}) = \MtsTrue\}$.

		~~Umgekehrt kann eine $n$-stellige aussagenlogische \Relation\ $J'$ mittels:
		$J''(\vec{a}) \MtsDefEq \MtsTrue \text{ für } \vec{a} \in J', \MtsFalse \text{ sonst}$, für $\vec{a} \in \MtsMn$, als $n$-stellige Operation aufgefasst werden.

		~~Falls $J(\vec{a})=\MtsTrue$ ist $\vec{a} \in J'$ und somit $J''(\vec{a})=\MtsTrue$.
		Für $J(\vec{a})=\MtsFalse$ ist $\vec{a} \notin J'$ und somit $J''(\vec{a})=\MtsFalse$.
		Also ist $J=J''$ und so können die aussagenlogischen $n$-stelligen \Relationen\ und \Operationen\ einander eineindeutig zugeordnet werden.

		~~Daher sind in der Aussagenlogik \Relationen\ und \Operationen\ nicht von vornherein unterscheidbar.
		Wegen der Verabredungen in \vrefsub{sub-Beispielsymbole} muss für die verwendeten \Junktoren\ daher jeweils wohl definiert sein, ob sie als \Relation\ und \Operation\ zu verstehen sind.
	}
\end{description}

\subsection[Relationen und Operationen]{\Relationen\ und \Operationen}% --------
\label{sub-Beispielsymbole}

Als Beispielsymbol für unäre \Operationen\ wird \chrqt{\defSym{\BspOpU}} und für binäre \Operationen\ \chrqt{\defSym{\BspOpB}} verwendet.
Beispielsymbole für binäre Relationen sind \chrqt{\defSym{\BspRel}} und \chrqt{\defSym{\BspRelEq}}, für ihre \Umkehrrelationen \chrqt{\defSym{\BspRelBck}} \textbzw\ \chrqt{\defSym{\BspRelBckEq}} sowie für ihre \defFt{Negationen} \chrqt{\defSym{\BspRelN}} \textbzw\ \chrqt{\defSym{\BspRelEqN}}.%
\footnote{%
	Die Relationen brauchen keine Ordnungsrelationen sein, auch wenn die angegebenen Symbole dies nahe legen.
	Wenn eine der Relationen \BspRel, \BspRelEq, \BspRelBck\ oder \BspRelBckEq\ definiert ist,
	sind wegen \eqref{def-relback}, \eqref{def-BspRelEq} und \eqref{def-BspRel} auch die anderen drei Relationen definiert sowie wegen \eqref{def-BspRelN} auch \BspRelN, \BspRelEqN, \defSym{\BspRelBckN} und \defSym{\BspRelBckEqN}.
	Der senkrechte Strich bei den Negationen kann auch schräg sein, wie \textzB\ bei \MtsEqN.
}
Wenn nichts anderes gesagt wird, gelte mit diesen Symbolen bei gegebenem \chrqt{\BspRel} stets:
\begin{align}
	& (A \defSymBin{\BspRelBck} B) & \MtsDefEquiv \quad &  (B \BspRel A)
	\label{def-relback}  \\
	& (A \defSymBin{\BspRelN}    B) & \MtsDefEquiv \quad & ((A \BspRel B) \text{ gilt nicht})
	\label{def-BspRelN}  \formulatoleft\formulatoleft\formulatoleft
\end{align}
Dabei ist \chrqt{\BspRelBck} ist die waagerechte Spiegelung von \chrqt{\BspRel} und statt des schrägen kann bei der Negation auch ein senkrechter Strich genommen werden.

Ist \chrqt{\BspRelBck}, \chrqt{\BspRelEq} oder \chrqt{\BspRelBckEq}, statt \chrqt{\BspRel} gegeben, so müssen die Symbole entsprechend ausgetauscht werden.
Entsprechend für die nächsten beiden Definitionen.

Je nachdem ob \BspRel\ oder \BspRelEq\ gegeben ist gelte ferner:
\begin{align}
	& (A \defSymBin{\BspRelEq}   B) & \MtsDefEquiv \quad & ((A \BspRel   B) \MtsOr  (A \MtsEq B))
	\label{def-BspRelEq} \\
	& (A \defSymBin{\BspRel}     B) & \MtsDefEquiv \quad & ((A \BspRelEq B) \MtsAnd (A \MtsEqN B))
	\label{def-BspRel}   \formulatoleft\formulatoleft\formulatoleft
\end{align}

Man beachte, dass, wenn man \chrqt{\MtsDefEquiv} durch \chrqt{\MtsEquiv} ersetzt, weder \eqref{def-BspRelEq} aus \eqref{def-BspRel} folgt noch umgekehrt.
\eqref{def-BspRelEq} und \eqref{def-BspRel} folgen aber dann auseinander, wenn aus \chrqt{\BspRel} die Ungleichheit \textbzw\ aus der Gleichheit \chrqt{\BspRelEq} folgt.
Beispiele dazu sind \vrefintab{tab-Gegenbeispiel} angegeben.
%
\begin{table}[H]
	\centering
	\setlength\extrarowheight{1.5pt}
	\begin{tabularx}{9.7cm}{|@{~\extracolsep{\fill}}c|cccc|l|}
		\hline
		~            &$A,\;          A$&$A,\;          B$&$B,\;A$&$B,\;          B$&
		\\
		\hline
		~\MtsEq  &$A=            A$&                 &       &$B=            B$&
		\\
		\hline
		~\BspRel  &                 &$A\BspRel   B$&       &                 &
		\text{Es gilt \eqref{def-BspRelEq}}
		\\
		~\BspRelEq&$A\BspRelEq A$&$A\BspRelEq B$&       &$B\BspRelEq B$&
		\text{und \eqref{def-BspRel}}
		\\
		\hline
		~\BspRel  &                 &$A\BspRel   B$&       &$B\BspRel   B$&
		\text{Es gilt \eqref{def-BspRelEq}}
		\\
		~\BspRelEq&$A\BspRelEq A$&$A\BspRelEq B$&       &$B\BspRelEq B$&
		\text{aber nicht \eqref{def-BspRel}}
		\\
		\hline
		~\BspRel  &                 &$A\BspRel   B$&       &                 &
		\text{Es gilt \eqref{def-BspRel}}
		\\
		~\BspRelEq&$A\BspRelEq A$&$A\BspRelEq B$&       &                 &
		\text{aber nicht \eqref{def-BspRelEq}}
		\\
		\hline
	\end{tabularx}
	\caption{Beispiele für \BspRel\ und \BspRelEq}
	\label{tab-Gegenbeispiel}% Erst nach '\caption'!
\end{table}
%
Wird eine binäre \Relation\ \BspRel\ zusammen mit einer binären \Operation\ \BspOpB\ oder einer weiteren binären \Relation\ $\approx$ verwendet wird, treffen wir folgende Vereinbarung:%
\footnote{%
	wird auch in der Literatur verwendet, \textzB\ \textzB~\cite{bib:Rautenberg}, Notationen Seite~xxi
}
\begin{align}
	&   A \BspOpB  B \BspRel C & \text{ steht für }
	&&& A \BspOpB  B \quad \MtsAnd \quad B \BspRel C \\
	&   A \BspRel B \BspOpB  C & \text{ steht für }
	&&& A \BspRel B \quad \MtsAnd \quad B \BspOpB  C \\
	&   A \BspRel B \approx    C & \text{ steht für }
	&&& A \BspRel B \quad \MtsAnd \quad B \approx    C \formulatoleft
\end{align}
Besondere Vereinbarungen für die unäre \Operation\ \chrqt{\BspOpU} treffen wir nicht.

Für den Fall fehlender Klammern sind die Prioritäten \vrefintab{tab-Prioritaeten} angegeben.
Damit wären dann alle Klammern in \datsub{sub-Beispielsymbole} überflüssig.

\subsection{Prioritäten}% ------------------------------------------------------
\label {sub-Prioritaeten}

\vrefDtab{tab-Prioritaeten} listet zur Vermeidung von Klammern die Prioritäten der in diesem Dokument verwendeten \Operationen, \Relationen, \Junktoren\ und \Definitionen\ in absteigender Folge von höherer zu niedrigerer Priorität, \textdh\ von starker zu schwacher Bindung auf.%
\footnote{Priorität 1 ist höher und bindet damit stärker als Priorität 2, usw.}
Das Weglassen redundanter Klammern wird in \datcha{cha-Grundlagen} nicht weiter thematisiert.%
\footnote{%
	Gesetzt den Fall, dass \ASBA\ die \Voraussetzungen\ und \Folgerungen\ eines mathematischen Satzes richtig und die \Beweisschritte, \textzB\ durch fehlerhafte Interpretation einer \Formel, falsch einliest, ansonsten aber richtig arbeitet.
	Dann kann man folgende Fälle unterscheiden:\\
	--- Ein falscher Satz kann dadurch nicht als richtig bewertet werden.\\
	--- Ein richtiger Satz wird wahrscheinlich auch bei eigentlich richtigem \Beweis\ als nicht bewiesen gelten, was natürlich unbefriedigend ist.\\
	--- In äußerst unwahrscheinlichen Fällen kann dabei auch ein eigentlich falscher \Beweis\ in einen richtigen verwandelt werden, was zwar schön ist, aber leider steht in der Dokumentation dann ein falscher \Beweis.\\
	In keinem Fall wird durch diesen Fehler die Menge der richtigen Sätze durch einen falschen Satz "`verunreinigt"'.
}
Zur besseren Verständlichkeit werden aber gelegentlich auch redundante Klammern verwendet, insbesondere wenn Prioritäten unklar oder in der Literatur auch anders definiert sind.
Die Prioritäten der \Junktoren\ wurden aus~\cite{bib:Rautenberg} Kapitel~1.1 Seite~5 entnommen und ergänzt und die der \Metaoperationen\ daran angeglichen.

\begin{table}[p]
	\centering
	\begin{threeparttable}
		\setlength\extrarowheight{3pt}
		\begin{tabularx}{12.5cm}{|@{~~}l|@{\extracolsep{\fill}}l|}
			\hline
			Klammern & $(\quad)$ \quad $\quad$ \chrqt{$\quad$} \quad \seqqt{$\quad$} \quad \strqt{$\quad$} \\
			\hline\hline
			\multicolumn{2}{|c|}{\Operationen\ haben unterschiedliche Priorität.} \\
			\hline
			Unäre \Operationen\ \Tnote{1} \Tnote{2} & $\BspOpU \quad \FrmNot \quad \MtsNot$ \\
			\hline
			Binäre \Operationen\ für Mengen &
			\begin{tabular}{@{\extracolsep{\fill}}l}
				$ \times $ \\
				\hline
				$ \cup $   \\
				\hline
				$ \cap $   \\
			\end{tabular}  \\
			\hline
			Binäre \Operationen\ \Tnote{1} & $ \BspOpB $ \\
			\hline
			Binäre \Junktoren\ \Tnote{2} &
			\begin{tabular}{@{\extracolsep{\fill}}l}
				$ \FrmAnd \quad \FrmNand               $ \\
				\hline
				$ \FrmOr  \quad \FrmXor \quad \FrmNor  $ \\
				\hline
				$ \FrmRep \quad \FrmImp                $ \\
				\hline
				$ \FrmEquiv                            $ \\
			\end{tabular}                                \\
			\hline\hline
			\multicolumn{2}{|c|}{Binäre Relationen haben gleiche Priorität.} \\
			\hline
			Binäre Relationen für Mengen \Tnote{3}
			& $ \MtsIn \quad \MtsNi \quad \MtsSubset \quad \MtsSubsetEq \quad \MtsSupset \quad \MtsSupsetEq $ \\
			\hdashline
			Binäre \Relationen\ \Tnote{1}
			& $ \BspRel \quad \BspRelN \quad \BspRelEq \quad \BspRelEqN \quad \BspRelBck \quad \BspRelBckEq$ \\
			\hdashline
			\Gleichheitsrelation\ \Tnote{4}
			& $ \MtsEq \quad \MtsEqN \quad \MtsAequiv \quad \MtsAequivN $ \\
			\hdashline
			\Ableitungsrelation\  \Tnote{5}
			& $ \MtsDerive $ \\
			\hdashline
			\Ersetzung\ \Tnote{5}
			& $ \MtsSwap \quad \MtsSubst $  \\
			\hline\hline
			\multicolumn{2}{|c|}{Sonstige binäre Verknüpfungen haben unterschiedliche Priorität.} \\
			\hline
			\Definition\ \Tnote{6} & $ \MtsDefEq $ \\
			\hline
			Binäre \Metaoperationen\ \Tnote{7} \Tnote{8} &
			\begin{tabular}{@{\extracolsep{\fill}}l}
				$ \MtsAnd$ \\
				\hline
				$ \MtsOr $ \\
				\hline
				$ \MtsUnd  $ \\
				\hline
				$ \MtsRep \quad \MtsEquiv \quad \MtsImp $
			\end{tabular}     \\
			\hline
			\Metadefinition\ \Tnote{6} & $ \MtsDefEquiv $ \\
			\hline\hline
			\multicolumn{2}{|c|}{Natürliche Sprache} \\
			\hline
			\parbox[][1.1cm][c]{6.3cm}{%
				Innerhalb natürlicher Sprache deren Strukturelemente, \textzB\ Satzzeichen \Tnote{9}%
			}
			& . \quad , \quad ; \quad usw. \\
			\hline
		\end{tabularx}
		\begin{tablenotes}
			\footnotesize
			\item[1] \vrefseesub{sub-Beispielsymbole}
			\item[2] \vrefseetab{tab-Symbole}
			\item[3] \vrefseesub{sub-Bezeichnungen}
			\item[4] \vrefseesubsub{subsub-Vergleiche}
			\item[5] \vrefseesub{sub-Basisregeln}
			\item[6] \vrefseesubsub{subsub-Definitionen}
			\item[7] \vrefseesub{sub-AussagenUndMetaoperationen}
			\item[8] \chrqt{\MtsUnd} wird nur bei den \Schlussregeln\ (\vrefseesub{sub-Schlussregeln}) verwendet.
			Zwar bezeichnen \chrqt{\MtsAnd} und \chrqt{\MtsUnd} dieselbe \Operation, aber je nach verwendetem Symbol hat sie eine unterschiedliche Priorität.
			\item[9] Innerhalb von \Formeln\ können Satzzeichen eine andere Bedeutung und Priorität haben.
		\end{tablenotes}
	\end{threeparttable}
	\caption{Prioritäten in abnehmender Reihenfolge}
	\label{tab-Prioritaeten}% Erst nach '\caption'!
\end{table}

Für \Operationen\ derselben Priorität wählen wir in diesem Dokument Rechtsklammerung%
\footnote{%
	Die Symbole unärer \Operationen\ stehen in diesem Dokument stets links \emph{vor} dem Operanden, so dass es für sie nur Rechtsklammerung geben kann.
	Zur Rechtsklammerung bei binären Operationen ein Zitat aus~\cite{bib:Rautenberg} Kapitel~1.1 Seite~5:
	"`Diese hat gegenüber Linksklammerung Vorteile bei der Niederschrift von Tautologien in \FrmImp, [...]"'.
	Die meisten Autoren bevorzugen Linksklammerung, was natürlicher erscheint.
	Dann sollte man aber für die Potenz doch noch Rechtsklammerung wählen, sonst ist \seqqt{$ a^{x^y} = (a^x)^y = a^{(x*y)} $} und nicht wie wahrscheinlich erwünscht \seqqt{$a^{(x^y)}$}.
}.

\section[Beweise in ASBA]{\Beweise\ in \ASBA}% ================================0
\beginsection            {\Beweise\ in \ASBA}
\label                {sec-BeweiseASBA}

Die Regeln zur Formulierung und Prüfung der \Beweise\ müssen in \ASBA\ fest codiert werden.
Sie sind quasi die \Axiome\ von \ASBA\ und sollten daher möglichst wenig voraussetzen.
In \ASBA\ wird dazu ein \emph{Genzen-Kalkül}%
\footnote{%
	\citesee{bib:Rautenberg} Kapitel~1.4 und~\cite{bib:Schlussregel,bib:NatuerlichesSchliessen}
} verwendet.
Die Definition von \emph{\Schlussregel} und \emph{\Beweis} ist in diesem Dokument \ASBA-spezifisch, um später eine leichtere Umsetzung in ein Programm zu erreichen.
Insbesondere müssen alle abzuspeichernden Mengen endlich sein.
Dies berücksichtigen wir in den Beispielen, fordern zunächst aber nicht notwendig Beschränktheit.
Zuerst brauchen wir aber noch ein paar Definitionen.

\subsection{Definitionen und Verabredungen}% -----------------------------------
\label                  {sub-Verabredungen}

Zu \chrqt{\MtsLen} und \chrqt{\MtsSet} Vergleiche die Definition von \emph{$n$-\Tupel} \vrefinsub{sub-weitereBezeichnungen}.

\begin{align}
	& |M|                          & \MtsDefEq \quad & \text{Kardinalität von } M
	&&\text{, die \defFt{Anzahl der Elemente} aus } M
	\label{def-Anzahl}
	\\
	& \defSym{\MtsMn}     & \MtsDefEq \quad & M \times \dots \times M \quad \text{ , für } n \in \MtsINo
	&&\text{, das \defFt{kartesische Produkt} aus $n$ Mengen } M
	\label{def-kartesischesProdukt}
	\\
	& \MtsMo                  &    \MtsEq \quad & \{()\}
	&&\text{, wobei $()$ das \defFt{0-\Tupel} ist}
	\label{def-Mo}
	\\
	& \defSym{\MtsTup}(M) & \MtsDefEq \quad & \{ \vec{a} \mid \vec{a} \in \MtsMn \FrmAnd n \in \MtsINo \}
	&&\text{, die Menge der \defTxt{\Tupel} \defFt{über} $M$, ihre \defTxt{\Tupelmenge}}
	\label{def-Tupelmenge}
	\\
	& \links{(A,B)}                & \MtsDefEq \quad & A
	&& \text{, die \defFt{linke Seite} eines geordneten Paares.}
	\label{def-links}
	\\
	& \rechts{(A,B)}               & \MtsDefEq \quad & B
	&& \text{, die \defFt{rechte Seite} eines geordneten Paares.}
	\label{def-rechts}
	\\
	& \defSym{\MtsPot}(M)      & \MtsDefEq \quad & \{ A \mid A \MtsSubsetEq M \}
	&&\text{, die \defTxt{\Potenzmenge} der Menge } M
	\label{def-Potenzmenge}
	\\
	& \defSym{\MtsPotf}(M)     & \MtsDefEq \quad & \{ A \mid A \MtsSubsetEq M \FrmAnd |A| \in \MtsINo\}
	&& \text{, die \defFt{endlichen Teilmengen} von } M
	\label{def-endlichePotenzmenge}
	\\
	& \defSym{\MtsRel}(M)      & \MtsDefEq \quad & \{ R \mid R \MtsSubsetEq M \times M\}
	&& \text{, die Menge der \defFt{binären Relationen} in } M
	\label{def-Relationsmenge}
	\\
	& \defSym{\MtsRelf}(M)     & \MtsDefEq \quad & \{ R \mid R \MtsSubsetEq M \times M \FrmAnd |R| \in \MtsINo\}
	&& \text{, die \defFt{endlichen binären Relationen} in } M
	\label{def-endlicheRelationsmenge}
	\\
	& \defSym{\MtsDeriveR}     & \MtsDefEq \quad & R
	&& \text{, für Relationen } R \in \MtsRelAllDerive
	\label{def-Ableitung}
\end{align}
Offensichtlich gilt für Mengen $M$ und $N$:
\begin{align}
	& \MtsPotf(M) \MtsSubsetEq \MtsPot          (M)
	& ,          \qquad
	& \MtsRelf(M) \MtsSubsetEq \MtsRel          (M)
	\label{eq-Setf} \\
	& \MtsRel (M) =            \MtsPot (M \times M)=\MtsPot (M^2)
	& ,          \qquad
	& \MtsRelf(M) =            \MtsPotf(M \times M)=\MtsPotf(M^2)
	\label{eq-relPot} \\
	& \MtsPot (M) \MtsSubset   \MtsPot          (N)
	& \MtsEquiv \qquad
	& \MtsPotf(M) \MtsSubset   \MtsPotf         (N)
	& \MtsEquiv \qquad
	&               M  \MtsSubset                          N
	\label{eq-potPot} \\
	& \MtsRel (M) \MtsSubset   \MtsRel          (N)
	& \MtsEquiv \qquad
	& \MtsRelf(M) \MtsSubset   \MtsRelf         (N)
	& \MtsEquiv \qquad
	&               M  \MtsSubset                          N
	\label{eq-relRel} \\
	&                                 \vec{a}  \in \MtsTup(M^2)
	& \MtsEquiv \qquad  & \MtsSet(\vec{a}) \in \MtsRelf    (M)
	\label{eq-vecrel}
\end{align}

\subsection[Formeln und Ableitungen]{\Formeln\ und \Ableitungen}% --------------
\label             {sub-Ableitungen}

Im Folgenden sei \MtsSprache\ stets eine gegebene Menge von \Formeln, \textzB\ alle korrekten \Formeln\ der \Aussagenlogik\ oder der \Praedikatenlogik.
Für die folgenden Betrachtungen ist aber nur nötig, dass die Elemente aus \MtsSprache\ \Zeichenfolgen\ sind.
Die Teilmengen von \MtsSprache\ nennen wir \defTxt{\Formelmengen}.
Es sind genau die Elemente aus \MtsPotSprache.

Bei einem Beweis werden aus einer \Formelmenge\ $\Gamma$ von \Axiomen\ und schon bewiesenen \Formeln\ mittels zulässiger
\footnote{%
	Was \emph{zulässig} heißt, muss im entsprechenden Kontext jeweils definiert sein.
	Üblicherweise sind das bestimmte Ableitungsregeln und Ersetzungen.
}
\Ableitungen\ die \Formeln\ einer \Formelmenge\ $\Delta$ abgeleitet; Schreibweise: \seqqt{$\Gamma \MtsDerive \Delta$}.

Für Teilmengen $\Gamma$ und $\Delta$ von \MtsSprache\ sei also:
\begin{itemize}
	\item $\Gamma \defSymBin{\MtsDerive} \Delta \MtsDefEquiv$ $\Gamma$ \defTxt{\ableitbar} $\Delta$; oder auch $\Gamma$ \defTxt{\beweisbar} $\Delta$.
	%
	\item $\Gamma \defSymBin{\MtsDerive} \Delta$ nennen wir auch eine \defTxt{\Ableitung} \defFt{in} \MtsSprache.
	Damit ist $(\Gamma,\Delta)$ ein Element einer binären Relation \MtsDerive\ in \MtsPotSprache, einer sogenannten \defTxt{\Ableitungsrelation}.
	%
	\item Wenn wir von einer Ableitung $\drvFt{a}$ sprechen, meinen wir immer ein Element einer \Ableitungsrelation, \textdh\ ein geordnetes Paar, \textzB\ $(\Gamma, \Delta) \in \MtsPotSprache \times \MtsPotSprache$, dargestellt als $\Gamma \MtsDerive \Delta$.
	%
	\item Um möglicherweise verschiedene \Ableitungsrelationen\ unterscheiden zu können, indizieren wir \chrqt{$\defSym{\MtsDerive}$} \textggf\ mit der zugrundeliegenden \Relation\ R, \textdh\ wir schreiben \chrqt{$\defSym{\MtsDeriveR}$} und sprechen dann von \defTxt{$R$-\ableitbar}, \defFt{$R$-\beweisbar} und \defFt{$R$-\Ableitung}.
\end{itemize}
%
Zur Vereinfachung der Darstellung und besseren Lesbarkeit treffen wir noch folgende Vereinbarungen für die beiden Seiten von \seqqt{$\Gamma \MtsDerive \Delta$} (natürlich nur, wenn dies nicht zu Verwechslungen führt):
\begin{itemize}
	\item Eine Aufzählung von \Formelmengen\ und einzelnen \Formeln\ steht für die Vereinigung der \Formelmengen\ mit der Menge der einzeln angegebenen \Formeln.
	\textZB\ steht \seqqt{$\Gamma, \alpha \MtsDerive \beta$} für \seqqt{$(\Gamma \cup \{\alpha\}) \MtsDerive \{\beta\}$}.
	%
	\item Diese Aufzählungen können auch leer sein und stehen dann für die leere Menge. \textZB\ steht \seqqt{$\MtsDerive\; \alpha \FrmImp (\beta \FrmImp \alpha)$} für \seqqt{$\emptyset \MtsDerive \{\alpha \FrmImp (\beta \FrmImp \alpha)\}$}.
	%
	\item Ist die Aufzählung links vom Relationssymbol \chrqt{\MtsDerive} leer, kann auch das Relationssymbol wegfallen.
	Im letzten Beispiel also einfach \seqqt{$\{\alpha \FrmImp (\beta \FrmImp \alpha)\}$}.
	Das entspricht dann einem \defTxt{\Axiom}.
\end{itemize}
%
Im Folgenden halten wir uns bei der Verwendung von Buchstaben so weit wie möglich an folgende Vereinbarungen:%
\footnote{Die letzte Gleichung ergibt sich aus \vreffor{eq-relPot}.}
\begin{align}
	&  \text{griechisch, klein:}       && \alpha, \beta, \gamma, \dots
	&& \text{\Formel}                  && \in \qquad \; \; \MtsSprache
	\\
	&  \text{griechisch, groß:}        && \Gamma, \Delta, \Theta, \dots
	&& \text{\Formelmenge}             && \in \quad \; \MtsPotSprache
	\\
	&  \text{lateinisch, fett, klein:} && \drvFt{a}, \drvFt{b}, \drvFt{c}, \dots
	&& \text{\Ableitung}               && \in \quad \; \MtsAllDerive
	\\
	&  \text{lateinisch, fett, groß:}  && \DrvFt{A}, \DrvFt{B}, \DrvFt{C}, \dots
	&& \text{\Ableitungsrelation}      && \in \MtsPotAllDerive = \MtsRelAllDerive
\end{align}
Damit definieren wir folgende Aussagen:
\begin{align}
	\frac{\; \DrvFt{A}  \;}{\; \DrvFt{B} \;}
	& \quad \MtsDefEquiv \quad
	\text{ Mit den \Ableitungen\ aus $\DrvFt{A}$ lassen sich die aus $\DrvFt{B}$ ableiten.}
	\label{def-AB}
	\\
	\frac{\; \vec{\drvFt{a}} \;}{\; \vec{\drvFt{b}} \;} \qquad
	& \quad \MtsDefEquiv \quad
	\text{ Mit den Komponenten aus $\vec{\drvFt{a}}$ lassen sich die aus $\vec{\drvFt{b}}$ ableiten.}
	\label{def-ab}
	\\
	\frac{\drvFt{a}_1 \MtsUnd \dots \MtsUnd \drvFt{a}_n}{\drvFt{b}_1 \MtsUnd \dots \MtsUnd \drvFt{b}_m}
	& \quad \MtsDefEquiv \quad
	\text{ Mit den \Ableitungen\ $\drvFt{a}_i$ lassen sich die $\drvFt{b}_j$ ableiten.}
	\label{def-aabb}
\end{align}
wobei in der letzten Definition $1 \le i \le n$ und $1 \le j \le m$ sei und die $\drvFt{a}_i$ und die $\drvFt{b}_j$ dabei jeweils beliebig permutiert werden können.
\chrqt{\defSym{\MtsUnd}} und Bruchstrich stehen für die \Metaoperationen\ \chrqt{\MtsAnd} und \chrqt{\MtsImp}.%
\footnote{%
	Der Bruchstrich hat die übliche Priorität, \MtsUnd\ die schwächste.
	Man beachte, dass Zähler und Nenner auch leer sein können, \textdh\ $n$ und $m$ gleich $0$ sein dürfen.
	In der Praxis liegen sie bei kleinen Werten, typischerweise 0, 1 oder 2.
}
Wir nennen alle drei Formen \defTxt{\Schlussregeln}%
\footnote{%
	Genau genommen nur um die Darstellung einer Schlussregel.
	Die Exakte Definition erfolgt \vrefinsub{sub-Schlussregeln}.
}.
Die Elemente aus $A$ \textbzw\ die Komponenten $a_i$ nennen wir die \defTxt{\Voraussetzungen} und die Elemente aus $B$ \textbzw\ die Komponenten $b_j$ die \defTxt{\Folgerungen} der \Schlussregel.
Offensichtlich gilt:
\begin{align}
	& \frac{a_1 \MtsUnd \dots \MtsUnd a_n}{b_1 \MtsUnd \dots \MtsUnd b_m} \; \MtsEquiv \; \frac{\; \vec{a} \;}{\; \vec{b} \;} \; \MtsEquiv \; \frac{\MtsSet(\vec{a})}{\MtsSet(\vec{b})} \label{eq-AB}
\end{align}
Wir nennen eine \Schlussregel\ auch einen \defTxt{\formalenSatz} und nennen sie \defTxt{\beschraenkt}, wenn sie nur endlich viele \Voraussetzungen\ und \Folgerungen\ hat.
Die \Schlussregeln\ nach \eqref{def-ab} und \eqref{def-aabb} sind per se beschränkt.
Die nach \eqref{def-AB} genau dann, wenn $\DrvFt{A}$ und $\DrvFt{B}$ endliche Mengen sind, \textdh\ wenn sie Elemente aus%TODO Text fehlt

Die Mengen der \Voraussetzungen\ und \Folgerungen\ dürfen auch leer sein.
Dies führt zu den folgenden Spezialfällen:
\begin{itemize}
	\item[] Eine \Schlussregel\ $\frac{A}{\emptyset}$ ohne \Folgerungen\ ist immer gültig.
	%
	\item[] Ein Menge $B$ von Ableitungen, die als \Axiome\ dienen sollen, kann als \Schlussregel\ $\frac{\emptyset}{B}$ ohne \Voraussetzungen\ repräsentiert werden.
\end{itemize}

\subsection[Schlussregeln]{\Schlussregeln}% ------------------------------------
\label {sub-Schlussregeln}

Wir betrachten zuerst noch die Menge der binären Relationen\vrefnotesub{sub-weitereBezeichnungen} in \MtsPotSprache.
Sei also $R$ eine solche binäre Relation und $A \in R$.
Dann gilt wegen~\eqref{def-links}, \eqref{def-rechts}, \eqref{def-Potenzmenge}, \eqref{def-Relationsmenge} und~\vreffor{def-Ableitung}:
\begin{align}
	&  A \in R \in \MtsRelAllDerive   \\
	&  A = (\links{A},\rechts{A})
	&& \text{und es gilt}
	&& \links{A}, \rechts{A} \MtsSubsetEq \MtsSprache \\
	&  \links{A} \MtsDeriveR \rechts{A}
	&& \text{oder einfach}
	&& \links{A} \MtsDerive  \rechts{A}
	&& \text{ist eine $R$-\Ableitung}                  \\
	&  \links{A} \; \text{$R$-\ableitbar} \; \rechts{A}
	&& \text{oder einfach} \qquad
	&& \links{A} \;\; \text{\ableitbar} \;\; \rechts{A}
	\formulatoleft
\end{align}

Nach diesen Vorbereitungen fassen wir noch mal zusammen:\\
Ein geordnetes Paar $(\MtsVoraussetzungSet, \MtsFolgerungSet) \in \MtsPotAllDerive^2 = \MtsRelAllDerive^2$ heißt eine
\defTxt{\Schlussregel} \defFt{für} \MtsSprache, geschrieben $\frac{\MtsVoraussetzungSet}{\MtsFolgerungSet}$; und es gilt:
\begin{align}
	& \MtsVoraussetzungSet \in \MtsRelAllDerive
	&& \text{, die \defTxt{\Voraussetzungen}}
	&& \text{, eine Menge von \defFt{\MtsVoraussetzungSet-\Ableitungen}.}
	\label{def-ruleRelationVoraussetzungen}
	\\
	& \MtsFolgerungSet   \in \MtsRelAllDerive
	&& \text{, die \defTxt{\Folgerungen}}
	&& \text{, eine Menge von   \defFt{\MtsFolgerungSet-\Ableitungen}.}
	\label{def-ruleRelationFolgerungen}
	\\
	& \drvFt{a} \in \MtsVoraussetzungSet \quad \MtsImp
	&& \drvFt{a} = (\Gamma, \Delta) \; \MtsAnd \; \Gamma, \Delta \in \MtsPotSprache
	&& \text{, Schreibweise: } \Gamma \MtsDerive_{\MtsVoraussetzungSet} \Delta
	\\
	& \drvFt{a} \in \MtsFolgerungSet \quad \MtsImp
	&& \drvFt{a} = (\Gamma, \Delta) \; \MtsAnd \; \Gamma, \Delta \in \MtsPotSprache
	&& \text{, Schreibweise: } \Gamma \MtsDerive_{\MtsFolgerungSet} \Delta
	\formulatoleft
\end{align}
mit $\Gamma$ und $\Delta$ jeweils passend.

***** Fehlende Verweise: \Ableitungsmenge, \FrmEqN, \MtsTrue, \MtsDerive, \MtsDeriveR. *****

Die \Schlussregel\ entspricht der \Aussage:
\begin{itemize}
	\item[] \emph{Mit den \Voraussetzungen\ aus \MtsVoraussetzungSet\ lassen sich alle \Folgerungen\ aus \MtsFolgerungSet\ ableiten}%
	\footnote{mittels noch zu definierender \emph{\zulaessigerUmwandlungen}}.
\end{itemize}
Die \Schlussregel\ heißt \defFt{allgemeingueltig}, wenn aus den \Voraussetzungen\ alle \Folgerungen\ abgleitet werden können.
In diesem Fall kann sie zur \zulaessigenUmwandlung\ von weiteren \Formeln\ dienen.

Die Mengen der \Voraussetzungen\ und \Folgerungen\ sowie die beiden Seiten einer \Ableitung\ dürfen auch leer sein.
Dies führt zu den folgenden semantischen Spezialfällen:
\begin{itemize}
	\item Eine \Ableitung\ $(A,\emptyset)$ ist trivial allgemeingültig.
	Daher können solche Voraussetzungen und Folgerungen ohne Probleme weggelassen werden.
	%
	\item Ein Menge $B$ von \Formeln, die \Axiome\ sein sollen, kann durch eine \Voraussetzung\ $(\emptyset,B)$ repräsentiert werden.
	%
	\item Ein Menge $B$ von \Formeln, die als allgemeingültig zu beweisen sind, kann durch eine \Folgerung\ $(\emptyset,B)$ repräsentiert werden.
\end{itemize}
%
Wenn eine Schlussregel $\frac{\MtsVoraussetzungSet}{\MtsFolgerungSet}$ beschränkt ist, sind \MtsVoraussetzungSet\ und \MtsFolgerungSet\ endliche Mengen und es gibt wegen~\vreffor{eq-vecrel} zwei \Tupel\ $\vec{\MtsVoraussetzung}, \vec{\MtsFolgerung} \in \MtsTup(\MtsAllDerive)$, so dass gilt:
\footnote{%
	Statt $\ge$ könnte in \eqref{eq-SRTb} auch \MtsEq\ genommen werden.
	Dann müssten die $\MtsVoraussetzung_n$ und die $\MtsFolgerung_m$ jeweils paarweise verschieden sein, was wir nicht voraussetzen wollen.
}
\begin{align}
	&     \MtsVoraussetzungSet    & \MtsEq \quad & \MtsSet(\vec{\MtsVoraussetzung})
	&,\;& \MtsFolgerungSet        & \MtsEq \quad & \MtsSet(\vec{\MtsFolgerung})
	\label{eq-SRTa}          \\
	&     N                       &    \ge \quad & |\MtsVoraussetzungSet|
	&,\;& M                       &    \ge \quad & |\MtsFolgerungSet|
	&,\;& \text{mit } N, M \in \MtsINo
	\label{eq-SRTb}          \\
	& \vec{\MtsVoraussetzung}     & \MtsEq \quad & \{\MtsVoraussetzung_1,\dots,\MtsVoraussetzung_N \}
	&,\;& \vec{\MtsFolgerung}     & \MtsEq \quad & \{\MtsFolgerung_1,\dots,\MtsFolgerung_M\}
	\label{eq-SRTc}          \\
	&       \MtsVoraussetzung_n   & \MtsEq \quad & ( \links{\MtsVoraussetzung}_n, \rechts{\MtsVoraussetzung}_n )
	&,\;& \MtsFolgerung_m         & \MtsEq \quad & ( \links{\MtsFolgerung}_m, \rechts{\MtsFolgerung}_m )
	&,\;& \text{für } 1 \le n \le N \text{ , } 1 \le m \le M
	\label{eq-SRTd}          \\
	& \links{\MtsVoraussetzung}_n & \MtsDerive_{\MtsVoraussetzungSet} \quad & \rechts{\MtsVoraussetzung}_n
	&,\;& \links{\MtsFolgerung}_m & \MtsDerive_{\MtsFolgerungSet}     \quad & \rechts{\MtsFolgerung}_m
	&,\;& \text{für } 1 \le n \le N \text{ , } 1 \le m \le M
	\label{eq-SRTe}          \formulatoleft
\end{align}
also
\begin{align}
	&  \vec{\MtsVoraussetzung} & = \quad & \{(\links{\MtsVoraussetzung}_n,
	\rechts{\MtsVoraussetzung}_n) \mid 1 \le n \le N \}
	\label{def-Voraussetzungen}
	\\
	&  \vec{\MtsFolgerung}   & = \quad & \{(\links{\MtsFolgerung}_m,
	\rechts{\MtsFolgerung}_m)   \mid 1 \le m \le M \}
	\label{def-Folgerungen} \formulatoleft\formulatoleft
\end{align}
und wir nennen auch das Paar $(\vec{\MtsVoraussetzung}, \vec{\MtsFolgerung})$ \Schlussregel.
Diese ist per se \beschraenkt\ und ein Element aus $\MtsTup(\MtsAllDerive)^2$.
Nun haben wir alternative Schreibweisen für \beschraenkte\ \Schlussregeln:%
\footnote{%
	Nach \eqref{def-AB}, \eqref{def-ab} und \vreffor{def-aabb} sind die "`Brüche"' \Aussagen, und keine Paare mehr.
	Die Äquivalenz der Aussagen steht schon in \vreffor{eq-AB}
}
\[
	\frac{             \MtsVoraussetzungSet}{             \MtsFolgerungSet} \; \MtsEquiv \;
	\frac{\MtsSet(\vec{\MtsVoraussetzung}) }{\MtsSet(\vec{\MtsFolgerung}) } \; \MtsEquiv \;
	\frac{        \vec{\MtsVoraussetzung}  }{        \vec{\MtsFolgerung}  } \; \MtsEquiv \;
	\frac{
		\links{\MtsVoraussetzung}_1 \MtsDerive_{\MtsVoraussetzungSet} \rechts{\MtsVoraussetzung}_1 \MtsUnd
		\dots \MtsUnd
		\links{\MtsVoraussetzung}_N \MtsDerive_{\MtsVoraussetzungSet} \rechts{\MtsVoraussetzung}_N }{
		\links{\MtsFolgerung}_1     \MtsDerive_{\MtsFolgerungSet}     \rechts{\MtsFolgerung}_1     \MtsUnd
		\dots \MtsUnd
		\links{\MtsFolgerung}_M     \MtsDerive_{\MtsFolgerungSet}     \rechts{\MtsFolgerung}_M
	}
	\quad \text{\defFt{, \defTxt{\Schlussregel}} oder \defTxt{\formalerSatz}}
	\tag{\tagFS} \label{def-FS}
\]

\subsection[Beweise]{\Beweise}% ------------------------------------------------
\label {sub-Beweise}

Für einen \defTxt{\Beweis} in \ASBA\ ist stets gegeben:%
\footnote{%
	\ASBA\ selbst kann nur endliche Mengen aBspeichern.
	Für \ASBA muss daher einschränkend $\MtsSchlussregelSet \in \MtsRelf(\MtsRelf(\MtsPotf(\MtsSprache)))$ und $\MtsErgebnisSet \in \MtsRelf(\MtsPotf(\MtsSprache))$ sein.
}
\begin{align}
	& \MtsSprache     &           \quad &
	&& \text{, eine Menge von \Formeln, die zugrundeliegende \defTxt{\Sprache}.}
	\label{def-Sprache}      \\
	& \MtsErsetzungSet   & \MtsSubsetEq \quad & \{ \MtsErsetzung \mid \MtsErsetzung : \MtsSprache \rightarrow \MtsSprache \}
	&& \text{, eine Menge von \Funktionen, die \defTxt{\Ersetzungen}.}
	\label{def-Ersetzung} \\
	& \MtsSchlussregelSet & \in       \quad & \MtsRelSchlussregel
	&& \text{, eine Menge von \defTxt{\Schlussregeln}.}
	\label{def-Schlussregel} \\
	& \MtsErgebnisSet        & \in       \quad & \MtsRelAllDerive
	&& \text{, eine Menge von \Ableitungen, die \defTxt{\Ergebnisse}.}
	\label{def-Folgerung} &&
\end{align}
%
Die \emph{\Ersetzungen} sorgen \textzB\ dafür, dass aus einer \allgemeingueltigenFormel\ wie  \seqqt{$\alpha \FrmImp (\beta \FrmImp \alpha)$} \textzB\ die \allgemeingueltigeFormel\ \seqqt{$\gamma \FrmImp (\delta \FrmImp \gamma)$} abgeleitet werden kann.
%
Die \emph{\Schlussregeln} geben erlaubte Schlussfolgerungen aus gegebenen Elementen an und umfassen auch die Voraussetzungen eines Satzes.
Die \emph{\Ergebnisse} schließlich sind das, was mittels eines Beweises aus den gegebenen Voraussetzungen \MtsSprache, \MtsErsetzungSet\ und \MtsSchlussregelSet\ gefolgert werden soll.

Im Fall von \beschraenkten\ \Schlussregeln\ können statt \MtsSchlussregelSet\ und \MtsErgebnisSet\ auch
\begin{align}
	& \vec{\MtsSchlussregel} & \in \quad & \MtsTup(\MtsTup(\MtsAllDerive)^2)
	&& \text{, ein \Tupel\ aus \defTxt{\Schlussregeln}.}
	\label{def-Schlussregelvector} \\
	& \vec{\MtsErgebnis}        & \in \quad & \quad \; \MtsTup(\MtsAllDerive)
	&& \text{, ein \Tupel\ aus \defTxt{\Ableitungen}, die \defTxt{\Ergebnisse}.}
	\label{def-Folgerungsvector}    \formulatoleft
\end{align}
gegeben sein. Mit
\begin{align}
	& \MtsSchlussregelSet \MtsDefEq \{ (\MtsSet(\vec{\MtsVoraussetzung}), \MtsSet(\vec{\MtsFolgerung})) \mid (\vec{\MtsVoraussetzung}, \vec{\MtsFolgerung}) \in \MtsSet(\vec{\MtsSchlussregel}) \}
	\\
	& \MtsErgebnisSet \MtsDefEq \MtsSet(\vec{\MtsErgebnis})
\end{align}
ergibt sich wegen \eqref{eq-Setf} und \vreffor{eq-vecrel} wieder die erste Form.

\subsection[Beispiel für einen Beweis]{Beispiel für einen \Beweis}% ------------
\label {sub-Beispielbeweis}

\todo{Nacharbeiten}     %TODO *** Nacharbeiten ***

\todo{Hier weitermachen}%TODO *** hier weitermachen ***

Zur Veranschaulichung ein Beispiel:\citenote{bib:HilbertKalkuelModusPonens}
\begin{align}
	& \MtsErsetzung_{\alpha,\beta}(\delta) & \MtsDefEq \quad & \text{das }\delta \text{, bei dem alle Vorkommen von $\alpha$ durch $\beta$ ersetzt wurden} \\
	& \MtsSprache & \MtsDefEq \quad & \text{die Menge aller \Formeln\ der aussagenlogischen \Sprache} \\
	& \MtsVoraussetzung_1    & \MtsDefEq \quad & (A, \{\alpha\}) \\
	& \MtsVoraussetzung_2    & \MtsDefEq \quad & (B, \{\alpha \FrmImp \beta\}) \\
	& \MtsVoraussetzung_3    & \MtsDefEq \quad & (A \cup B, \{\beta\}) \\
	& \MtsErsetzungSet   & \MtsDefEq \quad & \{\MtsErsetzung_{\alpha,\delta}, \MtsErsetzung_{\beta,B}, \MtsErsetzung_{\beta,B\FrmImp \delta}, \MtsErsetzung_{\gamma,\delta} \} \\
	& \MtsSchlussregelSet & \MtsDefEq \quad & \dots \\
	&          & \chi_1 \; \MtsDefEq \quad & \alpha \FrmImp (\beta \FrmImp \alpha) \\
	&          & \chi_2 \; \MtsDefEq \quad & (\alpha \FrmImp (\beta \FrmImp \gamma)) \FrmImp ((\alpha \FrmImp \beta) \FrmImp (\alpha \FrmImp \gamma)) \\
	& \MtsAxiomSet          & \MtsDefEq \quad & \{\chi_1, \chi_2\} \\
	& \MtsFolgerungRel     & \MtsDefEq \quad & \dots
	\formulatoleft
\end{align}
%TODO Beispiel vervollständigen

\subsection[Beweisschritte]{\Beweisschritte}% ----------------------------------
\label {sub-Beweisschritte}

%TODO Elimination von Voraussetzungen behandeln!
Ein \Beweis%
\footnote{\citesee{bib:Rautenberg} Kapitel~1.6 und~3.6}
in \ASBA\ besteht aus
\begin{align}
	& \text{einer \Schlussregel} && \frac{\MtsVoraussetzungSet}{\MtsFolgerungSet}
	\\
	& \text{einer Folge} && \MtsBeweisschrittTup = (\MtsBeweisschritt_1, \MtsBeweisschritt_2, ..., \MtsBeweisschritt_K)
	&& \text{von \emph{\Beweisschritten} } \MtsBeweisschritt_k
	&& \text{, die \defTxt{\Beweisschrittfolge}}
	\label{def-Beweisschrittfolge}
	\\
	& \text{einer Folge} && \MtsUmwandlungTup = (\MtsUmwandlung_1, \MtsUmwandlung_2, ..., \MtsUmwandlung_K)
	&& \text{von \emph{\Umwandlungen} } \MtsUmwandlung_k
	&& \text{, die \defTxt{\Umwandlungsfolge}}
	\label{def-Umwandlungsfolge}
\end{align}
Dabei ist $K$ ein Element aus \MtsINo, $0 \le k \le K$, die \defTxt{\Beweisschritte} $\MtsBeweisschritt_k$ sind \Schlussregeln\ und die \Umwandlungen\ $\MtsUmwandlung_k$ werden später definiert.
%TODO Verweis auf Definition der Umwandlungen fehlt
Wir definieren noch:
\begin{align}
	& \MtsBeweisschrittSet_k & \MtsDefEq \quad & \{\MtsBeweisschritt_1, \MtsBeweisschritt_2, ..., \MtsBeweisschritt_k\} & \quad \text{, für~~} 0 \le k \le K
	\label{def-Beweisschrittebis} \\
	& \MtsBeweisschrittSet   & \MtsDefEq \quad & \MtsBeweisschrittSet_K \label{def-Beweisschrittmenge}
	\formulatoleft\formulatoleft\formulatoleft
\end{align}
und nennen \MtsBeweisschrittSet\ die \defTxt{\Beweisschrittmenge} der \Beweisschrittfolge\ \MtsBeweisschrittTup.
Dann ist $\MtsBeweisschrittSet_0=\emptyset$ und $\MtsBeweisschrittSet_i\MtsSubsetEq\MtsBeweisschrittSet_j\MtsSubsetEq\MtsBeweisschrittSet$ für $0\le i\le j\le K$.
-- Wir nennen die \Beweisschrittfolge\ auch eine \defTxt{\Ableitung} aus \MtsFolgerungSet\ aus \MtsVoraussetzungSet.

%TODO Rolle der Umwandlungen erläutern
Jeder \Beweisschritt\ $ \MtsBeweisschritt_k \text{ für } 1 \le k \le K $ muss entweder eine \Voraussetzung\ aus \MtsVoraussetzungSet\ oder durch Anwendung einer \allgemeingueltigenSchlussregel\ auf eine Teilmenge von $\MtsBeweisschrittSet_{k-1}$ eine wahre \Formel\ oder eine weitere \allgemeingueltigeSchlussregel\ sein.
Schließlich muss noch
\[ \MtsFolgerungSet \MtsSubsetEq \MtsBeweisschrittSet \]
sein, da jede \Folgerung\ aus \MtsFolgerungSet\ in der Folge \MtsBeweisschrittTup\ vorkommen und somit Element der Menge \MtsBeweisschrittSet\ sein muss.

========================================================================
%TODO===================================================================

Bevor die \Schlussregeln\ weiter behandelt werden, werden noch Elemente der \emph{\Aussagenlogik} und der \emph{\Praedikatenlogik} behandelt.
Wir stützen uns dabei weitgehend auf~\cite{bib:Rautenberg}, ohne das jedes Mal anzugeben.

\section[Aussagenlogik]{\Aussagenlogik}% =======================================
\beginsection          {\Aussagenlogik}
\label              {sec-Aussagenlogik}

\color{gray}%%% Anfang grauer Text
\subsection[Konstante und Operationen]{Konstante und \Operationen}% ------------
\label               {sub-Operationen}

\vrefDtab{tab-Symbole}%
\footnote{%
	Die \tablename\ basiert auf den Wahrheitstafeln in~\cite{bib:JunktorMoeglich} Kapitel~2.2 und~\cite{bib:Rautenberg} Kapitel~1.1 Seite~3.
}
definiert für die zweiwertige Logik Konstante und \Junktoren\ über die \Wahrheitswerte\ ihrer Anwendung.
So ergeben sich, abhängig von den \Wahrheitswerten\ der Operanden $A$ und $B$,%
\footnote{%
	$A$ und $B$ können hier beliebige \Aussagen\ sein --- auch \Formeln\ ---, die jeweils genau einen \Wahrheitswert\ repräsentieren.
}
die in der \tablename\ angegebenen \Wahrheitswerte\ für die \Operationen.
Die mit 0, 1 und 2 benannten Spalten werden jeweils nur für die 0-, 1- und 2-stelligen \Junktoren, \textdh\ für die Konstanten, die unären und die binären \Junktoren\ ausgefüllt.
Dabei werden die Konstanten als 0-stellige \Junktoren\ angesehen.
Hat der Inhalt einer Zelle keine Relevanz, steht dort ein Minuszeichen, ist kein Wert bekannt, so bleibt sie leer.

\begin{table}[p]
	% Wahrheitswerte
	\newcommand*{\texttrue} {W}%            in einem Kommentar stets 'W'
	\newcommand*{\textfalse}{F}%            in einem Kommentar stets 'F'
	% Zähler für Prioritäten ---------------------------------------------------
	\newcounter{prio}    \setcounter{prio}    {1}
	\newcounter{pnot}    \setcounter{pnot}    {\value{prio}}
	\stepcounter{prio}% - - - - - - - - - - - - - - - - - -
	\newcounter{pand}    \setcounter{pand}    {\value{prio}}
	\newcounter{pnand}   \setcounter{pnand}   {\value{prio}}
	\stepcounter{prio}% - - - - - - - - - - - - - - - - - -
	\newcounter{por}     \setcounter{por}     {\value{prio}}
	\newcounter{pnor}    \setcounter{pnor}    {\value{prio}}
	\newcounter{pxor}    \setcounter{pxor}    {\value{prio}}
	\stepcounter{prio}% - - - - - - - - - - - - - - - - - -
	\newcounter{pimp}    \setcounter{pimp}    {\value{prio}}
	\newcounter{prep}    \setcounter{prep}    {\value{prio}}
	\stepcounter{prio}% - - - - - - - - - - - - - - - - - -
	\newcounter{pequiv}  \setcounter{pequiv}  {\value{prio}}
%%%	% Farben
%%%	\definecolor{cNormalUse}{rgb}{.80,.80,.80}
%%%	\definecolor{cRareUse}{rgb}{.90,.90,.99}
	% Trennlinien
	\newcommand*{\tablegroup}{\hdashline[6pt/3pt]}
	\newcommand*{\tableline} {\hdashline[3pt/3pt]}
	\newcommand*{\gapline}   {\cdashline{1-1}[1pt/3pt]\cdashline{9-11}[1pt/3pt]}
	\begin{threeparttable}
		\setlength\tabcolsep{3pt}
		\setlength\extrarowheight{1.5pt}
		\small
		\begin{tabularx}{\linewidth}{|c||c:cc:cccc|X:X|c|}
			\hline% -- Tabellenanfang --------------------------------------
			$A$ & - & \texttrue & \textfalse &%
			\texttrue  & \texttrue  & \textfalse & \textfalse &
			- & \Aussage\ $A$ & - \\
			\tableline%.................................................
			B & - & -       & -        &%
			\texttrue  & \textfalse & \texttrue  & \textfalse &
			- & \Aussage\ $B$ & - \\
			\hline% -- Überschrift -----------------------------------------
			\textbf{\Junktor}\Tnote{1} &
			\textbf{0}\Tnote{2} &
			\multicolumn{2}{c:}{\textbf{1}} &
			\multicolumn{4}{c|}{\textbf{2}} &
			\textbf{Name}\Tnote{3} &
			\textbf{Sprechweise} &
			\textbf{Prio}\Tnote{4} \\
			\hline\hline% == Konstante =====================================
%%%			\rowcolor{cRareUse}
			\defSym{\FrmTrue}
			& \texttrue  & - & - & - & - & - & -
			& Verum
			& \defFt{\TxtTrue} & - \\
			\tableline%.................................................
%%%			\rowcolor{cRareUse}
			\defSym{\FrmFalse}
			& \textfalse & - & - & - & - & - & -
			& Falsum
			& \defFt{\TxtFalse} & - \\
			\hline\hline% == unäre Junktoren ===============================
			& - & \texttrue & \texttrue  & - & - & - & - & & & - \\
			\tableline%.................................................
%%%			\rowcolor{cNormalUse}
			$(\dots)$
			& - & \texttrue & \textfalse & - & - & - & -
			& Klammerung
			& $A$ \defFt{ist geklammert} & -\Tnote{5} \\
			\tableline%.................................................
%%%			\rowcolor{cNormalUse}
			\defSym{\FrmNot}
			& - & \textfalse & \texttrue  & - & - & - & -
			& Negation
			& \defFt{Nicht} $A$ & \thepnot\Tnote{6} \\
			\tableline%.................................................
			& - & \textfalse & \textfalse & - & - & - & - & & & -  \\
			\hline\hline% == binäre Junktoren ==============================
			& - & - & - &\texttrue&\texttrue&\texttrue&\texttrue
			& Tautologie
			& & - \\
			\tableline%.................................................
%%%			\rowcolor{cNormalUse}
			\defSym{\FrmOr}
			& - & - & - &\texttrue&\texttrue&\texttrue&\textfalse
			& Disjunktion; Adjunktion;\newline Alternative
			& $A$ \defFt{oder} $B$ & \thepor \\
			\tableline%.................................................
%%%			\rowcolor{cRareUse}
			\defSym{\FrmRep} $\Leftarrow$ $\subset$
			& - & - & - &\texttrue&\texttrue&\textfalse&\texttrue
			& Replikation; Konversion;\newline konverse Implikation
			& $A$ \defFt{folgt aus} $B$ & \theprep \\
			\tableline%.................................................
			$\rfloor$
			& - & - & - &\texttrue&\texttrue&\textfalse&\textfalse
			& Präpendenz
			& Identität von $A$ & - \\
			\tablegroup% -----------------------------------------------
%%%			\rowcolor{cNormalUse}
			\defSym{\FrmImp} $\Rightarrow$ $\supset$
			& - & - & - &\texttrue&\textfalse&\texttrue&\texttrue
			& Implikation; Subjunktion;\newline Konditional
			& \defFt{Aus} $A$ \defFt{folgt} $B$; Wenn $A$ dann $B$;\newline
			$A$ nur dann wenn $B$ & \thepimp \\
			\tableline%.................................................
			$\lfloor$
			& - & - & - &\texttrue&\textfalse&\texttrue&\textfalse
			& Postpendenz
			& Identität von $B$ & - \\
			\tableline%.................................................
%%%			\rowcolor{cNormalUse}
			\defSym{\FrmEquiv} $\Leftrightarrow$
			& - & - & - &\texttrue&\textfalse&\textfalse&\texttrue
			& ~\Aequivalenz; Bijunktion;\newline Bikonditional
			& $A$ \defFt{genau dann wenn} $B$;\newline
			$A$ dann und nur dann wenn $B$
			& \thepequiv \\
			\tableline%.................................................
%%%			\rowcolor{cNormalUse}
			\defSym{\FrmAnd} $\&$ $\cdot$
			& - & - & - &\texttrue&\textfalse&\textfalse&\textfalse
			& Konjunktion
			& $A$ \defFt{und} $B$; Sowohl $A$ als auch $B$ & \thepand \\
			\tablegroup% -----------------------------------------------
%%%			\rowcolor{cRareUse}
			\defSym{\FrmNand} $\barwedge$ $\mid$
			& - & - & - &\textfalse&\texttrue&\texttrue&\texttrue
			& NAND; Unverträglichkeit;\newline Sheffer-Funktion
			& \defFt{Nicht zugleich} $A$ \defFt{und} $B$ & \thepnand \\
			\tableline%.................................................
%%%			\rowcolor{cRareUse}
			\defSym{\FrmXor} $\dot\vee$ $\veebar$ $\oplus$
			& - & - & - &\textfalse&\texttrue&\texttrue&\textfalse
			& XOR; Antivalenz;\newline ausschließende Disjunktion
			& \defFt{Entweder} $A$ \defFt{oder} $B$ & \thepxor \\
			\gapline%. . . . . . . . . . . . . . . . . . . . . . . . . .
			$\nleftrightarrow$ $\nLeftrightarrow$ $\nequiv$
			& - & - & - &"&"&"&"
			& ~\Kontravalenz
			& & - \\
			\tableline%.................................................
			$\lceil$
			& - & - & - &\textfalse&\texttrue&\textfalse&\texttrue
			& Postnonpendenz
			& Negation von $B$ & - \\
			\tableline%.................................................
			$\nrightarrow$ $\nRightarrow$ $\nsupset$
			& - & - & - &\textfalse&\texttrue&\textfalse&\textfalse
			& Postsektion
			& & - \\
			\tablegroup% -----------------------------------------------
			$\rceil$
			& - & - & - &\textfalse&\textfalse&\texttrue&\texttrue
			& Pränonpendenz
			& Negation von $A$ & - \\
			\tableline%.................................................
			$\nleftarrow$ $\nLeftarrow$ $\nsubset$
			& - & - & - &\textfalse&\textfalse&\texttrue&\textfalse
			& Präsektion
			& & - \\
			\tableline%.................................................
%%%			\rowcolor{cRareUse}
			\defSym{\FrmNor} $\overline\vee$
			& - & - & - &\textfalse&\textfalse&\textfalse&\texttrue
			& NOR; Nihilation;\newline Peirce-Funktion
			& \defFt{Weder} $A$ \defFt{noch} $B$ & \thepnor \\
			\tableline%.................................................
			& - & - & - &\textfalse&\textfalse&\textfalse&\textfalse
			& Kontradiktion
			& & - \\
			\hline% -- Tabellenende ----------------------------------------
			\multicolumn{11}{l}{~} \\
			\multicolumn{11}{l}{\parbox{\linewidth-6pt}{
				Um vollständig zu sein, \textdh\ alle 22 möglichen Kombinationen von \Wahrheitswerten\ für höchstens zwei Variable zu berücksichtigen, enthält die \tablename\ auch viele ungebräuchliche \Symbole\ und \Operationen.
%%%				Die Zeilen mit den Klammern und den gebräuchlichsten \Junktoren\ sind in der \tablename\ grau hinterlegt.
%%%				Hellgrau hinterlegt sind Zeilen mit weniger gebräuchlichen \Junktoren.
				\Junktoren\ ohne Angabe einer Priorität sind in diesem Dokument nicht weiter von Interesse.
				--- Im Folgenden werden von den in der Tabelle aufgeführten \Junktoren\ nur noch \defSym{\FrmFalse}, \defSym{\FrmTrue}, \defSym{\FrmNot}, \defSym{\FrmAnd}, \defSym{\FrmOr}, \defSym{\FrmImp}, \defSym{\FrmRep}, \defSym{\FrmEquiv}, \defSym{\FrmNand}, \defSym{\FrmNor} und \defSym{\FrmXor} verwendet.
			}} \\
			\multicolumn{11}{l}{~} \\
			\hline% -- Fußnoten --------------------------------------------
		\end{tabularx}
		\begin{tablenotes}
			\footnotesize
			%
			\item[1] Die \Junktoren\ \chrqt{$\subset$}, \chrqt{$\supset$}, \chrqt{$\nsubset$} und \chrqt{$nsupset$} haben hier nicht die Bedeutung der entsprechenden \Operationen\ der \Mengenlehre\ und dürfen nicht damit verwechselt werden; entsprechendes gilt für \chrqt{$+$} und \chrqt{$\cdot$} mit Addition und Multiplikation.
			%
			\item[2] 0-stellige \Junktoren\ sind Konstante, hier \emph{\Wahrheitswerte}.
			%
			\item[3] Ist eine Zelle in dieser Spalte leer, so ist die zugehörige Zeile nur vorhanden, um alle binären \Junktoren\ aufzuführen.
			%
			\item[4] Je kleiner die Zahl, je höher die Priorität.
			%
			\item[5] Klammerung ist genau genommen keine \Operation\ und wird nicht nur bei logischen, sondern auch bei anderen Ausdrücken verwendet. Ihre Priorität - sofern man überhaupt davon sprechen kann - kann nur höher als die aller \Junktoren\ sein.
			%
			\item[6] Die Priorität der unären \Operationen\ muss höher sein als die aller mehrwertigen, also auch der binären \Operationen.
			Wenn die Symbole aller unären \Operationen\ auf derselben Seite des Operanden stehen, brauchen sie eigentlich keine Priorität, da die Auswertung nur von innen (dem Operanden) nach außen erfolgen kann.
			Nur wenn es sowohl links-, als auch rechtsseitige unäre \Operationen\ gibt, muss man für diese Prioritäten definieren.
			%
		\end{tablenotes}
	\end{threeparttable}
	\caption{Definition von aussagenlogischen Symbolen.}
	\label{tab-Symbole}% Erst nach '\caption'!
\end{table}

Für einige \Junktorsymbole%
\footnote{%
	Symbole, die für \Junktoren\ verwendet werden.
},
Namen und Sprechweisen sind auch Alternativen angegeben.
Die durchgestrichenen (\textdh\ negierten) Symbole sind ungebräuchlich und nur aus formalen Gründen aufgeführt.
Wenn für eine bestimmte Kombination von \Wahrheitswerten\ mehr als eine Zeile angegeben ist, so können die zugehörigen \Junktoren\ zwar formal verschieden sein, liefern in der zweiwertigen \Aussagenlogik\ jedoch dieselben Ergebnisse.

Die zur Einsparung von Klammern definierten Prioritäten sind \vrefintab{tab-Prioritaeten} angegeben.%
\footnote{Zur Erinnerung: Es gilt Rechtsklammerung. \vrefseesub{sub-Prioritaeten}}

\subsection{Formalisierung}% ---------------------------------------------------
\label {sub-Formalisierung}

Da sie die Grundlage --- quasi das Fundament --- des mathematischen Inhalts von \ASBA\ sind, müssen die \Axiome, \Saetze, \Beweise, \textusw\ der \Aussagenlogik\ (und später der \Praedikatenlogik) in streng formaler Form vorliegen.%
\footnote{%
	Die Formalisierung stützt sich auf~\cite{bib:AussagenlogikFormalerZugang}; \alsoname~\cite{bib:LogikDe, bib:LogikEn}.
}
Da Computerprogramme mit der \emph{\PolnischenNotation}%
\footnote{%
	Bei der \defTxt{\PolnischenNotation} stehen die Operanden \textbzw\ Argumente von \Relationen\ und \Funktionen\ stets rechts von den Relations- und Funktionssymbolen.
	Dadurch kann auf Gliederungszeichen wie Klammern und Kommata verzichtet werden.
	Noch einfacher für Computer ist die \defFt{umgekehrte Polnische Notation}, bei der die Operanden und Argumente links von den Symbolen stehen.
}
besser umgehen können und Klammern dort überflüssig sind, werden viele \Formeln\ auch parallel in der Polnischen Notation angegeben.
Dies wird auf Wunsch auch bei Ausgaben von \ASBA\ so gehandhabt.

\subsubsection{Bausteine der aussagenlogischen Sprache}% - - - - - - - - - - - -
\label {subsub-Bausteine}

%TODO * hier bei der Aussagenlogik weitermachen
Zur Einteilung der \Junktoren\ werden die folgenden Mengen definiert:
\begin{align}
	& \defSym{\FrmCon}              & \MtsDefEq \quad & \{ \FrmTrue, \FrmFalse \}
	&& \text {, Menge der \defFt{aussagenlogischen Konstanten}}
	\idx{Konstante, aussagenlogisch, Menge davon} \label{def-C}
	\\
	& \defSym{\FrmUna}              & \MtsDefEq \quad & \{ \FrmNot \}
	&& \text{, Menge der \defFt{unären \Junktoren}}
	\idx{Junktor, unär, Menge davon}              \label{def-U}
	\\
	& \defSym{\FrmBin}              & \MtsDefEq \quad &
	\{ \FrmAnd, \FrmOr, \FrmXor, \FrmImp, \FrmEquiv, \FrmRep, \FrmNand, \FrmNor \}
	&& \text{, Menge der \defFt{binären \Junktoren}}
	\idx{Junktor, binär, Menge davon}             \label{def-B}
\end{align}

Um damit \Formeln\ zu bilden, werden noch Variable gebraucht:
\begin{align}
	& \defSym{\FrmVar}  & \MtsDefEq \quad & \{ \Frmvar_n \mid n \in \MtsINo \}
	&&&&
	&& \text{, Menge der \defFt{aussagenlogischen Variablen}} \label{def-FrmVar}
	&&
\end{align}

Die Mengen \FrmCon, \FrmUna, \FrmBin\ und \FrmVar\ müssen paarweise disjunkt sein.
-- Damit können die folgende Mengen definiert werden:
\begin{align}
	& \defSym{\FrmJun}  & \MtsDefEq      \qquad & \FrmCon \cup \FrmUna \cup \FrmBin
	&& \text{, Menge der \defTxt{\Junktorsymbole}}
	\idx{Junktorsymbole, Menge der} \label{def-FrmJun}
	\\
	& \defSym{\FrmABC}  & \MtsDefEq      \qquad & \FrmVar \cup \FrmJun
	&& \text{, \defFt{Alphabet der aussagenlogischen \Sprache}
	(\defFt{für} \FrmJun)}       \label{def-FrmABC}
	\\
	& \defSym{\FrmJunx} & \MtsSubsetEq\; \qquad & \FrmJun
	&& \text{, eine Teilmenge von } \FrmJun \text{ für eine Indexvariable } x
	~                               \label{def-FrmJunx}
	\\
	& \defSym{\FrmABCx} & \MtsDefEq      \qquad & \FrmVar \cup \FrmJunx \quad
	&& \text{, das Alphabet der aussagenlogischen Sprache
	für } \FrmJunx                \label{def-FrmABCx}
\end{align}

Für Elemente aus \FrmVar\ verwenden wir normalerweise die kleinen, lateinischen Buchstaben $a$, $b$, $c$, usw.

\subsubsection{Aussagenlogische Formeln}%  - - - - - - - - - - - - - - - - - - -
\label                  {subsub-Formeln}

Neben dem Alphabet \FrmABC\ \textbzw\ \FrmABCx\ werden noch Klammern als Gliederungszeichen verwendet.
Damit können nun rekursiv für jede Teilmenge $\FrmJun_x$ von \FrmJun\ zwei Mengen von aussagenlogischen \Formeln\ definiert werden, wobei wir für diese \Formeln\ die kleinen, griechischen Buchstaben $\alpha$, $\beta$, $\gamma$, \textusw\ verwenden.

$\defSym{\FrmForx}$ sei die Menge der auf folgende Weise definierten \defTxt{aussagenlogischen \Formeln} \defFt{mit Klammerung} zum Alphabet \defSym{\FrmABCx}\%
\idx{Formel, aussagenlogisch mit Klammerung}:
\begin{align}
	&                    & \FrmVar                    \MtsSubset \FrmForx
	&&& \text{, die Variablen}  \label{def-ForQ}
	\\
	&                    & \FrmJun_x \cap \FrmCon   \MtsSubset \FrmForx
	&&& \text{, die Konstanten} \label{def-ForJK}
	\\
	\alpha \in\FrmForx & \quad\MtsImp & (\BspOpU\alpha)\in\FrmForx
	&&& \text{, für} \quad \BspOpU \in \FrmUna \cap \FrmJunx
	\label{def-ForU}
	\\
	\alpha,\beta\in\FrmForx & \quad\MtsImp & (\alpha\BspOpB\beta) \in\FrmForx
	&&& \text{, für} \quad \BspOpB  \in \FrmBin \cap \FrmJunx
	\label{def-ForB}
	\formulatoleft
\end{align}

Nur die auf diese Weise konstruierten \Formeln\ sind Elemente aus \FrmForx.
-- Für \FrmJunx\ = \FrmJun\ sei noch $\defSym{\FrmFor} \MtsDefEq \FrmForx$.

$\defSym{\FrmForpx}$ sei die Menge der auf folgende Weise definierten \defTxt{\aussagenlogischenFormeln} \defFt{in Polnischer Notation}\idx{Formel, aussagenlogisch in Polnischer Notation}:
\begin{align}
	&                          & \FrmVar                   \MtsSubset \FrmForpx
	&&& \text{, die Variablen}  \label{def-ForQp}
	\\
	&                          & \FrmJunx \cap \FrmCon   \MtsSubset \FrmForpx
	&&& \text{, die Konstanten} \label{def-ForJKp}
	\\
	\alpha      \in\FrmForpx & \quad\MtsImp & \BspOpU\alpha \in\FrmForpx
	&&& \text{, für}  \quad \BspOpU \in \FrmUna  \cap \FrmJunx
	\label{def-ForUp}
	\\
	\alpha,\beta\in\FrmForpx & \quad\MtsImp & \BspOpB\alpha\beta\in\FrmForpx
	&&& \text{, für}  \quad \BspOpB  \in \FrmBin  \cap \FrmJunx
	\label{def-ForBp} \formulatoleft
\end{align}

Nur die auf diese Weise konstruierten \Formeln\ sind Elemente aus \FrmForpx.
-- Für \FrmJunx\ = \FrmJun\ sei noch $\defSym{\FrmForp} \MtsDefEq \FrmForpx$.

Wie man leicht sieht, gilt
\begin{equation}
	\FrmJunx      \: \MtsSubset \: \FrmJun_y  \: \MtsSubsetEq \: \FrmJun \MtsImp
	\begin{cases}
		\FrmABCx  \: \MtsSubset \: \FrmABC_y  \: \MtsSubsetEq \: \FrmABC \\
		\FrmForx  \; \MtsSubset \: \FrmFor_y  \; \MtsSubsetEq \: \FrmFor \\
		\FrmForpx \, \MtsSubset \: \FrmForp_y \, \MtsSubsetEq \: \FrmForp
	\end{cases}
\end{equation}
und weiterhin gibt es eine bijektive Abbildung von \FrmFor\ nach \FrmForp. Auf einen \Beweis\ verzichten wir.
%
Durch Anwendung der Klammerregeln \vrefvonsubsub{subsub-Bausteine} lassen sich in der Regel noch viele Klammern der \Formeln\ aus \FrmForx\ einsparen.
Die \Formeln\ aus \FrmForpx\ sind frei von Klammern.
Die Namen der \Junktoren\ finden sich \vrefintab{tab-Symbole}.

Die \Formeln, die nach einer der Regeln \eqref{def-ForU}, \eqref{def-ForB}, \eqref{def-ForUp} oder \eqref{def-ForBp} gebildet wurden, sind offensichtlich \defFt{\zerlegbar}, die anderen, \textdh\ Variablen und Konstanten (aus \FrmVar\ \textbzw\ \FrmCon), sind \defFt{\unzerlegbar}. Letztere bezeichnet man auch als \defFt{\atomare\ \Formeln}.

\subsection[Definition von Junktoren durch andere]{\Definition\ von \Junktoren\ durch andere}
\label                {sub-JunktorDef}

Im folgenden gelte für zwei aussagenlogische \Formeln\ $\alpha$ und $\beta$:
\begin{itemize}
	\item[] $\alpha \defSymBin{\MtsEq}    \beta \quad \MtsDefEquiv$ \quad $\alpha$ und $\beta$
	stimmen als \Zeichenkette\ überein.
	%
	\item[] $\alpha \defSymBin{\MtsEquiv} \beta \quad \MtsDefEquiv$ \quad
	\parbox[t]{13cm}{$\alpha$ und $\beta$ können mit Hilfe erlaubter \Ersetzungen\ und geltender \Axiome\ --- \vrefseesub{sub-Axiome} --- ineinander überführt werden.}
\end{itemize}
%TODO === Verweise auf Definitionen prüfen und hinzufügen

Es werden verschiedene Teilmengen von \FrmJun\ eingeführt, die jeweils ausreichen um alle anderen Elemente aus \FrmJun\ zu definieren:
\begin{align}
	&\FrmJun_{\iBool}&\MtsDefEq\quad&\{\FrmNot,\FrmAnd,\FrmOr\}\label{def-Jbool}
	\qquad \text{(\defTxt{\BoolescheSignatur})}
	\\
	&\FrmJun_{\iAnd} &\MtsDefEq\quad&\{\FrmNot,\FrmAnd \}\label{def-Jand}
	\\
	&\FrmJun_{\iOr}  &\MtsDefEq\quad&\{\FrmNot,\FrmOr  \}\label{def-Jor}
	\\
	&\FrmJun_{\iImp} &\MtsDefEq\quad&\{\FrmNot,\FrmImp \}\label{def-Jimp}
	\\
	&\FrmJun_{\iRep} &\MtsDefEq\quad&\{\FrmNot,\FrmRep \}\label{def-Jrep}
	\\
	&\FrmJun_{\iNand}&\MtsDefEq\quad&\{\FrmNand        \}\label{def-Jnand}
	\\
	&\FrmJun_{\iNor} &\MtsDefEq\quad&\{\FrmNor         \}\label{def-Jnor}
	\formulatoleft\formulatoleft\formulatoleft
\end{align}
Solche Teilmengen heißen \logischeSignatur.

Im Folgenden stehen jeweils links die \Formeln\ in üblicher Schreibweise vollständig geklammert und rechts in Polnischer Notation (ohne Klammern).
Ferner seien $\alpha$ und $\beta$ beliebige, nicht notwendig verschiedene \Formeln\ aus der passenden Menge \FrmForx\ \textbzgl\ der um die mit Hilfe der Definitionen erweiterten \Formelmenge.

Ausgehend von den \Junktoren\ aus der \BooleschenSignatur\ $\FrmJun_{\iBool}$ werden die restlichen \Junktoren\ aus \FrmJun\ definiert.
Die Definitionen sind in zwei Gruppen eingeteilt, und zwar die mit den \Junktoren\ aus $\FrmJun_{\iAnd}$:
\begin{align}
	% folgt ------------------------
	(\alpha \defSymBin{\FrmImp}   \beta) &\;\MtsDefEq\; (\FrmNot (\alpha \FrmAnd  (\FrmNot \beta))) &
	\defSymBin{\FrmImp}  \alpha   \beta  &\;\MtsDefEq\;  \FrmNot    \FrmAnd \alpha \FrmNot \beta
	\label{def-imp}
	\\
	% sofern -----------------------
	(\alpha \defSymBin{\FrmRep}   \beta) &\;\MtsDefEq\; (\FrmNot (\beta \FrmAnd  (\FrmNot \alpha))) &
	\defSymBin{\FrmRep}   \beta  \alpha  &\;\MtsDefEq\;  \FrmNot    \FrmAnd \beta \FrmNot \alpha
	\label{def-rep}
	\\
	% genau dann -------------------
	(\alpha \defSymBin{\FrmEquiv} \beta) &\;\MtsDefEq\; ((\alpha \FrmImp \beta) \FrmAnd (\alpha \FrmRep \beta)) &
	\defSymBin{\FrmEquiv} \alpha  \beta  &\;\MtsDefEq\; \FrmAnd \FrmImp \alpha \beta \FrmRep \alpha \beta
	\label{def-equiv}
	\\
	%falsch ------------------------
	\defSym{\FrmFalse}                   &\;\MtsDefEq\; (\Frmvar_0 \FrmAnd (\FrmNot \Frmvar_0)) &
	\defSym{\FrmFalse}                   &\;\MtsDefEq\;  \FrmAnd \Frmvar_0  \FrmNot \Frmvar_0   \label{def-false}
	\\
	% NAND -------------------------
	(\alpha \defSymBin{\FrmNand}  \beta) &\;\MtsDefEq\; (\FrmNot (\alpha \FrmAnd \beta )) &
	\defSymBin{\FrmNand}  \alpha  \beta  &\;\MtsDefEq\;  \FrmNot  \FrmAnd \alpha \beta \label{def-nand}
\end{align}
und die mit den \Junktoren\ aus $\FrmJun_{\iOr}$:
\begin{align}
	% NOR --------------------------
	(\alpha \defSymBin{\FrmNor}   \beta) & \;\MtsDefEq\; (\FrmNot (\alpha \FrmOr \beta))   &
	\defSymBin{\FrmNor}   \alpha  \beta  & \;\MtsDefEq\;  \FrmNot  \FrmOr \alpha \beta \label{def-nor}
	\\
	% plus -------------------------
	(\alpha \defSymBin{\FrmXor}   \beta) & \;\MtsDefEq\; ((\alpha \FrmOr \beta) \FrmAnd (\FrmNot (\alpha \FrmAnd \beta)))&
	\defSymBin{\FrmXor}   \alpha  \beta  & \;\MtsDefEq\;  \FrmAnd \FrmOr\alpha\beta \FrmNot \FrmAnd \alpha \beta
	\label{def-add}
	\\
	% wahr -------------------------
	\defSym{\FrmTrue} & \;\MtsDefEq\; (\Frmvar_0 \FrmOr (\FrmNot \Frmvar_0)) &
	\defSym{\FrmTrue} & \;\MtsDefEq\;  \FrmOr \Frmvar_0  \FrmNot \Frmvar_0
	\label{def-true}
\end{align}

Ist \chrqt{\FrmOr} oder \chrqt{\FrmAnd} nicht vorgegeben, \textdh\ wird von den Elementen aus $\FrmJun_{\iAnd}$ \textbzgl\ $\FrmJun_{\iOr}$ statt von denen aus $\FrmJun_{\iBool}$ ausgegangen, so muss man den fehlenden \Junktor\ mittels der passenden der beiden folgenden Definitionen einführen:
\begin{align}
	% and --> or -------------------
	(\alpha \FrmOr \beta)  & \;\MtsDefEq\; (\FrmNot((\FrmNot\alpha)\FrmAnd(\FrmNot\beta))) &
	\FrmOr \alpha  \beta   & \;\MtsDefEq\;  \FrmNot \FrmAnd \FrmNot \alpha \FrmNot \beta
	\label{def-orand} \\
	% or --> and -------------------
	(\alpha \FrmAnd \beta) & \;\MtsDefEq\; (\FrmNot((\FrmNot\alpha)\FrmOr(\FrmNot\beta)))  &
	\FrmAnd \alpha  \beta  & \;\MtsDefEq\;  \FrmNot \FrmOr \FrmNot \alpha \FrmNot \beta
	\label{def-andor}
\end{align}
Nun sind wieder alle \Junktoren\ definiert.

Entsprechend wird bei Vorgabe von $\FrmJun_{\iImp}$ \textbzgl\ $\FrmJun_{\iRep}$ die passende der beiden folgenden Definitionen ausgewählt:
\begin{align}
	% imp --> or -------------------
	(\alpha \FrmOr  \beta) & \;\MtsDefEq\; ((\FrmNot \alpha) \FrmImp \beta)         &
	\FrmOr \alpha   \beta  & \;\MtsDefEq\;   \FrmImp \FrmNot \alpha \beta
	\label{def-orrep}
	\\
	% rep --> and ------------------
	(\alpha \FrmAnd \beta) & \;\MtsDefEq\; (\FrmNot ((\FrmNot \beta) \FrmRep \alpha)) &
	\FrmAnd \alpha  \beta  & \;\MtsDefEq\;  \FrmNot \FrmRep \FrmNot \beta \alpha
	\label{def-andrep}
\end{align}
woraufhin dann \eqref{def-imp} \textbzgl\ \eqref{def-rep} als Gleichung nachzuweisen ist.
Da aus \eqref{def-rep} durch \Vertauschung\ der Variablen unmittelbar
\begin{align}
	(\alpha \FrmRep \beta) & \;\MtsEquiv\; (\beta \FrmImp \alpha) &
	\FrmRep \alpha  \beta  & \;\MtsEquiv\;  \FrmImp \beta \alpha  \label{eq-repimp}
\end{align}
folgt, vermindert sich der Aufwand dazu erheblich.

Bei Vorgabe von $\FrmJun_{\iNand}$ \textbzgl\ $\FrmJun_{\iNor}$ schließlich werden die passenden Definition aus
\begin{align}
	% nor --> not ------------------
	(\FrmNot \alpha) & \;\MtsDefEq\; (\alpha \FrmNor \alpha)  &
	\FrmNot  \alpha  & \;\MtsDefEq\;  \FrmNor \alpha \alpha   \label{def-notnor} \\
	% nand --> not -----------------
	(\FrmNot \alpha) & \;\MtsDefEq\; (\alpha \FrmNand \alpha) &
	\FrmNot  \alpha  & \;\MtsDefEq\;  \FrmNand \alpha \alpha  \label{def-notnand}
\end{align}
und, da \chrqt{\FrmNot} jetzt definiert ist, aus
\begin{align}
	% nor --> or -------------------
	(\alpha \FrmOr \beta)  & \;\MtsDefEq\; (\FrmNot(\alpha \FrmNor \beta))  &
	\FrmOr \alpha  \beta   & \;\MtsDefEq\;  \FrmNot \FrmNor \alpha \beta
	\label{def-ornor} \\
	% nand --> and -----------------
	(\alpha \FrmAnd \beta) & \;\MtsDefEq\; (\FrmNot(\alpha \FrmNand \beta)) &
	\FrmAnd \alpha  \beta  & \;\MtsDefEq\;  \FrmNot \FrmNand \alpha \beta
	\label{def-andnand}
\end{align}
ausgewählt und es ist \eqref{def-nand} \textbzgl\ \eqref{def-nor} als Gleichung nachzuweisen.

Abschließend ist noch nachzuweisen, dass mit Hilfe der jeweils passenden der Definitionen \eqref{def-imp} bis \eqref{def-andnand}, ausgehend vom jeweils passenden \FrmForx, genau die gesamte \Formelmenge\ \FrmFor\ erzeugt werden kann.

\subsection[Aussagenlogisches Axiomensystem]{Aussagenlogisches \Axiomensystem}%
\label                   {sub-Axiome}

Ausgehend von der \logischenSignatur\ $\FrmJun_\iAnd = \{\FrmNot, \FrmAnd\}$ und der \vrefdef{def-imp} von \chrqt{\FrmImp} werden die folgenden vier logischen \Axiome\ definiert:
\begin{align}
	&
	(\alpha\FrmImp\beta\FrmImp\gamma)\FrmImp(\alpha\FrmImp\beta)\FrmImp(\alpha\FrmImp\gamma)
	\formulaspace &
	& \FrmImp\FrmImp\alpha\FrmImp\beta\gamma\FrmImp\FrmImp\alpha\beta\FrmImp\alpha\gamma \\
	%
	& \alpha \FrmImp \beta \FrmImp \alpha \FrmAnd \beta
	\formulaspace &
	& \FrmImp \alpha \FrmImp \beta \FrmAnd \alpha \beta \\
	%
	& \alpha \FrmAnd \beta \FrmImp \alpha \;; \quad \alpha \FrmAnd \beta \FrmImp \beta
	\formulaspace &
	& \FrmImp \FrmAnd \alpha \beta \alpha \;; \quad \FrmImp \FrmAnd \alpha \beta \beta\\
	%
	&(\alpha \FrmImp \FrmNot \beta) \FrmImp (\beta \FrmImp \FrmNot \alpha)
	\formulaspace &
	& \FrmImp \FrmImp \alpha \FrmNot \beta \FrmImp \beta \FrmNot \alpha
	\formulatoleft
	%
\end{align}
\todo{Aussagenlogik weiter bearbeiten.}
%TODO Aussagenlogik weiter bearbeiten.

Siehe \defTxt{\Aussagenlogik} im Glossar.

\glsdesc{Aussagenlogik}

\section[Prädikatenlogik]{\Praedikatenlogik}% ==================================
\beginsection            {\Praedikatenlogik}
\label                {sec-Praedikatenlogik}

\todo{Prädikatenlogik bearbeiten.}
%TODO Prädikatenlogik bearbeiten.

Siehe \defTxt{\Praedikatenlogik} im Glossar.

\glsdesc{Praedikatenlogik}

\section[Mengenlehre]{\Mengenlehre}% ===========================================
\beginsection        {\Mengenlehre}
\label            {sec-Mengenlehre}

\todo{Mengenlehre bearbeiten.}
%TODO Mengenlehre bearbeiten.

Siehe \defTxt{\Mengenlehre} im Glossar.

\glsdesc{Mengenlehre}

\Endchapter
\color{black}%%% Ende grauer Text

	%%############################################################################%%
%%                                                                            %%
%% Datei:  ASBA-Design.tex                                                    %%
%% Inhalt: Kapitel "Design"                                                   %%
%%                                                                            %%
%% Copyright (C) 2017  Winfried Teschers                                      %%
%%                                                                            %%
%% This program is free software: you can redistribute it and/or modify       %%
%% it under the terms of the GNU Affero General Public License as published   %%
%% by the Free Software Foundation, either version 3 of the License, or       %%
%% (at your option) any later version.                                        %%
%%                                                                            %%
%% This program is distributed in the hope that it will be useful,            %%
%% but WITHOUT ANY WARRANTY; without even the implied warranty of             %%
%% MERCHANTABILITY or FITNESS FOR A PARTICULAR PURPOSE.  See the              %%
%% GNU Affero General Public License for more details.                        %%
%%                                                                            %%
%% You should have received a copy of the GNU Affero General Public License   %%
%% along with this program.  If not, see <http://www.gnu.org/licenses/>.      %%
%%                                                                            %%
%% Dr. Winfried Teschers                                                      %%
%% Anton-Günther-Straße 26c                                                   %%
%% 91083 Baiersdorf                                                           %%
%% Germany                                                                    %%
%%                                                                            %%
%% e-mail: winfried.teschers@t-online.de                                      %%
%%                                                                            %%
%%############################################################################%%

% !TeX root = ASBA.tex
% !TeX encoding = UTF-8
% !TeX spellcheck = de_DE

\chapter     {Design}% #########################################################
\beginchapter{Design}
\label   {cha-Design}

Dieses Projekt soll Open Source sein.
Daher gilt für die Dokumente die \emph{GNU Free Documentation License (FDL)} und für die Software die \emph{GNU Affero General Public License (APGL)}.
Die \emph{GNU General Public License (GPL)} reicht für die Software nicht aus, da das Programm auch mittels eines Servers betrieben werden kann und soll.
Damit das Projekt gegebenenfalls durch verschiedene Entwickler gleichzeitig bearbeitet werden kann und wegen des Konfigurationsmanagements wurde es als ein GitHub Projekt erstellt (\citesee{bib:ASBA}).

Wenn die Lizenzen nicht mitgeliefert wurden, können sie unter \url{http://www.gnu.org/licenses/} gefunden werden.

\section     {Anforderungen}% ==================================================
\beginsection{Anforderungen}
\label   {sec-Anforderungen}

Die Anforderungen ergeben sich zunächst aus den Zielen \vrefinsec{sec-Ziele}.
Die beiden Ziele~\ref{Ziel-Daten}~\emph{Daten} und~\ref{Ziel-Lizenz}~\emph{Lizenz} sind für die Entwicklung von \ASBA\ von sekundärer Bedeutung und werden daher in diesem \sectionname\ nicht übernommen.
Die anderen Ziele werden noch verfeinert.

\todo{Ziele aus Abschnitt "'Ziele"' in Anforderungen umwandeln.}
%TODO Ziele aus Abschnitt "'Ziele"' in Anforderungen umwandeln.
%
\begin{enumerate}

	\item \label{Anforderung-Form} \emph{Form}:
	Die Daten liegt in formaler, geprüfter Form vor.
	(\vrefseeziel{Ziel-Form})

	\item \label{Anforderung-Eingaben} \emph{Eingaben}:
	Die Eingabe von Daten erfolgt in einer formalen Syntax unter Verwendung der üblichen mathematischen Schreibweise.
	Folgende Daten können eingegeben werden:
	\begin{enumerate}
		\item \Axiome
		\item \Saetze
		\item \Beweise
		\item \Fachbegriffe
		\item \Fachgebiete
		\item \Ausgabeschemata
	\end{enumerate}
	Dabei sind alle Begriffe nur innerhalb eines Fachgebiets und seiner untergeordneten \Fachgebiete\ gültig, solange sie nicht umdefiniert werden.
	Das oberste \Fachgebiet\ ist die ganze Mathematik.
	--- \vrefseeziel{Ziel-Eingaben}

	\item \label{Anforderung-Pruefung} \emph{Prüfung}:
	Vorhandene \Beweise\ können automatisch geprüft werden.
	--- \vrefseeziel{Ziel-Pruefung}

	\item \label{Anforderung-Ausgaben} \emph{Ausgaben}:
	Die Ausgabe kann in einer eindeutigen, formalen Syntax gemäß vorhandener \Ausgabeschemata\ erfolgen.
	--- \vrefseeziel{Ziel-Ausgaben} - \emph{Ausgabe in Polnischer Notation}

	\item \label{Anforderung-Auswertungen} \emph{Auswertungen}:
	Zusätzlich zur Ausgabe der Daten sind verschiedene Auswertungen möglich.
	Insbesondere kann zu jedem \Beweis\ angegeben werden, wie lang er ist und welche \Axiome\ und Sätze%
	\footnote{Sätze, die quasi als \Axiome\ verwendet werden.}
	er benötigt.
	--- \vrefseeziel{Ziel-Auswertungen}

	\item \label{Anforderung-Anpassbarkeit} \emph{Anpassbarkeit}:
	\Fachbegriffe\ und die Darstellung bei der Ausgabe können mit Hilfe von --- gegebenenfalls unbenannten --- untergeordneten \Fachgebieten\ angepasst werden.
	--- \vrefseeziel{Ziel-Anpassbarkeit}

	\item \label{Anforderung-Individualitaet} \emph{Individualität}:
	\Axiome\ und Sätze können für jeden \Beweis\ individuell vorausgesetzt werden.
	Dabei sind fachgebietsspezifische \Fachbegriffe\ erlaubt.
	--- \vrefseeziel{Ziel-Individualitaet})

	\item \label{Anforderung-Internet} \emph{Internet}:
	Die Daten können auf mehrere Dateien verteilt sein.
	Ein Teil davon --- oder sogar alle --- können im Internet liegen.
	--- \vrefseeziel{Ziel-Internet}

	\item \label{Anforderung-Kommunikation} \emph{Kommunikation}:
	Die Kommunikation mit \ASBA\ kann mit den \Fachbegriffen\ der einzelnen \Fachgebiete\ erfolgen.
	--- \vrefseeziel{Ziel-Kommunikation}

	\item \label{Anforderung-Zugriff} \emph{Zugriff}:
	Der Zugriff auf \ASBA\ kann lokal und über das Internet erfolgen.
	--- \vrefseeziel{Ziel-Zugriff}

	\item \label{Anforderung-Unabhaengigkeit} \emph{Unabhängigkeit}:
	\ASBA\ kann offline und online arbeiten.
	--- \vrefseeziel{Ziel-Unabhaengigkeit}

	\item \label{Anforderung-Rekursion} \emph{Rekursion}:
	Es kann rekursiv über alle verwendeten Dateien --- auch solchen, die im Internet liegen --- ausgewertet werden.
	--- \vrefseeziel{Ziel-Rekursion}

	\item \label{Anforderung-Bedienbarkeit} \emph{Bedienbarkeit}:
	\ASBA\ ist einfach zu bedienen.
	--- \vrefseeziel{Ziel-Bedienbarkeit}

\end{enumerate}

\section[Axiome]{\Axiome}% =====================================================
\beginsection   {\Axiome}
\label       {sec-Axiome}
\todo{Axiome auswählen und definieren.}
%TODO Axiome auswählen und definieren.

\section[Beweise]{\Beweise}% ===================================================
\beginsection    {\Beweise}
\label        {sec-Beweise}
\todo{Schlussregeln auswählen und Beweise definieren.}
%TODO Schlussregeln auswählen und Beweise definieren.

\section     {Datenstruktur}% ==================================================
\beginsection{Datenstruktur}
\label   {sec-Datenstruktur}
\todo{Datenstruktur abstrakt und in XML definieren.}
%TODO Datenstruktur abstrakt und in XML definieren.

\section     {Bausteine}% ======================================================
\beginsection{Bausteine}
\label   {sec-Bausteine}
\todo{Bausteine? definieren.}
%TODO Bausteine? definieren.

\Endchapter

	%%############################################################################%%
%%                                                                            %%
%% Datei:  ASBA-Anhang.tex                                                    %%
%% Inhalt: Anhang                                                             %%
%%                                                                            %%
%% Copyright (C) 2017  Winfried Teschers                                      %%
%%                                                                            %%
%% This program is free software: you can redistribute it and/or modify       %%
%% it under the terms of the GNU Affero General Public License as published   %%
%% by the Free Software Foundation, either version 3 of the License, or       %%
%% (at your option) any later version.                                        %%
%%                                                                            %%
%% This program is distributed in the hope that it will be useful,            %%
%% but WITHOUT ANY WARRANTY; without even the implied warranty of             %%
%% MERCHANTABILITY or FITNESS FOR A PARTICULAR PURPOSE.  See the              %%
%% GNU Affero General Public License for more details.                        %%
%%                                                                            %%
%% You should have received a copy of the GNU Affero General Public License   %%
%% along with this program.  If not, see <http://www.gnu.org/licenses/>.      %%
%%                                                                            %%
%% Dr. Winfried Teschers                                                      %%
%% Anton-Günther-Straße 26c                                                   %%
%% 91083 Baiersdorf                                                           %%
%% Germany                                                                    %%
%%                                                                            %%
%% e-mail: winfried.teschers@t-online.de                                      %%
%%                                                                            %%
%%############################################################################%%

% !TeX root = ASBA.tex
% !TeX encoding = UTF-8
% !TeX spellcheck = de_DE

\appendix
\renewcommand*{\Chaptername}{\appendixname}

\chapter{Anhang}% ##############################################################
\beginchapter{Anhang}
\label{cha:Anhang}

\section{Werkzeuge}% ===========================================================
\beginsection{Werkzeuge}
\label{sec:Werkzeuge}

Da dies ein Open Source Projekt sein soll,
müssen alle Werkzeuge,
die zum Ablauf der Software erforderlich sind,
ebenfalls Open Source sein.
Für die reine Entwicklung sollte das auch gelten, muss es aber nicht.

\paragraph{Werkzeuge zur Übersetzung der Quelldateien}% --------------------

\begin{enumerate}

	\item\label{Werkzeug:LaTeX}
	Ein Übersetzer für \LaTeX Quellcode (*.tex).
	-- Verwendet wird \emph{MiK\TeX}.

	\item\label{Werkzeug:C++}
	Ein Übersetzer für C++ Quellcode (*.c, *.cpp, *.h, *.hpp).
	-- Verwendet wird \emph{Visual Studio Community 2017}.

	\setcounter{Enumi}{\value{enumi}}% Nummerierung wird fortgesetzt.
\end{enumerate}
Nicht unbedingt nötig, aber sinnvoll:
\begin{enumerate}
	\setcounter{enumi}{\value{Enumi}}% Nummerierung wird fortgesetzt.

	\item\label{Werkzeug:Dokumentation}
	Ein Dokumentationssystem für in C++ Quellcode und darin enthaltene Doxygen Kommentare (*.c, *.cpp, *.h, *.hpp).
	-- Verwendet wird \emph{Doxygen} mit Konfigurationsdatei \strqt{Doxyfile}.

	\item\label{Werkzeug:Konfigurationsmanagement}
	Ein Konfigurationsmanagementsystem zur Verwaltung der Quelldateien.
	-- Verwendet wird \emph{GitHub}.

	\setcounter{Enumi}{\value{enumi}}% Nummerierung wird fortgesetzt.
\end{enumerate}

\paragraph{Werkzeuge für die Entwicklung}% -------------------------------------

\begin{enumerate}
	\setcounter{enumi}{\value{Enumi}}% Nummerierung wird fortgesetzt.

	\item\label{Werkzeug:GitHub}\emph{GitHub}
	als Online Konfigurationsmanagementsystem
	zur Zusammenarbeit verschiedener Entwickler.
	\tourl{https://github.com/}
	-- Lizenz \seename~\cite{bib:GPLii}

	\item\label{Werkzeug:Git}GitHub benötigt
	\emph{Git} als Konfigurationsmanagementsystem.
	\tourl{https://git-scm.com/}
	-- Lizenz \seename~\cite{bib:GPLii}

	\item\label{Werkzeug:MiKTeX}\emph{MiK\TeX}
	für Dokumentation und Ausgaben in \LaTeX.
	\tourl{https://miktex.org/}
	-- Lizenz \seename~\cite{bib:MiKTeX}

	\item\label{Werkzeug:VSC}angedacht: \emph{Visual Studio Community 2017}%
	\footnote{%
		Visual Studio Community ist zwar nicht Open Source,
		darf aber zur Entwicklung von Open Source Software
		unentgeltlich verwendet werden.%
	}
	(\emph{VS}) als Entwicklungsumgebung für C++.
	\tourl{https://www.visualstudio.com/downloads/}
	-- Lizenz \seename~\cite{bib:EULA}

	\item\label{Werkzeug:VSC DB}angedacht: In \emph{Visual Studio Community 2015}
	integrierte Datenbank für \glsIdxPl{Axiom}, \glsIdxPl{Satz}, \glsIdxPl{Beweis},
	\glsIdxPl{Fachbegriff} und \glsIdxPl{Fachgebiet}.
	-- Lizenz \seename~\cite{bib:EULA}

	\item\label{Werkzeug:RapidXml}angedacht: \emph{RapidXml}
	für Ein- und Ausgabe in XML.
	\tourl{http://rapidxml.sourceforge.net/index.htm}
	-- Lizenz \seename{} wahlweise~\cite{bib:BSLi} oder~\cite{bib:MIT}
	\footnote{%
		RapidXml stellt eine C++ Header-Datei zur Verfügung.
		Wenn diese im Quellcode eines Programms enthalten ist,
		gilt das ganze Programm als Open Source.
		Wenn diese Header-Datei nur
		in einer Bibliothek innerhalb eines Projekts verwendet wird,
		so gilt nur diese Bibliothek als Open Source.%
	}

	\item\label{Werkzeug:Doxygen}angedacht: \emph{Doxygen}
	als Dokumentationssystem für C++.
	\tourl{http://www.stack.nl/~dimitri/doxygen/}
	-- Lizenz \seename~\cite{bib:GPLii}

	\item\label{Werkzeug:Ghostscript}angedacht: Doxygen benötigt \emph{Ghostscript}
	als Interpreter für Postscript und PDF.
	\tourl{http://ghostscript.com/}
	-- Lizenz \seename~\cite{bib:AGPL}

	\item\label{Werkzeug:Graphviz}angedacht: Doxygen
	benötigt \emph{Graphviz} mit \emph{Dot}
	zur Erzeugung und Visualisierung von Graphen.
	\tourl{http://www.graphviz.org/Home.php}
	-- Lizenz \seename~\cite{bib:EPL}

	\setcounter{Enumi}{\value{enumi}}% Nummerierung wird fortgesetzt.
\end{enumerate}

\paragraph{Werkzeuge zur Bearbeitung der Quelldateien}% ------------------------

\begin{enumerate}
	\setcounter{enumi}{\value{Enumi}}% Nummerierung wird fortgesetzt.

	\item\label{Werkzeug:TeXstudio}\emph{\TeX studio} als Editor für \LaTeX.
	\tourl{http://www.texstudio.org/}
	-- Lizenz \seename~\cite{bib:GPLii}\\
	\TeX studio benötigt einen Interpreter für Perl:

	\item\label{Werkzeug:Perl}\emph{Strawberry Perl}
	als Interpreter für Perl.
	\tourl{http://strawberryperl.com/}
	-- Lizenz:
	Various OSI-compatible Open Source licenses,
	or given to the public domain

	\item\label{Werkzeug:Notepadpp}\emph{Notepad++} als Text-Editor.
	\tourl{https://notepad-plus-plus.org/}
	-- Lizenz \seename~\cite{bib:GPLi}

	\item\label{Werkzeug:WinMerge}\emph{WinMerge}
	zum Vergleich von Dateien und Verzeichnissen.
	\tourl{http://winmerge.org/}
	-- Lizenz \seename~\cite{bib:GPLi}

	\setcounter{Enumi}{\value{enumi}}% Nummerierung wird fortgesetzt.
\end{enumerate}

%%%	\paragraph{Im Projekt \emph{qedeq} verwendete Werkzeuge}% ----------------------
%%%
%%%	\begin{itemize}
%%%		\setcounter{enumi}{\value{Enumi}}% Nummerierung wird fortgesetzt.
%%%
%%%		\item\label{Werkzeug:Java}\emph{Java}
%%%		als Programmiersprache und Laufzeitumgebung.
%%%		\tourl{https://www.java.com/de/download/win10.jsp}
%%%		-- Lizenz \seename~\cite{bib:JavaSE}
%%%
%%%		\item\label{Werkzeug:Apache Ant}\emph{Apache Ant}
%%%		als Java Bibliothek und Kommandozeilen-Werkzeug
%%%		um Java Programme zu erzeugen.
%%%		\tourl{http://ant.apache.org/}
%%%		-- Lizenz \seename~\cite{bib:Apacheii}
%%%
%%%		\item\label{Werkzeug:Checkstyle}\emph{Checkstyle}
%%%		zur statischen Code-Analyse für Java.
%%%		\tourl{http://checkstyle.sourceforge.net/}
%%%		-- Lizenz \seename~\cite{bib:LGPLii}
%%%
%%%		\item\label{Werkzeug:Clover}\emph{Clover}%
%%%		\footnote{%
%%%			Clover ist proprietäre Software, aber auf Anfrage frei für 30 Tage.
%%%			Danach ist eine einmalige Lizenzgebühr fällig.%
%%%		}
%%%		als Testwerkzeug zur Analyse der Code-Abdeckung.
%%%		\tourl{https://www.atlassian.com/software/clover/}
%%%		-- Lizenz \seename~\cite{bib:Clover}
%%%
%%%		\item\label{Werkzeug:Eclipse Java}\emph{Eclipse IDE for Java Developers}
%%%		als Entwicklungsumgebung für Java.
%%%		\tourl{http://www.eclipse.org/downloads/packages/eclipse-ide-java-developers/neon1a/}
%%%		-- Lizenz \seename~\cite{bib:OSI}
%%%
%%%		\item\label{Werkzeug:JUnit}\emph{JUnit}
%%%		zur Erzeugung von wiederholbaren Tests.
%%%		\tourl{http://junit.org/junit4/}
%%%		-- Lizenz \seename~\cite{bib:EPL}
%%%
%%%		\item\label{Werkzeug:Xerces2}\emph{Xerces2} als XML-Parser in Java.
%%%		\tourl{http://xerces.apache.org/xerces2-j/}
%%%		-- Lizenzen \seename~\cite{bib:Apacheii, bib:SAX, bib:WDCDL, bib:WDCSNL}
%%%
%%%		\setcounter{Enumi}{\value{enumi}}% Nummerierung wird fortgesetzt.
%%%	\end{itemize}

\section{Offene Aufgaben}% =====================================================
\beginsection{Offene Aufgaben}
\label{sec:Offene Aufgaben}

\begin{enumerate}
	\item TODOs bearbeiten
	\item Eingabeprogramm erstellen (liest XML)
	\item Prüfprogramm erstellen
	\item Ausgabeprogramm erstellen (schreibt XML)
	\item Formelausgabe erstellen (erzeugt \LaTeX{} aus XML)
	\item \glsIdxPl{Axiom} sammeln und eingeben
	\item \glsIdxPl{Satz} sammeln und eingeben
	\item \glsIdxPl{Beweis} sammeln und eingeben
	\item \glsIdxPl{Fachbegriff} und Symbole sammeln und eingeben
	\item \glsIdxPl{Fachgebiet} sammeln und eingeben
	\item \glsIdxPl{Ausgabeschema} sammeln und eingeben
\end{enumerate}

\Endchapter

	%%############################################################################%%
%%                                                                            %%
%% Datei:  ASBA-Ideen.tex                                                     %%
%% Inhalt: Kapitel "Ideen" --- Nur vorübergehend ---                          %%
%%                                                                            %%
%% Copyright (C) 2017  Winfried Teschers                                      %%
%%                                                                            %%
%% This program is free software: you can redistribute it and/or modify       %%
%% it under the terms of the GNU Affero General Public License as published   %%
%% by the Free Software Foundation, either version 3 of the License, or       %%
%% (at your option) any later version.                                        %%
%%                                                                            %%
%% This program is distributed in the hope that it will be useful,            %%
%% but WITHOUT ANY WARRANTY; without even the implied warranty of             %%
%% MERCHANTABILITY or FITNESS FOR A PARTICULAR PURPOSE.  See the              %%
%% GNU Affero General Public License for more details.                        %%
%%                                                                            %%
%% You should have received a copy of the GNU Affero General Public License   %%
%% along with this program.  If not, see <http://www.gnu.org/licenses/>.      %%
%%                                                                            %%
%% Dr. Winfried Teschers                                                      %%
%% Anton-Günther-Straße 26c                                                   %%
%% 91083 Baiersdorf                                                           %%
%% Germany                                                                    %%
%%                                                                            %%
%% e-mail: winfried.teschers@t-online.de                                      %%
%%                                                                            %%
%%############################################################################%%

% !TeX root = ASBA.tex
% !TeX encoding = UTF-8
% !TeX spellcheck = de_DE

\chapter{Ideen}% ###############################################################
\beginchapter{Ideen}
\label{cha-Ideen}

\section{Schlussregeln}% =======================================================
\beginsection{Schlussregeln}
\label{sec-Schlussregeln}

In diesem \sectionname\ geht es um \zulaessige\ \Transformationen, \textdh\ \allgemeingueltige\ \Schlussregeln.
Dazu gehören zunächst die \Basisregeln.
Dann aber auch alle aus den \Basisregeln\ und den bis dahin \allgemeingueltigen\ \Schlussregeln\ korrekt abgeleiteten neuen \Schlussregeln.
Die \Schlussregeln\ haben die Form eines Formalen \Satzes.

\subsection{Basisregeln}% ------------------------------------------------------
\label{sub-Basisregeln}

Gemäß \cite{bib:Rautenberg} Kapitel~1.4 \emph{Ein vollständiger aussagenlogischer Kalkül} werden sechs \Basisregeln\ definiert. Zuvor werden aber noch einige \Definitionen\ gebraucht. Dazu seien $n$, $m$, $k$ und $l$ natürliche Zahlen (auch~0), $\alpha$, $\alpha_i$, $\beta$ und $\beta_j$ \Formeln\, $X$, $X_i$, $Y$ und $Y_j$ Mengen von \Formeln\ und
\begin{align}
	%
	&X&&\defeq&&X_1\cup X_2\cup...\cup X_n\cup\{\alpha_1,\alpha_2,...,\alpha_m\}
	\\
	&Y&&\defeq&&Y_1\cup Y_2\cup...\cup Y_k\cup\{\beta_1, \beta_2, ...,\beta_l \}
	\formulatoleft\formulatoleft
\end{align}

$X$ und $Y$ können auch die leere Menge sein. Damit wird definiert:
\begin{align}
	& \alpha \definition{\symderive} \beta \quad \symmetadefeq \quad
	\parbox[t]{11cm}{%
	$\beta$ ist mittels schrittweiser Anwendung \emph{\zulaessiger\ \Transformationen} (siehe weiter unten) aus $\alpha$ \ableitbar.
	Sprechweise: Aus $\alpha$ ist $\beta$ \definition{\ableitbar} oder \definition{\beweisbar};
	kurz: \enquote{$\alpha$ \emph{\ableitbar} $\beta$} \textbzw\ \enquote{$\alpha$ \emph{\beweisbar} $\beta$}
	-- Es kann auch \chrqt{$\alpha$} durch \chrqt{$X$} und/oder \chrqt{$\beta$} durch \chrqt{$Y$} ersetzt werden.
	}
	\label{def-ableitbar}
	\\
	& \definition{\symderive} \beta \quad \symmetadefeq \quad \emptyset \symderive \beta \qquad \text{(\chrqt{\symderive} kann dann auch ganz entfallen)}
	\\
	& X_1,X_2,...,X_n,\alpha_1,\alpha_2,...,\alpha_m\quad
	\definition{\symderive}\quad Y_1,Y_2,...,Y_n, \beta_1, \beta_2,..., \beta_m\quad
	\symmetadefeq \quad X \symderive Y
	\label{def-ableitbarKurz}
	\formulatoleft
\end{align}

Eine \definition{\zulaessige} \definition{\Transformation} ist die Anwendung einer \emph{\Substitution}{\vrefnotesub{sub-Identitaetsregeln} (siehe unten), einer \emph{\Basisregel} (siehe unten) oder einer davon abgeleiteten sonstigen \emph{\Schlussregel}, \textzB\ aus \vrefsub{sub-Schlussregeln}.
Bei den \Schlussregeln\ und der \Substitution\ \chrqt{\symsubst} soll das Komma stärker binden als \chrqt{\symderive}, \chrqt{\symsubst} und \chrqt{\symsrand}, wobei \chrqt{\symsrand} für \enquote{und} \textbzw\ \chrqt{\symmetaand}\vrefnotesub{sub-AussagenUndMetaoperationen} steht und schwächer bindet als \chrqt{\symderive} und \chrqt{\symsubst}.%
\footnote{siehe Fußnote~3 \vrefvontab{tab-Prioritaeten}}

Zur der Auswahl der \Basisregeln, der Formulierung und der Bezeichnungen wird auf~\cite{bib:Rautenberg,bib:NatuerlichesSchliessen} zurückgegriffen.
Wie in~\cite{bib:NatuerlichesSchliessen} steht \chrqt{$\mathrm{E}$} für \enquote{-Einführung} und \chrqt{$\mathrm{B}$} für \enquote{-Beseitigung} (oder \enquote{-Elimination}) von \Junktoren.%
\footnote{%
	In der \Monotonieregel\ wird hier, anders als in~\cite{bib:Rautenberg}, \seqqt{$X,Y$} statt \seqqt{$ Y \text{ , für } Y \symsupseteq X $} genommen. Das ist gleichwertig, vermeidet aber den Zusatz \seqqt{$ \text{ , für } Y \symsupseteq X $}.
	Außerdem werden bei den Bezeichnungen \seqqt{$(\symland 1)$} und \seqqt{$(\symland 2)$} gemäß~\cite{bib:NatuerlichesSchliessen} durch \seqqt{$(\andE)$} \textbzw\ \seqqt{$(\andB)$} ersetzt.
}

Im Folgenden seien $\alpha$ und $\beta$ \Formeln\ und $X$ und $Y$ Mengen von \Formeln.
Für die sechs \Basisregeln\ werden dann nur noch die \Junktoren \chrqt{\symlnot} und \chrqt{\symland} benötigt.
Bei den weiteren \Schlussregeln\ wird noch \chrqt{\symlimp} gemäß der Definition~\vref{def-imp} verwendet.

\begin{align}
	& \frac{}{\alpha\symderive\alpha}
	& & (\text{\definition{\Anfangsregel}})
	\tag{\tagAR} \label{def-AR}
	\\\\
	& \frac{X\symderive\alpha}{X,Y\symderive\alpha}
	& & (\text{\definition{\Monotonieregel}})
	\tag{\tagMR} \label{def-MR}
	\\\\
	& \frac{X\symderive\alpha,\lnot\alpha}{X\symderive\beta}
	& & (\text{Einführung/Beseitigung der Negation Teil 1})
	\tag{\tagnota} \label{def-nota}
	\\\\
	& \frac{X,\alpha\symderive\beta \srand X,\symlnot\alpha\symderive\beta}{X\symderive\beta}
	& & (\text{Einführung/Beseitigung der Negation Teil 2})
	\tag{\tagnotb} \label{def-notb}
	\\\\
	& \frac{X\symderive\alpha,\beta}{X\symderive\alpha\symland\beta}
	& & (\text{Einführung der Konjunktion})
	\tag{\tagandE} \label{def-andE}
	\\\\
	& \frac{X\symderive\alpha\land\beta}{X\symderive\alpha,\beta}
	& & (\text{Beseitigung der Konjunktion})
	\tag{\tagandB} \label{def-andB}
	\formulatoleft
\end{align}

In einer \Schlussregel\ werden die \Formeln%
\footnote{hier: \Aussagen\ in einer formalen Form.}
über dem Querstrich als \definition{\Voraussetzungen} und die unter dem Querstrich als \definition{\Folgerungen} der Regel bezeichnet.
Eine \Schlussregel\ steht für die \Aussage, dass mit ihren \Voraussetzungen\ auch auch ihre \Folgerungen\ gelten.
-- Im Gegensatz zu den weiteren \Schlussregeln\ werden die oben aufgelisteten \Basisregeln\ nicht weiter hinterfragt, \textdh\ sie gelten quasi als \Axiome.

\subsection{Identitätsregeln}% --------------------------------------------------------
\label{sub-Identitaetsregeln}

%TODO Durch Substitution ersetzen?
Die \zulaessigen\ \Transformationen, \textdh\ die Anwendung der \Schlussregeln, erfordern \zulaessige\ \Substitutionen.
Damit wird dem Gleichheits- oder Identitätszeichen \chrqt{\symeq} mittels Einführungs- und Beseitigungsregel eine Bedeutung verliehen.%
\footnote{\citesee{bib:NatuerlichesSchliessen}}
Dazu seien $\alpha$, $\beta$ und $\gamma$ \vergleichbare\footnote{siehe Ende \vrefvonsub{sub-AussagenUndMetaoperationen}}\Formeln.

Zunächst wird definiert:
\begin{align}
	\gamma(\alpha \definition{\symsubst} \beta) \quad \defeq \quad
	\parbox[t]{11cm}{%
		Die \Formel, die man erhält, wenn in $\gamma$ alle oder nur einige Vorkommen von $\alpha$ durch $\beta$ ersetzt werden.
		-- Gegebenenfalls muss noch die Auswahl der Ersetzungen angegeben werden, andernfalls werden alle Vorkommen ersetzt.
		Letzteres heißt dann \defn{vollständige} \Substitution.
	} \label{def-SubstitutionAlt}\\
	\gamma(\alpha \definition{\symswap} \beta) \quad \defeq \quad
	\parbox[t]{11cm}{%
		Die \Formel, die man erhält, die man erhält, wenn in $\gamma$ alle $\alpha$ und $\beta$ miteinander vertauscht werden.
		Dazu ist es nötig, das $\alpha$ und $\beta$ voneinander unabhängig sind, vorzugsweise zwei verschiedene Variable.
	} \label{def-Vertauschung}
\end{align}

\seqqt{$ \alpha \symsubst \beta $} heißt \definition{\Substitution} und \seqqt{$ \alpha \symswap \beta $} \defn{\Vertauschung} oder kurz \defn{Tausch}.
-- Sei noch $S = (s_1, s_2, ...)$ eine endliche Folge aus \Substitutionen, die auch \Vertauschungen\ enthalten und auch leer sein kann.

Dann wird definiert:
\begin{align}
	\gamma(S) & \quad \defeq \quad \gamma(s_1)(s_2)... \label{def-SubstitutionenAlt}\\
	\gamma(\emptyset) & \quad \; = \quad \gamma & \text{(nur zur Verdeutlichung)}\\
	\gamma(s_1,s_2,...) & \quad \defeq \quad \gamma(S)
\end{align}

Die \Vertauschung\ ist eine spezielle Form der \Substitution.
Wenn $x$ und $y$ zwei verschiedene Variable, die in $\alpha$, $\beta$ und $\gamma$ nicht vorkommen, gilt:
\[
	\gamma(\alpha \swap \beta) = \gamma(\alpha\subst x, \beta\subst y,  y\subst\alpha, x \subst\beta)
\]

Sei zusätzlich noch $s$ eine \Substitution.
Folgende Sprechweisen werden verwendet:
\begin{itemize}
	\renewcommand*{\itemindent}{1,5cm}
	\renewcommand*{\labelsep}{5pt}
	\item [$\gamma(\alpha \subst \beta)$ :] In $\gamma$ wird $\alpha$ (\defn{vollständig}) \defn{durch $\beta$ substituiert}.
	\item [$\gamma(\alpha \swap \beta)$ :] In $\gamma$ werden $\alpha$ und $\beta$ \defn{vertauscht}.
	\item [$\gamma(s)$ :] $s$ wird auf $\gamma$ \defn{angewendet}.
	\item [$\gamma(S)$ :] Die \Substitutionen\ aus S werden in der angegebenen Reihenfolge auf $\gamma$ angewendet.
	\item [$\gamma(S)$ :] $S$ wird auf $\gamma$ angewendet.
\end{itemize}
%
Bei obiger Definition der \Substitution\ bleibt noch offen, unter welchen \Voraussetzungen\ sie angewendet werden darf. Das soll hier nicht untersucht werden. In diesem \sectionname\ genügt es, das nur \Vertauschung\ und vollständige \Substitution\ verwendet werden.
In diesen Fällen sind beliebige \Substitutionen\ von Variablen durch \Formeln\ erlaubt.

Ist $\gamma$ wie oben und $S$ eine Menge aus \Substitutionen.

Nun können die beiden \Identitaetsregeln\ definiert werden:
\begin{align}
	& \frac{}{\alpha\symeq\alpha}
	& & (\text{Einführung der Identität})
	\tag{\tageqE} \label{def-eqE}
	\\\\
	& \frac{\alpha\symeq\beta \srand \gamma}{\gamma(\alpha\subst\beta)}
	& & (\text{Beseitigung der Identität})
	\tag{\tageqB} \label{def-eqB}
	\formulatoleft
\end{align}

Die \Identitaetsregeln\ werden hier eingeführt, um die \Substitution\ zu rechtfertigen.
Wie die \Basisregeln\ gelten sie als \Axiome, würden also eigentlich dazu gehören.
Da sie aber nicht weiter verwendet werden, werden sie hier nicht zu den \Basisregeln\ gezählt.

\subsection{Weitere Schlussregeln}% --------------------------------------------
\label{sub-weitereSchlussregeln}

In~\cite{bib:Rautenberg} werden aus den \Basisregeln\ mittels \zulaessiger\ \Transformationen\ weitere \Schlussregeln\ abgeleitet.%
%TODO Identitätsregeln kommen bei Rautenberg später vor. ???
\footnote{%
	In~\cite{bib:Rautenberg} werden die \Identitaetsregeln\ zwar weder aufgeführt noch angewandt, ohne \Substitution\ geht es aber nicht.
}
Man vergleiche auch mit~\cite{bib:NatuerlichesSchliessen}.

\begin{align}
	& \frac{X,\symlnot\alpha\symderive\alpha}{X\symderive\alpha}
	& & (\text{Beseitigung der Negation; Indirekter \Beweis})
	\tag{\tagnotc} \label{def-notc}
	\\\\
	& \frac{X,\symlnot\alpha\symderive\beta,\symlnot\beta}{X\symderive\alpha}
	& & (\text{Reductio ad absurdum})
	\tag{\tagnotd} \label{def-notd}
	\\\\
	& \frac{X,\alpha\symderive\beta}{X\symderive\alpha\symlimp\beta}
	& & (\text{Einführung der Implikation})
	\tag{\tagimpE} \label{def-impE}
	\\\\
	& \frac{X\symderive\alpha\symlimp\beta}{X,\alpha\symderive\beta}
	& & (\text{Beseitigung der Implikation})
	\tag{\tagimpB} \label{def-impB}
	\\\\
	& \frac{X\symderive\alpha \srand X,\alpha\symderive\beta}{X\symderive\beta}
	& & (\text{\definition{\Schnittregel}})
	\tag{\tagSR} \label{def-SR}
	\\\\
	& \frac{X\symderive\alpha \srand \alpha\symlimp\beta}{X\symderive\beta}
	& & (\text{\definition{\Abtrennungsregel} -- \emph{Modus ponens}})
	\tag{\tagTR} \label{def-TR}
	\formulatoleft
\end{align}

Dabei werden zum \Beweis\ der \Schlussregeln\ in~\cite{bib:Rautenberg} folgende \Basisregeln\ verwendet:
\begin{itemize}
	\renewcommand*{\itemindent}{3cm}
	\renewcommand*{\labelsep}{5pt}
	\item[\Schlussregel\ ~:] verwendete \Basisregeln
	\item[\ref{def-notc} ~:] \ref{def-AR}, \ref{def-MR}, \ref{def-notb}
	\item[\ref{def-notd} ~:] \ref{def-AR}, \ref{def-MR}, \ref{def-nota}, \ref{def-notb}
	\item[\ref{def-impE} ~:] \ref{def-AR}, \ref{def-MR}, \ref{def-nota}, \ref{def-notb}, \ref{def-andE}
	\item[\ref{def-impB} ~:] \ref{def-AR}, \ref{def-MR}, \ref{def-nota}, \ref{def-notb}, \ref{def-andB}
	\item[\ref{def-SR}   ~:] \ref{def-AR}, \ref{def-MR}, \ref{def-nota}, \ref{def-notb}
	\item[\ref{def-TR}   ~:] \ref{def-AR}, \ref{def-MR}, \ref{def-nota}, \ref{def-notb}, \ref{def-andE}
\end{itemize}
%
\subsection{Beispiel einer Ableitung}% -----------------------------------------
\label{sub-BeispielAbleitung}

Als Beispiel wird hier die \Schnittregel\ aus den \Basisregeln\ abgeleitet.%
\footnote{%
	Die Form der Tabelle ist angelehnt an~\cite{bib:NatuerlichesSchliessen} Kapitel~2.2.4 \emph{Eine Beispielableitung}.
}
Dazu wird verabredet, dass \vrefintab{tab-AbleitungSchnittregel} der Inhalt der Zelle in der Zeile $i$ und der Spalte $(X_n)$ mit $X_i$ bezeichnet wird.
Zur kürzeren Darstellung wird statt auf die vollständigen Spaltenüberschriften nur auf die dort notierten $(X_n)$ verwiesen. Dass in der Spalte $(n)$ stets die Zeilennummer steht, wird im folgenden nicht mehr extra erwähnt.

Für die ausgefüllten Felder wird nun definiert:%
\footnote{%
	Eigentlich müsste man für jede \Substitution\ aus $S_i$ eine eigene Zeile vorsehen.
	Um die Tabellen für die \Beweise\ kürzer zu halten, werden aufeinanderfolgende \Substitutionen\ zusammengefasst.
}
\begin{align}
	R_i & \defeq
	\left\{
		\begin{array}{l}
			\text{- \enquote{\Voraussetzung} = Die \Aussage\ $A_i$ ist eine \Voraussetzung.}\\
			\text{- \enquote{\Folgerung} = Die \Aussage\ $A_i$ ist eine \Folgerung.}\\
			\text{- \enquote{Annahme} = Die \Aussage\ $A_i$ wird vorübergehend als zutreffend angenommen.}\\
			\text{- $j$ = Verweis auf die \Schlussregel\ $\overline{R}_j$ für ein $j < i$.}\\
			\text{- Verweis (ohne Klammern) auf eine \allgemeingueltige\ \Schlussregel.}
		\end{array}
	\right.
	\\
	S_i & \defeq \text{Die Folge aus den anzuwendenden \Substitutionen.}
	\\
	\overline{R}_i & \defeq \text{Das Ergebnis der in der angegebenen Reihenfolge angewendeten}\\
	& \quad\;\; \text{\Substitutionen aus $S_i$ auf die \Schlussregel\ $R_i$}
	\\
	Z_i & \defeq \text{Die Indizes $j$ (mit $j < i$) als Verweise auf eine oder mehrere \Aussagen\ $A_j$,}\\
	& \quad\;\;\text{ welche zusammen genau die \Voraussetzungen\ der \Schnittregel\ } \overline{R}_i \text{ erfüllen.}
	\\
	A_i & \defeq \text{\Folgerung(en) der \Schlussregel\ $\overline{R}_i$ --}\\
	& \quad\;\; \text{auch in Form der Indizes von einem oder mehreren von $Aj$ (mit $j < i$).}\\
	& \quad\;\; \text{In der Ergebniszeile kann hier auch die bewiesene \Aussage\ als \Schlussregel\ stehen.}
	\\
	D_i & \defeq \text{die Indizes der $A_j$, von denen $A_i$ abhängig ist.}
\end{align}

Bis zur Zeile $i$ hat man die folgende \Schlussregel\ bewiesen:
\[ \frac{A_{i_1} \srand A_{i_2} ...}{A_i} \quad \text{, für alle } i_j \in D_i \]
Sei nun
\[
	\Gamma_i \defeq
	\left\{
		\begin{array}{ll}
			\text{leer}    & \text{ für } R_i = \text{\enquote{\Voraussetzung}} \\
			\text{leer}    & \text{ für } R_i = \text{\enquote{\Folgerung}}     \\
			\text{leer}    & \text{ für } R_i = \text{\enquote{Annahme}}        \\
			\overline{R_j} & \text{ für } R_i = j \quad \text{(eine \defn{interne} \Schlussregel)} \\
			\text{die \Schlussregel} & \text{ für } R_i = \text{Verweis auf eine \defn{externe} \Schlussregel}
		\end{array}
	\right.
\]
Damit gilt für die Einträge in einer Zeile $i$:
\begin{itemize}
	\item Wenn $\Gamma_i$ nicht leer ist, ist $R_i$ eine \Schlussregel\ mit $R_i = \Gamma_i(S_i)$%
	\footnote{%
	    %TODO Makro für Verweis benutzen
		siehe Definition~\eqref{def-SubstitutionenAlt} \vrefvonsub{sub-Identitaetsregeln}
	}.
	\item Wenn $A_i$ nicht leer ist, ist $R_i = \dfrac{A_{z_1} \srand A_{z_2} \srand ...}{A_i}$ (alle $z_j \in Z_i$).
	\item Wenn $A_i$ nicht leer ist, ist bis jetzt die \Schlussregel\ $\dfrac{A_{d_1} \srand A_{d_2} \srand ...}{A_i}$ (alle $d_j \in D_i$) schon bewiesen.
\end{itemize}
$S_i$, $Z_i$ und $D_i$ dürfen dabei auch leer sein.

\begin{table}[!htb]
	\setlength\tabcolsep{1pt}
	\setlength\extrarowheight{7pt}
	\newcommand*{\centerParbox}[2]{\parbox{#1}{\centering #2}}
	\newcommand*{\titleCell}[3]{\centerParbox{#1}{\textbf{#2} (#3)}}
	\newcommand*{\SnCell}[1]{\centerParbox{1.85cm}{#1}}
	\newcommand*{\DnCell}[1]{\centerParbox{1.95cm}{#1}}
	\begin{tabular}{|c||c|c|c|c|c|c|}
		\hline
		\titleCell{0.95cm}{Zeile}                       {$n$} &
		\titleCell{1.05cm}{Regel}                     {$R_n$} &
		\titleCell{1.85cm}{Substitu"=tionen}          {$S_n$} &
		\titleCell{1.80cm}{erzeugte Regel} {$\overline{R}_n$} &
		\titleCell{2.15cm}{angewendet auf ...}        {$Z_n$} &
		\titleCell{1.65cm}{\Aussage}          {$A_n$} &
		\titleCell{1.95cm}{Abhängig"=keiten}          {$D_n$}
		\\\hline\hline
		1 & \centerParbox{1.35cm}{Voraus"=setzung} & & & & $X \symderive \alpha$ & 1
		\\\hline
		2 & \centerParbox{1.35cm}{Voraus"=setzung} & & & & $X,\alpha \symderive \beta$ & 2
		\\\hline
		3 & \centerParbox{1.00cm}{Folge"=rung} & & & & $X \symderive \beta$ & 3
		\\\hline
		4 & \ref{def-MR} & & $\dfrac{X \symderive \alpha}{X, Y \symderive \alpha}$ & & &
		\\\hline
		5 & 4 & $Y \subst \symlnot\alpha$ & $\dfrac{X \symderive \alpha}{X, \symlnot\alpha \symderive \alpha}$ & 1 & $X, \symlnot\alpha \symderive \alpha$ & 1
		\\\hline
		6 & \ref{def-AR} & & $ \dfrac{}{\alpha \symderive \alpha} $ & & &
		\\\hline
		7 & 6 & $\alpha \subst \symlnot\alpha$ & $\dfrac{}{\symlnot\alpha \symderive \symlnot\alpha}$ & & $\symlnot\alpha \symderive \symlnot\alpha$ &
		\\\hline
		8 & 4 & \SnCell{%
			$\alpha \subst \symlnot\alpha$\\
			$X \subst \symlnot\alpha$\\
			$Y \subst X$
		} & $\dfrac{\symlnot\alpha \symderive \symlnot\alpha}{X,\symlnot\alpha \symderive \symlnot\alpha}$ & 7 & $X,\symlnot\alpha \symderive \symlnot\alpha$ &
		\\\hline
		9 & \ref{def-nota} & & $\dfrac{X \symderive \alpha, \symlnot\alpha}{X \symderive \beta}$ & & &
		\\\hline
		10 & 9 & $X \subst X, \symlnot\alpha$ & $\dfrac{X,\symlnot\alpha \symderive \alpha, \symlnot\alpha}{X,\symlnot\alpha \symderive \beta}$ & 5, 8 & $X,\symlnot\alpha \symderive \beta$ & 1
		\\\hline
		11 & \ref{def-notb} & & $\dfrac{X,\alpha \symderive \beta \srand X,\symlnot\alpha \symderive \beta}{X \symderive \beta}$ & 2, 10 & 3 & 1, 2
		\\\hline\hline
		12 & \centerParbox{1.4cm}{\ref{def-AR}, \ref{def-MR}, \ref{def-nota}, \ref{def-notb}} & & $\dfrac{A_1 \srand A_2}{A_3}$ & & $\dfrac{X \symderive \alpha \srand X, \alpha \symderive \beta}{X \symderive \beta}$ &
		\\\hline
	\end{tabular}
	\caption{\Ableitung\ der \Schnittregel\ aus den \Basisregeln}
	\label{tab-AbleitungSchnittregel}
\end{table}

Die Erzeugung einer Tabelle analog zu~\vref{tab-AbleitungSchnittregel} wird im folgenden beschrieben.
Zellen, für die kein Inhalt angegeben wird, bleiben leer.
Rückwärts-Referenzen auf schon ausgefüllte Zellinhalte sind jederzeit möglich.
Das Eintragen der Zeilennummer $i$ wird nicht extra erwähnt.
-- Die Tabelle und die Beschreibung sind so ausführlich, damit man daraus leicht ein Computerprogramm erstellen kann.
%
\begin{enumerate}
	%
	\item Am Anfang der Tabelle werden zuerst \Voraussetzungen, dann zu beweisende \Folgerungen\ und schließlich Annahmen aufgeführt.%
	\footnote{%
		Die Angabe ist dann erforderlich, wenn darauf verwiesen wird.
		Durch die Auflistung hat man aber einen vollständigen Überblick über die \Voraussetzungen\ und \Folgerungen\ eines \Beweises\ und die Zwischenannahmen.
		Auf jede nötige \Voraussetzung\ und jede verwendete Zwischenannahme wird in der Spalte $(Z_n$) mindestens einmal verwiesen, so dass sie auch aufgeführt werden müssen.
		Die Angabe der \Folgerungen\ erleichtert die Erstellung einer \emph{Ergebniszeile} (\seename Punkt~\ref{item-Ergebniszeile}).
	}
	Jede der drei Gruppen kann auch leer sein und es ist auch möglich, die Zeilen an anderen Stellen der Tabelle anzugeben, spätestens aber, wenn darauf verwiesen wird.
	Für jede \Voraussetzung, \Folgerung\ und Annahme gibt es eine Zeile:
	\begin{enumerate}
		\item $R_i =$ \enquote{\Voraussetzung}, \enquote{\Folgerung} oder \enquote{Annahme}.
		\item $A_i =$ Die aktuelle \Voraussetzung, \Folgerung\ oder Annahme.
		\item $D_i =$ $i$ \quad (ein Verweis auf $A_i$).
	\end{enumerate}
	%
	\item In den nächsten Zeilen werden die \Beweisschritte\ aufgeführt, für jeden Schritt eine Zeile.

	Zunächst kann $R_i$ kann auf zwei Arten erzeugt werden:
	\begin{enumerate}
		\setcounter{enumii}{\value{Enumii}}% Nummerierung wird fortgesetzt.
		\item
		\begin{enumerate}
			\item $R_i =$ Verweis auf eine \allgemeingueltige\ \Schlussregel.
			\item $\overline{R}_i =$ Die \Schlussregel, auf die verwiesen wird.
		\end{enumerate}
		\setcounter{Enumii}{\value{enumii}}% Nummerierung wird fortgesetzt.
	\end{enumerate}
	oder
	\begin{enumerate}
		\item
		\begin{enumerate}
			\item $R_i = j$, wenn die schon bewiesene \Schlussregel\ $\overline{R}_j$ (mit $j < i$) angewendet werden soll.
			\item $S_i =$ Die auf die \Schlussregel\ $R_i$ anzuwendende \Substitution.
			\item $\overline{R}_i =$ Das Ergebnis der \Substitution\ $S_i$ auf die \Schlussregel\ $R_i$.
		\end{enumerate}
		\setcounter{Enumii}{\value{enumii}}% Nummerierung wird fortgesetzt.
	\end{enumerate}
	Man beachte, dass die \Schlussregel\ $\overline{R}_i$, stets \allgemeingueltig\ ist, da sie ausschließlich aus \allgemeingueltigen\ \Schlussregeln\ mittels \Substitutionen\ abgeleitet worden ist.
	Daher gibt es auch keine Beschränkung weiterer \Substitutionen\ durch irgendwelche Abhängigkeiten.

	Nun kann die Zeile beendet werden, oder es geht weiter mit:
	\begin{enumerate}
		\setcounter{enumii}{\value{Enumii}}% Nummerierung wird fortgesetzt.
		\item \label{item-Anwendung} $Z_n =$ Die Indizes aller $A_j$ (mit $j < i$), die eine \Voraussetzung\ der \Schlussregel\ $\overline{R}_i$ sind, möglichst in der verwendeten Reihenfolge.
		-- Für jedes angegebene $j$ werden noch die Abhängigkeiten $D_j$ den Abhängigkeiten $D_i$ hinzugefügt.
		%
		\item $A_i =$ \Folgerung(en) der \Schlussregel\ $\overline{R}_i$.
		-- Wenn diese \Folgerungen\ schon als \Aussagen\ $A_j$ (mit $j < i$) vorhanden sind, können auch einfach deren Indizes eingetragen werden.
		Damit werden die Zusammenhänge und der Abschluss des \Beweises\ besser ersichtlich.
		%
		\item $D_i =$ Die Verweise wurden schon in (\ref{item-Anwendung}) eingetragen.%
		\footnote{Wenn $D_n$ leer ist, dann ist $A_n$ allgemeingültig.}
		%
	\end{enumerate}
	Der \Beweis\ muss so lange fortgeführt werden, bis alle \Folgerungen\ als \Aussagen\ in der Spalte $(A_n)$ erschienen und dort jeweils nur von den gegebenen \Voraussetzungen\ abhängig sind.
	%
	\item \label{item-Ergebniszeile} In einer \defn{Ergebniszeile}, die dann die letzte ist, kann noch die bewiesene Behauptung in Form einer \Schlussregel\ formuliert und in einer passenden Spalte notiert werden.
	Zusätzlich können dort auch noch alle verwendeten \Schlussregeln\ gesammelt werden.
	Dies kann \textzB\ folgendermaßen geschehen:
	\begin{enumerate}
		%
		\item $(R_n) =$ Verweise auf alle verwendeten externen \Schlussregeln.
		%
		\item $(\overline{R}_n) =$ Die bewiesene Behauptung als \Schlussregeln, wobei alle $A_i$, die \Voraussetzungen\ sind, als \Voraussetzung\ und alle $A_j$, die \Folgerungen\ sind, als \Folgerung\ eingesetzt werden, jeweils in der Form \enquote{$A_i$} \textbzgl\ \enquote{$A_j$}.
		Das ergibt dann:
		\[ \frac{A_{i_1} \srand A_{i_2} \srand ...}{A_{j_1} \srand A_{j_2} \srand ...} \]
		%
		\item $(A_n) =$ $\overline{R}_i$, wobei die \Voraussetzungen\ und \Folgerungen\ aufgelöst werden.
		%
		\item $(D_n) =$ Die Vereinigung aller Abhängigkeiten der \Folgerungen\, vermindert um die \Voraussetzungen.
		-- Wenn das Feld dabei nicht leer bleibt, ist der \Beweis\ missglückt!
		%
	\end{enumerate}
	%
\end{enumerate}
%
Ein weiteres Beispiel \vrefintab{tab-AbleitungKontraposition} soll verdeutlichen, wie Abhängigkeiten von Zwischenannahmen wieder beseitigt werden können.%
\footnote{\citesee{bib:NatuerlichesSchliessen}, Kapitel 2.2.4 \emph{Eine Beispielableitung}}

\begin{table}[!htb]
	\setlength\tabcolsep{1pt}
	\setlength\extrarowheight{7pt}
	\newcommand*{\centerParbox}[2]{\parbox{#1}{\centering #2}}
	\newcommand*{\titleCell}[3]{\centerParbox{#1}{\textbf{#2} (#3)}}
	\newcommand*{\SnCell}[1]{\centerParbox{2.30cm}{#1}}
	\newcommand*{\DnCell}[1]{\centerParbox{1.95cm}{#1}}
	\begin{tabular}{|c||c|c|c|c|c|c|}
		\hline
		\titleCell{0.95cm}{Zeile}                       {$n$} &
		\titleCell{1.05cm}{Regel}                     {$R_n$} &
		\titleCell{1.85cm}{Substitu"=tionen}          {$S_n$} &
		\titleCell{1.80cm}{erzeugte Regel} {$\overline{R}_n$} &
		\titleCell{2.15cm}{angewendet auf ...}        {$Z_n$} &
		\titleCell{1.65cm}{\Aussage}          {$A_n$} &
		\titleCell{1.95cm}{Abhängig"=keiten}          {$D_n$}
		\\\hline \hline
		1 & \centerParbox{1.00cm}{Folge"=rung} & & & & $(\alpha\symlimp\beta)\symlimp(\symlnot\beta\symlimp\symlnot\alpha)$ & 1
		\\\hline
		2 & \centerParbox{1.20cm}{An"=nahme} & & & & $\alpha\symlimp\beta$ & 2
		\\\hline
		3 & \centerParbox{1.20cm}{An"=nahme} & & & & $\symlnot\beta$ & 3
		\\\hline
		4 & \centerParbox{1.20cm}{An"=nahme} & & & & $\alpha$ & 4
		\\\hline
		5 & \impB & & $\dfrac{X \symderive \alpha\symlimp\beta}{X,\alpha \symderive \beta}$ & & &
		\\\hline
		6 & -1 & $X \subst \emptyset$ & $\dfrac{\alpha\symlimp\beta}{\alpha \symderive \beta}$ & 2 & $\alpha \symderive \beta $ & 2
		\\\hline
		7 & \SR & & $\dfrac{X \symderive \alpha \srand X,\alpha \symderive \beta}{X \symderive \beta}$ & & &
		\\\hline
		8 & -1 & $X \subst \emptyset$ & $\dfrac{\alpha \srand \alpha \symderive \beta}{\beta}$ & 4, 6 & $\beta $ & 4, 6
		\\\hline
		9' & \ref{def-andE} & & $\dfrac{X \symderive \alpha, \beta}{X \symderive \alpha \symland \beta}$ & & &
		\\\hline
		10' & -1 & $X \subst \emptyset$ & $\dfrac{\alpha \srand \beta}{\alpha \symland \beta}$ & & &
		\\\hline
		11' & -1 &\SnCell{
			$\alpha \swap \beta$\\
			$\alpha \subst \symlnot\beta$
		}  & $\dfrac{\beta \srand \symlnot\beta}{\beta \symland \symlnot\beta}$ & 8, 3 & $\beta \symland \symlnot\beta$ &
		\\\hline
		9 & \ref{def-nota} & & $\dfrac{X \symderive \alpha, \symlnot\alpha}{X \symderive \beta}$ & & &
		\\\hline
		10 & -1 & $X \subst \emptyset$ & $\dfrac{\alpha \srand \symlnot\alpha}{\beta}$ & & &
		\\\hline
		11 & -1 & \SnCell{
			$\alpha \swap \beta$\\
			$\alpha \subst \symlnot\alpha$
		} & $\dfrac{\beta \srand \symlnot\beta}{\symlnot\alpha}$ & 8, 3 & $\symlnot\alpha$ & 2, 3, 4
		\\\hline
		12 & \impE & & $\dfrac{X, \alpha \symderive \beta}{X \symderive \alpha\symlimp\beta}$ & & &
		\\\hline
		13 & -1 & $X \subst \emptyset$ & $\dfrac{\alpha \symderive \beta}{\alpha\symlimp\beta}$ & & &
		\\\hline
		14 & -1 & \SnCell{
			$\alpha \swap \beta$\\
			$\alpha \subst \symlnot\alpha$\\
			$\beta \subst \symlnot\beta$
		} & $\dfrac{\symlnot\beta \symderive \symlnot\alpha}{\symlnot\beta\symlimp\symlnot\alpha}$ & 3, 11, ??? & $\symlnot\beta\symlimp\symlnot\alpha$ & 2, 3, 4, ???
		\\\hline
		15 & \impE+1 & \SnCell{
			$\alpha \subst \gamma$\\
			$\beta \subst \delta$\\
			$\gamma \subst \alpha\symlimp\beta$\\
			$\delta \subst \symlnot\beta\symlimp\symlnot\alpha$
		} & $\dfrac{\alpha\symlimp\beta \symderive \symlnot\beta\symlimp\symlnot\alpha}
		{(\alpha\symlimp\beta)\symlimp(\symlnot\beta\symlimp\symlnot\alpha)}$ & 2, 14 &
		$(\alpha\symlimp\beta)\symlimp(\symlnot\beta\symlimp\symlnot\alpha)$ & 2, 3, 4, ???
		\\\hline\hline
		16 & \centerParbox{1.5cm}{\impE, \impB, \SR} & & $\dfrac{}{A_1}$ & & $\dfrac{}{(\alpha\symlimp\beta)\symlimp(\symlnot\beta\symlimp\symlnot\alpha)}$ &
		\\\hline
	\end{tabular}
	\caption{\Ableitung\ der \Kontraposition\ aus \allgemeingueltigen\ \Schlussregeln}
	\label{tab-AbleitungKontraposition}
\end{table}

\todo{Beispielableitung der Kontraposition vervollständigen}%%%
%TODO Beispielableitung der Kontraposition vervollständigen %%%

\Endchapter


	\beforechapter
	\likechapter{Verzeichnisse}
	\Endchapter

	%section{Tabellenverzeichnis}% =============================================
	\likechapter[section]{\listtablename}
%%%	\addcontentsline{lot}{chapter}{Test der 2. Seite Tabellenverzeichnis}%%%
	\label{dic:Tabellenverzeichnis}
	\listoftables
	\Thispagestyle%              veränderte Kopfzeile für diese Seite aktivieren

	%section{Abbildungsverzeichnis}% ===========================================
	\likechapter[section]{\listfigurename}
%%%	\addcontentsline{lof}{chapter}{Test der 2. Seite Abbildungsverzeichnis}%%%
	\begin{minipage}{\linewidth-10.95pt}
		\label{dic:Abbildungsverzeichnis}
		\listoffigures
	\end{minipage}
	\Thispagestyle%              veränderte Kopfzeile für diese Seite aktivieren

	%%############################################################################%%
%%                                                                            %%
%% Datei:  ASBA-Literaturverzeichnis.tex                                      %%
%% Inhalt: Literaturverzeichnis                                               %%
%%                                                                            %%
%% Copyright (C) 2017  Winfried Teschers                                      %%
%%                                                                            %%
%% This program is free software: you can redistribute it and/or modify       %%
%% it under the terms of the GNU Affero General Public License as published   %%
%% by the Free Software Foundation, either version 3 of the License, or       %%
%% (at your option) any later version.                                        %%
%%                                                                            %%
%% This program is distributed in the hope that it will be useful,            %%
%% but WITHOUT ANY WARRANTY; without even the implied warranty of             %%
%% MERCHANTABILITY or FITNESS FOR A PARTICULAR PURPOSE.  See the              %%
%% GNU Affero General Public License for more details.                        %%
%%                                                                            %%
%% You should have received a copy of the GNU Affero General Public License   %%
%% along with this program.  If not, see <http://www.gnu.org/licenses/>.      %%
%%                                                                            %%
%% Dr. Winfried Teschers                                                      %%
%% Anton-Günther-Straße 26c                                                   %%
%% 91083 Baiersdorf                                                           %%
%% Germany                                                                    %%
%%                                                                            %%
%% e-mail: winfried.teschers@t-online.de                                      %%
%%                                                                            %%
%%############################################################################%%

% !TeX root = ASBA.tex
% !TeX encoding = UTF-8
% !TeX spellcheck = de_DE

%chapter{Literaturverzeichnis}% ################################################

\begin{flushleft}
	\begin{thebibliography}{12}
		\likechapter[section]{\bibname}  % erst hier!
		\label{dic:Literaturverzeichnis} % erst hier!

		\bibitem{bib:Rautenberg}Wolfgang Rautenberg,
		\emph{Einführung in die Mathematische Logik}:
		Ein Lehrbuch, 3.\@ Auflage, Vieweg+Teubner 2008

		\bibitem{bib:Apacheii}\emph{Apache License}, Version 2.0
		$\rightarrow$%
		\footnote{%
			Der Pfeil~($\rightarrow$) verweist stets auf einen Link zu einer Seite im Internet.%
		}%
		~\url{http://www.apache.org/licenses/LICENSE-2.0}
		02.01.2004%
		\footnote{%
			Das Datum hinter dem Link gibt -- je nachdem welches bekannt ist -- das Datum der letzten Änderung, den Stand der Seite oder das Datum, an dem die Seite angeschaut wurde an.
			Sind mehrere Daten vorhanden, wird das erste vorhandene in der angegebenen Reihenfolge genommen.
			-- Dies gilt für alle hier aufgelisteten Seiten im Internet.%
		}

		\bibitem{bib:BSLi}\emph{Boost Software License} 1.0
		\tourl{http://www.boost.org/users/license.html}
		17.08.2003

		\bibitem{bib:EPL}\emph{Eclipse Public License} Version 1.0
		\tourl{http://www.eclipse.org/org/documents/epl-v10.php}
		09.03.2017

		\bibitem{bib:AGPL}\emph{GNU Affero General Public License}
		\tourl{http://www.gnu.org/licenses/agpl}
		19.11.2007

		\bibitem{bib:GPLi}\emph{GNU General Public License}
		\tourl{http://www.gnu.org/licenses/old-licenses/gpl-1.0}
		02.1989

		\bibitem{bib:GPLii}\emph{GNU General Public License}, Version 2
		\tourl{http://www.gnu.org/licenses/old-licenses/gpl-2.0}
		06.1991

		\bibitem{bib:LGPLii}\emph{GNU Lesser General Public License},
		Version 2.1
		\tourl{http://www.gnu.org/licenses/old-licenses/lgpl-2.1}
		02.1999

		\bibitem{bib:Clover}Lizenz für \emph{Clover}
		\tourl{https://www.atlassian.com/software/clover}
		2017

		\bibitem{bib:EULA}Lizenz
		für \emph{Microsoft Visual Studio Express 2015}
		\tourl{https://www.visualstudio.com/de/license-terms/mt171551/}
		2017

		\bibitem{bib:MiKTeX}Lizenz für \emph{MikTeX}
		\tourl{https://miktex.org/kb/copying}
		14.01.2014

		\bibitem{bib:SAX}Lizenz für \emph{SAX}
		\tourl{http://www.saxproject.org/copying.html}
		05.05.2000

		\bibitem{bib:MIT}\emph{MIT License}
		\tourl{https://opensource.org/licenses/MIT/}
		09.03.2017

		\bibitem{bib:JavaSE}\emph{Oracle Binary Code License Agreement}
		\tourl{http://java.com/license}
		02.04.2013

		\bibitem{bib:OSI}\emph{OSI Certified Open Source Software}
		\tourl{https://opensource.org/pressreleases/certified-open-source.php}
		16.06.1999

		\bibitem{bib:WDCDL}\emph{W3C Document License}
		\tourl{http://www.w3.org/Consortium/Legal/2015/doc-license}
		01.02.2015

		\bibitem{bib:WDCSNL}\emph{W3C Software Notice and License}
		\tourl{http://www.w3.org/Consortium/Legal/2002/copyright-software-20021231.html}
		13.05.2015

		\bibitem{bib:HilbertII}\emph{Hilbert II -- Introduction}
		\tourl{http://www.qedeq.org/}
		20.01.2014

		\bibitem{bib:qedeq}\emph{Formal Correct Mathematical Knowledge}:
		GitHub Repository vom Projekt Hilbert II
		\tourl{https://github.com/m-31/qedeq/}
		04.08.2016

		\bibitem{bib:ASBA}\emph{ASBA
			-- Axiome, Sätze, Beweise und Auswertungen}.
		Projekt zur maschinellen Überprüfung von mathematischen Beweisen
		und deren Ausgabe in lesbarer Form:
		GitHub Repository vom Projekt ASBA
		-- in Bearbeitung
		\tourl{https://github.com/Dr-Winfried/ASBA}

		\bibitem{bib:LogikDe}Meyling, Michael:
		\emph{Anfangsgründe der mathematischen Logik}
		\tourl{http://www.qedeq.org/current/doc/math/qedeq\_logic\_v1\_de.pdf}
		24.~Mai~2013 (in Bearbeitung)

		\bibitem{bib:PraedikatenlogikDe}Meyling, Michael:
		\emph{Formale Prädikatenlogik}
		\tourl{http://www.qedeq.org/current/doc/math/qedeq\_formal\_logic\_v1\_de.pdf}
		24.~Mai~2013 (in Bearbeitung)

		\bibitem{bib:MengenlehreDe}Meyling, Michael:
		\emph{Axiomatische Mengenlehre}
		\tourl{http://www.qedeq.org/current/doc/math/qedeq\_set\_theory\_v1\_de.pdf}
		24.~Mai~2013 (in Bearbeitung)

		\bibitem{bib:LogikEn}Meyling, Michael:
		\emph{Elements of Mathematical Logic}
		\tourl{http://www.qedeq.org/current/doc/math/qedeq\_logic\_v1\_en.pdf}
		24.~Mai~2013 (in Bearbeitung)

		\bibitem{bib:PraedikatenlogikEn}Meyling, Michael:
		\emph{Formal Predicate Calculus}
		\tourl{http://www.qedeq.org/current/doc/math/qedeq\_formal\_logic\_v1\_en.pdf}
		24.~Mai~2013 (in Bearbeitung)

		\bibitem{bib:MengenlehreEn}Meyling, Michael:
		\emph{Axiomatic Set Theory}
		\tourl{http://www.qedeq.org/current/doc/math/qedeq\_set\_theory\_v1\_en.pdf}
		24.~Mai~2013 (in Bearbeitung)

		\bibitem{bib:Junktor}Wikipedia:
		\emph{Aussagenlogik} \chaptername~2.2 \emph{Mögliche Junktoren}
		\tourl{https://de.wikipedia.org/wiki/Junktor\#M.C3.B6gliche\_Junktoren}
		20.01.2016

		\bibitem{bib:Aussagenlogik}Wikipedia:
		\emph{Aussagenlogik} \chaptername~4 \emph{Formaler Zugang}
		\tourl{https://de.wikipedia.org/wiki/Aussagenlogik\#Formaler\_Zugang}
		13.02.2017

		\bibitem{bib:Identitaet}Wikipedia:
		\emph{Identität (Logik)} \chaptername~2.3
		\emph{Identität in der Informatik}
		\tourl{https://de.wikipedia.org/wiki/Identit\%C3\%A4t\_(Logik)\#Identit.C3.A4t\_in\_der\_Informatik}
		18.05.2017

		\bibitem{bib:Kalkuel}Wikipedia:
		\emph{Kalkül}
		\tourl{https://de.wikipedia.org/wiki/Kalk\%C3\%BCl}
		26.02.2017

		\bibitem{bib:Mengenlehre}Wikipedia:
		\emph{Mengenlehre}
		\tourl{https://de.wikipedia.org/wiki/Mengenlehre}
		03.03.2017

		\bibitem{bib:Praedikatenlogik}Wikipedia:
		\emph{Prädikatenlogik erster Stufe}
		\tourl{https://de.wikipedia.org/wiki/Pr\%C3\%A4dikatenlogik\_erster\_Stufe}
		24.02.2017

		\bibitem{bib:Schlussregel}Wikipedia:
		\emph{Schlussregel}
		\tourl{https://de.wikipedia.org/wiki/Schlussregel}
		01.05.2017

		\bibitem{bib:NatuerlichesSchliessen}Wikipedia:
		\emph{Natürliches Schließen}
		\tourl{https://de.wikipedia.org/wiki/Systeme\_nat\%C3\%BCrlichen\_Schlie\%C3\%9Fens}
		01.05.2017

	\end{thebibliography}
\end{flushleft}

\Endchapter


% Indizes und Glossaries #######################################################

% Um Kopf- und Fußzeilen und korrekte Referenzen zu erhalten, müssen die
% Programme 'splitindices.pl' und 'makeglossaries.exe' aufgerufen und die
% erzeugten Dateien 'ASBA-*.ind' und 'ASBA.gls' modifiziert werden. Dies tut das
% Batchprogramm 'erzeugeIndizes.bat' und fügt mit Hilfe des Batchprogramms
% 'einfuegen.bat' in die Dateien 'ASBA-*.ind' und 'ASBA.gls' nach der ersten
% bzw. zweiten Zeile je eine Zeile mit "  \insert*" bzw. "\insertglo" und dem
% anschließenden Kommentar "% -- Eingefuegt" ein.
%
% Nach der ersten Übersetzung dieser Datei muss das Programm 'einfuegen.bat'
% manuell aufgerufen werden. Danach ist noch eine Übersetzung notwendig.
%
% Die Kommandos '\insert*' und '\insertglo' müssen vor Aufruf von '\printindex'
% bzw. '\printglossaries' mit '\newcommand' definiert werden.
%
% '*' steht für die vorhandenen Indizes 'idx' und 'sym'.

	\beforechapter% Kopfzeile aktivieren und neue Seite
	\idxdictionary{Index}% Index ===============================================
	\newcommand*{\insertidx}{%
		\label{idx:Index}
	}
	\printindex[idx]

	\beforechapter% Kopfzeile aktivieren und neue Seite
	\idxdictionary{Symbolverzeichnis}% =========================================
	\newcommand*{\insertsym}{%
		\label{idx:Symbolverzeichnis}
	}
	\printindex[sym]

	\beforechapter% Kopfzeile aktivieren und neue Seite
	\glodictionary{\glossaryname}% Index =======================================
	\newcommand*{\insertglo}{%
		\label{glo:Glossar}
		\Thispagestyle
		\addcontentsline{toc}{section}{\glossaryname}% Eintrag ins Inhaltsverzeichnis
	}
	\printglossaries

\end{document}

% Ende des Dokuments ###########################################################
