%%############################################################################%%
%%                                                                            %%
%% Datei:  ASBA.tex                                                           %%
%% Inhalt: Erzeugung des Projektdokuments von ASBA.                           %%
%%                                                                            %%
%% Copyright (C) 2017  Winfried Teschers                                      %%
%%                                                                            %%
%% This program is free software: you can redistribute it and/or modify       %%
%% it under the terms of the GNU Affero General Public License as published   %%
%% by the Free Software Foundation, either version 3 of the License, or       %%
%% (at your option) any later version.                                        %%
%%                                                                            %%
%% This program is distributed in the hope that it will be useful,            %%
%% but WITHOUT ANY WARRANTY; without even the implied warranty of             %%
%% MERCHANTABILITY or FITNESS FOR A PARTICULAR PURPOSE.  See the              %%
%% GNU Affero General Public License for more details.                        %%
%%                                                                            %%
%% You should have received a copy of the GNU Affero General Public License   %%
%% along with this program.  If not, see <http://www.gnu.org/licenses/>.      %%
%%                                                                            %%
%% Dr. Winfried Teschers                                                      %%
%% Anton-Günther-Straße 26c                                                   %%
%% 91083 Baiersdorf                                                           %%
%% Germany                                                                    %%
%%                                                                            %%
%% e-mail: winfried.teschers@t-online.de                                      %%
%%                                                                            %%
%%############################################################################%%

% !TeX root = ASBA.tex
% !TeX encoding = UTF-8
% !TeX spellcheck = de_DE

%%############################################################################%%
%%                                                                            %%
%% Datei:  ASBA-Vorspann.tex                                                  %%
%% Inhalt: Vorspann für die Datei ASBA.txt                                    %%
%%                                                                            %%
%% Copyright (C) 2017  Winfried Teschers                                      %%
%%                                                                            %%
%% This program is free software: you can redistribute it and/or modify       %%
%% it under the terms of the GNU Affero General Public License as published   %%
%% by the Free Software Foundation, either version 3 of the License, or       %%
%% (at your option) any later version.                                        %%
%%                                                                            %%
%% This program is distributed in the hope that it will be useful,            %%
%% but WITHOUT ANY WARRANTY; without even the implied warranty of             %%
%% MERCHANTABILITY or FITNESS FOR A PARTICULAR PURPOSE.  See the              %%
%% GNU Affero General Public License for more details.                        %%
%%                                                                            %%
%% You should have received a copy of the GNU Affero General Public License   %%
%% along with this program.  If not, see <http://www.gnu.org/licenses/>.      %%
%%                                                                            %%
%% Dr. Winfried Teschers                                                      %%
%% Anton-Günther-Straße 26c                                                   %%
%% 91083 Baiersdorf                                                           %%
%% Germany                                                                    %%
%%                                                                            %%
%% e-mail: winfried.teschers@t-online.de                                      %%
%%                                                                            %%
%%############################################################################%%

% !TeX root = ASBA.tex
% !TeX encoding = UTF-8
% !TeX spellcheck = de_DE

\documentclass[english, ngerman, parskip=half, headsepline, footsepline, fleqn, notitlepage]{scrreprt}

% Pakete #######################################################################

% allgemein --------------------------------------------------------------------
\usepackage[utf8]{inputenc}% Input encoding specification
\usepackage[T1]{fontenc}
\usepackage{lmodern}
\usepackage{scrlayer-scrpage}
\usepackage{geometry}% Flexible and complete interface to document dimensions.
\usepackage{microtype}% Subliminal refinements towards typographical perfection.
\usepackage{graphicx}% Alternative interface to graphics functions.
\usepackage{pict2e}% New implementation of picture commands.
\usepackage{multicol}% An environment for multicolumn output
\usepackage{babel}% Multilingual support for plain TeX or LaTeX.
\usepackage[autostyle]{csquotes}% Contex sensitive quotation facilities

% mathematische Pakete ---------------------------------------------------------
\usepackage{amsmath}% Mathematical facilities for LaLeX from ASM
\usepackage{amsfonts}% TeX fonts from the American Mathematical Society.
\usepackage{amssymb}% Symbols from the American Mathematical Society.
\usepackage{mathtools}% Mathematical tools to use with asmmath.
\usepackage{mathabx}% Three series of mathematical symbols.
\usepackage{mathpazo}% Fonts to typeset mathematics to match palatino.
%\usepackage{cancel}% Place lines through mathematical formulae.

% Tabellen ---------------------------------------------------------------------
\usepackage[table]{xcolor}% Driver-independent color extensions - vor 'color'?
%\usepackage{ctable}% Flexible typesetting of table and figure using key/value
%% Das Paket ctable fast die Eigenschaften der Pakete
\usepackage{array}%
\usepackage{tabularx}% Erweiterung von tabular*
\usepackage{booktabs}% Nicer layout of tables
%% zusammen und lädt zusätzlich noch die Pakete
\usepackage{rotating}% Rotating tools, including rotated full page floats
\usepackage{xspace}% Behandelt Zwischenraum nach Makros
\usepackage{color}% LaTeX support for color
\usepackage{xkeyval}% Extension of the keyval package
% Ende der von 'ctable' geladenen Pakete
\usepackage{threeparttable}% Tables with captions and notes all the same width.
\usepackage{multirow}% Create tabular cell spanning multiple rows.
\usepackage{diagbox}% Table heads with diagonal lines.
\usepackage{arydshln}% Draw dash-lines in array/tabular.
\usepackage{caption}% Customizing captions in floating environments.

% Indizes ----------------------------------------------------------------------
%\usepackage{makeidx}% Indexing - Entweder 'makeidx' oder 'splitidx'
\usepackage[protected]{splitidx}% mehrere Indizes - statt 'makeidx'
%\usepackage{hvindex}% Support for indexing - after 'babel
%\usepackage{showidx}% Index auf Seitenrand anzeigen - Zum Testen der Indizes
%TODO Fehler: 'showindex' gibt direkt und nicht auf Rand aus
\usepackage{glossaries}% Create glossaries and lists of acronyms
%TODO Fehler: 'hyperfirst' hat keine Wirkung'.
%\usepackage{glossaries-german}% German language module for glossaries package

% Verweise ---------------------------------------------------------------------
%\usepackage[germanb]{minitoc}\dosectoc% Unterverzeichnisse erstellen
%TODO Fehler: 'minitoc' funktioniert nicht
\usepackage{varioref}
\usepackage[colorlinks,linktoc=all]{hyperref}% Extensive support for hypertext
\usepackage{glossaries}% Create glossaries and lists of acronyms
% lädt     {glossaries-german}% German language module for glossaries package

% Einstellung von globalen Werten und Makro-Redefinitionen #####################

\geometry{textwidth=170mm,textheight=256mm,twoside}% optional Option 'showframe'

% Kopfzeilen ===================================================================
\newcommand*{\texthead}[1]{\textnormal{\textsf{\textbf{#1}}}}% Schriftart
\newcommand*{\Lehead}[1]{\lehead{\texthead{#1}}}
\newcommand*{\Cehead}[1]{\cehead{\texthead{#1}}}
\newcommand*{\Rehead}[1]{\rehead{\texthead{#1}}}
\newcommand*{\Lohead}[1]{\lohead{\texthead{#1}}}
\newcommand*{\Cohead}[1]{\cohead{\texthead{#1}}}
\newcommand*{\Rohead}[1]{\rohead{\texthead{#1}}}
\newcommand*{\Ohead}[1]{\ohead{\texthead{#1}}}
\newcommand*{\Chead}[1]{\chead{\texthead{#1}}}
\newcommand*{\Ihead}[1]{\ihead{\texthead{#1}}}
\newcommand*{\Ofoot}[1]{\ofoot{\textnormal{\textbf{#1}}}}
\newcommand*{\Cfoot}[1]{\cfoot{\textnormal{#1}}}
\newcommand*{\Ifoot}[1]{\ifoot{\textnormal{#1}}}
\newcommand*{\Pagestyle}{\pagestyle{scrheadings}}
\newcommand*{\Thispagestyle}{\thispagestyle{scrheadings}}

% Kopfzeilen mit 'scrlayer-scrpage'
%         \Lehead \Cehead \Rehead | \Lohead \Cohead \Rohead
% \Ohead: \Lehead                                   \Rohead
% \Chead:         \Cehead                   \Cohead
% \Ihead:                 \Rehead   \Lohead
% ASBA <Chapter-Überschrift> \Chaptername~\thechapter
%                            \sectionname~\thesection <Section-Überschrift> ASBA
%Initialisierung
\Ohead{ASBA}%               bleibt unverändert
\Chead{Inhaltsverzeichnis}% wird laufend verändert
\Ihead{}%                   wird laufend verändert

% Kapitel ======================================================================
\newcommand*{\Chaptername}{\chaptername}% wird mit 'Anhang' überschrieben

\newcommand*{\beforechapter}{% direkt vor \chapter (auch im Kommentar)
	\Thispagestyle%        Kopfzeile für diese Seite aktivieren - vor \clearpage
	\clearpage%            neue Seite
}
\newcommand*{\beginchapter}[1]{%        direkt nach \chapter
	\Chead{#1}%                         Kopfzeile Mitte = <Kapitelname>
	\Ihead{\Chaptername~\thechapter}%   Kopfzeile Innen = Kapitel/Anhang <Nr.>
	\Pagestyle%                         veränderte Kopfzeile aktivieren
	\Thispagestyle%                     ... auch für diese Seite, da ...
}%                                      '\chapter' Kopf-/Fußzeilen deaktiviert
\newcommand*{\likechapter}[2][chapter]{%statt \beginchapter für Inhalts-,
	%                                   Tabellen- und Abbildungsverzeichnis
	\Chead{#2}%                         Mitte in der Kopfzeile = <Kapitelname>
	\Ihead{}%                           Innen in der Kopfzeile = <leer>
	\Pagestyle%                         veränderte Kopfzeile aktivieren ...
	\Thispagestyle%                     sicherheitshalber auch für diese Seite.
	\addcontentsline{toc}{#1}{#2}%      Eintrag ins Inhaltsverzeichnis
	%TODO Fehler: 2. Seite Kopf Mitte für Literaturverzeichnis = Abbildungsverzeichnis
}
\newcommand*{\Endchapter}{% am Ende eines Kapitels
	\Thispagestyle% sicherheitshalber Kopfzeile für diese Seite aktivieren
}
% Indizes ======================================================================
\newcommand*{\idxdictionary}[1]{%       nur für Indices
	\Thispagestyle%       Kopfzeile für diese Seite aktivieren - vor \clearpage
	\clearpage%                         neue Seite
	\extendtheindex{}{%                 aktiviert die Kopfzeile für Index-Seiten
		\Chead{#1}%                     Mitte in der Kopfzeile = <Kapitelname>
		\Ihead{}%                       Innen in der Kopfzeile = <leer>
		\Pagestyle%                     veränderte Kopfzeile aktivieren
		\Thispagestyle%                 ... auch für diese Seite
		\addcontentsline{toc}{section}{#1}% Eintrag ins Inhaltsverzeichnis
	}{}{}
}
\newcommand*{\glodictionary}[1]{%       nur für Glossary
	\Chead{#1}%                         Mitte in der Kopfzeile = <Kapitelname>
	\Ihead{}%                           Innen in der Kopfzeile = <leer>
	\Pagestyle%                         veränderte Kopfzeile aktivieren
}

% Abschnitte ===================================================================
\newcommand*{\beginsection}[1]{%        direkt nach \section
	\Cohead{#1}%                        oben rechts mittig = <Abschnittsname>
	\Lohead{\sectionname~\thesection}%  oben rechts innen = <Abschnittsnummer>
	\Pagestyle%                         veränderte Kopfzeile aktivieren
}

% Fußzeilen ====================================================================
\Ofoot{\thepage}
\Cfoot{Winfried Teschers}
\Ifoot{\today}
\Pagestyle%                                       aktiviert Kopf- und Fußzeilen

% Fußnoten ---------------------------------------------------------------------
\deffootnote[10pt]% Markenbreite
{10pt}% Einzug - für Blocksatz: Markenbreite
{0pt}% Absatzeinzug für Folgeabsätze
{\makebox[9pt][r]{\textsuperscript{\thefootnotemark)} }}% Zeichen; < Markenbreite
\deffootnotemark {\textsuperscript{\thefootnotemark)}}

% Vordefinierte Werte ändern ===================================================
\setcounter{tocdepth}{3}%    Tiefe des Inhaltsverzeichnisses: 2 => subsection
\setcounter{secnumdepth}{3}% Nummerierung:                    3 => subsubsection
\setlength\extrarowheight{1pt}% Tabellenzellenhöhe vergrößern
\captionsetup{labelfont=bf}%    Tabellenbeschriftung in bf = bold font

% Empfehlung aus: Herbert Voß, LaTeX Referenz, 3. Auflage, Berlin 2014; S. 37f
\renewcommand{\floatpagefraction}{0.7}% Empfehlung: 0.5-0.8 Voreinstellung: 0.9
\renewcommand{\textfraction}{0.15}%                 0.1-0.3                 0.05
\renewcommand{\topfraction}{0.8}%                   0.5-0.85                0.9
\renewcommand{\bottomfraction}{0.5}%                0.2-0.5                 0.9
\setcounter{topnumber}{3}%                                                  2
\setcounter{totalnumber}{15}%                                               3

% Neue Elemente ----------------------------------------------------------------
\newcounter{Enumi}% für unterbrochene Listennummerierung

% Bildelemente #################################################################

\newcommand*{\textbild}[1]{\textbf{\textsf{#1}}}% Textauszeichnungen für Text im Bild
\newcommand*{\Datei}[4][0.5]{% #2 x #3 = (-#2/2,-#3/2),(#2/2,#3/2)
	% [Eck-Radius], Breite, Höhe, Name
	\put(0,0){\oval[#1](#2,#3)}
	\put(0,0){\makebox(0,0){\textbild{#4}}}
}
\newcommand*{\Datenbank}[5]{% 2(#1) x 2(#2+#3) = (-#1,-#2-#3),(+#1,+#2+#3)
	% Halbmesser x, Halbmesser y, halbe Höhe, Name - Ursprung in der Mitte
	\put(0.0,-#3){
		\qbezier(-#1,0.0)(-#1,-#2)(0.0,-#2)
		\qbezier(+#1,0.0)(+#1,-#2)(0.0,-#2)
	}
	\put(0,0){\Line(-#1,-#3)(-#1, #3)}
	\put(0,0){\Line( #1,-#3)( #1, #3)}
	\put(0.0,#3){
		\qbezier(-#1,0.0)(-#1,-#2)(0.0,-#2)
		\qbezier( #1,0.0)( #1,-#2)(0.0,-#2)
		\qbezier(-#1,0.0)(-#1, #2)(0.0, #2)
		\qbezier( #1,0.0)( #1, #2)(0.0, #2)
	}
	\makebox(0,0){\textbild{#4}}
	\makebox(0,-#3){\textbild{#5}}
}
% '\Männchen' (mit 'ä') führt zu Fehler
\newcommand*{\Maennchen}{% 1x2 = (-0.5,-1.7),(+0.5,+0.3) Ursprung im Kopf
	\put(0,0){\circle{0.6}}
	\Line(0.0,-0.3)(0.0,-1.2)
	\polyline(-0.5,-0.3)(0.0,-0.6)(0.5,-0.3)
	\polyline(-0.5,-1.7)(0.0,-1.2)(0.5,-1.7)
}
\newcommand*{\Marker}[2][0.5]{% 2x#1 x 2x#1 - Kreis mit Text
	{
		\linethickness{0.5pt}
		\color{white}
		\put(0,0){\circle*{#1}}
		\color{black}
		\put(0,0){\circle{#1}}
		\put(0,0){\makebox(0,0){\small\textbild{#2}}}
	}
}
\newcommand*{\marker}[2][0.5]{% 2x#1 x 2x#1 - Kreis mit Text - grau
	{
		\linethickness{0.5pt}
		\color{white}
		\put(0,0){\circle*{#1}}
		\color{gray}
		\put(0,0){\circle{#1}}
		\put(0,0){\makebox(0,0){\small\textbild{#2}}}
	}
}
\newcommand*{\Papier}[3]{% 3x#1+#2 = (0.0,-#2),(2.1,#1) Ursprung links unten
	% Länge (Höhe), Länge Abschluss, Name
	\polyline(0.0,-0.01)(+0.0,+#1)(+2.8,+#1)(+2.8,-0.01)
	\qbezier(+2.1,+#2)(+2.6,+#2)(+2.8,0.0)
	\qbezier(+2.1,+#2)(+1.6,+#2)(+1.4,0.0)
	\qbezier(+0.7,-#2)(+1.2,-#2)(+1.4,0.0)
	\qbezier(+0.7,-#2)(+0.2,-#2)(+0.0,0.0)
	\put(0,0){\makebox(+2.8,+#1){\textbild{#3}}}
}
\newcommand*{\Terminal}[1]{% 2x2 =(-1.0,-1.4),(+1.0,+0.6)Ursprung im Monitor
	% Bildschirm
	%		\put(0,0){\polygon(-1.0,-0.6)(+1.0,-0.6)(+1.0,+0.6)(-1.0,+0.6)}
	\put(-1.0,-0.6){\framebox(2,1.2){#1}}
	\put(0,0){\oval[0.1](1.65,0.85)}
	% Hals
	\put(0,0){\Line(-0.2,-0.6)(-0.2,-1.0)}
	\put(0,0){\Line(+0.2,-0.6)(+0.2,-1.0)}
	% Tastatur
	\multiput(-1.0,-1.0)(+0.0,-0.133){4}{\line(1,0){2.0}}
	\multiput(-1.0,-1,0)(+0.2,+0.0){11}{\line(0,-1){0.4}}
}
\newcommand*{\Wolke}[1]{% 3.0x1.5 = (-1.5,-1.0),(+1.5,+0.5)
	% unterer Bogen
	\qbezier(-1.5,+0.0)(-1.5,-1.0)(-0.0,-1.0)
	\qbezier(+1.5,+0.0)(+1.5,-1.0)(+0.0,-1.0)
	% oberer Bogen rechts
	\qbezier(+1.5,+0.0)(+1.5,+0.5)(+0.8,+0.5)
	\qbezier(+0.4,+0.4)(+0.4,+0.5)(+0.8,+0.5)
	% oberer Bogen Mitte
	\qbezier(+0.5,+0.2)(+0.5,+0.5)(+0.0,+0.5)
	\qbezier(-0.4,+0.4)(-0.4,+0.5)(-0.0,+0.5)
	% oberer Bogen links
	\qbezier(-0.3,+0.2)(-0.3,+0.5)(-0.8,+0.5)
	\qbezier(-1.5,+0.0)(-1.5,+0.5)(-0.8,+0.5)
	\put(-1.5,-1.0){\makebox(3.0,1.5){\textbild{#1}}}
}

% Metasprachliche Symbole ######################################################

\newcommand*{\metaund}{\&\&}%               Und-Symbol  für Texte
\newcommand*{\metaoder}{||}%                Oder-Symbol für Texte
\newcommand*{\metaand}{\;\metaund\;}%       Und-Symbol  für Formeln
\newcommand*{\metaor}{\;\metaoder\;}%       Oder-Symbol für Formeln
% Nur im Mathematikmodus!
\newcommand*{\metaimp}{\Rightarrow}%        aus ... folgt ...
\newcommand*{\metarep}{\Leftarrow}%         ... folgt aus ...
\newcommand*{\metaequiv}{\Leftrightarrow}%  ... genau dann wenn ...
\newcommand*{\metadefeq}{:\Leftrightarrow}% ... definitionsgemäß genau dann wenn
\newcommand*{\defeq}{:=}%                   ... definitionsgemäß gleich ...
\newcommand*{\eq}{=}%                       ... gleich ...

% Mathematische Symbole ########################################################

% Beispieloperatoren ===========================================================
% \*bsp
\newcommand*{\opbsp}{\circ}
\newcommand*{\relbsp}{\sim}
\newcommand*{\releqbsp}{\simeq}
\newcommand*{\lrelbsp}{\lhd}
\newcommand*{\rrelbsp}{\rhd}
\newcommand*{\lreleqbsp}{\unlhd}
\newcommand*{\rreleqbsp}{\unrhd}

% Definitionen für die Tabelle der Junktoren ===================================
% \l*  -           logischer Operator
% \ln* - negierter logischer Operator
% Logische Operatoren als Addition und Multiplikation
\newcommand*{\ladd}{+}
\newcommand*{\lmult}{\cdot}
% Wahrheitswerte ---------------------------------------------------------------
\newcommand*{\texttrue}{W}%  in einem Kommentar stets 'W'
\newcommand*{\textfalse}{F}% in einem Kommentar stets 'F'
% Konstante --------------------------------------------------------------------
\newcommand*{\ltrue}{\top}%      W - wahr
%\newcommand*{\lnfalse}{\notbot}% " - nicht falsch
\newcommand*{\lfalse}{\bot}%     F - falsch
%\newcommand*{\lntrue}{\nottop}%  " - nicht wahr
% unäre Operatoren -------------------------------------------------------------
%                                                 W F - Aussage A
%\newcommand*{\lutrue}{\operatorname{\top}}%      W W - wahr [unär]
%\newcommand*{\lnufalse}{\operatorname{\notbot}}% " " - nicht falsch [unär]
%                                                 W F - A
%             \lnot                               F W - nicht
%\newcommand*{\lufalse}{\operatorname{\bot}}%     F F - falsch [unär]
%\newcommand*{\lnutrue}{\operatorname{\nottop}}%  " " - nicht wahr [unär]
% binäre Operatoren ------------------------------------------------------------
%                                                    W W F F - Aussage A
%                                                    W F W F - Aussage B
%  - - - - - - - - - - - - - - - - - - - - - - - - - - - - - - - - - - - - - - -
%\newcommand*{\lbtrue}{\operatorname{\top}}%         W W W W - wahr [binär]
%\newcommand*{\lnbfalse}{\operatorname{\notbot}}%    " " " " - nicht falsch
%            \lor                                    W W W F - A oder B
\newcommand*{\lrep}{\leftarrow}%                     W W F W - A folgt aus B
\newcommand*{\lrepA}{\Leftarrow}%
\newcommand*{\lrepB}{\subset}%
\newcommand*{\lleft}{\operatorname{\rfloor}}%        W W F F - A
%  - - - - - - - - - - - - - - - - - - - - - - - - - - - - - - - - - - - - - - -
\newcommand*{\limp}{\rightarrow}%                    W F W W - aus A folgt B
\newcommand*{\limpA}{\Rightarrow}%
\newcommand*{\limpB}{\supset}%
\newcommand*{\lright}{\operatorname{\lfloor}}%       W F W F - B
\newcommand*{\lequiv}{\leftrightarrow}%              W F F W - A genau dann,
\newcommand*{\lequivA}{\Leftrightarrow}%                       wenn B
%            \lnxor                                  " " " " - nicht
%                                                         (entweder A oder B)
%            \land                                   W F F F - A und B
\newcommand*{\landA}{\&}
\newcommand*{\landB}{\lmult}
%  - - - - - - - - - - - - - - - - - - - - - - - - - - - - - - - - - - - - -
\newcommand*{\lnand}{\uparrow}%                      F W W W - nicht
\newcommand*{\lnandA}{\barwedge}%                              (A und B)
\newcommand*{\lnandB}{\mid}%
\newcommand*{\lxor}{\ladd}%                          F W W F - entweder A
\newcommand*{\lxorA}{\operatorname{\dot\lor}}%                 oder B
\newcommand*{\lxorB}{\veebar}%
\newcommand*{\lxorC}{\oplus}%
\newcommand*{\lnequiv}{\nleftrightarrow}%            " " " " - nicht
\newcommand*{\lnequivA}{\nLeftrightarrow}%            (A genau dann, wenn B)
\newcommand*{\lnequivB}{\notequiv}%
\newcommand*{\lnright}{\lceil}%                      F W F W - nicht B
\newcommand*{\lnimp}{\nrightarrow}%                  F W F F - nicht
\newcommand*{\lnimpA}{\nRightarrow}%                         (aus A folgt B)
\newcommand*{\lnimpB}{\nsupset}%
%  - - - - - - - - - - - - - - - - - - - - - - - - - - - - - - - - - - - - -
\newcommand*{\lnleft}{\rceil}%                       F F W W - nicht A
\newcommand*{\lnrep}{\nleftarrow}%                   F F W F - nicht
\newcommand*{\lnrepA}{\nLeftarrow}%                          (A folgt aus B)
\newcommand*{\lnrepB}{\nsubset}%
\newcommand*{\lnor}{\downarrow}%                     F F F W - nicht
\newcommand*{\lnorA}{\operatorname{\overline\vee}}%            (A oder B)
%\newcommand*{\lbfalse}{\operatorname{\bot}}%        F F F F - falsch[binär]
%\newcommand*{\lnbtrue}{\operatorname{\nottop}}%     " " " " - nicht wahr

% Verwendete Mengenbezeichnungen ===============================================

% \gs* = globales Symbol
\newcommand*{\gsN}{\mathbb{N}}%    Menge der natürlichen Zahlen ohne 0
\newcommand*{\gsNo}{\mathbb{N}_0}% Menge der natürlichen Zahlen einschließlich 0

% \as* = aussagenlogisches Symbol
\newcommand*{\asA}{\mathcal{A}}%       Alphabet der Sprache
\newcommand*{\asAx}{\mathcal{A}_x}%    ... davon eine Teilmenge bzgl. \asJx
\newcommand*{\asAy}{\mathcal{A}_y}%    entsprechend \asAx
\newcommand*{\asB}{\mathcal{B}}%       Menge der binären Operatoren
\newcommand*{\asC}{\mathcal{C}}%       Menge der Konstanten
\newcommand*{\asF}{\mathcal{F}}%       Menge der Formeln
\newcommand*{\asFp}{\mathcal{F}^p}%    ... in polnischer Notation
\newcommand*{\asFx}{\mathcal{F}_x}%    Teilenge der Formeln
\newcommand*{\asFxp}{\mathcal{F}_x^p}% ... bzgl. \asJx in polnischer Notation
\newcommand*{\asFy}{\mathcal{F}_y}%    entsprechend \asFx
\newcommand*{\asFyp}{\mathcal{F}_y^p}% entsprechend \asFxp
\newcommand*{\asJ}{\mathcal{J}}%       Menge der Junktoren
\newcommand*{\asJx}{\mathcal{J}_x}%    Teilmenge der Junktoren
\newcommand*{\asJy}{\mathcal{J}_y}%    entsprechend \asJx
\newcommand*{\asM}{\mathcal{M}}%       Metaoperatoren und "Gleichheiten"
\newcommand*{\asU}{\mathcal{U}}%       Menge der unären Operatoren
\newcommand*{\asV}{\mathcal{V}}%       Menge der atomaren Formeln
\newcommand*{\asX}{\mathcal{X}}%       Mengenvariable
% verschiedene Indizes für Teilmengen von \asA, \asF, \asFp und \asJ
\newcommand*{\xAnd}{\mathrm{and}}%
\newcommand*{\xBool}{\mathrm{bool}}%
\newcommand*{\xImp}{\mathrm{imp}}%
\newcommand*{\xNand}{\mathrm{nand}}%
\newcommand*{\xNor}{\mathrm{nor}}%
\newcommand*{\xOr}{\mathrm{or}}%
\newcommand*{\xRep}{\mathrm{rep}}%

% sonstige mathematische Zeichen ===============================================

\newcommand*{\abltb}{\vdash}%           ... ableitbar ...
\newcommand*{\subst}{\curvearrowright}% ... substituiert durch ...

% sonstige Kommandos für den Mathematiksatz ####################################

\mathtoolsset{showonlyrefs,showmanualtags}% Nur mit \ref referenzierte Gleichungen, aber alle manuellen Tags

% sonstige nützliche Kommandos #################################################

% Im Parameter von '\turl' muss '\' vor jedem Zeichen aus '{}#&%$' ein '\'
% stehen und '\' / '~' durch '\textbackslash' / '\textasctilde' ersetzt werden.
\newcommand*{\tourl}[1]{$\rightarrow$~\url{#1}}
\newcommand*{\formulatoleft}{&&&&&&&&&&}%  Um Formeln nach links zu komprimieren
\newcommand*{\formulaspace}{&&&&}%         Für Platz zwischen den Formeln
\newcommand*{\todo}[1]{\textbf{>~>~>~#1~<~<~<}}% für TODOs
\newcommand*{\charqt}[1]{'#1'}%          Quotierung von Zeichen    (character) %\newcommand*{\charqt}[1]{\guilsinglleft#1\guilsinglright}%  ... Alternative
\newcommand*{\strqt}[1]{\enquote{#1}}%   Quotierung von Zeichenketten (string)
%\newcommand*{\strqt}[1]{\guillemotleft#1\guillemotright}%     ... Alternative
% Nur mit Parameter im Mathematikmodus!
\newcommand*{\symqt}[1]{\charqt{#1}}%    Quotierung einzelner Symbole (symbol)
\newcommand*{\forqt}[1]{\strqt{#1}}%     Quotierung von Formeln      (formula)

% Strukturbezeichnungen ergänzen
\newcommand*{\sectionname}{Abschnitt}
\newcommand*{\subsectionname}{Unterabschnitt}
\newcommand*{\subsubsectionname}{Paragraph}

% Abkürzungen mit Punkten; zur Unterscheidung vom Satzende
\newcommand*{\textbzgl}{bzgl.\@ }
\newcommand*{\textbzw}{bzw.\@ }
\newcommand*{\textdh}{d.\@\,h.\@ }
\newcommand*{\textggf}{ggf.\@ }
\newcommand*{\textiAlg}{i.\@\,Alg.\@ }
\newcommand*{\textua}{u.\@\,a.\@ }
\newcommand*{\textusw}{usw.\@ }
\newcommand*{\textzB}{z.\@\,B.\@ }
\newcommand*{\textZB}{Z.\@\,B.\@ }

% Ergebnis von Makros verschwinden lassen
\newcommand*{\hidden}[1]{}

% Glossareinträge ##############################################################

% Indices und Symbole ==========================================================
\makeindex
\newindex[Symbolverzeichnis]{sym}
\newindex[Index]{idx}
\newcommand*{\Idx}[1]{#1\idx{#1}}%   normaler Index
\newcommand*{\Sym}[1]{#1\sym{$#1$}}% Symbol - Nur im Mathematikmodus verwenden!

% Glossareinträge ==============================================================
\GlsSetQuote{+}% wegen Gebrauch von ngerman; see glossaries guide for beginners
\makeglossaries
\setacronymstyle{long-sc-short}

% Symbol/Index und Glossareintrag ==============================================

\newcommand*{\idx}[1]{\sindex[idx]{#1}}
\newcommand*{\sym}[1]{\sindex[sym]{#1}}
\newcommand*{\glsSym}[1]{\glspl{#1}\sym{\gls{#1}}}% Symbol - Im Mathematikmodus!
\newcommand*{\glsIdx}[1]{\gls{#1}\idx{\gls{#1}}}%        normal
\newcommand*{\GlsIdx}[1]{\Gls{#1}\idx{\gls{#1}}}%        groß
\newcommand*{\glsIdxBg}[2]{#2\idx{\gls{#1}}}%                     Beugung
\newcommand*{\GlsIdxBg}[2]{#2\idx{\gls{#1}}}%            groß und Beugung
\newcommand*{\glsIdxPl}[1]{\glspl{#1}\idx{\gls{#1}}}%             Plural
\newcommand*{\GlsIdxPl}[1]{\Glspl{#1}\idx{\gls{#1}}}%    groß und Plural
\newcommand*{\glsIdxX}[1]{\glsIdxPl{#1}}% Sonderfall
\newcommand*{\GlsIdxX}[1]{\GlsIdxPl{#1}}% Sonderfall und groß
\newcommand*{\Tag}[1]{\tag{\glsPl{#1}}\sym{\gls{#1}}}% Glossary, Symbol, Tag in Formel

% ==============================================================================

% Symbole für Beispieloperatoren -----------------------------------------------

\newglossaryentry{lrelbsp}{
	name={$ \lrelbsp$},
	plural={\lrelbsp},% im Mathematikmodus
	description={%
		Ein Beispielsymbol für eine Relation mit Umkehrrelation $\rrelbsp$%
	}
}
\newglossaryentry{lreleqbsp}{
	name={$ \lreleqbsp$},
	plural={\lreleqbsp},% im Mathematikmodus
	description={%
		Ein Beispielsymbol für eine Relation mit Gleichheit und Umkehrrelation $\rreleqbsp$%
	}
}
\newglossaryentry{relbsp}{
	name={$ \relbsp$},
	plural={\relbsp},% im Mathematikmodus
	description={%
		Ein Beispielsymbol für eine Relation%
	}
}
\newglossaryentry{releqbsp}{
	name={$ \releqbsp$},
	plural={\releqbsp},% im Mathematikmodus
	description={%
		Ein Beispielsymbol für eine Relation mit Gleichheit%
	}
}
\newglossaryentry{rrelbsp}{
	name={$ \rrelbsp$},
	plural={\rrelbsp},% im Mathematikmodus
	description={%
		Ein Beispielsymbol für eine Relation mit Umkehrrelation $\lrelbsp$%
	}
}
\newglossaryentry{rreleqbsp}{
	name={$ \rreleqbsp$},
	plural={\rreleqbsp},% im Mathematikmodus
	description={%
		Ein Beispielsymbol für eine Relation mit Gleichheit und Umkehrrelation $\lreleqbsp$%
	}
}

% Symbole für Metaoperatoren ---------------------------------------------------

\newglossaryentry{defeq}{
	name={$ \defeq$},
	plural={\defeq},% im Mathematikmodus
	description={%
		Ein \glsIdx{MetaoperatorV}: ... definitionsgemäß gleich ..%
	}
}
\newglossaryentry{eq}{
	name={$ \eq$},
	plural={\eq},% im Mathematikmodus
	description={%
		Ein \glsIdx{MetaoperatorV}: ... gleich (ist dasselbe wie, ist identisch zu) ..%
	}
}
\newglossaryentry{equiv}{
	name={$ \equiv$},
	plural={\equiv},% im Mathematikmodus
	description={%
		Ein (Meta-)Operator: ... äquivalent (ist das gleiche wie, ist so wie) zu ..%
	}
}
\newglossaryentry{metaand}{
	name={$ \metaund$},
	plural={\metaand},% im Mathematikmodus
	description={%
		Ein \glsIdx{MetaoperatorV}: ... oder ... (\seealso~\glsSym{mid})%
	}
}
\newglossaryentry{metadefeq}{
	name={$ \metadefeq$},
	plural={\metadefeq},% im Mathematikmodus
	description={%
		Ein \glsIdx{MetaoperatorV}: ... definitionsgemäß gleich (definitionsgemäß genau dann, wenn) ..%
	}
}
\newglossaryentry{metaequiv}{
	name={$ \metaequiv$},
	plural={\metaequiv},% im Mathematikmodus
	description={%
		Ein \glsIdx{MetaoperatorV}: ... genau dann wenn ..%
	}
}
\newglossaryentry{metaimp}{
	name={$ \metaimp$},
	plural={\metaimp},% im Mathematikmodus
	description={%
		Ein \glsIdx{MetaoperatorV}: ... dann auch ..%
	}
}
\newglossaryentry{metaor}{
	name={$ \metaoder$},
	plural={\metaor},% im Mathematikmodus
	description={%
		Ein \glsIdx{MetaoperatorV}: ... oder ..%
	}
}
\newglossaryentry{metarep}{
	name={$ \metarep$},
	plural={\metarep},% im Mathematikmodus
	description={%
		Ein \glsIdx{MetaoperatorV}: ... sofern ..%
	}
}
\newglossaryentry{mid}{
	name={$ \mid$},
	plural={\mid},% im Mathematikmodus
	description={%
		Ein \glsIdx{MetaoperatorV}: ... und ... (\seealso~\glsSym{metaand})%
	}
}
\newglossaryentry{ne}{
	name={$ \ne$},
	plural={\ne},% im Mathematikmodus
	description={%
		Ein \glsIdx{MetaoperatorV}: ... ungleich (nicht dasselbe wie, nicht identisch zu) ..%
	}
}
\newglossaryentry{notequiv}{
	name={$ \notequiv$},
	plural={\notequiv},% im Mathematikmodus
	description={%
		Ein (Meta-)Operator: ... nicht äquivalent (ist nicht das gleiche wie, ist nicht so wie) ..%
	}
}

% sonstige mathematische Symbole -----------------------------------------------

\newglossaryentry{abltb}{
	name={$ \abltb$},
	plural={\abltb},% im Mathematikmodus
	description={%
		Ableitungsrelation: ... ableitbar ...
		(\seename\ \emph{\glsIdx{ableitbar}})%
	}
}
\newglossaryentry{lfalse}{
	name={$ \lfalse$},
	plural={\lfalse},% im Mathematikmodus
	description={%
		Eine Aussagenlogische Konstante: Falsch%
	}
}
\newglossaryentry{ltrue}{
	name={$ \ltrue$},
	plural={\ltrue},% im Mathematikmodus
	description={%
		Eine Aussagenlogische Konstante: Wahr%
	}
}
\newglossaryentry{subst}{
	name={$ \subst$},
	plural={\subst},% im Mathematikmodus
	description={%
		Substitution: ... substituiert durch ...
		(siehe die Definition in \subsectionname~\vref{sub:Basisregeln})%
	}
}

% Symbole für Mengen -----------------------------------------------------------

\newglossaryentry{gsN}{
	name={$ \gsN$},
	plural={\gsN},% im Mathematikmodus
	description={%
		Die Menge der natürlichen Zahlen ohne 0%
	}
}
\newglossaryentry{gsNo}{
	name={$ \gsNo$},
	plural={\gsNo},% im Mathematikmodus - erfolgt im falschen Zeichensatz!
	description={%
		Die Menge der natürlichen Zahlen einschließlich 0%
	}
}
\newglossaryentry{asA}{
	name={$ \asA$},
	plural={\asA},% im Mathematikmodus
	description={%
		Das Alphabet der aussagenlogischen Sprache%
	}
}
\newglossaryentry{asAx}{
	name={$ \asAx$},
	plural={\asAx},% im Mathematikmodus
	description={%
		Eine Teilmenge des Alphabets $\asA$ der aussagenlogischen Sprache%
	}
}
\newglossaryentry{asB}{
	name={$ \asB$},
	plural={\asB},% im Mathematikmodus
	description={%
		Die Menge der aussagenlogischen, binären Operatoren%
	}
}
\newglossaryentry{asC}{
	name={$ \asC$},
	plural={\asC},% im Mathematikmodus
	description={%
		Die Menge der aussagenlogischen Konstanten%
	}
}
\newglossaryentry{asF}{
	name={$ \asF$},
	plural={\asF},% im Mathematikmodus
	description={%
		Die Menge der aussagenlogischen Formeln mit Klammerung%
	}
}
\newglossaryentry{asFp}{
	name={$ \asFp$},
	plural={\asFp},% im Mathematikmodus
	description={%
		Die Menge der aussagenlogischen Formeln in polnischer Notation%
	}
}
\newglossaryentry{asFx}{
	name={$ \asFx$},
	plural={\asFx},% im Mathematikmodus
	description={%
		Eine Teilmenge der Menge $\asF$ der aussagenlogischen Formeln mit Klammerung%
	}
}
\newglossaryentry{asFxp}{
	name={$ \asFxp$},
	plural={\asFxp},% im Mathematikmodus
	description={%
		Eine Teilmenge der Menge $\asF$ der aussagenlogischen Formeln in polnischer Notation%
	}
}
\newglossaryentry{asJ}{
	name={$ \asJ$},
	plural={\asJ},% im Mathematikmodus
	description={%
		Die Menge der aussagenlogischen Operatoren%
	}
}
\newglossaryentry{asJx}{
	name={$ \asJx$},
	plural={\asJx},% im Mathematikmodus
	description={%
		Eine Teilmenge der Menge $\asJ$ der aussagenlogischen Operatoren
	}
}
\newglossaryentry{asM}{
	name={$ \asM$},
	plural={\asM},% im Mathematikmodus
	description={%
		Die Menge der \glsIdxPl{MetaoperatorV} und der mit Gleichheit verwandten Symbole%
	}
}
\newglossaryentry{asU}{
	name={$ \asU$},
	plural={\asU},% im Mathematikmodus
	description={%
		Die Menge der aussagenlogischen unären Operatoren%
	}
}
\newglossaryentry{asV}{
	name={$ \asV$},
	plural={\asV},% im Mathematikmodus
	description={%
		Die Menge der aussagenlogischen \glsIdxPl{atomareFormelA}%
	}
}

% Schlussregeln ----------------------------------------------------------------

\newcommand*{\tagAR}{AR}% Argument für \tag
\newglossaryentry{AR}{
	name={(AR)},
	plural={(\text{AR})},% im Mathematikmodus
	description={%
		\glsIdx{Anfangsregel}%
	}
}
\newcommand*{\tagMR}{MR}% Argument für \tag
\newglossaryentry{MR}{
	name={(MR)},
	plural={(\text{MR})},% im Mathematikmodus
	description={%
		\glsIdx{Monotonieregel}%
	}
}
\newcommand*{\tagSR}{SR}% Argument für \tag
\newglossaryentry{SR}{
	name={(SR)},
	plural={(\text{SR})},% im Mathematikmodus
	description={%
		\glsIdx{Schnittregel} (Modus ponens)%
	}
}
\newcommand*{\tagTR}{TR}% Argument für \tag
\newglossaryentry{TR}{
	name={(TR)},
	plural={(\text{TR})},% im Mathematikmodus
	description={%
		\glsIdx{Abtrennungsregel}%
	}
}
\newcommand*{\tagandB}{$\land$B}% Argument für \tag
\newglossaryentry{andB}{
	name={($ \land     $B)},
	plural={(\land\text{B})},% im Mathematikmodus
	description={%
		Beseitigung von \symqt{$\land$}%
	}
}
\newcommand*{\tagandE}{$\land$E}% Argument für \tag
\newglossaryentry{andE}{
	name={($ \land     $E)},
	plural={(\land\text{E})},% im Mathematikmodus
	description={%
		Einführung von \symqt{$\land$}%
	}
}
\newcommand*{\tagimpB}{$\limp$B}% Argument für \tag
\newglossaryentry{impB}{
	name={($ \limp     $B)},
	plural={(\limp\text{B})},% im Mathematikmodus
	description={%
		Beseitigung von \symqt{$\limp$}%
	}
}
\newcommand*{\tagimpE}{$\limp$E}% Argument für \tag
\newglossaryentry{impE}{
	name={($ \limp     $E)},
	plural={(\limp\text{E})},% im Mathematikmodus
	description={%
		Einführung von \symqt{$\limp$}%
	}
}
\newcommand*{\tagnota}{$\lnot$1}% Argument für \tag
\newglossaryentry{nota}{
	name={($ \lnot     $1)},
	plural={(\lnot\text{1})},% im Mathematikmodus
	description={%
		Einführung/Beseitigung von \symqt{$\lnot$} Teil 1%
	}
}
\newcommand*{\tagnotb}{$\lnot$2}% Argument für \tag
\newglossaryentry{notb}{
	name={($ \lnot     $2)},
	plural={(\lnot\text{2})},% im Mathematikmodus
	description={%
		Einführung/Beseitigung von \symqt{$\lnot$} Teil 2%
	}
}
\newcommand*{\tagnotc}{$\lnot$3}% Argument für \tag
\newglossaryentry{notc}{
	name={($ \lnot     $3)},
	plural={(\lnot\text{3})},% im Mathematikmodus
	description={%
		Beweistechnik \strqt{Indirekter Beweis}%
	}
}
\newcommand*{\tagnotd}{$\lnot$4}% Argument für \tag
\newglossaryentry{notd}{% statt "notE"
	name={($ \lnot     $4)},
	plural={(\lnot\text{4})},% im Mathematikmodus
	description={%
		Reductio ad absurdum (indirekter Beweis)%
	}
}
%%%\newglossaryentry{orB}{
%%%	name={($ \lor     $B)},
%%%	plural={(\lor\text{B})},% im Mathematikmodus
%%%	description={%
%%%		Beseitigung von \symqt{$\lor$}%
%%%	}
%%%}
%%%\newglossaryentry{orE}{
%%%	name={($ \lor     $E)},
%%%	plural={(\lor\text{E})},% im Mathematikmodus
%%%	description={%
%%%		Einführung von \symqt{$\lor$}%
%%%	}
%%%}

% keine Symbole mehr -----------------------------------------------------------

\newglossaryentry{ableitbar}{
	name={ableitbar},
	plural={ableitbare},
	description={%
		Wenn sich eine Formel $\beta$ aus einer Formel $\alpha$ mittels zulässiger Transaktionen ableiten lässt, heißt $\beta$ ableitbar aus $\alpha$.
		Sprechweise: \forqt{$\alpha$ ableitbar $\beta$}.
		Eine oder beide Formeln $\alpha$ \textbzw $\beta$ dürfen dabei durch Formelmengen ersetzt werden.
		(\seealso\ \symqt{\glsIdx{abltb}} und \glsIdx{Ableitungsrelation})
		-- Synonym: \glsIdx{beweisbar}%
	}
}
\newglossaryentry{Ableitungsrelation}{
	name={Ableitungsrelation},
	plural={Ableitungsrelationen},
	description={%
		Die Relation \symqt{\glsIdx{abltb}}%
	}
}
\newglossaryentry{Abtrennungsregel}{
	name={Abtrennungsregel},
	plural={Abtrennungsregeln},
	description={%
		Eine \emph{\glsIdx{Schlussregel}} - siehe~\glsIdx{TR}%
	}
}
\newglossaryentry{Anfangsregel}{
	name={Anfangsregel},
	plural={Anfangsregeln},
	description={%
		Eine \emph{\glsIdx{Schlussregel}} um beginnen zu können - siehe~\glsIdx{AR}%
	}
}
\newcommand*{\ASBA}{\glsIdx{ASBA}}
\newacronym{ASBA}{ASBA}{
	Programmsystem, das \textbf{A}xiome, \textbf{S}ätze, \textbf{B}eweise und \textbf{A}uswertungen behandeln kann%
}
\newglossaryentry{atomareFormelA}{
	name={atomare Formel},     % eine ...
	plural={atomaren Formeln}, % alle ...
	description={%
		Eine Formel, die sich nicht weiter zerlegen lässt%
	}%
}
\newglossaryentry{Ausgabeschema}{
	name={Ausgabeschema},
	plural={Ausgabeschemata},
	description={%
		Ein Schema, mit dem bestimmte mathematische Objekte ausgegeben werden sollen%
	}
}
\newglossaryentry{Aussage}{
	name={Aussage},
	plural={Aussagen},
	description={%
		Eine Aussage in natürlicher Sprache oder als Formel, die einen Wahrheitswert liefert%
	}
}
\newglossaryentry{Aussagenlogik}{
	name={Aussagenlogik},
	description={%
		\seename\ \sectionname~\vref{sec:Aussagenlogik}%
	}
}
\newglossaryentry{Axiom}{
	name={Axiom},
	plural={Axiome},
	description={%
		Eine Formel, die unbewiesen als wahr angesehen wird%
	}
}
\newglossaryentry{Basisregel}{
	name={Basisregel},
	plural={Basisregeln},
	description={%
		Eine \emph{\glsIdx{Schlussregel}}, die nicht mehr auf andere zurückgeführt wird%
	}
}
\newglossaryentry{Beweis}{
	name={Beweis},
	plural={Beweise},
	description={%
		Eine zulässige Ableitung von Folgerungen aus gegebenen Voraussetzungen%
	}
}
\newglossaryentry{beweisbar}{
	name={beweisbar},
	plural={beweisbare},
	description={%
		Synonym zu \emph{ableitbar}%
	}
}

\newglossaryentry{Beweisschritt}{
	name={Beweisschritt},
	plural={Beweisschritte},
	description={%
		Eine Vorschrift, wie aus vorgegebenen Aussagen eine weitere folgt%
	}
}
\newglossaryentry{BoolscheSignatur}{
	name={Boolsche Signatur},
	plural={Boolschen Signatur},% Dativ
	description={%
		Die \glsIdx{logischeSignaturV} $\{\lnot, \land, \lor\}$%
	}
}
\newglossaryentry{Fachbegriff}{
	name={Fachbegriff},
	plural={Fachbegriffe},
	description={%
		Ein Name für einen mathematischen Begriff%
	}
}
\newglossaryentry{Fachgebiet}{
	name={Fachgebiet},
	plural={Fachgebiete},
	description={%
		Ein Teil der Mathematik mit einer zugehörigen Basis aus Axiomen, Sätzen und spezifischen Fachbegriffen und Darstellungen%
	}
}
\newglossaryentry{FormalelementV}{
	name={formales Element},   % ein   ...
	plural={formale Elemente}, % viele ...
	description={%
		Ein mathematisches Element in formaler Schreibweise.
		Bis auf wenige Aussagen kommen darin \emph{\glsIdxPl{MetaausdruckV}} nicht mehr vor%
	}%
}
\newglossaryentry{intEigenschaftA}{
	name={interessierende Eigenschaft},      % eine ...
	plural={interessierenden Eigenschaften}, % alle ...
	description={%
		Solche Eigenschaften von Ausdrücken, die im aktuellen Zusammenhang von Interesse sind.%
	}%
}
\newglossaryentry{logischeSignaturV}{
	name={logische Signatur},     % eine  ...
	plural={logische Signaturen}, % viele ...
	description={%
		Eine in \emph{\glsIdx{Metasprache}} verfasste Aussage, die auch zusammengesetzt sein kann%
	}%
}
\newglossaryentry{MetaausdruckV}{
	name={metasprachlicher Ausdruck},   % ein   ...
	plural={metasprachliche Ausdrücke}, % viele ...
	description={%
		Eine in normaler Sprache verfasste Aussage, die auch zusammengesetzt sein kann%
	}%
}
\newglossaryentry{MetaaussageV}{
	name={metasprachliche Aussage},    % eine   ...
	plural={metasprachliche Aussagen}, % viele ...
	description={%
		Eine in \emph{\glsIdx{Metasprache}} verfasste Aussage, die auch zusammengesetzt sein kann%
	}%
}
\newglossaryentry{MetaoperatorV}{
	name={metasprachlicher Operator},    % ein   ...
	plural={metasprachliche Operatoren}, % viele ...
	description={%
		Ein Operator, dessen Operanden \emph{\glsIdxPl{MetaausdruckV}} sind%
	}%
}
\newglossaryentry{Metasprache}{
	name={Metasprache},
	plural={Metasprachen},
	description={%
		Eine Sprache, in der Aussagen über Elemente einer anderen Sprache getroffen werden können%
	}%
}
\newglossaryentry{Monotonieregel}{
	name={Monotonieregel},
	plural={Monotonieregeln},
	description={%
		Eine \emph{\glsIdx{Schlussregel}} - siehe~\glsIdx{MR}%
	}
}
\newglossaryentry{Praedikat}{
	name={Prädikat},
	plural={Prädikate},
	description={%
		Ein Element der \emph{\glsIdx{Praedikatenlogik}} (\see \sectionname~\vref{sec:Praedikatenlogik}).
		\textZB kann man eine \forqt{$Gruppe$} als ein zweistelliges Prädikat \forqt{$Gruppe(G,+)$} definieren, in dem $G$ eine Menge und \symqt{$+$} eine Operation, \textdh eine (zweistellige) Funktion \forqt{$+: G \times G \rightarrow G$} ist, so dass die Gruppenaxiome erfüllt sind%
	}
}
\newglossaryentry{Praedikatenlogik}{
	name={Prädikatenlogik},
	description={%
		\seename\ \sectionname~\vref{sec:Praedikatenlogik}%
	}
}
\newglossaryentry{Satz}{
	name={Satz},
	plural={Sätze},
	description={%
		Eine mathematische Aussage, dass eine bestimmte Folgerung aus gegebenen Voraussetzungen abgeleitet werden kann%
	}
}
\newglossaryentry{Schlussregel}{
	name={Schlussregel},
	plural={Schlussregeln},
	description={%
		Eine Regel für eine (zulässige) Umwandlung von Formeln%
	}
}
\newglossaryentry{Schnittregel}{
	name={Schnittregel},
	plural={Schnittregeln},
	description={%
		Eine \emph{\glsIdx{Schlussregel}} - siehe~\glsIdx{SR}%
	}
}
\newglossaryentry{vergleichbar}{
	name={vergleichbar},
	plural={vergleichbare},
	description={
		Zwei \emph{\glsIdxPl{MetaausdruckV}} \textbzw \emph{\glsIdxPl{FormalelementV}} heißen -- auf eine bestimmte Art -- vergleichbar, wenn sie auf diese Art (\textzB als Zeichenketten oder als vergleichbare Ergebnisse von Formeln) verglichen werden können.
		Die Art muss implizit bekannt oder explizit angegeben sein.
		Meistens genügt es zu wissen, was für \glsIdxPl{MetaausdruckV} \textbzw \glsIdxPl{FormalelementV} es sind.
		Sie müssen dann nur von derselben Art sein%
	}%
}
\newglossaryentry{Wahrheitswert}{
	name={Wahrheitswert},
	plural={Wahrheitswerte},
	description={%
		Wahrheitswerte sind die Werte \strqt{wahr} und \strqt{falsch}, oft auch als \strqt{true} und \strqt{false} oder einfach \charqt{1} und \charqt{0} bezeichnet%
	}
}
\newglossaryentry{zulaessigeTransformation}{
	name={zulässige Transformation},
	plural={zulässige Transformation},
	description={%
		Die Anwendung einer \glsIdx{Basisregel} oder einer davon abgeleiteten sonstigen \glsIdx{Schlussregel}%
	}
}

%%############################################################################%%
%%                                                                            %%
%% Datei:  ASBA-Vorspann-Logik.tex                                            %%
%% Inhalt: Vorspann Logik für ASBA                                            %%
%%                                                                            %%
%% Copyright (C) 2017  Winfried Teschers                                      %%
%%                                                                            %%
%% This program is free software: you can redistribute it and/or modify       %%
%% it under the terms of the GNU Affero General Public License as published   %%
%% by the Free Software Foundation, either version 3 of the License, or       %%
%% (at your option) any later version.                                        %%
%%                                                                            %%
%% This program is distributed in the hope that it will be useful,            %%
%% but WITHOUT ANY WARRANTY; without even the implied warranty of             %%
%% MERCHANTABILITY or FITNESS FOR A PARTICULAR PURPOSE.  See the              %%
%% GNU Affero General Public License for more details.                        %%
%%                                                                            %%
%% You should have received a copy of the GNU Affero General Public License   %%
%% along with this program.  If not, see <http://www.gnu.org/licenses/>.      %%
%%                                                                            %%
%% Dr. Winfried Teschers                                                      %%
%% Anton-Günther-Straße 26c                                                   %%
%% 91083 Baiersdorf                                                           %%
%% Germany                                                                    %%
%%                                                                            %%
%% e-mail: winfried.teschers@t-online.de                                      %%
%%                                                                            %%
%%############################################################################%%

% !TeX root = ASBA.tex
% !TeX encoding = UTF-8
% !TeX spellcheck = de_DE

% Glossareinträge werden in "ASBA-Vorspann-Glossar" definiert.
% Elemente, die in anderen Dateien als "ASBA-Logik.tex" verwendet werden, werden in "ASBA-Vorspann.tex" definiert.
% Namensbestandteile mit besonderer Bedeutung:
%   Bezeichnungen:
%     <Name>Set     = Menge    von <Name>n (keine bis alle)
%     <Name>Rel     = Relation von <Name>n
%     <Name>Tup     = Tupel    von <Name>n
%     All<Name>     = Menge  aller <Name>n
%   Ergebnisse von Operationen auf Mengen:
%     Pot<Menge>    = Menge    der Teilmengen                   von <Name>
%     Rel<Menge>    = Menge    der binären Relationen           auf <Name>
%     Relf<Menge>   = Menge    der endlichen binären Relationen auf <Name>
%     Tup<Menge>    = Menge    der Tupel                        auf <Name>
%   Sonstiges:
%     <Relation>Bck = Umkehrrelation
%     <Operator>Eq  = Operator oder Gleich
%     <Operator>N   = negierter Operator
%     <Oper.>BckEqN = Kombination in dieser Reihenfolge
%     \Tag<Tag>     =
%
% Folgende Makros sind alles "eigene"
%   \Ltr...    - nur Zeichen für (die Menge)     ...       (für       Textmodus)
%   \Str...    - nur Text    für (die Operation) ...       (für       Textmodus)
%   \StrMtsIdx...    - Text für einen Index für        ...       (für Mathematikmodus)
%   \Raw...    - Symbol/Text ohne Verweis;erfordert bei Symbolen Mathematikmodus
%   \...Bsp... - Beispielsymbol für Formel- oder Metasprache (f.Mathematikmodus)
%   \Bsp...    -                          mit Verweis ins Symbolverzeichnis
%   \...Mts... - Symbol der Metasprache                    (für Mathematikmodus)
%   \Mts...    -                          mit Verweis ins Symbolverzeichnis
%   \...Ojk... - Symbol der Formelsprache                  (für Mathematikmodus)
%   \Ojk...    -                          mit Verweis ins Symbolverzeichnis
%   \...Txt... - individuelle Bezeichnung                  (für       Textmodus)
%   \Txt...    -                          mit Verweis in  Glossar und Index
%
% Weitere "eigene" Kombinationen: OpU, OpB, Bck, Idx, Ltr, Txt

% Beispielsymbole für Operationen und Relationen ===============================
% \RawBsp*; OpU=unär, OpB=binär, Rel=Relation, Bck=Umkehr-; N=nicht, Eq=gleich
\newcommand*{\RawBspOpU}      {\mathbin{\circleddash}}
\newcommand*{\RawBspOpB}      {\mathbin{\circledast}}
\newcommand*{\RawBspRel}      {\mathrel{\prec}}
\newcommand*{\RawBspRelEq}    {\mathrel{\preceq}}
\newcommand*{\RawBspRelBck}   {\mathrel{\succ}}
\newcommand*{\RawBspRelBckEq} {\mathrel{\succeq}}
\newcommand*{\RawBspRelN}     {\mathrel{\nprec}}
\newcommand*{\RawBspRelEqN}   {\mathrel{\npreceq}}
\newcommand*{\RawBspRelBckN}  {\mathrel{\nsucc}}
\newcommand*{\RawBspRelBckEqN}{\mathrel{\nsucceq}}

% Metasymbole ==================================================================
% \RawMts*
% Metaoperationen, -relationen u.a. (für Aussagen) -----------------------------
\newcommand*{\RawMtsNot}     {\mathbin{\thicksim}}%               ... gilt nicht
\newcommand*{\RawMtsAnd}     {\mathbin{\&}}%                        ... und  ...
\newcommand*{\RawMtsOr}      {\mathbin{\parallel}}%                 ... oder ...
\newcommand*{\RawMtsImp}     {\mathrel{\Rightarrow}}% von ... folgt          ...
\newcommand*{\RawMtsRep}     {\mathrel{\Leftarrow}}%      ... folgt von      ...
\newcommand*{\RawMtsEquiv}   {\mathrel{\Leftrightarrow}}% .. genau dann, wenn ..
\newcommand*{\RawMtsEq}      {\mathrel{=\mkern-8mu=}}%           ...  gleich ...
\newcommand*{\RawMtsEqN}     {\mathrel{=\mkern-15mu/\mkern-15mu=}}% . ungleich .
\newcommand*{\RawMtsAequiv}  {\mathrel{\equiv}}%     ...       äquivalent zu ...
\newcommand*{\RawMtsAequivN} {\mathrel{\nequiv}}%    ... nicht äquivalent zu ...
\newcommand*{\RawMtsDefEquiv}{\mathrel{:\mkern-2mu\RawMtsEquiv}}% def.gemäß -"-
\newcommand*{\RawMtsDefEq}   {\mathrel{:\mkern-2mu\RawMtsEq}}% def.gemäß gleich
\newcommand*{\RawMtsUnd}     {\mathbin{\mid}}%nur in Schlussregeln: ... und  ...
\newcommand*{\RawMtsDerive}  {\mathrel{\vdash}}%          ... ableitbar      ...
\newcommand*{\RawMtsSwap}    {\mathbin{\leftrightarrows}}% .. vertauscht mit ...
\newcommand*{\RawMtsSubst}   {\mathbin{\leftarrowtail}}%.. substituiert durch ..
% Elementrelationen (für Elemente und Mengen) ----------------------------------
\newcommand*{\RawMtsIn}       {\in}%        ist  Element aus (der Menge)
\newcommand*{\RawMtsNi}       {\ni}%       (die Menge) enthält nicht das Element
\newcommand*{\RawMtsInN}      {\notin}%     ist  Element aus (der Menge)
\newcommand*{\RawMtsNiN}      {\notni}%    (die Menge) enthält nicht das Element
\newcommand*{\RawMtsSetSep}   {\mid}%  Der Trennstrich in einer Mengendefinition
% Mengenrelationen (für Mengen) ------------------------------------------------
\newcommand*{\RawMtsSubset}   {\subset}%    ist        echte  Teilmenge von
\newcommand*{\RawMtsSubsetEq} {\subseteq}%  ist (gleich oder) Teilmenge von
\newcommand*{\RawMtsSubsetN}  {\nsubset}%   ist  keine echte  Teilmenge von
\newcommand*{\RawMtsSubsetEqN}{\nsubseteq}% ist  keine        Teilmenge von
\newcommand*{\RawMtsSupset}   {\supset}%    ist        echte  Obermenge von
\newcommand*{\RawMtsSupsetEq} {\supseteq}%  ist (gleich oder) Obermenge von
\newcommand*{\RawMtsSupsetN}  {\nsupset}%   ist  keine echte  Obermenge von
\newcommand*{\RawMtsSupsetEqN}{\nsupseteq}% ist  keine        Obermenge von
% Mengenoperationen (für Mengen) -----------------------------------------------
\newcommand*{\RawMtsCap}      {\cap}%       Durchschnitt          von Mengen
\newcommand*{\RawMtsCup}      {\cup}%       Vereinigung           von Mengen
\newcommand*{\RawMtsSetminus} {\setminus}%  Differenz             von Mengen
\newcommand*{\RawMtsTimes}    {\times}%     karthesisches Produkt von Mengen
\newcommand*{\RawMtsEmptyset} {\emptyset}%  die leere Menge
% Komponentenrelationen (für Komponenten und Folgen) ---------------------------
\newcommand*{\RawMtsSeqIn}    {\sqsubset\mkern-19mu-}%  ist  Komponente (der Folge)
\newcommand*{\RawMtsSeqNi}    {\sqsupset\mkern-19mu-}%  (die Folge) enthält nicht das Symbol
\newcommand*{\RawMtsSeqInN}   {\nsqsubset\mkern-19mu-}% ist  Komponente (der Folge)
\newcommand*{\RawMtsSeqNiN}   {\nsqsupset\mkern-19mu-}% (die Folge) enthält nicht das Symbol
% Folgenoperationen (für Folgen) -----------------------------------------------
\newcommand*{\RawMtsCat}      {\sqcup}%       Verkettung           von Folgen
% Folgenrelationen (für Folgen) ------------------------------------------------
\newcommand*{\RawMtsSubseq}   {\sqsubset}%    ist        echte  Teilmenge von
\newcommand*{\RawMtsSubseqEq} {\sqsubseteq}%  ist (gleich oder) Teilmenge von
\newcommand*{\RawMtsSubseqN}  {\nsqsubset}%   ist  keine echte  Teilmenge von
\newcommand*{\RawMtsSubseqEqN}{\nsqsubseteq}% ist  keine        Teilmenge von
\newcommand*{\RawMtsSupseq}   {\sqsupset}%    ist        echte  Obermenge von
\newcommand*{\RawMtsSupseqEq} {\sqsupseteq}%  ist (gleich oder) Obermenge von
\newcommand*{\RawMtsSupseqN}  {\nsqsupset}%   ist  keine echte  Obermenge von
\newcommand*{\RawMtsSupseqEqN}{\nsqsupseteq}% ist  keine        Obermenge von

% Text-, Meta- und Objekt-Wahrheitswerte
\newcommand*{\RawTxtFalse}    {\emph{\StrTxtFalse}}%                      (Text)
\newcommand*{\RawTxtTrue}     {\emph{\StrTxtTrue}}%                       (Text)
\newcommand*{\RawMtsFalse}    {\mathord{\mathrm{\StrMtsFalse}}}%F - falsch (Sym)
\newcommand*{\RawMtsTrue}     {\mathord{\mathrm{\StrMtsTrue}}}% W - wahr   (Sym)
\newcommand*{\RawOjkFalse}    {\mathord{\bot}}%                         (Symbol)
\newcommand*{\RawOjkTrue}     {\mathord{\top}}%                         (Symbol)

% Definitionen für die Tabelle der Junktoren -----------------------------------
% \RawOjk*
% Wahrheitswert von A                            W W F F
% Wahrheitswert von B                            W F W F
% unäre Operationen ------------------------------------------------------------
\newcommand*{\RawOjkNot}      {\lnot}%           F W - - - nicht A
% binäre Operationen -----------------------------------------------------------
\newcommand*{\RawOjkAnd}      {\land}%           W F F F - A und B
\newcommand*{\RawOjkOr}       {\lor}%            W W W F - A oder B
\newcommand*{\RawOjkImp}      {\rightarrow}%     W F W W - von A folgt B
\newcommand*{\RawOjkRep}      {\leftarrow}%      W W F W - A folgt von B
\newcommand*{\RawOjkEquiv}    {\leftrightarrow}% W F F W - A genau dann wenn B
\newcommand*{\RawOjkNand}     {\uparrow}%        F W W W - nicht   (A und  B)
\newcommand*{\RawOjkNor}      {\downarrow}%      F F F W - weder    A noch B
\newcommand*{\RawOjkXor}      {\dot\lor}%        F W W F - entweder A oder B

% außerhalb der Tabelle --------------------------------------------------------
\newcommand*{\RawOjkEq}       {=}%          Gleichheit   in Formeln
\newcommand*{\RawOjkEqN}      {\ne}%        Ungleichheit in Formeln
% Quantoren --------------------------------------------------------------------
\newcommand*{\RawMtsForall}   {\forall}%     für alle          <x>         gilt:
\newcommand*{\RawMtsExists}   {\exists}%     es gibt       ein <x> für das gilt:
\newcommand*{\RawMtsExione}   {\exists!}%    es gibt genau ein <x> für das gilt:
\newcommand*{\RawOjkForall}   {\bigwedge}%   für alle          <x>         gilt:
\newcommand*{\RawOjkExists}   {\bigvee}%     es gibt       ein <x> für das gilt:
\newcommand*{\RawOjkExione}   {\dot\bigvee}% es gibt genau ein <x> für das gilt:

% weitere Symbole --------------------------------------------------------------
\newcommand*{\RawMtsFktSep}   {:}%                 f \MtsFktSep A \MtsFktArrow B
\newcommand*{\RawMtsFktArrow} {\rightarrow}

% Neue Metaoperationen ---------------------------------------------------------
\DeclareMathOperator{\RawMtsValue}  {\StrMtsValue}% Wert  einer      Formel
\DeclareMathOperator{\RawMtsGraph}  {\StrMtsGraph}% Graph einer      Funktion/Relation
\DeclareMathOperator{\RawMtsTraeger}{\StrMtsTraeger}% Trägermenge einer       Relation
\DeclareMathOperator{\RawMtsStel}   {\StrMtsStel}% Stelligkeit einer Funktion/Relation
\DeclareMathOperator{\RawMtsStelF}  {\StrMtsStel_f}% Stelligkeit für [F]unktionen
\DeclareMathOperator{\RawMtsStelR}  {\StrMtsStel_r}% Stelligkeit für [R]elationen
\DeclareMathOperator{\RawMtsQb}     {\StrMtsQb}% Quellbereich einer partiellen Funktion
\DeclareMathOperator{\RawMtsDb}     {\StrMtsDb}% Definitionsbereich einer      Funktion
\DeclareMathOperator{\RawMtsZb}     {\StrMtsZb}% Zielbereich        einer      Funktion
\DeclareMathOperator{\RawMtsWb}     {\StrMtsWb}% Wertebereich       einer      Funktion
\DeclareMathOperator{\RawMtsLen}    {\StrMtsLen}% Länge            eines/r Tupels/Folge
\DeclareMathOperator{\RawMtsSet}    {\StrMtsSet}% Komponentenmenge eines/r Tupels/Folge

% Schriftarten
\newcommand*{\varFt}[1]  {\mathbf{#1}}% Variable aus Alphabet   (Kleinbuchstabe)
\newcommand*{\DrvFt}[1]  {\mathbf{#1}}% Mengen von Ableitungen   (Großbuchstabe)
\newcommand*{\drvFt}[1]  {\mathbf{#1}}% ein Element davon (Klein-/Großbuchstabe)
\newcommand*{\IdxFt}[1]  {\mathrm{#1}}% fester Index      (Klein-/Großbuchstabe)
\newcommand*{\SetFt}[1] {\mathcal{#1}}% vorgegebene Menge        (Großbuchstabe)
\newcommand*{\ElmFt}[1]          {#1}%  ein Element davon        (Großbuchstabe)
\newcommand*{\SetOp}[1]{\mathfrak{#1}}% Mengenoperation                   (Text)

% spezielle Indizes (rechts oben)
\newcommand*{\LtrMtsIdxLogisch} {A}%    die Logik betreffend
% spezielle Indizes für Teilmengen (rechts unten)
\newcommand*{\StrMtsIdxBin}     {b}%    binär
\newcommand*{\StrMtsIdxCon}     {c}%    constant; konstant
\newcommand*{\StrMtsIdxUna}     {u}%    unär
\newcommand*{\StrMtsIdxAnd}     {and}%  Signatur not, and
\newcommand*{\StrMtsIdxBool}    {bool}% Signatur not, and, or
\newcommand*{\StrMtsIdxImp}     {imp}%  Signatur not, imp
\newcommand*{\StrMtsIdxNand}    {nand}% Signatur      nand
\newcommand*{\StrMtsIdxNor}     {nor}%  Signatur      nor
\newcommand*{\StrMtsIdxOr}      {or}%   Signatur not, or
\newcommand*{\StrMtsIdxRep}     {rep}%  Signatur not, rep
% spezielle Indizes mit Schriftart
\newcommand*{\RawMtsIdxPolnisch}{\IdxFt{\LtrMtsIdxPolnisch}}% ^
\newcommand*{\RawMtsIdxLogisch} {\IdxFt{\LtrMtsIdxLogisch}}%  ^
\newcommand*{\RawMtsIdxEndlich} {\IdxFt{\LtrMtsIdxEndlich}}%  _
\newcommand*{\RawMtsIdxGraph}   {\IdxFt{\LtrMtsIdxGraph}}%    _
\newcommand*{\RawMtsIdxBin}     {\IdxFt{\StrMtsIdxBin}}
\newcommand*{\RawMtsIdxCon}     {\IdxFt{\StrMtsIdxCon}}
\newcommand*{\RawMtsIdxUna}     {\IdxFt{\StrMtsIdxUna}}
\newcommand*{\RawMtsIdxAnd}     {\IdxFt{\StrMtsIdxAnd}}
\newcommand*{\RawMtsIdxBool}    {\IdxFt{\StrMtsIdxBool}}
\newcommand*{\RawMtsIdxImp}     {\IdxFt{\StrMtsIdxImp}}
\newcommand*{\RawMtsIdxNand}    {\IdxFt{\StrMtsIdxNand}}
\newcommand*{\RawMtsIdxNor}     {\IdxFt{\StrMtsIdxNor}}
\newcommand*{\RawMtsIdxOr}      {\IdxFt{\StrMtsIdxOr}}
\newcommand*{\RawMtsIdxRep}     {\IdxFt{\StrMtsIdxRep}}

% Indexoperationenen
\newcommand*{\links} [1] {#1^{\scriptscriptstyle <}}% linkes  Element vom Paar
\newcommand*{\rechts}[1] {#1^{\scriptscriptstyle >}}% rechtes Element vom Paar

% abgeleitete Mengen - ohne Verweis ins Glossar
\newcommand*{\RawMtsFol} {\SetOp{\LtrMtsFol}}%  Menge der Folgen             auf
\newcommand*{\RawMtsFolf}{\RawMtsFol_{\RawMtsIdxEndlich}}% ... nur die endlichen Folgen
\newcommand*{\RawMtsTup} {\SetOp{\LtrMtsTup}}%  Menge der Tupel              auf
\newcommand*{\RawMtsPot} {\SetOp{\LtrMtsPot}}%  Menge der Teilmengen         von
\newcommand*{\RawMtsPotf}{\RawMtsPot_{\RawMtsIdxEndlich}}%... nur die endlichen Teilmengen
\newcommand*{\RawMtsRel} {\SetOp{\LtrMtsRel}}%  Menge der binären Relationen auf
\newcommand*{\RawMtsRelf}{\RawMtsRel_{\RawMtsIdxEndlich}}%... nur die endlichen Relationen

% natürliche Zahlen u.a. - ohne Verweis ins Glossar
% alternativ: '{\fam5' statt '\mathbb{'
\newcommand*{\RawMtsIN}   {{\fam5\LtrMtsIN}}%  Menge der natürlichen Zahlen ohne 0
%\newcommand*{\RawMtsIN}{\mathbb{\LtrMtsIN}}%  Menge der natürlichen Zahlen ohne 0
\newcommand*{\RawMtsINo}        {\RawMtsIN_0}% Menge der natürlichen Zahlen mit  0
\newcommand*{\RawMtsMo}               {M^0}
\newcommand*{\RawMtsMn}               {M^n}

% weitere Mengen
\newcommand*{\RawMtsUniversum}        {\SetOp{\LtrMtsUniversum}}%  Diskursuniversum
\newcommand*{\RawMtsSprache}          {\SetFt{\LtrMtsSprache}}% Formel-Sprache
\newcommand*{\RawMtsRelAllDerive}     {\ensuremath{\RawMtsRel(\RawMtsPot(\RawMtsSprache))}}%   R(P(L))
% ... - mit Verweis ins Glossar
\newcommand*{\MtsPotSprache}          {\ensuremath{\MtsPot(\MtsSprache)}}%        P(L)
\newcommand*{\MtsPotfSprache}         {\ensuremath{\MtsPotf(\MtsSprache)}}%      Pe(L)
\newcommand*{\MtsAllDerive}           {\ensuremath{\MtsPotSprache^2}}%            P(L)^2
\newcommand*{\MtsPotAllDerive}        {\ensuremath{\MtsPot(\MtsAllDerive)}}%    P(P(L)^2)
\newcommand*{\MtsRelAllDerive}        {\ensuremath{\MtsRel(\MtsPotSprache)}}%   R(P(L))
\newcommand*{\MtsAllSchlussregel}     {\ensuremath{\MtsPotAllDerive^2}}%        P(P(L)^2)^2
\newcommand*{\MtsRelSchlussregel}     {\ensuremath{\MtsRel(\MtsRelAllDerive)}}% R(P(L)^2)
\newcommand*{\MtsPotfAllDerive}       {\ensuremath{\MtsPotf(\MtsAllDerive)}}%  Pe(P(L)^2)
\newcommand*{\MtsRelfAllDerive}       {\ensuremath{\MtsRelf(\MtsPotSprache)}}% Re(P(L))

% Elemente und Mengen für Beweise - ohne Verweis ins Glossar
\newcommand*{\RawMtsPraemisse}        {\drvFt{\LtrMtsPraemisse}}%         eine      Prämisse
\newcommand*{\RawMtsPraemisseSet}     {\SetFt{\LtrMtsPraemisseSet}}%      Menge der Prämissen
\newcommand*{\RawMtsPraemisseRel}     {\RawMtsDerive_{\RawMtsPraemisseSet}}%... als Relation
\newcommand*{\RawMtsKonklusion}       {\drvFt{\LtrMtsKonklusion}}%        eine      Konklusion
\newcommand*{\RawMtsKonklusionSet}    {\SetFt{\LtrMtsKonklusionSet}}%     Menge der Konklusionen
\newcommand*{\RawMtsKonklusionRel}    {\RawMtsDerive_{\RawMtsKonklusionSet}}%.. als Relation
\newcommand*{\RawMtsErgebnis}         {\drvFt{\LtrMtsErgebnis}}%          ein       Ergebnis
\newcommand*{\RawMtsErgebnisSet}      {\SetFt{\LtrMtsErgebnisSet}}%       Menge von Ergebnissen
\newcommand*{\RawMtsErgebnisRel}      {\RawMtsDerive_{\RawMtsErgebnisSet}}% ... als Relation
\newcommand*{\RawMtsBeweisschritt}    {\ElmFt{\LtrMtsBeweisschritt}}%     ein       Beweisschritt
\newcommand*{\RawMtsBeweisschrittTup} {\vec{\RawMtsBeweisschritt}}%       Folge der Beweisschritte
\newcommand*{\RawMtsBeweisschrittSet} {\SetFt{\LtrMtsBeweisschrittSet}}%  Menge der Beweisschritte
\newcommand*{\RawMtsTransformation}   {\ElmFt{\LtrMtsTransformation}}%    eine      Transformation
\newcommand*{\RawMtsTransformationTup}{\SetFt{\LtrMtsTransformation}}%    Folge von Transformationen
\newcommand*{\RawMtsSchlussregel}     {\ElmFt{\LtrMtsSchlussregel}}%      eine      Schlussregel
\newcommand*{\RawMtsSchlussregelSet}  {\SetFt{\RawMtsSchlussregel}}%      Menge von Schlussregeln
\newcommand*{\RawMtsErsetzung}        {\ElmFt{\LtrMtsErsetzung}}%         eine      Ersetzung
\newcommand*{\RawMtsErsetzungSet}     {\SetFt{\LtrMtsErsetzung}}%         Menge von Ersetzungen
\newcommand*{\RawMtsAxiom}            {\ElmFt{\LtrMtsAxiom}}%             ein       Axiom
\newcommand*{\RawMtsAxiomSet}         {\SetFt{\LtrMtsAxiom}}%             Menge von Axiomen

% Mengen der Aussagenlogik - ohne Verweis ins Glossar
\newcommand*{\RawOjkvar} {\varFt{\LtrOjkvar}}% Variablensymbol
\newcommand*{\RawOjkVar} {\SetFt{\LtrOjkVar}}% Menge der Variablensymbole
\newcommand*{\RawOjkABC} {\SetFt{\LtrOjkABC}}% Menge der Buchstaben (Alphabet) der aussagenlogischen Sprache
\newcommand*{\RawOjkJun} {\SetFt{\LtrOjkJun}}% Menge der Junktoren
\newcommand*{\RawOjkCon} {\RawOjkJun_{\RawMtsIdxCon}}% Menge der         Konstantensymbole
\newcommand*{\RawOjkUna} {\RawOjkJun_{\RawMtsIdxUna}}% Menge der unären  Operationssymbole
\newcommand*{\RawOjkBin} {\RawOjkJun_{\RawMtsIdxBin}}% Menge der binären Operationssymbole
\newcommand*{\RawOjkFor} {\SetFt{\LtrOjkFor}^{\RawMtsIdxLogisch}}% Menge der aussagenlogischen Formeln
\newcommand*{\RawOjkForp}{\SetFt{\LtrOjkFor}^{\RawMtsIdxLogisch\RawMtsIdxPolnisch}}% ...in Polnischer Notation

% Prädikate und praedikatähnliche Makros =======================================

\newcommand*{\FunktionDef}[3]{\ensuremath{   #1    \MtsFktSep #2 \MtsFktArrow #3 }}
\newcommand*   {\MengeDef}[2]{\ensuremath{\{ #1    \MtsSetSep #2               \}}}% mit Verweis ins Glossar
\newcommand*{\RawMengeDef}[2]{\ensuremath{\{ #1 \RawMtsSetSep #2               \}}}% ohne Verweis ins Glossar

% sonstige Makro für den Mathematiksatz ########################################

\mathtoolsset{showonlyrefs,showmanualtags}% Nur mit \ref referenzierte Gleichungen, aber alle manuellen Tags

% Gleichung im Rahmen - siehe Rautenberg Seite 389
%%% #1=Rahmenfarbe, #2=Hintergrundfarbe, #3=mathematische Formel, #4=Marke
%%%\makeatletter
%%%\def\myMathBox    {@ifnextchar[{\my@MBoxi}     {\my@MBoxii[black]}}
%%%\def\my@MBoxi [#1]{@ifnextchar[{\my@MBoxii[#1]}{\my@MBoxii[white]}}
%%%\def\my@MBoxii[#1][#2]#3#4{%
%%%	\par
%%%	\noindent\fcolorbox{#1}{#2}{%
%%%		\parbox{\linewidth-1.5\labelwidth-2\fboxrule-2\fboxsep{#3}}%
%%%	}%
%%%	\parbox{1.5\labelwidth}{%
%%%		\begin{eqnarray}\label{#4}\end{eqnarray}%
%%%	}
%%%	\par
%%%}
%%%\makeatother

%%############################################################################%%
%%                                                                            %%
%% Datei:  ASBA-Vorspann-Glossary.tex                                         %%
%% Inhalt: Vorspann Glossareinträge für ASBA                                  %%
%%                                                                            %%
%% Copyright (C) 2017  Winfried Teschers                                      %%
%%                                                                            %%
%% This program is free software: you can redistribute it and/or modify       %%
%% it under the terms of the GNU Affero General Public License as published   %%
%% by the Free Software Foundation, either version 3 of the License, or       %%
%% (at your option) any later version.                                        %%
%%                                                                            %%
%% This program is distributed in the hope that it will be useful,            %%
%% but WITHOUT ANY WARRANTY; without even the implied warranty of             %%
%% MERCHANTABILITY or FITNESS FOR A PARTICULAR PURPOSE.  See the              %%
%% GNU Affero General Public License for more details.                        %%
%%                                                                            %%
%% You should have received a copy of the GNU Affero General Public License   %%
%% along with this program.  If not, see <http://www.gnu.org/licenses/>.      %%
%%                                                                            %%
%% Dr. Winfried Teschers                                                      %%
%% Anton-Günther-Straße 26c                                                   %%
%% 91083 Baiersdorf                                                           %%
%% Germany                                                                    %%
%%                                                                            %%
%% e-mail: winfried.teschers@t-online.de                                      %%
%%                                                                            %%
%%############################################################################%%

% !TeX root = ASBA.tex
% !TeX encoding = UTF-8
% !TeX spellcheck = de_DE

% Elemente, die keine Glossareinträge sind, werden in "ASBA-Vorspann.tex" und "ASBA-Mathematik-Vorspann.tex" definiert.
%TODO Symbole in  Glossar und Index eintragen

% Symbole für Mengen -----------------------------------------------------------

\newglossaryentry{gsN}{
	name      ={$\gsN$},
	plural     ={\gsN},% im Mathematikmodus
	description ={Die Menge der natürlichen Zahlen ohne 0.}
}
\newglossaryentry{gsNo}{
	name      ={$\gsNo$},
	%TODO falscher Zeichensatz?
	plural     ={\gsNo},% im Mathematikmodus
	description ={Die Menge der natürlichen Zahlen einschließlich 0.}
}
\newglossaryentry{alABC}{
	name      ={$\alABC$},
	plural     ={\alABC},% im Mathematikmodus
	description ={Das Alphabet der aussagenlogischen Sprache.}
}
\newglossaryentry{alABCx}{
	name      ={$\alABC_x$},
	plural     ={\alABC_x},% im Mathematikmodus
	description ={
		Eine Teilmenge des Alphabets $\alABC$ der aussagenlogischen Sprache.
	}
}
\newglossaryentry{alBin}{
	name      ={$\alBin$},
	plural     ={\alBin},% im Mathematikmodus
	description ={Die Menge der aussagenlogischen, binären Operatoren.}
}
\newglossaryentry{alCon}{
	name      ={$\alCon$},
	plural     ={\alCon},% im Mathematikmodus
	description ={Die Menge der aussagenlogischen Konstanten.}
}
\newglossaryentry{alFor}{
	name      ={$\alFor$},
	plural     ={\alFor},% im Mathematikmodus
	description ={Die Menge der aussagenlogischen \emph{Formeln} mit Klammerung.}
}
\newglossaryentry{alForp}{
	name      ={$\alForp$},
	plural     ={\alForp},% im Mathematikmodus
	description ={
		Die Menge der aussagenlogischen \emph{Formeln} in polnischer Notation.
	}
}
\newglossaryentry{alForx}{
	name      ={$\alFor_x$},
	plural     ={\alFor_x},% im Mathematikmodus
	description ={
		Eine Teilmenge der Menge $\alFor$ der aussagenlogischen \emph{Formeln} mit Klammerung.
	}
}
\newglossaryentry{alForxp}{
	name      ={$\alForp_x$},
	plural     ={\alForp_x},% im Mathematikmodus
	description ={
		Eine Teilmenge der Menge $\alFor$ der aussagenlogischen \emph{Formeln} in polnischer Notation.
	}
}
\newglossaryentry{alJun}{
	name      ={$\alJun$},
	plural     ={\alJun},% im Mathematikmodus
	description ={Die Menge der aussagenlogischen \emph{Junktoren}.}
}
\newglossaryentry{alJunx}{
	name      ={$\alJun_x$},
	plural     ={\alJun_x},% im Mathematikmodus
	description ={
		Eine Teilmenge der Menge $\alJun$ der aussagenlogischen Operatoren.
	}
}
\newglossaryentry{alUna}{
	name      ={$\alUna$},
	plural     ={\alUna},% im Mathematikmodus
	description ={Die Menge der aussagenlogischen unären Operatoren.}
}
\newglossaryentry{alVar}{
	name      ={$\alVar$},
	plural     ={\alVar},% im Mathematikmodus
	description ={Die Menge der aussagenlogischen Variablen.}
}

% Symbole für Beispieloperatoren -----------------------------------------------

\newglossaryentry{lrelbsp}{
	name      ={$\lrelbsp$},
	plural     ={\lrelbsp},% im Mathematikmodus
	description ={
		Ein Beispielsymbol für eine Relation mit Umkehrrelation $\rrelbsp$.
	}
}
\newglossaryentry{lreleqbsp}{
	name      ={$\lreleqbsp$},
	plural     ={\lreleqbsp},% im Mathematikmodus
	description ={
		Ein Beispielsymbol für eine Relation mit Gleichheit und Umkehrrelation $\rreleqbsp$.
	}
}
\newglossaryentry{relbsp}{
	name      ={$\relbsp$},
	plural     ={\relbsp},% im Mathematikmodus
	description ={Ein Beispielsymbol für eine Relation.}
}
\newglossaryentry{releqbsp}{
	name      ={$\releqbsp$},
	plural     ={\releqbsp},% im Mathematikmodus
	description ={Ein Beispielsymbol für eine Relation mit Gleichheit.}
}
\newglossaryentry{relnbsp}{
	name      ={$\relnbsp$},
	plural     ={\relnbsp},% im Mathematikmodus
	description ={Verneinung von $\relbsp$.}
}
\newglossaryentry{rrelbsp}{
	name      ={$\rrelbsp$},
	plural     ={\rrelbsp},% im Mathematikmodus
	description ={
		Ein Beispielsymbol für eine Relation mit Umkehrrelation $\lrelbsp$.
	}
}
\newglossaryentry{rreleqbsp}{
	name      ={$\rreleqbsp$},
	plural     ={\rreleqbsp},% im Mathematikmodus
	description ={
		Ein Beispielsymbol für eine Relation mit Gleichheit und Umkehrrelation $\lreleqbsp$.
	}
}

% Meta-Symbole -----------------------------------------------------------------

\newglossaryentry{defeq}{
	name      ={$:=$},
	plural      ={:=},% im Mathematikmodus
	description ={Definition: \textdots\ definitionsgemäß gleich \textdots}
}
\newglossaryentry{derive}{
	name      ={$\derivesym$},
	plural     ={\derive},% im Mathematikmodus
	description ={
		\emph{Ableitungsrelation}: \textdots\ ableitbar \textdots
		-- siehe \emph{ableitbar}.
	}
}
\newglossaryentry{eq}{
	name      ={$\eq$},
	plural     ={\eq},% im Mathematikmodus
	description ={
		Eine \emph{Metarelation}: \textdots\ gleich (ist dasselbe wie, ist identisch zu) \textdots
	}
}
\newglossaryentry{equiv}{
	name      ={$\equiv$},
	plural     ={\equiv},% im Mathematikmodus
	description ={
		Eine \emph{Metarelation}: \textdots\ äquivalent zu (ist das gleiche wie, ist so wie) \textdots
	}
}
\newglossaryentry{metaand}{
	name      ={$\metaandsym$},
	plural     ={\metaand},% im Mathematikmodus
	description ={Ein \emph{Metaoperator}: \textdots\ und \textdots}
}
\newglossaryentry{metadefeq}{
	name      ={$\metadefeq$},
	plural     ={\metadefeq},% im Mathematikmodus
	description ={
		\emph{Metadefinition}: \textdots\ definitionsgemäß gleich (definitionsgemäß genau dann, wenn) \textdots
	}
}
\newglossaryentry{metaequiv}{
	name      ={$\metaequiv$},
	plural     ={\metaequiv},% im Mathematikmodus
	description ={Eine \emph{Metarelation}: \textdots\ genau dann wenn \textdots}
}
\newglossaryentry{metaimp}{
	name      ={$\metaimp$},
	plural     ={\metaimp},% im Mathematikmodus
	description ={Eine \emph{Metarelation}: \textdots\ dann auch \textdots}
}
\newglossaryentry{metaor}{
	name      ={$\metaorsym$},
	plural     ={\metaor},% im Mathematikmodus
	description ={Ein \emph{Metaoperator}: \textdots\ oder \textdots}
}
\newglossaryentry{metarep}{
	name      ={$\metarep$},
	plural     ={\metarep},% im Mathematikmodus
	description ={Eine \emph{Metarelation}: \textdots\ sofern \textdots}
}
\newglossaryentry{ne}{
	name      ={$\ne$},
	plural     ={\ne},% im Mathematikmodus
	description ={
		Eine (Meta-)Operator: \textdots\ ungleich (nicht dasselbe wie, nicht identisch zu) \textdots
	}
}
\newglossaryentry{nequiv}{
	name      ={$\nequiv$},
	plural     ={\nequiv},% im Mathematikmodus
	description ={
		Eine \emph{Metarelation}: \textdots\ nicht äquivalent (ist nicht das gleiche wie, ist nicht so wie) \textdots
	}
}
\newglossaryentry{subst}{
	name      ={$\subst$},
	plural     ={\subst},% im Mathematikmodus
	description ={
		\emph{Substitution}: \textdots\ substituiert durch \textdots\
		-- siehe die Definition \vrefinsub{sub:Identitätsregeln}.
	}
}
\newglossaryentry{swap}{
	name      ={$\swap$},
	plural     ={\swap},% im Mathematikmodus
	description ={
		\emph{Vertauschung}: \textdots\ vertauscht mit \textdots\
		-- siehe die Definition \vrefinsub{sub:Identitätsregeln}.
	}
}
\newglossaryentry{srand}{
	name      ={$\srand$},
	plural     ={\srand},% im Mathematikmodus
	description ={
		Ein \emph{Metaoperator}: \textdots\ und \textdots\
		-- wird nur bei den \emph{Schlussregeln} verwendet.
	}
}

% sonstige mathematische Symbole -----------------------------------------------

\newglossaryentry{lfalse}{
	name      ={$\lfalse$},
	plural     ={\lfalse},% im Mathematikmodus
	description ={Eine aussagenlogische Konstante (\emph{Wahrheitswert}): Falsch.}
}
\newglossaryentry{ltrue}{
	name      ={$\ltrue$},
	plural     ={\ltrue},% im Mathematikmodus
	description ={Eine aussagenlogische Konstante (\emph{Wahrheitswert}): Wahr.}
}

% Schlussregeln ----------------------------------------------------------------

\newcommand* {\tagAR}{AR}% Argument für \tag - im Textmodus
\newcommand*    {\AR}{(\text{AR})}% im Mathematikmodus
\newglossaryentry{AR}{
	name      ={$\AR$},
	plural     ={\AR},% im Mathematikmodus
	description ={\emph{Anfangsregel}}
}
\newcommand* {\tagFS}{FS}% Argument für \tag - im Textmodus
\newcommand*    {\FS}{(\text{FS})}% im Mathematikmodus
\newglossaryentry{FS}{
	name      ={$\FS$},
	plural     ={\FS},% im Mathematikmodus
	description ={\emph{formaler Satz}}
}
\newcommand* {\tagMR}{MR}% Argument für \tag - im Textmodus
\newcommand*    {\MR}{(\text{MR})}% im Mathematikmodus
\newglossaryentry{MR}{
	name      ={$\MR$},
	plural     ={\MR},% im Mathematikmodus
	description ={\emph{Monotonieregel}}
}
\newcommand* {\tagSR}{SR}% Argument für \tag - im Textmodus
\newcommand*    {\SR}{(\text{SR})}% im Mathematikmodus
\newglossaryentry{SR}{
	name      ={$\SR$},
	plural     ={\SR},% im Mathematikmodus
	description ={\emph{Schnittregel} (Modus ponens)}
}
\newcommand* {\tagTR}{TR}% Argument für \tag - im Textmodus
\newcommand*    {\TR}{(\text{TR})}% im Mathematikmodus
\newglossaryentry{TR}{
	name      ={$\TR$},
	plural     ={\TR},% im Mathematikmodus
	description ={\emph{Abtrennungsregel}}
}
\newcommand* {\tageqB}{$\eq$B}% Argument für \tag - im Textmodus
\newcommand*    {\eqB}{(\eq\text{B})}% im Mathematikmodus
\newglossaryentry{eqB}{
	name      ={$\eqB$},
	plural     ={\eqB},% im Mathematikmodus
	description ={Beseitigung von \chrqt{$\eq$}}
}
\newcommand* {\tageqE}{$\eq$E}% Argument für \tag - im Textmodus
\newcommand*    {\eqE}{(\eq\text{E})}% im Mathematikmodus
\newglossaryentry{eqE}{
	name      ={$\eqE$},
	plural     ={\eqE},% im Mathematikmodus
	description ={Einführung von \chrqt{$\eq$}}
}
\newcommand* {\tagandB}{$\land$B}% Argument für \tag - im Textmodus
\newcommand*    {\andB}{(\land\text{B})}% im Mathematikmodus
\newglossaryentry{andB}{
	name      ={$\andB$},
	plural     ={\andB},% im Mathematikmodus
	description ={Beseitigung von \chrqt{$\land$}}
}
\newcommand* {\tagandE}{$\land$E}% Argument für \tag - im Textmodus
\newcommand*    {\andE}{(\land\text{E})}% im Mathematikmodus
\newglossaryentry{andE}{
	name      ={$\andE$},
	plural     ={\andE},% im Mathematikmodus
	description ={Einführung von \chrqt{$\land$}}
}
\newcommand* {\tagimpB}{$\limp$B}% Argument für \tag - im Textmodus
\newcommand*    {\impB}{(\limp\text{B})}% im Mathematikmodus
\newglossaryentry{impB}{
	name      ={$\impB$},
	plural     ={\impB},% im Mathematikmodus
	description ={Beseitigung von \chrqt{$\limp$}}
}
\newcommand* {\tagimpE}{$\limp$E}% Argument für \tag - im Textmodus
\newcommand*    {\impE}{(\limp\text{E})}% im Mathematikmodus
\newglossaryentry{impE}{
	name      ={$\impE$},
	plural     ={\impE},% im Mathematikmodus
	description ={Einführung von \chrqt{$\limp$}}
}
\newcommand* {\tagnota}{$\lnot$1}% Argument für \tag - im Textmodus
\newcommand*    {\nota}{(\lnot\text{1})}% im Mathematikmodus
\newglossaryentry{nota}{
	name={ $\nota$},
	plural     ={\nota},% im Mathematikmodus
	description ={Einführung/Beseitigung von \chrqt{$\lnot$} Teil 1}
}
\newcommand* {\tagnotb}{$\lnot$2}% Argument für \tag - im Textmodus
\newcommand*    {\notb}{(\lnot\text{2})}% im Mathematikmodus
\newglossaryentry{notb}{
	name      ={$\notb$},
	plural     ={\notb},% im Mathematikmodus
	description ={Einführung/Beseitigung von \chrqt{$\lnot$} Teil 2}
}
\newcommand* {\tagnotc}{$\lnot$3}% Argument für \tag - im Textmodus
\newcommand*    {\notc}{(\lnot\text{3})}% im Mathematikmodus
\newglossaryentry{notc}{
	name      ={$\notc$},
	plural     ={\notc},% im Mathematikmodus
	description ={Beweistechnik \enquote{Indirekter \emph{Beweis}}}
}
\newcommand* {\tagnotd}{$\lnot$4}% Argument für \tag - im Textmodus
\newcommand*    {\notd}{(\lnot\text{4})}% im Mathematikmodus
\newglossaryentry{notd}{% statt "notE"
	name      ={$\notd$},
	plural     ={\notd},% im Mathematikmodus
	description ={Reductio ad absurdum (Indirekter \emph{Beweis})}
}
%%%\newcommand* {\tagorB}{$\lor$B}% Argument für \tag - im Textmodus
%%%\newcommand*    {\orB}{(\lor\text{B})}% im Mathematikmodus
%%%\newglossaryentry{orB}{
%%%	name      ={$\obB$},
%%%	plural     ={\orB},% im Mathematikmodus
%%%	description ={Beseitigung von \chrqt{$\lor$}}
%%%}
%%%\newcommand* {\tagorE}{$\lor$E}% Argument für \tag - im Textmodus
%%%\newcommand*    {\orE}{(\lor\text{E})}% im Mathematikmodus
%%%\newglossaryentry{orE}{
%%%	name      ={$\obE$},
%%%	plural     ={\orE},% im Mathematikmodus
%%%	description ={Beseitigung von \chrqt{$\lor$}}
%%%}

% Fachbegriffe #################################################################
%TODO Alle Fachbegriffe in Glossar und Index eintragen

%A === A === A === A === A === A === A === A === A === A === A === A === A === A

\newcommand*{\ASBA}{\glsIdx{ASBA}}
\newacronym{ASBA}{ASBA}{
	Programmsystem, das \textbf{A}xiome, \textbf{S}ätze, \textbf{B}eweise und \textbf{A}uswertungen behandeln kann.
}
\newcommand*    {\ableitbar} {\glsIdx  {ableitbar}}
\newcommand*    {\ableitbare}{\glsIdxPl{ableitbar}}
\newglossaryentry{ableitbar}{
	name        ={ableitbar},
	plural      ={ableitbare},
	description ={
		Wenn sich eine \emph{Formel} $\beta$ aus einer anderen \emph{Formel} $\alpha$ mittels zulässiger Transaktionen ableiten lässt, heißt $\beta$ \emph{ableitbar} aus $\alpha$.
		Sprechweise: \seqqt{$ \alpha \text{ableitbar} \beta $}.
		Eine oder beide \emph{Formeln} $\alpha$ \textbzw\ $\beta$ dürfen dabei durch \emph{Formelmengen} ersetzt werden.
		-- siehe \emph{Ableitungsrelation} und \chrqt{$\derivesym$}.
		\newline
		Synonym: \emph{beweisbar}.%
	}
}
\newcommand*    {\Ableitungsrelation}{\glsIdx{Ableitungsrelation}}
\newglossaryentry{Ableitungsrelation}{
	name        ={Ableitungsrelation},
	description ={Die Relation \chrqt{$\derivesym$}.}
}
\newcommand*    {\Abtrennungsregel}{\glsIdx{Abtrennungsregel}}
\newglossaryentry{Abtrennungsregel}{
	name        ={Abtrennungsregel},
	description ={Eine \emph{Schlussregel} -- siehe~\emph{TR}}
}
\newcommand*    {\allgemeingueltigeSchlussregel}  {\glsIdx  {allgemeingueltige-Schlussregel}}
\newcommand*    {\allgemeingueltigenSchlussregel} {\glsIdxBg{allgemeingueltige-Schlussregel}{allgemeingültigen Schlussregel}}
\newcommand*    {\allgemeingueltigeSchlussregeln} {\glsIdxPl{allgemeingueltige-Schlussregel}}
\newcommand*    {\allgemeingueltigenSchlussregeln}{\glsIdxBg{allgemeingueltige-Schlussregel}{allgemeingültigen Schlussregeln}}
\newglossaryentry{allgemeingueltige-Schlussregel}{
	name        ={allgemeingültige Schlussregel},
	plural      ={allgemeingültige Schlussregeln},
	description ={
		Eine \emph{Schlussregel} die aus den \emph{Basisregeln} und den schon bekannten \emph{allgemeingültigen Schlussregeln} abgeleitet werden kann.
	}
}
\newcommand*    {\Anfangsregel}{\glsIdx{Anfangsregel}}
\newglossaryentry{Anfangsregel}{
	name        ={Anfangsregel},
	description ={
		Eine \emph{Schlussregel} um beginnen zu können -- siehe~\emph{AR}.}
}
\newcommand*    {\atomareFormel} {\glsIdx  {atomare-Formel}}
\newcommand*    {\atomareFormeln}{\glsIdxPl{atomare-Formel}}
\newglossaryentry{atomare-Formel}{
	name        ={atomare Formel},
	plural      ={atomare Formeln},
	description ={Eine \emph{Formel}, die sich nicht weiter zerlegen lässt.}
}
\newcommand*    {\Ausgabeschema}  {\glsIdx  {Ausgabeschema}}
\newcommand*    {\Ausgabeschemata}{\glsIdxPl{Ausgabeschema}}
\newglossaryentry{Ausgabeschema}{
	name        ={Ausgabeschema},
	plural      ={Ausgabeschemata},
	description ={
		Ein Schema, mit dem bestimmte mathematische \emph{Objekte} ausgegeben werden sollen.
	}
}
\newcommand*    {\Aussage} {\glsIdx  {Aussage}}
\newcommand*    {\Aussagen}{\glsIdxPl{Aussage}}
\newglossaryentry{Aussage}{
	name        ={Aussage},
	plural      ={Aussagen},
	description ={
		Eine \emph{Aussage} in natürlicher Sprache oder als \emph{Formel}, die einen \emph{Wahrheitswert} liefert.
	}
}
\newcommand*    {\Aussagenlogik}{\glsIdx{Aussagenlogik}}
\newglossaryentry{Aussagenlogik}{
	name        ={Aussagenlogik},
	description ={-- \vrefseesec{sec:Aussagenlogik}.}
}
\newcommand*    {\Axiom}  {\glsIdx  {Axiom}}
\newcommand*    {\Axiome} {\glsIdxPl{Axiom}}
\newcommand*    {\Axiomen}{\glsIdxBg{Axiom}{Axiomen}}
\newglossaryentry{Axiom}{
	name        ={Axiom},
	plural      ={Axiome},
	description ={Eine Formel, die unbewiesen als wahr angesehen wird.}
}
\newcommand*    {\Axiomensystem} {\glsIdx  {Axiomensystem}}
\newcommand*    {\Axiomensysteme}{\glsIdxPl{Axiomensystem}}
\newglossaryentry{Axiomensystem}{
	name        ={Axiomensystem},
	plural      ={Axiomensysteme},
	description ={Eine Menge von \emph{Axiomen}.}
}

%B === B === B === B === B === B === B === B === B === B === B === B === B === B

\newcommand*    {\Basisregel} {\glsIdx  {Basisregel}}
\newcommand*    {\Basisregeln}{\glsIdxPl{Basisregel}}
\newglossaryentry{Basisregel}{
	name        ={Basisregel},
	plural      ={Basisregeln},
	description ={
		Eine \emph{Schlussregel}, die nicht mehr auf andere zurückgeführt wird.
		Obwohl das auch auf die \emph{Identitätsregeln} zutrifft, werden diese hier aber nicht dazu gezählt.
	}
}
\newcommand*    {\Beweis}  {\glsIdx  {Beweis}}
\newcommand*    {\Beweises}{\glsIdxBg{Beweis}{Beweises}}
\newcommand*    {\Beweise} {\glsIdxPl{Beweis}}
\newcommand*    {\Beweisen}{\glsIdxBg{Beweis}{Beweisen}}
\newglossaryentry{Beweis}{
	name        ={Beweis},
	plural      ={Beweise},
	description ={
		Eine zulässige Ableitung von Folgerungen aus gegebenen Voraussetzungen.
		-- 	\newline\vrefseesec{sec:BeweiseASBA}.
	}
}
%TODO === hier weitermachen
\newcommand*    {\beweisbar} {\glsIdx  {beweisbar}}
\newcommand*    {\beweisbare}{\glsIdxPl{beweisbar}}
\newglossaryentry{beweisbar}{
	name        ={beweisbar},
	plural      ={beweisbare},
	description ={Synonym zu \emph{ableitbar}.}
}
\newcommand*    {\Beweisschritt}  {\glsIdx  {Beweisschritt}}
\newcommand*    {\Beweisschritte} {\glsIdxPl{Beweisschritt}}
\newcommand*    {\Beweisschritten}{\glsIdxBg{Beweisschritt}{Beweisschritten}}
\newglossaryentry{Beweisschritt}{
	name        ={Beweisschritt},
	plural      ={Beweisschritte},
	description ={
		Eine Vorschrift, wie aus vorgegebenen \emph{Aussagen} (den \emph{Voraussetzungen}) eine weitere (die \emph{Folgerung}) folgt.
	}
}
\newglossaryentry{Beweisschrittfolge}{
	name        ={Beweisschrittfolge},
	plural      ={Beweisschrittfolgen},
	description ={Eine Folge von \emph{Beweisschritten}.}
}
\newglossaryentry{Beweisschrittmenge}{
	name        ={Beweisschrittmenge},
	plural      ={Beweisschrittmengen},
	description ={
		Die Menge der \emph{Beweisschritte}, \textdh\ der Glieder der \emph{Beweisschrittfolge} eines \emph{Beweises}.
	}
}
\newglossaryentry{Boolsche-Signatur}{
	name        ={Boolsche Signatur},
	plural      ={Boolsche Signaturen},
	description ={Die \emph{logische Signatur} $\{\lnot, \land, \lor\}$.}
}

%F === F === F === F === F === F === F === F === F === F === F === F === F === F

\newcommand*    {\Fachbegriff}  {\glsIdx  {Fachbegriff}}
\newcommand*    {\Fachbegriffe} {\glsIdxPl{Fachbegriff}}
\newcommand*    {\Fachbegriffen}{\glsIdxBg{Fachbegriff}{Fachbegriffen}}
\newglossaryentry{Fachbegriff}{
	name        ={Fachbegriff},
	plural      ={Fachbegriffe},
	description ={Ein Name für einen mathematischen Begriff.}
}
\newcommand*    {\Fachgebiet}  {\glsIdx  {Fachgebiet}}
\newcommand*    {\Fachgebiete} {\glsIdxPl{Fachgebiet}}
\newcommand*    {\Fachgebieten}{\glsIdxBg{Fachgebiet}{Fachgebieten}}
\newglossaryentry{Fachgebiet}{
	name        ={Fachgebiet},
	plural      ={Fachgebiete},
	description ={
		Ein Teil der Mathematik mit einer zugehörigen Basis aus \Axiomen, Sätzen und spezifischen \Fachbegriffen\ und Darstellungen.
	}
}
\newglossaryentry{Folgerung}{
	name        ={Folgerung},
	plural      ={Folgerungen},
	description ={
		Die Folgerungen einer \emph{Schlussregel} $\frac{\prerequisiteset}{\conclusionset}$ sind die Elemente von $\conclusionset$.
	}
}
%TODO Folgerungsmenge
\newglossaryentry{formaler-Satz}{
	name        ={formaler Satz},
	plural      ={formale  Sätze},
	description ={
		Formale Darstellung eines mathematischen Satzes. -- siehe~\emph{FS}.
	}
}
\newglossaryentry{Formel}{
	name        ={Formel},
	plural      ={Formeln},
	description ={
		Unter einer \emph{Formel} verstehen wir in diesem Dokument stets eine mathematische \emph{Formel}.
		Diese kann auch mehrdimensional sein, lässt sich aber mittels geeigneter Definitionen immer eindeutig als eine \emph{Zeichenfolge} schreiben.
		\emph{Schlussregeln} betrachten wir \emph{nicht} als \emph{Formeln}.
	}
}
\newglossaryentry{Formelmenge}{
	name        ={Formelmenge},
	plural      ={Formelmengen},
	description ={
		\textIAlg\ eine Menge von Formeln $\formula$ \textbzw\ Worten.
		Man nennt $\formulaset$ auch \emph{Sprache}.
	}
}

%G === G === G === G === G === G === G === G === G === G === G === G === G === G

\newglossaryentry{Gleichheitsrelation}{
	name        ={Gleichheitsrelation},
	plural      ={Gleichheitsrelationen},
	description ={
		Eine mit der Gleichheit verwandte Relation: $\eq$, $\ne$, $\equiv$ und $\nequiv$.
	}
}

%I === I === I === I === I === I === I === I === I === I === I === I === I === I

%TODO Identitätsregel nötig?
\newglossaryentry{Identitaetsregel}{
	name        ={Identitätsregel},
	plural      ={Identitätsregeln},
	description ={
		Eigentlich eine \emph{Basisregel} zur Identität.
		Da die \emph{Identitätsregeln} nur zur Rechtfertigung der \emph{Substitution} verwendet werden, werden sie hier nicht zu den \emph{Basisregeln} gezählt.
	}
}
\newglossaryentry{interessierende-Eigenschaft}{
	name        ={interessierende Eigenschaft},
	plural      ={interessierende Eigenschaften},
	description ={
		Solche Eigenschaften von \emph{Objekten}, die im aktuellen Zusammenhang von Interesse sind.
	}
}

%J === J === J === J === J === J === J === J === J === J === J === J === J === J

\newglossaryentry{Junktor}{
	name        ={Junktor},
	plural      ={Junktoren},
	description ={Ein Operatorsymbol, \textdh\ ein Symbol für einen Operator.}
}

%K === K === K === K === K === K === K === K === K === K === K === K === K === K

\newglossaryentry{Kontraposition}{
	name        ={Kontraposition},
	plural      ={Kontraposition},
	description ={
		Die allgemeingültige \emph{Aussage}: $ (\alpha \limp \beta) \limp (\lnot\beta \limp \lnot\alpha) $.
	}
}

%L === L === L === L === L === L === L === L === L === L === L === L === L === L

\newglossaryentry{logische-Signatur}{
	name        ={logische Signatur},
	plural      ={logische Signaturen},
	description ={
		Eine Teilmenge von $\alJun$, die ausreicht, alle anderen Elemente von $\alJun$ zu definieren.
	}
}

%M === M === M === M === M === M === M === M === M === M === M === M === M === M

\newglossaryentry{Mengenlehre}{
	name={Mengenlehre},
	description ={-- \vrefseesec{sec:Mengenlehre}.}
}
\newglossaryentry{Metaoperator}{
	name        ={Metaoperator},
	plural      ={Metaoperatoren},
	description ={
		Ein Operator der \emph{Metasprache}: $\metaandsym$, $\metaorsym$ und $\srand$.
	}
}
\newglossaryentry{Metarelation}{
	name        ={Metarelation},
	plural      ={Metarelationen},
	description ={
		Eine Relation der \emph{Metasprache}: $\metaimp$, $\metarep$ und $\metaequiv$.
	}
}
\newglossaryentry{Metasprache}{
	name        ={Metasprache},
	plural      ={Metasprachen},
	description ={
		Eine Sprache, in der \emph{Aussagen} über Elemente einer anderen Sprache getroffen werden können.
		In diesem Dokument ist dies immer die normale Sprache.
		-- \vrefseesec{sec:Metasprache}.
	}
}
\newglossaryentry{Monotonieregel}{
	name        ={Monotonieregel},
	plural      ={Monotonieregeln},
	description ={
		Eine \emph{Schlussregel}. -- siehe~\emph{MR}.
	}
}

%O === O === O === O === O === O === O === O === O === O === O === O === O === O

\newglossaryentry{Objekt}{
	name        ={Objekt},
	plural      ={Objekte},
	description ={
		Symbole, \emph{Formeln} und \emph{Aussagen} sowie Mengen, \emph{Zeichenfolgen}, Zahlen; ganz allgemein reale oder gedachte Dinge an sich.
	}
}

%P === P === P === P === P === P === P === P === P === P === P === P === P === P

\newglossaryentry{Praedikat}{
	name        ={Prädikat},
	plural      ={Prädikate},
	description ={
		Ein Element der \emph{Prädikatenlogik} -- \vrefseesec{sec:Prädikatenlogik}.\\
		\textZB\ kann man eine Gruppe als ein zweistelliges Prädikat $\mathrm{Gruppe}(G,+)$ definieren, in dem $G$ eine Menge und $+$ eine Operation, \textdh\ eine (zweistellige) Funktion $ +: G \times G \rightarrow G $ ist, so dass die Gruppenaxiome erfüllt sind.
	}
}
\newcommand*    {\Praedikatenlogik}{\glsIdx{Praedikatenlogik}}
\newglossaryentry{Praedikatenlogik}{
	name={Prädikatenlogik},
	description ={-- \vrefseesec{sec:Prädikatenlogik}.}
}

%S === S === S === S === S === S === S === S === S === S === S === S === S === S

\newcommand*    {\Satz}   {\glsIdx  {Satz}}
\newcommand*    {\Satzes} {\glsIdxBg{Satz}{Satzes}}
\newcommand*    {\Saetze} {\glsIdxPl{Satz}}
\newcommand*    {\Saetzen}{\glsIdxBg{Satz}{Sätzen}}
\newglossaryentry{Satz}{
	name        ={Satz},
	plural      ={Sätze},
	description ={
		Eine mathematische \emph{Aussage}, dass bestimmte Folgerungen aus gegebenen Voraussetzungen abgeleitet werden können.
	}
}
\newglossaryentry{Schlussregel}{
	name        ={Schlussregel},
	plural      ={Schlussregeln},
	description ={
		Eine \emph{Schlussregel} $\frac{\prerequisiteset}{\conclusionset}$ entspricht der \emph{Aussage}:
		Wenn alle \emph{Voraussetzungen} $\prerequisite$ aus $\prerequisiteset$ zutreffen, dann auch alle \emph{Folgerungen} $\conclusion$ aus $\conclusionset$.
		Wenn diese \emph{Aussage} zutrifft, kann die Schlussregel zur zulässigen Umwandlung von \emph{Formeln} dienen.
	}
}
\newglossaryentry{Schlussregelmenge}{
	name        ={Schlussregelmenge},
	plural      ={Schlussregelmengen},
	description ={
		Eine Menge von \emph{Schlussregeln}, meistens mit $\conclusionruleset$ bezeichnet.
	}
}
\newglossaryentry{Schnittregel}{
	name        ={Schnittregel},
	plural      ={Schnittregeln},
	description ={Eine \emph{allgemeingültige Schlussregel}. -- siehe~\emph{SR}}.
}
\newglossaryentry{Sprache}{
	name        ={Sprache},
	plural      ={Sprachen},
	description ={-- siehe \emph{Formelmenge}.}
}
\newglossaryentry{Substitution}{ %TODO ggf. überarbeiten
	name        ={Substitution},
	plural      ={Substitutionen},
	description ={
		Die Ersetzung von einem, mehreren oder allen \emph{formalen Elementen} ($\alpha$) in einem anderen \emph{formalen Element} ($\gamma$) durch ein drittes \emph{formales Element} ($\beta$)
		-- formal: $\gamma(\alpha\subst\beta)$.
		Wenn alle $\alpha$ in $\gamma$ durch $\beta$ ersetzt werden, ist die \emph{Substitution vollständig}.
		-- \vrefseesub{sub:Identitätsregeln}.
	}
}
%TODO Substitutionsmenge
\newglossaryentry{Symbol}{
	name        ={Symbol},
	plural      ={Symbole},
	description ={
		Ein einfaches Symbol ist ein druckbares typographisches Zeichen.
		Ein zusammengesetztes Symbol besteht aus mehreren einfachen Symbolen.
		In beiden Fällen wird ein Symbol als unzerlegbar angesehen.
		\vrefseesec{sec:Notationen}.
	}
}

%T === T === T === T === T === T === T === T === T === T === T === T === T === T

\newglossaryentry{Transformation}{
	name        ={Transformation},
	plural      ={Transformationen},
	description ={
		Eine Umformung oder Erzeugung einer \emph{Formel} aus einer vorgegebenen Menge von \emph{Formeln},
		\textdh\ die Anwendung einer \emph{Schlussregel}.
	}
}
\newglossaryentry{Transformationsmenge}{
	name        ={Transformationsmenge},
	plural      ={Transformationsmenge},
	description ={
		Eine Menge von \emph{Transformationen}.
	}
}

%V === V === V === V === V === V === V === V === V === V === V === V === V === V

\newglossaryentry{vergleichbar}{
	name        ={vergleichbar},
	plural      ={vergleichbare},
	description ={
		Zwei \emph{Objekte} $A$ und $B$ sind vergleichbar, wenn beide von derselben Art sind, \textdh\ wenn \textzB\ jeweils beide Mengen, \emph{Zeichenfolgen}, Zahlen, \textusw\ sind.
		Dabei muss bei \emph{Formeln} zwischen der Formel an sich und dem Ergebnis der Formel unterschieden werden.
		-- \vrefseesec{subsub:Vergleichbar}.
	}
}
\newglossaryentry{Vertauschung}{ %TODO ggf. überarbeiten
	name        ={Vertauschung},
	plural      ={Vertauschungen},
	description ={
		Die \emph{Vertauschung} von zwei unabhängigen \emph{formalen Elementen} ($\alpha$ und $\beta$) in einem anderen formalen Element ($\gamma$)
		-- formal: $\gamma(\alpha\swap\beta)$.
		Die Vertauschung ist eine spezielle Form der \emph{Substitution}.
		-- siehe die Definition~\eqref{def:Vertauschung} \vrefinsub{sub:Identitätsregeln}.
	}
}
\newglossaryentry{Voraussetzung}{
	name        ={Voraussetzung},
	plural      ={Voraussetzungen},
	description ={
		Die Voraussetzungen einer \emph{Schlussregel} $\frac{\prerequisiteset}{\conclusionset}$ sind die Elemente von $\prerequisiteset$.
	}
}
%TODO Voraussetzungsmenge

%W === W === W === W === W === W === W === W === W === W === W === W === W === W

\newglossaryentry{Wahrheitswert}{
	name        ={Wahrheitswert},
	plural      ={Wahrheitswerte},
	description ={
		Wahrheitswerte sind die Werte \chrqt{$\ltrue$} und \chrqt{$\lfalse$}, oft auch mit \chrqt{$\mathrm{wahr}$} und \chrqt{$\mathrm{falsch}$}, \chrqt{$\mathrm{true}$} und \chrqt{$\mathrm{false}$} oder einfach \chrqt{$1$} und \chrqt{$0$} bezeichnet.
	}
}

%Z === Z === Z === Z === Z === Z === Z === Z === Z === Z === Z === Z === Z === Z

\newglossaryentry{Zeichenfolge}{
	name        ={Zeichenfolge},
	plural      ={Zeichenfolgen},
	description ={
		Folgen von unzerlegbaren Zeichen und Symbolen, wobei Leerstellen und sonstiger Zwischenraum nicht zählen und nur zur besseren Darstellung dienen.
		Dabei sind als spezielle Symbole auch \emph{Zeichenketten} erlaubt, solange die Zerlegung eindeutig bleibt.
		\textZB\ kann \chrqt{$sin$} als ein einzelnes Symbol -- für die Sinusfunktion -- aufgefasst werden, aber auch als Folge der Buchstaben \chrqt{s}, \chrqt{i} und \chrqt{b}.
		\emph{Formeln} werden immer als Zeichenfolgen aufgefasst.
	}
}
\newglossaryentry{Zeichenkette}{
	name        ={Zeichenkette},
	plural      ={Zeichenketten},
	description ={
		Folgen von unzerlegbaren Zeichen, auch Leerstellen und sonstigem Zwischenraum.
		-- siehe auch \emph{Zeichenfolge}.
	}
}
\newglossaryentry{zulaessige-Transformation}{%TODO ggf. überarbeiten
	name        ={zulässige Transformation},
	plural      ={zulässige Transformationen},
	description ={
		Eine \emph{Transformation} aus einer vorgegebenen Menge von Transformationen oder eine daraus zulässiger weise abgeleitete Transformation.
	}
}

%%############################################################################%%
%%                                                                            %%
%% Datei:  ASBA-Glossar-Texte.tex                                             %%
%% Inhalt: Vorspann Glossareinträge für ASBA                                  %%
%%                                                                            %%
%% Copyright (C) 2017  Winfried Teschers                                      %%
%%                                                                            %%
%% This program is free software: you can redistribute it and/or modify       %%
%% it under the terms of the GNU Affero General Public License as published   %%
%% by the Free Software Foundation, either version 3 of the License, or       %%
%% (at your option) any later version.                                        %%
%%                                                                            %%
%% This program is distributed in the hope that it will be useful,            %%
%% but WITHOUT ANY WARRANTY; without even the implied warranty of             %%
%% MERCHANTABILITY or FITNESS FOR A PARTICULAR PURPOSE.  See the              %%
%% GNU Affero General Public License for more details.                        %%
%%                                                                            %%
%% You should have received a copy of the GNU Affero General Public License   %%
%% along with this program.  If not, see <http://www.gnu.org/licenses/>.      %%
%%                                                                            %%
%% Dr. Winfried Teschers                                                      %%
%% Anton-Günther-Straße 26c                                                   %%
%% 91083 Baiersdorf                                                           %%
%% Germany                                                                    %%
%%                                                                            %%
%% e-mail: winfried.teschers@t-online.de                                      %%
%%                                                                            %%
%%############################################################################%%

% !TeX root = ASBA.tex
% !TeX encoding = UTF-8
% !TeX spellcheck = de_DE

% ### Glossar und Index ########################################################

% ==============================================================================
% \* - Ausgabe als Text und Eintrag und Verweis ins Glossar
% Fachbegriffe =================================================================

\iftestFlg% Definition von Dummy Glossareinträgen; gleichzeitig Kopiervorlagen
	\newVerweis     {\dummyDummy}{\glstext}{dummyDummy}
	\newglossaryentry{dummyDummy}{
		name       =        {---, dummy \addIdx[% Ausgabe             im Glossar
			name   =        {---, dummy},%        Ausgabe             im Index
			sort   =      {Dummy, dummy}]  {dummyDummy}},%Reihenfolge im Index
		sort       =      {Dummy, dummy},%                Reihenfolge im Glossar
		text       ={dummy Dummy},%               Ausgabe             im Text
		user1      ={},%             alternative Ausgabe im Text für \newVerweis
		user2      ={},%             alternative Ausgabe im Text für \newVerweis
		see        ={},%             Verweise ins Symbolverzeichnis und Glossar
		description={\todoBeschreiben%
		}
	}
	\newVerweis     {\Dummy}{\glstext}{Dummy}
	\newglossaryentry{Dummy}{
		name        ={Dummy \addIdx   {Dummy}},% Ausgabe/Reihenfolge sonst
		text        ={Dummy},%                   Ausgabe             im Text
		user1       ={},%            alternative Ausgabe im Text für \newVerweis
		user2       ={},%            alternative Ausgabe im Text für \newVerweis
		see         ={},%            Verweise ins Symbolverzeichnis und Glossar
		description ={\todoBeschreiben%
		}
	}
	% dsgl. mit Absätzen in der Beschreibung
	\newVerweis         {\WikiDummy}{\glstext}{WikiDummy}
	\longnewglossaryentry{WikiDummy}{
		name            ={WikiDummy \addIdx   {WikiDummy}},
		text            ={WikiDummy},
		user1           ={},
		user2           ={},
		see             ={},
	}{\todoBeschreiben%
		\wikicite{bib:Wikipedia}{
		}
	}
\else\fi

% TODO ### symbol := Symbol (falls vorhenden); nur dann: am Seitenrand ausgeben: mit \SymbolAmRand{label}

% TODO ### Selbstreferenzen auflösen

% TODO ### Vorkommen prüfen und durch Makro ersetzen

% TODO symbol  ={\formaleDefinition{symbol}},%  Symbol ohne Verweis, falls vorhenden

% TODO \todo... nach description einfügen

%A === A === A === A === A === A === A === A === A === A === A === A === A === A

\newsynonym{\Abbildung}{Abbildung}{\Funktion}

\newVerweis     {\ableitbar} {\glstext}{ableitbar}
\newVerweis[e]  {\ableitbare}{\glstext}{ableitbar}
\newglossaryentry{ableitbar}{
	name        ={ableitbar \addIdx    {ableitbar}},
	text        ={ableitbar},
	see         ={Ableitungsrelation},
	description ={\todoPruefen%
		Wenn sich eine \Formel\ $\beta$ aus einer anderen \Formel\ $\alpha$ mittels \zulaessiger\ \Transformationen\ ableiten lässt, heißt $\beta$ \GloFt{ableitbar} aus $\alpha$.
		Sprechweise: $\alpha$ \GloFt{ableitbar}\synonym{\beweisbar} $\beta $.
		Eine oder beide \Formeln\ $\alpha$ \textbzw\ $\beta$ dürfen dabei durch \Formelmengen\ ersetzt werden.
	}
}

\newVerweis         {\Ableitung}  {\glstext}{Ableitung}
\newVerweis[en]     {\Ableitungen}{\glstext}{Ableitung}
\longnewglossaryentry{Ableitung}{
	name            ={Ableitung \addIdx     {Ableitung}},
	text            ={Ableitung},
	see             ={Ableitungsmenge,Ableitungsrelation,Konklusion,Logik,Praemisse,Schlussregel},
}{\todoPruefen%
	\wikicite{bib:Ableitung}{
		Eine \wikiBoldFt{Ableitung}, \wikiBoldFt{Herleitung}, oder \wikiLinkFt{Deduktion} ist in der \wikiLinkFt{Logik} die Gewinnung von \wikiLinkFt{Aussagen} aus anderen Aussagen. Dabei werden \wikiLinkFt{Schlussregeln} auf \wikiLinkFt{Prämissen} angewandt, um zu \wikiLinkFt{Konklusionen} zu gelangen. Welche Schlussregeln dabei erlaubt sind, wird durch das verwendete \wikiLinkFt{Kalkül} bestimmt.

		Die Ableitung ist zusammen mit der \wikiLinkFt{semantischen Konklusion} einer der zwei logischen Methoden, um auf die Konklusion zu kommen.
	}
	Eine Ableitung ist für \ASBA\ eine \Aussage\ $A \MtsDerive B$ \textbzw\ allgemeiner $A \MtsDeriveR B$ mit $A,B \MtsSubsetEq \MtsSprache$, wobei \MtsSprache\ eine \Sprache\ ist.
	Dies entspricht einem \Element\ $(A,B)$ aus einer \Ableitungsrelation\ \MtsDerive\ \textbzw\ \MtsDeriveR\ (\textdh\ $(A,B) \in R_{\MtsIdxGraph}$.
	Die semantische Aussage ist die, das die \Formeln\ aus $B$ aus den \Formeln\ aus $A$ abgeleitet werden können.
}

\newVerweis     {\Ableitungsmenge} {\glstext}{Ableitungsmenge}
\newcommand*    {\Ableitungsmengen}[1][]{\glstext[#1]{Ableitungsmenge}[n]}
\newglossaryentry{Ableitungsmenge}{
	name        ={Ableitungsmenge \addIdx    {Ableitungsmenge}},
	text        ={Ableitungsmenge},
	description ={\todoPruefen%
		Eine \Menge\ von \Ableitungen, letztlich nichts anderes als eine \Ableitungsrelation.
	}
}

\newVerweis     {\Ableitungsrelation}  {\glstext}{Ableitungsrelation}
\newVerweis[en] {\Ableitungsrelationen}{\glstext}{Ableitungsrelation}
\newglossaryentry{Ableitungsrelation}{
	name        ={Ableitungsrelation \addIdx     {Ableitungsrelation}},
	text        ={Ableitungsrelation},
	see         ={Ableitung},
	description ={\todoPruefen%
		Eine \binaere\ \Relation\ \MtsDerive\ aus \MtsAllDerive.
		Für $R \in \MtsAllDerive$ auch mit \MtsDeriveR\ bezeichnet.
	}
}

\newVerweis     {\Abtrennungsregel}{\glstext}{Abtrennungsregel}
\newglossaryentry{Abtrennungsregel}{
	name        ={Abtrennungsregel \addIdx   {Abtrennungsregel}},
	text        ={Abtrennungsregel},
	see         ={TR},
	description ={\todoPruefen%
		Eine \Schlussregel.
	}
}

%%%\newVerweis     {\Aequivalenz}  {\glstext}{Aequivalenz}
%%%\newVerweis[en] {\Aequivalenzen}{\glstext}{Aequivalenz}
%%%\newglossaryentry{Aequivalenz}{
%%%	name        ={Äquivalenz \addIdx[
%%%		name    ={Äquivalenz}]            {Aequivalenz}},
%%%	text        ={Äquivalenz},
%%%	see         ={MtsAequiv},
%%%	description ={\todoPruefen%
%%%		Eine \Gleichheitsrelation:
%%%		Zwei Objekte $A$ und $B$ sind \GloFt{äquivalent}\alternativi{ähnlich}, $A \MtsAequiv B$, wenn sie in den \interessierendenEigenschaften\ für \MtsAequiv\ übereinstimmen.
%%%	}
%%%}

\newVerweis        {\Aequivalenzrelation}  {\glstext}{Aequivalenzrelation}
\newVerweis[en]    {\Aequivalenzrelationen}{\glstext}{Aequivalenzrelation}
\longnewglossaryentry{Aequivalenzrelation}{
	name            ={Äquivalenzrelation \addIdx[
		name        ={Äquivalenzrelation}]           {Aequivalenzrelation}},
	text            ={Äquivalenzrelation},
}{\todoPruefen%
	Eine \GloFt{Äquivalenzrelation} ist eine \binaere\ \Relation\ auf einer \Menge\ $M$ mit folgenden Eigenschaften
	(dabei sei $\sim$ die \gloFt{Äquivalenzrelation}):
	\begin{align}
		&\text{\DefFt{reflexiv }}   &:&&\qquad  &a \sim a \\
		&\text{\DefFt{transitiv }}  &:&&\qquad((&a \sim b) \MtsAnd (b \sim c)) \MtsImp (a \sim c)\\
		&\text{\DefFt{symmetrisch }}&:&&\qquad (&a \sim b) \MtsImp (b \sim a)
		\formulatoleft
	\end{align}
	jeweils für alle \Elemente\ $a$, $b$ und $c$ aus $M$.
}

\newVerweis     {\Allquantor} {\glstext}{Allquantor}
\newglossaryentry{Allquantor}{
	name        ={Allquantor \addIdx    {Allquantor}},
	text        ={Allquantor},
	description ={\todoPruefen%
		Man nennt den \Quantor\ \MtsForall\ \textbzw\ \OjkForall\ auch \GloFt{Allquantor}.
	}
}

\newVerweis     {\Existenzquantor} {\glstext}{Existenzquantor}
\newglossaryentry{Existenzquantor}{
	name        ={Existenzquantor \addIdx    {Existenzquantor}},
	text        ={Existenzquantor},
	description ={\todoPruefen%
		Man nennt den \Quantor\ \MtsExists\ \textbzw\ \OjkExists\ auch \GloFt{Existenzquantor}.
	}
}

\newVerweis     {\Alphabet} {\glstext}{Alphabet}
\newcommand*    {\Alphabets}[1][]{\glstext[#1]{Alphabet}[s]}
\newglossaryentry{Alphabet}{
	name        ={Alphabet \addIdx    {Alphabet}},
	text        ={Alphabet},
	description ={\todoBeschreiben%
	}
}

\newVerweis     {\Anfangsregel}{\glstext}{Anfangsregel}
\newglossaryentry{Anfangsregel}{
	name        ={Anfangsregel \addIdx   {Anfangsregel}},
	text        ={Anfangsregel},
	description ={\todoPruefen%
		Die \Schlussregel\ \glsAR\ um anfangen zu können.
		\glsentrydesc{AR}
	}
}

\newVerweis     {\ASBA}{\glstext} {ASBA}
\newglossaryentry{ASBA}{
	name        ={ASBA \addIdx   {ASBA}},
	text        ={ASBA},
	description ={\todoPruefen%
		ist ein Akronym für „\DefFt{A}xiome, \DefFt{S}ätze, \DefFt{B}eweise und \DefFt{A}uswertungen“.
		Es bezeichnet das \hier\ beschriebene Programmsystem, das zu eingegebenen \Axiomen, \Saetzen\ und \Beweisen\ letztere prüft, Auswertungen generiert und unter Zuhilfenahme gegebener \Ausgabeschemata\ eine Ausgabe im \LaTeX-Format in mathematisch üblicher Schreibweise mit \Formeln\ erstellt.
	}
}

\newVerweis     {\atomar}  {\glstext}{atomar}
\newVerweis     {\Atomar}  {\Glstext}{atomar}
\newVerweis[e]  {\atomare} {\glstext}{atomar}
\newVerweis[e]  {\Atomare} {\glstext}{atomar}
\newVerweis[en] {\atomaren}{\glstext}{atomar}
\newVerweis[es] {\atomares}{\glstext}{atomar}
\newglossaryentry{atomar}{
	name        ={atomar \addIdx     {atomar}},
	text        ={atomar},
	see         ={zerlegbar},
	description ={\todoPruefen%
		Das Attribut \GloFt{atomar} kann auf \Aussagen, \Formeln\ und \Symbole\ angewendet werden.
		\Atomar\ sind solche, die keine echten \Teilobjekte\ gleicher \Objektart\ enthalten.
	}
}

\dummyVerweis   {\Ausdruck}{\glstext}{Ausdruck}% ToDo=Ausdruck --> Formel?

\dummyVerweis   {\logischerAusdruck}   {\glstext}  {logischer Ausdruck}% ToDo=Ausdruck, logischer
\dummyVerweis   {\logischenAusdruecke} {\glsuseri} {logischen Ausdrücke}
\dummyVerweis[n]{\logischenAusdruecken}{\glsuseri} {logischen Ausdrücke}
\dummyVerweis    {\logischeAusdruecke} {\glsuserii}{logische  Ausdrücke}

\dummyVerweis   {\metasprachlicherAusdruck}   {\glstext} {metasprachlicher Ausdruck}% ToDo=Ausdruck, metasprachlicher
\dummyVerweis[n]{\metasprachlichenAusdruecken}{\glsuseri}{metasprachlichen Ausdrücke}

\newVerweis     {\Ausgabeschema}  {\glstext}{Ausgabeschema}
\newVerweis[ta] {\Ausgabeschemata}{\glstext}{Ausgabeschema}
\newglossaryentry{Ausgabeschema}{
	name        ={Ausgabeschema \addIdx     {Ausgabeschema}},
	text        ={Ausgabeschema},
	description ={\todoGeprueft%
		Ein \GloFt{Ausgabeschema} ist für \ASBA\ eine Beschreibung, wie ein bestimmtes mathematisches \Objekt\ ausgegeben werden soll.
		Dies kann \textzB\ ein Stück \LaTeX-Code mit entsprechenden Parametern sein.
	}
}

\newVerweis         {\Aussage}{\glstext}{Aussage}
\newVerweis[n]     {\Aussagen}{\glstext}{Aussage}
\longnewglossaryentry{Aussage}{
	name            ={Aussage \addIdx   {Aussage}},
	text            ={Aussage},
}{\todoOk%
	\wikicite{bib:Aussage}{
		Eine \wikiBoldFt{Aussage} im Sinn der \wikiLinkFt{aristotelischen Logik} ist ein sprachliches Gebilde, von dem es sinnvoll ist zu \wikiItalicFt{fragen}, ob es \wikiLinkFt{wahr} oder falsch ist (so genanntes Aristotelisches \wikiLinkFt{Zweiwertigkeitsprinzip}). Es ist nicht erforderlich, \wikiItalicFt{sagen} zu können, ob das Gebilde wahr oder falsch ist. Es genügt, dass die Frage nach Wahrheit („Zutreffen“) oder Falschheit („Nicht-Zutreffen“) sinnvoll ist, – was zum Beispiel bei Fragesätzen, Ausrufen und Wünschen nicht der Fall ist. Aussagen sind somit Sätze, die \wikiLinkFt{Sachverhalte} beschreiben und denen man einen \wikiLinkFt{Wahrheitswert} zuordnen kann.
	}
	Dies gilt natürlich auch, wenn \metasprachlicheSymbole\ verwendet werden, wovon wir im Folgenden reichlich Gebrauch machen.
	Da man \Relationen\ und \logischenAusdruecken\ ebenfalls einen \Wahrheitswert\ zuordnen kann%
	\footnote{%
		Zumindest prinzipiell nach Ersetzung von \Variablen\ durch konkrete Werte.
	},
	können wir sie auch als \gloFt{Aussagen} behandeln.
	Es handelt sich dann um \defTxt{\logischeA}, im Gegensatz zu \defTxt{\metasprachlichenAussagen}.
}

\newVerweis     {\atomareAussage} {\glstext}{atomareAussage}
\newVerweis[n]  {\atomareAussagen}{\glstext}{atomareAussage}
\newglossaryentry{atomareAussage}{
	name       =            {---, atomare \addIdx[
		name   =            {---, atomare},
		sort   =        {Aussage, atomare}] {atomareAussage}},
	sort       =        {Aussage, atomare},
	text       ={atomare Aussage},
	description={\todoOk%
		Eine \Aussage\ heißt \defTxt{\atomar}\synonym{\defTxt{\unzerlegbar}}, wenn sie nicht \zerlegbar\ ist, \textdh\ wenn sie keine \echteTeilaussage\ enthält.
	}
}

\newVerweis      {\formaleAussage}  {\glstext} {formaleAussage}
\newVerweis[n]   {\formalenAussagen}{\glsuseri}{formaleAussage}
\newglossaryentry {formaleAussage}{
	name       =  {formaleAussage \addIdx      {formaleAussage}},
	name       =             {---, formale \addIdx[
		name   =             {---, formale},
		sort   =         {Aussage, formale}]   {formaleAussage}},
	sort       =         {Aussage, formale},
	text       ={formale  Aussage},
	user1      ={formalen Aussage},
	description={\todoOk%
		Eine \GloFt{formale Aussage} ist eine \Aussage\ in \Objektsprache.
	}
}

\newVerweis      {\metasprachlicheAussage} {\glstext}       {metasprachlicheAussage}
\newVerweis[n]   {\metasprachlicheAussagen}{\glstext}       {metasprachlicheAussage}
\newVerweis[n]  {\metasprachlichenAussagen}{\glsuseri}      {metasprachlicheAussage}
\newglossaryentry {metasprachlicheAussage}{
	name       =                     {---, metasprachliche \addIdx[
		name   =                     {---, metasprachliche},
		sort   =                 {Aussage, metasprachliche}]{metasprachlicheAussage}},
	sort       =                 {Aussage, metasprachliche},
	text       ={metasprachliche  Aussage},
	user1      ={metasprachlichen Aussage},
	description={\todoOk%
		Eine \GloFt{metasprachliche Aussage} ist eine \Aussage\ in \Metasprache.
	}
}

\newVerweis     {\logischeAussage} {\glstext} {logischeAussage}
\newVerweis[n]  {\logischeAussagen}{\glstext} {logischeAussage}
\newVerweis     {\logischeA}       {\glsuseri}{logischeAussage}
\newglossaryentry{logischeAussage}{
	name       =             {---, logische \addIdx[
		name   =             {---, logische},
		sort   =         {Aussage, logische}] {logischeAussage}},
	sort       =         {Aussage, logische},
	text       ={logische Aussage},
	user1      ={logische},
	description={\todoOk%
		\GloFt{Logische \Aussagen} sind \logischeAusdruecke, wozu auch Ergebnisse von \Relationen\ sowie Ergebnisse von \Funktionen\ mit \Wertebereich\ aus den \Wahrheitswerten\ gehören können.
	}
}

\newVerweis     {\parametrisierteAussage}{\glstext}        {parametrisierteAussage}
\newglossaryentry{parametrisierteAussage}{
	name       =                    {---, parametrisierte \addIdx[
		name   =                    {---, parametrisierte},
		sort   =                {Aussage, parametrisierte}]{parametrisierteAussage}},
	sort       =                {Aussage, parametrisierte},
	text       ={parametrisierte Aussage},
	user1      ={parametrisiert},
	description={\todoGeprueft
		Eine \Aussage\ heißt \GloFt{parametrisiert}, wenn sie mindestens einen \Parameter\ enthält.
	}
}

\newVerweis     {\zerlegbareAussage}{\glstext}   {zerlegbareAussage}
\newVerweis     {\zerlegbarA}       {\glsuseri}  {zerlegbareAussage}
\newglossaryentry{zerlegbareAussage}{
	name       =               {---, zerlegbare \addIdx[
		name   =               {---, zerlegbare},
		sort   =           {Aussage, zerlegbare}]{zerlegbareAussage}},
	sort       =           {Aussage, zerlegbare},
	text       ={zerlegbare Aussage},
	user1      ={zerlegbar},
	description={\todoOk%
		Eine \Aussage\ heißt \defTxt{\zerlegbar}\alternativi[%
			--- wir unterscheiden allerdings die beiden \Begriffe.
			Aus \zerlegbar\ folgt zusammengesetzt, aber nicht immer umgekehrt.
		]{zusammengesetzt}
		wenn sie mindestens eine \echteTeilaussage\ enthält.
	}
}

\newVerweis     {\Aussagedefinition}  {\glstext}{Aussagedefinition}
\newVerweis[en] {\Aussagedefinitionen}{\glstext}{Aussagedefinition}
\newglossaryentry{Aussagedefinition}{
	name        ={Aussagedefinition \addIdx     {Aussagedefinition}},
	text        ={Aussagedefinition},
	see         ={Objektdefinition},
	description ={\todoOk%
		Eine \Metadefinition: Die formale Definition einer \Aussage.
		\ifmarginparFlg\newline\else\fi
		\seqqt{$A \MtsDefEquiv B$} steht für \standsfor{$A$ ist \DefFt{definitionsgemäß äquivalent zu} $B$} für \Aussagen\ $A$ und $B$.
		Gewissermaßen ist $A$ nur eine andere Schreibweise für $B$.
	}
}

\newVerweis     {\Aussagenbereich}{\glstext} {Aussagenbereich}
\newglossaryentry{Aussagenbereich}{
	name        ={Aussagenbereich \addIdx    {Aussagenbereich}},
	text        ={Aussagenbereich},
	description ={\todoOk%
		Der \GloFt{Aussagenbereich} \MtsAussagen\ ist der \Bereich\ aller \formalenAussagen, \textdh\ der \Aussagen\ in \Objektsprache.
		Es kann $\MtsAussagen \MtsSubsetEq \MtsUniversum$ gelten, muss es aber nicht.
	}
}

\newVerweis         {\Aussagenlogik}{\glstext} {Aussagenlogik}
\newVerweis         {\AussagenL}    {\glsuseri}{Aussagenlogik}
\longnewglossaryentry{Aussagenlogik}{
	name            ={Aussagenlogik \addIdx    {Aussagenlogik}},
	text            ={Aussagenlogik},
	user1           ={Aussagen-},
	see             ={Aussage,Junktor,Logik,Praedikatenlogik,Wahrheitswert},
}{\todoPruefen%
	\wikicite{bib:Aussagenlogik}{
		Die \wikiBoldFt{Aussagenlogik} ist ein Teilgebiet der \wikiLinkFt{Logik}, das sich mit Aussagen und deren Verknüpfung durch \wikiLinkFt{Junktoren} befasst, ausgehend von strukturlosen \wikiLinkFt{Elementaraussagen} (Atomen), denen ein \wikiLinkFt{Wahrheitswert} zugeordnet wird. In der \wikiItalicFt{klassischen Aussagenlogik} wird jeder Aussage genau einer der zwei Wahrheitswerte „wahr“ und „falsch“ zugeordnet. Der Wahrheitswert einer zusammengesetzten Aussage lässt sich ohne zusätzliche Informationen aus den Wahrheitswerten ihrer Teilaussagen bestimmen.
	}
}

\newVerweis     {\Auswertung}  {\glstext}{Auswertung}
\newVerweis[en] {\Auswertungen}{\glstext}{Auswertung}
\newglossaryentry{Auswertung}{
	name        ={Auswertung \addIdx     {Auswertung}},
	text        ={Auswertung},
	description ={\todoOk%
		Eine \GloFt{Auswertung} ist für \ASBA\ eine statistische oder andere Auswertung, die bestimmten Elementen der Datei \textbzw\ Datenbank zugeordnet sind.
		\textZB\ können zu einem \Satz\ alle für einen \Beweis\ notwendigen \Axiome\ angegeben werden.
	}
}

\newVerweis     {\Axiom}  {\glstext}{Axiom}
\newVerweis[e]  {\Axiome} {\glstext}{Axiom}
\newVerweis[en] {\Axiomen}{\glstext}{Axiom}
\newglossaryentry{Axiom}{
	name        ={Axiom \addIdx     {Axiom}},
	text        ={Axiom},
	see         ={MtsAxiom,MtsAxiomSet},
	description ={\todoOk%
		Ein \GloFt{Axiom} ist eine \Aussage, die nicht aus anderen Aussagen abgeleitet werden kann.
		Es können wie bei \Saetzen\ \Praemissen\ und \Konklusionen\ vorhanden sein, aber keine \Beweise.
	}
}

\newVerweis     {\Axiomensystem} {\glstext}{Axiomensystem}
\newVerweis[e]  {\Axiomensysteme}{\glstext}{Axiomensystem}
\newglossaryentry{Axiomensystem}{
	name        ={Axiomensystem \addIdx    {Axiomensystem}},
	text        ={Axiomensysteme},
	description ={\todoGeprueft%
		Eine \Menge\ von \Axiomen.
	}
}

%B === B === B === B === B === B === B === B === B === B === B === B === B === B

\newVerweis     {\Basisregel} {\glstext}{Basisregel}
\newVerweis[n]  {\Basisregeln}{\glstext}{Basisregel}
\newglossaryentry{Basisregel}{
	name        ={Basisregel \addIdx    {Basisregel}},
	text        ={Basisregel},
	description ={\todoPruefen%
		Eine \Schlussregel, die nicht mehr auf andere zurückgeführt wird.
		Obwohl das auch auf die \Identitaetsregeln\ zutrifft, werden diese \hier\ aber nicht dazu gezählt.
	}
}

\newVerweis     {\Baustein} {\glstext}{Baustein}
\newVerweis[e]  {\Bausteine}{\glstext}{Baustein}
\newglossaryentry{Baustein}{
	name        ={Baustein \addIdx    {Baustein}},
	text        ={Baustein},
	description ={\todoBeschreiben%
	}
}

\newVerweis         {\Begriff}  {\glstext}{Begriff}
\newVerweis[e]      {\Begriffe} {\glstext}{Begriff}
\newVerweis[en]     {\Begriffen}{\glstext}{Begriff}
\longnewglossaryentry{Begriff}{
	name            ={Begriff \addIdx     {Begriff}},
	text            ={Begriff},
	see             ={Bezeichnung},
}{\todoOk%
	\wikicite{bib:Begriff}{
		Mit dem Ausdruck \wikiBoldFt{Begriff} (\wikiLinkFt{mittelhochdeutsch} und \wikiLinkFt{frühneuhochdeutsch} \wikiItalicFt{begrif} oder \wikiItalicFt{begrifunge}) ist allgemein der \wikiLinkFt{Bedeutungsinhalt} einer \wikiLinkFt{Bezeichnung} angesprochen. Die Abgrenzung zwischen Begriffen und rein gedanklichen (mentalen) Einheiten erfolgt jedoch oft unscharf: Teilweise wird ein \wikiItalicFt{Begriff} als „mentale Informationseinheit“ beschrieben, (also genauso wie in der Kognitionswissenschaft das Konzept). Präziser ist die Abgrenzung des \wikiItalicFt{Begriffes} als \wikiItalicFt{Konzept, das sprachlich benannt ist}, oder geradezu als die \wikiItalicFt{Kombination aus einer sprachlichen Bezeichnung und dem entsprechenden Konzept}.
	}
}

\newVerweis     {\Beispielsymbol}{\glstext}{Beispielsymbol}
\newglossaryentry{Beispielsymbol}{
	name        ={Beispielsymbol \addIdx   {Beispielsymbol}},
	text        ={Beispielsymbol},
	see         ={Symbol},
	description ={\todoBeschreiben%
	}
}

\dummyVerweis{\Belegung}{\glstext}{Belegung}% ToDo=Belegung (einer Sprache)

\newVerweis         {\Benennung}  {\glstext}{Benennung}
\newVerweis[en]     {\Benennungen}{\glstext}{Benennung}
\longnewglossaryentry{Benennung}{
	name            ={Benennung \addIdx     {Benennung}},
	text            ={Benennung},
	see             ={Bezeichnung,Symbol},
}{\todoOk%
	\wikicite{bib:Benennung}{
		Eine \wikiBoldFt{Benennung} ist die \wikiLinkFt{Bezeichnung} eines Gegenstandes durch ein \wikiLinkFt{Wort} oder mehrere Wörter.[1] Die Benennung gilt in der Sprachwissenschaft und in der \wikiLinkFt{Terminologielehre} als die sprachliche Form, mit der \wikiLinkFt{Begriffe} ins Bewusstsein gerufen werden.[2] Eine Benennung ist insofern die Versprachlichung einer Vorstellung.[2] Der weiter gefasste Oberbegriff \wikiItalicFt{Bezeichnung} beinhaltet demgegenüber, neben der \wikiItalicFt{Benennung}, auch nichtsprachliches, wie Nummern, Notationen und Symbole.[3] Bei einer \wikiLinkFt{fachsprachlichen} Benennung spricht man auch von einem \wikiLinkFt{Fachausdruck} oder Terminus.[2] Benennungen kommen als Einwort- und als \wikiLinkFt{Mehrwortbenennungen}, auch Mehrworttermini genannt, vor.
	}
}

\newVerweis     {\Bereich}  {\glstext}{Bereich}
\newVerweis[e]  {\Bereiche} {\glstext}{Bereich}
\newVerweis[en] {\Bereichen}{\glstext}{Bereich}
\newglossaryentry{Bereich}{
	name        ={Bereich \addIdx     {Bereich}},
	text        ={Bereich},
	see         ={Element,Klasse,leererBereich,Menge},
	description ={\todoOk%
		Ein \GloFt{Bereich} ist eine Zusammenfassung von \Aussagen\ und \Objekten.
		Für solche Zusammenfassungen brauchen wir nur wenige Eigenschaften, die explizit angegeben werden.
		Die in einem \gloFt{Bereich} zusammengefassten \Aussagen\ und \Objekte\ bezeichnen wir wie üblich als seine \DefFt{\Elemente}.
		\Klassen\ und \Mengen\ sind spezielle \gloFt{Bereiche}.%
		\footnote{In der Tat ist \MtsUniversum\ nur eine \Klasse\ und keine \Menge.}
	}
}

\newVerweis     {\Bereichsoperation}  {\glstext}{Bereichsoperation}
\newVerweis[en] {\Bereichsoperationen}{\glstext}{Bereichsoperation}
\newglossaryentry{Bereichsoperation}{
	name        ={Bereichsoperation \addIdx     {Bereichsoperation}},
	text        ={Bereichsoperation},
	description ={\todoOk%
		Eine \GloFt{Bereichsoperation} ist eine \Operation\ auf \Bereichen.
		Hoier sind <es die \Operationen\ \MtsCup, \MtsCap und \MtsTimes.
	}
}

\newVerweis     {\Bereichsrelation}  {\glstext}{Bereichsrelation}
\newVerweis[en] {\Bereichsrelationen}{\glstext}{Bereichsrelation}
\newglossaryentry{Bereichsrelation}{
	name        ={Bereichsrelation \addIdx     {Bereichsrelation}},
	text        ={Bereichsrelation},
	description ={\todoOk%
		Eine \GloFt{Bereichsrelationen} ist eine \Relation\ zwischen \Bereichen.
		\Hier\ sind es die acht \Relationen\ \MtsSubset, \MtsSubsetEq, \MtsSupset, \MtsSupsetEq, \MtsSubsetN, \MtsSubsetEqN, \MtsSupsetN und \MtsSupsetEqN.
	}
}

\newVerweis     {\beschraenkt}  {\glstext}{beschraenkt}
\newVerweis[e]  {\beschraenkte} {\glstext}{beschraenkt}
\newVerweis[en] {\beschraenkten}{\glstext}{beschraenkt}
\newglossaryentry{beschraenkt}{
	name        ={beschränkt \addIdx[
		name    ={beschränkt}]            {beschraenkt}},
	text        ={beschränkt},
	description ={\todoPruefen%
		Eine \Schlussregel\ heißt \beschraenkt, wenn sie nur endlich viele Prämissen und Konklusionen hat.
	}
}

\newVerweis         {\Beweis}  {\glstext}{Beweis}
\newVerweis[e]      {\Beweise} {\glstext}{Beweis}
\newVerweis[es]     {\Beweises}{\glstext}{Beweis}
\newVerweis[en]     {\Beweisen}{\glstext}{Beweis}
\longnewglossaryentry{Beweis}{
	name            ={Beweis \addIdx     {Beweis}},
	text            ={Beweis},
	see             ={Ableitung,Aussage,Axiom},
}{\todoOk%
	\wikicite{bib:Beweis}{
		Ein \wikiBoldFt{Beweis} ist in der Mathematik die als fehlerfrei anerkannte Herleitung der Richtigkeit bzw. der Unrichtigkeit einer \wikiBoldFt{Aussage} aus einer Menge von \wikiLinkFt{Axiomen}, die als wahr vorausgesetzt werden, und anderen Aussagen, die bereits bewiesen sind. Um den Beweis klar vom gültigen Schluss zu unterscheiden, spricht man auch vom \wikiBoldFt{axiomatischen Beweis}.

		Umfangreichere Beweise von mathematischen Sätzen werden in der Regel in mehrere kleine Teilbeweise aufgeteilt, siehe dazu \wikiLinkFt{Satz} und \wikiLinkFt{Hilfssatz}.

		In der \wikiLinkFt{Beweistheorie}, einem Teilgebiet der \wikiLinkFt{mathematischen Logik}, werden Beweise formal als \wikiLinkFt{Ableitungen} aufgefasst und selbst als mathematische Objekte betrachtet, um etwa die Beweisbarkeit oder Unbeweisbarkeit von Sätzen aus gegebenen Axiomen selbst zu beweisen.
	}
	Ein \GloFt{Beweis} besteht aus einer \Folge\ von \Beweisschritten, die aus gegebenen \Praemissen\ \Konklusionen\ ableitet.
}
\newsynonym{\beweisbar}{beweisbar}{\ableitbar}

\newVerweis     {\Beweisschritt}  {\glstext}{Beweisschritt}
\newVerweis[e]  {\Beweisschritte} {\glstext}{Beweisschritt}
\newVerweis[en] {\Beweisschritten}{\glstext}{Beweisschritt}
\newglossaryentry{Beweisschritt}{
	name        ={Beweisschritt \addIdx     {Beweisschritt}},
	text        ={Beweisschritt},
	see         ={MtsBeweisschritt,MtsBeweisschrittSet,MtsBeweisschrittTup},
	description ={
		Eine Vorschrift, wie aus vorgegebenen \Aussagen\ (den \Praemissen) weitere (die \Konklusionen) folgen.
	}
}

\newVerweis     {\Beweisschrittfolge} {\glstext}{Beweisschrittfolge}
\newVerweis[n]  {\Beweisschrittfolgen}{\glstext}{Beweisschrittfolge}
\newglossaryentry{Beweisschrittfolge}{
	name        ={Beweisschrittfolge \addIdx    {Beweisschrittfolge}},
	text        ={Beweisschrittfolge},
	description ={
		Eine Folge von \Beweisschritten.
	}
}

\newVerweis     {\Beweisschrittmenge} {\glstext}{Beweisschrittmenge}
\newVerweis[n]  {\Beweisschrittmengen}{\glstext}{Beweisschrittmenge}
\newglossaryentry{Beweisschrittmenge}{
	name        ={Beweisschrittmenge \addIdx    {Beweisschrittmenge}},
	text        ={Beweisschrittmenge},
	description ={
		Eine \Menge\ von \Beweisschritten, insbesondere die \Menge\ der Glieder einer \Beweisschrittfolge.
	}
}

\newVerweis         {\Bezeichnung}  {\glstext}{Bezeichnung}
\newVerweis[en]     {\Bezeichnungen}{\glstext}{Bezeichnung}
\longnewglossaryentry{Bezeichnung}{
	name            ={Bezeichnung \addIdx     {Bezeichnung}},
	text            ={Bezeichnung},
	see             ={Begriff,Benennung,Symbol},
}{\todoOk%
	\wikicite{bib:Bezeichnung}{
		Eine \wikiBoldFt{Bezeichnung} ist die Repräsentation eines \wikiLinkFt{Begriffs} mit sprachlichen oder anderen Mitteln. Erfolgt diese Repräsentation mittels Wörtern, handelt es sich um eine \wikiLinkFt{Benennung}. Eine nichtsprachliche Bezeichnung kann durch ein \wikiLinkFt{Symbol} erfolgen.
	}
}

\newVerweis     {\binaer}  {\glstext}{binaer}
\newVerweis[e]  {\binaere} {\glstext}{binaer}
\newVerweis[en] {\binaeren}{\glstext}{binaer}
\newglossaryentry{binaer}{
	name        ={binär \addIdx[
		name    ={binär}]            {binaer}},
	text        ={binär},
	see         ={unaer},
	description ={\todoPruefen%
		Eine \Operation, \Funktion\ oder \Relation\ heißt \GloFt{binär}, wenn ihre \Stelligkeit\ gleich 2 ist.
	}
}

%D === D === D === D === D === D === D === D === D === D === D === D === D === D

\newVerweis         {\Darstellung}  {\glstext}{Darstellung}
\newVerweis[en]     {\Darstellungen}{\glstext}{Darstellung}
\longnewglossaryentry{Darstellung}{
	name            ={Darstellung \addIdx     {Darstellung}},
	text            ={Darstellung},
}{\todoErgaenzen%
	\wikicite{bib:Darstellung}{
		Unter \wikiBoldFt{Darstellung} (zur semantischen Wurzel \wikiItalicFt{dar}- „öffentlich übergeben“, vergleiche Darbietung, \wikiLinkFt{Darlehen}, darreichen) versteht man die Umsetzung von \wikiLinkFt{Sachverhalten}, \wikiLinkFt{Ereignissen} oder abstrakten Konzepten mittels \wikiLinkFt{Zeichen}, performativer \wikiLinkFt{Handlungen} oder \wikiLinkFt{Modellen}. Historisch reicht die Darstellung von der \wikiLinkFt{mündlichen Überlieferung} über das \wikiLinkFt{Schauspiel} bis zur \wikiLinkFt{Computergrafik} und schließt zahlreiche Vermittlungsmethoden zwischen \wikiLinkFt{Text}, \wikiLinkFt{Bild} und künstlerischer \wikiLinkFt{Aufführung} ein.
	}
	Die \GloFt{Darstellung} mathematischer \Objekte\ geschieht auf mehreren Ebenen
}

\newVerweis     {\interneDarstellung}{\glstext} {interneDarstellung}
\newglossaryentry{interneDarstellung}{
	name       =                 {---, interne \addIdx[
		name   =                 {---, interne},
		sort   =         {Darstellung, interne}]{interneDarstellung}},
	sort       =         {Darstellung, interne},
	text       ={interne Darstellung},
	description={\todoBeschreiben%
	}
}

\newVerweis     {\logischeDarstellung}{\glstext} {logischeDarstellung}
\newVerweis    {\logischenD}          {\glsuseri}{logischeDarstellung}
\newglossaryentry{logischeDarstellung}{
	name       =                 {---, logische \addIdx[
		name   =                 {---, logische},
		sort   =         {Darstellung, logische}]{logischeDarstellung}},
	sort       =         {Darstellung, logische},
	text       ={logische Darstellung},
	user1      ={logischen},
	description={\todoBeschreiben%
	}
}

\newVerweis     {\Darstellungsweise} {\glstext}{Darstellungsweise}
\newVerweis[n]  {\Darstellungsweisen}{\glstext}{Darstellungsweise}
\newglossaryentry{Darstellungsweise}{
	name        ={Darstellungsweise \addIdx    {Darstellungsweise}},
	text        ={Darstellungsweise},
	description ={\todoPruefen%
		Die Art der \Darstellung\ mathematischer \Objekte.
	}
}

\newVerweis     {\Definitionsbereich} {\glstext} {Definitionsbereich}
\newVerweis[e]  {\Definitionsbereiche}{\glstext} {Definitionsbereich}
\newVerweis     {\DefinitionsB}       {\glsuseri}{Definitionsbereich}
\newglossaryentry{Definitionsbereich}{
	name        ={Definitionsbereich \addIdx     {Definitionsbereich}},
	text        ={Definitionsbereich},
	user1       ={Definitions},
	see         ={MtsDb,Quellbereich,Funktion},
	description ={\todoPruefen%
		Für eine \Funktion\ \FunktionDef{f}{A}{B} ist $\MtsDb(f)A$ ihr \Definitionsbereich\ (domain).
	}
}

\newVerweis     {\Differenz} {\glstext}{Differenz}
\newglossaryentry{Differenz}{
	name        ={Differenz \addIdx    {Differenz}},
	text        ={Differenz},
	description ={\todoErgaenzen%
		Eine \Bereichsoperation:
	}
}

\newVerweis         {\Diskursuniversum} {\glstext}{Diskursuniversum}
\longnewglossaryentry{Diskursuniversum}{
	name            ={Diskursuniversum \addIdx    {Diskursuniversum}},
	text            ={Diskursuniversum},
	symbol          ={\formaleDefinition{\RawMtsUniversum}},
	see             ={Aussage,Begriff,Logik},
}{\todoPruefen%
	\wikicite{bib:Diskursuniversum}{
		Unter einem \wikiBoldFt{Diskursuniversum} versteht man in der \wikiLinkFt{Logik} und \wikiLinkFt{Sprachphilosophie} die Gesamtheit der Gegenstände, auf die sich Aussagen wie „alle Gegenstände sind … “ (\wikiLinkFt{Allaussage}) oder „es gibt keine Gegenstände, die … sind“ (negative \wikiLinkFt{Existenzaussage}) beziehen. Solche Aussagen sind nur sinnvoll, wenn die Bedeutung von „Gegenstand“ auf einen bestimmten Bereich, das Diskursuniversum, eingeschränkt wird. Ausmaß und Art der Einschränkung hängen vom Inhalt und vom Zusammenhang der Aussagen ab. Es gibt daher nicht nur ein Diskursuniversum, sondern verschiedene Diskursuniversen.

		Der englische Ausdruck \wikiBoldFt{Universe of Discourse} wird auch in der deutschsprachigen Logik- und Informatikliteratur verwendet. Er geht auf \wikiLinkFt{Augustus De Morgan} (1847) zurück und bezeichnet den Bereich der Gegenstände (im weitesten Sinn), über die überhaupt geredet werden soll.

		Missverständnisse und Streit entstehen in der Logik wie im Alltag oft dadurch, dass Personen „aneinander vorbei“ von verschiedenen Dingen reden. Jemand behauptet z. B., dass es keine geflügelten Pferde gibt. Sein Widerpart weist dies mit dem Hinweis auf den \wikiLinkFt{Pegasus} zurück. Beide bewegen sich gedanklich in verschiedenen Welten. Ihr Streit lässt sich schlichten, wenn sie sich auf ein gemeinsames Diskursuniversum einigen, d. h. aushandeln, wovon die Rede (der \wikiLinkFt{Diskurs}) sein soll, ob nur von physisch existierenden Pferden oder auch von \wikiLinkFt{Fabelwesen}.

		Auch beim Gebrauch negativer (komplementärer) \wikiLinkFt{Begriffe} spielt das Diskursuniversum eine Rolle. Ausdrücke wie „Nichtschwimmer“, „Nichtfachmann“, „Nichtwähler“ können sinnvoll nur auf Personen angewandt werden. Die Nichtwähler bilden mit den Wählern zusammen das auf wahlberechtigte Personen eingeschränkte Diskursuniversum. Die Einschränkung geschieht beim Gebrauch solcher Begriffe automatisch. Wird die Automatik außer Betrieb gesetzt, indem man z. B. einen stillgelegten Schornstein als Nichtraucher bezeichnet, entsteht ein Wortspiel. Allgemein gilt für jeden Begriff: wird er mit dem zugehörigen negativen Begriff vereinigt (genauer: werden deren \wikiLinkFt{Extensionen} vereinigt), so bilden beide zusammen das Diskursuniversum oder den Bereich der Anwendungsfälle des positiv bestimmten Komplementärbegriffs:

		[eine Tabelle]

		In der \wikiLinkFt{Mengenlehre} entspricht dem Diskursuniversum die \wikiLinkFt{Grundmenge}, die Mengen entsprechen den Begriffen, die \wikiLinkFt{Komplemente} von Mengen der Negation von Begriffen. In der \wikiLinkFt{Prädikatenlogik} entspricht dem Diskursuniversum der Bereich der \wikiLinkFt{Definitionsmenge}, den die \wikiLinkFt{Gegenstandsvariable} einer \wikiLinkFt{quantifizierten} Aussage durchlaufen kann.

		Das \wikiItalicFt{Universe of Discourse} wird in der Logik zumeist abgekürzt mit \wikiItalicFt{U}, in der Informatik auch mit \wikiItalicFt{UoD}.

		Das \wikiItalicFt{U} ist in der Regel eine Teilmenge aller existierenden Objekte und insbesondere in der Prädikatenlogik der bei der Verwendung von \wikiLinkFt{Quantoren} festgelegte oder vorausgesetzte Objektbereich.
	}
	\SymbolAmRand{Diskursuniversum}
	Das \GloFt{Diskursuniversum} \MtsUniversum\ ist der vorgegebene \Bereich\ aller \Objekte, die in \Aussagen\ einen \Parameter\ ersetzen dürfen.
}

\newVerweis     {\Durchschnitt} {\glstext}{Durchschnitt}
\newglossaryentry{Durchschnitt}{
	name        ={Durchschnitt \addIdx    {Durchschnitt}},
	text        ={Durchschnitt},
	description ={\todoErgaenzen%
		Eine \Bereichsoperation:
	}
}

%E === E === E === E === E === E === E === E === E === E === E === E === E === E

\newVerweis     {\echt} {\glstext}{echt}
\newVerweis[e]  {\echte}{\glstext}{echt}
\newglossaryentry{echt}{
	name        ={echt \addIdx    {echt}},
	text        ={echt},
	description ={\todoGeprueft%
		Attribut für \Oberaussage, \Oberfolge, \Oberformel, \Oberobjekt, \Obermenge, \Obersprache, \Obersymbol, \Teilaussage, \Teilfolge,  \Teilformel, \Teilobjekt, \Teilmenge, \Teilsprache\ und \Teilsymbol.
	}
}

\newVerweis     {\Eigenschaft}  {\glstext}{Eigenschaft}
\newVerweis[en] {\Eigenschaften}{\glstext}{Eigenschaft}
\newglossaryentry{Eigenschaft}{
	name        ={Eigenschaft \addIdx   {Eigenschaft}},
	text        ={Eigenschaft},
	description ={\todoPruefen%
		Ist $x$ ein \Parameter\ einer \Aussage\ $A$, so ist die \Aussage\ \statement{$x$ hat die \Eigenschaft\ $A$} gleichbedeutend damit, das $A$ gilt.
		Wir schreiben etwas unpräzise auch $A(x)$, besonders dann, wenn auch $A(y)$ für $y \ne x$ von Interesse ist.
	}
}

% TODO ### ab hier weiter prüfen
\newVerweis      {\interessierendeEigenschaft}  {\glstext}      {interessierendeEigenschaft}
\newVerweis     {\interessierendenEigenschaft}  {\glsuseri}     {interessierendeEigenschaft}
\newVerweis[en] {\interessierendenEigenschaften}{\glsuseri}     {interessierendeEigenschaft}
\newglossaryentry {interessierendeEigenschaft}{
	name       =                 {Eigenschaft, interessierende \addIdx[
		name   =                 {Eigenschaft, interessierende},
		sort   =                 {Eigenschaft, interessierende}]{interessierendeEigenschaft}},
	sort       =                 {Eigenschaft, interessierende},
	text       ={interessierende  Eigenschaft},
	user1      ={interessierenden Eigenschaft},
	description={\todoPruefen%
		Solche Eigenschaften von \Objekten, die im aktuellen Zusammenhang von Interesse sind, \textzB\ einen bestimmten Wert zu haben, \Element\ aus einer bestimmten \Menge\ zu sein, ein bestimmtes \Objekt\ zu bezeichnen, usw.
	}
}

\newVerweis         {\Element}  {\glstext}{Element}
\newVerweis[e]      {\Elemente} {\glstext}{Element}
\newVerweis[en]     {\Elementen}{\glstext}{Element}
\longnewglossaryentry{Element}{
	name            ={Element \addIdx     {Element}},
	text            ={Element},
	see             ={Mengenlehre,Relation},
}{\todoOk%
	\wikicite{bib:Element}{
		Ein \wikiBoldFt{Element} in der \wikiLinkFt{Mathematik} ist immer im Rahmen der \wikiLinkFt{Mengenlehre} oder \wikiLinkFt{Klassenlogik} zu verstehen. Die grundlegende \wikiLinkFt{Relation}, wenn $x$ ein Element ist und $M$ eine \wikiLinkFt{Menge} oder \wikiLinkFt{Klasse} ist, lautet:
		\begin{quote}
			„$x$ ist Element von $M$“ oder mit Hilfe des \wikiLinkFt{Elementzeichens} „$x \in M$“.
		\end{quote}
		Die Mengendefinition von \wikiLinkFt{Georg Cantor} beschreibt anschaulich, was unter einem Element im Zusammenhang mit einer Menge zu verstehen ist:
		\begin{quote}
			„Unter einer ‚Menge‘ verstehen wir jede Zusammenfassung $M$ von bestimmten wohlunterschiedenen Objekten $m$ unserer Anschauung oder unseres Denkens (welche die ‚Elemente‘ von $M$ genannt werden) zu einem Ganzen.“
		\end{quote}
		Diese anschauliche Mengenauffassung der \wikiLinkFt{naiven Mengenlehre} erwies sich als nicht widerspruchsfrei. Heute wird daher eine \wikiLinkFt{axiomatische} Mengenlehre benutzt, meist die \wikiLinkFt{Zermelo-Fraenkel-Mengenlehre}, teilweise auch eine allgemeinere \wikiLinkFt{Klassenlogik}.
	}
	\Hier\ sind \GloFt{Elemente} stets \Aussagen\ oder \Objekte\ und wir schreiben immer \enquote{\gloFt{Element} \ManFt{aus}} und lassen neben \Mengen\ und \Klassen\ auch \Bereiche\ zu.
}

\newVerweis     {\Elementoperation}  {\glstext}{Elementoperation}
\newVerweis[en] {\Elementoperationen}{\glstext}{Elementoperation}
\newglossaryentry{Elementoperation}{
	name        ={Elementoperation \addIdx     {Elementoperation}},
	text        ={Elementoperation},
	description ={\todoBeschreiben%
	}
}

\newVerweis     {\Elementrelation}  {\glstext}{Elementrelation}
\newVerweis[en] {\Elementrelationen}{\glstext}{Elementrelation}
\newglossaryentry{Elementrelation}{
	name        ={Elementrelation \addIdx     {Elementrelation}},
	text        ={Elementrelation},
	see         ={Komponentenrelation},
	description ={\todoOk%
		Eine \GloFt{Elementrelation} ist eine \Relation\ zwischen einem \Element\ und einem \Bereich.
		Hier sind es die vier \Relationen\ \MtsIn, \MtsNi, \MtsInN und \MtsNiN.
	}
}

\newVerweis     {\Ergebnis}   {\glstext}{Ergebnis}
\newVerweis[se] {\Ergebnisse} {\glstext}{Ergebnis}
\newcommand*    {\Ergebnissen}[1][]{\glstext[#1]{Ergebnis}[sen]}
\newglossaryentry{Ergebnis}{
	name        ={Ergebnis \addIdx      {Ergebnis}},
	text        ={Ergebnis},
	see         ={MtsErgebnis,MtsErgebnisSet,MtsErgebnisRel},
	description ={\todoPruefen%
		Eine \Ableitung:
		Ein \Ergebnis\ eines \Beweises.
	}
}

\newVerweis     {\Ergebnismenge} {\glstext}{Ergebnismenge}
\newVerweis[n]  {\Ergebnismengen}{\glstext}{Ergebnismenge}
\newglossaryentry{Ergebnismenge}{
	name        ={Ergebnismenge \addIdx    {Ergebnismenge}},
	text        ={Ergebnismenge},
	description ={\todoPruefen%
		Eine \Ableitungsmenge:
		Die \Menge\ \MtsErgebnisSet\ der \Ergebnisse\ eines \Beweises.
	}
}

\newVerweis     {\Ersetzung}  {\glstext}{Ersetzung}
\newVerweis[en] {\Ersetzungen}{\glstext}{Ersetzung}
\newglossaryentry{Ersetzung}{
	name        ={Ersetzung \addIdx     {Ersetzung}},
	text        ={Ersetzung},
	description ={\todoPruefen%
		Eine \Funktion\ zur \Transformation\ einer \Formel\ mittels \Ersetzung\ in eine gleichwertige.
		Die \Ersetzung\ heißt \zulaessig, wenn sie vorgegebene Regeln erfüllt.
	}
}

\newVerweis     {\Ersetzungsmenge} {\glstext}{Ersetzungsmenge}
\newVerweis[n]  {\Ersetzungsmengen}{\glstext}{Ersetzungsmenge}
\newglossaryentry{Ersetzungsmenge}{
	name        ={Ersetzungsmenge \addIdx    {Ersetzungsmenge}},
	text        ={Ersetzungsmenge},
	description ={\todoPruefen%
		Eine \Menge\ von \Ersetzungen, meistens mit \MtsErsetzungSet\ bezeichnet.
	}
}

%F === F === F === F === F === F === F === F === F === F === F === F === F === F

\newVerweis         {\Fachbegriff}  {\glstext}{Fachbegriff}
\newVerweis[e]      {\Fachbegriffe} {\glstext}{Fachbegriff}
\newVerweis[en]     {\Fachbegriffen}{\glstext}{Fachbegriff}
\longnewglossaryentry{Fachbegriff}{
	name            ={Fachbegriff \addIdx     {Fachbegriff}},
	text            ={Fachbegriff},
	see             ={Begriff,Fachgebiet},
}{\todoOk%
	\wikicite{bib:Terminus}{
		Ein \wikiBoldFt{Terminus} oder \wikiBoldFt{Fachbegriff} ist eine \wikiLinkFt{definierte} \wikiLinkFt{Benennung} für einen \wikiLinkFt{Begriff} innerhalb der \wikiLinkFt{Fachsprache} eines \wikiLinkFt{Fachgebietes}. Synonyme dazu sind auch \wikiBoldFt{Term} oder \wikiBoldFt{Terminus technicus} (lateinisch \wikiItalicFt{terminus technicus}; \wikiLinkFt{Genus} \wikiItalicFt{m.}; \wikiLinkFt{Pl.} \wikiItalicFt{Termini technici}, kurz \wikiItalicFt{Termini}). \wikiItalicFt{Terminus} kann allerdings neben der rein sprachlichen \wikiItalicFt{Benennung} auch den Bedeutungsinhalt, den \wikiItalicFt{Begriff} selbst, ansprechen.

		Eine vergleichbare Bezeichnung ist \wikiBoldFt{Fachwort}. Ein \wikiBoldFt{Fachausdruck} ist ein \wikiLinkFt{sprachlicher Ausdruck}, der in einer Fachsprache verwendet wird und dort eine spezielle Bedeutung besitzt. \wikiItalicFt{Fachausdruck} gilt gegenüber \wikiItalicFt{Fachwort} als ein geeigneteres Ersatzwort für Terminus. Denn ein Terminus kann nicht nur in der Form einer Einwortbenennung, sondern auch als \wikiLinkFt{Mehrwortbenennung} (auch \wikiItalicFt{Mehrwortterminus}) vorliegen.

		Die Menge aller Termini eines Fachgebietes (die Benennungen aller Begriffe) bildet die jeweilige fachspezifische \wikiLinkFt{Terminologie} (den \wikiLinkFt{Fachwortschatz}). Mit der Untersuchung und Aufstellung von Terminologien beschäftigt sich die \wikiLinkFt{Terminologielehre}. Wenn ein Fachwortschatz standardisiert oder normiert ist, spricht man auch von einem \wikiLinkFt{Thesaurus} oder \wikiLinkFt{kontrollierten Vokabular} und nennt die darin enthaltenen Termini \wikiLinkFt{Deskriptoren}.
	}
	Ein \GloFt{Fachbegriff} ist für \ASBA\ eine \Benennung\ für einen \Begriff\ aus einem \Fachgebiet.
	Insbesondere kann es auch ein spezielles \Symbol\ sein.
}

\newVerweis         {\Fachgebiet}  {\glstext}{Fachgebiet}
\newVerweis[s]      {\Fachgebiets} {\glstext}{Fachgebiet}
\newVerweis[e]      {\Fachgebiete} {\glstext}{Fachgebiet}
\newVerweis[en]     {\Fachgebieten}{\glstext}{Fachgebiet}
\longnewglossaryentry{Fachgebiet}{
	name            ={Fachgebiet \addIdx     {Fachgebiet}},
	text            ={Fachgebiet},
}{\todoOk%
	\wikicite{bib:Fachgebiet}{
		\wikiBoldFt{Fachgebiet} (auch \wikiBoldFt{Fachbereich} oder \wikiBoldFt{Fachrichtung} oder \wikiBoldFt{Domäne}) ist das auf ein bestimmtes \wikiLinkFt{Wissensgebiet} begrenzte \wikiLinkFt{Wissen}.
	}
	Ein \GloFt{Fachgebiet} ist für \ASBA\ ein Teilgebiet der Mathematik mit einer zugehörigen Basis aus \Axiomen, \Saetzen, \Fachbegriffen\ und \Darstellungsweisen, \textzB\ \Logik\ und \Mengenlehre.

	Ein \GloFt{Fachgebiet} kann bei \ASBA\ sehr klein sein und im Extremfall kein einziges \Element\ enthalten.
	\emph{Umgebung} wäre vielleicht eine bessere \Bezeichnung, ist aber schon ein verbreiteter \Fachbegriff, so dass \hier\ die \Bezeichnung\ "`Fachgebiet"' verwendet wird.
}

\newVerweis         {\Folge} {\glstext}{Folge}
\newVerweis[n]      {\Folgen}{\glstext}{Folge}
\longnewglossaryentry{Folge}{
	name            ={Folge \addIdx    {Folge}},
	text            ={Folge},
}{\todoGeprueft%
	\wikicite{bib:Folge}{
		Als \wikiBoldFt{Folge} oder \wikiBoldFt{Sequenz} wird in der \wikiLinkFt{Mathematik} eine Auflistung (\wikiLinkFt{Familie}) von endlich oder unendlich vielen fortlaufend nummerierten Objekten (beispielsweise Zahlen) bezeichnet. Dasselbe Objekt kann in einer Folge auch mehrfach auftreten. Das Objekt mit der Nummer $i$, man sagt auch: mit dem Index $i$ i, wird $i$-tes Glied oder $i$-te Komponente der Folge genannt. Endliche wie unendliche Folgen finden sich in allen Bereichen der Mathematik. Mit unendlichen Folgen, deren Glieder Zahlen sind, beschäftigt sich vor allem die \wikiLinkFt{Analysis}.

		Ist $n$ die Anzahl der Glieder einer endlichen Folge, so spricht man von einer Folge der Länge $n$, einer $n$-gliedrigen Folge oder von einem $n$-Tupel. Die Folge ohne Glieder, deren Index-Bereich also leer ist, wird leere Folge, 0-gliedrige Folge oder 0-Tupel genannt.
	}
%%%%	Ein \GloFt{Folge}\alternativi{Sequenz} $\vec{a}$ ist eine Aneinanderreihung ihrer \defTxt{\Komponenten} $a_i$, $i \in \MtsINo$, geschrieben $[a_1, a_2, \dots]$.
%%%%	Sind alle \Komponenten\ \Elemente\ aus einer \Menge\ $M$, so heißt $\vec{a}$ eine \GloFt{Folge} \DefFt{auf} $M$ oder \DefFt{von} \Elementen\ aus $M$.
%%%%	Hat die \GloFt{Folge} nur endlich viele \Komponenten, so heißt sie \DefFt{endlich} und die Anzahl $\MtsLen(\vec{a})$ ihrer \Komponenten\ ihre \DefFt{Länge}.
%%%%	Ist die Länge gleich $0$, so sprechen wir von der \defTxt{\leerenFolge} und bezeichnen sie mit \seqqt{$()$}.
%%%%	Eine endliche \GloFt{Folge} der Länge $n$ heißt auch \DefFt{$n$-\Tupel} und die leere \GloFt{Folge} demnach \DefFt{$0$-\Tupel}.
}

\newVerweis      {\leereFolge} {\glstext} {leereFolge}
\newVerweis[n]   {\leereFolgen}{\glstext} {leereFolge}
\newVerweis     {\leerenFolge} {\glsuseri}{leereFolge}
\newglossaryentry {leereFolge}{
	name       =         {---, leere \addIdx[
		name   =         {---, leere},
		sort   =       {Folge, leere}]    {leereFolge}},
	sort       =       {Folge, leere},
	text       ={leere  Folge},
	user1      ={leeren Folge},
	see        ={MtsLen,Folge,Tupel},
	description={\todoPruefen%
		Eine \Folge\ heißt \GloFt{leer}, wenn ihre Länge $0$ ist, \textdh\ wenn sie keine \Komponenten\ besitzt.
	}
}

\newVerweis     {\Folgenmenge}{\glstext}{Folgenmenge}
\newglossaryentry{Folgenmenge}{
	name        ={Folgenmenge \addIdx   {Folgenmenge}},
	text        ={Folgenmenge},
	description ={\todoBeschreiben%
	}
}

\newVerweis     {\Folgenoperation}  {\glstext}{Folgenoperation}
\newVerweis[en] {\Folgenoperationen}{\glstext}{Folgenoperation}
\newglossaryentry{Folgenoperation}{
	name        ={Folgenoperation \addIdx     {Folgenoperation}},
	text        ={Folgenoperation},
	description ={\todoBeschreiben%
	}
}

\newVerweis     {\Folgenrelation}  {\glstext}{Folgenrelation}
\newVerweis[en] {\Folgenrelationen}{\glstext}{Folgenrelation}
\newglossaryentry{Folgenrelation}{
	name        ={Folgenrelation \addIdx     {Folgenrelation}},
	text        ={Folgenrelation},
	description ={\todoBeschreiben%
	}
}

\newcommand*{\Folgerungen}[1][]{\glstext[#1]{Folgerung}[en]}
\newsynonym{\Folgerung}{Folgerung}{\Konklusion}

\newsynonym{\Folgerungsmenge}{Folgerungsmenge}{\Konklusionsmenge}

\newVerweis     {\Formationsregel} {\glstext}{Formationsregel}
\newVerweis[n]  {\Formationsregeln}{\glstext}{Formationsregel}
\newglossaryentry{Formationsregel}{
	name        ={Formationsregel \addIdx    {Formationsregel}},
	text        ={Formationsregel},
	description ={\todoBeschreiben%
	}
}

\newVerweis     {\Formel} {\glstext}{Formel}
\newVerweis[n]  {\Formeln}{\glstext}{Formel}
\newglossaryentry{Formel}{
	name        ={Formel \addIdx    {Formel}},
	text        ={Formel},
	description ={\todoPruefen%
		Unter einer \GloFt{Formel} verstehen wir stets eine mathematische Formel.
		Diese kann aus einem einzigen \Symbol\ bestehen (\atomare\ \gloFt{Formel}), andererseits aber auch mehrdimensional sein, lässt sich dann aber mittels geeigneter Definitionen immer eindeutig als eine \Symbolfolge\ schreiben.
		%%% besser: Formel = \Element\ aus einer Sprache?
	}
}

\newVerweis     {\allgemeingueltigeFormel}{\glstext}         {allgemeingueltigeFormel}
\newVerweis    {\allgemeingueltigenFormel}{\glsuseri}        {allgemeingueltigeFormel}
\newglossaryentry{allgemeingueltigeFormel}{
	name       =                     {---, allgemeingültige \addIdx[
		name   =                     {---, allgemeingültige},
		sort   =                  {Formel, allgemeingültige}]{allgemeingueltigeFormel}},
	sort       =                  {Formel, allgemeingültige},
	text       ={allgemeingültige  Formel},
	user1      ={allgemeingültigen Formel},
	description={\todoPruefen%
		Eine \Formel\ heißt \GloFt{allgemeingültig}, wenn sie aus den \Axiomen\ und \allgemeingueltigenSchlussregeln\ abgeleitet werden kann.
	}
}

\newVerweis      {\aussagenlogischeFormel} {\glstext}        {aussagenlogischeFormel}
\newVerweis[n]   {\aussagenlogischeFormeln}{\glstext}        {aussagenlogischeFormel}
\newVerweis     {\aussagenlogischenFormel} {\glsuseri}       {aussagenlogischeFormel}
\newVerweis[n]  {\aussagenlogischenFormeln}{\glsuseri}       {aussagenlogischeFormel}
\newVerweis      {\aussagenlogischeF}      {\glsuserii}      {aussagenlogischeFormel}
\newglossaryentry {aussagenlogischeFormel}{
	name       =                     {---, aussagenlogische \addIdx[
		name   =                     {---, aussagenlogische},
		sort   =                  {Formel, aussagenlogische}]{aussagenlogischeFormel}},
	sort       =                  {Formel, aussagenlogische},
	text       ={aussagenlogische  Formel},
	user1      ={aussagenlogischen Formel},
	user2      ={aussagenlogische},
	description={\todoPruefen%
		Eine \Formel\ heißt \GloFt{aussagenlogisch}, wenn sie ein \Element\ aus \OjkFor\ ist.
	}
}

\newVerweis      {\praedikatenlogischeFormel} {\glstext}           {praedikatenlogischeFormel}
\newVerweis[n]   {\praedikatenlogischeFormeln}{\glstext}           {praedikatenlogischeFormel}
\newglossaryentry {praedikatenlogischeFormel}{
	name       =                        {---, praedikatenlogische \addIdx[
		name   =                        {---, praedikatenlogische},
		sort   =                     {Formel, praedikatenlogische}]{praedikatenlogischeFormel}},
	sort       =                     {Formel, praedikatenlogische},
	text       ={praedikatenlogische  Formel},
	description={\todoPruefen%
		Eine \Formel\ heißt \GloFt{prädikatenlogisch}, wenn sie ein \Element\ aus \OjkFor\ ist.%TODO andere Menge
	}
}

\newVerweis     {\Formelmenge} {\glstext}{Formelmenge}
\newVerweis[n]  {\Formelmengen}{\glstext}{Formelmenge}
\newglossaryentry{Formelmenge}{
	name        ={Formelmenge \addIdx    {Formelmenge}},
	text        ={Formelmenge},
	description ={\todoPruefen%
		Eine \Menge\ von \Formeln, oft mit \MtsSprache\ bezeichnet.
		Man nennt \MtsSprache\ auch eine \Sprache\ und ihre \Elemente\ \Woerter, insbesondere dann, wenn es eindeutige Regeln zur Konstruktion von \MtsSprache\ gibt.
		Wir bevorzugen „\Formel“ und „\Formelmenge“.
	}
}

\newVerweis      {\MtsFktSep}    {\glsuseri} {Funktion}
\newVerweis      {\MtsFktArrow}  {\glsuserii}{Funktion}
\newVerweis         {\Funktion}  {\glstext}  {Funktion}
\newVerweis[en]     {\Funktionen}{\glstext}  {Funktion}
\longnewglossaryentry{Funktion}{
	name            ={Funktion \addIdx       {Funktion}},
	text            ={Funktion},
	user1           ={:},
	user2           ={\ensuremath{\RawMtsFktArrow}},
	see             ={Abbildung,Element,Menge,Objekt,Relation},
}{\todoPruefen%
	\wikicite{bib:Funktion}{
		In der \wikiLinkFt{Mathematik} ist eine \wikiBoldFt{Funktion} (lateinisch \wikiItalicFt{functio}) oder \wikiBoldFt{Abbildung} eine Beziehung (\wikiLinkFt{Relation}) zwischen zwei \wikiLinkFt{Mengen}, die jedem Element der einen Menge (Funktionsargument, unabhängige Variable, $x$-Wert) genau ein Element der anderen Menge (Funktionswert, abhängige Variable, $y$-Wert) zuordnet. Der Funktionsbegriff wird in der Literatur unterschiedlich definiert, jedoch geht man generell von der Vorstellung aus, dass Funktionen \wikiLinkFt{mathematischen Objekten} mathematische Objekte zuordnen, zum Beispiel jeder reellen Zahl deren Quadrat.  Das Konzept der Funktion oder Abbildung nimmt in der modernen Mathematik eine zentrale Stellung ein; es enthält als Spezialfälle unter anderem \wikiLinkFt{parametrische Kurven}, Skalar- und \wikiLinkFt{Vektorfelder}, \wikiLinkFt{Transformationen}, \wikiLinkFt{Operationen}, \wikiLinkFt{Operatoren} und vieles mehr.
	}
	Eine \GloFt{$n$-\stellige\ Funktion} $f$ von einer \Menge\ $A = A_1 \MtsTimes \dots \MtsTimes A_n$, dem \Definitionsbereich, in eine \Menge\ $B$, den \Zielbereich, ist eine ($n$+1)-\stellige\ \Relation\ $(G,A_1,\dots,A_n,B)$ derart, dass es für jedes $\vec{a} = (a_1,\dots,a_n)$ mit $a_i \in A_i$ genau ein $b \in B$ gibt mit $(a_1,\dots,a_n,b) \in f$.
	Dieses $b$ wird auch mit \seqqt{$f(a_1,\dots,a_n)$} , \seqqt{$f a_1 \dots a_n$} , \seqqt{$f(\vec{a})$} oder \seqqt{$f\vec{a}$} bezeichnet.
	\\Schreibweise: \seqqt{\FunktionDef{f}{A}{B}} \textbzw\ \seqqt{$\FunktionDef{f}{A_1 \MtsTimes \dots \MtsTimes A_n}{B}$}
}

\newVerweis     {\Funktionssymbol}  {\glstext}{Funktionssymbol}
\newVerweis[e]  {\Funktionssymbole} {\glstext}{Funktionssymbol}
\newVerweis[en] {\Funktionssymbolen}{\glstext}{Funktionssymbol}
\newglossaryentry{Funktionssymbol}{
	name        ={Funktionssymbol \addIdx     {Funktionssymbol}},
	text        ={Funktionssymbol},
	description ={\todoPruefen%
		Ein \Symbol\ für eine \Funktion.
	}
}

\newVerweis     {\Funktionswert} {\glstext}{Funktionswert}
\newVerweis[e]  {\Funktionswerte}{\glstext}{Funktionswert}
\newglossaryentry{Funktionswert}{
	name        ={Funktionswert \addIdx    {Funktionswert}},
	text        ={Funktionswert},
	description ={\todoPruefen%
		einer \Funktion.
	}
}

%G === G === G === G === G === G === G === G === G === G === G === G === G === G

\newVerweis     {\Gleichheit}{\glstext}{Gleichheit}
\newglossaryentry{Gleichheit}{
	name        ={Gleichheit \addIdx   {Gleichheit}},
	text        ={Gleichheit},
	description ={\todoPruefen%
		Eine \Gleichheitsrelation:
		Zwei Objekte $A$ und $B$ sind \DefFt{gleich} (dasselbe; identisch), $A \MtsEq B$, wenn sie in den \interessierendenEigenschaften\ für \MtsEq\ übereinstimmen.
	}
}

\newVerweis     {\Gleichheitsrelation}  {\glstext}{Gleichheitsrelation}
\newVerweis[en] {\Gleichheitsrelationen}{\glstext}{Gleichheitsrelation}
\newglossaryentry{Gleichheitsrelation}{
	name        ={Gleichheitsrelation \addIdx     {Gleichheitsrelation}},
	text        ={Gleichheitsrelation},
	description ={\todoPruefen%
%%%		Eine mit \Gleichheit\ verwandte \Relation: \MtsEq, \MtsEqN, \MtsAequiv\ und \MtsAequivN.
		Eine mit \Gleichheit\ verwandte \Relation: \MtsEq und \MtsEqN.
	}
}

\newVerweis     {\Gliederungszeichen}{\glstext}{Gliederungszeichen}
\newglossaryentry{Gliederungszeichen}{
	name        ={Gliederungszeichen \addIdx   {Gliederungszeichen}},
	text        ={Gliederungszeichen},
	description ={\todoBeschreiben%
	}
}

\newVerweis     {\Graph}  {\glstext}{Graph}
\newVerweis[en] {\Graphen}{\glstext}{Graph}
\newglossaryentry{Graph}{
	name        ={Graph \addIdx     {Graph}},
	text        ={Graph},
	see      ={MtsGraph},
	description ={\todoPruefen%
		einer \Funktion\ oder \Relation.
	}
}

%I === I === I === I === I === I === I === I === I === I === I === I === I === I

\newVerweis     {\Identitaetsregel} {\glstext}{Identitaetsregel}
\newVerweis[n]  {\Identitaetsregeln}{\glstext}{Identitaetsregel}
\newglossaryentry{Identitaetsregel}{
	name        ={Identitätsregel \addIdx[
		name    ={Identitätsregel}]           {Identitaetsregel}},
	text        ={Identitätsregel},
	description ={\todoPruefen%
		Eigentlich eine \Basisregel\ zur Identität.
		Da die \Identitaetsregeln\ nur zur Rechtfertigung der \Ersetzung\ verwendet werden, werden sie \hier\ nicht zu den \Basisregeln\ gezählt.
	}
}

%J === J === J === J === J === J === J === J === J === J === J === J === J === J

\newVerweis         {\Junktor}  {\glstext}{Junktor}
\newVerweis[en]     {\Junktoren}{\glstext}{Junktor}
\longnewglossaryentry{Junktor}{
	name            ={Junktor \addIdx     {Junktor}},
	text            ={Junktor},
	see             ={Metajunktor},
}{\todoPruefen%
	\wikicite{bib:Junktor}{
		Ein \wikiBoldFt{Junktor} (von \wikiLinkFt{lat.} \wikiItalicFt{iungere} „verknüpfen, verbinden“) ist eine \wikiLinkFt{logische Verknüpfung} zwischen Aussagen innerhalb der \wikiLinkFt{Aussagenlogik}, also ein logischer \wikiLinkFt{Operator}. Junktoren werden auch Konnektive, Konnektoren, Satzoperatoren, Satzverknüpfer, Satzverknüpfungen, Aussagenverknüpfer, logische Bindewörter, Verknüpfungszeichen oder Funktoren genannt und als \wikiLinkFt{logische Partikel} klassifiziert.

		Sprachlich wird zwischen der jeweiligen Verknüpfung selbst (zum Beispiel der \wikiLinkFt{Konjunktion}) und dem sie bezeichnenden Wort beziehungsweise Sprachzeichen (zum Beispiel dem Wort „und“ beziehungsweise dem Zeichen „\OjkAnd“) oft nicht unterschieden.
	}
	Ein \GloFt{Junktor} ist eine \aussagenlogischeOperation\ oder -\aRelation.
	Da die Werte einer aussagenlogischen \Operation\ \Wahrheitswerte\ sind, kann man einen \Junktor\ auch stets als \Relation\ verstehen.
}

\newVerweis     {\binaererJunktor}  {\glstext} {binaererJunktor}
\newVerweis[en] {\binaerenJunktoren}{\glsuseri}{binaererJunktor}
\newglossaryentry{binaererJunktor}{
	name        =            {---, binärer \addIdx[
		name    =            {---, binärer},
		sort    =        {Junktor, binärer}]   {binaererJunktor}},
	sort        =        {Junktor, binärer},
	text        ={binärer Junktor},
	user1       ={binären Junktor},
	description ={\todoBeschreiben%
	}
}

\newVerweis     {\unaererJunktor}  {\glstext} {unaererJunktor}
\newVerweis[en] {\unaerenJunktoren}{\glsuseri}{unaererJunktor}
\newglossaryentry{unaererJunktor}{
	name        =           {---, unärer \addIdx[
		name    =           {---, unärer},
		sort    =       {Junktor, unärer}]    {unaererJunktor}},
	sort        =       {Junktor, unärer},
	text        ={unärer Junktor},
	user1       ={unären Junktor},
	description ={\todoBeschreiben%
	}
}

\newVerweis     {\Junktorsymbol} {\glstext}{Junktorsymbol}
\newVerweis[e]  {\Junktorsymbole}{\glstext}{Junktorsymbol}
\newglossaryentry{Junktorsymbol}{
	name        ={Junktorsymbol \addIdx    {Junktorsymbol}},
	text        ={Junktorsymbol},
	description ={\todoPruefen%
		Ein \Symbol\ für einen \Junktor.
	}
}

%K === K === K === K === K === K === K === K === K === K === K === K === K === K

\newVerweis         {\Kalkuel}{\glstext}{Kalkuel}
\longnewglossaryentry{Kalkuel}{
	name            ={Kalkuel \addIdx   {Kalkuel}},
	text            ={Kalkül},
	see             ={Axiom,Logik},
}{\todoErgaenzen%
	\wikicite{bib:Kalkuel}{
		Als der oder das \wikiBoldFt{Kalkül} (französisch \wikiItalicFt{calcul} „Rechnung“; von \wikiLinkFt{lateinisch} \wikiItalicFt{calculus} „\wikiLinkFt{Rechenstein}“, „\wikiLinkFt{Spielstein}“) versteht man in den formalen Wissenschaften wie \wikiLinkFt{Logik} und \wikiLinkFt{Mathematik} ein System von Regeln, mit denen sich aus gegebenen Aussagen (\wikiLinkFt{Axiomen}) weitere Aussagen ableiten lassen. Kalküle, auf eine Logik selbst angewandt, werden auch Logikkalküle genannt.
	}
	\todo{Beschreibung fehlt noch}
}

\newVerweis     {\Klammerung}{\glstext}{Klammerung}
\newglossaryentry{Klammerung}{
	name        ={Klammerung \addIdx   {Klammerung}},
	text        ={Klammerung},
	description ={\todoBeschreiben%
	}
}

\newVerweis         {\Klasse} {\glstext}{Klasse}
\newVerweis[n]      {\Klassen}{\glstext}{Klasse}
\longnewglossaryentry{Klasse}{
	name            ={Klasse \addIdx    {Klasse}},
	text            ={Klasse},
	see             ={Menge,Mengenlehre}
}{\todoOk%
	\wikicite{bib:Klasse}{
		Als \wikiBoldFt{Klasse} gilt in der \wikiLinkFt{Mathematik}, \wikiLinkFt{Klassenlogik} und \wikiLinkFt{Mengenlehre} eine Zusammenfassung beliebiger Objekte, definiert durch eine logische Eigenschaft, die alle Objekte der Klasse erfüllen. Vom Klassenbegriff ist der Mengenbegriff zu unterscheiden. Nicht alle Klassen sind automatisch auch Mengen, weil Mengen zusätzliche Bedingungen erfüllen müssen. Mengen sind aber stets Klassen und werden daher auch in der Praxis in Klassenschreibweise notiert.
	}
	Eine \GloFt{Klasse} ist eine \Bereich, deren \Elemente\ genau die \Objekte\ mit einer bestimmten \Eigenschaft\ sind.
	%TODO Eigenschaft(x) definieren
	Schreibweise: \MengeDef{x}{Eigenschaft(x)}
	-- Jede \Menge\ ist auch eine \gloFt{Klasse} und jede \gloFt{Klasse} ein \Bereich.
}

\newVerweis         {\Klassenlogik} {\glstext}{Klassenlogik}
\longnewglossaryentry{Klassenlogik}{
	name            ={Klassenlogik \addIdx    {Klassenlogik}},
	text            ={Klassenlogik},
	see             ={Klasse,Logik},
}{\todoErgaenzen%
	\wikicite{bib:Klassenlogik}{
		Die \wikiBoldFt{Klassenlogik} ist im weiteren Sinn eine \wikiLinkFt{Logik}, deren Objekte als Klassen bezeichnet werden. Im engeren Sinn spricht man von einer Klassenlogik nur dann, wenn \wikiLinkFt{Klassen} durch eine Eigenschaft ihrer Elemente beschrieben werden. Diese Klassenlogik ist daher eine Verallgemeinerung der \wikiLinkFt{Mengenlehre}, die nur eine eingeschränkte Klassenbildung erlaubt.
	}
}

\newVerweis     {\Komponente} {\glstext}{Komponente}
\newVerweis[n]  {\Komponenten}{\glstext}{Komponente}
\newglossaryentry{Komponente}{
	name        ={Komponente \addIdx    {Komponente}},
	text        ={Komponente},
	see         ={Folge,Tupel},
	description ={\todoPruefen%
		Die \Komponenten\ einer \Folge\ $\vec{a} = (a_1, a_2, \dots)$ sind die $a_i$.
		$a_i$ heißt die \GloFt{$i$-te \Komponente} von $\vec{a}$.
	}
}

\newVerweis     {\Komponentenmenge}  {\glstext}{Komponentenmenge}
\newglossaryentry{Komponentenmenge}{
	name        ={Komponentenmenge \addIdx     {Komponentenmenge}},
	text        ={Komponentenmenge},
	see         ={Menge},
	description ={\todoPruefen%
		$\MtsSet(\vec{a}) \MtsDefEq \RawMengeDef{a}{a \MtsSeqIn \vec{a}}$ ist die \GloFt{Komponentenmenge} einer \Folge\ \textbzw\ eines \Tupels\ $\vec{a}$.
	}
}

\newVerweis     {\Komponentenrelation}  {\glstext}{Komponentenrelation}
\newVerweis[en] {\Komponentenrelationen}{\glstext}{Komponentenrelation}
\newglossaryentry{Komponentenrelation}{
	name        ={Komponentenrelation \addIdx     {Komponentenrelation}},
	text        ={Komponentenrelation},
	see         ={Elementrelation},
	description ={\todoPruefen%
		Eine \GloFt{Komponentenrelation} ist eine Relation zwischen einer (möglichen) \Komponente\ und einer \Folge: \MtsSeqIn, \MtsSeqNi, \MtsSeqInN und \MtsSeqNiN
	}
}

\newVerweis     {\Konklusion}  {\glstext}{Konklusion}
\newVerweis[en] {\Konklusionen}{\glstext}{Konklusion}
\newglossaryentry{Konklusion}{
	name        ={Konklusion \addIdx     {Konklusion}},
	text        ={Konklusion},
	see         ={Schlussregel},
	description ={\todoPruefen%
		Eine \Ableitung:
		Die \Konklusionen\ einer \Schlussregel\ $\frac{\MtsPraemisseSet}{\MtsKonklusionSet}$ \textbzw\ $\frac{\MtsPraemisseSet}{\MtsKonklusionSet}$ sind die \Elemente\ aus \MtsKonklusionSet\ \textbzw\ \MtsKonklusionRel.
		Die \Konklusionen\ werden normalerweise mit $\MtsKonklusion_i$ bezeichnet.
	}
}

\newVerweis     {\Konklusionsmenge} {\glstext}{Konklusionsmenge}
\newVerweis[n]  {\Konklusionsmengen}{\glstext}{Konklusionsmenge}
\newglossaryentry{Konklusionsmenge}{
	name        ={Konklusionsmenge \addIdx    {Konklusionsmenge}},
	text        ={Konklusionsmenge},
	description ={\todoPruefen%
		Eine \Ableitungsmenge:
		Die \Menge\ \MtsKonklusionSet\ der \Konklusionen\ einer \Schlussregel\ \textbzw\ eines \Beweises.
	}
}

\newVerweis         {\Konstante} {\glstext}{Konstante}
\newVerweis[n]     {\Konstanten}{\glstext}{Konstante}
\longnewglossaryentry{Konstante}{
	name            ={Konstante \addIdx    {Konstante}},
	text            ={Konstante},
	see             ={Symbol,Variable},
}{\todoPruefen%
	\wikicite{bib:Konstante}{
		Allgemein ist eine \wikiBoldFt{Konstante} (von \wikiLinkFt{lateinisch} \wikiItalicFt{constans} „feststehend“) ein Zeichen beziehungsweise ein Sprachausdruck mit einer „genau bestimmte[n]Bedeutung, die im Laufe der Überlegungen unverändert bleibt“[1]. Die Konstante ist damit ein Gegenbegriff zur \wikiLinkFt{Variablen}.
	}
}

\newVerweis      {\aussagenlogischeKonstante} {\glstext}        {aussagenlogischeKonstante}
\newVerweis     {\aussagenlogischenKonstante} {\glsuseri}       {aussagenlogischeKonstante}
\newVerweis[n]  {\aussagenlogischenKonstanten}{\glsuseri}       {aussagenlogischeKonstante}
\newglossaryentry {aussagenlogischeKonstante}{
	name       =                        {---, aussagenlogische \addIdx[
		name   =                        {---, aussagenlogische},
		sort   =                  {Konstante, aussagenlogische}]{aussagenlogischeKonstante}},
	sort       =                  {Konstante, aussagenlogische},
	text       ={aussagenlogische  Konstante},
	user1      ={aussagenlogischen Konstante},
	description={\todoPruefen%
		Eine \Konstante\ heißt \GloFt{aussagenlogisch}, wenn sie ein \Element\ aus \OjkCon\ ist.
	}
}

\newVerweis     {\Kontraposition}{\glstext}{Kontraposition}
\newglossaryentry{Kontraposition}{
	name        ={Kontraposition \addIdx   {Kontraposition}},
	text        ={Kontraposition},
	description ={\todoPruefen%
		Die allgemeingültige \Aussage: $ (\alpha \OjkImp \beta) \OjkImp (\OjkNot\beta \OjkImp \OjkNot\alpha) $.
	}
}

%%%\newVerweis     {\Kontravalenz}{\glstext}{Kontravalenz}
%%%\newglossaryentry{Kontravalenz}{
%%%	name        ={Kontravalenz \addIdx   {Kontravalenz}},
%%%	text        ={Kontravalenz},
%%%	description ={\todoPruefen%
%%%		Eine \Gleichheitsrelation:
%%%		Zwei Objekte $A$ und $B$ sind \DefFt{nicht äquivalent} (nicht ähnlich), $A \MtsAequivN B$, wenn sie in mindestens einer \interessierendenEigenschaft\ für \MtsAequiv\ nicht übereinstimmen.
%%%	}
%%%}

%L === L === L === L === L === L === L === L === L === L === L === L === L === L

\newVerweis         {\Logik}  {\glstext}{Logik}
\newVerweis[en]     {\Logiken}{\glstext}{Logik}
\longnewglossaryentry{Logik}{
	name            ={Logik \addIdx     {Logik}},
	text            ={Logik},
	see             ={atomar,Aussage,Aussagenlogik,Praedikatenlogik,Schlussregel},
}{\todoGeprueft%
	\wikicite{bib:Logik}{
		Mit \wikiBoldFt{Logik} (von \wikiLinkFt{altgriechisch} [\textdots]‚denkende Kunst‘, ‚Vorgehensweise‘) oder auch \wikiBoldFt{Folgerichtigkeit} wird im Allgemeinen das \wikiLinkFt{vernünftige Schlussfolgern} und im Besonderen dessen Lehre – die \wikiBoldFt{Schlussfolgerungslehre} oder auch \wikiBoldFt{Denklehre} – bezeichnet. In der Logik wird die Struktur von \wikiLinkFt{Argumenten} im Hinblick auf ihre \wikiLinkFt{Gültigkeit} untersucht, unabhängig vom Inhalt der \wikiLinkFt{Aussagen}. Bereits in diesem Sinne spricht man auch von „formaler“ Logik. Traditionell ist die Logik ein Teil der \wikiLinkFt{Philosophie}. Ursprünglich hat sich die traditionelle Logik in Nachbarschaft zur \wikiLinkFt{Rhetorik} entwickelt. Seit dem 20. Jahrhundert versteht man unter Logik überwiegend {symbolische Logik}, die auch als grundlegende \wikiLinkFt{Strukturwissenschaft}, z. B. innerhalb der \wikiLinkFt{Mathematik} und der \wikiLinkFt{theoretischen Informatik}, behandelt wird.

		Die moderne symbolische Logik verwendet statt der \wikiLinkFt{natürlichen Sprache} eine \wikiLinkFt{künstliche Sprache} (Ein Satz wie \wikiItalicFt{Der Apfel ist rot} wird z. B. in der \wikiLinkFt{Prädikatenlogik} als $f(a)$ formalisiert, wobei $a$ für \wikiItalicFt{Der Apfel} und $f$ für \wikiItalicFt{ist rot} steht) und verwendet streng \wikiLinkFt{definierte Schlussregeln}. Ein einfaches Beispiel für ein solches \wikiLinkFt{formales System} ist die \wikiLinkFt{Aussagenlogik} (dabei werden sogenannte \wikiLinkFt{atomare Aussagen} durch Buchstaben ersetzt). Die symbolische Logik nennt man auch \wikiLinkFt{mathematische Logik} oder formale Logik im engeren Sinn.
	}
}

\newVerweis         {\mathematischeLogik}{\glstext}       {mathematischeLogik}
\longnewglossaryentry{mathematischeLogik}{
	name            =                {---, mathematische \addIdx[
		name        =                {---, mathematische},
		sort        =              {Logik, mathematische}]{mathematischeLogik}},
	sort            =              {Logik, mathematische},
	text            ={mathematische Logik},
	see             ={Mengenlehre,Fachgebiet},
}{\todoGeprueft%
	\wikicite{bib:mathematischeLogik}{
		Die \wikiBoldFt{mathematische Logik}, auch \wikiBoldFt{symbolische Logik}, (alternativer Sprachgebrauch auch \wikiItalicFt{Logistik}), ist ein Teilgebiet der \wikiLinkFt{Mathematik}, insbesondere als Methode der \wikiLinkFt{Metamathematik} und eine Anwendung der modernen \wikiLinkFt{formalen Logik}. Oft wird sie wiederum in die Teilgebiete \wikiLinkFt{Modelltheorie}, \wikiLinkFt{Beweistheorie}, \wikiLinkFt{Mengenlehre} und \wikiLinkFt{Rekursionstheorie} aufgeteilt. Forschung im Bereich der mathematischen Logik hat zum Studium der \wikiLinkFt{Grundlagen der Mathematik} beigetragen und wurde auch durch dieses motiviert. Infolgedessen wurde sie auch unter dem Begriff \wikiItalicFt{Metamathematik} bekannt.

		Ein Aspekt der Untersuchungen der mathematischen Logik ist das Studium der Ausdrucksstärke von formalen Logiken und formalen \wikiLinkFt{Beweissystemen}. Eine Möglichkeit, die \wikiLinkFt{Komplexität} solcher Systeme zu messen, besteht darin, festzustellen, was damit bewiesen oder definiert werden kann.

		Früher wurde die mathematische Logik auch \wikiItalicFt{symbolische Logik} (als Gegensatz zur \wikiLinkFt{philosophischen Logik}) genannt, wobei jener Name mittlerweile nur noch für gewisse Aspekte der \wikiLinkFt{Beweistheorie} verwendet wird.
	}
}

%M === M === M === M === M === M === M === M === M === M === M === M === M === M

\newVerweis      {\MtsSetSep}{\glsuseri}{Menge}
\newVerweis         {\Menge} {\glstext} {Menge}
\newVerweis[n]      {\Mengen}{\glstext} {Menge}
\longnewglossaryentry{Menge}{
	name            ={Menge \addIdx     {Menge}},
	text            ={Menge},
	user1           ={\ensuremath{\RawMtsSetSep}},
	see             ={Berich,Element,Folge,leereMenge,Mengenlehre,Tupel},
}{\todoGeprueft%
	\wikicite{bib:Menge}{
		Eine \wikiBoldFt{Menge} ist ein Verbund, eine Zusammenfassung von einzelnen \wikiLinkFt{Elementen}. Die \wikiItalicFt{Menge} ist eines der wichtigsten und grundlegenden Konzepte der Mathematik, mit ihrer Betrachtung beschäftigt sich die \wikiLinkFt{Mengenlehre}.

		Bei der Beschreibung einer Menge geht es ausschließlich um die Frage, welche Elemente in ihr enthalten sind. Es wird nicht danach gefragt, ob ein Element mehrmals enthalten ist oder ob es eine Reihenfolge unter den Elementen gibt. Eine Menge muss kein Element enthalten – es gibt genau eine Menge ohne Elemente, die „\wikiLinkFt{leere Menge}“. In der Mathematik sind die Elemente einer Menge häufig Zahlen, Punkte eines \wikiLinkFt{Raumes} oder ihrerseits Mengen. Das Konzept ist jedoch auf beliebige Objekte anwendbar: z. B. in der \wikiLinkFt{Statistik} auf Stichproben, in der Medizin auf Patientenakten, am Marktstand auf eine Tüte mit Früchten.

		Ist die Reihenfolge der Elemente von Bedeutung, dann spricht man von einer endlichen oder unendlichen \wikiLinkFt{Folge}, wenn sich die Folgenglieder mit den natürlichen Zahlen aufzählen lassen (das erste, das zweite, usw.). Endliche Folgen heißen auch \wikiLinkFt{Tupel}. In einem Tupel oder einer Folge können Elemente auch mehrfach vorkommen. Ein Gebilde, das wie eine Menge Elemente enthält, wobei es zusätzlich auf die Anzahl der Exemplare jedes Elements ankommt, jedoch nicht auf die Reihenfolge, heißt \wikiLinkFt{Multimenge}.
	}
	Eine \GloFt{Menge} ist eine \Klasse\ mit zusätzlichen Eigenschaften.
}

\newVerweis     {\leereMenge}{\glstext}{leereMenge}
\newglossaryentry{leereMenge}{
	name       =        {---, leere \addIdx[
		name   =        {---, leere},
		sort   =      {Menge, leere}]  {leereMenge}},
	sort       =      {Menge, leere},
	text       ={leere Menge},
	description={\todoPruefen%
		\MtsEmptyset, die \GloFt{leere Menge}, ist die einzige \Menge\ ohne \Elemente.
		Sie wird auch mit \seqqt{$\{\}$} bezeichnet.
	}
}

\newVerweis         {\Mengenlehre}{\glstext}{Mengenlehre}
\longnewglossaryentry{Mengenlehre}{
	name            ={Mengenlehre \addIdx   {Mengenlehre}},
	text            ={Mengenlehre},
	see             ={Axiom,Fachgebiet,Menge,Objekt},
}{\todoPruefen%
	\wikicite{bib:Mengenlehre}{
		Die \wikiBoldFt{Mengenlehre} ist ein grundlegendes \wikiLinkFt{Teilgebiet der Mathematik}, das sich mit der Untersuchung von \wikiLinkFt{Mengen}, also von Zusammenfassungen von \wikiLinkFt{Objekten}, beschäftigt. Die gesamte Mathematik, wie sie heute üblicherweise gelehrt wird, ist in der Sprache der Mengenlehre formuliert und baut auf den \wikiLinkFt{Axiomen der Mengenlehre} auf. Die meisten mathematischen Objekte, die in Teilbereichen wie \wikiLinkFt{Algebra}, \wikiLinkFt{Analysis}, \wikiLinkFt{Geometrie}, \wikiLinkFt{Stochastik} oder \wikiLinkFt{Topologie} behandelt werden, um nur einige wenige zu nennen, lassen sich als Mengen definieren. Gemessen daran ist die Mengenlehre eine recht junge Wissenschaft; erst nach der Überwindung der \wikiLinkFt{Grundlagenkrise der Mathematik} im frühen 20. Jahrhundert konnte die Mengenlehre ihren heutigen, zentralen und grundlegenden Platz in der Mathematik einnehmen.
	}
}

\newVerweis     {\Mengenoperation}  {\glstext}{Mengenoperation}
\newVerweis[en] {\Mengenoperationen}{\glstext}{Mengenoperation}
\newglossaryentry{Mengenoperation}{
	name        ={Mengenoperation \addIdx     {Mengenoperation}},
	text        ={Mengenoperation},
	description ={\todoBeschreiben%
	}
}

\newsynonym{\Mengenprodukt}{Mengenprodukt}{\kartesischesProdukt}

\newVerweis     {\Mengenrelation}  {\glstext}{Mengenrelation}
\newVerweis[en] {\Mengenrelationen}{\glstext}{Mengenrelation}
\newglossaryentry{Mengenrelation}{
	name        ={Mengenrelation \addIdx     {Mengenrelation}},
	text        ={Mengenrelation},
	description ={\todoBeschreiben%
	}
}

\newVerweis     {\Metadefinition}  {\glstext}{Metadefinition}
\newVerweis[en] {\Metadefinitionen}{\glstext}{Metadefinition}
\newglossaryentry{Metadefinition}{
	name        ={Metadefinition \addIdx     {Metadefinition}},
	text        ={Metadefinition},
	description ={\todoPruefen%
		Eine \Metaoperation: Die formale Definition einer \Aussage\ (\Aussagedefinition) \textbzw\ eines \Objekts\ (\Objektdefinition).
	}
}

\newVerweis     {\Metaformel} {\glstext}{Metaformel}
\newVerweis[n]  {\Metaformeln}{\glstext}{Metaformel}
\newglossaryentry{Metaformel}{
	name        ={Metaformel \addIdx    {Metaformel}},
	text        ={Metaformel},
	description ={\todoPruefen%
		Eine \Formel\ der \formalenMetasprache.
	}
}

\newVerweis     {\Metajunktor}  {\glstext}{Metajunktor}
\newVerweis[en] {\Metajunktoren}{\glstext}{Metajunktor}
\newglossaryentry{Metajunktor}{
	name        ={Metajunktor \addIdx     {Metajunktor}},
	text        ={Metajunktor},
	see         ={Junktor},
	description ={\todoBeschreiben%
	}
}

\newVerweis     {\Metaoperation}  {\glstext} {Metaoperation}
\newVerweis[en] {\Metaoperationen}{\glstext} {Metaoperation}
\newVerweis[en]    {\Moperationen}{\glsuseri}{Metaoperation}
\newglossaryentry{Metaoperation}{
	name        ={Metaoperation \addIdx      {Metaoperation}},
	text        ={Metaoperation},
	user1       =    {operation},
	see         ={Objektoperation},
	description ={\todoPruefen%
		Eine \Operation\ der \Metasprache: \MtsAnd, \MtsOr\ oder \MtsUnd.
	}
}

\newVerweis     {\Metarelation}  {\glstext} {Metarelation}
\newVerweis[en] {\Metarelationen}{\glstext} {Metarelation}
\newVerweis[en]    {\Mrelationen}{\glsuseri}{Metarelation}
\newglossaryentry{Metarelation}{
	name        ={Metarelation \addIdx      {Metarelation}},
	text        ={Metarelation},
	user1       =    {relation},
	see         ={Objektrelation},
	description ={\todoPruefen%
		Eine \Relation\ der \Metasprache: \MtsImp, \MtsRep\ oder \MtsEquiv.
	}
}

\newVerweis     {\Metasprache} {\glstext}{Metasprache}
\newVerweis[n]  {\Metasprachen}{\glstext}{Metasprache}
\newglossaryentry{Metasprache}{
	name        ={Metasprache \addIdx    {Metasprache}},
	text        ={Metasprache},
	see         ={Objektsprache},
	description ={\todoOk%
		Die \Sprache, in der \Aussagen\ über eine andere \Sprache\ getroffen werden können.
		\Hier\ ist dies immer die normale Umgangssprache.
		Ihre \Syntax\ ist gegeben, \textbzgl\ der \Semantik\ bemühen wir uns um exakte Definitionen der \Begriffe\ und \Bezeichnungen.
	}
}

\dummyVerweis   {\Modell}{\glstext}{Modell}% ToDo=Modell

\newVerweis      {\formaleMetasprache}{\glstext}  {formaleMetasprache}
\newVerweis     {\formalenMetasprache}{\glsuseri} {formaleMetasprache}
\newVerweis     {\formalenM}          {\glsuserii}{formaleMetasprache}
\newglossaryentry {formaleMetasprache}{
	name       =                 {---, formale \addIdx[
		name   =                 {---, formale},
		sort   =         {Metasprache, formale}]  {formaleMetasprache}},
	sort       =         {Metasprache, formale},
	text       ={formale  Metasprache},
	user1      ={formalen Metasprache},
	user2      ={formalen},
	description={\todoOk%
		Die \Metasprache, deren Ausdrucksmittel nur \atomare\ \Aussagen\ und definierte \Metasymbole\ sind.
		\Hier\ ist ihre Syntax und Semantik passend für \ASBA\ definiert, in der Regel parallel zur \Praedikatenlogik.
	}
}

\newVerweis     {\Metasymbol} {\glstext}{Metasymbol}
\newVerweis[e]  {\Metasymbole}{\glstext}{Metasymbol}
\newglossaryentry{Metasymbol}{
	name        ={Metasymbol \addIdx    {Metasymbol}},
	text        ={Metasymbol},
	see         ={Objektsymbol},
	description ={\todoPruefen%
		Ein \Symbol\ der \formalenMetasprache.
	}
}

\newVerweis     {\Metavariable} {\glstext} {Metavariable}
\newVerweis[n]     {\Mvariablen}{\glsuseri}{Metavariable}
\newglossaryentry{Metavariable}{
	name        ={Metavariable \addIdx     {Metavariable}},
	text        ={Metavariable},
	user1       =    {variable},
	description ={\todoPruefen%
		Eine \Variable\ der \formalenMetasprache.
	}
}

\newVerweis     {\Monotonieregel}{\glstext}{Monotonieregel}
\newglossaryentry{Monotonieregel}{
	name        ={Monotonieregel \addIdx   {Monotonieregel}},
	text        ={Monotonieregel},
	see         ={MR},
	description ={\todoPruefen%
		Eine \Schlussregel.
	}
}

%N === N === N === N === N === N === N === N === N === N === N === N === N === N

\newVerweis     {\Negation}  {\glstext}{Negation}
\newVerweis[en] {\Negationen}{\glstext}{Negation}
\newglossaryentry{Negation}{
	name        ={Negation \addIdx     {Negation}},
	text        ={Negation},
	description ={\todoPruefen%
		Die \GloFt{Negation} \emph{von} einer \binaeren\ \Relation\ $(G,A,B)$ ist die \Relation\ $(H,A,B)$ mit $H = (A \MtsTimes B) \MtsSetminus G\}$.
		Üblicherweise wird das zugehörige \Relationssymbol\ mit einem schrägen oder vertikalen Strich durchgestrichen.
		Die \gloFt{Negation} der \Umkehrrelation\ einer \Relation\ ist gleich der \Umkehrrelation\ ihrer \gloFt{Negation}.
	}
}

%O === O === O === O === O === O === O === O === O === O === O === O === O === O

\newVerweis     {\Oberaussage} {\glstext}{Oberaussage}
\newVerweis[n]  {\Oberaussagen}{\glstext}{Oberaussage}
\newglossaryentry{Oberaussage}{
	name        ={Oberaussage \addIdx    {Oberaussage}},
	text        ={Oberaussage},
	description ={\todoOk%
		Eine \Aussage\ $A$ ist genau dann eine \GloFt{Oberaussage} einer \Aussage\ $B$, wenn $B$ eine \Teilaussage\ von $A$ ist.
	}
}

\newVerweis      {\echteOberaussage}{\glstext} {echteOberaussage}
\newVerweis     {\echtenOberaussage}{\glsuseri}{echteOberaussage}
\newglossaryentry {echteOberaussage}{
	name       =               {---, echte \addIdx[
		name   =               {---, echte},
		sort   =       {Oberaussage, echte}]   {echteOberaussage}},
	sort       =       {Oberaussage, echte},
	text       ={echte  Oberaussage},
	user1      ={echten Oberaussage},
	description={\todoOk%
		Eine \Aussage\ $A$ ist genau dann eine \GloFt{echte Oberaussage} einer \Aussage\ $B$, wenn $B$ eine \echteTeilaussage\ von $A$ ist.
	}
}

\newVerweis     {\Oberbereich} {\glstext} {Oberbereich}
\newVerweis[n]  {\Oberbereichn}{\glstext} {Oberbereich}
\newglossaryentry{Oberbereich}{
	name        ={Oberbereich \addIdx     {Oberbereich}},
	text        ={Oberbereich},
	see         ={Teilbereich},
	description ={\todoOk%
		Ein \Bereich\ $A$ ist ist genau dann ein \GloFt{Oberbereich} von einem \Bereich\ $B$, wenn $A \MtsSupsetEq B$ ist.
	}
}

\newVerweis     {\echterOberbereich}{\glstext}  {echterOberbereich}
\newglossaryentry{echterOberbereich}{
	name       =               {---, echter \addIdx[
		name   =               {---, echter},
		sort   =       {Oberbereich, echter}]   {echterOberbereich}},
	sort       =       {Oberbereich, echter},
	text       ={echter Oberbereich},
	see        ={echterTeilbereich},
	description={\todoOk%
		Ein \Bereich\ $A$ ist ist genau dann ein \GloFt{echter Oberbereich} von einem \Bereich\ $B$, wenn $A \MtsSupset B$ ist.
	}
}

\newVerweis     {\Oberfolge} {\glstext}{Oberfolge}
\newVerweis[n]  {\Oberfolgen}{\glstext}{Oberfolge}
\newglossaryentry{Oberfolge}{
	name        ={Oberfolge \addIdx    {Oberfolge}},
	text        ={Oberfolge},
	description ={\todoPruefen%
		Eine \Folge\ $A$ ist genau dann eine \GloFt{Oberfolge} einer \Folge\ $B$, wenn $B$ eine \Teilfolge\ von $A$ ist.
	}
}

\newVerweis      {\echteOberfolge}{\glstext} {echteOberfolge}
\newVerweis     {\echtenOberfolge}{\glsuseri}{echteOberfolge}
\newglossaryentry {echteOberfolge}{
	name       =              {---, echte \addIdx[
		name   =              {---, echte},
		sort   =       {Oberfolge, echte}]   {echteOberfolge}},
	sort       =       {Oberfolge, echte},
	text       ={echte  Oberfolge},
	user1      ={echten Oberfolge},
	description={\todoPruefen%
		Eine \Folge\ $A$ ist genau dann eine \GloFt{echte Oberfolge} einer \Folge\ $B$, wenn $B$ eine \echteTeilfolge\ von $A$ ist.
	}
}

\newVerweis     {\Oberformel} {\glstext}{Oberformel}
\newVerweis[n]  {\Oberformeln}{\glstext}{Oberformel}
\newglossaryentry{Oberformel}{
	name        ={Oberformel \addIdx    {Oberformel}},
	text        ={Oberformel},
	description ={\todoPruefen%
		Eine \Formel\ $A$ ist genau dann eine \GloFt{Oberformel} einer \Formel\ $B$, wenn $B$ eine \Teilformel\ von $A$ ist.
	}
}

\newVerweis      {\echteOberformel}{\glstext} {echteOberformel}
\newVerweis     {\echtenOberformel}{\glsuseri}{echteOberformel}
\newglossaryentry {echteOberformel}{
	name       =              {---, echte \addIdx[
		name   =              {---, echte},
		sort   =       {Oberformel, echte}]   {echteOberformel}},
	sort       =       {Oberformel, echte},
	text       ={echte  Oberformel},
	user1      ={echten Oberformel},
	description={\todoPruefen%
		Eine \Formel\ $A$ ist genau dann eine \GloFt{echte Oberformel} einer \Formel\ $B$, wenn $B$ eine \echteTeilformel\ von $A$ ist.
	}
}

\newVerweis     {\Obermenge} {\glstext}{Obermenge}
\newVerweis[n]  {\Obermengen}{\glstext}{Obermenge}
\newglossaryentry{Obermenge}{
	name        ={Obermenge \addIdx    {Obermenge}},
	text        ={Obermenge},
	see         ={Oberbereich,Teilmenge},
	description ={\todoOk%
		Eine \Menge\ $A$ ist genau dann eine \GloFt{Obermenge} von einer \Menge\ $B$, wenn $A \MtsSupsetEq B$ ist.
	}
}

\newVerweis      {\echteObermenge}{\glstext}  {echteObermenge}
\newVerweis     {\echtenObermenge}{\glsuseri} {echteObermenge}
\newVerweis      {\echteOM}       {\glsuserii}{echteObermenge}
\newglossaryentry {echteObermenge}{
	name       =             {---, echte \addIdx[
		name   =             {---, echte},
		sort   =       {Obermenge, echte}]    {echteObermenge}},
	sort       =       {Obermenge, echte},
	text       ={echte  Obermenge},
	user1      ={echten Obermenge},
	user2      ={echte},
	see        ={echterOberbereich,echteTeilmenge},
	description={\todoOk%
		Eine \Menge\ $A$ ist genau dann eine \GloFt{echte Obermenge} von einer \Menge\ $B$, wenn $A \MtsSupset B$ ist.
	}
}

\newVerweis     {\Oberobjekt} {\glstext}{Oberobjekt}
\newVerweis[e]  {\Oberobjekte}{\glstext}{Oberobjekt}
\newglossaryentry{Oberobjekt}{
	name        ={Oberobjekt \addIdx    {Oberobjekt}},
	text        ={Oberobjekt},
	description ={\todoPruefen%
		Eine \Objekt\ $A$ ist genau dann ein \GloFt{Oberobjekt} eines \Objekts\ $B$, wenn $B$ ein \Teilobjekt\ von $A$ ist.
	}
}

\newVerweis     {\echtesOberobjekt}{\glstext} {echtesOberobjekt}
\newVerweis     {\echtenOberobjekt}{\glsuseri}{echtesOberobjekt}
\newglossaryentry{echtesOberobjekt}{
	name       =              {---, echtes \addIdx[
		name   =              {---, echtes},
		sort   =       {Oberobjekt, echtes}]  {echtesOberobjekt}},
	sort       =       {Oberobjekt, echtes},
	text       ={echtes Oberobjekt},
	user1      ={echten Oberobjekt},
	description={\todoPruefen%
		Ein \Objekt\ $A$ ist genau dann ein \GloFt{echtes Oberobjekt} eines \Objekts\ $B$, wenn $B$ ein \echtesTeilobjekt\ von $A$ ist.
	}
}

\newVerweis     {\Obersprache} {\glstext} {Obersprache}
\newVerweis[e]  {\Obersprachee}{\glstext} {Obersprache}
\newglossaryentry{Obersprache}{
	name        ={Obersprache \addIdx     {Obersprache}},
	text        ={Obersprache},
	user1       =    {sprache},
	description ={\todoPruefen%
		Eine \Sprache\ $A$ ist genau dann eine \GloFt{Obersprache} einer \Sprache\ $B$, wenn $B$ eine \Teilsprache\ von $A$ ist.
	}
}

\newVerweis     {\echteObersprache}{\glstext}  {echteObersprache}
\newglossaryentry{echteObersprache}{
	name        =               {---, echte \addIdx[
		name    =               {---, echte},
		sort    =       {Obersprache, echte}]  {echteObersprache}},
	sort        =       {Obersprache, echte},
	text        ={echte Obersprache},
	user1       ={echten Obersprache},
	user2       =           {sprache},
	description={\todoPruefen%
		Eine \Sprache\ $A$ ist genau dann eine \GloFt{echte Obersprache} einer \Sprache\ $B$, wenn $B$ eine \echteTeilsprache\ von $A$ ist.
	}
}

\newVerweis     {\Obersymbol} {\glstext}{Obersymbol}
\newVerweis[e]  {\Obersymbole}{\glstext}{Obersymbol}
\newglossaryentry{Obersymbol}{
	name        ={Obersymbol \addIdx    {Obersymbol}},
	text        ={Obersymbol},
	description ={\todoPruefen%
		Eine \Symbol\ $A$ ist genau dann ein \GloFt{Obersymbol} eines \Symbols\ $B$, wenn $B$ ein \Teilsymbol\ von $A$ ist.
	}
}

\newVerweis     {\echtesObersymbol}{\glstext} {echtesObersymbol}
\newVerweis     {\echtenObersymbol}{\glsuseri}{echtesObersymbol}
\newglossaryentry{echtesObersymbol}{
	name       =              {---, echtes \addIdx[
		name   =              {---, echtes},
		sort   =       {Obersymbol, echtes}]  {echtesObersymbol}},
	sort       =       {Obersymbol, echtes},
	text       ={echtes Obersymbol},
	user1      ={echten Obersymbol},
	description={\todoPruefen%
		Eine \Symbol\ $A$ ist genau dann ein \GloFt{echtes Obersymbol} eines \Symbols\ $B$, wenn $B$ ein \echtesTeilsymbol\ von $A$ ist.
	}
}

\newVerweis         {\Objekt}  {\glstext}{Objekt}
\newVerweis[e]      {\Objekte} {\glstext}{Objekt}
\newVerweis[s]      {\Objekts} {\glstext}{Objekt}
\newVerweis[en]     {\Objekten}{\glstext}{Objekt}
\longnewglossaryentry{Objekt}{
	name            ={Objekt \addIdx     {Objekt}},
	text            ={Objekt},
}{\todoOk%
	\wikicite{bib:mathematischesObjekt}{
		Als \wikiBoldFt{mathematische Objekte} werden die \wikiLinkFt{abstrakten} \wikiLinkFt{Objekte} bezeichnet, die in den verschiedenen \wikiLinkFt{Teilgebieten der Mathematik} beschrieben und untersucht werden. \wikiLinkFt{Grundlegende} Beispiele sind \wikiLinkFt{Zahlen}, \wikiLinkFt{Mengen} und \wikiLinkFt{geometrische Körper}, weiterführend sind beispielsweise \wikiLinkFt{Graphen}, \wikiLinkFt{Integrale} und \wikiLinkFt{Kohomologien}. Die Fragen zur Existenz und zu der Natur von mathematischen Objekten sind zentral in der \wikiLinkFt{Philosophie der Mathematik}. Die zeitgenössische Mathematik hingegen klammert diese Fragestellungen aus und beschäftigt sich \wikiLinkFt{innerstrukturell} mit ihnen. Dies schließt Bereiche wie \wikiLinkFt{Mengenlehre}, \wikiLinkFt{Prädikatenlogik}, \wikiLinkFt{Modelltheorie} und \wikiLinkFt{Kategorientheorie} mit ein, in denen die (sonst übergeordneten) mathematischen Strukturen wie \wikiLinkFt{Axiome}, \wikiLinkFt{Schlussregeln} und \wikiLinkFt{Beweise} erforscht werden, die damit selbst zu mathematischen Objekten werden. Die Ansichten darüber, was mathematische Objekte sind, haben sich im Lauf der \wikiLinkFt{Geschichte der Mathematik} stark gewandelt.
	}
	Ein \GloFt{Objekt} ist \hier\ immer ein \Element\ aus \MtsUniversum.
}

\newVerweis     {\formalesObjekt} {\glstext} {formalesObjekt}
\newVerweis     {\formalenObjekte}{\glsuseri}{formalesObjekt}
\newglossaryentry{formalesObjekt}{
	name       = {formalesObjekt \addIdx     {formalesObjekt}},
	name       =            {---, formales \addIdx[
		name   =            {---, formales},
		sort   =         {Objekt, formales}] {formalesObjekt}},
	sort       =         {Objekt, formales},
	text       ={formales Objekt},
	user1      ={formalen Objekte},
	description={\todoOk%
		Ein \GloFt{formales Objekt} ist ein \Objekt, das in \Aussagen\ in \Objektsprache\ einen \Parameter\ ersetzen darf.
		Es ist notwendigerweise in \Objektsprache\ geschrieben.
	}
}

\newVerweis     {\metasprachlichesObjekt} {\glstext}        {metasprachlichesObjekt}
\newVerweis      {\metasprachlicheObjekte}{\glspl}          {metasprachlichesObjekt}
\newglossaryentry{metasprachlichesObjekt}{
	name       = {metasprachlichesObjekt \addIdx            {metasprachlichesObjekt}},
	name       =                    {---, metasprachliches \addIdx[
		name   =                    {---, metasprachliches},
		sort   =                 {Objekt, metasprachliches}]{metasprachlichesObjekt}},
	sort       =                 {Objekt, metasprachliches},
	text       ={metasprachliches Objekt},
	plural     ={metasprachliche  Objekte},
	description={\todoPruefen%
		Ein \GloFt{metasprachliches Objekt} ist ein \Objekt\ in \Metasprache.
	}
}

\newVerweis     {\Objektart}  {\glstext}{Objektart}
\newVerweis[en] {\Objektarten}{\glstext}{Objektart}
\newglossaryentry{Objektart}{
	name        ={Objektart \addIdx     {Objektart}},
	text        ={Objektart},
	description ={\todoBeschreiben%
	}
}

\newVerweis     {\Objektbereich}{\glstext} {Objektbereich}
\newglossaryentry{Objektbereich}{
	name        ={Objektbereich \addIdx    {Objektbereich}},
	text        ={Objektbereich},
	description ={\todoOk%
		Der \GloFt{Objektbereich} \MtsObjekte\ ist der \Bereich\ aller \formalenObjekte, \textdh\ der \Objekte, die in \Aussagen\ in \Objektsprache\ einen \Parameter\ ersetzen dürfen.
		Diese Objekte sind notwendigerweise auch in \Objektsprache\ geschrieben und offensichtlich ist $\MtsObjekte \MtsSubsetEq \MtsUniversum$.
	}
}

\newVerweis     {\Objektdefinition}  {\glstext}{Objektdefinition}
\newVerweis[en] {\Objektdefinitionen}{\glstext}{Objektdefinition}
\newglossaryentry{Objektdefinition}{
	name        ={Objektdefinition \addIdx     {Objektdefinition}},
	text        ={Objektdefinition},
	see         ={Aussagedefinition},
	description ={\todoOk%
		Eine \Metadefinition: Die formale Definition eines \Objekts.
		\ifmarginparFlg\newline\else\fi
		\seqqt{$A \MtsDefEq B$} steht für \standsfor{$A$ ist \DefFt{definitionsgemäß gleich} $B$} für \Objekte\ $A$ und $B$.
		Gewissermaßen ist $A$ nur eine andere Schreibweise für $B$.
	}
}

\newVerweis     {\Objektformel} {\glstext}{Objektformel}
\newVerweis[n]  {\Objektformeln}{\glstext}{Objektformel}
\newglossaryentry{Objektformel}{
	name        ={Objektformel \addIdx    {Objektformel}},
	text        ={Objektformel},
	description ={\todoPruefen%
		Eine \Formel\ der \Objektsprache.
	}
}

\newVerweis     {\Objektkonstante} {\glstext}{Objektkonstante}
\newVerweis[n]  {\Objektkonstanten}{\glstext}{Objektkonstante}
\newglossaryentry{Objektkonstante}{
	name        ={Objektkonstante \addIdx    {Objektkonstante}},
	text        ={Objektkonstante},
	description ={\todoPruefen%
		Eine \Konstante\ der \Objektsprache.
	}
}

\newVerweis     {\Objektoperation}  {\glstext} {Objektoperation}
\newVerweis[en] {\Objektoperationen}{\glstext} {Objektoperation}
\newVerweis[en]      {\Ooperationen}{\glsuseri}{Objektoperation}
\newglossaryentry{Objektoperation}{
	name        ={Objektoperation \addIdx      {Objektoperation}},
	text        ={Objektoperation},
	user1       =      {operation},
	see         ={Metaoperation},
	description ={\todoPruefen%
		Eine \Operation\ der \Objektsprache: \OjkAnd, \OjkOr.
	}
}

\newVerweis     {\Objektrelation}  {\glstext} {Objektrelation}
\newVerweis[en] {\Objektrelationen}{\glstext} {Objektrelation}
\newVerweis[en]      {\Orelationen}{\glsuseri}{Objektrelation}
\newglossaryentry{Objektrelation}{
	name        ={Objektrelation \addIdx      {Objektrelation}},
	text        ={Objektrelation},
	user1       =      {relation},
	see         ={Metarelation},
	description ={\todoPruefen%
		Eine \Relation\ der \Objektsprache: \OjkImp, \OjkRep\ oder \OjkEquiv.
	}
}

\newVerweis     {\Objektsprache} {\glstext}{Objektsprache}
\newVerweis[n]  {\Objektsprachen}{\glstext}{Objektsprache}
\newglossaryentry{Objektsprache}{
	name        ={Objektsprache \addIdx    {Objektsprache}},
	text        ={Objektsprache},
	description ={\todoOk%
		Die \Sprache, über die mittels einer (\formalenM) \Metasprache\ "`geredet"' wird.
		Unser \Objekt, mit dem mathematische \Beweise\ formuliert werden sollen, ist die \Logik.
		Demnach sind die Ausdrucksmittel der \Objektsprache\ die der \Logik.
		Wir verwenden \hier\ die \Praedikatenlogik\ oder, als \echteTeilsprache, die \Aussagenlogik.
	}
}

\newVerweis     {\Objektsymbol} {\glstext}{Objektsymbol}
\newVerweis[e]  {\Objektsymbole}{\glstext}{Objektsymbol}
\newglossaryentry{Objektsymbol}{
	name        ={Objektsymbol \addIdx    {Objektsymbol}},
	text        ={Objektsymbol},
	see         ={Metasymbol},
	description ={\todoPruefen%
		Ein \Symbol\ der \Objektsprache.
	}
}

\newVerweis     {\Operation}  {\glstext}{Operation}
\newVerweis[en] {\Operationen}{\glstext}{Operation}
\newglossaryentry{Operation}{
	name        ={Operation \addIdx     {Operation}},
	text        ={Operation},
	description ={\todoPruefen%
		Eine \GloFt{Operation} ist eine --- meistens \binaere, \textdh\ zweiwertige --- \Funktion\ $M^n \MtsFktArrow M$ mit $n \MtsIn \MtsINo$.
		Für eine \binaere, \textdh\ $n = 2$, \gloFt{Operation} $\FunktionDef{\BspOpB}{M \MtsTimes M}{M}$ schreibt man meistens $x \BspOpB y$ statt $\BspOpB(x,y)$.
		Für $n = 0$ kann man die \gloFt{Operation} mit einer \Konstanten\ identifizieren.
	}
}

\newVerweis      {\aussagenlogischeOperation}  {\glstext}       {aussagenlogischeOperation}
\newVerweis[en]  {\aussagenlogischeOperationen}{\glstext}       {aussagenlogischeOperation}
\newVerweis[en] {\aussagenlogischenOperationen}{\glsuseri}      {aussagenlogischeOperation}
\newVerweis[en]                 {\aOperationen}{\glsuserii}     {aussagenlogischeOperation}
\newglossaryentry {aussagenlogischeOperation}{
	name       =                        {---, aussagenlogische \addIdx[
		name   =                        {---, aussagenlogische},
		sort   =                  {Operation, aussagenlogische}]{aussagenlogischeOperation}},
	sort       =                  {Operation, aussagenlogische},
	text       ={aussagenlogische  Operation},
	user1      ={aussagenlogischen Operation},
	user2      =                  {Operation},
	description={\todoErgaenzen%
		Die \GloFt{aussagenlogischen Operationen} sind ...
	}
}

\newVerweis     {\Operationssymbol} {\glstext}{Operationssymbol}
\newVerweis[e]  {\Operationssymbole}{\glstext}{Operationssymbol}
\newglossaryentry{Operationssymbol}{
	name        ={Operationssymbol \addIdx    {Operationssymbol}},
	text        ={Operationssymbol},
	description ={\todoPruefen%
		Ein \Symbol\ für eine \Operation.
	}
}

\newVerweis         {\Ordnungsrelation}  {\glstext}{Ordnungrelation}
\newVerweis[en]     {\Ordnungsrelationen}{\glstext}{Ordnungrelation}
\longnewglossaryentry{Ordnungsrelation}{
	name            ={Ordnungsrelation \addIdx[
		name        ={Ordnungsrelation}]           {Ordnungsrelation}},
	text            ={Ordnungsrelation},
}{\todoPruefen%
	Eine \GloFt{Ordnungsrelation} ist ein \binaere\ \Relation\ auf einer \Menge\ $M$ mit der folgenden Eigenschaft
	(dabei sei $\preceq$ die \gloFt{Ordnungsrelation}):
	\begin{align}
		&\text{\DefFt{transitiv }}:\qquad ((a \preceq b) \MtsAnd (b \preceq c)) \MtsImp (a \preceq c) \formulatoleft
	\end{align}
	jeweils für alle \Elemente\ $a$, $b$ und $c$ aus $M$.
}

%P === P === P === P === P === P === P === P === P === P === P === P === P === P

\newVerweis     {\geordnetesPaar} {\glstext}  {geordnetesPaar}
\newVerweis[e]  {\geordnetenPaare}{\glsuseri} {geordnetesPaar}
\newglossaryentry{geordnetesPaar}{
	name       =           {Paar, geordnetes \addIdx[
		name   =           {Paar, geordnetes}]{geordnetesPaar}},
	text       ={geordnetes Paar},
	user1      ={geordneten Paar},
	description={\todoBeschreiben%
	}
}

\newVerweis     {\Parameter} {\glstext}{Parameter}
\newVerweis[n]  {\Parametern}{\glstext}{Parameter}
\newVerweis[s]  {\Parameters}{\glstext}{Parameter}
\newglossaryentry{Parameter}{
	name        ={Parameter \addIdx    {Potenzmenge}},
	text        ={Parameter},
	see         ={Aussage,Variable},
	description ={\todoGeprueft%
		Die \GloFt{Parameter} einer \Aussage\ sind deren \freieVariablen.
	}
}

\newVerweis      {\PolnischeNotation}  {\glstext}  {PolnischeNotation}
\newVerweis[en]  {\PolnischeNotationen}{\glstext}  {PolnischeNotation}
\newVerweis      {\PolnischenNotation} {\glsuseri} {PolnischeNotation}
\newVerweis      {\PolnischerNotation} {\glsuserii}{PolnischeNotation}
\newglossaryentry{PolnischeNotation}{
	name        =           {Notation, Polnische \addIdx[
		name    =           {Notation, Polnische},
		text    ={Polnische  Notation}]            {PolnischeNotation}},
	text        ={Polnische  Notation},
	user1       ={Polnischen Notation},
	user2       ={Polnischer Notation},
	description ={\todoPruefen%
		Bei der \GloFt{Polnischen Notation} stehen die Argumente von \Relationen\ und \Funktionen\ stets rechts von den \RelationsS- und \Funktionssymbolen.
		Dadurch kann auf \Gliederungszeichen\ wie Klammern und Kommata verzichtet werden.
		Noch einfacher für Computer ist die \GloFt{umgekehrte Polnische Notation}, bei der die Argumente immer links stehen.
	}
}

\newVerweis     {\Potenzmenge} {\glstext}{Potenzmenge}
\newVerweis[n]  {\Potenzmengen}{\glstext}{Potenzmenge}
\newglossaryentry{Potenzmenge}{
	name        ={Potenzmenge \addIdx    {Potenzmenge}},
	text        ={Potenzmenge},
	description ={\todoPruefen%
		Die \Potenzmenge\ $\MtsPot(M)$ einer \Menge\ $M$ ist die \Menge\ ihrer \Teilmengen.
	}
}

\newVerweis     {\Praedikat} {\glstext}{Praedikat}
\newVerweis[e]  {\Praedikate}{\glstext}{Praedikat}
\newVerweis[s]  {\Praedikats}{\glstext}{Praedikat}
\newglossaryentry{Praedikat}{
	name        ={Prädikat \addIdx[
		name    ={Prädikat}]           {Praedikat}},
	text        ={Prädikat},
	description ={\todoPruefen%
		Ein Element der \Praedikatenlogik. ---
		\textZB\ kann man eine Gruppe als ein zwei\stelliges\ \Praedikat\ $\PreFt{Gruppe}(G,+)$ definieren, in dem $G$ eine \Menge\ und $+$ eine \Operation, \textdh\ eine \binaere\ (zwei\stellige) \Funktion\ $ +: G \MtsTimes G \rightarrow G $ ist, so dass die Gruppenaxiome erfüllt sind.
	}
}

\newVerweis         {\Praedikatenlogik}{\glstext}{Praedikatenlogik}
\longnewglossaryentry{Praedikatenlogik}{
	name            ={Prädikatenlogik \addIdx[
		name        ={Prädikatenlogik}]          {Praedikatenlogik}},
	text            ={Prädikatenlogik},
	see             ={Aussagenlogik,Logik},
}{
	\wikicite{bib:Praedikatenlogik}{
		Die \wikiBoldFt{Prädikatenlogiken} (auch \wikiBoldFt{Quantorenlogiken}) bilden eine Familie \wikiLinkFt{logischer} Systeme, die es erlauben, einen weiten und in der Praxis vieler Wissenschaften und deren Anwendungen wichtigen Bereich von Argumenten zu formalisieren und auf ihre Gültigkeit zu überprüfen. Auf Grund dieser Eigenschaft spielt die Prädikatenlogik eine große Rolle in der \wikiLinkFt{Logik} sowie in \wikiLinkFt{Mathematik}, \wikiLinkFt{Informatik}, \wikiLinkFt{Linguistik} und \wikiLinkFt{Philosophie}.
	}
}

\newVerweis     {\Praemisse}  {\glstext}{Praemisse}
\newVerweis[n]  {\Praemissen}{\glstext}{Praemisse}
\newglossaryentry{Praemisse}{
	name        ={Prämisse \addIdx      {Praemisse}},
	text        ={Prämisse},
	see         ={Schlussregel},
	description ={\todoPruefen%
		Eine \Ableitung:
		Die \Praemissen\ einer \Schlussregel\ $\frac{\MtsPraemisseSet}{\MtsKonklusionSet}$ \textbzw\ $\frac{\MtsPraemisseSet}{\MtsKonklusionSet}$ sind die \Elemente\ aus \MtsPraemisseSet\ \textbzw\ \MtsPraemisseRel.
		Die \Praemissen\ werden normalerweise mit $\MtsPraemisse_i$ bezeichnet.
	}
}

\newVerweis     {\Praemissenmenge} {\glstext}{Praemissenmenge}
\newVerweis[n]  {\Praemissenmengen}{\glstext}{Praemissenmenge}
\newglossaryentry{Praemissenmenge}{
	name        = {Prämissenmenge \addIdx    {Praemissenmenge}},
	text        = {Prämissenmenge},
	description ={\todoPruefen%
		Eine \Ableitungsmenge:
		Die \Menge\ \MtsPraemisseSet\ der \Praemissen\ einer \Schlussregel\ \textbzw\ eines \Beweises.
	}
}

\newVerweis         {\kartesischesProdukt}{\glstext}      {kartesischesProdukt}
\newVerweis          {\kartesischeProdukt}{\glsuseri}     {kartesischesProdukt}
\longnewglossaryentry{kartesischesProdukt}{
	name            =             {Produkt, kartesisches \addIdx[
		name        =             {Produkt, kartesisches}]{kartesischesProdukt}},
	text            ={kartesisches Produkt},
	user1           ={kartesische  Produkt},
}{\todoPruefen%
	\wikicite{bib:kartesischesProdukt}{
		Das \wikiBoldFt{kartesische Produkt} oder \wikiBoldFt{Mengenprodukt} ist in der Mengenlehre eine grundlegende Konstruktion, aus gegebenen Mengen eine neue Menge zu erzeugen. [\textdots] Das kartesische Produkt zweier Mengen ist die Menge aller geordneten Paare von Elementen der beiden Mengen, wobei die erste Komponente ein Element der ersten Menge und die zweite Komponente ein Element der zweiten Menge ist. Allgemeiner besteht das kartesische Produkt mehrerer Mengen aus der Menge aller Tupel von Elementen der Mengen, wobei die Reihenfolge der Mengen und damit der entsprechenden Elemente fest vorgegeben ist. Die Ergebnismenge des kartesischen Produkts wird auch \wikiBoldFt{Produktmenge}, \wikiBoldFt{Kreuzmenge} oder \wikiBoldFt{Verbindungsmenge} genannt. [\textdots]
	}
}

%Q === Q === Q === Q === Q === Q === Q === Q === Q === Q === Q === Q === Q === Q

\newVerweis[en]     {\Quantoren}{\glstext}{Quantor}
\newVerweis         {\Quantor}  {\glstext}{Quantor}
\longnewglossaryentry{Quantor}{
	name            ={Quantor \addIdx     {Quantor}},
	text            ={Quantor},
	see             ={Allquantor,Existenzquantor,Junktor,Praedikatenlogik},
}{\todoErgaenzen%
	\wikicite{bib:Quantor}{
		Ein \wikiBoldFt{Quantor} oder \wikiBoldFt{Quantifikator}, die Re-Latinisierung des von \wikiLinkFt{C. S. Peirce} eingeführten Ausdrucks „quantifier“, ist ein \wikiLinkFt{Operator} der \wikiLinkFt{Prädikatenlogik}. Neben den \wikiLinkFt{Junktoren} sind die Quantoren Grundzeichen der Prädikatenlogik. Allen Quantoren gemeinsam ist, dass sie \wikiLinkFt{Variablen} \wikiLinkFt{binden}.

		Die beiden gebräuchlichsten Quantoren sind der \wikiItalicFt{Existenzquantor} (in natürlicher Sprache zum Beispiel als „mindestens ein“ ausgedrückt) und der \wikiItalicFt{Allquantor} (in natürlicher Sprache zum Beispiel als „alle“ oder „jede/r/s“ ausgedrückt). Andere Arten von Quantoren sind \wikiItalicFt{Anzahlquantoren} wie „ein“ oder „zwei“, die sich auf Existenz- beziehungsweise Allquantor zurückführen lassen, und Quantoren wie „manche“, „einige“ oder „viele“, die auf Grund ihrer Unbestimmtheit in der \wikiLinkFt{klassischen Logik} nicht verwendet werden.
	}
}

\newVerweis     {\logischerQuantor} {\glstext} {logischerQuantor}
\newglossaryentry{logischerQuantor}{
	name       =              {---, logischer \addIdx[
		name   =              {---, logischer},
		sort   =          {Quantor, logischer}]{logischerQuantor}},
	sort       =          {Quantor, logischer},
	text       ={logischer Quantor},
	description={\todoBeschreiben%
	}
}

\newVerweis     {\metasprachlicherQuantor} {\glstext}        {metasprachlicherQuantor}
\newglossaryentry{metasprachlicherQuantor}{
	name       =                     {---, metasprachlicher \addIdx[
		name   =                     {---, metasprachlicher},
		sort   =                 {Quantor, metasprachlicher}]{metasprachlicherQuantor}},
	sort       =                 {Quantor, metasprachlicher},
	text       ={metasprachlicher Quantor},
	description={\todoBeschreiben%
	}
}

\newVerweis     {\Quellbereich} {\glstext} {Quellbereich}
\newVerweis[e]  {\Quellbereiche}{\glstext} {Quellbereich}
\newVerweis     {\QuellB}       {\glsuseri}{Quellbereich}
\newglossaryentry{Quellbereich}{
	name        ={Quellbereich \addIdx     {Quellbereich}},
	text        ={Quellbereich},
	user1       ={Quell},
	see         ={Definitionsbereich,Menge},
	description ={\todoPruefen%
		Für die \Funktion \FunktionDef{f}{A}{B} ist die \Menge\ $\MtsQb(f) \MtsDefEq \RawMengeDef{a \in A}{f(a) \text{ existiert}}$ ihr \Quellbereich%
		\footnote{%
			Der \GloFt{Quellbereich} $\MtsQb(f)$ unterscheidet sich nur bei \DefFt{partiellen} \Funktionen\ vom \Definitionsbereich\ $\MtsDb(f)$, \textdh\ solchen \Funktionen, für die $f(a)$ nicht für alle $a \MtsIn A$ definiert ist.
		}
		(source).
	}
}

%R === R === R === R === R === R === R === R === R === R === R === R === R === R

\newVerweis         {\Relation}  {\glstext}{Relation}
\newVerweis[en]     {\Relationen}{\glstext}{Relation}
\longnewglossaryentry{Relation}{
	name            ={Relation \addIdx     {Relation}},
	text            ={Relation},
	see             ={Aequivalenzrelation,Begriff,Menge,Objekt,Ordnungsrelation},
}{\todoPruefen%
	\wikicite{bib:Relation}{
		Eine \wikiBoldFt{Relation} (\wikiLinkFt{lateinisch} \wikiItalicFt{relatio} „Beziehung“, „Verhältnis“) ist allgemein eine Beziehung, die zwischen Dingen bestehen kann. Relationen im Sinne der \wikiLinkFt{Mathematik} sind ausschließlich diejenigen Beziehungen, bei denen stets klar ist, ob sie bestehen oder nicht; Objekte können also nicht „bis zu einem gewissen Grade“ in einer Relation zueinander stehen. Damit ist eine einfache \wikiLinkFt{mengentheoretische} Definition des Begriffs möglich: Eine Relation $R$ ist eine Menge von $n$-\wikiLinkFt{Tupeln}. In der Relation $R$ zueinander stehende Dinge bilden $n$-Tupel, die Element von $R$ sind.

		Wird nicht ausdrücklich etwas anderes angegeben, versteht man unter einer Relation gemeinhin eine zweistellige oder binäre Relation. Bei einer solchen Beziehung bilden dann jeweils zwei Elemente $a$ und $b$ ein \wikiLinkFt{geordnetes Paar} $(a,b)$. Stammen dabei $a$ und $b$ aus verschiedenen Grundmengen $A$ und $B$, so heißt die Relation \wikiItalicFt{heterogen} oder „Relation \wikiItalicFt{zwischen} den Mengen $A$ und $B$.“ Stimmen die Grundmengen überein ($A = B$), dann heißt die Relation \wikiItalicFt{homogen} oder „Relation \wikiItalicFt{in} bzw. \wikiItalicFt{auf} der Menge $A$.“

		Wichtige Spezialfälle, zum Beispiel \wikiLinkFt{Äquivalenzrelationen} und \wikiLinkFt{Ordnungsrelationen}, sind Relationen \wikiItalicFt{auf} einer Menge.

		Heute sehen manche Autoren den Begriff Relation nicht unbedingt als auf Mengen beschränkt an, sondern lassen jede aus geordneten Paaren bestehende \wikiLinkFt{Klasse} als Relation gelten.
	}
	Eine \DefFt{$n$-\stellige} \GloFt{Relation} $R$ ist ein (1+$n$)-\Tupel\ $(G,A_1,\dots,A_n)$ mit $G \MtsSubsetEq A_1 \MtsTimes \dots \MtsTimes A_n)$.
}

\newVerweis      {\aussagenlogischeRelation}  {\glstext}       {aussagenlogischeRelation}
\newVerweis[en]  {\aussagenlogischeRelationen}{\glstext}       {aussagenlogischeRelation}
\newVerweis[en] {\aussagenlogischenRelationen}{\glsuseri}      {aussagenlogischeRelation}
\newVerweis                     {\aRelation}  {\glsuserii}     {aussagenlogischeRelation}
\newVerweis[en]         {\aRelationen}{\glsuserii}{aussagenlogischeRelation}
\newglossaryentry {aussagenlogischeRelation}{
	name       =                       {---, aussagenlogische \addIdx[
		name   =                       {---, aussagenlogische},
		sort   =                  {Relation, aussagenlogische}]{aussagenlogischeRelation}},
	sort       =                  {Relation, aussagenlogische},
	text       ={aussagenlogische  Relation},
	user1      ={aussagenlogischen Relation},
	user2      =                  {Relation},
	description={\todoErgaenzen%
		Die \GloFt{aussagenlogischen} \Relationen\ sind ...
	}
}

\newVerweis     {\Relationssymbol} {\glstext} {Relationssymbol}
\newVerweis[e]  {\Relationssymbole}{\glstext} {Relationssymbol}
\newVerweis     {\RelationsS}      {\glsuseri}{Relationssymbol}
\newglossaryentry{Relationssymbol}{
	name        ={Relationssymbol \addIdx     {Relationssymbol}},
	text        ={Relationssymbol},
	user1       ={Relations},
	description ={\todoPruefen%
		Ein \Symbol\ für eine \Relation.
	}
}

%S === S === S === S === S === S === S === S === S === S === S === S === S === S

\newVerweis     {\Satz}   {\glstext}{Satz}
\newVerweis[es] {\Satzes} {\glstext}{Satz}
\newVerweis     {\Saetze} {\glspl}  {Satz}
\newVerweis[n]  {\Saetzen}{\glspl}  {Satz}
\newglossaryentry{Satz}{
	name        ={Satz \addIdx      {Satz}},
	text        ={Satz},
	plural      ={Sätze},
	description ={\todoOk%
		Ein \GloFt{Satz} ist eine \Aussage, bestehend aus einer Anzahl von \Praemissen\ und \Konklusionen\ und einem \Beweis, der die \Konklusionen\ aus den \Praemissen\ ableitet.
	}
}

\newVerweis     {\formalerSatz} {\glstext} {formalerSatz}
\newVerweis     {\formalenSatz} {\glsuseri}{formalerSatz}
\newglossaryentry{formalerSatz}{
	name       =          {---, formaler \addIdx[
		name   =          {---, formaler},
		sort   =         {Satz, formaler}] {formalerSatz}},
	sort       =         {Satz, formaler},
	text       ={formaler Satz},
	user1      ={formalen Satz},
	see        ={FS},
	description={\todoPruefen%
		Formale \Darstellung\ eines mathematischen \Satzes.
	}
}

\newVerweis         {\Schlussregel} {\glstext}{Schlussregel}
\newVerweis[n]      {\Schlussregeln}{\glstext}{Schlussregel}
\longnewglossaryentry{Schlussregel}{
	name            ={Schlussregel \addIdx    {Schlussregel}},
	text            ={Schlussregel},
	see             ={MtsSchlussregel,MtsSchlussregelSet,Kalkuel},
}{\todoPruefen%
	\wikicite{bib:Schlussregel}{
		Eine \wikiBoldFt{Schlussregel} (oder \wikiItalicFt{Inferenzregel}) bezeichnet eine Transformationsregel (Umformungsregel) in einem \wikiLinkFt{Kalkül} der \wikiLinkFt{formalen Logik}, d. h. eine \wikiLinkFt{syntaktische} Regel, nach der es erlaubt ist, von bestehenden Ausdrücken einer formalen Sprache zu neuen Ausdrücken überzugehen. Dieser regelgeleitete Übergang stellt eine \wikiLinkFt{Schlussfolgerung} dar.
	}
	Eine \Schlussregel\ $\frac{\MtsPraemisseSet}{\MtsKonklusionSet}$ entspricht der \Aussage:
	\begin{quote}
		Wenn alle \Praemissen\ $\MtsPraemisse \MtsIn \MtsPraemisseSet$ zutreffen, dann auch alle \Konklusionen\ $\MtsKonklusion \MtsIn \MtsKonklusionSet$.
	\end{quote}
	Wenn diese \Aussage\ zutrifft, kann die Schlussregel zur \zulaessigen\ \Transformation\ von \Formeln\ dienen.
}

\newVerweis     {\allgemeingueltigeSchlussregel} {\glstext}        {allgemeingueltigeSchlussregel}
\newVerweis[n]  {\allgemeingueltigeSchlussregeln}{\glstext}        {allgemeingueltigeSchlussregel}
\newVerweis    {\allgemeingueltigenSchlussregel} {\glsuseri}       {allgemeingueltigeSchlussregel}
\newVerweis[n] {\allgemeingueltigenSchlussregeln}{\glsuseri}       {allgemeingueltigeSchlussregel}
\newglossaryentry{allgemeingueltigeSchlussregel}{
	name       =                           {---, allgemeingültige \addIdx[
		name   =                           {---, allgemeingültige},
		sort   =                  {Schlussregel, allgemeingültige}]{allgemeingueltigeSchlussregel}},
	sort       =                  {Schlussregel, allgemeingültige},
	text       ={allgemeingültige  Schlussregel},
	user1      ={allgemeingültigen Schlussregel},
	description={\todoPruefen%
		Eine \Schlussregel\ heißt \GloFt{allgemeingültig}, wenn sie aus den \Basisregeln\ und schon bekannten \allgemeingueltigenSchlussregeln\ abgeleitet werden kann.
	}
}

\newVerweis     {\Schlussregelmenge} {\glstext}{Schlussregelmenge}
\newcommand*    {\Schlussregelmengen}[1][]{\glstext[#1]{Schlussregelmenge}n[]}
\newglossaryentry{Schlussregelmenge}{
	name        ={Schlussregelmenge \addIdx    {Schlussregelmenge}},
	text        ={Schlussregelmenge},
	see         ={MtsSchlussregelSet},
	description ={\todoPruefen%
		Eine \Menge\ von \Schlussregeln, meistens mit \MtsSchlussregelSet\ bezeichnet.
	}
}

\newVerweis     {\Schnittregel}{\glstext}{Schnittregel}
\newglossaryentry{Schnittregel}{
	name        ={Schnittregel \addIdx   {Schnittregel}},
	text        ={Schnittregel},
	see         ={SR},
	description ={\todoPruefen%
		Eine \allgemeingueltigeSchlussregel.
	}
}

\newVerweis         {\Semantik} {\glstext}{Semantik}
\longnewglossaryentry{Semantik}{
	name            ={Semantik \addIdx    {Semantik}},
	text            ={Semantik},
}{\todoGeprueft%
	\wikicite{bib:Wikipedia}{
		\wikiBoldFt{Semantik} [\textdots], auch \wikiBoldFt{Bedeutungslehre}, nennt man die Theorie oder Wissenschaft von der Bedeutung der Zeichen. \wikiItalicFt{Zeichen} können hierbei beliebige \wikiLinkFt{Symbole} sein, insbesondere aber auch \wikiLinkFt{Sätze}, Satzteile, \wikiLinkFt{Wörter} oder \wikiLinkFt{Wortteile}.
	}
	In der \formalenMetasprache\ und der \Objektsprache\ sind die Zeichen die \Symbole\ und \Formeln.
}

\newVerweis         {\Signatur}{\glstext}{Signatur}
\longnewglossaryentry{Signatur}{
	name            ={Signatur \addIdx   {Signatur}},
	text            ={Signatur},
	see             ={Abbildung,Logik,Praedikatenlogik,Sprache,Stelligkeit,Symbol},
}{\todoPruefen%
	\wikicite{bib:Signatur}{
		In der \wikiLinkFt{mathematischen Logik} besteht eine \wikiBoldFt{Signatur} aus der \wikiLinkFt{Menge} der \wikiLinkFt{Symbole}, die in der betrachteten \wikiLinkFt{Sprache} zu den üblichen, rein logischen Symbolen hinzukommt, und einer \wikiLinkFt{Abbildung}, die jedem Symbol der Signatur eine \wikiLinkFt{Stelligkeit} eindeutig zuordnet. Während die logischen Symbole wie  $\forall ,\exists ,\land ,\lor ,\rightarrow ,\leftrightarrow ,\neg$ stets als „für alle“, „es gibt ein“, „und“, „oder“, „folgt“, „äquivalent zu“ bzw. „nicht“ interpretiert werden, können durch die semantische \wikiLinkFt{Interpretation} der Symbole der Signatur verschiedene \wikiLinkFt{Strukturen} (insbesondere Modelle von Aussagen der Logik) unterschieden werden. Die Signatur ist der spezifische Teil einer \wikiLinkFt{elementaren Sprache}.

		Beispielsweise lässt sich die gesamte \wikiLinkFt{Zermelo-Fraenkel-Mengenlehre} in der Sprache der \wikiLinkFt{Prädikatenlogik erster Stufe} und dem einzigen Symbol \MtsIn (neben den rein logischen Symbolen) formulieren; in diesem Fall ist die Symbolmenge der Signatur gleich $\{\MtsIn\}$.
	}
}

\newVerweis      {\BoolescheSignatur}{\glstext}  {BoolescheSignatur}
\newVerweis     {\BooleschenSignatur}{\glsuseri} {BoolescheSignatur}
\newglossaryentry {BoolescheSignatur}{
	name       =                {---, Boolesche \addIdx[
		name   =                {---, Boolesche},
		sort   =           {Signatur, Boolesche}]{BoolescheSignatur}},
	sort       =           {Signatur, Boolesche},
	text       ={Boolesche  Signatur},
	user1      ={Booleschen Signatur},
	description={\todoPruefen%
		Die \logischeSignatur\ $\{\OjkNot, \OjkAnd, \OjkOr\}$.
	}
}

\newVerweis      {\logischeSignatur}  {\glstext} {logischeSignatur}
\newVerweis[en]  {\logischeSignaturen}{\glstext} {logischeSignatur}
\newVerweis     {\logischenSignatur}  {\glsuseri}{logischeSignatur}
\newglossaryentry {logischeSignatur}{
	name       =               {---, logische \addIdx[
		name   =               {---, logische},
		sort   =          {Signatur, logische}]  {logischeSignatur}},
	sort       =          {Signatur, logische},
	text       ={logische  Signatur},
	user1      ={logischen Signatur},
	description={\todoPruefen%
		Abweichend von der Definition von \Signatur\ in \Wikipedia\ ist eine \GloFt{logische Signatur} eine \Teilmenge\ von \OjkJun, ausreichend um damit und mit \OjkVar\ und Klammerung alle anderen \Elemente\ aus \OjkJun\ zu definieren.
	}
}

\newVerweis     {\Sprache} {\glstext}{Sprache}
\newVerweis[n]  {\Sprachen}{\glstext}{Sprache}
\newglossaryentry{Sprache}{
	name        ={Sprache \addIdx    {Sprache}},
	text        ={Sprache},
	description ={\todoPruefen%
		--- Siehe \Formelmenge.
	}
}

\newVerweis      {\aussagenlogischeSprache}{\glstext}         {aussagenlogischeSprache}
\newVerweis     {\aussagenlogischenSprache}{\glsuseri}        {aussagenlogischeSprache}
\newglossaryentry {aussagenlogischeSprache}{
	name       =                      {---, aussagenlogische \addIdx[
		name   =                      {---, aussagenlogische},
		sort   =                  {Sprache, aussagenlogische}]{aussagenlogischeSprache}},
	sort       =                  {Sprache, aussagenlogische},
	text       ={aussagenlogische  Sprache},
	user1      ={aussagenlogischen Sprache},
	description={\todoBeschreiben%
	}
}

\newVerweis     {\Sprachebene} {\glstext}{Sprachebene}
\newVerweis[n]  {\Sprachebenen}{\glstext}{Sprachebene}
\newglossaryentry{Sprachebene}{
	name        ={Sprachebene \addIdx    {Sprachebene}},
	text        ={Sprachebene},
	description ={\todoOk%
		Wir unterscheiden \hier\ drei \GloFt{Sprachebenen}: Die obere Ebene mit der \Metasprache, die mittlere mit der \formalenMetasprache\ und die untere mit der \Objektsprache.
		Mit einer \Sprache\ einer höheren Ebene kann man \textua\ \Aussagen\ über \Sprachen\ mit niedrigere Ebene treffen.
	}
}

\newVerweis     {\stellig}  {\glstext}{stellig}
\newVerweis[e]  {\stellige} {\glstext}{stellig}
\newVerweis[es] {\stelliges}{\glstext}{stellig}
\newVerweis[er] {\stelliger}{\glstext}{stellig}
\newglossaryentry{stellig}{
	name        ={$n$-stellig \addIdx[
		name    ={$n$-stellig},
		sort    ={stellig}]           {stellig}},
	sort        ={stellig},
	text        ={stellig},
	see         ={MtsStelF,MtsStelR},
	description ={\todoPruefen%
		Eine \Funktion, \Relation\ oder ein \Praedikat\ mit der \Stelligkeit\ $n \MtsIn \MtsINo$ nennt man \GloFt{$n$-stellig}.
	}
}

\newVerweis     {\Stelligkeit}  {\glstext}{Stelligkeit}
\newVerweis[en] {\Stelligkeiten}{\glstext}{Stelligkeit}
\newglossaryentry{Stelligkeit}{
	name        ={Stelligkeit \addIdx     {Stelligkeit}},
	text        ={Stelligkeit},
	see         ={MtsStelF,MtsStelR},
	description ={\todoPruefen%
		einer \Funktion, \Relation\ oder eines \Praedikats.
	}
}

\newVerweis     {\Symbol}  {\glstext} {Symbol}
\newVerweis[e]  {\Symbole} {\glstext} {Symbol}
\newVerweis[s]  {\Symbols} {\glstext} {Symbol}
\newVerweis[en] {\Symbolen}{\glstext} {Symbol}
\newglossaryentry{Symbol}{
	name        ={Symbol \addIdx      {Symbol}},
	text        ={Symbol},
	see         ={Beispielsymbol,Metasymbol,Objektsymbol},
	description ={\todoPruefen%
		Ein \DefFt{einfaches} \GloFt{Symbol} ist ein druckbares typographisches Zeichen, das als Einheit angesehen wird.
		Ein \DefFt{zusammengesetztes} \GloFt{Symbol} besteht aus mehreren einfachen \gloFt{Symbolen}.
		Wird ein \gloFt{Symbol}, das kann auch ein zusammengesetztes \gloFt{Symbol} sein, stets als Einheit angesehen, nennen wir es \defTxt{\atomar}\alternativi{unzerlegbar}, andernfalls \defTxt{\zerlegbar}.
		Im Einzelfall muss für ein Symbol definiert werden, ob es zerlegt werden kann oder nicht.
		Ein \emph{einfaches} \gloFt{Symbol} ist offensichtlich immer \atomar.
	}
}

\newVerweis     {\aussagenlogischesSymbol}  {\glstext}       {aussagenlogischesSymbol}
\newVerweis[en] {\aussagenlogischenSymbolen}{\glstext}       {aussagenlogischesSymbol}
\newglossaryentry{aussagenlogischesSymbol}{
	name       =                       {---, aussagenlogisches \addIdx[
		name   =                       {---, aussagenlogisches},
		sort   =                  {Symbol, aussagenlogische}]{aussagenlogischesSymbol}},
	sort       =                  {Symbol, aussagenlogische},
	text       ={aussagenlogisches Symbol},
	user1      ={aussagenlogischen Symbol},
	description={\todoErgaenzen%
		Die \GloFt{aussagenlogischen} \Symbole\ sind ...
	}
}

\newVerweis     {\metasprachlichesSymbol} {\glstext}         {metasprachlichesSymbol}
\newVerweis      {\metasprachlicheSymbole}{\glsuseri}        {metasprachlichesSymbol}
\newglossaryentry{metasprachlichesSymbol}{
	name       =                     {---, metasprachliches \addIdx[
		name   =                     {---, metasprachliches},
		sort   =                  {Symbol, metasprachliches}]{metasprachlichesSymbol}},
	sort       =                  {Symbol, metasprachliches},
	text       ={metasprachliches Symbol},
	user1      ={metasprachliche  Symbole},
	description={\todoBeschreiben%
	}
}

\newVerweis     {\zusammengesetztesSymbol} {\glstext}         {zusammengesetztesSymbol}
\newVerweis      {\zusammengesetzteSymbole}{\glsuseri}        {zusammengesetztesSymbol}
\newglossaryentry{zusammengesetztesSymbol}{
	name       =                     {---, zusammengesetztes \addIdx[
		name   =                     {---, zusammengesetztes},
		sort   =                  {Symbol, zusammengesetztes}]{zusammengesetztesSymbol}},
	sort       =                  {Symbol, zusammengesetztes},
	text       ={zusammengesetztes Symbol},
	user1      ={zusammengesetzte  Symbole},
	description={\todoBeschreiben%
	}
}

\newVerweis     {\Symbolfolge} {\glstext}{Symbolfolge}
\newVerweis[n]  {\Symbolfolgen}{\glstext}{Symbolfolge}
\newglossaryentry{Symbolfolge}{
	name        ={Symbolfolge \addIdx    {Symbolfolge}},
	text        ={Symbolfolge},
	see         ={Zeichenkette},
	description ={\todoPruefen%
		Eine \GloFt{Symbolfolge} ist eine \Folge\ von \atomaren\ \Symbolen.
	}
}

\newVerweis         {\Syntax} {\glstext}{Syntax}
\longnewglossaryentry{Syntax}{
	name            ={Syntax \addIdx    {Syntax}},
	text            ={Syntax},
	see             ={Semantik,Sprache},
}{\todoGeprueft%
	\wikicite{bib:Wikipedia}{
		Unter \wikiBoldFt{Syntax} [\textdots] versteht man allgemein ein Regelsystem zur Kombination elementarer Zeichen zu zusammengesetzten Zeichen in natürlichen oder künstlichen Zeichensystemen. Die Zusammenfügungsregeln der Syntax stehen hierbei den Interpretationsregeln der \wikiLinkFt{Semantik} gegenüber.
	}
	Wir nennen in der \formalenMetasprache\ und der \Objektsprache\ die elementaren Zeichen \Symbole\ und die zusammengesetzten Zeichen \Formeln.
}

%T === T === T === T === T === T === T === T === T === T === T === T === T === T

\newVerweis     {\Teilaussage} {\glstext} {Teilaussage}
\newVerweis        {\Taussage} {\glsuseri}{Teilaussage}
\newVerweis[n]  {\Teilaussagen}{\glstext} {Teilaussage}
\newglossaryentry{Teilaussage}{
	name        ={Teilaussage \addIdx     {Teilaussage}},
	text        ={Teilaussage},
	user1       =    {aussage},
	description ={\todoOk%
		Eine \Aussage\ $A$ heißt eine \GloFt{Teilaussage}\synonym{\defTxt{\Unteraussage}} \DefFt{von} einer \Aussage\ $B$, wenn sie Teil von $B$ ist und man sie ohne Bedeutungsänderung von $B$ dort klammern könnte.
	}
}

\newVerweis      {\echteTeilaussage}{\glstext}  {echteTeilaussage}
\newVerweis     {\echtenTeilaussage}{\glsuseri} {echteTeilaussage}
\newVerweis             {\eTaussage}{\glsuserii}{echteTeilaussage}
\newglossaryentry {echteTeilaussage}{
	name       =               {---, echte \addIdx[
		name   =               {---, echte},
		sort   =       {Teilaussage, echte}]    {echteTeilaussage}},
	sort       =       {Teilaussage, echte},
	text       ={echte  Teilaussage},
	user1      ={echten Teilaussage},
	user2      =           {aussage},
	description={\todoOk%
		Eine \Teilaussage\ $A$ einer \Aussage\ $B$ heißt \GloFt{echte Teilaussage} von $B$, wenn $A$ verschieden von $B$ ist.
	}
}

\newVerweis     {\Teilbereich} {\glstext} {Teilbereich}
\newVerweis[n]  {\Teilbereichn}{\glstext} {Teilbereich}
\newglossaryentry{Teilbereich}{
	name        ={Teilbereich \addIdx     {Teilbereich}},
	text        ={Teilbereich},
	see         ={Oberbereich},
	description ={\todoOk%
		Ein \Bereich\ $A$ ist genau dann ein \GloFt{Teilbereich} von einem \Bereich\ $B$, wenn $A \MtsSubsetEq B$ ist.
	}
}

\newVerweis     {\echterTeilbereich}{\glstext}  {echterTeilbereich}
\newVerweis     {\echtenTeilbereich}{\glsuseri} {echterTeilbereich}
\newVerweis             {\eTbereich}{\glsuserii}{echterTeilbereich}
\newglossaryentry{echterTeilbereich}{
	name       =               {---, echter \addIdx[
		name   =               {---, echter},
		sort   =       {Teilbereich, echter}]   {echterTeilbereich}},
	sort       =       {Teilbereich, echter},
	text       ={echter Teilbereich},
	see        ={echterOberbereich},
	description={\todoOk%
		Ein \Bereich\ $A$ ist genau dann ein \GloFt{echter Teilbereich} von einem \Bereich\ $B$, wenn $A \MtsSubset B$ ist.
	}
}

\newVerweis     {\Teilfolge} {\glstext} {Teilfolge}
\newVerweis        {\Tfolge} {\glsuseri}{Teilfolge}
\newVerweis[n]  {\Teilfolgen}{\glstext} {Teilfolge}
\newglossaryentry{Teilfolge}{
	name        ={Teilfolge \addIdx     {Teilfolge}},
	text        ={Teilfolge},
	user1       =    {folge},
	description ={\todoBeschreiben%
	}
}

\newVerweis      {\echteTeilfolge}{\glstext}  {echteTeilfolge}
\newVerweis     {\echtenTeilfolge}{\glsuseri} {echteTeilfolge}
\newVerweis             {\eTfolge}{\glsuserii}{echteTeilfolge}
\newglossaryentry {echteTeilfolge}{
	name       =             {---, echte \addIdx[
		name   =             {---, echte},
		sort   =       {Teilfolge, echte}]    {echteTeilfolge}},
	sort       =       {Teilfolge, echte},
	text       ={echte  Teilfolge},
	user1      ={echten Teilfolge},
	user2      =           {folge},
	description={\todoBeschreiben%
	}
}

\newVerweis     {\Teilformel} {\glstext} {Teilformel}
\newVerweis[n]  {\Teilformeln}{\glstext} {Teilformel}
\newVerweis        {\Tformel} {\glsuseri}{Teilformel}
\newglossaryentry{Teilformel}{
	name        ={Teilformel \addIdx     {Teilformel}},
	text        ={Teilformel},
	user1       =    {formel},
	description ={\todoBeschreiben%
	}
}

\newVerweis      {\echteTeilformel}{\glstext}  {echteTeilformel}
\newVerweis     {\echtenTeilformel}{\glsuseri} {echteTeilformel}
\newVerweis             {\eTformel}{\glsuserii}{echteTeilformel}
\newglossaryentry {echteTeilformel}{
	name       =              {---, echte \addIdx[
		name   =              {---, echte},
		sort   =       {Teilformel, echte}]    {echteTeilformel}},
	sort       =       {Teilformel, echte},
	text       ={echte  Teilformel},
	user1      ={echten Teilformel},
	user2      =           {formel},
	description={\todoBeschreiben%
	}
}

\newVerweis     {\Teilmenge} {\glstext} {Teilmenge}
\newVerweis        {\Tmenge} {\glsuseri}{Teilmenge}
\newVerweis[n]  {\Teilmengen}{\glstext} {Teilmenge}
\newglossaryentry{Teilmenge}{
	name        ={Teilmenge \addIdx     {Teilmenge}},
	text        ={Teilmenge},
	user1       =    {menge},
	see         ={Obermenge,Teilbereich},
	description ={\todoOk%
		Eine \Menge\ $A$ ist ist genau dann eine \GloFt{Teilmenge} von einer \Menge\ $B$, wenn $A \MtsSubsetEq B$ ist.
	}
}

\newVerweis      {\echteTeilmenge}{\glstext}  {echteTeilmenge}
\newVerweis     {\echtenTeilmenge}{\glsuseri} {echteTeilmenge}
\newVerweis             {\eTmenge}{\glsuserii}{echteTeilmenge}
\newglossaryentry {echteTeilmenge}{
	name       =             {---, echte \addIdx[
		name   =             {---, echte},
		sort   =       {Teilmenge, echte}]    {echteTeilmenge}},
	sort       =       {Teilmenge, echte},
	text       ={echte  Teilmenge},
	user1      ={echten Teilmenge},
	user2      =           {menge},
	see         ={echteObermenge,echterTeilbereich},
	description={\todoOk%
		Eine \Menge\ $A$ ist ist genau dann eine \GloFt{echte Teilmenge} von einer \Menge\ $B$, wenn $A \MtsSubset B$ ist.
	}
}

\newVerweis     {\Teilobjekt} {\glstext} {Teilobjekt}
\newVerweis        {\Tobjekt} {\glsuseri}{Teilobjekt}
\newVerweis[e]  {\Teilobjekte}{\glstext} {Teilobjekt}
\newglossaryentry{Teilobjekt}{
	name        ={Teilobjekt \addIdx     {Teilobjekt}},
	text        ={Teilobjekt},
	user1       =    {objekt},
	description ={\todoBeschreiben%
	}
}

\newVerweis     {\echtesTeilobjekt}{\glstext}  {echtesTeilobjekt}
\newVerweis     {\echtenTeilobjekt}{\glsuseri} {echtesTeilobjekt}
\newVerweis             {\eTobjekt}{\glsuserii}{echtesTeilobjekt}
\newglossaryentry{echtesTeilobjekt}{
	name        =              {---, echtes \addIdx[
		name    =              {---, echtes},
		sort    =       {Teilobjekt, echtes}]  {echtesTeilobjekt}},
	sort        =       {Teilobjekt, echtes},
	text        ={echtes Teilobjekt},
	user1       ={echten Teilobjekt},
	user2       =           {objekt},
	description ={\todoBeschreiben%
	}
}

\newVerweis     {\Teilsprache} {\glstext} {Teilsprache}
\newglossaryentry{Teilsprache}{
	name        ={Teilsprache \addIdx     {Teilsprache}},
	text        ={Teilsprache},
	description ={\todoBeschreiben%
	}
}

\newVerweis     {\echteTeilsprache}{\glstext}  {echteTeilsprache}
\newglossaryentry{echteTeilsprache}{
	name        =               {---, echte \addIdx[
		name    =               {---, echte},
		sort    =       {Teilsprache, echte}]  {echteTeilsprache}},
	sort        =       {Teilsprache, echte},
	text        ={echte Teilsprache},
	description ={\todoBeschreiben%
	}
}

\newVerweis     {\Teilsymbol} {\glstext} {Teilsymbol}
\newVerweis        {\Tsymbol} {\glsuseri}{Teilsymbol}
\newVerweis[e]  {\Teilsymbole}{\glstext} {Teilsymbol}
\newglossaryentry{Teilsymbol}{
	name        ={Teilsymbol \addIdx     {Teilsymbol}},
	text        ={Teilsymbol},
	user1       =    {symbol},
	description ={\todoBeschreiben%
	}
}

\newVerweis     {\echtesTeilsymbol}{\glstext}  {echtesTeilsymbol}
\newVerweis     {\echtenTeilsymbol}{\glsuseri} {echtesTeilsymbol}
\newVerweis             {\eTsymbol}{\glsuserii}{echtesTeilsymbol}
\newglossaryentry{echtesTeilsymbol}{
	name       =              {---, echtes \addIdx[
		name   =              {---, echtes},
		sort   =       {Teilsymbol, echtes}]   {echtesTeilsymbol}},
	sort       =       {Teilsymbol, echtes},
	text       ={echtes Teilsymbol},
	user1      ={echten Teilsymbol},
	user2      =           {symbol},
	description={\todoBeschreiben%
	}
}

\newVerweis     {\Traegermenge} {\glstext}{Traegermenge}
\newVerweis[n]  {\Traegermengen}{\glstext}{Traegermenge}
\newglossaryentry{Traegermenge}{
	name        ={Trägermenge \addIdx[
		name    ={Trägermenge}]           {Traegermenge}},
	text        ={Trägermenge},
	see         ={MtsTraeger},
	description ={\todoPruefen%
		einer \Relation.
	}
}

\newVerweis     {\Transformation}  {\glstext}{Transformation}
\newVerweis[en] {\Transformationen}{\glstext}{Transformation}
\newglossaryentry{Transformation}{
	name        ={Transformation \addIdx     {Transformation}},
	text        ={Transformation},
	see         ={MtsTransformation,MtsTransformationTup,zulaessigeTransformation},
	description ={\todoPruefen%
		Eine Umformung oder Erzeugung einer \Formel\ aus einer vorgegebenen \Menge\ von \Formeln, \textdh\ die Anwendung einer \Schlussregel.
	}
}

\newVerweis      {\zulaessigeTransformation}  {\glstext}  {zulaessigeTransformation}
\newVerweis[en]  {\zulaessigeTransformationen}{\glstext}  {zulaessigeTransformation}
\newVerweis     {\zulaessigenTransformation}  {\glsuseri} {zulaessigeTransformation}
\newVerweis[en] {\zulaessigenTransformationen}{\glsuseri} {zulaessigeTransformation}
\newVerweis[en] {\zulaessigerTransformationen}{\glsuserii}{zulaessigeTransformation}
\newglossaryentry{zulaessigeTransformation}{
	name        =                      {---, zulässige \addIdx[
		name    =                      {---, zulässige},
		sort    =           {Transformation, zulässige}]  {zulaessigeTransformation}},
	sort        =           {Transformation, zulässige},
	text        ={zulässige  Transformation},
	user1       ={zulässigen Transformation},
	user2       ={zulässiger Transformation},
	description ={\todoPruefen%
		Eine \Transformation\ heißt \GloFt{zulässig}, wenn sie \Element\ aus einer vorgegebenen \Menge\ von \Transformationen\ oder eine daraus zulässigerweise abgeleitete \Transformation\ ist.
	}
}

\newVerweis     {\Transformationsfolge} {\glstext}{Transformationsfolge}
\newVerweis[n]  {\Transformationsfolgen}{\glstext}{Transformationsfolge}
\newglossaryentry{Transformationsfolge}{
	name        ={Transformationsfolge \addIdx    {Transformationsfolge}},
	text        ={Transformationsfolge},
	see         ={MtsTransformation,MtsTransformationTup,Transformation},
	description ={\todoPruefen%
		Eine Folge von \Transformationen.
	}
}

\newVerweis     {\Transformationsregel} {\glstext}{Transformationsregel}
\newVerweis[n]  {\Transformationsregeln}{\glstext}{Transformationsregel}
\newglossaryentry{Transformationsregel}{
	name        ={Transformationsregel \addIdx    {Transformationsregel}},
	text        ={Transformationsregel},
	description ={\todoBeschreiben%
	}
}

\newVerweis         {\Tupel} {\glstext}{Tupel}
\newVerweis[s]      {\Tupels}{\glstext}{Tupel}
\longnewglossaryentry{Tupel}{
	name            ={Tupel \addIdx    {Tupel}},
	text            ={Tupel},
	see             ={Folge,Komponente,Menge,Objekt,Symbolfolge,Zeichenkette},
}{\todoPruefen%
	\wikicite{bib:Tupel}{
		\wikiBoldFt{Tupel} (abgetrennt von \wikiLinkFt{mittellat.} \wikiItalicFt{quintuplus} ‚fünffach‘, \wikiItalicFt{septuplus} ‚siebenfach‘, \wikiItalicFt{centuplus} ‚hundertfach‘ etc.) sind in der \wikiLinkFt{Mathematik} neben \wikiLinkFt{Mengen} eine wichtige Art und Weise, \wikiLinkFt{mathematische Objekte} zusammenzufassen. Ein Tupel besteht aus einer \wikiLinkFt{Liste} endlich vieler, nicht notwendigerweise voneinander verschiedener Objekte. Dabei spielt, im Gegensatz zu Mengen, die Reihenfolge der Objekte eine Rolle. Es gibt verschiedene Möglichkeiten, Tupel formal als Mengen darzustellen. Tupel finden in vielen Bereichen der Mathematik Verwendung, zum Beispiel als \wikiLinkFt{Koordinaten} von Punkten oder als \wikiLinkFt{Vektoren} in mehrdimensionalen \wikiLinkFt{Vektorräumen}.

		Von Tupeln unabhängig von ihrer Länge ist selten die Rede. Vielmehr verwendet man das Wort \wikiBoldFt{$n$-Tupel} und die im nächsten Abschnitt genannten Spezialfälle davon dann, wenn sich aus dem Zusammenhang die Länge als feste Zahl oder als benannte Konstante wie $n$ ergibt. Betrachtet man dagegen viele endliche Folgen unterschiedlicher Längen von Elementen einer Grundmenge, spricht man von endlichen Folgen oder definiert einen neuen Begriff, der oft mit „Kette“ zusammengesetzt ist, z. B. \wikiLinkFt{Zeichenkette}, \wikiLinkFt{Additionskette}.
	}
	Ein \GloFt{$n$-Tupel}\alternativi{Vektor} $\vec{a}$ ist eine endliche \Folge\alternativi{Sequenz} $(a_1, \dots, a_n)$ \DefFt{von} seinen \DefFt{Komponenten} $a_i$.
	Sind alle Komponenten \Elemente\ aus derselben \Menge\ $M$, so heißt $\vec{a}$ ein $n$-\Tupel\ \DefFt{auf} $M$.
}

\newVerweis     {\Tupelmenge} {\glstext}{Tupelmenge}
\newVerweis[n]  {\Tupelmengen}{\glstext}{Tupelmenge}
\newglossaryentry{Tupelmenge}{
	name        ={Tupelmenge \addIdx    {Tupelmenge}},
	text        ={Tupelmenge},
	description ={\todoPruefen%
		Die \Tupelmenge\ $\MtsTup(M)$ einer \Menge\ $M$ ist die \Menge\ aller $n$-Tupel aus $M^n$ für alle $n \in \MtsINo$.
	}
}

%U === U === U === U === U === U === U === U === U === U === U === U === U === U

\newVerweis     {\Umkehrrelation}  {\glstext}{Umkehrrelation}
\newVerweis[en] {\Umkehrrelationen}{\glstext}{Umkehrrelation}
\newglossaryentry{Umkehrrelation}{
	name        ={Umkehrrelation \addIdx     {Umkehrrelation}},
	text        ={Umkehrrelation},
	see         ={Menge},
	description ={\todoPruefen%
		Die \GloFt{Umkehrrelation}\alternativiii{konverse Relation}{Konverse }{inverse Relation } \emph{von} einer \binaeren\ \Relation\ $(G,A,B)$ ist die \Relation\ $(H,B,A)$ mit $H = \RawMengeDef{(b,a)}{(a,b) \in G}$.
		Üblicherweise wird das zugehörige \Relationssymbol\ gespiegelt.
		Die \gloFt{Umkehrrelation} der \Negation\ einer \Relation\ ist gleich der \Negation\ ihrer \gloFt{Umkehrrelation}.
	}
}

\newVerweis     {\unaer}  {\glstext}{unaer}
\newVerweis[e]  {\unaere} {\glstext}{unaer}
\newVerweis[en] {\unaeren}{\glstext}{unaer}
\newVerweis[er] {\unaerer}{\glstext}{unaer}
\newglossaryentry{unaer}{
	name        ={unär \addIdx[
		name    ={unär}]            {unaer}},
	text        ={unär},
	see         ={binaer},
	description ={\todoPruefen%
		Eine \Operation, \Funktion\ oder \Relation\ heißt \GloFt{unär}, wenn ihre \Stelligkeit\ gleich 1 ist.
	}
}

\newVerweis     {\Ungleichheit}{\glstext}{Ungleichheit}
\newglossaryentry{Ungleichheit}{
	name        ={Ungleichheit \addIdx   {Ungleichheit}},
	text        ={Ungleichheit},
	description ={\todoPruefen%
		Eine \Gleichheitsrelation:
		Zwei Objekte $A$ und $B$ sind \DefFt{nicht gleich}\alternativii{nicht dasselbe}{nicht identisch} $A \MtsEqN B$, wenn sie in mindestens einer \interessierendenEigenschaft\ für \MtsEq\ nicht übereinstimmen.
	}
}

\newsynonym{\Unteraussage}{Unteraussage}{\Teilaussage}
\newsynonym{\Unterformel} {Unterformel} {\Teilformel}
\newsynonym{\Untermenge}  {Untermenge}  {\Teilmenge}
\newsynonym{\Unterobjekt} {Unterobjekt} {\Teilobjekt}
\newsynonym{\Untersymbol} {Untersymbol} {\Teilsymbol}

\newsynonym{\unzerlegbar} {unzerlegbar} {\atomar}

%V === V === V === V === V === V === V === V === V === V === V === V === V === V

\newVerweis         {\Variable} {\glstext}{Variable}
\newVerweis[n]      {\Variablen}{\glstext}{Variable}
\longnewglossaryentry{Variable}{
	name            ={Variable \addIdx    {Variable}},
	text            ={Variable},
	see             ={Konstante},
}{\todoGeprueft%
	\wikicite{bib:Variable}{
		Eine \wikiBoldFt{Variable} ist ein Name für eine Leerstelle in einem logischen oder mathematischen Ausdruck.[1]Der Begriff leitet sich vom lateinischen \wikiLinkFt{Adjektiv} \wikiItalicFt{variabilis} (veränderlich) ab. Gleichwertig werden auch die Begriffe \wikiItalicFt{Platzhalter} oder \wikiItalicFt{Veränderliche} benutzt. Als „Variable“ dienten früher Wörter oder Symbole, heute verwendet man zur \wikiLinkFt{mathematischen Notation} in der Regel Buchstaben als Zeichen. Wird anstelle der Variablen ein konkretes Objekt eingesetzt, so ist darauf zu achten, dass überall dort, wo die Variable auftritt, auch dasselbe Objekt benutzt wird.
	}
}

\newVerweis      {\aussagenlogischeVariable} {\glstext}         {aussagenlogischeVariable}
\newVerweis[n]  {\aussagenlogischenVariablen}{\glsuseri}        {aussagenlogischeVariable}
\newVerweis     {\aussagenlogischenV}        {\glsuserii}       {aussagenlogischeVariable}
\newglossaryentry {aussagenlogischeVariable}{
	name       =                       {---, aussagenlogische \addIdx[
		name   =                       {---, aussagenlogische},
		sort   =                  {Variable, aussagenlogische}] {aussagenlogischeVariable}},
	sort       =                  {Variable, aussagenlogische},
	text       ={aussagenlogische  Variable},
	user1      ={aussagenlogischen Variable},
	user2      ={aussagenlogischen},
	description={\todoPruefen%
		Die \GloFt{aussagenlogischen} \Variablen\ sind die \Elemente\ aus \OjkVar.
	}
}

\newVerweis      {\freieVariable} {\glstext}  {freieVariable}
\newVerweis[n]   {\freieVariablen}{\glstext}  {freieVariable}
\newVerweis[n]  {\freienVariablen}{\glsuseri} {freieVariable}
\newVerweis       {\freiV}        {\glsuserii}{freieVariable}
\newVerweis[e]   {\freieV}        {\glsuserii}{freieVariable}
\newglossaryentry {freieVariable}{
	name       =            {---, freie \addIdx[
		name   =            {---, freie},
		sort   =       {Variable, freie}]{freieVariable}},
	sort       =       {Variable, freie},
	text       ={freie  Variable},
	user1      ={freien Variable},
	user2      ={frei},
	description={\todoGeprueft%
		Eine \Variable\ heißt \GloFt{frei}, wenn sie nicht \gebundenV\ ist.
	}
}

\newVerweis      {\gebundeneVariable} {\glstext}  {gebundeneVariable}
\newVerweis[n]   {\gebundeneVariablen}{\glstext}  {gebundeneVariable}
\newVerweis[n]  {\gebundenenVariablen}{\glsuseri} {gebundeneVariable}
\newVerweis       {\gebundenV}        {\glsuserii}{gebundeneVariable}
\newVerweis[e]   {\gebundeneV}        {\glsuserii}{gebundeneVariable}
\newglossaryentry {gebundeneVariable}{
	name       =            {---, gebundene \addIdx[
		name   =            {---, gebundene},
		sort   =       {Variable, gebundene}]{gebundeneVariable}},
	sort       =       {Variable, gebundene},
	text       ={gebundene  Variable},
	user1      ={gebundenen Variable},
	user2      ={gebunden},
	description={\todoGeprueft%
		Eine \Variable\ heißt durch einen bestimmten \Quantor\ \GloFt{gebunden}, wenn sie die zum \Quantor\ gehörige \Variable\ ist und im zugehörigen \Ausdruck\ auch \freiV\ vorkommt.
	}
}

\newVerweis      {\logischeVariable} {\glstext} {logischeVariable}
\newVerweis[n]   {\logischeVariablen}{\glstext} {logischeVariable}
\newVerweis      {\logischeV}        {\glsuseri}{logischeVariable}
\newglossaryentry {logischeVariable}{
	name       =               {---, logische \addIdx[
		name   =               {---, logische},
		sort   =          {Variable, logische}] {logischeVariable}},
	sort       =          {Variable, logische},
	text       ={logische  Variable},
	user1      ={logische},
	description={\todoPruefen%
		Die \GloFt{logischen} \Variablen\ entsprechen den \aussagenlogischenV.
	}
}

\newVerweis     {\metasprachlicheVariable}{\glstext}        {metasprachlicheVariable}
\newVerweis     {\metasprachlicheV}       {\glsuseri}       {metasprachlicheVariable}
\newglossaryentry{metasprachlicheVariable}{
	name       =                     {---, metasprachliche \addIdx[
		name   =                     {---, metasprachliche},
		sort   =                {Variable, metasprachliche}]{metasprachlicheVariable}},
	sort       =                {Variable, metasprachliche},
	text       ={metasprachliche Variable},
	user1      ={metasprachliche},
	description={\todoErgaenzen%
		Die \GloFt{metasprachlichen} \Variablen\ sind die \Elemente\ aus ...
	}
}

\newVerweis     {\Vereinigung} {\glstext}{Vereinigung}
\newglossaryentry{Vereinigung}{
	name        ={Vereinigung \addIdx    {Vereinigung}},
	text        ={Vereinigung},
	description ={\todoErgaenzen%
		Eine \Bereichsoperation: \todo{Beschreibung fehlt noch}
	}
}

\newVerweis     {\vergleichbar} {\glstext}{vergleichbar}
\newVerweis     {\Vergleichbar} {\Glstext}{vergleichbar}
\newVerweis[e]  {\vergleichbare}{\glstext}{vergleichbar}
\newglossaryentry{vergleichbar}{
	name        ={vergleichbar \addIdx    {vergleichbar}},
	text        ={vergleichbar},
	description ={\todoPruefen%
		Zwei \Objekte\ $A$ und $B$ sind \vergleichbar, wenn beide von derselben \Objektart\ sind, \textdh\ wenn beide \textzB\ jeweils Mengen, \Symbolfolgen, Zahlen, \textusw\ sind.
		Dabei muss bei \Formeln\ zwischen der \Formel\ an sich und ihrem \emph{Wert} oder \emph{Ergebnis} unterschieden werden.
		%%%  Wert und Ergebnis definieren?
	}
}

\newVerweis     {\Verkettung} {\glstext}{Verkettung}
\newglossaryentry{Verkettung}{
	name        ={Verkettung \addIdx    {Verkettung}},
	text        ={Verkettung},
	description ={\todoBeschreiben%
	}
}

\newVerweis     {\Vertauschung}  {\glstext}{Vertauschung}
\newVerweis[en] {\Vertauschungen}{\glstext}{Vertauschung}
\newglossaryentry{Vertauschung}{
	name        ={Vertauschung \addIdx     {Vertauschung}},
	text        ={Vertauschung},
	description ={\todoPruefen%
		Die \GloFt{Vertauschung} von zwei unabhängigen Teil-\Formeln\ ($\alpha$ und $\beta$) in einer anderen \Formel\ ($\gamma$)
		\\--- Formal: $\gamma(\alpha \MtsSwap \beta)$.
		Die \gloFt{Vertauschung} ist eine spezielle Form der \Ersetzung.
	}
}

\newsynonym{\Voraussetzung}{Voraussetzung}{\Praemisse}

%W === W === W === W === W === W === W === W === W === W === W === W === W === W

\newVerweis         {\Wahrheitswert}  {\glstext}{Wahrheitswert}
\newVerweis[e]      {\Wahrheitswerte} {\glstext}{Wahrheitswert}
\newVerweis[en]     {\Wahrheitswerten}{\glstext}{Wahrheitswert}
\longnewglossaryentry{Wahrheitswert}{
	name            ={Wahrheitswert \addIdx     {Wahrheitswert}},
	text            ={Wahrheitswert},
	see             ={atomar,Aussage,Element,Junktor,Logik,Satz,Teilaussage},
}{\todoOk%
	\wikicite{bib:Wahrheitswert}{
		Ein \wikiBoldFt{Wahrheitswert} ist in \wikiLinkFt{Logik} und \wikiLinkFt{Mathematik} ein \wikiItalicFt{logischer Wert}, den eine Aussage in Bezug auf Wahrheit annehmen kann.

		In der zweiwertigen \wikiLinkFt{klassischen Logik} kann eine Aussage nur entweder \wikiItalicFt{wahr} oder \wikiItalicFt{falsch} sein, die Menge der Wahrheitswerte $\{W, F\}$ hat so zwei Elemente. In \wikiLinkFt{mehrwertigen Logiken} enthält die \wikiLinkFt{Wahrheitswertemenge} mehr als zwei Elemente, z. B. in einer \wikiLinkFt{dreiwertigen Logik} oder einer \wikiLinkFt{Fuzzy-Logik}, die damit zu den \wikiLinkFt{nichtklassischen} zählen. Hier wird dann auch neben Wahrheitswerten von \wikiItalicFt{Quasiwahrheitswerten}, \wikiItalicFt{Pseudowahrheitswerten} oder \wikiItalicFt{Geltungswerten} gesprochen.

		Die Abbildung der Menge von Aussagen einer (meist formalen) Sprache auf die Wahrheitswertemenge wird \wikiLinkFt{Wahrheitswertzuordnung}  genannt und ist eine aussagenlogisch spezifische \wikiLinkFt{Bewertungsfunktion}. In der klassischen Logik kann auch explizit die Klasse aller wahren Aussagen beziehungsweise die Klasse aller falschen Aussagen definiert werden. Die Abbildung von Wahrheitswerten der (\wikiLinkFt{atomaren}) Teilaussagen einer zusammengesetzten Aussage auf die Wahrheitswertemenge heißt \wikiLinkFt{Wahrheitswertefunktion} oder Wahrheitsfunktion. Die Wertetabelle dieser \wikiLinkFt{Funktion} im mathematischen Sinn wird auch als \wikiLinkFt{Wahrheitstafel} bezeichnet und häufig dazu verwendet, die Bedeutung wahrheitsfunktionaler \wikiLinkFt{Junktoren} anzugeben.
	}
	\nurImGlossar{
		Wir verwenden nur die beiden \GloFt{Wahrheitswerte} der zweiwertigen klassischen \Logik, die wir (in der \Metasprache) mit \chrqt{\TxtTrue} und \chrqt{\TxtFalse} bezeichnen.
		In der \formalenMetasprache\ hingegen verwenden wir \chrqt{\MtsTrue} und \chrqt{\MtsFalse} und in der \Objektsprache\ \chrqt{\OjkTrue} und \chrqt{\OjkFalse}.
		In der Literatur findet man auch einfach \chrqt{$1$} und \chrqt{$0$}.

		Ist statt Wahrheit nur Beweisbarkeit von Interesse, so gelangt man zum Intuitionismus, in dem der Satz vom ausgeschlossenen Dritten\citenote{bib:TertiumNonDatur} nicht gilt.

		\wikiciteChapter{bib:Intuitionismus}{1}{
			Die Wahrheit eines mathematischen Satzes wird im Intuitionismus bezogen auf die Möglichkeit, einen entsprechenden Beweis zu formulieren. Wahrheit entsteht also erst durch die Verifizierung. Wahre Sätze oder von ihnen beschriebene Objekte haben keine Existenz unabhängig von tatsächlichen Denkprozessen. Dies steht im Kontrast unter anderem zum sog. \wikiLinkFt{Platonismus} in der Philosophie der Mathematik.
		}
	}
	\nichtImGlossar{
		Ein \GloFt{Wahrheitswert} ist ein \Wert, den eine \Aussage\ in Bezug auf Wahrheit annehmen kann.
		Für die \Darstellung\ der \Wahrheitswerte\ abhängig von der \Sprachebene\ und dem logischen Wert der Aussage definieren wir:
		\begin{table}[H]
			\centering
			\begin{threeparttable}
				\setlength\extrarowheight{3pt}
				\begin{tabularx}{10cm}{l@{\extracolsep{\fill}}|cc|c|}
					& \multicolumn{2}{c|}{ Aussagewert } & \\
					\TabFt{\Sprachebene} & \TabFt{wahr} & \TabFt{falsch} & \TabFt{Symbolart} \\
					\hline
					\Metasprache          & \TxtTrue & \TxtFalse & normaler Text \\
					\formaleMetasprache~~ & \MtsTrue & \MtsFalse & \Metasymbol   \\
					\Objektsprache        & \OjkTrue & \OjkFalse & \Objektsymbol \\
					\hline
				\end{tabularx}
			\end{threeparttable}
			\caption{\Darstellung\ der \Wahrheitswerte}
			\label{tab:Wahrheitswerte}% Erst nach '\caption'!
		\end{table}
		Wir schließen nicht aus, dass es weitere \gloFt{Wahrheitswerte} gibt.
	}
}

\newVerweis     {\aussagenlogischerWahrheitswert}{\glstext}          {aussagenlogischerWahrheitswert}
\newglossaryentry{aussagenlogischerWahrheitswert}{
	name       =                            {---, aussagenlogischer \addIdx[
		name   =                            {---, aussagenlogischer},
		sort   =                  {Wahrheitswert, aussagenlogischer}]{aussagenlogischerWahrheitswert}},
	sort       =                  {Wahrheitswert, aussagenlogischer},
	text       ={aussagenlogischer Wahrheitswert},
	description={\todoGeprueft%
		Es gib nur die beiden \GloFt{aussagenlogischen \Wahrheitswerte} \OjkTrue\ und \OjkFalse.
	}
}

\newVerweis     {\metasprachlicherWahrheitswert}{\glstext}         {metasprachlicherWahrheitswert}
\newVerweis      {\metasprachlicheWahrheitswert}{\glsuseri}        {metasprachlicherWahrheitswert}
\newglossaryentry{metasprachlicherWahrheitswert}{
	name       =                           {---, metasprachlicher \addIdx[
		name   =                           {---, metasprachlicher},
		sort   =                 {Wahrheitswert, metasprachlicher}]{metasprachlicherWahrheitswert}},
	sort       =                 {Wahrheitswert, metasprachlicher},
	text       ={metasprachlicher Wahrheitswert},
	user1      ={metasprachliche  Wahrheitswert},
	description={\todoPruefen%
		Es gib die beiden \GloFt{metasprachlichen \Wahrheitswerte} in Textform (\TxtTrue, \TxtFalse) und in der \formalenMetasprache\ (\MtsTrue, \MtsFalse).
	}
}

\newVerweis     {\Wert}  {\glstext}{Wert}
\newVerweis[en] {\Werten}{\glstext}{Wert}
\newglossaryentry{Wert}{
	name        ={Wert \addIdx     {Wert}},
	text        ={Wert},
	description ={\todoGeprueft%
		Der \GloFt{Wert} einer \Formel\ ergibt sich rekursiv aus der \Belegung\ der \Symbole, aus denen die \Formel\ besteht.
		Beispielsweise hat die \Formel\ \seqqt{a+b=c} mit der \Belegung\ von \chrqt{$a$}, \chrqt{$b$}, \chrqt{$c$}, \chrqt{$+$} und \chrqt{$=$} durch die Zahlen Eins, Zwei und Drei, den Additionsoperator und die Gleichheit den \gloFt{Wert} "`\TxtTrue"'.%
		\footnote{Genau genommen \MtsTrue, was wiederum standardmäßig die \Belegung\ \TxtTrue hat.}
		Belegt man bei sonst gleicher Belegung \chrqt{$c$} mit Vier, so ist der \gloFt{Wert} hingegen "`\TxtFalse"'.
	}
}

\newVerweis     {\logischerWert}{\glstext}  {logischerWert}
\newVerweis      {\logischeWert}{\glsuseri} {logischerWert}
\newglossaryentry{logischerWert}{
	name       =           {---, logischer \addIdx[
		name   =           {---, logischer},
		sort   =          {Wert, logischer}]{logischerWert}},
	sort       =          {Wert, logischer},
	text       ={logischer Wert},
	user1      ={logische  Wert},
	description={\todoGeprueft%
		Synonym zu \Wahrheitswert.
	}
}

\newVerweis     {\Wertebereich} {\glstext}{Wertebereich}
\newVerweis[e]  {\Wertebereiche}{\glstext}{Wertebereich}
\newglossaryentry{Wertebereich}{
	name        ={Wertebereich \addIdx    {Wertebereich}},
	text        ={Wertebereich},
	see         ={MtsWb,Zielbereich,Funktion},
	description ={\todoPruefen%
		einer \Funktion.
	}
}

\newVerweis         {\Wikipedia}{\glstext}{Wikipedia}
\longnewglossaryentry{Wikipedia}{
	name            ={Wikipedia \addIdx   {Wikipedia}},
	text            ={Wikipedia},
}{\todoPruefen%
	\wikicite{bib:Wikipedia}{
		Wikipedia ist ein Projekt zum Aufbau einer [Internet-\nobreak]Enzyklopädie aus freien Inhalten.
	}
}

\newVerweis     {\Wort}   {\glstext}{Wort}
\newVerweis[e]  {\Worte}  {\glstext}{Wort}
\newVerweis     {\Woerter}{\glspl}  {Wort}
\newglossaryentry{Wort}{
	name        ={Wort \addIdx      {Wort}},
	text        ={Wort},
	plural      ={Wörter},
	see         ={Formelmenge},
	description ={\todoPruefen%
		Synonym: \Formel\ ---
		Ein \Element\ aus einer \Sprache.
	}
}

%Z === Z === Z === Z === Z === Z === Z === Z === Z === Z === Z === Z === Z === Z

\newVerweis         {\natuerlicheZahl}  {\glstext} {natuerlicheZahl}
\newVerweis        {\natuerlichenZahl}  {\glsuseri}{natuerlicheZahl}
\newVerweis[en]    {\natuerlichenZahlen}{\glsuseri}{natuerlicheZahl}
\longnewglossaryentry{natuerlicheZahl}{
	name           =            {Zahl, natürliche \addIdx[
		name       =            {Zahl, natürliche}]{natuerlicheZahl}},
	text           ={natürliche  Zahl},
	user1          ={natürlichen Zahl},
}{\todoBeschreiben%
	Eine verbreitete Version für die Definition der \Menge\ \MtsINo\ der \GloFt{natürlichen Zahlen} ist folgende:
	\begin{itemize}
		\item [] $\emptyset \MtsIn \MtsINo$
		\item [] $n \MtsIn \MtsINo \MtsImp n \MtsCup \{n\} \MtsIn \MtsINo$
		\item [] Nur die so definierten \Elemente\ sind \Elemente\ aus \MtsINo.
	\end{itemize}
	Man nennt $n \MtsCup \{n\}$ auch den \DefFt{Nachfolger} von $n$ und es gilt:
	\begin{itemize}
		\item [] $n \MtsSubset \MtsINo$               , für $n   \MtsIn \MtsINo$
		\item [] $n  <  m \MtsEquiv n \MtsSubset   m$ , für $n,m \MtsIn \MtsINo$
		\item [] $n \le m \MtsEquiv n \MtsSubsetEq m$ , für $n,m \MtsIn \MtsINo$
	\end{itemize}
}

\newVerweis     {\Zeichenkette} {\glstext}{Zeichenkette}
\newVerweis[n]  {\Zeichenketten}{\glstext}{Zeichenkette}
\newglossaryentry{Zeichenkette}{
	name        ={Zeichenkette \addIdx    {Zeichenkette}},
	text        ={Zeichenkette},
	see         ={Symbolfolge},
	description ={\todoPruefen%
		Eine Folge von (typographischen) Zeichen, auch Leerstellen und sonstigem Zwischenraum.
	}
}

\newVerweis     {\zerlegbar}  {\glstext}{zerlegbar}
\newVerweis[e]  {\zerlegbare} {\glstext}{zerlegbar}
\newVerweis[e]  {\Zerlegbare} {\Glstext}{zerlegbar}
\newVerweis[es] {\zerlegbares}{\glstext}{zerlegbar}
\newglossaryentry{zerlegbar}{
	name        ={zerlegbar \addIdx     {zerlegbar}},
	text        ={zerlegbar},
	see         ={atomar},
	description ={\todoPruefen%
		Eine \Aussage, \Formel, \Folge\ oder \Symbol, die eine \echteTeilaussage,  -\eTfolge, -\eTformel\ \textbzw. -\eTsymbol\ enthalten, heißt \GloFt{zerlegbar}.
	}
}

\newVerweis     {\Ziel} {\glstext}{Ziel}
\newVerweis[e]  {\Ziele}{\glstext}{Ziel}
\newglossaryentry{Ziel}{
	name        ={Ziel \addIdx    {Ziel}},
	text        ={Ziel},
	description ={\todoPruefen%
		Ein \GloFt{Ziel} ist \hier\ eine Anforderungen an \ASBA.
	}
}

\newVerweis     {\Zielbereich} {\glstext}{Zielbereich}
\newVerweis[e]  {\Zielbereiche}{\glstext}{Zielbereich}
\newglossaryentry{Zielbereich}{
	name        ={Zielbereich \addIdx    {Zielbereich}},
	text        ={Zielbereich},
	see         ={MtsZb,Wertebereich,Funktion},
	description ={\todoPruefen%
		einer \Funktion.
	}
}

\newVerweis     {\zulaessig}  {\glstext}{zulaessig}
\newVerweis[e]  {\zulaessige} {\glstext}{zulaessig}
\newVerweis[en] {\zulaessigen}{\glstext}{zulaessig}
\newVerweis[er] {\zulaessiger}{\glstext}{zulaessig}
\newglossaryentry{zulaessig}{
	name        ={zulässig \addIdx[
		name    ={zulässig}]            {zulaessig}},
	text        ={zulässig},
	see         ={Formel,Transformation,Ersetzung},
	description ={\todoPruefen%
		Eine Eigenschaft von \Formel, \Transformation\ und \Ersetzung.
	}
}


% Titelseite ###################################################################

\titlehead{
	{\Large Dr. Winfried Teschers}\\
	Anton-Günther-Straße 26c\\91083 Baiersdorf\\
	{\footnotesize winfried.teschers@t-online.de}
}
\subject{Projektdokument}
\title{{\Huge ASBA}\\Axiome, Sätze, Beweise und Auswertungen}
\subtitle{Projekt zur maschinellen Überprüfung von mathematischen \Beweisen\ und deren Ausgabe in lesbarer Form}
\author{Winfried Teschers}
\date{\today}
\publishers{\vspace{1cm}\normalsize
	Es wird ein Programmsystem beschrieben, das zu eingegebenen \Axiomen, \Saetzen\ und \Beweisen\ letztere prüft, \Auswertungen\ generiert und unter Zuhilfenahme gegebener \Ausgabeschemata\ eine Ausgabe im \LaTeX-Format in mathematisch üblicher Schreibweise mit \Formeln\ erstellt.
}

% Dokument #####################################################################

\begin{document}
	\maketitle
	~\vfill Copyright \copyright\ 2018 Winfried Teschers\bigskip

	\begin{otherlanguage}{english}
		Permission is granted to copy, distribute and/or modify this document under the terms of the GNU Free Documentation License, Version~1.3 or any later version published by the Free Software Foundation; with no Invariant Sections, no Front-Cover Texts, and no Back-Cover Texts.
		You should have received a copy of the GNU Free Documentation License along with this document.
		If not, see \url{http://www.gnu.org/licenses/}.
	\end{otherlanguage}

	%chapter{Inhaltsverzeichnis}% ##############################################
	\tableofcontents
	\Endchapter

	%%############################################################################%%
%%                                                                            %%
%% Datei:  ASBA-Vorwort.tex                                                   %%
%% Inhalt: Kapitel "Vorwort" und "Vereinbarungen"                             %%
%%                                                                            %%
%% Copyright (C) 2017  Winfried Teschers                                      %%
%%                                                                            %%
%% This program is free software: you can redistribute it and/or modify       %%
%% it under the terms of the GNU Affero General Public License as published   %%
%% by the Free Software Foundation, either version 3 of the License, or       %%
%% (at your option) any later version.                                        %%
%%                                                                            %%
%% This program is distributed in the hope that it will be useful,            %%
%% but WITHOUT ANY WARRANTY; without even the implied warranty of             %%
%% MERCHANTABILITY or FITNESS FOR A PARTICULAR PURPOSE.  See the              %%
%% GNU Affero General Public License for more details.                        %%
%%                                                                            %%
%% You should have received a copy of the GNU Affero General Public License   %%
%% along with this program.  If not, see <http://www.gnu.org/licenses/>.      %%
%%                                                                            %%
%% Dr. Winfried Teschers                                                      %%
%% Anton-Günther-Straße 26c                                                   %%
%% 91083 Baiersdorf                                                           %%
%% Germany                                                                    %%
%%                                                                            %%
%% e-mail: winfried.teschers@t-online.de                                      %%
%%                                                                            %%
%%############################################################################%%

% !TeX root = ASBA.tex
% !TeX encoding = UTF-8
% !TeX spellcheck = de_DE

%\chapter                     {Vorwort}% #######################################
\phantomsection% sichert korrekten Link im Inhaltsverzeichnis
\label                    {cha:Vorwort}
~\vskip 1.6cm
\likeChapterFt                {Vorwort}
\vskip 0.8cm
\beginchapter[]               {Vorwort}
\addcontentsline{toc}{chapter}{Vorwort}% Eintrag ins Inhaltsverzeichnis

Schon während meiner aktiven Zeit habe ich davon geträumt, ein Programm zu erstellen, mit dem man mathematische Sätze und Beweise speichern und überprüfen kann.
Es sollte auch statistische Auswertungen beherrschen und \textua\ Fragen beantworten können wie \textzB\
"`Welche Axiome sind zum Beweis eines bestimmten Satzes erforderlich?"' oder
"`Wie viele Beweisschritte erfordert ein bestimmter Beweis?"'.
Ein Beweis mit weniger Axiomen und weniger Beweisschritten wäre dann vorzuziehen.

Einige Jahre nach meiner Pensionierung habe ich Ende 2016 endlich damit angefangen, das Projekt ASBA zu starten.
Im Internet habe ich das Projekt "`Hilbert II"' \cite{bib:HilbertII} gefunden, dass eine ähnliche Zielsetzung hat.
Ich habe dann mit dem Projektleiter Michael Meyling Kontakt aufgenommen und war zuversichtlich, Synergien nutzen zu können.
Leider hat sich dann herausgestellt, dass mein Ansatz viel umfangreicher und somit mit "`Hilbert II"' wohl nicht kompatibel ist.
Daher betreibe ich ASBA als ein Ein-Mann-Projekt und dies wird bis zur Fertigstellung der ersten Version dieses Dokuments wohl so bleiben müssen.
Vielleicht ergibt sich dann ja eine Zusammenarbeit mit anderen Enthusiasten.

Da \hier\ viele mathematische Formeln vorkommen und ASBA auch \LaTeX-Code generieren soll, ist es in \LaTeX\ verfasst.
Dieses für mich neue Textsystem war eine große, spannende Herausforderung und ist einer der Gründe für die lange Dauer der Erstellung dieses Dokuments.
Hinzu kommt, dass ich keinen Termindruck habe und endlich mal 100\% versuchen kann -- in meinem Job wurde ich daran aus verständlichen Gründen gehindert.

ASBA soll eine Basis für die Überprüfung und Archivierung mathematischer Sätze und Beweise sein.
Daher halte ich es für unerlässlich, alle verwendeten Begriffe und Bezeichnungen (\textdh\ Benennungen und Symbole) eindeutig genug zu definieren (100\%!).
Natürlich will ich mich dabei an die übliche Nomenklatur halten.
Aber was ist üblich?
Steht \MtsSubset\ für "`Teilmenge"' oder "`echteTeilmenge"'?
Ist $0$ ein Element aus \MtsIN\ oder nicht?
Daher habe ich versucht, alle wichtigen, verwendeten Bezeichnungen der Mathematik, mit dem Schwerpunkt Logik, aber auch der formalen Metasprache streng zu definieren, normalerweise im Text, teilweise aber nur in einer Fußnote, auf jeden Fall aber im Glossar.
Dort sind auch manche Bezeichnungen aufgeführt, die im Text nicht definiert wurden.

\bigskip

Baiersdorf, den 07. Dezember 2018

Winfried Teschers

\Endchapter

\newpage

%\chapter                     {Vereinbarungen}% ################################
\phantomsection% sichert korrekten Link im Inhaltsverzeichnis
\label                    {cha:Vereinbarungen}
~\vskip 1.6cm
\likeChapterFt                {Vereinbarungen}
\vskip 0.8cm
\beginchapter[]               {Vereinbarungen}
\addcontentsline{toc}{chapter}{Vereinbarungen}% Eintrag ins Inhaltsverzeichnis

\Hier\ werden verschiedene Textauszeichnungen mit folgenden Bedeutungen verwendet:
\begin{itemize}

	\item In mathematischen Formeln:
	\begin{itemize}
		\item $\Varft      {Variable\ allgemein}$; normalerweise ein Buchstabe.
		\item $\varft          {Variablensymbol}$; normalerweise ein Kleinbuchstabe.
		\item $\Conft                {Konstante}$; normalerweise ein Wort.
		\item $\Idxft        {Konstanter\ Index}$; normalerweise ein Buchstabe.
		\item $\Setft  {VORGEGEBENE\ \ BEREICHE}$; normalerweise ein Großbuchstabe.%
			\footnote{Kleinbuchstaben gibt es in dieser Schriftart nicht.}
		\item $\Elmft          {Element\ daraus}$; normalerweise ein Großbuchstabe.
		\item $\sOpft        {Bereichsoperation}$; normalerweise ein Wort.
		\item $\Drvft{Bereich\ von\ Ableitungen}$; normalerweise ein Großbuchstabe.
		\item $\drvft          {Element\ daraus}$; normalerweise ein Buchstabe.
		\item $\Preft               {Pr\"adikat}$; normalerweise ein Wort.
	\end{itemize}

	\item In Zitaten aus \Wikipedia:
	\begin{itemize}
		\item \likeWikiFt  {Wie im Original.}
		\item \wikiBoldFt  {Wie im Original.}
		\item \wikiItalicFt{Wie im Original.}
		\item \wikiLinkFt  {Wie im Original, aber ohne Link.}
	\end{itemize}

	\item In sonstigem Text (ohne Überschriften):
	\begin{itemize}
		\item \likeLinkFt{Interner Link.}; auch in Überschriften.
			Die Farbe kann mit anderen Textauszeichnungen kombiniert werden.
		\item \likeBibFt{Nummer als Link ins Literaturverzeichnis.}
		\item \CharFt   {Zeichen [in Zeichenketten].}
		\item \DefFt    {Definition.}
		\item \OptFt    {Optionale  Teile von Sprechweisen.}
		\item \ManFt    {Notwendige Teile von Sprechweisen.}
		\item \GloFt    {Erstmalige Selbstreferenz (ohne Link).}
		\item \gloFt               {Selbstreferenz (ohne Link).}
		\item \likePreFt{Prädikat.}
		\iftestFlg
			\item
			\begin{offen}
				Teile, deren Bearbeitung zurückgestellt ist.
			\end{offen}
		\else\fi
	\end{itemize}

\end{itemize}

Fußnoten dienen nur zu weiteren Erläuterungen sowie Verweisen in dieses Dokument und die Literatur.
Daher können sie auch etwas "`lascher"' formuliert sein.
Für das Verständnis des Textes sollten sie nicht nötig sein, es reichen Grundkenntnisse der Mathematik.

\Endchapter

	%%############################################################################%%
%%                                                                            %%
%% Datei:  ASBA-Analyse.tex                                                   %%
%% Inhalt: Kapitel "Analyse"                                                  %%
%%                                                                            %%
%% Copyright (C) 2017  Winfried Teschers                                      %%
%%                                                                            %%
%% This program is free software: you can redistribute it and/or modify       %%
%% it under the terms of the GNU Affero General Public License as published   %%
%% by the Free Software Foundation, either version 3 of the License, or       %%
%% (at your option) any later version.                                        %%
%%                                                                            %%
%% This program is distributed in the hope that it will be useful,            %%
%% but WITHOUT ANY WARRANTY; without even the implied warranty of             %%
%% MERCHANTABILITY or FITNESS FOR A PARTICULAR PURPOSE.  See the              %%
%% GNU Affero General Public License for more details.                        %%
%%                                                                            %%
%% You should have received a copy of the GNU Affero General Public License   %%
%% along with this program.  If not, see <http://www.gnu.org/licenses/>.      %%
%%                                                                            %%
%% Dr. Winfried Teschers                                                      %%
%% Anton-Günther-Straße 26c                                                   %%
%% 91083 Baiersdorf                                                           %%
%% Germany                                                                    %%
%%                                                                            %%
%% e-mail: winfried.teschers@t-online.de                                      %%
%%                                                                            %%
%%############################################################################%%

% !TeX root = ASBA.tex
% !TeX encoding = UTF-8
% !TeX spellcheck = de_DE

\chapter     {Analyse}% ########################################################
\beginchapter{Analyse}
\label   {cha:Analyse}

In der Mathematik gibt es eine unüberschaubare Menge an \Axiomen, \Saetzen, \Beweisen, \Fachbegriffen\ und \Fachgebieten.
Zu den meisten \Fachgebieten\ gibt es noch ungelöste Probleme.

Es fehlt ein System, das einen Überblick bietet und die Möglichkeit, \Beweise\ automatisch zu überprüfen.
Außerdem sollte all dies in üblicher mathematischer Schreibweise ein- und ausgegeben werden können.
In diesem Dokument werden die Grundlagen für das zu entwickelnde Programmsystem \defTxt{\ASBA} (ein Akronym für "`\DefFt{A}xiome, \DefFt{S}ätze, \DefFt{B}eweise und \DefFt{A}uswertungen"') behandelt.

Ein Programmsystem mit ähnlicher Aufgabenstellung findet sich im GitHub Projekt \emph{Hilbert~II} (\cite{bib:HilbertII, bib:qedeq}).
Einige Ideen sind von dort übernommen worden.

\section     {Fragen}% =========================================================
\beginsection{Fragen}
\label   {sec:Fragen}

Einige der Fragen, die in diesem Zusammenhang auftauchen,
werden nun formuliert:
\begin{enumerate}
	%
	\item \label{Frage:Grundlagen} \DefFt{Grundlagen}:
	Was sind die Grundlagen?
	\textZB\ welche \Logik\ und welche \Mengenlehre.
	%
	\item \label{Frage:Basis} \DefFt{Basis}:
	Welche wichtigen \Axiome, \Saetze, \Beweise, \Fachbegriffe\ und \Fachgebiete\ gibt es?
	Welche davon sind Standard?
	%
	\item \label{Frage:Axiome} \defTxt{\Axiome}:
	Welche \Axiome\ werden bei einem \Satz\ oder \Beweis\ vorausgesetzt?
	Allgemein anerkannte oder auch strittige, wie \textzB\ den \emph{\Satz\ vom ausgeschlossenen Dritten} (\emph{tertium non datur}) oder das \emph{Auswahlaxiom}.
	%
	\item \label{Frage:Beweis} \defTxt{\Beweis}:
	Ist ein \Beweis\ fehlerfrei?
	%
	\item \label{Frage:Konstruktion} \DefFt{Konstruktion}:
	Gibt es einen konstruktiven \Beweis?
	%
	\item \label{Frage:Vergleiche} \DefFt{Vergleiche}:
	Welcher \Beweis\ ist besser?
	Nach welchem Kriterium?
	\textZB\ elegant, kurz, einsichtig oder wenige \Axiome.
	Was heißt eigentlich \emph{elegant}?
	%
	\item \label{Frage:Definitionen} \DefFt{Definitionen}:
	Was ist mit einem \Fachbegriff\ jeweils genau gemeint?
	\textZB\ \emph{Stetigkeit}, \emph{Integral} und \emph{Analysis}.
	%
	\item \label{Frage:Abhaengigkeiten} \DefFt{Abhängigkeiten}:
	Wie heißt ein \Fachbegriff\ in einer anderen Sprache?
	Ist wirklich dasselbe gemeint?
	Was ist mit \Fachbegriffen\ in verschiedenen \Fachgebieten?
	%
	\item \label{Frage:Ueberblick} \DefFt{Überblick}:
	Ist ein \Axiom, \Satz, \Beweis\ oder \Fachbegriff\ schon einmal --- \textggf\ abweichend --- definiert, formuliert oder bewiesen worden?
	%
	\item \label{Frage:Darstellung} \defTxt{\Darstellung}:
	Wie kann man einen \Satz\ und den zugehörigen \Beweis\ --- \textggf\ auch spezifisch für ein \Fachgebiet\ --- darstellen?
	%
	\item \label{Frage:Forschung} \DefFt{Forschung}:
	Welche Probleme gibt es noch zu erforschen.
	%
\end{enumerate}

\section     {Eigenschaften}% ==================================================
\beginsection{Eigenschaften}
\label   {sec:Eigenschaften}

\ASBA\ soll ausgehend von den Fragen in \vrefsec{sec:Fragen} entwickelt werden, und die folgenden Eigenschaften haben:
\begin{enumerate}
	%
	\item \label{Eigenschaft:Daten} \DefFt{Daten}:
	\Axiome, \Saetze, \Beweise, \Fachbegriffe\ und \Fachgebiete\ können in formaler Form gespeichert werden --- auch (noch) nicht oder unvollständig bewiesene \Saetze.
	Dabei soll die übliche mathematische Schreibweise verwendet werden können.
	%
	\item \label{Eigenschaft:Definitionen} \DefFt{Definitionen}:
	Es können \Fachbegriffe\ für \Axiome, \Saetze, \Beweise\ und \Fachgebiete\ --- letztere mit eigenen \Axiomen, \Saetzen, \Beweisen, \Fachbegriffen\ und über- oder untergeordneten \Fachgebieten\ --- definiert werden.
	Die Definitionen dürfen an anderer Stelle definierte \Fachbegriffe\ und \Fachgebiete\ verwenden.%
	\footnote{Rekursive Definitionen sollten ebenfalls möglich sein.}
	%
	\item \label{Eigenschaft:Pruefung} \DefFt{Prüfung}:
	Vorhandene \Beweise\ können automatisch geprüft werden.
	%
	\item \label{Eigenschaft:Ausgaben} \DefFt{Ausgaben}:
	Die \Axiome, \Saetze\ und \Beweise\ können in üblicher Schreibweise --- abhängig von Sprache und \Fachgebiet\ --- ausgegeben werden.
	%
	\item \label{Eigenschaft:Auswertungen} \DefFt{Auswertungen}:
	Zusätzlich zur Ausgabe der gespeicherten Daten sind verschiedene Auswertungen möglich, unter anderem für die meisten der unter \vrefsec{sec:Fragen} behandelten Fragen.
	%
	\setcounter{Enumi}{\value{enumi}}% Nummerierung wird fortgesetzt.
\end{enumerate}
%
Damit \ASBA\ nicht umsonst erstellt wird und möglichst breite Verwendung findet, werden noch zwei Punkte angefügt:
\begin{enumerate}
	\setcounter{enumi}{\value{Enumi}}% Nummerierung wird fortgesetzt.
	%
	\item \label{Eigenschaft:Lizenz} \DefFt{Lizenz}:
	Die Software ist \emph{Open Source}.
	%
	\item \label{Eigenschaft:Akzeptanz} \DefFt{Akzeptanz}:
	\ASBA\ wird von Mathematikern akzeptiert und verwendet.
\end{enumerate}
%
\vreftab{tab:Fragen2Eigenschaften} zeigt, wie sich die Eigenschaften zu den Fragen \vrefinsec{sec:Fragen} verhalten.
Mit einem X werden die Spalten einer Zeile markiert, deren zugehörige Eigenschaften zur Beantwortung der entsprechenden Frage beitragen sollen.
Idealerweise sollte die Erfüllung aller angegebenen Eigenschaften alle gestellten Fragen beantworten, was allerdings illusorisch ist.
%
% Abstände für die nächsten drei Tabellen
\newcommand*{\vsL}{\hspace{-1.0cm}}  % für 1-stellige Zahlen
\newcommand*{\vsl}{\hspace{-6pt}\vsL}% für 2-stellige Zahlen
\newcommand*{\vsc}{\hspace{6pt}}     % für die gedrehten Überschriften
%
\begin{table}[H]
	\begin{tabularx}{\linewidth}
		{@{\hspace{.5cm}}rl@{\extracolsep{\fill}}|*{7}{c}@{\hspace{1cm}}|}
		\multicolumn{2}{l|}{\diagbox[height=3.0cm,width=4.5cm]%
			{\TabFt{Frage}\\~}{\\\TabFt{Eigenschaft}}}
		&\rotatebox{90}{%
			\mbox{\vsL\ref{Eigenschaft:Daten}        \vsc Daten        }}
		&\rotatebox{90}{%
			\mbox{\vsL\ref{Eigenschaft:Definitionen} \vsc Definitionen }}
		&\rotatebox{90}{%
			\mbox{\vsL\ref{Eigenschaft:Pruefung}     \vsc Prüfung      }}
		&\rotatebox{90}{%
			\mbox{\vsL\ref{Eigenschaft:Ausgaben}     \vsc Ausgaben     }}
		&\rotatebox{90}{%
			\mbox{\vsL\ref{Eigenschaft:Auswertungen} \vsc Auswertungen }}
		&\rotatebox{90}{%
			\mbox{\vsL\ref{Eigenschaft:Lizenz}       \vsc Lizenz       }}
		&\rotatebox{90}{%
			\mbox{\vsL\ref{Eigenschaft:Akzeptanz}    \vsc Akzeptanz    }}
		\\\hline
		\ref{Frage:Grundlagen}      & Grundlagen
		& X & X & - & X & X & - & - \\
		\ref{Frage:Basis}           & Basis
		& X & X & - & X & X & - & - \\
		\ref{Frage:Axiome}          & \Axiome
		& X & X & - & X & X & - & - \\
		\hdashline[2pt/2pt]
		\ref{Frage:Beweis}          & \Beweis
		& X & - & X & X & - & - & - \\
		\ref{Frage:Konstruktion}    & Konstruktion
		& X & - & - & X & - & - & - \\
		\ref{Frage:Vergleiche}      & Vergleiche
		& X & - & - & - & X & - & - \\
		\hdashline[2pt/2pt]
		\ref{Frage:Definitionen}    & Definitionen
		& X & X & - & X & - & - & - \\
		\ref{Frage:Abhaengigkeiten} & Abhängigkeiten
		& X & - & - & X & - & - & - \\
		\ref{Frage:Ueberblick}      & Überblick
		& X & - & - & - & X & - & - \\
		\hdashline[2pt/2pt]
		\ref{Frage:Darstellung}     & \Darstellung
		& - & X & - & X & - & - & - \\
		\ref{Frage:Forschung}       & Forschung
		& X & - & - & - & X & - & - \\
		\hline
	\end{tabularx}
	\caption{%
		Fragen (\ref{sec:Fragen}) $\to$ Eigenschaften (\ref{sec:Eigenschaften})
	}
	\label{tab:Fragen2Eigenschaften}% Erst nach '\caption'!
\end{table}

\section[Ziele]{\Ziele}% =======================================================
\beginsection  {\Ziele}
\label      {sec:Ziele}

Um die Eigenschaften von \vrefsec{sec:Eigenschaften} zu erreichen, werden für \ASBA\ die folgenden \Ziele%
\footnote{%
	Es sind eigentlich Anforderungen.
	Diese \Bezeichnung\ wird aber schon \vrefincha{cha:Design} verwendet.
}
gesetzt:
\begin{enumerate}
	%
	\item \label{Ziel:Daten} \DefFt{Daten}:
	Die verteilte Datenbank von \ASBA\ enthält möglichst viele wichtige \Axiome, \Saetze, \Beweise, \Fachbegriffe, \Fachgebiete\ und \Ausgabeschemata%
	\footnote{%
		Um den Punkt~\ref{Eigenschaft:Ausgaben} \vrefvonsec{sec:Eigenschaften} erfüllen zu können, werden noch fachgebietsspezifische \Ausgabeschemata\ benötigt, welche die Art der Ausgaben beschreiben.
	}.
	%
	\item \label{Ziel:Form} \DefFt{Form}:
	Die Daten liegen in formaler, geprüfter Form vor.
	%
	\item \label{Ziel:Eingaben} \DefFt{Eingaben}:
	Die Eingabe von Daten erfolgt in einer formalen \Syntax\ unter Verwendung der üblichen mathematischen Schreibweise.
	%
	\item \label{Ziel:Pruefung} \DefFt{Prüfung}:
	\Beweise\ können automatisch geprüft\footnote{%
		An dieser Stelle soll \ASBA\ soll keine \Beweise\ finden --- das ist \Ziel\ von Punkt \ref{Ziel:Beweisunterstuetzung}, sondern nur vorhandene prüfen.
	}
	werden.
	%
	\item \label{Ziel:Ausgaben} \DefFt{Ausgaben}:
	Die Ausgabe kann in einer eindeutigen, formalen \Syntax\ gemäß vorhandener \Ausgabeschemata\ erfolgen.
	%
	\item \label{Ziel:Auswertungen} \DefFt{Auswertungen}:
	Zusätzlich zur Ausgabe der Daten sind verschiedene Auswertungen möglich.
	Insbesondere kann zu jedem \Beweis\ angegeben werden, wie lang er ist und welche \Axiome\ und \Saetze%
	\footnote{%
		\Saetze, die quasi als \Axiome\ verwendet werden.
	}
	er benötigt.
	%
	\item \label{Ziel:Anpassbarkeit} \DefFt{Anpassbarkeit}:
	\Fachbegriffe\ und die \Darstellung\ bei der Ausgabe können mit Hilfe von --- gegebenenfalls unbenannten --- untergeordneten \Fachgebieten\ angepasst werden.
	%
	\item \label{Ziel:Individualitaet} \DefFt{Individualität}:
	\Axiome\ und \Saetze\ können für jeden \Beweis\ individuell vorausgesetzt werden.
	Dabei sind fachgebietsspezifische \Fachbegriffe\ erlaubt.
	%
	\item \label{Ziel:Internet} \DefFt{Internet}:
	Die Daten können auf mehrere Dateien verteilt sein.
	Ein Teil davon --- oder sogar alle --- können im Internet liegen.
	%
	\item \label{Ziel:Kommunikation} \DefFt{Kommunikation}:
	Die Kommunikation mit \ASBA\ kann mit den \Fachbegriffen\ der einzelnen \Fachgebiete\ erfolgen.
	%
	\item \label{Ziel:Zugriff} \DefFt{Zugriff}:
	Der Zugriff auf \ASBA\ kann lokal und über das Internet erfolgen.
	%
	\item \label{Ziel:Unabhaengigkeit} \DefFt{Unabhängigkeit}:
	\ASBA\ kann online und offline arbeiten.
	%
	\item \label{Ziel:Rekursion} \DefFt{Rekursion}:
	Es kann rekursiv über alle verwendeten Dateien --- auch solchen, die im Internet liegen --- ausgewertet werden.
	%
	\item \label{Ziel:Bedienbarkeit} \DefFt{Bedienbarkeit}:
	\ASBA\ ist einfach zu bedienen.
	%
	\item \label{Ziel:Lizenz} \DefFt{Lizenz}:
	Die Software ist \emph{Open Source}.
	%
	\item \label{Ziel:Zwischenspeicher} \DefFt{Zwischenspeicher}:
	Wichtige Auswertungen können an vorhandenen Dateien angehängt oder separat in eigenen Dateien gespeichert werden.
	%
	\item \label{Ziel:Beweisunterstuetzung} \DefFt{Beweisunterstützung}:
	\ASBA\ hilft bei der Erstellung von \Beweisen.
	%
\end{enumerate}
%
Punkt \ref{Ziel:Zwischenspeicher} wurde noch angefügt, damit \ASBA\ effizient arbeiten kann und um die Akzeptanz zu erhöhen.
Um letzteres zu erreichen, dafür ist auch Punkt \ref{Ziel:Beweisunterstuetzung} nützlich.
Es bietet sich ja auch an, die Fähigkeiten, die \ASBA\ mit der Prüfung von Beweisen haben wird, auch auf die Erstellung von Beweisen anzuwenden.
Die Reihenfolge der \Ziele\ stellt noch keine Priorisierung fest.

\vrefDtab{tab:Eigenschaften2Ziele} zeigt wieder, wie sich die Ziele zu den Eigenschaften \vrefinsec{sec:Eigenschaften} verhalten.
Mit einem X werden wieder die Spalten einer Zeile markiert, deren zugehörige Ziele zur Sicherstellung der entsprechenden Eigenschaft beitragen sollen.
Idealerweise sollte durch Erreichen aller aufgestellten Ziele \ASBA\ alle angegebenen Eigenschaften aufweisen, was wahrscheinlich ebenfalls illusorisch ist.
%
\begin{table}[H]
	\begin{tabularx}{\linewidth}
		{@{\hspace{.2cm}}rl@{\extracolsep{\fill}}|*{17}{c}@{\hspace{0.2cm}}|}
		\multicolumn{2}{l|}{\diagbox[height=3.0cm,width=3.6cm]%
			{\TabFt{Eigenschaft}\\~}{\\\\\TabFt{Ziel}}}
		&\rotatebox{90}{%
			\mbox{\vsL\ref{Ziel:Daten}                \vsc Daten              }}
		&\rotatebox{90}{%
			\mbox{\vsL\ref{Ziel:Form}                 \vsc Form               }}
		&\rotatebox{90}{%
			\mbox{\vsL\ref{Ziel:Eingaben}             \vsc Eingaben           }}
		&\rotatebox{90}{%
			\mbox{\vsL\ref{Ziel:Pruefung}             \vsc Prüfung            }}
		&\rotatebox{90}{%
			\mbox{\vsL\ref{Ziel:Ausgaben}             \vsc Ausgaben           }}
		&\rotatebox{90}{%
			\mbox{\vsL\ref{Ziel:Auswertungen}         \vsc Auswertungen       }}
		&\rotatebox{90}{%
			\mbox{\vsL\ref{Ziel:Anpassbarkeit}        \vsc Anpassbarkeit      }}
		&\rotatebox{90}{%
			\mbox{\vsL\ref{Ziel:Individualitaet}      \vsc Individualität     }}
		&\rotatebox{90}{%
			\mbox{\vsL\ref{Ziel:Internet}             \vsc Internet           }}
		&\rotatebox{90}{%
			\mbox{\vsl\ref{Ziel:Kommunikation}        \vsc Kommunikation      }}
		&\rotatebox{90}{%
			\mbox{\vsl\ref{Ziel:Zugriff}              \vsc Zugriff            }}
		&\rotatebox{90}{%
			\mbox{\vsl\ref{Ziel:Unabhaengigkeit}      \vsc Unabhängigkeit     }}
		&\rotatebox{90}{%
			\mbox{\vsl\ref{Ziel:Rekursion}            \vsc Rekursion          }}
		&\rotatebox{90}{%
			\mbox{\vsl\ref{Ziel:Bedienbarkeit}        \vsc Bedienbarkeit      }}
		&\rotatebox{90}{%
			\mbox{\vsl\ref{Ziel:Lizenz}               \vsc Lizenz             }}
		&\rotatebox{90}{%
			\mbox{\vsl\ref{Ziel:Zwischenspeicher}     \vsc Zwischenspeicher   }}
		&\rotatebox{90}{%
			\mbox{\vsl\ref{Ziel:Beweisunterstuetzung} \vsc Beweisunterstützung}}
		\\\hline
		\ref{Eigenschaft:Daten}         & Daten%
		& X & X & X & - & - & - & - & - & - & - & - & - & - & - & - & - & - \\
		\ref{Eigenschaft:Definitionen}  & Definitionen%
		& X & - & X & - & - & - & - & - & - & - & - & - & - & - & - & - & - \\
		\ref{Eigenschaft:Pruefung}      & Prüfung
		& - & - & - & X & - & - & - & - & - & - & - & - & - & - & - & - & - \\
		\hdashline[2pt/2pt]
		\ref{Eigenschaft:Ausgaben}      & Ausgaben%
		& - & - & - & - & X & - & - & - & - & - & - & - & - & - & - & - & - \\
		\ref{Eigenschaft:Auswertungen}  & Auswertungen%
		& - & - & - & - & - & X & - & - & - & - & - & - & - & - & - & - & - \\
		\ref{Eigenschaft:Lizenz}        & Lizenz%
		& - & - & - & - & - & - & - & - & - & - & - & - & - & - & X & - & - \\
		\hdashline[2pt/2pt]
		\ref{Eigenschaft:Akzeptanz}     & Akzeptanz%
		& X & X & X & X & X & X & X & X & X & X & X & X & X & X & X & X & X \\
		\hline
	\end{tabularx}
	\caption{%
		Eigenschaften (\ref{sec:Eigenschaften}) $\to$ Ziele (\ref{sec:Ziele})
	}
	\label{tab:Eigenschaften2Ziele}% Erst nach '\caption'!
\end{table}

\section     {Zusammenfassung}% ================================================
\beginsection{Zusammenfassung}
\label   {sec:Zusammenfassung}

\begin{table}[H]
	\begin{tabularx}{\linewidth}
		{@{\hspace{.2cm}}rl@{\extracolsep{\fill}}|*{17}{c}@{\hspace{0.2cm}}|}
		\multicolumn{2}{l|}{\diagbox[height=3.0cm,width=4.0cm]%
			{\TabFt{Frage}\\~}{\\\\\TabFt{Ziel}}}
		&\rotatebox{90}{%
			\mbox{\vsL\ref{Ziel:Daten}                \vsc Daten              }}
		&\rotatebox{90}{%
			\mbox{\vsL\ref{Ziel:Form}                 \vsc Form               }}
		&\rotatebox{90}{%
			\mbox{\vsL\ref{Ziel:Eingaben}             \vsc Eingaben           }}
		&\rotatebox{90}{%
			\mbox{\vsL\ref{Ziel:Pruefung}             \vsc Prüfung            }}
		&\rotatebox{90}{%
			\mbox{\vsL\ref{Ziel:Ausgaben}             \vsc Ausgaben           }}
		&\rotatebox{90}{%
			\mbox{\vsL\ref{Ziel:Auswertungen}         \vsc Auswertungen       }}
		&\rotatebox{90}{%
			\mbox{\vsL\ref{Ziel:Anpassbarkeit}        \vsc Anpassbarkeit      }}
		&\rotatebox{90}{%
			\mbox{\vsL\ref{Ziel:Individualitaet}      \vsc Individualität     }}
		&\rotatebox{90}{%
			\mbox{\vsL\ref{Ziel:Internet}             \vsc Internet           }}
		&\rotatebox{90}{%
			\mbox{\vsl\ref{Ziel:Kommunikation}        \vsc Kommunikation      }}
		&\rotatebox{90}{%
			\mbox{\vsl\ref{Ziel:Zugriff}              \vsc Zugriff            }}
		&\rotatebox{90}{%
			\mbox{\vsl\ref{Ziel:Unabhaengigkeit}      \vsc Unabhängigkeit     }}
		&\rotatebox{90}{%
			\mbox{\vsl\ref{Ziel:Rekursion}            \vsc Rekursion          }}
		&\rotatebox{90}{%
			\mbox{\vsl\ref{Ziel:Bedienbarkeit}        \vsc Bedienbarkeit      }}
		&\rotatebox{90}{%
			\mbox{\vsl\ref{Ziel:Lizenz}               \vsc Lizenz             }}
		&\rotatebox{90}{%
			\mbox{\vsl\ref{Ziel:Zwischenspeicher}     \vsc Zwischenspeicher   }}
		&\rotatebox{90}{%
			\mbox{\vsl\ref{Ziel:Beweisunterstuetzung} \vsc Beweisunterstützung}}
		\\\hline
		\ref{Frage:Grundlagen}      & Grundlagen%
		& X & X & X & - & X & X & x & - & - & - & - & - & - & - & - & - & - \\
		\ref{Frage:Basis}           & Basis%
		& X & X & X & - & X & X & x & x & - & - & - & - & - & - & - & - & - \\
		\ref{Frage:Axiome}          & \Axiome%
		& X & X & X & - & X & X & x & - & - & - & - & - & - & - & - & - & - \\
		\hdashline[2pt/2pt]
		\ref{Frage:Beweis}          & \Beweis%
		& X & X & X & X & X & - & - & x & - & - & - & - & - & - & - & - & - \\
		\ref{Frage:Konstruktion}    & Konstruktion%
		& X & X & X & - & X & - & - & x & - & - & - & - & - & - & - & - & - \\
		\ref{Frage:Vergleiche}      & Vergleiche%
		& X & X & X & - & - & X & - & x & - & - & - & - & - & - & - & - & - \\
		\hdashline[2pt/2pt]
		\ref{Frage:Definitionen}    & Definitionen%
		& X & X & X & - & X & - & x & - & - & - & - & - & - & - & - & - & - \\
		\ref{Frage:Abhaengigkeiten} & Abhängigkeiten%
		& X & X & X & - & X & - & x & - & - & - & - & - & - & - & - & - & - \\
		\ref{Frage:Ueberblick}      & Überblick%
		& X & X & X & - & - & X & x & - & - & - & - & - & - & - & - & - & - \\
		\hdashline[2pt/2pt]
		\ref{Frage:Darstellung}     & \Darstellung%
		& X & - & X & - & X & - & x & - & - & - & - & - & - & - & - & - & - \\
		\ref{Frage:Forschung}       & Forschung%
		& X & X & X & - & - & X & x & - & - & - & - & - & - & - & - & - & - \\
		\hline
		\multicolumn{19}{l|}{Die nächsten beiden Punkte
			sind Eigenschaften aus \vrefsec{sec:Eigenschaften}:}\\
		\hline
		\ref{Eigenschaft:Lizenz}    & Lizenz%
		& - & - & - & - & - & - & - & - & - & - & - & - & - & - & X & - & - \\
		\ref{Eigenschaft:Akzeptanz} & Akzeptanz%
		& X & X & X & X & X & X & X & X & X & X & X & X & X & X & X & X & X \\
		\hline
	\end{tabularx}
	\caption{Fragen (\ref{sec:Fragen}) $\to$ Ziele (\ref{sec:Ziele})}
	\label{tab:Fragen2Ziele}% Erst nach '\caption'!
\end{table}
%
\vrefDtab{tab:Fragen2Ziele} ist eine Kombination der Tabellen~ \ref{tab:Fragen2Eigenschaften} und~\ref{tab:Eigenschaften2Ziele} und zeigt, wie sich die Ziele \vrefinsec{sec:Ziele} zu den Fragen \vrefinsec{sec:Fragen} verhalten.
Auch in dieser Tabelle werden mit einem X die Spalten einer Zeile markiert, deren zugehörige Ziele für die Beantwortung der entsprechenden Frage nötig sind.
Mit einem kleinen x werden sie markiert, wenn sie zur Beantwortung der Fragen nicht nötig, aber von Interesse sind.
Idealerweise sollte das Erreichen aller aufgestellten Ziele alle gestellten Fragen beantworten, was natürlich auch illusorisch ist.

\clearpage

\section[Die Umgebung von \glsentrytext{ASBA}]{Die Umgebung von \ASBA}%
\beginsection                                 {Die Umgebung von \ASBA}
\label                                        {sec:Umgebung}

\vrefInfig{fig:Umgebung} wird beschrieben, welche Interaktionen \ASBA\ mit der Umgebung hat, \textdh\ welche Ein- und Ausgaben existieren und woher sie kommen \textbzw\ wohin sie gehen.

\begin{figure}[H]
	\setlength\unitlength{1cm}
	\begin{picture}(17.0,9.5)(-8.4,-4.5)
		% Hilfsgitter während der Bildbearbeitung
		%\color{lightgray}
		%\multiput(-8.4,-4.5)(+0.0,1.0){10}{\line(1,0){17.0}}
		%\multiput(-7.9,-4.5)(+1.0,0.0){17}{\line(0,1){ 9.5}}
		\linethickness{1.5pt}
		% Hintergrund (grau) ===============================================
		\color{gray}
		% rechts: externes ASBA mit Pfeilen --------------------------------
		\put(+3.00,+0.50){\framebox(2.40,1.60){\huge\ImageFt{\ASBA}}}
		\put(+3.00,+0.50){\makebox(2.40,1.50)[t]{\ImageFt{externes}}}
		\put(+4.00,+2.12){\vector(-1,+4){0.35}}% <--- externes ASBA
		\put(+4.00,+3.55){\vector(+1,-4){0.36}}% ---> externes ASBA
		\put(+3.81,+2.82){\marker{a}}
		% rechts oben: externe Datenbank mit Pfeilen -----------------------
		\put(+7.30,+3.80){\Datenbank{1.20}{0.40}{0.80}{\small externe}{\small Datenbank}}
		\put(+5.41,+2.10){\vector(+1,+2){0.70}}% <--- externes ASBA
		\put(+6.14,+2.89){\vector(-1,-2){0.72}}% ---> externes ASBA
		\put(+5.60,+2.48){\marker{b}}
		% Verbindung Auswertungen ---> Männchen ----------------------------
		\put(+5.60,-3.30){\vector(-1,0){4.05}}% Auswertungen ---> Männchen
		\put(+3.50,-3.30){\marker{c}}
		% Verbindung Männchen <---> Terminal -------------------------------
		\put(-2.00,-3.00){\vector(+1,0){2.45}}% Männchen <--- Terminal
		\put(+0.40,-3.30){\vector(-1,0){2.40}}% Männchen ---> Terminal
		\put(-0.75,-3.15){\marker{d}}
		% Verbindung Terminal <---> Datei ----------------------------------
		\put(-5.50,-1.50){\vector(+3,-2){1.50}}% Terminal <--- Datei
		\put(-4.01,-2.80){\vector(-3,+2){1.60}}% Terminal ---> Datei
		\put(-5.00,-2.10){\marker{e}}
		% Vordergrund (schwarz) ============================================
		\color{black}
		% rechts oben: Wolke mit Pfeilen -----------------------------------
		\put(+3.40,+4.50){\Wolke{Internet}}
		\put(+2.00,+4.04){\vector(-1,-3){0.82}}% ---> ASBA
		\put(+1.53,+1.53){\vector(+1,+3){0.75}}% <--- ASBA
		\put(+1.55,+2.70){\Marker{1}}
		% links oben: Datenbank mit Pfeilen --------------------------------
		\put(-7.00,+3.50){\Datenbank{1.50}{0.50}{1.00}{\large \ASBA}{\large Datenbank}}
		\put(-5.50,+3.75){\vector(+7,-4){3.95}}% ---> ASBA
		\put(-1.51,+1.10){\vector(-7,+4){4.00}}% <--- ASBA
		\put(-3.70,+2.40){\Marker{2}}
		% links Mitte: Datei mit Pfeilen -----------------------------------
		\put(-7.00,-1.00){\Datei{3.00}{2.00}{\ASBA}{Datei}}
		\put(-5.50,-0.80){\vector(+4,+1){3.95}}% ---> ASBA
		\put(-1.51,-0.10){\vector(-4,-1){4.00}}% <--- ASBA
		\put(-3.70,-0.45){\Marker{3}}
		%links unten: Rechner mit Pfeilen ----------------------------------
		\put(-3.00,-3.10){\Terminal{Terminal}}
		\put(-2.50,-2.48){\vector(+1,+2){0.98}}% ---> ASBA
		\put(-1.09,-0.52){\vector(-1,-2){0.98}}% <--- ASBA
		\put(-2.00,-1.50){\Marker{4}}
		% Mitte unten: Männchen mit Pfeilen --------------------------------
		\put(+1.00,-2.80){\Maennchen}
		\put(+0.85,-2.50){\vector(0,+1){2.00}}% ---> ASBA
		\put(+1.15,-0.51){\vector(0,-1){2.00}}% <--- ASBA
		\put(+0.80,-1.52){\Marker{5}}
		% rechts unten: Papier mit Pfeil -----------------------------------
		\put(+5.60,-4.20){\Papier{+2.00}{+0.30}{\ASBA}{Ausgabe}}
		\put(+1.51,-0.55){\vector(+2,-1){+4.10}}% <--- ASBA
		\put(+3.25,-1.55){\Marker{6}}
		% Mitte: ASBA ------------------------------------------------------
		\linethickness{3pt}
		\put(-1.5,-0.5){\framebox(3.0,2.0){\Huge\ImageFt{\ASBA}}}
	\end{picture}
	\caption{Die Umgebung von \ASBA}
	\label{fig:Umgebung}% Erst nach '\caption'!
\end{figure}

In den \vrefinfig{fig:Umgebung} abgebildeten Datenflüssen (1) bis (6) und (a) bis (e) werden die folgenden Daten übertragen:
\begin{itemize}
	\newcommand*{\vonnach}  [2]{#1 $\rightarrow$ #2}
	\newcommand*{\nachvon}  [2]{\vonnach{#2}{#1}}
	\newcommand*{\hinundher}[2]{#1 $\leftrightarrow$ #2}
	%
	\item[(1)]\label{dat:Internet}
	\begin{description}
		\item[\vonnach{\ASBA}{Internet}]\label{dat:ausInternet}
		Inhalte der Datenbank.
		\item[\nachvon{\ASBA}{Internet}]\label{dat:inInternet}
		Inhalte der externen Datenbank.
	\end{description}
	%
	\item[(2)]\label{dat:Datenbank}
	\begin{description}
		\item[\vonnach{Datenbank}{\ASBA}]\label{dat:ausDatenbank}
		Inhalte der Datenbank und Antworten auf Datenbankanweisungen.
		\item[\nachvon{Datenbank}{\ASBA}]\label{dat:inDatenbank}
		Inhalte der Datei, der externen Datenbank und Datenbankanweisungen.
	\end{description}
	%
	\item[(3)]\label{dat:Datei}
	\begin{description}
		\item[\vonnach{Datei}{\ASBA}]\label{dat:ausDatei}
		Inhalte der Datei.
		\item[\nachvon{Datei}{\ASBA}]\label{dat:inDatei}
		Die Datei wird um zusätzliche Auswertungen ergänzt, \textzB\ ob die \Beweise\ korrekt sind, welche \Axiome\ und \Saetze\ --- auch externe aus dem Internet --- verwendet wurden, Länge des \Beweises\ usw.
	\end{description}
	%
	\item[(4)]\label{dat:Terminal}
	\begin{description}
		\item[\vonnach{Terminal}{\ASBA}]\label{dat:ausTerminal}
		Anweisungen, Daten und Batchprogramme.
		\item[\nachvon{Terminal}{\ASBA}]\label{dat:inTerminal}
		Antworten auf Anweisungen, Auswertungen usw.
	\end{description}
	Außerdem interaktive Ein- und Ausgabe durch einen Anwender, wie in (5) beschrieben.
	%
	\item[(5)]\label{dat:Anwender}
	\begin{description}
		\item[\hinundher{Anwender}{\ASBA}]\label{dat:mitAnwender}
		Interaktive Ein- und Ausgaben durch einen Anwender mit Komponenten von (3), (4) und (6).
		--- Die Kommunikation läuft \textiAlg\ über ein Terminal.
	\end{description}
	%
	\item[(6)]\label{dat:Ausgabe}
	\begin{description}
		\item[\nachvon{Ausgabe}{\ASBA}]\label{dat:inAusgabe}
		Inhalte von Datei und Datenbank in lesbarer Form, \textua\ mit Hilfe von \Ausgabeschemata\ auch mit \Formeln.
		Die Ausgabe kann auch in eine Datei erfolgen,
		\textzB\ im \LaTeX-Format.
	\end{description}
	%
	\item[(a)]\label{dat:extInternet}
	\begin{description}
		\item[\vonnach{Internet}{externes \ASBA}]\label{dat:ausextInternet}
		Inhalte der Datenbank.
		\item[\nachvon{Internet}{externes \ASBA}]\label{dat:inextInternet}
		Inhalte der externen Datenbank.
	\end{description}
	%
	\item[(b)]\label{dat:extDatenbank}
	\begin{description}
		\item[\vonnach{externe Datenbank}{externes \ASBA}]
		\label{dat:ausextDatenbank} Inhalte der externen Datenbank.
		\item[\nachvon{externe Datenbank}{externes \ASBA}]
		\label{dat:inextDatenbank} Inhalte der Datenbank.
	\end{description}
	%
	\item[(c)]\label{dat:AusgabeAnwender}
	\begin{description}
		\item[\vonnach{Ausgabe}{Anwender}]\label{dat:Ausgabe2Anwender}
		Alle Daten der Ausgabe.
	\end{description}
	%
	\item[(d)] \label{dat:AnwenderTerminal}
	\begin{description}
		\item[\hinundher{Anwender}{Terminal}]\label{dat:Anwender22Terminal}
		Interaktive Ein- und Ausgabe durch einen Anwender, wie in (5) beschrieben.
	\end{description}
	%
	\item[(e)] \label{dat:TerminalDatei}
	\begin{description}
		\item[\hinundher{Terminal}{Datei}]\label{dat:Terminal22Datei}
		Erstellen und Bearbeiten der Datei durch einen Anwender.
		--- siehe (d)
	\end{description}
	%
\end{itemize}
Die Datenflüsse (a) bis (e) erfolgen außerhalb von \ASBA\ und werden nicht weiter behandelt.

Die Datenbank und die Datei enthalten im Prinzip die gleichen Daten, wobei sie in der Datei im Textformat in lesbarer Form und in der Datenbank in einem internen Format vorliegen.
Zudem enthält die Datenbank \textiAlg\ sehr viel mehr Daten. Es handelt sich dabei jeweils um die folgenden Daten:
\begin{description}
	\item[\defTxt{\Axiome}]         \label{Daten:Axiom}         \glsBeschreibung{Axiom}
	\item[\defTxt{\Saetze}]         \label{Daten:Satz}          \glsBeschreibung{Satz}
	\item[\defTxt{\Beweise}]        \label{Daten:Beweis}        \glsBeschreibung{Beweis}
	\item[\defTxt{\Fachbegriffe}]   \label{Daten:Fachbegriff}   \glsBeschreibung{Fachbegriff}
	\item[\defTxt{\Fachgebiete}]    \label{Daten:Fachgebiet}    \glsBeschreibung{Fachgebiet}
	\item[\defTxt{\Ausgabeschemata}]\label{Daten:Ausgabeschema} \glsBeschreibung{Ausgabeschema}
	\item[\defTxt{\Auswertungen}]   \label{Daten:Auswertung}    \glsBeschreibung{Auswertung}
\end{description}
Alle Daten können interne und externe Verweise enthalten.

\begin{offen}%%%

\section[Basis von Beweisen]{Basis von \Beweisen}% =============================
\beginsection               {Basis von \Beweisen}
\label                             {sec:BeweisBasis}

Da ein Computerprogramm erstellt werden soll, muss die Grundstruktur des Vorgehens bei \Beweisen\ definiert werden.%
\footnote{\seename~\cite{bib:Kalkuel}}

\begin{description}
	%
	\item[Die \logischeDarstellung] von mathematischen \Aussagen, wozu auch \Axiome\ und \Saetze\ gehören, erfolgt, da es sich immer um \Formeln\ handelt, an besten mit \Symbolfolgen%
	\footnote{%
		Die \interneDarstellung\ der \Symbolfolgen\ kann zur Optimierung von \ASBA\ von der \logischenD\ abweichen.
	},
	\textdh\ Folgen von Zeichen und Symbolen, in denen Zwischenraum --- insbesondere Leerstellen --- nicht zählen.
	Mehrdimensionale \Formeln, wie \textzB\ Matrizen, Baumstrukturen, Funktionsschemata und anderes, können auch als (eindimensionale) Symbolfolgen dargestellt werden.%
	\footnote{%
		\textZB\ könnte man eine 2$\times$2-Matrix
		$\begin{bmatrix} a & b \\ c & d \end{bmatrix}$
		auch darstellen als Folge von Zeilen: \seqqt{$[(a,b),(c,d)]$}, oder noch einfacher: \seqqt{$[a,b;c,d]$}.
		In \ASBA\ wird die \LaTeX-Syntax verwendet.
		\\Damit wird die soeben angegebene Matrix codiert durch \seqqt{\$\textbackslash begin\{bmatrix\}a\&b\textbackslash\textbackslash c\&d\textbackslash end\{bmatrix\}\$}.
	}
	\Beweise\ sind letztendlich nichts anderes, als erlaubte \Transformationen\ dieser \Symbolfolgen.
	%
	\item[\Bausteine] sind Grundelemente, auch \DefFt{Zeichen} oder \DefFt{(Satz-)Buchstaben} genannt, aus denen die Symbolfolgen bestehen dürfen, und müssen definiert werden.
	%
	\item[\Formationsregeln] dienen zur Festlegung, wie man aus den Bausteinen Ausdrücke erzeugen kann, und müssen ebenfalls definiert werden.
	%
	\item[\Saetze] lassen sich als eine \Menge\ von \Formeln, den \Praemissen, wozu auch \Axiome\ und andere \Saetze\ gehören können, einer weiteren \Menge\ von \Formeln\ (\Symbolfolgen), den \Konklusionen, und der Angabe eines \Beweises\ darstellen.
	%
	\item[\Beweise] zu gegebenen \Praemissen\ und \Konklusionen\ lassen sich als \Folge\ von \Transformationen, beginnend mit den \Praemissen\ und endend mit den \Konklusionen, darstellen.
	%
	\item[\Transformationsregeln] definieren, welche \Transformationen\ mit gegebenen \Formelmengen\ zulässig sind.%
	\footnote{\seename~\cite{bib:Rautenberg,bib:Schlussregel,bib:NatuerlichesSchliessen}}
	%
\end{description}

\end{offen}%%%

\Endchapter

	%%############################################################################%%
%%                                                                            %%
%% Datei:  ASBA-Logik.tex                                                     %%
%% Inhalt: Kapitel "Logische Grundlagen"                                      %%
%%                                                                            %%
%% Copyright (C) 2017  Winfried Teschers                                      %%
%%                                                                            %%
%% This program is free software: you can redistribute it and/or modify       %%
%% it under the terms of the GNU Affero General Public License as published   %%
%% by the Free Software Foundation, either version 3 of the License, or       %%
%% (at your option) any later version.                                        %%
%%                                                                            %%
%% This program is distributed in the hope that it will be useful,            %%
%% but WITHOUT ANY WARRANTY; without even the implied warranty of             %%
%% MERCHANTABILITY or FITNESS FOR A PARTICULAR PURPOSE.  See the              %%
%% GNU Affero General Public License for more details.                        %%
%%                                                                            %%
%% You should have received a copy of the GNU Affero General Public License   %%
%% along with this program.  If not, see <http://www.gnu.org/licenses/>.      %%
%%                                                                            %%
%% Dr. Winfried Teschers                                                      %%
%% Anton-Günther-Straße 26c                                                   %%
%% 91083 Baiersdorf                                                           %%
%% Germany                                                                    %%
%%                                                                            %%
%% e-mail: winfried.teschers@t-online.de                                      %%
%%                                                                            %%
%%############################################################################%%

% !TeX root = ASBA.tex
% !TeX encoding = UTF-8
% !TeX spellcheck = de_DE

\chapter     {Logische Grundlagen}% #######################################
\beginchapter{Logische Grundlagen}
\label            {cha:Grundlagen}

\footnoteForNotDefinedItem

\begin{offen}%%%
	Die logischen Grundlagen werden einerseits gebraucht, um die erlaubten \Beweisschritte\vrefnotesec{sub:Beweisschritte} zu definieren, andererseits dienen sie auch zum Testen von \ASBA.
	Daher werden sie in \datcha{cha:Grundlagen} ausführlicher behandelt, als für die Erstellung von \ASBA\ erforderlich ist.
	Alle \hier\ aufgeführten \Axiome, \Saetze\ und \Beweise\ sollen dazu kodiert und die \Beweise\ dann von \ASBA\ verifiziert werden.
\end{offen}%%%

Speziell in \datcha{cha:Grundlagen} wollen wir mit möglichst exakt definierten \defGlo{\Begriffen} und den zugehörigen einheitlichen, systematischen \defGlo{\Bezeichnungen} (\textdh\ \defGlo{\Benennungen} und \Symbolen) arbeiten.
Wenn sie \likeHyperTxt{in dieser} Schriftart erscheinen, gibt es eine Definition im Symbolverzeichnis (ab \Pageref{dic:Symbolverzeichnis}) oder Glossar (ab \Pageref{dic:Glossar})%
\footnote{%
	Möglicherweise steht dort statt einer Definition auch nur eine Referenz zur Definition im laufenden Text.
},
und diese Bedeutung ist dann gemeint.
Gleichzeitig ist damit im PDF-Dokument ein Link dorthin verbunden.
An Stellen, wo eine \Benennung\footnote{%
	Für Symbole gilt kann leider nur die Farbe, nicht die Schriftart geändert werden.%
} definiert wird, wird sie \likeHyperDef{in dieser} Schriftart ausgegeben.
Wenn die \Benennung\ mit der Fußnote "`\footnotemark[0]"' versehen ist, steht die vollständige Definition nur im Glossar und nicht im laufenden Text.
Eine vertiefende Beschreibung im Glossar oder Symbolverzeichnis ist unabhängig davon immer möglich.

\begin{center}
	\begin{tabular}{|l|c|c|}
		\hline
		Die Sache an sich:  & \multicolumn{2}{c|}{\Begriff}     \\
		\hline
		Darstellung:        & \multicolumn{2}{c|}{\Bezeichnung} \\
		Darstellungsmittel: & \Benennung      & \Symbol         \\
		\hline
	\end{tabular}
\end{center}

Wenn im Text "`wir"' verwendet wird, geht es um Definitionen, die von allgemein bekannten möglicherweise abweichen.\footnote{%
	"`Wir"' und nicht "`ich"', da der Leser eingeschlossen werden soll und in Zukunft möglicherweise auch andere Autoren an diesem Dokument beteiligt sein werden.
}
Die Verwendung von "`wir"' ist allerdings möglicherweise nicht konsistent und soll nur als Hinweis dienen.

\section     {Metasprache}% ====================================================
\beginsection{Metasprache}
\label   {sec:Metasprache}

Wenn man über eine Sprache, die sogenannte \Objektsprache, spricht, braucht man eine zweite Sprache, die sogenannte \Metasprache, in der \Aussagen\ über erstere getroffen werden können.%
\footnote{%
	Die beiden Sprachen können auch übereinstimmen, \textzB\ wenn man über die natürliche Sprache spricht.
}
Wenn die \Objektsprache\ die der Mathematik ist, wählt man üblicherweise die natürliche Sprache als \Metasprache.
Leider ist diese oft ungenau, nicht immer eindeutig und abhängig vom Zusammenhang, in dem sie gesprochen wird.%
\footnote{%
	Man betrachte die beiden formal verschiedenen \Aussagen\ \statement{Studenten und Rentner zahlen die Hälfte.} und \statement{Studenten oder Rentner zahlen die Hälfte.}, die beide das gleiche meinen.
	--- Entnommen aus \cite{bib:Rautenberg} \sectionname~1.2 Bemerkung 1.
}
Um diese Probleme in den Griff zu bekommen, kann die \Metasprache\ auch formalisiert werden.
Durch diese Formalisierung erinnert sie dann schon an mathematische \Formeln.
Die \defGlo{\Sprachebenen} sollten aber sorgfältig unterschieden werden.

\subsection[Sprachebenen]{\Sprachebenen}% --------------------------------------
\label {sub:Sprachebenen}

\begin{description}
	\item[\defTxt{\Metasprache}] \Hier\ die obere \Sprachebene:
	\glsBeschreibung{Metasprache}
	Ihre \Syntax\ und \Semantik\ wird \hier\ nicht behandelt.

	\item[\defTxt{\formaleMetasprache}] \Hier\ die mittlere \Sprachebene:
	\glsBeschreibung{formaleMetasprache}
	Ihre \Syntax\ und \Semantik\ werden im Folgenden noch entwickelt.

	\item[\defTxt{\Objektsprache}] \Hier\ die untere \Sprachebene:
	\glsBeschreibung{Objektsprache}

	Die entsprechende \Syntax\ wird im Folgenden noch entwickelt.
	Die \Semantik\ kann bis zu einem gewissen Grad offen bleiben, um so auch Raum für alternative \Logiken\ zu lassen.
\end{description}

\subsection[Aussagen]{\Aussagen}% ----------------------------------------------
\label {sub:Aussagen}

Wir definieren zunächst noch einige \Begriffe.
\begin{description}
	\item[\defTxt{\Wahrheitswert}] \glsBeschreibung{Wahrheitswert}
	\item[\defTxt{\Aussage}]       \glsBeschreibungMitWiki{Aussage}
\end{description}
Beispiele für \Aussagen\ in \Metasprache\ sind
\begin{enumerate}
	\item[(a)] \label{Bsp:a} \statement{Morgen scheint die Sonne.}
	\item[(b)] \label{Bsp:b} \statement{Ich bin 1,83\,m groß.}
	\item[(c)] \label{Bsp:c} \statement{Ich habe ein rotes Auto und das kann 200\,km/h schnell fahren.}
	\item[(d)] \label{Bsp:d} \statement{Alle Iren haben rote Haare.}
\end{enumerate}
Wie (c) zeigt, kann eine \Aussage\ auch aus anderen \Aussagen\ zusammengesetzt sein.
Wir definieren daher:
\begin{description}
	\item[\defTxt{\Teilaussage}]       \glsBeschreibung{Teilaussage}
	\item[\defTxt{\echteTeilaussage}]  \glsBeschreibung{echteTeilaussage}
%%%	\item[\defTxt{\Oberaussage}]       \glsBeschreibung{Oberaussage}
%%%	\item[\defTxt{\echteOberaussage}]  \glsBeschreibung{echteOberaussage}
	\item[\defTxt{\atomareAussage}]    \glsBeschreibung{atomareAussage}
	\item[\defTxt{\zerlegbareAussage}] \glsBeschreibung{zerlegbareAussage}
\end{description}
Während (a) und (b) \atomare\ \Aussagen\ sind, ist (c) \zerlegbarA.
Für alle vier \Aussagen\ ist es sinnvoll zu fragen, ob sie gelten oder nicht;
für (a) allerdings nur im Nachhinein und für den zweiten Teil von (c) nur weil klar ist, worauf sich "`das"' bezieht.
Offensichtlich muss manchmal der Zusammenhang, in dem eine \Aussage\ formuliert wird, bekannt sein.
\textZB\ ist die Bedeutung von "`Ich"' nur dann bekannt, wenn man weiss, von wem die \Aussage\ ist.

In (b) kann man "`Ich"' durch irgend eine andere Person $X$ ersetzten\footnote{Dass dann auch "`bin"' durch "`ist"' ersetzt werden muss, ist von untergeordneter Bedeutung.} und (d) kann umformuliert werden.
Es ergeben sich dann die \Aussagen
\begin{enumerate}
	\item[(e)] \label{Bsp:e} \statement{$X$ ist 1,83\,m groß}
	\item[(f)] \label{Bsp:f} \statement{Für alle $X$ gilt: Wenn $X$ ein Ire ist, dann hat $X$ rote Haare.}
\end{enumerate}
Den \Wahrheitswert\ von (e) kann man erst dann bestimmen, wenn der \Wert\ der \Variablen\ $X$ bekannt ist, während bei (f) alle zulässigen Werte für $X$ im Prinzip schon bekannt sind.
Man sagt, dass die \Variable $X$ in (e) \freiV\ und in (f) \gebundenV\ vorkommt.
Die \freienVariablen\ einer \Aussage\ nennt man auch ihre \Parameter\ und eine \Aussage\ mit mindestens einem \Parameter\ eine \parametrisierteAussage.

Die \Parameter\ einer \Aussage\ dürfen, soweit nicht anderweitig eingeschränkt, durch jedes zulässige \DefFt{\Objekt}\footnote{\glsBeschreibung{Objekt}} ersetzt werden.
Denkbar sind \Symbole, \Formeln\ und \Aussagen\ sowie \Mengen, \Symbolfolgen\ und Zahlen; ganz allgemein jedes reale oder gedachte Ding an sich.

Wir definieren noch für \Aussagen\ \textbzw\ \Objekte\ $A$ und $B$:%
\footnote{%
	Üblicherweise werden mit den Definitionen neue, \textggf\ parametrisierte, \Begriffe\ und \Symbole\ eingeführt.
	Die Anforderungen an $A$ und $B$ sind intuitiv klar.
	Insbesondere darf $B$ nicht von $A$ abhängig sein.
	Rekursive Definitionen sind allerdings zulässig.
	Man betrachtet dann die gegebenen Definitionen mit \Parametern\ als eine \Menge\ von Definitionen, in denen für bestimmte \Parameter\ alle möglichen \Ersetzungen\ durchgeführt wurden.
	Dann muss diese \Menge\ nur noch in die richtige Reihenfolge gebracht werden können.
}
\begin{description}
	\item[\defTxt{\Eigenschaft}] \glsBeschreibung{Eigenschaft}
	%
	\item[$\defSym{\MtsDefEq}$] \defTxt{\Objektdefinition} \glsBeschreibung{Objektdefinition}%
	\footnote{%
		Nach den Definitionen von \MtsDefEquiv\ und \MtsDefEq\ sind zwei Ausdrücke $P$ und $Q$ schon dann gleich, wenn nach der Ersetzung aller Vorkommen von $A$ durch $B$ sowohl in $P$ als auch in $Q$ die resultierenden Ausdrücke $\overline{P}$ und $\overline{Q}$ gleich sind.
	}
	%
	\item[$\defSym{\MtsDefEquiv}$] \defTxt{\Aussagedefinition} \glsBeschreibung{Aussagedefinition}%
	\footnote{%
		Wenn \Aussagen\ auch \Objekte\ sind, kann \MtsDefEquiv durch \MtsDefEq ersetzt werden.
	}
\end{description}

\subsection[Bereiche]{\Bereiche}% ----------------------------------------------
\label {sub:Bereiche}

Wir definieren:
\begin{description}
	\item[\defTxt{\Bereich}%
	,     \defTxt{\Element}]                        \glsBeschreibung{Bereich}
\end{description}
und für \Aussagen\ und \Objekte\ $a$ und \Bereiche\ $A$:
\begin{align}
	a \defSymBin{\MtsIn} A \quad \MtsDefEquiv \quad
	\text{$a$ ist ein \DefFt{\Element\ aus} $A$} \label{def:MtsIn}\formulatoleft
\end{align}
\MtsIn\ ist eine \Relation.
Gemäß~\eqref{def:relback} \Pageref{def:relback} ist \defSymBin{\MtsNi} die \Umkehrrelationen\ zu \MtsIn\ (Sprechweise: \emph{\textdots\ enthält als \Element\ \textdots}.
Schließlich sind \defSymBin{\MtsInN} und \defSymBin{\MtsNiN} gemäß~\eqref{def:relnot} \Pageref{def:relnot} noch die zugehörigen \Negationen.
Diese vier \Relationen\ bezeichnen wir als \defTxt{\Elementrelationen}.

Wir definieren noch für \Bereiche\ $A$ und $B$%
\footnote{%
	In der Literatur wird \chrqt{\MtsSubset} oft in der Bedeutung von \chrqt{\MtsSubsetEq} verwendet.
	Wir verwenden \chrqt{\MtsSubset} jedoch nur, wenn wir explizit \Ungleichheit\ verlangen.
}
\begin{align}
	& A \defSymBin{\MtsSubsetEq} B & \MtsDefEquiv \quad &
	\parbox[t]{10cm}{alle \Elemente\ aus $A$ sind auch \Elemente\ aus $B$
		\newline Sprechweise: $A$ ist ein \defTxt{\Teilbereich} von $B$}
	\label{def:MtsSubeq} \\
	& A \defSymBin{\MtsEq}       B & \MtsDefEquiv \quad &
	A \MtsSubsetEq B \text{ und } B \MtsSubsetEq A
	\label{def:MtsEq}    \\
	& A \defSymBin{\MtsSubset}   B & \MtsDefEquiv \quad &
	\parbox[t]{10cm}{$A$ \MtsSubsetEq $B$ und nicht $A$ \MtsEq $B$
		\newline Sprechweise: $A$ ist \defTxt{\echterTeilbereich} von $B$}
	\label{def:MtsSub}   \formulatoleft
\end{align}

Gemäß~\eqref{def:relback} \Pageref{def:relback} sind \defSymBin{\MtsSupset} und \defSymBin{\MtsSupsetEq} die \Umkehrrelationen\ zu \MtsSubset\ und \MtsSubsetEq\ (Sprechweisen: \emph{\textdots\ ist [\defTxt{\echteOM}] \defTxt{\Obermenge} von \textdots}).
Es gelten entsprechende Gleichungen wie~\eqref{def:releq} und~\eqref{def:relbsp} \Pageref{def:releq}.
Schließlich sind \defSymBin{\MtsSubsetN}, \defSymBin{\MtsSubsetEqN}, \defSymBin{\MtsSupsetN} und \defSymBin{\MtsSupsetEqN} gemäß~\eqref{def:relnot} \Pageref{def:relnot} noch die zugehörigen \Negationen.
Diese acht \Relationen\ bezeichnen wir als \defTxt{\Bereichsrelationen}.

Wir definieren:
\begin{description}
	\item[\defTxt{\Diskursuniversum} \MtsUniversum] \glsBeschreibung{Diskursuniversum}
	\item[\defTxt{\Aussagenbereich}  \MtsAussagen]  \glsBeschreibung{Aussagenbereich}
	\item[\defTxt{\Objektbereich}    \MtsObjekte]   \glsBeschreibung{Objektbereich}
	\item[\defSym{\MtsIN}~] ist der \Bereich\ der \defGlo{\natuerlichenZahlen}  ohne           $0$
	\item[\defSym{\MtsINo}] ist der \Bereich\ der \defGlo{\natuerlichenZahlen} (einschließlich $0$)
\end{description}
Wenn wir von einer \natuerlichenZahl\ sprechen, meinen wir immer ein \Element\ aus \MtsINo.

%TODO Bis hier Korrektur gelesen

\subsection[Metaoperationen]{\Metaoperationen}% --------------------------------
\label  {sub:Metaoperationen}

\Zerlegbare\ \Aussagen\ wie (c) können zum Teil formalisiert werden.
Dies wird mit den folgenden Definitionen erreicht:%
\footnote{%
	Damit es nicht zu Verwechslungen führt, verwenden wir für die metasprachliche Negation nicht das logische Symbol \chrqt{\OjkNot}.
	Wegen \eqref{def:relback} \Pageref{def:relback} ist die Definition von \chrqt{\MtsRep} überflüssig, wird wegen der angegebenen Sprechweise aber dennoch angegeben.
}
\begin{align}
	%
	&    \defSymUna{\MtsNot}   A & \MtsDefEquiv \qquad &
	\text{$A$ \DefFt{gilt nicht}.}
	\\
	%
	& A \defSymBin{\MtsImp}   B & \MtsDefEquiv \qquad &
	\text{\DefFt{Wenn} $A$ gilt \DefFt{dann} gilt auch $B$.}
	\\
	& A \defSymBin{\MtsRep}   B & \MtsDefEquiv \qquad &
	\text{$A$ gilt \DefFt{sofern} $B$ gilt.}
	\\
	& A \defSymBin{\MtsEquiv} B & \MtsDefEquiv \qquad &
	\text{$A$ gilt \DefFt{genau dann wenn} $B$ gilt.}
	\\
	& A \defSymBin{\MtsAnd}   B & \MtsDefEquiv \qquad &
	\text{$A$ \DefFt{und}  $B$.}
	\\
	& A \defSymBin{\MtsOr}    B & \MtsDefEquiv \qquad &
	\text{$A$ \DefFt{oder} $B$.}
	\formulatoleft
\end{align}

Offensichtlich sind das alles ebenfalls \Aussagen, jetzt aber teilweise formalisiert.
(c) lässt sich dann ausdrücken als \statement{\statement{Ich habe ein rotes Auto} \MtsAnd\ \statement{das kann 200\,km/h schnell fahren.}}.
\seqqt{$A \defSymBin{\MtsRep} B$} ist nur eine andere Schreibweise für \seqqt{$B \MtsImp A$}.
-- Ein Symbol für "`nicht"' wird \hier\ nicht gebraucht.

Wir nennen \MtsAnd\ und \MtsOr\ \defTxt{\Metaoperationen} und \MtsImp, \MtsRep\ und \MtsEquiv\ \defTxt{\Metarelationen}%
\footnote{%
	Man könnte \Metaoperationen\ und \Metarelationen\ auch als \DefFt{Metajunktoren} bezeichnen. Zur Unterscheidung von \Operationen\ und \Relationen\ vergleiche aber auch die Fußnote~\ref{def:Junktor} auf \Pageref{def:Junktor}.
}.
Die damit gebildeten \Aussagen\ können natürlich auch geklammert werden, um die Reihenfolge der Auswertung eindeutig zu machen.
Für den Fall fehlender Klammern sind ihre Prioritäten \vrefintab{tab:Prioritaeten} angegeben.

Um Verwechslungen mit den \Junktoren\ zu vermeiden, verwenden wir für die metasprachlichen \Operationen\ "`und"' und "`oder"' die Symbole \chrqt{\MtsAnd} und \chrqt{\MtsOr}.
$A$ und $B$ können als Operanden von \chrqt{\MtsEquiv}, \chrqt{\MtsAnd} und \chrqt{\MtsOr} vertauscht werden, ohne das Ergebnis zu ändern.%
\footnote{%
	\textDh\ die \Operationen\ \chrqt{\MtsEquiv}, \chrqt{\MtsAnd} und \chrqt{\MtsOr} sind \emph{kommutativ}.
}
Wird in einer (Teil"~)\Aussage\ nur eine der \Operationen\ \MtsAnd\ oder \MtsOr\ verwendet, können die Klammern dort weggelassen und die Operationen in beliebiger Reihenfolge ausgewertet werden, wiederum ohne das Ergebnis zu ändern.%
\footnote{%
	\textDh\ die \Operationen\ \MtsAnd\ und \MtsOr\ sind auch \emph{assoziativ}.
	Bei den den logischen \Operationen\ \OjkAnd\ und \OjkOr\ müssen Kommutativität und Assoziativität durch \Axiome\ gefordert werden.
	Die Kommutativität von \MtsEquiv\ kann abgeleitet werden.
}
Zusammengefasst ist die Reihenfolge der \Operationen\ und der Auswertung dort beliebig.

\subsection[Mit Gleichheit verwandte Relationen]{Mit \Gleichheit\ verwandte \Relationen}
\label     {sub:Gleichheit}

\subsubsection[Vergleichbar]{\Vergleichbar}% - - - - - - - - - - - - - - - - - -
\label {subsub:Vergleichbar}

Zwei \Objekte\ $A$ und $B$ sind \defTxt{\vergleichbar}, wenn beide von derselben \Objektart\ sind, \textdh\ wenn \textzB\ jeweils beide Mengen, \Symbolfolgen, Zahlen, \textusw\ sind.
Dabei muss bei \Formeln\ zwischen der \Formel\ an sich und dem Ergebnis der \Formel\ unterschieden werden. Siehe Beispiel (a).

Intuitiv scheint klar zu sein, was damit  gemeint ist.
Wenn aber entschieden werden muss, ob \textzB\ (a) "`1+1"' gleich "`2"' oder (b) "`1+1"' gleich "`1 + 1"' ist, muss man erst entscheiden, von welcher \Objektart\ die beiden zu vergleichenden Ausdrücke sind, \textdh\ \emph{wie} verglichen wird.
Wenn sie als jeweiliges Ergebnis der beiden \Formeln, verglichen werden, dann ist (a) richtig.
Wenn sie als \Formeln, \textdh\ als \Symbolfolgen, verglichen werden, ist (a) falsch.
Wenn die Ausdrücke in (b) als \Symbolfolgen\ verglichen werden, dann ist (b) richtig.
Wenn sie als \Zeichenketten\ verglichen werden, ist (b) falsch.

Die folgende Tabelle fasst dass zusammen:

\begin{center}
	\begin{tabular}{|c|c|c|c|}
		\hline
		$        A $  &        $B$        & \Objektart\    & $A$ gleich $B$ \\
		\hline
		$       1+1$  &        $2$        & \Objekt       & richtig \\
		\seqqt{$1+1$} & \seqqt{$2$}       & \Formel       & falsch  \\
		\seqqt{$1+1$} & \seqqt{$1\;+\;1$} & \Symbolfolge & richtig \\
		\strqt{1+1}   & \strqt{1 + 1}     & \Zeichenkette & falsch  \\
		\hline
	\end{tabular}
\end{center}

\subsubsection{Vergleiche}%- - - - - - - - - - - - - - - - - - - - - - - - - - -
\label {subsub:Vergleiche}

$A$ und $B$ seien \Objekte.
Dann definieren wir:

\begin{description}
	%
	\item[$\defSym{\MtsEq}$] \defTxt{\Gleichheit} \label{def:Gleichheit}
	\seqqt{$A \MtsEq B$} heißt, dass $A$ und $B$ in den \interessierendenEigenschaften\ für \MtsEq\ übereinstimmen.%
	\footnote{%
		\textZB\ sind zwei \Junktoren\ üblicherweise dann gleich, wenn sie stets denselben \emph{\Wahrheitswert} liefern.
		Ihre \Bezeichnungen\ können dabei durchaus verschieden sein, interessieren bei der Feststellung der \Gleichheit\ aber nicht.
		\textZB\ bezeichnen \chrqt{\MtsAnd} und \chrqt{\MtsUnd} dieselbe \Operation, haben aber verschiedene Priorität. --- \vrefseetab{tab:Prioritaeten}
	}
	Sprechweisen: \standsfor{$A$ ist \emph{dasselbe} wie $B$} oder \standsfor{$A$ ist \emph{identisch} zu $B$}
	--- Inwieweit die \Begriffe\ \emph{Gleichheit} und \emph{Identität} korrelieren, wird \hier\ nicht erörtert.\citenote{bib:Identitaet}
	%
	\item[$\defSym{\MtsEqN}$] \defTxt{\Ungleichheit} \label{def:Ungleichheit}
	\seqqt{$A \MtsEqN B$} heißt, dass $A$ und $B$ in mindestens einer \interessierendenEigenschaft\ für \MtsEq\ nicht übereinstimmen.
	Sprechweisen: \standsfor{$A$ ist \emph{nicht dasselbe} wie $B$} (aber vielleicht das gleiche; siehe \MtsEquiv) oder \standsfor{$A$ ist \emph{nicht identisch} zu $B$}.
	%
%%%	\item[$\defSym{\MtsAequiv}$] \defTxt{\Aequivalenz} \label{def:Aequivalenz}
%%%	\seqqt{$A \MtsAequiv B$} heißt, dass $A$ und $B$ in den \interessierendenEigenschaften\ für \MtsAequiv\ übereinstimmen.
%%%	Sprechweisen: \standsfor{$A$ ist \emph{das gleiche} wie $B$} (aber nicht unbedingt dasselbe; siehe \MtsEq) oder \standsfor{$A$ ist \emph{so wie} $B$}.
%%%	--- Es kann auch verschiedene Äquivalenzen geben, für die dann verschiedene \Bezeichnungen\ verwendet werden.
%%%	%
%%%	\item[$\defSym{\MtsAequivN}$] \defTxt{\Kontravalenz} \label{def:Kontravalenz}
%%%	\seqqt{$A \MtsAequivN B$} heißt, dass $A$ und $B$ in mindestens einer \interessierendenEigenschaft\ für \MtsAequivN\ nicht übereinstimmen.
%%%	Sprechweisen: \standsfor{$A$ ist \emph{nicht das gleiche} wie $B$} oder \standsfor{$A$ ist \emph{nicht so wie} $B$}.
%%%	%
\end{description}

%%%\MtsEq, \MtsEqN, \MtsAequiv\ und \MtsAequivN\ bezeichnen wir als  \defTxt{\Gleichheitsrelationen}.
%%%\Gleichheit\ und \Aequivalenz\ sind \defTxt{\Aequivalenzrelationen}, \textdh\ sie sind \emph{reflexiv} ($a \sim a$), \emph{transitiv} ($((a \sim b) \MtsAnd (b \sim c)) \MtsImp (a \sim c)$) und \emph{symmetrisch} ($(a \sim b) \MtsImp (b \sim a)$)
\MtsEq und \MtsEqN\ bezeichnen wir als  \defTxt{\Gleichheitsrelationen}.
\Gleichheit\ ist eine \defTxt{\Aequivalenzrelation}, \textdh\ sie ist \emph{reflexiv} ($a \sim a$), \emph{transitiv} ($((a \sim b) \MtsAnd (b \sim c)) \MtsImp (a \sim c)$) und \emph{symmetrisch} ($(a \sim b) \MtsImp (b \sim a)$)
-- jeweils für alle zulässigen Objekte $a$, $b$ und $c$.

%%%Jede \interessierendeEigenschaft\ für \MtsAequiv\ oder eine andere \Aequivalenz\ muss auch eine für \MtsEq\ sein.
%%%Daraus folgt insbesondere, dass mit $(A \MtsEq B)$ auch $(A \MtsAequiv B)$ und mit $(A \MtsAequivN B)$ auch $(A \MtsEqN B)$ gilt.

\subsection[Bezeichnungen]{\Bezeichnungen}% ------------------------------------
\label {sub:Bezeichnungen}

\begin{description}

	% ----- Symbol -------------------------------------------------------------
	%TODO Unterschied einfach und atomar(unzerlegbar), zusammengesetzt und zerlegbar
	\item [\Symbole] umfassen neben speziellen \Symbolen\ auch Buchstaben, Ziffern und Sonderzeichen.
	\Symbole, für die es kein eigenes typographisches Zeichen gibt, können auch durch Aufeinanderfolge mehrerer typographischer Zeichen, \textiAlg\ lateinische Buchstaben, dargestellt werden.
	Wir nennen sie dann \DefFt{zusammengesetzte Symbole}, im Gegensatz zu den \DefFt{einfachen Symbolen}.
	Charakteristisch für ein Symbol ist, dass es ohne Bedeutungsverlust nicht zerlegt werden kann.
	Ein \zusammengesetztesSymbol\ ist \textiAlg\ \zerlegbar, kann aber auch als \atomar, \textdh\ \unzerlegbar, definiert werden, wie \textzB\ $\sin$ als \Symbol\ für die Sinusfunktion.
	\Symbole\ werden \chrqt{so} quotiert; \zerlegbare\ können aber auch wie \Symbolfolgen\ quotiert werden.
	--- Die Quotierung ist kein Bestandteil des \Symbols!

	Wird für bestimmte \Objekte\ ein \Symbol\ verwendet, so nennen wir dies ein \defTxt{\Objektsymbol}.
	Ist das Objekt eine Funktion, Operation, Relation \textusw, so nennen wir das Symbol ein \DefFt{Funktionssymbol}, \DefFt{Operationssymbol}, \DefFt{Relationssymbol} usw.

	% ----- Zeichenkette -------------------------------------------------------
	\item [\Zeichenketten] sind Folgen von einfachen \Symbolen, in denen im Prinzip auch Leerstellen und andere nicht druckbare Zeichen zulässig sind.%
	\footnote{%
		Da beim Ausdruck optisch nicht entschieden werden kann, ob ein Zwischenraum (white space) aus einem Tabulator oder \textevtl\ mehreren Leerzeichen besteht, verwenden wir nur einzelne Leerzeichen als Zwischenraumzeichen und vermeiden Zeilenumbrüche.
	}
	Damit Leerstellen in \Zeichenketten\ leicht bestimmt und sogar gezählt werden können,
	werden \Zeichenketten\ stets \strqt{in dieser} Schriftart und Quotierung dargestellt.
	--- Die Quotierung ist kein Bestandteil der \Zeichenkette!

	% ----- Symbolfolge -------------------------------------------------------
	\item [\Symbolfolgen] sind ähnlich wie \Zeichenketten, außer das sie als Bausteine neben einfachen auch zusammengesetzte, aber \atomare\ \Symbole\ enthalten können und Leerzeichen und andere Zwischenraumzeichen nicht zählen.
	Letztere dienen nur der optischen Trennung der \Symbole\ und der besseren Lesbarkeit.
	\Symbolfolgen\ werden stets \seqqt{in dieser} Quotierung dargestellt.
	--- Die Quotierung ist kein Bestandteil der \Symbolfolge!

	% ----- Formel -------------------------------------------------------------
	\item [\Formeln] \label{def:Formel} sind \hier\ immer nach vorgegebenen Regeln aufgebaute \Symbolfolgen%
	\footnote{%
		Es kann verschiedene Arten von \Formeln\ geben, \textzB\ \aussagenlogischeF, prädikatenlogische und solche, die ein Taschenrechner auswerten kann.
	}.
	Daher werden sie wie \Symbolfolgen\ quotiert.
	--- Die Quotierung ist kein Bestandteil der \Symbolfolge!

	Man kann eine \Formel\ auch dadurch charakterisieren, dass sie ein \Element\ aus einer vorgegebenen \Menge\ \MtsSprache\ von \Symbolfolgen\ ist.%
	\footnote{%
		Die \Formel\ wird dann auch \defTxt{\Wort} der \defTxt{\Sprache} \MtsSprache\ genannt - besonders, wenn die \Elemente\ aus \MtsSprache\ \Zeichenketten\ statt \Symbolfolgen\ sind.
		Wir bleiben der Klarheit willen bei "`\Formel"'.
	}
	Das ist dann so ziemlich die einfachste Regel.

	Wenn eine \Symbolfolge\ nicht korrekt nach den vorgegebenen Regeln aufgebaut ist \textbzw\ kein \Element\ aus der vorgegebenen \Menge\ \MtsSprache\ ist, werden wir sie \emph{nicht} als \Formel\ bezeichnen, auch nicht als "`fehlerhafte Formel"' oder ähnlich.
	Sie ist dann einfach keine \Formel.

	% ----- Objekt -------------------------------------------------------------
	\item [\Objekte] sind \textzB\ \Symbole, \Zeichenketten, \Symbolfolgen\ und \Formeln, oder auch \Aussagen, Mengen, Zahlen, \textusw\ --- ganz allgemein reale oder gedachte Dinge an sich.
	Eine \Formel, die nicht quotiert ist, steht für den Wert dieser \Formel, der dann wieder ein \Objekt\ ist.
	Entsprechend steht ein \Symbol, das nicht quotiert ist, für das dadurch bezeichnete \Objekt.
	\textZB\ bezeichnet das \Symbol\ \chrqt{\MtsIN} die \Menge\ \MtsIN der natürlichen Zahlen ohne 0.

\end{description}

\subsection{Quotierung}% -------------------------------------------------------
\label {sub:Quotierung}

Zur Verdeutlichung der soeben definierten Quotierungen ein Beispiel:\footnote{%
	Was \atomare\ und was \zerlegbare\ \Symbole\ sind, muss jeweils definiert werden, \textbzw\ ergibt sich aus dem Zusammenhang.
}

\begin{tabular}{llll}
	&        $\sin$  & \Objekt
	& die Sinusfunktion
	\\
	& \chrqt{$\sin$} & \Bezeichnung
	& für das \Objekt
	\\
	& \seqqt{$\sin$} & \Symbolfolge\ (\Formel)
	& aus dem zusammengesetzten, \atomaren\ \Symbol\ \chrqt{$\sin$}
	\\
	& \seqqt {$sin$} & \Symbolfolge\ (\Formel)
	& aus den einfachen \Symbolen\ \chrqt{$s$}, \chrqt{$i$} und \chrqt{$n$}
	\\
	& \strqt  {sin}  & \Zeichenkette
	& aus den einfachen \Symbolen\ \chrqt{\CharFt{s}}, \chrqt{\CharFt{i}} und \chrqt{\CharFt{n}}
\end{tabular}

Die \Bezeichnung\ eines \Objekts\ kann auch aus mehreren Symbolen bestehen, \textdh\ einer \Symbolfolge\ oder sogar einer ganzen \Formel; \textzB\ ist die Bezeichnung für das indizierte \Objekt\ $a_i$ gleich \seqqt{$a_i$}.

\subsection[Weitere Bezeichnungen]{Weitere \Bezeichnungen}% --------------------
\label  {sub:weitereBezeichnungen}

\begin{description}

%TODO überarbeiten, Dopplungen meiden, mit Glossar abgleichen; eindeutige Bezeichnungen
	% ----- Folge --------------------------------------------------------------
	\item[\Folge] %TODO Folge beschreiben

	% ----- Tupel --------------------------------------------------------------
	\item [\Tupel] Ein \DefFt{$n$-\Tupel} ist eine endliche Folge $\vec{a} = (a_1, \dots, a_n)$ mit folgenden Eigenschaften:
	\begin{itemize}
		\item $n$, die \DefFt{Länge}, \textdh\ die Anzahl der \DefFt{Komponenten} aus $\vec{a}$, ist eine natürliche Zahl.

		$\defSymUna{\MtsLen} \vec{a} \MtsDefEq \defSym{\MtsLen}(\vec{a}) \MtsDefEq n$
		%
		\item Die $a_i$ für $1 \le i \le n$ sind \Elemente\ meist vorgegebener \Mengen.
		%
		\item $\defSymUna{\MtsSet} \vec{a} \MtsDefEq \defSym{\MtsSet}(\vec{a}) \MtsDefEq$ die \Menge\ aller Komponenten $a_i$ aus $\vec{a}$.
	\end{itemize}
	Für $n=0$ ist $\vec{a} = ()$, das \DefFt{leere \Tupel} oder \DefFt{$0$-\Tupel}.

	Wo immer $\vec{a}$ und $a_i$ mit $i \in \MtsINo$ gemeinsam vorkommen, ist $a_i$ die $i$-te Komponente aus $\vec{a}$.

	% ----- Relation -----------------------------------------------------------
	\item [\Relation] Eine \DefFt{$n$-stellige \Relation}\citenote{bib:RelationMehrstellig} $R$ ist ein (1+$n$)-\Tupel\ $(G,A_1,\dots,A_n$) mit folgenden Eigenschaften:
	\begin{itemize}
		\item $n$, die \DefFt{relationale \Stelligkeit}, ist eine natürliche Zahl.

		$\MtsStelR R \MtsDefEq \MtsStelR(R) \MtsDefEq n$
		%
		\item Die $A_i$ für $1 \le i \le n$ sind Mengen, die \defTxt{\Traegermengen} (carrier) von $R$.

		$\MtsTraeger_i R \MtsDefEq \MtsTraeger_i(R) \MtsDefEq A_i$
		%
		\item $G$, der \defTxt{\Graph} von $R$, ist eine \Teilmenge\ des kartesischen Produkts $A_1 \MtsTimes \dots \MtsTimes A_n$.

		$\MtsGraph R \MtsDefEq \MtsGraph(R) \MtsDefEq G \quad$ (oft einfach mit $R$ bezeichnet)
		%
		\item $R(a_1,\dots,a_n) \MtsDefEquiv (a_1,\dots,a_n) \in G$
	\end{itemize}
	Für $n=0$ ist $G \MtsSubsetEq \{()\}$%
	\footnote{%
		Das kartesische Produkt enthält nur noch das $0$-\Tupel\ $()$.
	},
	\textdh\ $R()$ ist entweder \TxtTrue\ (\MtsTrue) oder \TxtFalse\ (\MtsFalse).
	\\Für $n=1$ ist $G \MtsSubsetEq A_1$, \textdh\ $R$ kann als \Teilmenge\ von $A_1$ aufgefasst werden.
	\\Für $n=2$ heißt die Relation \defTxt{\binaer} und man schreibt \seqqt{$x R y$} statt \seqqt{$R(x,y)$} \textbzw\ \seqqt{$(x,y) \in R$}.

	Ist $R=(G,M,\dots,M)$, so heißt $R$ eine $n$-stellige Relation \DefFt{auf}\alternativi{in} $M$.

	Ist $|G|$ endlich, so nennen wir auch $R$ \DefFt{endlich}.

	% ----- Umkehrrelation -----------------------------------------------------
	\item [\Umkehrrelation] Die \defTxt{\Umkehrrelation} \DefFt{von}\alternativi{für} einer \binaeren\ Relation $(G,A,B)$ ist die Relation $(G',B,A)$ mit $G' \MtsDefEq \MengeDef{(b,a)}{(a,b) \in G}$.
	Üblicherweise wird das zugehörige Relationssymbol gespiegelt.

	% ----- Funktion -----------------------------------------------------------
	\item [\Funktion] Eine \defTxt{$n$-stellige \Funktion}\citenote{bib:FunktionMengentheoretisch} ist ein (1+$n$+1)-\Tupel\ $f = (G,A_1,\dots,A_n,B)$ mit folgenden Eigenschaften:
	\begin{itemize}
		\item $n$, die \defTxt{\Stelligkeit}%
		\footnote{%
			Die Werte der Stelligkeit als Relation und als Funktion sind verschieden, \textdh\ es gilt stets: $\MtsStelR(f) = \MtsStelF(f) + 1$.
		},
		ist eine natürliche Zahl.

		$\MtsStelF f \MtsDefEq \MtsStelF(f) \MtsDefEq n$

		\item $f$ ist eine ($n$+1)-stellige Relation.

		\item Zu jedem $n$-\Tupel\ $\vec{a} = (a_1,\dots,a_n)$ mit $a_i \in A_i$ für $1 \le i \le n$ gibt es genau ein $b \in B$ mit $(a_1,\dots,a_n,b) \in G$, den \defTxt{\Funktionswert} von $\vec{a}$.

		$f\vec{a} \MtsDefEq f a_1 \dots a_n \MtsDefEq f(\vec{a}) \MtsDefEq f(a_1,\dots,a_n) \MtsDefEq b$
		\footnote{%
			$f(a_1,\dots,a_n)$ und $f(a_1,\dots,a_n,b)$ sind wohl zu unterscheiden.
			Ersteres ist ein Funktionsaufruf mit einem Funktionswert, letzteres eine Relation mit einem Wahrheitswert.
		}

		\item $A = A_1 \MtsTimes \dots \MtsTimes A_n$ ist der \defTxt{\Definitionsbereich} (domain) von $f$.

		$\MtsDb f \MtsDefEq \MtsDb(f) \MtsDefEq A_1 \MtsTimes \dots \MtsTimes A_n$

		\item $B$ ist der \defTxt{\Zielbereich} (target) von $f$

		$\MtsZb f \MtsDefEq \MtsZb(f)$
	\end{itemize}
	Für $n = 0$ ist $G = ((),b)$ für ein $b \in B$ und somit $f() = b$. $f$ kann damit auch als Konstante $b$ aufgefasst werden.%
	\footnote{%
		Bei der Schreibweise ohne Klammern steht da statt \seqqt{$f()$} nur noch \seqqt{$f$} und statt \seqqt{$f()=b$}, insgesamt also nur noch \seqqt{$f=b$}.
	}

	Man sagt: $f$ ist eine $n$-stellige \Funktion\ von $A_1 \MtsTimes \dots \MtsTimes A_n$ \DefFt{nach}\alternativi{in} $B$ (Schreibweise: $\FunktionDef{f}{A_1 \MtsTimes \dots \MtsTimes A_n}{B}$) oder, im Fall $n=1$, $f$ ist eine Funktion von $A$ nach $B$ (Schreibweise: \FunktionDef{f}{A}{B}).
	Mit $A \MtsDefEq A_1 \MtsTimes \dots \MtsTimes A_n$ kann für $n > 0$ jede Funktion als $1$-stellig aufgefasst werden.

	% ----- Operation ----------------------------------------------------------
	\item [\Operationen] in oder auf einer \Menge\ $M$ sind $n$-stellige Funktionen $\MtsMn \MtsFktArrow M$.
	Für eine \defTxt{\binaere}, \textdh\ 2-stellige \Operation\ \BspOpB\ schreibt man \textiAlg\ \seqqt{$x \BspOpB y$} statt \seqqt{$\BspOpB(x,y)$}.
	Wenn nicht anders angegeben, sind \Operationen\ stets \binaer.
	0-stellige \Operationen\ können wieder als Konstante aufgefasst werden.

	Um Missverständnisse zu vermeiden, werden wir die \Bezeichnung\ "`Operator"' nicht verwenden.

	% ----- Junktor ------------------------------------------------------------
	\item [\Junktoren] sind \aussagenlogischeRelationen\ und \aOperationen.%
	\footnote{\label{def:Junktor}%
		Ein $n$-stelliger \Junktor\ $J$ sei eine \Operation\ und somit eine \Funktion.
		Wegen $M = \{\MtsTrue,\MtsFalse\}$ kann er auch als eine $n$-stellige \Relation\ $J'$ aufgefasst werden:
		$J' \MtsDefEq \MengeDef{\vec{a} \in \MtsMn}{J(\vec{a}) = \MtsTrue}$.

		~~Umgekehrt kann eine $n$-stellige \aussagenlogischeRelation\ $J'$ mittels:
		$J''(\vec{a}) \MtsDefEq \MtsTrue \text{ für } \vec{a} \in J', \MtsFalse \text{ sonst}$, für $\vec{a} \in \MtsMn$, auch als $n$-stellige Operation aufgefasst werden.

		~~Falls $J(\vec{a})=\MtsTrue$ ist $\vec{a} \in J'$ und somit $J''(\vec{a})=\MtsTrue$.
		Für $J(\vec{a})=\MtsFalse$ ist $\vec{a} \notin J'$ und somit $J''(\vec{a})=\MtsFalse$.
		Also ist $J=J''$ und so können die $n$-stelligen \aussagenlogischenRelationen\ und \Operationen\ einander eineindeutig zugeordnet werden.

		~~Daher sind in der Aussagenlogik \Relationen\ und \Operationen\ nicht von vornherein unterscheidbar.
		Wegen der Verabredungen in \vrefsub{sub:Beispielsymbole} muss für die verwendeten \Junktoren\ daher jeweils wohl definiert sein, ob sie als \Relation\ und \Operation\ zu verstehen sind.
	}
\end{description}

\subsection[Relationen und Operationen]{\Relationen\ und \Operationen}% --------
\label{sub:Beispielsymbole}

Als Beispielsymbol für \unaere\ \Operationen\ wird \chrqt{\defSym{\BspOpU}} und für \binaere\ \Operationen\ \chrqt{\defSym{\BspOpB}} verwendet.
Beispielsymbole für \binaere\ Relationen sind \chrqt{\defSym{\BspRel}} und \chrqt{\defSym{\BspRelEq}}, für ihre \Umkehrrelationen\ \chrqt{\defSym{\BspRelBck}} \textbzw\ \chrqt{\defSym{\BspRelBckEq}} sowie für ihre \DefFt{Negationen} \chrqt{\defSym{\BspRelN}} \textbzw\ \chrqt{\defSym{\BspRelEqN}}.%
\footnote{%
	Die Relationen brauchen keine Ordnungsrelationen sein, auch wenn die angegebenen Symbole dies nahe legen.
	Wenn eine der Relationen \BspRel, \BspRelEq, \BspRelBck\ oder \BspRelBckEq\ definiert ist,
	sind wegen \eqref{def:relback}, \eqref{def:releq} und \eqref{def:relbsp} auch die anderen drei Relationen definiert sowie wegen \eqref{def:relnot} auch \BspRelN, \BspRelEqN, \defSym{\BspRelBckN} und \defSym{\BspRelBckEqN}.
	Der senkrechte Strich bei den Negationen kann auch schräg sein, wie \textzB\ bei \MtsEqN.
}
Wenn nichts anderes gesagt wird, gelte mit diesen Symbolen bei gegebenem \chrqt{\BspRel}%
\footnote{%
	entsprechend mit \chrqt{\BspRelBck}, \chrqt{\BspRelEq}, \chrqt{\BspRelBckEq} und anderen, nicht horizontal symmetrischen \Symbolen.
} stets:
\begin{align}
	& (A \defSymBin{\BspRelBck} B) & \MtsDefEquiv \quad &  (B \BspRel A)
	& \quad \text{, die \defTxt{\Umkehrrelation} von } \BspRel
	\label{def:relback} \\
	& (A \defSymBin{\BspRelN}    B) & \MtsDefEquiv \quad & \MtsNot (A \BspRel B)
	& \quad \text{, die \defTxt{\Negation}       von } \BspRel
	\label{def:relnot}  \formulatoleft
\end{align}
Dabei ist \chrqt{\BspRelBck} ist die waagerechte Spiegelung von \chrqt{\BspRel} und statt des senkrechten kann auch ein schräger Strich genommen werden.

Sei nun \BspRel\ gegeben und  \BspRelBckN\ die \Umkehrrelation\ der \Negation\ von \BspRel.
Dann gilt wegen \vref{def:relback} und \vref{def:relnot}
\[(A \BspRelBckN B) \MtsEquiv (B \BspRelN A) \MtsEquiv \MtsNot (B \BspRel A)\]
Sei nun umgekehrt \BspRelBckN\ die \Negation\ der \Umkehrrelation\ von \BspRel.
Dann gilt wegen \vref{def:relnot} und \vref{def:relback}
\[(A \BspRelBckN B) \MtsEquiv \MtsNot (A \BspRelBck B) \MtsEquiv \MtsNot (B \BspRel A)\]
Also stimmt die \Umkehrrelation\ der \Negation\ mit der \Negation\ der \Umkehrrelation\ überein und wir brauchen keine verschiedenen Symbole dafür.

Je nachdem ob \BspRel\ oder \BspRelEq\ gegeben ist%
\footnote{%
	entsprechend mit \BspRelBck\ oder \BspRelBckEq oder anderen nicht horizontal symmetrischen Paaren von \Symbolen.
}
gelte ferner:
\begin{align}
	& (A \defSymBin{\BspRelEq}   B) & \MtsDefEquiv \quad & ((A \BspRel   B) \MtsOr  (A \MtsEq B))
	\label{def:releq} \\
	& (A \defSymBin{\BspRel}     B) & \MtsDefEquiv \quad & ((A \BspRelEq B) \MtsAnd (A \MtsEqN B))
	\label{def:relbsp}   \formulatoleft\formulatoleft\formulatoleft
\end{align}

Man beachte, dass, wenn man \chrqt{\MtsDefEquiv} durch \chrqt{\MtsEquiv} ersetzt, weder \eqref{def:releq} aus \eqref{def:relbsp} folgt noch umgekehrt.
\eqref{def:releq} und \eqref{def:relbsp} folgen aber dann auseinander, wenn aus \chrqt{\BspRel} die Ungleichheit \textbzw\ aus der Gleichheit \chrqt{\BspRelEq} folgt.
Beispiele dazu sind \vrefintab{tab:Gegenbeispiel} angegeben.
%
\begin{table}[H]
	\centering
	\setlength\extrarowheight{1.5pt}
	\begin{tabularx}{9.7cm}{|@{~\extracolsep{\fill}}c|cccc|l|}
		\hline
		~          & $A,\;       A$ & $A,\;       B$ & $B,\;A$& $B,\;       B$ &
		\\
		\hline
		~\MtsEq    & $A=         A$ &                &        & $B=         B$ &
		\\
		\hline
		~\BspRel   &                & $A\BspRel   B$ &        &                &
		\text{Es gilt \eqref{def:releq}}
		\\
		~\BspRelEq & $A\BspRelEq A$ & $A\BspRelEq B$ &        & $B\BspRelEq B$ &
		\text{und \eqref{def:relbsp}}
		\\
		\hline
		~\BspRel   &                & $A\BspRel   B$ &        & $B\BspRel   B$ &
		\text{Es gilt \eqref{def:releq}}
		\\
		~\BspRelEq & $A\BspRelEq A$ & $A\BspRelEq B$ &        & $B\BspRelEq B$ &
		\text{aber nicht \eqref{def:relbsp}}
		\\
		\hline
		~\BspRel   &                & $A\BspRel   B$ &        &                &
		\text{Es gilt \eqref{def:relbsp}}
		\\
		~\BspRelEq & $A\BspRelEq A$ & $A\BspRelEq B$ &        &                &
		\text{aber nicht \eqref{def:releq}}
		\\
		\hline
	\end{tabularx}
	\caption{Beispiele für \BspRel\ und \BspRelEq}
	\label{tab:Gegenbeispiel}% Erst nach '\caption'!
\end{table}
%
Seien $\RawBspOpRel_1$ und $\RawBspOpRel_2$ \binaere\ \Operationen\ oder \Relationen\ (auch gemischt) und mindestens eins von beiden eine \Relation.
Dann treffen wir folgende Vereinbarung:%
\footnote{%
	wird auch in der Literatur verwendet, \textzB\ \textzB~\cite{bib:Rautenberg}, Notationen Seite~xxi
}
\[ \label{def:OpRel}
	A \RawBspOpRel_1  B \RawBspOpRel_2 C \text{ steht für }
	A \RawBspOpRel_1  B \quad \MtsAnd \quad B \RawBspOpRel_2 C
\]
Ist diese Interpretation nicht gewünscht, so müssen Klammern verwendet werden.

Für den Fall fehlender Klammern sind die Prioritäten \vrefintab{tab:Prioritaeten} angegeben.
Damit wären dann alle Klammern in \datsub{sub:Beispielsymbole} überflüssig.

\subsection{Prioritäten}% ------------------------------------------------------
\label {sub:Prioritaeten}

\vrefDtab{tab:Prioritaeten} listet zur Vermeidung von Klammern die Prioritäten der \hier\ verwendeten \Operationen, \Relationen, \Junktoren\ und \Metadefinitionen\ in absteigender Folge von höherer zu niedrigerer Priorität, \textdh\ von starker zu schwacher Bindung auf.%
\footnote{Priorität 1 ist höher und bindet damit stärker als Priorität 2, usw.}
Das Weglassen redundanter Klammern wird in \datcha{cha:Grundlagen} nicht weiter thematisiert.%
\footnote{%
	Gesetzt den Fall, dass \ASBA\ die \Praemissen\ und \Konklusionen\ eines mathematischen \Satzes\ richtig und die \Beweisschritte, \textzB\ durch fehlerhafte Interpretation einer \Formel, falsch einliest, ansonsten aber richtig arbeitet.
	Dann kann man folgende Fälle unterscheiden:\\
	--- Ein falscher \Satz\ kann dadurch nicht als richtig bewertet werden.\\
	--- Ein richtiger \Satz\ wird wahrscheinlich auch bei eigentlich richtigem \Beweis\ als nicht bewiesen gelten, was natürlich unbefriedigend ist.\\
	--- In äußerst unwahrscheinlichen Fällen kann dabei auch ein eigentlich falscher \Beweis\ in einen richtigen verwandelt werden, was zwar schön ist, aber leider steht in der Dokumentation dann ein falscher \Beweis.\\
	In keinem Fall wird durch diesen Fehler die \Menge\ der richtigen \Saetze\ durch einen falschen \Satz\ "`verunreinigt"'.
}
Zur besseren Verständlichkeit werden aber gelegentlich auch redundante Klammern verwendet, insbesondere wenn Prioritäten unklar oder in der Literatur auch anders definiert sind.
Die Prioritäten der \Junktoren\ wurden aus~\cite{bib:Rautenberg} Kapitel~1.1 Seite~5 entnommen und ergänzt und die der \Metaoperationen\ daran angeglichen.

\begin{table}[p]
	\centering
	\begin{threeparttable}
		\setlength\extrarowheight{3pt}
		\begin{tabularx}{12.5cm}{|@{~~}l|@{\extracolsep{\fill}}l|}
			\hline
			Klammern & $(\quad)$ \quad $\quad$ \chrqt{$\quad$} \quad \seqqt{$\quad$} \quad \strqt{$\quad$} \\
			\hline\hline
			\multicolumn{2}{|c|}{\Operationen\ haben unterschiedliche Priorität.} \\
			\hline
			Unäre \Operationen\ \Tnote{1} \Tnote{2} & $\BspOpU \quad \OjkNot \quad \MtsNot$ \\
			\hline
			Binäre \Bereichsoperationen &
			\begin{tabular}{@{\extracolsep{\fill}}l}
				$ \MtsTimes $ \\
				\hline
				$ \MtsCup $   \\
				\hline
				$ \MtsCap $   \\
			\end{tabular}  \\
			\hline
			Binäre \Operationen\ \Tnote{1} & $ \BspOpB $ \\
			\hline
			Binäre \Junktoren\ \Tnote{2} &
			\begin{tabular}{@{\extracolsep{\fill}}l}
				$ \OjkAnd \quad \OjkNand               $ \\
				\hline
				$ \OjkOr  \quad \OjkXor \quad \OjkNor  $ \\
				\hline
				$ \OjkRep \quad \OjkImp                $ \\
				\hline
				$ \OjkEquiv                            $ \\
			\end{tabular}                                \\
			\hline\hline
			\multicolumn{2}{|c|}{Binäre Relationen haben gleiche Priorität.} \\
			\hline
			Binäre \Elementrelationen \Tnote{3}
			& $ \MtsIn \quad \MtsInN \quad \MtsNi \quad \MtsNiN $ \\
			\hdashline
			Binäre \Bereichsrelationen \Tnote{3}
			& $ \MtsSubset \quad \MtsSubsetN \quad \MtsSubsetEq \quad \MtsSubsetEqN \quad \MtsSupset \quad \MtsSupsetN \quad \MtsSupsetEq \quad \MtsSupsetEqN $ \\
			\hdashline
			Binäre \Relationen\ \Tnote{1}
			& $ \BspRel \quad \BspRelN \quad \BspRelEq \quad \BspRelEqN \quad \BspRelBck \quad \BspRelBckN \quad \BspRelBckEq \quad \BspRelBckEqN $ \\
			\hdashline
			\Gleichheitsrelation\ \Tnote{4}
%%%			& $ \MtsEq \quad \MtsEqN \quad \MtsAequiv \quad \MtsAequivN $ \\
			& $ \MtsEq \quad \MtsEqN $ \\
			\hdashline
			\Ableitungsrelation\  \Tnote{5}
			& $ \MtsDerive $ \\
			\hdashline
			\Ersetzung\ \Tnote{5}
			& $ \MtsSwap \quad \MtsSubst $  \\
			\hline\hline
			\multicolumn{2}{|c|}{Sonstige \binaere\ Verknüpfungen haben unterschiedliche Priorität.} \\
			\hline
			\Objektdefinition\ \Tnote{6} & $ \MtsDefEq $ \\
			\hline
			Binäre \Metaoperationen\ \Tnote{7} \Tnote{8} &
			\begin{tabular}{@{\extracolsep{\fill}}l}
				$ \MtsAnd$ \\
				\hline
				$ \MtsOr $ \\
				\hline
				$ \MtsUnd  $ \\
				\hline
				$ \MtsRep \quad \MtsEquiv \quad \MtsImp $
			\end{tabular}     \\
			\hline
			\Aussagedefinition\ \Tnote{6} & $ \MtsDefEquiv $ \\
			\hline\hline
			\multicolumn{2}{|c|}{Natürliche Sprache} \\
			\hline
			\parbox[][1.1cm][c]{6.3cm}{%
				Innerhalb natürlicher Sprache deren Strukturelemente, \textzB\ Satzzeichen \Tnote{9}%
			}
			& . \quad , \quad ; \quad usw. \\
			\hline
		\end{tabularx}
		\begin{tablenotes}
			\footnotesize
			\item[1] \vrefseesub{sub:Beispielsymbole}
			\item[2] \vrefseetab{tab:Symbole}
			\item[3] \vrefseesub{sub:Bezeichnungen}
			\item[4] \vrefseesubsub{subsub:Vergleiche}
			\item[5] \vrefseesub{sub:Basisregeln}
			\item[6] \vrefseesubsub{bereich}
			\item[7] \vrefseesub{sub:Metaoperationen}
			\item[8] \chrqt{\MtsUnd} wird nur bei den \Schlussregeln\ (\vrefseesub{sub:Schlussregeln}) verwendet.
			\chrqt{\MtsAnd} und \chrqt{\MtsUnd} bezeichnen die gleiche \Operation, haben aber unterschiedliche Priorität.
			\item[9] Innerhalb von \Formeln\ können Satzzeichen eine andere Bedeutung und Priorität haben.
		\end{tablenotes}
	\end{threeparttable}
	\caption{Prioritäten in abnehmender Reihenfolge}
	\label{tab:Prioritaeten}% Erst nach '\caption'!
\end{table}

Für \Operationen\ derselben Priorität wählen wir \hier\ Rechtsklammerung%
\footnote{%
	Die Symbole \unaerer\ \Operationen\ stehen \hier\ stets links \emph{vor} dem Operanden, so dass es für sie nur Rechtsklammerung geben kann.
	Zur Rechtsklammerung bei \binaeren\ Operationen ein Zitat aus~\cite{bib:Rautenberg} Kapitel~1.1 Seite~5:
	"`Diese hat gegenüber Linksklammerung Vorteile bei der Niederschrift von Tautologien in \OjkImp, [...]"'.
	Die meisten Autoren bevorzugen Linksklammerung, was natürlicher erscheint.
	Dann sollte man aber für die Potenz doch noch Rechtsklammerung wählen, sonst ist \seqqt{$ a^{x^y} = (a^x)^y = a^{(x*y)} $} und nicht wie wahrscheinlich erwünscht \seqqt{$a^{(x^y)}$}.
}.

\section[Beweise in ASBA]{\Beweise\ in \ASBA}% ================================0
\beginsection            {\Beweise\ in \ASBA}
\label                {sec:BeweiseASBA}

Die Regeln zur Formulierung und Prüfung der \Beweise\ müssen in \ASBA\ fest codiert werden.
Sie sind quasi die \Axiome\ von \ASBA\ und sollten daher möglichst wenig voraussetzen.
In \ASBA\ wird dazu ein \emph{Genzen-Kalkül}%
\footnote{%
	\citesee{bib:Rautenberg} Kapitel~1.4 und~\cite{bib:Schlussregel,bib:NatuerlichesSchliessen}
} verwendet.
Die Definition von \emph{\Schlussregel} und \emph{\Beweis} ist \hier\ \ASBA-spezifisch, um später eine leichtere Programmierung zu erreichen.
Insbesondere müssen alle abzuspeichernden Mengen endlich sein.
Dies berücksichtigen wir in den Beispielen, fordern zunächst aber nicht notwendig Beschränktheit.
Zuerst brauchen wir aber noch ein paar Definitionen.

\subsection{Definitionen und Verabredungen}% -----------------------------------
\label                  {sub:Verabredungen}

Zu \chrqt{\MtsLen} und \chrqt{\MtsSet} Vergleiche die Definition von \emph{$n$-\Tupel} \vrefinsub{sub:weitereBezeichnungen}.

\begin{align}
	& |M|                          & \MtsDefEq \quad & \text{Kardinalität von } M
	&&\text{, die \DefFt{Anzahl der \Elemente} aus } M
	\label{def:Anzahl}
	\\
	& \defSym{\MtsMn}     & \MtsDefEq \quad & M \MtsTimes \dots \MtsTimes M \quad \text{ , für } n \in \MtsINo
	&&\text{, das \DefFt{kartesische Produkt} aus $n$ Mengen } M
	\label{def:kartesischesProdukt}
	\\
	& \MtsMo                  &    \MtsEq \quad & \{()\}
	&&\text{, wobei $()$ das \DefFt{0-\Tupel} ist}
	\label{def:Mo}
	\\
	& \defSym{\MtsTup}(M) & \MtsDefEq \quad & \MengeDef{\vec{a} \in \MtsMn}{n \in \MtsINo}
	&&\text{, die \Menge\ der \defTxt{\Tupel} \DefFt{über} $M$ (\defTxt{\Tupelmenge})}
	\label{def:Tupelmenge}
	\\
	& \links{(A,B)}                & \MtsDefEq \quad & A
	&& \text{, die \DefFt{linke Seite} eines geordneten Paares.}
	\label{def:links}
	\\
	& \rechts{(A,B)}               & \MtsDefEq \quad & B
	&& \text{, die \DefFt{rechte Seite} eines geordneten Paares.}
	\label{def:rechts}
	\\
	& \defSym{\MtsPot}(M)      & \MtsDefEq \quad & \MengeDef{A}{A \MtsSubsetEq M}
	&&\text{, die \defTxt{\Potenzmenge} der \Menge\ } M
	\label{def:Potenzmenge}
	\\
	& \defSym{\MtsPotf}(M)     & \MtsDefEq \quad & \MengeDef{A \MtsSubsetEq M}{|A| \in \MtsINo}
	&& \text{, die \DefFt{endlichen \Teilmengen} von } M
	\label{def:endlichePotenzmenge}
	\\
	& \defSym{\MtsRel}(M)      & \MtsDefEq \quad & \MengeDef{R}{R \MtsSubsetEq M \MtsTimes M}
	&& \text{, die \Menge\ der \DefFt{\binaeren\ \Relationen\ in} } M
	\label{def:Relationsmenge}
	\\
	& \defSym{\MtsRelf}(M)     & \MtsDefEq \quad & \MengeDef{R \MtsSubsetEq M \MtsTimes M}{|R| \in \MtsINo}
	&& \text{, die \DefFt{endlichen \binaeren\ \Relationen\ in} } M
	\label{def:endlicheRelationsmenge}
	\\
	& \defSym{\MtsDeriveR}     & \MtsDefEq \quad & R
	&& \text{, für Relationen } R \in \MtsRelAllDerive
	\label{def:Ableitung}
\end{align}
Offensichtlich gilt für Mengen $M$ und $N$:
\begin{align}
	& \MtsPotf(M) \MtsSubsetEq \MtsPot          (M)
	& ,          \qquad
	& \MtsRelf(M) \MtsSubsetEq \MtsRel          (M)
	\label{eq:Setf} \\
	& \MtsRel (M) =            \MtsPot (M \MtsTimes M)=\MtsPot (M^2)
	& ,          \qquad
	& \MtsRelf(M) =            \MtsPotf(M \MtsTimes M)=\MtsPotf(M^2)
	\label{eq:relPot} \\
	& \MtsPot (M) \MtsSubset   \MtsPot          (N)
	& \MtsEquiv \qquad
	& \MtsPotf(M) \MtsSubset   \MtsPotf         (N)
	& \MtsEquiv \qquad
	&               M  \MtsSubset                          N
	\label{eq:potPot} \\
	& \MtsRel (M) \MtsSubset   \MtsRel          (N)
	& \MtsEquiv \qquad
	& \MtsRelf(M) \MtsSubset   \MtsRelf         (N)
	& \MtsEquiv \qquad
	&               M  \MtsSubset                          N
	\label{eq:relRel} \\
	&                                 \vec{a}  \in \MtsTup(M^2)
	& \MtsEquiv \qquad  & \MtsSet(\vec{a}) \in \MtsRelf    (M)
	\label{eq:vecrel}
\end{align}

\subsection[Formeln und Ableitungen]{\Formeln\ und \Ableitungen}% --------------
\label             {sub:Ableitungen}

Im Folgenden sei \MtsSprache\ stets eine gegebene \Menge\ von \Formeln, \textzB\ alle korrekten \Formeln\ der \Aussagenlogik\ oder der \Praedikatenlogik.
Für die folgenden Betrachtungen ist aber nur nötig, dass die \Elemente\ aus \MtsSprache\ \Symbolfolgen\ sind.
Die \Teilmengen\ von \MtsSprache\ nennen wir \defTxt{\Formelmengen}.
Es sind genau die \Elemente\ aus \MtsPotSprache.

Bei einem \Beweis\ werden aus einer \Formelmenge\ $\Gamma$ von \Axiomen\ und schon bewiesenen \Formeln\ mittels zulässiger
\footnote{%
	Was \emph{zulässig} heißt, muss im entsprechenden Kontext jeweils definiert sein.
	Üblicherweise sind das bestimmte Ableitungsregeln und Ersetzungen.
}
\Ableitungen\ die \Formeln\ einer \Formelmenge\ $\Delta$ abgeleitet; Schreibweise: \seqqt{$\Gamma \MtsDerive \Delta$}.

Für \Teilmengen\ $\Gamma$ und $\Delta$ von \MtsSprache\ sei also:
\begin{itemize}
	\item $\Gamma \defSymBin{\MtsDerive} \Delta \MtsDefEquiv$ $\Gamma$ \defTxt{\ableitbar} $\Delta$; oder auch $\Gamma$ \defTxt{\beweisbar} $\Delta$.
	%
	\item $\Gamma \defSymBin{\MtsDerive} \Delta$ nennen wir auch eine \defTxt{\Ableitung} \DefFt{in} \MtsSprache.
	Damit ist $(\Gamma,\Delta)$ ein \Element\ aus einer \binaeren\ Relation \MtsDerive\ in \MtsPotSprache, einer sogenannten \defTxt{\Ableitungsrelation}.
	%
	\item Wenn wir von einer Ableitung $\drvft{a}$ sprechen, meinen wir immer ein \Element\ aus einer \Ableitungsrelation, \textdh\ ein geordnetes Paar, \textzB\ $(\Gamma, \Delta) \in \MtsPotSprache \MtsTimes \MtsPotSprache$, dargestellt als $\Gamma \MtsDerive \Delta$.
	%
	\item Um möglicherweise verschiedene \Ableitungsrelationen\ unterscheiden zu können, indizieren wir \chrqt{$\defSym{\MtsDerive}$} \textggf\ mit der zugrundeliegenden \Relation\ R, \textdh\ wir schreiben \chrqt{$\defSym{\MtsDeriveR}$} und sprechen dann von \defTxt{$R$-\ableitbar}, \DefFt{$R$-\beweisbar} und \DefFt{$R$-\Ableitung}.
\end{itemize}
%
Zur Vereinfachung der Darstellung und besseren Lesbarkeit treffen wir noch folgende Vereinbarungen für die beiden Seiten von \seqqt{$\Gamma \MtsDerive \Delta$} (natürlich nur, wenn dies nicht zu Verwechslungen führt):
\begin{itemize}
	\item Eine Aufzählung von \Formelmengen\ und einzelnen \Formeln\ steht für die Vereinigung der \Formelmengen\ mit der \Menge\ der einzeln angegebenen \Formeln.
	\textZB\ steht \seqqt{$\Gamma, \alpha \MtsDerive \beta$} für \seqqt{$(\Gamma \MtsCup \{\alpha\}) \MtsDerive \{\beta\}$}.
	%
	\item Diese Aufzählungen können auch leer sein und stehen dann für die \leereMenge.
	\\\textZB\ steht \seqqt{$\MtsDerive\; \alpha \OjkImp (\beta \OjkImp \alpha)$} für \seqqt{$\MtsEmptyset \MtsDerive \{\alpha \OjkImp (\beta \OjkImp \alpha)\}$}.
	%
	\item Ist die Aufzählung links vom Relationssymbol \chrqt{\MtsDerive} leer, kann auch das Relationssymbol wegfallen.
	Im letzten Beispiel also einfach \seqqt{$\{\alpha \OjkImp (\beta \OjkImp \alpha)\}$}.
	Das entspricht dann einem \defTxt{\Axiom}.
\end{itemize}
%
Im Folgenden halten wir uns bei der Verwendung von Buchstaben so weit wie möglich an folgende Vereinbarungen:%
\footnote{Die letzte Gleichung ergibt sich aus \vreffor{eq:relPot}.}
\begin{align}
	&  \text{griechisch, klein:}       && \alpha, \beta, \gamma, \dots
	&& \text{\Formel}                  && \in \qquad \; \; \MtsSprache
	\\
	&  \text{griechisch, groß:}        && \Gamma, \Delta, \Theta, \dots
	&& \text{\Formelmenge}             && \in \quad \; \MtsPotSprache
	\\
	&  \text{lateinisch, fett, klein:} && \drvft{a}, \drvft{b}, \drvft{c}, \dots
	&& \text{\Ableitung}               && \in \quad \; \MtsAllDerive
	\\
	&  \text{lateinisch, fett, groß:}  && \Drvft{A}, \Drvft{B}, \Drvft{C}, \dots
	&& \text{\Ableitungsrelation}      && \in \MtsPotAllDerive = \MtsRelAllDerive
\end{align}
Damit definieren wir folgende Aussagen:
\begin{align}
	\frac{\; \Drvft{A}  \;}{\; \Drvft{B} \;}
	& \quad \MtsDefEquiv \quad
	\text{ Mit den \Ableitungen\ aus $\Drvft{A}$ lassen sich die aus $\Drvft{B}$ ableiten.}
	\label{def:AB}
	\\
	\frac{\; \vec{\drvft{a}} \;}{\; \vec{\drvft{b}} \;} \qquad
	& \quad \MtsDefEquiv \quad
	\text{ Mit den Komponenten aus $\vec{\drvft{a}}$ lassen sich die aus $\vec{\drvft{b}}$ ableiten.}
	\label{def:ab}
	\\
	\frac{\drvft{a}_1 \MtsUnd \dots \MtsUnd \drvft{a}_n}{\drvft{b}_1 \MtsUnd \dots \MtsUnd \drvft{b}_m}
	& \quad \MtsDefEquiv \quad
	\text{ Mit den \Ableitungen\ $\drvft{a}_i$ lassen sich die $\drvft{b}_j$ ableiten.}
	\label{def:aabb}
\end{align}
wobei in der letzten Definition $1 \le i \le n$ und $1 \le j \le m$ sei und die $\drvft{a}_i$ und die $\drvft{b}_j$ dabei jeweils beliebig permutiert werden können.
\chrqt{\defSym{\MtsUnd}} und Bruchstrich stehen für die \Metaoperationen\ \chrqt{\MtsAnd} und \chrqt{\MtsImp}.%
\footnote{%
	Der Bruchstrich hat die übliche Priorität, \MtsUnd\ die schwächste.
	Man beachte, dass Zähler und Nenner auch leer sein können, \textdh\ $n$ und $m$ gleich $0$ sein dürfen.
	In der Praxis liegen sie bei kleinen Werten, typischerweise 0, 1 oder 2.
}
Wir nennen alle drei Formen \defTxt{\Schlussregeln}%
\footnote{%
	Genau genommen nur um die \Darstellung\ einer Schlussregel.
	Die Exakte Definition erfolgt \vrefinsub{sub:Schlussregeln}.
}.
Die \Elemente\ aus $A$ \textbzw\ die Komponenten $a_i$ nennen wir die \defTxt{\Praemissen} und die \Elemente\ aus $B$ \textbzw\ die Komponenten $b_j$ die \defTxt{\Konklusionen}\synonym{\defTxt{\Folgerungen}} der \Schlussregel.
Offensichtlich gilt:
\begin{align}
	& \frac{a_1 \MtsUnd \dots \MtsUnd a_n}{b_1 \MtsUnd \dots \MtsUnd b_m} \; \MtsEquiv \; \frac{\; \vec{a} \;}{\; \vec{b} \;} \; \MtsEquiv \; \frac{\MtsSet(\vec{a})}{\MtsSet(\vec{b})} \label{eq:AB}
\end{align}
Wir nennen eine \Schlussregel\ auch einen \defTxt{\formalenSatz} und nennen sie \defTxt{\beschraenkt}, wenn sie nur endlich viele \Praemissen\ und \Konklusionen\ hat.
Die \Schlussregeln\ nach \eqref{def:ab} und \eqref{def:aabb} sind per se beschränkt.
Die nach \eqref{def:AB} genau dann, wenn $\Drvft{A}$ und $\Drvft{B}$ endliche Mengen sind, \textdh\ wenn sie \Elemente\ aus ...%TODO Text fehlt

Die Mengen der \Praemissen\ und \Konklusionen\ dürfen auch leer sein.
Dies führt zu den folgenden Spezialfällen:
\begin{itemize}
	\item[] Eine \Schlussregel\ $\frac{A}{\MtsEmptyset}$ ohne \Konklusionen\ ist immer gültig.
	%
	\item[] Ein \Menge\ $B$ von Ableitungen, die als \Axiome\ dienen sollen, kann als \Schlussregel\ $\frac{\MtsEmptyset}{B}$ ohne \Praemissen\ repräsentiert werden.
\end{itemize}

\subsection[Schlussregeln]{\Schlussregeln}% ------------------------------------
\label {sub:Schlussregeln}

Wir betrachten zuerst noch die \Menge\ der \binaeren\ Relationen\vrefnotesub{sub:weitereBezeichnungen} in \MtsPotSprache.
Sei also $R$ eine solche \binaere\ Relation und $A \in R$.
Dann gilt wegen~\eqref{def:links}, \eqref{def:rechts}, \eqref{def:Potenzmenge}, \eqref{def:Relationsmenge} und~\vreffor{def:Ableitung}:
\begin{align}
	&  A \in R \in \MtsRelAllDerive   \\
	&  A = (\links{A},\rechts{A})
	&& \text{und es gilt}
	&& \links{A}, \rechts{A} \MtsSubsetEq \MtsSprache \\
	&  \links{A} \MtsDeriveR \rechts{A}
	&& \text{oder einfach}
	&& \links{A} \MtsDerive  \rechts{A}
	&& \text{ist eine $R$-\Ableitung}                  \\
	&  \links{A} \; \text{$R$-\ableitbar} \; \rechts{A}
	&& \text{oder einfach} \qquad
	&& \links{A} \;\; \text{\ableitbar} \;\; \rechts{A}
	\formulatoleft
\end{align}

Nach diesen Vorbereitungen fassen wir noch mal zusammen:\\
Ein geordnetes Paar $(\MtsPraemisseSet, \MtsKonklusionSet) \in \MtsPotAllDerive^2 = \MtsRelAllDerive^2$ heißt eine
\defTxt{\Schlussregel} \DefFt{für} \MtsSprache, geschrieben $\frac{\MtsPraemisseSet}{\MtsKonklusionSet}$; und es gilt:
\begin{align}
	& \MtsPraemisseSet \in \MtsRelAllDerive
	&& \text{, die \defTxt{\Praemissen}}
	&& \text{, eine \Menge\ von \DefFt{\MtsPraemisseSet-\Ableitungen}.}
	\label{def:ruleRelationPraemissen}
	\\
	& \MtsKonklusionSet   \in \MtsRelAllDerive
	&& \text{, die \defTxt{\Konklusionen}}
	&& \text{, eine \Menge\ von   \DefFt{\MtsKonklusionSet-\Ableitungen}.}
	\label{def:ruleRelationKonklusionen}
	\\
	& \drvft{a} \in \MtsPraemisseSet \quad \MtsImp
	&& \drvft{a} = (\Gamma, \Delta) \; \MtsAnd \; \Gamma, \Delta \in \MtsPotSprache
	&& \text{, Schreibweise: } \Gamma \MtsDerive_{\MtsPraemisseSet} \Delta
	\\
	& \drvft{a} \in \MtsKonklusionSet \quad \MtsImp
	&& \drvft{a} = (\Gamma, \Delta) \; \MtsAnd \; \Gamma, \Delta \in \MtsPotSprache
	&& \text{, Schreibweise: } \Gamma \MtsDerive_{\MtsKonklusionSet} \Delta
	\formulatoleft
\end{align}
mit $\Gamma$ und $\Delta$ jeweils passend.

***** Fehlende Verweise: \Ableitungsmenge, \OjkEqN, \MtsTrue, \MtsDerive, \MtsDeriveR. *****

Die \Schlussregel\ entspricht der \Aussage:
\begin{itemize}
	\item[] \emph{Mit den \Praemissen\ aus \MtsPraemisseSet\ lassen sich alle \Konklusionen\ aus \MtsKonklusionSet\ ableiten}%
	\footnote{mittels noch zu definierender \emph{\zulaessigerTransformationen}}.
\end{itemize}
Die \Schlussregel\ heißt \DefFt{allgemeingueltig}, wenn aus den \Praemissen\ alle \Konklusionen\ abgleitet werden können.
In diesem Fall kann sie zur \zulaessigenTransformation\ von weiteren \Formeln\ dienen.

Die Mengen der \Praemissen\ und \Konklusionen\ sowie die beiden Seiten einer \Ableitung\ dürfen auch leer sein.
Dies führt zu den folgenden semantischen Spezialfällen:
\begin{itemize}
	\item Eine \Ableitung\ $(A,\MtsEmptyset)$ ist trivial allgemeingültig.
	Daher können solche Prämissen und Konklusionen ohne Probleme weggelassen werden.
	%
	\item Ein \Menge\ $B$ von \Formeln, die \Axiome\ sein sollen, kann durch eine \Praemisse\ $(\MtsEmptyset,B)$ repräsentiert werden.
	%
	\item Ein \Menge\ $B$ von \Formeln, die als allgemeingültig zu beweisen sind, kann durch eine \Konklusion\ $(\MtsEmptyset,B)$ repräsentiert werden.
\end{itemize}
%
Wenn eine Schlussregel $\frac{\MtsPraemisseSet}{\MtsKonklusionSet}$ beschränkt ist, sind \MtsPraemisseSet\ und \MtsKonklusionSet\ endliche Mengen und es gibt wegen~\vreffor{eq:vecrel} zwei \Tupel\ $\vec{\MtsPraemisse}, \vec{\MtsKonklusion} \in \MtsTup(\MtsAllDerive)$, so dass gilt:
\footnote{%
	Statt $\ge$ könnte in \eqref{eq:SRTb} auch \MtsEq\ genommen werden.
	Dann müssten die $\MtsPraemisse_n$ und die $\MtsKonklusion_m$ jeweils paarweise verschieden sein, was wir nicht voraussetzen wollen.
}
\begin{align}
	&     \MtsPraemisseSet    & \MtsEq \quad & \MtsSet(\vec{\MtsPraemisse})
	&,\;& \MtsKonklusionSet        & \MtsEq \quad & \MtsSet(\vec{\MtsKonklusion})
	\label{eq:SRTa}          \\
	&     N                       &    \ge \quad & |\MtsPraemisseSet|
	&,\;& M                       &    \ge \quad & |\MtsKonklusionSet|
	&,\;& \text{mit } N, M \in \MtsINo
	\label{eq:SRTb}          \\
	& \vec{\MtsPraemisse}     & \MtsEq \quad & \{\MtsPraemisse_1,\dots,\MtsPraemisse_N \}
	&,\;& \vec{\MtsKonklusion}     & \MtsEq \quad & \{\MtsKonklusion_1,\dots,\MtsKonklusion_M\}
	\label{eq:SRTc}          \\
	&       \MtsPraemisse_n   & \MtsEq \quad & ( \links{\MtsPraemisse}_n, \rechts{\MtsPraemisse}_n )
	&,\;& \MtsKonklusion_m         & \MtsEq \quad & ( \links{\MtsKonklusion}_m, \rechts{\MtsKonklusion}_m )
	&,\;& \text{für } 1 \le n \le N \text{ , } 1 \le m \le M
	\label{eq:SRTd}          \\
	& \links{\MtsPraemisse}_n & \MtsDerive_{\MtsPraemisseSet} \quad & \rechts{\MtsPraemisse}_n
	&,\;& \links{\MtsKonklusion}_m & \MtsDerive_{\MtsKonklusionSet}     \quad & \rechts{\MtsKonklusion}_m
	&,\;& \text{für } 1 \le n \le N \text{ , } 1 \le m \le M
	\label{eq:SRTe}          \formulatoleft
\end{align}
also
\begin{align}
	&  \vec{\MtsPraemisse}  & = \quad & \MengeDef{(\links{\MtsPraemisse}_n,
	\rechts{\MtsPraemisse}_n)}{1 \le n \le N}
	\label {def:Praemissen}
	\\
	&  \vec{\MtsKonklusion} & = \quad & \MengeDef{(\links{\MtsKonklusion}_m,
	\rechts{\MtsKonklusion}_m)}{1 \le m \le M }
	\label {def:Konklusionen} \formulatoleft\formulatoleft
\end{align}
und wir nennen auch das Paar $(\vec{\MtsPraemisse}, \vec{\MtsKonklusion})$ \Schlussregel.
Diese ist per se \beschraenkt\ und ein \Element\ aus $\MtsTup(\MtsAllDerive)^2$.
Nun haben wir alternative Schreibweisen für \beschraenkte\ \Schlussregeln:%
\footnote{%
	Nach \eqref{def:AB}, \eqref{def:ab} und \vreffor{def:aabb} sind die "`Brüche"' \Aussagen, und keine Paare mehr.
	Die Äquivalenz der Aussagen steht schon in \vreffor{eq:AB}
}
\[
	\frac{             \MtsPraemisseSet}{             \MtsKonklusionSet} \; \MtsEquiv \;
	\frac{\MtsSet(\vec{\MtsPraemisse}) }{\MtsSet(\vec{\MtsKonklusion}) } \; \MtsEquiv \;
	\frac{        \vec{\MtsPraemisse}  }{        \vec{\MtsKonklusion}  } \; \MtsEquiv \;
	\frac{
		\links{\MtsPraemisse}_1 \MtsDerive_{\MtsPraemisseSet} \rechts{\MtsPraemisse}_1 \MtsUnd
		\dots \MtsUnd
		\links{\MtsPraemisse}_N \MtsDerive_{\MtsPraemisseSet} \rechts{\MtsPraemisse}_N }{
		\links{\MtsKonklusion}_1     \MtsDerive_{\MtsKonklusionSet}     \rechts{\MtsKonklusion}_1     \MtsUnd
		\dots \MtsUnd
		\links{\MtsKonklusion}_M     \MtsDerive_{\MtsKonklusionSet}     \rechts{\MtsKonklusion}_M
	}
	\quad \text{\DefFt{, \defTxt{\Schlussregel}} oder \defTxt{\formalerSatz}}
	\tagFS \label{def:FS}
\]

\subsection[Beweise]{\Beweise}% ------------------------------------------------
\label {sub:Beweise}

Für einen \defTxt{\Beweis} in \ASBA\ ist stets gegeben:%
\footnote{%
	\ASBA\ selbst kann nur endliche Mengen aBspeichern.
	Für \ASBA muss daher einschränkend $\MtsSchlussregelSet \in \MtsRelf(\MtsRelf(\MtsPotf(\MtsSprache)))$ und $\MtsErgebnisSet \in \MtsRelf(\MtsPotf(\MtsSprache))$ sein.
}
\begin{align}
	& \MtsSprache     &           \quad &
	&& \text{, eine \Menge\ von \Formeln, die zugrundeliegende \defTxt{\Sprache}.}
	\label{def:Sprache}      \\
	& \MtsErsetzungSet   & \MtsSubsetEq \quad & \MengeDef{\MtsErsetzung}{\FunktionDef{\MtsErsetzung}{\MtsSprache}{\MtsSprache}}
	&& \text{, eine \Menge\ von \Funktionen, die \defTxt{\Ersetzungen}.}
	\label{def:Ersetzung} \\
	& \MtsSchlussregelSet & \in       \quad & \MtsRelSchlussregel
	&& \text{, eine \Menge\ von \defTxt{\Schlussregeln}.}
	\label{def:Schlussregel} \\
	& \MtsErgebnisSet        & \in       \quad & \MtsRelAllDerive
	&& \text{, eine \Menge\ von \Ableitungen, die \defTxt{\Ergebnisse}.}
	\label{def:Konklusion} &&
\end{align}
%
Die \emph{\Ersetzungen} sorgen \textzB\ dafür, dass aus einer \allgemeingueltigenFormel\ wie  \seqqt{$\alpha \OjkImp (\beta \OjkImp \alpha)$} \textzB\ die \allgemeingueltigeFormel\ \seqqt{$\gamma \OjkImp (\delta \OjkImp \gamma)$} abgeleitet werden kann.
%
Die \emph{\Schlussregeln} geben erlaubte Schlussfolgerungen aus gegebenen \Elementen\ an und umfassen auch die Prämissen eines \Satzes.
Die \emph{\Ergebnisse} schließlich sind das, was mittels eines \Beweises\ aus den gegebenen Prämissen \MtsSprache, \MtsErsetzungSet\ und \MtsSchlussregelSet\ gefolgert werden soll.

Im Fall von \beschraenkten\ \Schlussregeln\ können statt \MtsSchlussregelSet\ und \MtsErgebnisSet\ auch
\begin{align}
	& \vec{\MtsSchlussregel} & \in \quad & \MtsTup(\MtsTup(\MtsAllDerive)^2)
	&& \text{, ein \Tupel\ aus \defTxt{\Schlussregeln}.}
	\label{def:Schlussregelvector} \\
	& \vec{\MtsErgebnis}        & \in \quad & \quad \; \MtsTup(\MtsAllDerive)
	&& \text{, ein \Tupel\ aus \defTxt{\Ableitungen}, die \defTxt{\Ergebnisse}.}
	\label{def:Konklusionsvector}    \formulatoleft
\end{align}
gegeben sein. Mit
\begin{align}
	& \MtsSchlussregelSet \MtsDefEq \MengeDef{(\MtsSet(\vec{\MtsPraemisse}), \MtsSet(\vec{\MtsKonklusion}))}{(\vec{\MtsPraemisse}, \vec{\MtsKonklusion}) \in \MtsSet(\vec{\MtsSchlussregel})}
	\\
	& \MtsErgebnisSet \MtsDefEq \MtsSet(\vec{\MtsErgebnis})
\end{align}
ergibt sich wegen \eqref{eq:Setf} und \vreffor{eq:vecrel} wieder die erste Form.

\subsection[Beispiel für einen Beweis]{Beispiel für einen \Beweis}% ------------
\label {sub:Beispielbeweis}

\todo{Nacharbeiten}     %TODO *** Nacharbeiten ***

\todo{Hier weitermachen}%TODO *** hier weitermachen ***

Zur Veranschaulichung ein Beispiel:\citenote{bib:HilbertKalkuelModusPonens}
\begin{align}
	& \MtsErsetzung_{\alpha,\beta}(\delta) & \MtsDefEq \quad & \text{das }\delta \text{, bei dem alle Vorkommen von $\alpha$ durch $\beta$ ersetzt wurden} \\
	& \MtsSprache & \MtsDefEq \quad & \text{die \Menge\ aller \Formeln\ der \aussagenlogischenSprache} \\
	& \MtsPraemisse_1    & \MtsDefEq \quad & (A, \{\alpha\}) \\
	& \MtsPraemisse_2    & \MtsDefEq \quad & (B, \{\alpha \OjkImp \beta\}) \\
	& \MtsPraemisse_3    & \MtsDefEq \quad & (A \MtsCup B, \{\beta\}) \\
	& \MtsErsetzungSet   & \MtsDefEq \quad & \{\MtsErsetzung_{\alpha,\delta}, \MtsErsetzung_{\beta,B}, \MtsErsetzung_{\beta,B\OjkImp \delta}, \MtsErsetzung_{\gamma,\delta} \} \\
	& \MtsSchlussregelSet & \MtsDefEq \quad & \dots \\
	&          & \chi_1 \; \MtsDefEq \quad & \alpha \OjkImp (\beta \OjkImp \alpha) \\
	&          & \chi_2 \; \MtsDefEq \quad & (\alpha \OjkImp (\beta \OjkImp \gamma)) \OjkImp ((\alpha \OjkImp \beta) \OjkImp (\alpha \OjkImp \gamma)) \\
	& \MtsAxiomSet          & \MtsDefEq \quad & \{\chi_1, \chi_2\} \\
	& \MtsKonklusionRel     & \MtsDefEq \quad & \dots
	\formulatoleft
\end{align}
%TODO Beispiel vervollständigen

\subsection[Beweisschritte]{\Beweisschritte}% ----------------------------------
\label {sub:Beweisschritte}

%TODO Elimination von Prämissen behandeln!
Ein \Beweis%
\footnote{\citesee{bib:Rautenberg} Kapitel~1.6 und~3.6}
in \ASBA\ besteht aus
\begin{align}
	& \text{einer \Schlussregel} && \frac{\MtsPraemisseSet}{\MtsKonklusionSet}
	\\
	& \text{einer Folge} && \MtsBeweisschrittTup = (\MtsBeweisschritt_1, \MtsBeweisschritt_2, ..., \MtsBeweisschritt_K)
	&& \text{von \emph{\Beweisschritten} } \MtsBeweisschritt_k
	&& \text{, die \defTxt{\Beweisschrittfolge}}
	\label{def:Beweisschrittfolge}
	\\
	& \text{einer Folge} && \MtsTransformationTup = (\MtsTransformation_1, \MtsTransformation_2, ..., \MtsTransformation_K)
	&& \text{von \emph{\Transformationen} } \MtsTransformation_k
	&& \text{, die \defTxt{\Transformationsfolge}}
	\label{def:Transformationsfolge}
\end{align}
Dabei ist $K$ ein \Element\ aus \MtsINo, $0 \le k \le K$, die \defTxt{\Beweisschritte} $\MtsBeweisschritt_k$ sind \Schlussregeln\ und die \Transformationen\ $\MtsTransformation_k$ werden später definiert.
%TODO Link auf Definition der Transformationen fehlt
Wir definieren noch:
\begin{align}
	& \MtsBeweisschrittSet_k & \MtsDefEq \quad & \{\MtsBeweisschritt_1, \dots, \MtsBeweisschritt_k\} & \quad \text{, für~~} 0 \le k \le K
	\label{def:Beweisschrittebis} \\
	& \MtsBeweisschrittSet   & \MtsDefEq \quad & \MtsBeweisschrittSet_K \label{def:Beweisschrittmenge}
	\formulatoleft\formulatoleft\formulatoleft
\end{align}
und nennen \MtsBeweisschrittSet\ die \defTxt{\Beweisschrittmenge} der \Beweisschrittfolge\ \MtsBeweisschrittTup.
Dann ist $\MtsBeweisschrittSet_0=\MtsEmptyset$ und $\MtsBeweisschrittSet_i\MtsSubsetEq\MtsBeweisschrittSet_j\MtsSubsetEq\MtsBeweisschrittSet$ für $0\le i\le j\le K$.
-- Wir nennen die \Beweisschrittfolge\ auch eine \defTxt{\Ableitung} aus \MtsKonklusionSet\ aus \MtsPraemisseSet.

%TODO Rolle der Transformationen erläutern
Jeder \Beweisschritt\ $ \MtsBeweisschritt_k \text{ für } 1 \le k \le K $ muss entweder eine \Praemisse\ aus \MtsPraemisseSet\ oder durch Anwendung einer \allgemeingueltigenSchlussregel\ auf eine \Teilmenge\ von $\MtsBeweisschrittSet_{k-1}$ eine wahre \Formel\ oder eine weitere \allgemeingueltigeSchlussregel\ sein.
Schließlich muss noch
\[ \MtsKonklusionSet \MtsSubsetEq \MtsBeweisschrittSet \]
sein, da jede \Konklusion\ aus \MtsKonklusionSet\ in der Folge \MtsBeweisschrittTup\ vorkommen und somit \Element\ aus der \Menge\ \MtsBeweisschrittSet\ sein muss.

========================================================================
%TODO ==================================================================

\Endchapter

	\begin{offen}%%%
		%%############################################################################%%
%%                                                                            %%
%% Datei:  ASBA-Ideen.tex                                                     %%
%% Inhalt: Kapitel "Ideen" --- Nur vorübergehend ---                          %%
%%                                                                            %%
%% Copyright (C) 2017  Winfried Teschers                                      %%
%%                                                                            %%
%% This program is free software: you can redistribute it and/or modify       %%
%% it under the terms of the GNU Affero General Public License as published   %%
%% by the Free Software Foundation, either version 3 of the License, or       %%
%% (at your option) any later version.                                        %%
%%                                                                            %%
%% This program is distributed in the hope that it will be useful,            %%
%% but WITHOUT ANY WARRANTY; without even the implied warranty of             %%
%% MERCHANTABILITY or FITNESS FOR A PARTICULAR PURPOSE.  See the              %%
%% GNU Affero General Public License for more details.                        %%
%%                                                                            %%
%% You should have received a copy of the GNU Affero General Public License   %%
%% along with this program.  If not, see <http://www.gnu.org/licenses/>.      %%
%%                                                                            %%
%% Dr. Winfried Teschers                                                      %%
%% Anton-Günther-Straße 26c                                                   %%
%% 91083 Baiersdorf                                                           %%
%% Germany                                                                    %%
%%                                                                            %%
%% e-mail: winfried.teschers@t-online.de                                      %%
%%                                                                            %%
%%############################################################################%%

% !TeX root = ASBA.tex
% !TeX encoding = UTF-8
% !TeX spellcheck = de_DE

\chapter{Ideen}% ###############################################################
\beginchapter{Ideen}
\label{cha:Ideen}

\section{Schlussregeln}% =======================================================
\beginsection{Schlussregeln}
\label{sec:Schlussregeln}
\hidden{\glsIdx{Schlussregel}}

In diesem \sectionname\ geht es um \glsIdxPl{zulaessige-Transformation}, \textdh\ \glsIdx{allgemeingueltige-Schlussregel}.
Dazu gehören zunächst die \glsIdxPl{Basisregel}.
Dann aber auch alle aus den \glsIdxPl{Basisregel} und den bis dahin \glsIdxBg{allgemeingueltige-Schlussregel}{allgemeingültigen Schlussregeln} korrekt abgeleiteten neuen \glsIdxPl{Schlussregel}.
Die \glsIdxPl{Schlussregel} haben die Form eines Formalen \glsIdx{Satz}es.

\subsection{Basisregeln}% ------------------------------------------------------
\label{sub:Basisregeln}
\hidden{\glsIdxPl{Basisregel}}

Gemäß \cite{bib:Rautenberg} Kapitel~1.4 \emph{Ein vollständiger aussagenlogischer Kalkül} werden sechs \glsIdxPl{Basisregel} definiert. Zuvor werden aber noch einige Definition gebraucht. Dazu seien $n$, $m$, $k$ und $l$ natürliche Zahlen (auch~0), $\alpha$, $\alpha_i$, $\beta$ und $\beta_j$ \glsIdxPl{Formel}, $X$, $X_i$, $Y$ und $Y_j$ Mengen von \glsIdxBg{Formel}{formalen Elementen} und
\begin{align}
	%
	&X&&\defeq&&X_1\cup X_2\cup...\cup X_n\cup\{\alpha_1,\alpha_2,...,\alpha_m\}
	\\
	&Y&&\defeq&&Y_1\cup Y_2\cup...\cup Y_k\cup\{\beta_1, \beta_2, ...,\beta_l \}
	\formulatoleft\formulatoleft
\end{align}
%
$X$ und $Y$ können auch die leere Menge sein. Damit wird definiert:
\begin{align}
	& \alpha \derive \beta \quad \metadefeq \quad
	\parbox[t]{10.5cm}{%
	$\beta$ ist mittels schrittweiser Anwendung \emph{\glsIdxBg{zulaessige-Transformation}{zulässiger Transformationen}} (siehe weiter unten) aus $\alpha$ ableitbar.
	Sprechweise: Aus $\alpha$ ist $\beta$ \emph{ableitbar} oder \emph{beweisbar};
	kurz: \enquote{$\alpha$ \emph{\glsIdx{ableitbar}} $\beta$} \textbzgl\ \enquote{$\alpha$ \emph{\glsIdx{beweisbar}} $\beta$}
	-- Es kann auch \chrqt{$\alpha$} durch \chrqt{$X$} und/oder \chrqt{$\beta$} durch \chrqt{$Y$} ersetzt werden.
	}
	\label{def:ableitbar}
	\\
	& \derive \beta \quad \metadefeq \quad \emptyset \derive \beta \qquad \text{(\chrqt{$\derivesym$} kann dann auch ganz entfallen)}
	\\
	&             X_1, X_2, ...,X_n, \alpha_1, \alpha_2, ..., \alpha_m \quad
	\derive \quad Y_1, Y_2, ...,Y_n,  \beta_1,  \beta_2,  ..., \beta_m \quad
	\metadefeq \quad X \derive Y
	\label{def:ableitbarKurz}
	\formulatoleft
\end{align}
%
Eine \emph{\glsIdx{zulaessige-Transformation}} ist die Anwendung einer \emph{\glsIdx{Substitution}}{\vrefnotesub{sub:Identitätsregeln} (siehe unten), einer \emph{\glsIdx{Basisregel}} (siehe unten) oder einer davon abgeleiteten sonstigen \emph{\glsIdx{Schlussregel}}, \textzB\ aus \vrefsub{sub:Schlussregeln}.
Bei den \glsIdxPl{Schlussregel} und der \glsIdx{Substitution} ($\subst$) soll das Komma stärker binden als \chrqt{$\derivesym$}, \chrqt{$\subst$} und \chrqt{$\srand$},
 wobei \chrqt{$\srand$} für \enquote{und} \textbzgl\ \chrqt{\glsIdx{metaand}}\vrefnotesub{sub:Aussagen} steht und schwächer bindet als \chrqt{$\derivesym$} und \chrqt{$\subst$}.%
\footnote{siehe Fußnote~3 \vrefvontab{tab:Prio-Aussagenlogik}} %TODO Kommentar zu \srand prüfen

Zur der Auswahl der \glsIdxPl{Basisregel}, der Formulierung und der Bezeichnungen wird auf~\cite{bib:Rautenberg,bib:NatuerlichesSchliessen} zurückgegriffen.
Wie in~\cite{bib:NatuerlichesSchliessen} steht \chrqt{$E$} für \enquote{-Einführung} und \chrqt{$B$} für \enquote{-Beseitigung} (oder \enquote{-Elimination}) von Operatoren.%
\footnote{%
	In der \glsIdx{Monotonieregel} wird hier, anders als in~\cite{bib:Rautenberg}, \seqqt{$X,Y$} statt \seqqt{$ Y \text{ , für } Y \supseteq X $} genommen. Das ist gleichwertig, vermeidet aber den Zusatz \seqqt{$ \text{ , für } Y \supseteq X $}.
	Außerdem werden bei den Bezeichnungen \seqqt{$(\land 1)$} und \seqqt{$(\land 2)$} gemäß~\cite{bib:NatuerlichesSchliessen} durch \seqqt{$\andE$} \textbzw\ \seqqt{$\andB$} ersetzt.
}

Im Folgenden seien $\alpha$ und $\beta$ wieder stets \glsIdxPl{Formel} und $X$ und $Y$ Mengen von \glsIdxBg{Formel}{formalen Elementen}.
Für die sechs \glsIdxPl{Basisregel} werden dann nur noch die logischen Operatoren \chrqt{$\lnot$} und \chrqt{$\land$} benötigt.
Bei den weiteren \glsIdxPl{Schlussregel} wird noch \chrqt{$\limp$} gemäß der Definition~\vref{def:imp} verwendet.
%
\begin{align}
	& \frac{}{\alpha\derive\alpha}
	& & (\text{\glsIdx{Anfangsregel}})
	\tag{\tagAR} \sym{\gls{AR}} \label{def:Anfangsregel}
	\\\\
	& \frac{X\derive\alpha}{X,Y\derive\alpha}
	& & (\text{\glsIdx{Monotonieregel}})
	\tag{\tagMR} \sym{\gls{MR}} \label{def:Monotonieregel}
	\\\\
	& \frac{X\derive\alpha,\lnot\alpha}{X\derive\beta}
	& & (\text{Einführung/Beseitigung der Negation Teil 1})
	\tag{\tagnota} \sym{\gls{nota}} \label{def:nota}
	\\\\
	& \frac{X,\alpha\derive\beta \srand X,\lnot\alpha\derive\beta}{X\derive\beta}
	& & (\text{Einführung/Beseitigung der Negation Teil 2})
	\tag{\tagnotb} \sym{\gls{notb}} \label{def:notb}
	\\\\
	& \frac{X\derive\alpha,\beta}{X\derive\alpha\land\beta}
	& & (\text{Einführung der Konjunktion})
	\tag{\tagandE} \sym{\gls{andE}} \label{def:andE}
	\\\\
	& \frac{X\derive\alpha\land\beta}{X\derive\alpha,\beta}
	& & (\text{Beseitigung der Konjunktion})
	\tag{\tagandB} \sym{\gls{andB}} \label{def:andB}
	\formulatoleft
\end{align}
%
In einer \glsIdx{Schlussregel} werden die \glsIdxBg{Formel}{formalen Elemente}%
\footnote{hier: \glsIdxPl{Aussage} in einer formalen Form.}
über dem Querstrich als \emph{\glsIdxPl{Voraussetzung}} und die unter dem Querstrich als \emph{\glsIdx{Folgerung}} der Regel bezeichnet.
Eine \glsIdx{Schlussregel} steht für die \glsIdx{Aussage}, dass mit ihren \glsIdxPl{Voraussetzung} auch auch ihre \glsIdxPl{Folgerung} gelten.
-- Im Gegensatz zu den weiteren \glsIdxPl{Schlussregel} werden die oben aufgelisteten Basisregeln nicht weiter hinterfragt, \textdh\ sie gelten quasi als \glsIdxPl{Axiom}.

\subsection{Identitätsregeln}% --------------------------------------------------------
\label{sub:Identitätsregeln}

Die zulässigen Transformationen, \textdh\ die Anwendung der \glsIdxPl{Schlussregel}, erfordern zulässige \glsIdxPl{Substitution}.
Damit wird dem Gleichheits- oder Identitätszeichen \chrqt{$\eq$} dann mittels Einführungs- und Beseitigungsregel eine Bedeutung verliehen.%
\footnote{siehe~\cite{bib:NatuerlichesSchliessen}}
Dazu seien $\alpha$, $\beta$ und $\gamma$ \glsIdxPl{vergleichbar}%
\footnote{siehe Ende \vrefvonsub{sub:Aussagen}}
\glsIdxPl{Formel}.
Zunächst wird definiert:
\begin{align}
	\gamma(\alpha \subst \beta) \quad \defeq \quad
	\parbox[t]{11cm}{%
		Das \glsIdxBg{Formel}{formale Element}, dass man erhält, wenn in $\gamma$ alle oder nur einige Vorkommen von $\alpha$ durch $\beta$ ersetzt werden.
		-- Gegebenenfalls muss noch die Auswahl der Ersetzungen angegeben werden, andernfalls werden alle Vorkommen ersetzt.
		Letzteres heißt dann \emph{vollständige} \glsIdx{Substitution}.
	} \label{def:Substitution}\\
	\gamma(\alpha \swap \beta) \quad \defeq \quad
	\parbox[t]{11cm}{%
		Das \glsIdxBg{Formel}{formale Element}, dass man erhält, wenn in $\gamma$ alle $\alpha$ und $\beta$ miteinander vertauscht werden.
		Dazu ist es nötig, das $\alpha$ und $\beta$ voneinander unabhängig sind, vorzugsweise zwei verschiedene Variable.
	} \label{def:Vertauschung}
\end{align}
\seqqt{$ \alpha \subst \beta $} heißt \emph{\glsIdx{Substitution}} und \seqqt{$ \alpha \swap \beta $} \emph{\glsIdx{Vertauschung}} oder kurz \emph{Tausch}.
-- Sei noch $S = (s_1, s_2, ...)$ eine endliche Folge von \glsIdxPl{Substitution}, die auch \glsIdxPl{Vertauschung} enthalten und auch leer sein kann. Dann wird definiert:
\begin{align}
	\gamma(S) & \quad \defeq \quad \gamma(s_1)(s_2)... \label{def:Substitutionen}\\
	\gamma(\emptyset) & \quad \; = \quad \gamma & \text{(nur zur Verdeutlichung)}\\
	\gamma(s_1,s_2,...) & \quad \defeq \quad \gamma(S)
\end{align}
%
Die \glsIdx{Vertauschung} ist eine spezielle Form der \glsIdx{Substitution}.
Wenn $x$ und $y$ zwei verschiedene Variable, die in $\alpha$, $\beta$ und $\gamma$ nicht vorkommen, gilt:
\[
	\gamma(\alpha \swap \beta) = \gamma(\alpha\subst x, \beta\subst y,  y\subst\alpha, x \subst\beta)
\]

Sei zusätzlich noch $s$ eine \glsIdx{Substitution}.
Folgende Sprechweisen werden verwendet:
\begin{itemize}
	\renewcommand*{\itemindent}{1,5cm}
	\renewcommand*{\labelsep}{5pt}
	\item [$\gamma(\alpha \subst \beta)$ :] In $\gamma$ wird $\alpha$ (vollständig) \emph{durch $\beta$ substituiert}.
	\item [$\gamma(\alpha \swap \beta)$ :] In $\gamma$ werden $\alpha$ und $\beta$ \emph{vertauscht}.
	\item [$\gamma(s)$ :] $s$ wird auf $\gamma$ \emph{angewendet}.
	\item [$\gamma(S)$ :] Die \glsIdxPl{Substitution} aus S werden in der angegebenen Reihenfolge auf $\gamma$ angewendet.
	\item [$\gamma(S)$ :] $S$ wird auf $\gamma$ angewendet.
\end{itemize}
%
Bei obiger Definition der \glsIdx{Substitution} bleibt noch offen, unter welchen \glsIdxPl{Voraussetzung} sie angewendet werden darf. Das soll hier nicht untersucht werden. In diesem \sectionname\ genügt es, das nur \glsIdx{Vertauschung} und vollständige \glsIdx{Substitution} verwendet werden.
In diesen Fällen sind beliebige \glsIdxPl{Substitution} von Variablen durch \glsIdxPl{Formel} erlaubt.

Ist $\gamma$ wie oben und $S$ eine Menge von \glsIdxPl{Substitution}.

Nun können die beiden \glsIdxPl{Identitaetsregel} definiert werden:
\begin{align}
	& \frac{}{\alpha\eq\alpha}
	& & (\text{Einführung der Identität})
	\tag{\tageqE} \sym{\gls{eqE}} \label{def:eqE}
	\\\\
	& \frac{\alpha\eq\beta \srand \gamma}{\gamma(\alpha\subst\beta)}
	& & (\text{Beseitigung der Identität})
	\tag{\tageqB} \sym{\gls{eqB}} \label{def:eqB}
	\formulatoleft
\end{align}
%
Die \glsIdxPl{Identitaetsregel} werden hier eingeführt, um die \glsIdx{Substitution} zu rechtfertigen.
Wie die \glsIdxPl{Basisregel} gelten sie als \glsIdxPl{Axiom}, würden also eigentlich dazu gehören.
Da sie aber nicht weiter verwendet werden, werden sie hier nicht zu den \glsIdxPl{Basisregel} gezählt.

\subsection{Weitere Schlussregeln}% --------------------------------------------
\label{sub:weitereSchlussregeln}

In~\cite{bib:Rautenberg} werden aus den Basisregeln mittels \glsIdxBg{zulaessige-Transformation}{zulässiger Transformationen} weitere \glsIdxPl{Schlussregel} abgeleitet.%
%TODO Identitätsregeln kommen bei Rautenberg später vor. ???
\footnote{%
	In~\cite{bib:Rautenberg} werden die \glsIdxPl{Identitaetsregel} zwar weder aufgeführt noch angewandt, ohne \glsIdx{Substitution} geht es aber nicht.
}
Man vergleiche auch mit~\cite{bib:NatuerlichesSchliessen}.
%
\begin{align}
	& \frac{X,\lnot\alpha\derive\alpha}{X\derive\alpha}
	& & (\text{Beseitigung der Negation; Indirekter \glsIdx{Beweis}})
	\tag{\tagnotc} \sym{\gls{notc}} \label{def:notc}
	\\\\
	& \frac{X,\lnot\alpha\derive\beta,\lnot\beta}{X\derive\alpha}
	& & (\text{Reductio ad absurdum})
	\tag{\tagnotd} \sym{\gls{notd}} \label{def:notd}
	\\\\
	& \frac{X,\alpha\derive\beta}{X\derive\alpha\limp\beta}
	& & (\text{Einführung der Implikation})
	\tag{\tagimpE} \sym{\gls{impE}} \label{def:impE}
	\\\\
	& \frac{X\derive\alpha\limp\beta}{X,\alpha\derive\beta}
	& & (\text{Beseitigung der Implikation})
	\tag{\tagimpB} \sym{\gls{impB}} \label{def:impB}
	\\\\
	& \frac{X\derive\alpha \srand X,\alpha\derive\beta}{X\derive\beta}
	& & (\text{\glsIdx{Schnittregel}})
	\tag{\tagSR} \sym{\gls{SR}} \label{def:SR}
	\\\\
	& \frac{X\derive\alpha \srand \alpha\limp\beta}{X\derive\beta}
	& & (\text{\glsIdx{Abtrennungsregel}--\emph{Modus ponens}})
	\tag{\tagTR} \sym{\gls{TR}} \label{def:TR}
	\formulatoleft
\end{align}
%
Dabei werden zum \glsIdx{Beweis} der \glsIdxPl{Schlussregel} in~\cite{bib:Rautenberg} folgende Basisregeln verwendet:
\begin{itemize}
	\renewcommand*{\itemindent}{1cm}
	\renewcommand*{\labelsep}{5pt}
	\item[\tagnotc~:] \tagAR, \tagMR,           \tagnotb
	\item[\tagnotd~:] \tagAR, \tagMR, \tagnota, \tagnotb
	\item[\tagimpE~:] \tagAR, \tagMR, \tagnota, \tagnotb, \tagandE
	\item[\tagimpB~:] \tagAR, \tagMR, \tagnota, \tagnotb          , \tagandB
	\item[\tagSR  ~:] \tagAR, \tagMR, \tagnota, \tagnotb
	\item[\tagTR  ~:] \tagAR, \tagMR, \tagnota, \tagnotb, \tagandE
\end{itemize}
%
\subsection{Beispiel einer Ableitung}% -----------------------------------------
\label{sub:BeispielAbleitung}

Als Beispiel wird hier die \glsIdx{Schnittregel} aus den Basisregeln abgeleitet.%
\footnote{%
	Die Form der Tabelle ist angelehnt an~\cite{bib:NatuerlichesSchliessen} Kapitel~2.2.4 \emph{Eine Beispielableitung}.
}
Dazu wird verabredet, dass \vrefintab{tab:AbleitungSchnittregel} der Inhalt der Zelle in der Zeile $i$ und der Spalte $(X_n)$ mit $X_i$ bezeichnet wird.
Zur kürzeren Darstellung wird statt auf die vollständigen Spaltenüberschriften nur auf die dort notierten $(X_n)$ verwiesen. Dass in der Spalte $(n)$ stets die Zeilennummer steht, wird im folgenden nicht mehr extra erwähnt.
-- Für die ausgefüllten Felder wird nun definiert:%
\footnote{%
	Eigentlich müsste man für jede \glsIdx{Substitution} aus $S_i$ eine eigene Zeile vorsehen.
	Um die Tabellen für die \glsIdxPl{Beweis} kürzer zu halten, werden aufeinanderfolgende \glsIdxPl{Substitution} zusammengefasst.
}
\begin{align}
	R_i & \defeq
	\left\{
		\begin{array}{l}
			\text{- \enquote{\glsIdx{Voraussetzung}} = Die \glsIdx{Aussage} $A_i$ ist eine \glsIdx{Voraussetzung}.}\\
			\text{- \enquote{\glsIdx{Folgerung}} = Die \glsIdx{Aussage} $A_i$ ist eine \glsIdx{Folgerung}.}\\
			\text{- \enquote{Annahme} = Die \glsIdx{Aussage} $A_i$ wird vorübergehend als zutreffend angenommen.}\\
			\text{- $j$ = Verweis auf die \glsIdx{Schlussregel} $\overline{R}_j$ für ein $j < i$.}\\
			\text{- Verweis (ohne Klammern) auf eine \glsIdx{allgemeingueltige-Schlussregel}.}
		\end{array}
	\right.
	\\
	S_i & \defeq \text{Die Reihe der anzuwendenden \glsIdxPl{Substitution}.}
	\\
	\overline{R}_i & \defeq \text{Das Ergebnis der in der angegebenen Reihenfolge angewendeten}\\
	& \quad\;\; \text{\glsIdxPl{Substitution} aus $S_i$ auf die \glsIdx{Schlussregel} $R_i$}
	\\
	Z_i & \defeq \text{Die Indizes $j$ (mit $j < i$) als Verweise auf eine oder mehrere \glsIdxPl{Aussage} $A_j$,}\\
	& \quad\;\; \text{welche zusammen genau die \glsIdxPl{Voraussetzung} der \glsIdx{Schnittregel} } \overline{R}_i \text{ erfüllen.}
	\\
	A_i & \defeq \text{\glsIdx{Folgerung}(en) der \glsIdx{Schlussregel} $\overline{R}_i$ --}\\
	& \quad\;\; \text{auch in Form der Indizes von einem oder mehreren von $Aj$ (mit $j < i$).}\\
	& \quad\;\; \text{In der Ergebniszeile kann hier auch die bewiesene \glsIdx{Aussage} als Schlussregel stehen.}
	\\
	D_i & \defeq \text{die Indizes der $A_j$, von denen $A_i$ abhängig ist.}
\end{align}
Bis zur Zeile $i$ hat man die folgende \glsIdx{Schlussregel} bewiesen:
\[ \frac{A_{i_1} \srand A_{i_2} ...}{A_i} \quad \text{, für alle } i_j \in D_i \]
Sei nun
\[
	\Gamma_i \defeq
	\left\{
		\begin{array}{ll}
			\text{leer}    & \text{ für } R_i = \text{\enquote{\glsIdx{Voraussetzung}}} \\
			\text{leer}    & \text{ für } R_i = \text{\enquote{\glsIdx{Folgerung}}}     \\
			\text{leer}    & \text{ für } R_i = \text{\enquote{Annahme}}       \\
			\overline{R_j} & \text{ für } R_i = j \quad \text{(eine \emph{interne} \glsIdx{Schlussregel})} \\
			\text{die \glsIdx{Schlussregel}} & \text{ für } R_i = \text{Verweis auf eine \emph{externe} \glsIdx{Schlussregel}}
		\end{array}
	\right.
\]
Damit gilt für die Einträge in einer Zeile $i$:
\begin{itemize}
	\item Wenn $\Gamma_i$ nicht leer ist, ist $R_i$ eine \glsIdx{Schlussregel} mit $R_i = \Gamma_i(S_i)$%
	\footnote{%
		siehe Definition~\eqref{def:Substitutionen} \vrefvonsub{sub:Identitätsregeln}
	}.
	\item Wenn $A_i$ nicht leer ist, ist $R_i = \dfrac{A_{z_1} \srand A_{z_2} \srand ...}{A_i}$ (alle $z_j \in Z_i$).
	\item Wenn $A_i$ nicht leer ist, ist bis jetzt die \glsIdx{Schlussregel} $\dfrac{A_{d_1} \srand A_{d_2} \srand ...}{A_i}$ (alle $d_j \in D_i$) schon bewiesen.
\end{itemize}
$S_i$, $Z_i$ und $D_i$ dürfen dabei auch leer sein.

\begin{table}[!htb]
	\setlength\tabcolsep{1pt}
	\setlength\extrarowheight{7pt}
	\newcommand*{\centerParbox}[2]{\parbox{#1}{\centering #2}}
	\newcommand*{\titleCell}[3]{\centerParbox{#1}{\textbf{#2} (#3)}}
	\newcommand*{\SnCell}[1]{\centerParbox{1.85cm}{#1}}
	\newcommand*{\DnCell}[1]{\centerParbox{1.95cm}{#1}}
	\begin{tabular}{|c||c|c|c|c|c|c|}
		\hline
		\titleCell{0.95cm}{Zeile}                       {$n$} &
		\titleCell{1.05cm}{Regel}                     {$R_n$} &
		\titleCell{1.85cm}{Substitu"=tionen}          {$S_n$} &
		\titleCell{1.80cm}{erzeugte Regel} {$\overline{R}_n$} &
		\titleCell{2.15cm}{angewendet auf ...}        {$Z_n$} &
		\titleCell{1.65cm}{\glsIdx{Aussage}}          {$A_n$} &
		\titleCell{1.95cm}{Abhängig"=keiten}          {$D_n$}
		\\\hline\hline
		1 & \centerParbox{1.35cm}{Voraus"=setzung} & & & & $X \derive \alpha$ & 1
		\\\hline
		2 & \centerParbox{1.35cm}{Voraus"=setzung} & & & & $X,\alpha \derive \beta$ & 2
		\\\hline
		3 & \centerParbox{1.00cm}{Folge"=rung} & & & & $X \derive \beta$ & 3
		\\\hline
		4 & \tagMR & & $\dfrac{X \derive \alpha}{X, Y \derive \alpha}$ & & &
		\\\hline
		5 & 4 & $Y \subst \lnot\alpha$ & $\dfrac{X \derive \alpha}{X, \lnot\alpha \derive \alpha}$ & 1 & $X, \lnot\alpha \derive \alpha$ & 1
		\\\hline
		6 & \tagAR & & $ \dfrac{}{\alpha \derive \alpha} $ & & &
		\\\hline
		7 & 6 & $\alpha \subst \lnot\alpha$ & $\dfrac{}{\lnot\alpha \derive \lnot\alpha}$ & & $\lnot\alpha \derive \lnot\alpha$ &
		\\\hline
		8 & 4 & \SnCell{%
			$\alpha \subst \lnot\alpha$\\
			$X \subst \lnot\alpha$\\
			$Y \subst X$
		} & $\dfrac{\lnot\alpha \derive \lnot\alpha}{X,\lnot\alpha \derive \lnot\alpha}$ & 7 & $X,\lnot\alpha \derive \lnot\alpha$ &
		\\\hline
		9 & \tagnota & & $\dfrac{X \derive \alpha, \lnot\alpha}{X \derive \beta}$ & & &
		\\\hline
		10 & 9 & $X \subst X, \lnot\alpha$ & $\dfrac{X,\lnot\alpha \derive \alpha, \lnot\alpha}{X,\lnot\alpha \derive \beta}$ & 5, 8 & $X,\lnot\alpha \derive \beta$ & 1
		\\\hline
		11 & \tagnotb & & $\dfrac{X,\alpha \derive \beta \srand X,\lnot\alpha \derive \beta}{X \derive \beta}$ & 2, 10 & 3 & 1, 2
		\\\hline\hline
		12 & \centerParbox{1.4cm}{\tagAR, \tagMR, \tagnota, \tagnotb} & & $\dfrac{A_1 \srand A_2}{A_3}$ & & $\dfrac{X \derive \alpha \srand X, \alpha \derive \beta}{X \derive \beta}$ &
		\\\hline
	\end{tabular}
	\caption{Ableitung der \glsIdx{Schnittregel} aus den \glsIdxPl{Basisregel}}
	\label{tab:AbleitungSchnittregel}
\end{table}

Die Erzeugung einer Tabelle analog zu~\vref{tab:AbleitungSchnittregel} wird im folgenden beschrieben.
Zellen, für die kein Inhalt angegeben wird, bleiben leer.
Rückwärts-Referenzen auf schon ausgefüllte Zellinhalte sind jederzeit möglich.
Das Eintragen der Zeilennummer $i$ wird nicht extra erwähnt.
-- Die Tabelle und die Beschreibung sind so ausführlich, damit man daraus leicht ein Computerprogramm erstellen kann.
%
\begin{enumerate}
	%
	\item Am Anfang der Tabelle werden zuerst \glsIdxPl{Voraussetzung}, dann zu beweisende \glsIdxPl{Folgerung} und schließlich Annahmen aufgeführt.%
	\footnote{%
		Die Angabe ist dann erforderlich, wenn darauf verwiesen wird.
		Durch die Auflistung hat man aber einen vollständigen Überblick über die \glsIdxPl{Voraussetzung} und \glsIdxPl{Folgerung} eines \glsIdx{Beweis}es und die Zwischenannahmen.
		Auf jede nötige \glsIdx{Voraussetzung} und jede verwendete Zwischenannahme wird in der Spalte $(Z_n$) mindestens einmal verwiesen, so dass sie auch aufgeführt werden müssen.
		Die Angabe der \glsIdxPl{Folgerung} erleichtert die Erstellung einer \emph{Ergebniszeile} (siehe Punkt~\ref{item:Ergebniszeile}).
	}
	Jede der drei Gruppen kann auch leer sein und es ist auch möglich, die Zeilen an anderen Stellen der Tabelle anzugeben, spätestens aber, wenn darauf verwiesen wird.
	Für jede \glsIdx{Voraussetzung}, \glsIdx{Folgerung} und Annahme gibt es eine Zeile:
	\begin{enumerate}
		\item $R_i =$ \enquote{\glsIdx{Voraussetzung}}, \enquote{\glsIdx{Folgerung}} oder \enquote{Annahme}.
		\item $A_i =$ Die aktuelle \glsIdx{Voraussetzung}, \glsIdx{Folgerung} oder Annahme.
		\item $D_i =$ $i$ \quad (ein Verweis auf $A_i$).
	\end{enumerate}
	%
	\item In den nächsten Zeilen werden die \glsIdxPl{Beweisschritt} aufgeführt, für jeden Schritt eine Zeile.

	Zunächst kann $R_i$ kann auf zwei Arten erzeugt werden:
	\begin{enumerate}
		\setcounter{enumii}{\value{Enumii}}% Nummerierung wird fortgesetzt.
		\item
		\begin{enumerate}
			\item $R_i =$ Verweis auf eine \glsIdx{allgemeingueltige-Schlussregel}.
			\item $\overline{R}_i =$ Die \glsIdx{Schlussregel}, auf die verwiesen wird.
		\end{enumerate}
		\setcounter{Enumii}{\value{enumii}}% Nummerierung wird fortgesetzt.
	\end{enumerate}
	oder
	\begin{enumerate}
		\item
		\begin{enumerate}
			\item $R_i = j$, wenn die schon bewiesene \glsIdx{Schlussregel} $\overline{R}_j$ (mit $j < i$) angewendet werden soll.
			\item $S_i =$ Die auf die \glsIdx{Schlussregel} $R_i$ anzuwendende \glsIdx{Substitution}.
			\item $\overline{R}_i =$ Das Ergebnis der \glsIdx{Substitution} $S_i$ auf die \glsIdx{Schlussregel} $R_i$.
		\end{enumerate}
		\setcounter{Enumii}{\value{enumii}}% Nummerierung wird fortgesetzt.
	\end{enumerate}
	Man beachte, dass die \glsIdx{Schlussregel} $\overline{R}_i$, stets allgemeingültig ist, da sie ausschließlich aus \glsIdxBg{allgemeingueltige-Schlussregel}{allgemeingültigen Schlussregeln} mittels \glsIdxPl{Substitution} abgeleitet worden ist.
	Daher gibt es auch keine Beschränkung weiterer \glsIdxPl{Substitution} durch irgendwelche Abhängigkeiten.

	Nun kann die Zeile beendet werden, oder es geht weiter mit:
	\begin{enumerate}
		\setcounter{enumii}{\value{Enumii}}% Nummerierung wird fortgesetzt.
		\item \label{item:Anwendung} $Z_n =$ Die Indizes aller $A_j$ (mit $j < i$), die eine \glsIdx{Voraussetzung} der \glsIdx{Schlussregel} $\overline{R}_i$ sind, möglichst in der verwendeten Reihenfolge.
		-- Für jedes angegebene $j$ werden noch die Abhängigkeiten $D_j$ den Abhängigkeiten $D_i$ hinzugefügt.
		%
		\item $A_i =$ \glsIdx{Folgerung}(en) der \glsIdx{Schlussregel} $\overline{R}_i$.
		-- Wenn diese \glsIdxPl{Folgerung} schon als \glsIdxPl{Aussage} $A_j$ (mit $j < i$) vorhanden sind, können auch einfach deren Indizes eingetragen werden.
		Damit werden die Zusammenhänge und der Abschluss des \glsIdx{Beweis}es besser ersichtlich.
		%
		\item $D_i =$ Die Verweise wurden schon in (\ref{item:Anwendung}) eingetragen.%
		\footnote{Wenn $D_n$ leer ist, dann ist $A_n$ allgemeingültig.}
		%
	\end{enumerate}
	Der \glsIdx{Beweis} muss so lange fortgeführt werden, bis alle \glsIdxPl{Folgerung} als \glsIdxPl{Aussage} in der Spalte $(A_n)$ erschienen und dort jeweils nur von den gegebenen \glsIdxPl{Voraussetzung} abhängig sind.
	%
	\item \label{item:Ergebniszeile} In einer \emph{Ergebniszeile}, die dann die letzte ist, kann noch die bewiesene Behauptung in Form einer \glsIdx{Schlussregel} formuliert und in einer passenden Spalte notiert werden.
	Zusätzlich können dort auch noch alle verwendeten \glsIdxPl{Schlussregel} gesammelt werden.
	Dies kann \textzB\ folgendermaßen geschehen:
	\begin{enumerate}
		%
		\item $(R_n) =$ Verweise auf alle verwendeten externen \glsIdxPl{Schlussregel}.
		%
		\item $(\overline{R}_n) =$ Die bewiesene Behauptung als \glsIdxPl{Schlussregel}, wobei alle $A_i$, die \glsIdxPl{Voraussetzung} sind, als \glsIdx{Voraussetzung} und alle $A_j$, die \glsIdxPl{Folgerung} sind, als \glsIdx{Folgerung} eingesetzt werden, jeweils in der Form \enquote{$A_i$} \textbzgl\ \enquote{$A_j$}.
		Das ergibt dann:
		\[ \frac{A_{i_1} \srand A_{i_2} \srand ...}{A_{j_1} \srand A_{j_2} \srand ...} \]
		%
		\item $(A_n) =$ $\overline{R}_i$, wobei die \glsIdxPl{Voraussetzung} und \glsIdxPl{Folgerung} aufgelöst werden.
		%
		\item $(D_n) =$ Die Vereinigung aller Abhängigkeiten der \glsIdxPl{Folgerung}, vermindert um die \glsIdxPl{Voraussetzung}.
		-- Wenn das Feld dabei nicht leer bleibt, ist der \glsIdx{Beweis} missglückt!
		%
	\end{enumerate}
	%
\end{enumerate}
%
Ein weiteres Beispiel \vrefintab{tab:AbleitungKontraposition} soll verdeutlichen, wie Abhängigkeiten von Zwischenannahmen wieder beseitigt werden können.%
\footnote{siehe~\cite{bib:NatuerlichesSchliessen},
Kapitel 2.2.4 \emph{Eine Beispielableitung}}

\begin{table}[!htb]
	\setlength\tabcolsep{1pt}
	\setlength\extrarowheight{7pt}
	\newcommand*{\centerParbox}[2]{\parbox{#1}{\centering #2}}
	\newcommand*{\titleCell}[3]{\centerParbox{#1}{\textbf{#2} (#3)}}
	\newcommand*{\SnCell}[1]{\centerParbox{2.30cm}{#1}}
	\newcommand*{\DnCell}[1]{\centerParbox{1.95cm}{#1}}
	\begin{tabular}{|c||c|c|c|c|c|c|}
		\hline
		\titleCell{0.95cm}{Zeile}                       {$n$} &
		\titleCell{1.05cm}{Regel}                     {$R_n$} &
		\titleCell{1.85cm}{Substitu"=tionen}          {$S_n$} &
		\titleCell{1.80cm}{erzeugte Regel} {$\overline{R}_n$} &
		\titleCell{2.15cm}{angewendet auf ...}        {$Z_n$} &
		\titleCell{1.65cm}{\glsIdx{Aussage}}          {$A_n$} &
		\titleCell{1.95cm}{Abhängig"=keiten}          {$D_n$}
		\\\hline \hline
		1 & \centerParbox{1.00cm}{Folge"=rung} & & & & $(\alpha\limp\beta)\limp(\lnot\beta\limp\lnot\alpha)$ & 1
		\\\hline
		2 & \centerParbox{1.20cm}{An"=nahme} & & & & $\alpha\limp\beta$ & 2
		\\\hline
		3 & \centerParbox{1.20cm}{An"=nahme} & & & & $\lnot\beta$ & 3
		\\\hline
		4 & \centerParbox{1.20cm}{An"=nahme} & & & & $\alpha$ & 4
		\\\hline
		5 & \tagimpB & & $\dfrac{X \derive \alpha\limp\beta}{X,\alpha \derive \beta}$ & & &
		\\\hline
		6 & -1 & $X \subst \emptyset$ & $\dfrac{\alpha\limp\beta}{\alpha \derive \beta}$ & 2 & $\alpha \derive \beta $ & 2
		\\\hline
		7 & \tagSR & & $\dfrac{X \derive \alpha \srand X,\alpha \derive \beta}{X \derive \beta}$ & & &
		\\\hline
		8 & -1 & $X \subst \emptyset$ & $\dfrac{\alpha \srand \alpha \derive \beta}{\beta}$ & 4, 6 & $\beta $ & 4, 6
		\\\hline
		9' & \tagandE & & $\dfrac{X \derive \alpha, \beta}{X \derive \alpha \land \beta}$ & & &
		\\\hline
		10' & -1 & $X \subst \emptyset$ & $\dfrac{\alpha \srand \beta}{\alpha \land \beta}$ & & &
		\\\hline
		11' & -1 &\SnCell{
			$\alpha \swap \beta$\\
			$\alpha \subst \lnot\beta$
		}  & $\dfrac{\beta \srand \lnot\beta}{\beta \land \lnot\beta}$ & 8, 3 & $\beta \land \lnot\beta$ &
		\\\hline
		9 & \tagnota & & $\dfrac{X \derive \alpha, \lnot\alpha}{X \derive \beta}$ & & &
		\\\hline
		10 & -1 & $X \subst \emptyset$ & $\dfrac{\alpha \srand \lnot\alpha}{\beta}$ & & &
		\\\hline
		11 & -1 & \SnCell{
			$\alpha \swap \beta$\\
			$\alpha \subst \lnot\alpha$
		} & $\dfrac{\beta \srand \lnot\beta}{\lnot\alpha}$ & 8, 3 & $\lnot\alpha$ & 2, 3, 4
		\\\hline
		12 & \tagimpE & & $\dfrac{X, \alpha \derive \beta}{X \derive \alpha\limp\beta}$ & & &
		\\\hline
		13 & -1 & $X \subst \emptyset$ & $\dfrac{\alpha \derive \beta}{\alpha\limp\beta}$ & & &
		\\\hline
		14 & -1 & \SnCell{
			$\alpha \swap \beta$\\
			$\alpha \subst \lnot\alpha$\\
			$\beta \subst \lnot\beta$
		} & $\dfrac{\lnot\beta \derive \lnot\alpha}{\lnot\beta\limp\lnot\alpha}$ & 3, 11, ??? & $\lnot\beta\limp\lnot\alpha$ & 2, 3, 4, ???
		\\\hline
		15 & \tagimpE+1 & \SnCell{
			$\alpha \subst \gamma$\\
			$\beta \subst \delta$\\
			$\gamma \subst \alpha\limp\beta$\\
			$\delta \subst \lnot\beta\limp\lnot\alpha$
		} & $\dfrac{\alpha\limp\beta \derive \lnot\beta\limp\lnot\alpha}
		{(\alpha\limp\beta)\limp(\lnot\beta\limp\lnot\alpha)}$ & 2, 14 &
		$(\alpha\limp\beta)\limp(\lnot\beta\limp\lnot\alpha)$ & 2, 3, 4, ???
		\\\hline\hline
		16 & \centerParbox{1.5cm}{\tagimpE, \tagimpB, \tagSR} & & $\dfrac{}{A_1}$ & & $\dfrac{}{(\alpha\limp\beta)\limp(\lnot\beta\limp\lnot\alpha)}$ &
		\\\hline
	\end{tabular}
	\caption{Ableitung der \glsIdx{Kontraposition} aus \glsIdxBg{allgemeingueltige-Schlussregel}{allgemeingültigen Schlussregeln}}
	\label{tab:AbleitungKontraposition}
\end{table}

\todo{Beispielableitung der Kontraposition vervollständigen}%%%
%TODO Beispielableitung der Kontraposition vervollständigen %%%

\Endchapter

	\end{offen}%%%
	%%############################################################################%%
%%                                                                            %%
%% Datei:  ASBA-Design.tex                                                    %%
%% Inhalt: Kapitel "Design"                                                   %%
%%                                                                            %%
%% Copyright (C) 2017  Winfried Teschers                                      %%
%%                                                                            %%
%% This program is free software: you can redistribute it and/or modify       %%
%% it under the terms of the GNU Affero General Public License as published   %%
%% by the Free Software Foundation, either version 3 of the License, or       %%
%% (at your option) any later version.                                        %%
%%                                                                            %%
%% This program is distributed in the hope that it will be useful,            %%
%% but WITHOUT ANY WARRANTY; without even the implied warranty of             %%
%% MERCHANTABILITY or FITNESS FOR A PARTICULAR PURPOSE.  See the              %%
%% GNU Affero General Public License for more details.                        %%
%%                                                                            %%
%% You should have received a copy of the GNU Affero General Public License   %%
%% along with this program.  If not, see <http://www.gnu.org/licenses/>.      %%
%%                                                                            %%
%% Dr. Winfried Teschers                                                      %%
%% Anton-Günther-Straße 26c                                                   %%
%% 91083 Baiersdorf                                                           %%
%% Germany                                                                    %%
%%                                                                            %%
%% e-mail: winfried.teschers@t-online.de                                      %%
%%                                                                            %%
%%############################################################################%%

% !TeX root = ASBA.tex
% !TeX encoding = UTF-8
% !TeX spellcheck = de_DE

\chapter{Design}% ##############################################################
\beginchapter{Design}
\label{cha:Design}

Dieses Projekt soll Open Source sein.
Daher gilt für die Dokumente die \emph{GNU Free Documentation License (FDL)} und für die Software die \emph{GNU Affero General Public License (APGL)}.
Die \emph{GNU General Public License (GPL)} reicht für die Software nicht aus, da das Programm auch mittels eines Servers betrieben werden kann und soll.
Damit das Projekt gegebenenfalls durch verschiedene Entwickler gleichzeitig bearbeitet werden kann und wegen des Konfigurationsmanagements wurde es als ein GitHub Projekt erstellt (siehe~\cite{bib:ASBA}).

Wenn die Lizenzen nicht mitgeliefert wurden, können sie unter \url{http://www.gnu.org/licenses/} gefunden werden.

\section{Anforderungen}% =======================================================
\beginsection{Anforderungen}
\label{sec:Anforderungen}

Die Anforderungen ergeben sich zunächst aus den Zielen \vrefinsec{sec:Ziele}.
Die beiden Ziele~\ref{Ziel:Daten}~\emph{Daten} und~\ref{Ziel:Lizenz}~\emph{Lizenz} sind für die Entwicklung von \ASBA\ von sekundärer Bedeutung und werden daher in diesem \sectionname\ nicht übernommen.
Die anderen Ziele werden noch verfeinert.

\todo{Ziele aus Abschnitt "'Ziele"' in Anforderungen umwandeln.}%%%
%TODO Ziele aus Abschnitt "'Ziele"' in Anforderungen umwandeln. %%%
%
\begin{enumerate}

	\item \label{Anforderung:Form} \emph{Form}:
	Die Daten liegt in formaler, geprüfter Form vor.
	(\vrefseeziel{Ziel:Form})

	\item \label{Anforderung:Eingaben} \emph{Eingaben}:
	Die Eingabe von Daten erfolgt in einer formalen Syntax unter Verwendung der üblichen mathematischen Schreibweise.
	Folgende Daten können eingegeben werden:
	\begin{enumerate}
		\item \Axiome
		\item \Saetze
		\item \Beweise
		\item \Fachbegriffe
		\item \Fachgebiete
		\item \Ausgabeschemata
	\end{enumerate}
	Dabei sind alle Begriffe nur innerhalb eines Fachgebiets und seiner untergeordneten \Fachgebiete\ gültig, solange sie nicht umdefiniert werden.
	Das oberste \Fachgebiet\ ist die ganze Mathematik.
	-- \vrefseeziel{Ziel:Eingaben}

	\item \label{Anforderung:Prüfung} \emph{Prüfung}:
	Vorhandene \Beweise\ können automatisch geprüft werden.
	-- \vrefseeziel{Ziel:Prüfung}

	\item \label{Anforderung:Ausgaben} \emph{Ausgaben}:
	Die Ausgabe kann in einer eindeutigen, formalen Syntax gemäß vorhandener \Ausgabeschemata\ erfolgen.
	-- \vrefseeziel{Ziel:Ausgaben}

	\item \label{Anforderung:Auswertungen} \emph{Auswertungen}:
	Zusätzlich zur Ausgabe der Daten sind verschiedene Auswertungen möglich.
	Insbesondere kann zu jedem \Beweis\ angegeben werden, wie lang er ist und welche \Axiome\ und Sätze%
	\footnote{Sätze, die quasi als \Axiome\ verwendet werden.}
	er benötigt.
	-- \vrefseeziel{Ziel:Auswertungen}

	\item \label{Anforderung:Anpassbarkeit} \emph{Anpassbarkeit}:
	\Fachbegriffe\ und die Darstellung bei der Ausgabe können mit Hilfe von -- gegebenenfalls unbenannten -- untergeordneten \Fachgebieten\ angepasst werden.
	-- \vrefseeziel{Ziel:Anpassbarkeit}

	\item \label{Anforderung:Individualität} \emph{Individualität}:
	\Axiome\ und Sätze können für jeden \Beweis\ individuell vorausgesetzt werden.
	Dabei sind fachgebietsspezifische \Fachbegriffe\ erlaubt.
	-- \vrefseeziel{Ziel:Individualität})

	\item \label{Anforderung:Internet} \emph{Internet}:
	Die Daten können auf mehrere Dateien verteilt sein.
	Ein Teil davon -- oder sogar alle -- können im Internet liegen.
	-- \vrefseeziel{Ziel:Internet}

	\item \label{Anforderung:Kommunikation} \emph{Kommunikation}:
	Die Kommunikation mit \ASBA\ kann mit den \Fachbegriffen\ der einzelnen \Fachgebiete\ erfolgen.
	-- \vrefseeziel{Ziel:Kommunikation}

	\item \label{Anforderung:Zugriff} \emph{Zugriff}:
	Der Zugriff auf \ASBA\ kann lokal und über das Internet erfolgen.
	-- \vrefseeziel{Ziel:Zugriff}

	\item \label{Anforderung:Unabhängigkeit} \emph{Unabhängigkeit}:
	\ASBA\ kann offline und online arbeiten.
	-- \vrefseeziel{Ziel:Unabhängigkeit}

	\item \label{Anforderung:Rekursion} \emph{Rekursion}:
	Es kann rekursiv über alle verwendeten Dateien -- auch solchen, die im Internet liegen -- ausgewertet werden.
	-- \vrefseeziel{Ziel:Rekursion}

	\item \label{Anforderung:Bedienbarkeit} \emph{Bedienbarkeit}:
	\ASBA\ ist einfach zu bedienen.
	-- \vrefseeziel{Ziel:Bedienbarkeit}

\end{enumerate}

\section{Axiome}% ==============================================================
\beginsection{\Axiome}
\label{sec:Axiome}
\todo{Axiome auswählen und definieren.}%%%
%TODO Axiome auswählen und definieren. %%%

\section{Beweise}% =============================================================
\beginsection{\Beweise}
\label{sec:Beweise}
\todo{Schlussregeln auswählen und Beweise definieren.}%%%
%TODO Schlussregeln auswählen und Beweise definieren. %%%

\section{Datenstruktur}% =======================================================
\beginsection{Datenstruktur}
\label{sec:Datenstruktur}
\todo{Datenstruktur abstrakt und in XML definieren.}%%%
%TODO Datenstruktur abstrakt und in XML definieren. %%%

\section{Bausteine}% ===========================================================
\beginsection{Bausteine}
\label{sec:Bausteine}
\todo{Bausteine? definieren.}%%%
%TODO Bausteine? definieren. %%%

\Endchapter

	%%############################################################################%%
%%                                                                            %%
%% Datei:  ASBA-Anhang.tex                                                    %%
%% Inhalt: Anhang                                                             %%
%%                                                                            %%
%% Copyright (C) 2017  Winfried Teschers                                      %%
%%                                                                            %%
%% This program is free software: you can redistribute it and/or modify       %%
%% it under the terms of the GNU Affero General Public License as published   %%
%% by the Free Software Foundation, either version 3 of the License, or       %%
%% (at your option) any later version.                                        %%
%%                                                                            %%
%% This program is distributed in the hope that it will be useful,            %%
%% but WITHOUT ANY WARRANTY; without even the implied warranty of             %%
%% MERCHANTABILITY or FITNESS FOR A PARTICULAR PURPOSE.  See the              %%
%% GNU Affero General Public License for more details.                        %%
%%                                                                            %%
%% You should have received a copy of the GNU Affero General Public License   %%
%% along with this program.  If not, see <http://www.gnu.org/licenses/>.      %%
%%                                                                            %%
%% Dr. Winfried Teschers                                                      %%
%% Anton-Günther-Straße 26c                                                   %%
%% 91083 Baiersdorf                                                           %%
%% Germany                                                                    %%
%%                                                                            %%
%% e-mail: winfried.teschers@t-online.de                                      %%
%%                                                                            %%
%%############################################################################%%

% !TeX root = ASBA.tex
% !TeX encoding = UTF-8
% !TeX spellcheck = de_DE

\appendix
\renewcommand*{\Chaptername}{\appendixname}

\chapter     {Anhang}% #########################################################
\beginchapter{Anhang}
\label   {cha:Anhang}

\section     {Werkzeuge}% ======================================================
\beginsection{Werkzeuge}
\label   {sec:Werkzeuge}

Da dies ein Open Source Projekt sein soll, müssen alle Werkzeuge, die zum Ablauf der Software erforderlich sind, ebenfalls Open Source sein.
Für die reine Entwicklung sollte das auch gelten, muss es aber nicht.

\paragraph{Werkzeuge zur Übersetzung der Quelldateien}% --------------------

\begin{enumerate}

	\item\label{Werkzeug:LaTeX}
	Ein Übersetzer für \LaTeX\ Quellcode (*.tex).
	--- Verwendet wird \emph{MiK\TeX}.

	\item\label{Werkzeug:Cpp}
	Ein Übersetzer für C++ Quellcode (*.c, *.cpp, *.h, *.hpp).
	--- Verwendet wird \emph{Visual Studio Community 2017}.

	\setcounter{Enumi}{\value{enumi}}% Nummerierung wird fortgesetzt.
\end{enumerate}
%
Nicht unbedingt nötig, aber sinnvoll:
\begin{enumerate}
	\setcounter{enumi}{\value{Enumi}}% Nummerierung wird fortgesetzt.

	\item\label{Werkzeug:Dokumentation}
	Ein Dokumentationssystem für in C++ Quellcode und darin enthaltene Doxygen Kommentare (*.c, *.cpp, *.h, *.hpp).
	--- Verwendet wird \emph{Doxygen} mit Konfigurationsdatei "`Doxyfile"'.

	\item\label{Werkzeug:Konfigurationsmanagement}
	Ein Konfigurationsmanagementsystem zur Verwaltung der Quelldateien.
	--- Verwendet wird \emph{GitHub}.

	\setcounter{Enumi}{\value{enumi}}% Nummerierung wird fortgesetzt.
\end{enumerate}

\paragraph{Werkzeuge für die Entwicklung}% -------------------------------------

\begin{enumerate}
	\setcounter{enumi}{\value{Enumi}}% Nummerierung wird fortgesetzt.

	\item\label{Werkzeug:GitHub}\emph{GitHub} als Online Konfigurationsmanagementsystem zur Zusammenarbeit verschiedener Entwickler.
	\\\tourl{https://github.com/}
	--- Lizenz \citesee{bib:GPLii}

	\item\label{Werkzeug:Git}GitHub benötigt \emph{Git} als Konfigurationsmanagementsystem.
	\\\tourl{https://git-scm.com/}
	--- Lizenz \citesee{bib:GPLii}

	\item\label{Werkzeug:MiKTeX}\emph{MiK\TeX} für Dokumentation und Ausgaben in \LaTeX.
	\\\tourl{https://miktex.org/}
	--- Lizenz \citesee{bib:MiKTeX}

	\item\label{Werkzeug:VSC}angedacht: \emph{Visual Studio Community 2017}%
	\footnote{%
		Visual Studio Community ist zwar nicht Open Source, darf aber zur Entwicklung von Open Source Software
		unentgeltlich verwendet werden.
	}
	(\emph{VS}) als Entwicklungsumgebung für C++.
	\\\tourl{https://www.visualstudio.com/downloads/}
	--- Lizenz \citesee{bib:EULA}

	\item\label{Werkzeug:VSC DB}angedacht: In \emph{Visual Studio Community 2015} integrierte Datenbank für \Ausgabeschemata, \Saetze, \Beweise, \Fachbegriffe\ und \Fachgebiete.
	--- Lizenz \citesee{bib:EULA}

	\item\label{Werkzeug:RapidXml}angedacht: \emph{RapidXml} für Ein- und Ausgabe in XML.
	\\\tourl{http://rapidxml.sourceforge.net/index.htm}
	--- Lizenz \citesee{bib:BSLi} oder wahlweise~\cite{bib:MIT}
	\footnote{%
		RapidXml stellt eine C++ Header-Datei zur Verfügung.
		Wenn diese im Quellcode eines Programms enthalten ist, gilt das ganze Programm als Open Source.
		Wenn diese Header-Datei nur in einer Bibliothek innerhalb eines Projekts verwendet wird, so gilt nur diese Bibliothek als Open Source.
	}

	\item\label{Werkzeug:Doxygen}angedacht: \emph{Doxygen} als Dokumentationssystem für C++.
	\\\tourl{http://www.stack.nl/~dimitri/doxygen/}
	--- Lizenz \citesee{bib:GPLii}

	\item\label{Werkzeug:Ghostscript}angedacht: Doxygen benötigt \emph{Ghostscript} als Interpreter für Postscript und PDF.
	\\\tourl{http://ghostscript.com/}
	--- Lizenz \citesee{bib:AGPL}

	\item\label{Werkzeug:Graphviz}angedacht: Doxygen benötigt \emph{Graphviz} mit \emph{Dot} zur Erzeugung und Visualisierung von Graphen.
	\\\tourl{http://www.graphviz.org/Home.php}
	--- Lizenz \citesee{bib:EPL}

	\setcounter{Enumi}{\value{enumi}}% Nummerierung wird fortgesetzt.
\end{enumerate}

\paragraph{Werkzeuge zur Bearbeitung der Quelldateien}% ------------------------

\begin{enumerate}
	\setcounter{enumi}{\value{Enumi}}% Nummerierung wird fortgesetzt.

	\item\label{Werkzeug:TeXstudio}\emph{\TeX studio} als Editor für \LaTeX.
	\\\tourl{http://www.texstudio.org/}
	--- Lizenz \citesee{bib:GPLii}
	\\\TeX studio benötigt einen Interpreter für Perl:

	\item\label{Werkzeug:Perl}\emph{Strawberry Perl} als Interpreter für Perl.
	\\\tourl{http://strawberryperl.com/}
	--- Lizenz: Various OSI-compatible Open Source licenses, or given to the public domain

	\item\label{Werkzeug:Notepadpp}\emph{Notepad++} als Text-Editor.
	\\\tourl{https://notepad-plus-plus.org/}
	--- Lizenz \citesee{bib:GPLi}

	\item\label{Werkzeug:WinMerge}\emph{WinMerge} zum Vergleich von Dateien und Verzeichnissen.
	\\\tourl{http://winmerge.org/}
	--- Lizenz \citesee{bib:GPLi}

	\setcounter{Enumi}{\value{enumi}}% Nummerierung wird fortgesetzt.
\end{enumerate}

\color{gray}%%% Anfang grauer Text ---------------------------------------------
\paragraph{Im Projekt \emph{qedeq} verwendete Werkzeuge}% ----------------------

\begin{itemize}
	\setcounter{enumi}{\value{Enumi}}% Nummerierung wird fortgesetzt.

	\item\label{Werkzeug:Java}\emph{Java} als Programmiersprache und Laufzeitumgebung.
	\\\tourl{https://www.java.com/de/download/win10.jsp}
	--- Lizenz \citesee{bib:JavaSE}

	\item\label{Werkzeug:Apache Ant}\emph{Apache Ant} als Java Bibliothek und Kommandozeilen-Werkzeug
	um Java Programme zu erzeugen.
	\\\tourl{http://ant.apache.org/}
	--- Lizenz \citesee{bib:Apacheii}

	\item\label{Werkzeug:Checkstyle}\emph{Checkstyle} zur statischen Code-Analyse für Java.
	\\\tourl{http://checkstyle.sourceforge.net/}
	--- Lizenz \citesee{bib:LGPLii}

	\item\label{Werkzeug:Clover}\emph{Clover}%
	\footnote{%
		Clover ist proprietäre Software, aber auf Anfrage frei für 30 Tage.
		Danach ist eine einmalige Lizenzgebühr fällig.
	}
	als Testwerkzeug zur Analyse der Code-Abdeckung.
	\\\tourl{https://www.atlassian.com/software/clover/}
	--- Lizenz \citesee{bib:Clover}

	\item\label{Werkzeug:Eclipse Java}\emph{Eclipse IDE for Java Developers} als Entwicklungsumgebung für Java.
	\\\tourl{http://www.eclipse.org/downloads/packages/eclipse-ide-java-developers/neon1a/}
	--- Lizenz \citesee{bib:OSI}

	\item\label{Werkzeug:JUnit}\emph{JUnit} zur Erzeugung von wiederholbaren Tests.
	\\\tourl{http://junit.org/junit4/}
	--- Lizenz \citesee{bib:EPL}

	\item\label{Werkzeug:Xerces2}\emph{Xerces2} als XML-Parser in Java.
	\\\tourl{http://xerces.apache.org/xerces2-j/}
	--- Lizenzen \citesee{bib:Apacheii, bib:SAX, bib:WDCDL, bib:WDCSNL}
\end{itemize}
\color{black}%%% Ende  grauer Text ---------------------------------------------

\section     [Die Struktur ausgewählter Begriffe]{Die Struktur ausgewählter \Begriffe}
\beginsection{Die Struktur ausgewählter Begriffe}
\label   {sec:Begriffsstruktur}

\begin{table}[H]
	\centering
	\begin{threeparttable}
		\setlength\extrarowheight{3pt}
		\begin{tabularx}{\linewidth}{c@{\extracolsep{\fill}}|c|c|c|c|}
			\hline% ------------------------------------------------------------
			\multicolumn{5}{c|}{\textbf{\Objekt}}\Tnote{1}
			\\
			\hline% ------------------------------------------------------------
		\end{tabularx}
		\begin{tablenotes}
			\footnotesize
			\item[1] Fußnote zur Tabelle
		\end{tablenotes}
	\end{threeparttable}
	\caption{\Bezeichnungen}
	\label{tab:Objekte}% Erst nach '\caption'!
\end{table}

\begin{table}[H]
	\begin{threeparttable}
		\setlength\extrarowheight{3pt}
		\begin{tabularx}{\linewidth}{c@{\extracolsep{\fill}}|c|c|c|c|}
			\hline% ------------------------------------------------------------
			\multicolumn{3}{c|}{\textbf{\Metasprache}}&
			\multicolumn{2}{c|}{\textbf{\Objektsprache}}
			\\
			\textbf{natürliche Sprache} & \multicolumn{2}{c|}{\textbf{\formaleMetasprache}}
			& \textbf{\Aussagenlogik} & \textbf{\Praedikatenlogik}
			\\
			\hline% ------------------------------------------------------------
			& \multicolumn{4}{c|}{\Symbole}
			\\
			& \multicolumn{2}{c|}{\Metasymbol}
			& \multicolumn{2}{c|}{\Objektsymbol}
			\\
			\hline% ------------------------------------------------------------
			& \multicolumn{4}{c|}{Beispielsymbole}
			\\
			\unaere\  \Operation
			& \multicolumn{4}{c|}{\BspOpU}
			\\
			\binaere\ \Operation
			& \multicolumn{4}{c|}{\BspOpB}
			\\
			\binaere\ \Relationen
			& \multicolumn{4}{c|}{\BspRel  \quad \BspRelEq  \quad \BspRelBck  \quad \BspRelBckEq \quad \BspRelN \quad \BspRelEqN \quad \BspRelBckN \quad \BspRelBckEqN}
			\\
			\hline% ------------------------------------------------------------
			\multicolumn{5}{c|}{\Wahrheitswerte}
			\\
			~                    \TxtTrue \quad \TxtFalse
			&\multicolumn{2}{c|}{\MtsTrue \quad \MtsFalse}
			&\multicolumn{2}{c|}{\OjkTrue \quad \OjkFalse}
			\\
			\hline% ------------------------------------------------------------
			& \multicolumn{4}{c|}{\Operation \quad \Relation \quad \Umkehrrelation \quad \Negation}
			\\
			& \multicolumn{2}{c|}{\Metaoperation \quad \Metarelation}
			& \multicolumn{2}{c|}{\Junktor}
			\\
			\hline% ------------------------------------------------------------
			~                       nicht
			& \multicolumn{2}{c|}{\MtsNot}
			& \multicolumn{2}{c|}{\OjkNot}
			\\
			~                         und \quad   oder \quad    dann
			& \multicolumn{2}{c|}{\MtsAnd \quad \MtsOr \quad \MtsImp}
			& \multicolumn{2}{c|}{\OjkAnd \quad \OjkOr \quad \OjkImp}
			\\
			~                     dann wenn \quad    wenn
			& \multicolumn{2}{c|}{\MtsEquiv \quad \MtsRep}
			& \multicolumn{2}{c|}{\OjkEquiv \quad \OjkRep}
			\\
			~                         und\Tnote{1} \quad entweder oder
			& \multicolumn{2}{c|}{\MtsUnd}
			& \multicolumn{2}{c|}{                             \OjkXor}
			\\
			~                    nicht und \quad nicht oder
			& \multicolumn{2}{c|}{ }
			& \multicolumn{2}{c|}{\OjkNand \quad \OjkNor}
			\\
			\hline% ------------------------------------------------------------
			~                     gleich \quad ungleich
			& \multicolumn{2}{c|}{\MtsEq \quad \MtsEqN}
			& \multicolumn{2}{c|}{\OjkEq \quad \OjkEqN}
			\\
			definitionsgemäß            gleich
			& \multicolumn{2}{c|}{\MtsDefEquiv}
			& \multicolumn{2}{c|}{ }
			\\
			definitionsgemäß         gleich
			& \multicolumn{2}{c|}{\MtsDefEq}
			& \multicolumn{2}{c|}{ }
			\\
			\hline% ------------------------------------------------------------
			\Quantoren
			& \multicolumn{2}{c|}{$\MtsForall \quad \MtsExists \quad \MtsExione$}
			&                   & $\OjkForall \quad \OjkExists \quad \OjkExione$
			\\
			\hline% ------------------------------------------------------------
			\Ersetzung \quad \Vertauschung
			& \multicolumn{2}{c|}{\MtsSubst \quad \MtsSwap}
			& \multicolumn{2}{c|}{ }
			\\
			\Ableitungsrelationen:
			& \multicolumn{2}{c|}{\MtsDerive \quad \MtsDeriveR \quad \MtsPraemisseRel \quad \MtsKonklusionRel \quad \MtsErgebnisRel}
			& \multicolumn{2}{c|}{ }
			\\
			\hline% ------------------------------------------------------------
			\Elementrelationen:
			& \multicolumn{2}{c|}{\MtsIn \quad \MtsNi \quad \MtsInN \quad \MtsNiN}
			& \multicolumn{2}{c|}{ }
			\\
			\Mengenrelationen:
			& \multicolumn{2}{c|}{\MtsSubset \quad \MtsSubsetEq \quad \MtsSubsetN \quad \MtsSubsetEqN \quad \MtsSupset \quad \MtsSupsetEq \quad \MtsSupsetN \quad \MtsSupsetEqN}
			& \multicolumn{2}{c|}{ }
			\\
			\Komponentenrelationen:
			& \multicolumn{2}{c|}{\MtsSeqIn \quad \MtsSeqNi \quad \MtsSeqInN \quad \MtsSeqNiN}
			& \multicolumn{2}{c|}{ }
			\\
			\Folgenrelationen:
			& \multicolumn{2}{c|}{\MtsSubseq \quad \MtsSubseqEq \quad \MtsSubseqN \quad \MtsSubseqEqN \quad \MtsSupseq \quad \MtsSupseqEq \quad \MtsSupseqN \quad \MtsSupseqEqN}
			& \multicolumn{2}{c|}{ }
			\\
			ausgewählte Mengen
			& \multicolumn{2}{c|}{\MtsIN \quad \MtsINo \quad \MtsUniversum \quad \Sprache }
			& \multicolumn{2}{c|}{ }
			\\
			\hline% ------------------------------------------------------------
			& \textbf{\unaer} & \textbf{\binaer}
			& \multicolumn{2}{c|}{ }
			\\
			\Mengenoperationen
			& \MtsPot \quad \MtsPotf \quad \MtsRel \quad \MtsRelf & \MtsCap \quad \MtsCup \quad \MtsSetminus \quad \MtsTimes
			& \multicolumn{2}{c|}{ }
			\\
			& \MtsFol \quad \MtsFolf \quad \MtsTup &
			& \multicolumn{2}{c|}{ }
			\\
			\hline% ------------------------------------------------------------
			\unaere\ \Operationen\ auf:
			& \textbf{\Relationen} & \textbf{\Funktionen}
			& \multicolumn{2}{c|}{ }
			\\
			& \MtsStelR            & \MtsStelF
			& \multicolumn{2}{c|}{ }
			\\
			\DefinitionsB- \quad \Zielbereich
			&                      & \MtsDb \quad \MtsZb
			& \multicolumn{2}{c|}{ }
			\\
			\QuellB- \quad \Wertebereich
			&                      & \MtsQb \quad \MtsWb
			& \multicolumn{2}{c|}{ }
			\\
			\Traegermenge
			& \multicolumn{2}{c|}{$\MtsTraeger \quad \MtsTraeger_i$}
			& \multicolumn{2}{c|}{ }
			\\
			\Graph
			& \multicolumn{2}{c|}{ \MtsGraph }
			& \multicolumn{2}{c|}{ }
			\\
			\hline% ------------------------------------------------------------
			\unaere\ \Operationen\ auf:
			& \multicolumn{2}{c|}{ \Folgen \quad \Tupel }
			& \multicolumn{2}{c|}{ }
			\\
			& \multicolumn{2}{c|}{ \MtsLen \quad \MtsSet }
			& \multicolumn{2}{c|}{ }
			\\
			\hline% ------------------------------------------------------------
		\end{tabularx}
		\begin{tablenotes}
			\footnotesize
			\item[] Die erste Spalte beschreibt die anderen Spalten.
			Die \textbf{fettgedruckten} Teile, und nur diese, gelten als Überschriften.
			\item[1] nur in Schlussregeln
		\end{tablenotes}
	\end{threeparttable}
	\caption{Ausgewählte \Bezeichnungen}
	\label{tab:Benennungen}% Erst nach '\caption'!
\end{table}


\section     {Offene Aufgaben}% ================================================
\beginsection{Offene Aufgaben}
\label   {sec:OffeneAufgaben}

\begin{enumerate}
	\item TODOs bearbeiten.
	\item Eingabeprogramm erstellen (liest XML).
	\item Prüfprogramm erstellen.
	\item Ausgabeprogramm erstellen (schreibt XML).
	\item Formelausgabe erstellen (erzeugt \LaTeX{} aus XML).
	\item \Axiome\ sammeln und eingeben.
	\item \Saetze\ sammeln und eingeben.
	\item \Beweise\ sammeln und eingeben.
	\item \Fachbegriffe\ und Symbole sammeln und eingeben.
	\item \Fachgebiete\ sammeln und eingeben.
	\item \Ausgabeschemata\ sammeln und eingeben.
\end{enumerate}

\Endchapter


	\chapter     {Verzeichnisse}% ##############################################
	\beginchapter{Verzeichnisse}
	\label   {app-Verzeichnisse}

	%section{Tabellenverzeichnis}% =============================================
	\phantomsection% sichert korrekten Link im Inhaltsverzeichnis
	\label{cha:Tabellenverzeichnis}
	\likesection{\listtablename}
	\begin{minipage}{\linewidth-10.95pt}
		\listoftables
	\end{minipage}
	\Endchapter

	%section{Abbildungsverzeichnis}% ===========================================
	\phantomsection% sichert korrekten Link im Inhaltsverzeichnis
	\label{cha:Abbildungsverzeichnis}
	\likesection{\listfigurename}
	\begin{minipage}{\linewidth-10.95pt}
		\listoffigures
	\end{minipage}
	\Endchapter

	%%############################################################################%%
%%                                                                            %%
%% Datei:  ASBA-Literaturverzeichnis.tex                                      %%
%% Inhalt: Literaturverzeichnis                                               %%
%%                                                                            %%
%% Copyright (C) 2017  Winfried Teschers                                      %%
%%                                                                            %%
%% This program is free software: you can redistribute it and/or modify       %%
%% it under the terms of the GNU Affero General Public License as published   %%
%% by the Free Software Foundation, either version 3 of the License, or       %%
%% (at your option) any later version.                                        %%
%%                                                                            %%
%% This program is distributed in the hope that it will be useful,            %%
%% but WITHOUT ANY WARRANTY; without even the implied warranty of             %%
%% MERCHANTABILITY or FITNESS FOR A PARTICULAR PURPOSE.  See the              %%
%% GNU Affero General Public License for more details.                        %%
%%                                                                            %%
%% You should have received a copy of the GNU Affero General Public License   %%
%% along with this program.  If not, see <http://www.gnu.org/licenses/>.      %%
%%                                                                            %%
%% Dr. Winfried Teschers                                                      %%
%% Anton-Günther-Straße 26c                                                   %%
%% 91083 Baiersdorf                                                           %%
%% Germany                                                                    %%
%%                                                                            %%
%% e-mail: winfried.teschers@t-online.de                                      %%
%%                                                                            %%
%%############################################################################%%

% !TeX root = ASBA.tex
% !TeX encoding = UTF-8
% !TeX spellcheck = de_DE

%chapter{Literaturverzeichnis}% ################################################

\begin{flushleft}
	\begin{thebibliography}{12}
		\likechapter[section]{\bibname}  % erst hier!
		\label{dic:Literaturverzeichnis} % erst hier!

		\bibitem{bib:Rautenberg}Wolfgang Rautenberg,
		\emph{Einführung in die Mathematische Logik}:
		Ein Lehrbuch, 3.\@ Auflage, Vieweg+Teubner 2008

		\bibitem{bib:Apacheii}\emph{Apache License}, Version 2.0
		$\rightarrow$%
		\footnote{%
			Der Pfeil~($\rightarrow$) verweist stets auf einen Link zu einer Seite im Internet.
		}
		\url{http://www.apache.org/licenses/LICENSE-2.0}
		01.2004%
		\footnote{%
			Das Datum hinter dem Link gibt -- je nachdem welches bekannt ist -- das Datum der letzten Änderung, den Stand der Seite oder das Datum, an dem die Seite angeschaut wurde an.
			Sind mehrere Daten vorhanden, wird das erste vorhandene in der angegebenen Reihenfolge genommen.
			-- Dies gilt für alle hier aufgelisteten Seiten im Internet.
		}

		\bibitem{bib:BSLi}\emph{Boost Software License} 1.0
		\tourl{http://www.boost.org/users/license.html}
		17.08.2003

		\bibitem{bib:EPL}\emph{Eclipse Public License} Version 1.0
		\tourl{http://www.eclipse.org/org/documents/epl-v10.php}
		09.03.2017

		\bibitem{bib:AGPL}\emph{GNU Affero General Public License}
		\tourl{http://www.gnu.org/licenses/agpl}
		19.11.2007

		\bibitem{bib:GPLi}\emph{GNU General Public License}
		\tourl{http://www.gnu.org/licenses/old-licenses/gpl-1.0}
		02.1989

		\bibitem{bib:GPLii}\emph{GNU General Public License}, Version 2
		\tourl{http://www.gnu.org/licenses/old-licenses/gpl-2.0}
		06.1991

		\bibitem{bib:LGPLii}\emph{GNU Lesser General Public License},
		Version 2.1
		\tourl{http://www.gnu.org/licenses/old-licenses/lgpl-2.1}
		02.1999

		\bibitem{bib:Clover}Lizenz für \emph{Clover}
		\tourl{https://www.atlassian.com/software/clover}
		2017

		\bibitem{bib:EULA}Lizenz
		für \emph{Microsoft Visual Studio Express 2015}
		\tourl{https://www.visualstudio.com/de/license-terms/mt171551/}
		2017

		\bibitem{bib:MiKTeX}Lizenz für \emph{MikTeX}
		\tourl{https://miktex.org/kb/copying}
		13.04.2017

		\bibitem{bib:SAX}Lizenz für \emph{SAX}
		\tourl{http://www.saxproject.org/copying.html}
		05.05.2000

		\bibitem{bib:MIT}\emph{MIT License}
		\tourl{https://opensource.org/licenses/MIT/}
		09.03.2017

		\bibitem{bib:JavaSE}\emph{Oracle Binary Code License Agreement}
		\tourl{http://java.com/license}
		02.04.2013

		\bibitem{bib:OSI}\emph{OSI Certified Open Source Software}
		\tourl{https://opensource.org/pressreleases/certified-open-source.php}
		16.06.1999

		\bibitem{bib:WDCDL}\emph{W3C Document License}
		\tourl{http://www.w3.org/Consortium/Legal/2015/doc-license}
		01.02.2015

		\bibitem{bib:WDCSNL}\emph{W3C Software Notice and License}
		\tourl{http://www.w3.org/Consortium/Legal/2002/copyright-software-20021231.html}
		13.05.2015

		\bibitem{bib:HilbertII}\emph{Hilbert II -- Introduction}
		\tourl{http://www.qedeq.org/}
		20.01.2014

		\bibitem{bib:qedeq}\emph{Formal Correct Mathematical Knowledge}:
		GitHub Repository vom Projekt Hilbert II
		\tourl{https://github.com/m-31/qedeq/}
		18.03.2017

		\bibitem{bib:ASBA}\emph{ASBA -- Axiome, Sätze, Beweise und Auswertungen}.
		Projekt zur maschinellen Überprüfung von mathematischen Beweisen
		und deren Ausgabe in lesbarer Form:
		GitHub Repository vom Projekt ASBA
		-- in Bearbeitung
		\tourl{https://github.com/Dr-Winfried/ASBA}

		\bibitem{bib:LogikDe}Meyling, Michael:
		\emph{Anfangsgründe der mathematischen Logik}
		\tourl{http://www.qedeq.org/current/doc/math/qedeq\_logic\_v1\_de.pdf}
		24.~Mai~2013 (in Bearbeitung)

		\bibitem{bib:PraedikatenlogikDe}Meyling, Michael:
		\emph{Formale Prädikatenlogik}
		\tourl{http://www.qedeq.org/current/doc/math/qedeq\_formal\_logic\_v1\_de.pdf}
		24.~Mai~2013 (in Bearbeitung)

		\bibitem{bib:MengenlehreDe}Meyling, Michael:
		\emph{Axiomatische Mengenlehre}
		\tourl{http://www.qedeq.org/current/doc/math/qedeq\_set\_theory\_v1\_de.pdf}
		24.~Mai~2013 (in Bearbeitung)

		\bibitem{bib:LogikEn}Meyling, Michael:
		\emph{Elements of Mathematical Logic}
		\tourl{http://www.qedeq.org/current/doc/math/qedeq\_logic\_v1\_en.pdf}
		24.~Mai~2013 (in Bearbeitung)

		\bibitem{bib:PraedikatenlogikEn}Meyling, Michael:
		\emph{Formal Predicate Calculus}
		\tourl{http://www.qedeq.org/current/doc/math/qedeq\_formal\_logic\_v1\_en.pdf}
		24.~Mai~2013 (in Bearbeitung)

		\bibitem{bib:MengenlehreEn}Meyling, Michael:
		\emph{Axiomatic Set Theory}
		\tourl{http://www.qedeq.org/current/doc/math/qedeq\_set\_theory\_v1\_en.pdf}
		24.~Mai~2013 (in Bearbeitung)

		\bibitem{bib:Aussagenlogik}Wikipedia:
		\emph{Aussagenlogik} \chaptername~4 \emph{Formaler Zugang}
		\tourl{https://de.wikipedia.org/wiki/Aussagenlogik\#Formaler\_Zugang}
		18.01.2018

		\bibitem{bib:Funktion}Wikipedia:
		\emph{Funktion (Mathematik)} \chaptername~2.1 \emph{Mengentheoretische Definition}
		\tourl{https://de.wikipedia.org/wiki/Funktion\_(Mathematik)\#Mengentheoretische\_Definition}
		27.01.2018

		\bibitem{bib:HilbertKalkuel}Wikipedia:
		\emph{Hilbert-Kalkül} \chaptername~1.4 \emph{Modus (ponendo) ponens}
		\tourl{https://de.wikipedia.org/wiki/Hilbert-Kalk\%C3\%BCl\#Modus\_(ponendo)\_ponens}
		18.06.16

		\bibitem{bib:Identitaet}Wikipedia:
		\emph{Identität (Logik)} \chaptername~2.3 \emph{Identität in der Informatik}
		\tourl{https://de.wikipedia.org/wiki/Identit\%C3\%A4t\_(Logik)\#Identit.C3.A4t\_in\_der\_Informatik}
		18.05.2017

		\bibitem{bib:Junktor}Wikipedia:
		\emph{Junktor} \chaptername~2.2 \emph{Mögliche Junktoren}
		\tourl{https://de.wikipedia.org/wiki/Junktor\#M.C3.B6gliche\_Junktoren}
		21.10.2017

		\bibitem{bib:Kalkuel}Wikipedia:
		\emph{Kalkül}
		\tourl{https://de.wikipedia.org/wiki/Kalk\%C3\%BCl}
		26.02.2017

		\bibitem{bib:Mengenlehre}Wikipedia:
		\emph{Mengenlehre}
		\tourl{https://de.wikipedia.org/wiki/Mengenlehre}
		17.01.2018

		\bibitem{bib:Praedikatenlogik}Wikipedia:
		\emph{Prädikatenlogik erster Stufe}
		\tourl{https://de.wikipedia.org/wiki/Pr\%C3\%A4dikatenlogik\_erster\_Stufe}
		26.11.2017

		\bibitem{bib:Relation}Wikipedia:
		\emph{Relation (Mathematik)} \chaptername~1.1 \emph{Mehrstellige Relation}
		\tourl{https://de.wikipedia.org/wiki/Relation\_(Mathematik)\#Mehrstellige\_Relation}
		27.01.2018

		\bibitem{bib:Schlussregel}Wikipedia:
		\emph{Schlussregel}
		\tourl{https://de.wikipedia.org/wiki/Schlussregel}
		29.03.2015

		\bibitem{bib:NatuerlichesSchliessen}Wikipedia:
		\emph{Systeme natürlichen Schließens}
		\tourl{https://de.wikipedia.org/wiki/Systeme\_nat\%C3\%BCrlichen\_Schlie\%C3\%9Fens}
		25.10.2017

		\clearpage % schon hier!
	\end{thebibliography}
\end{flushleft}



	% Glossar ==================================================================
	% Optionen -> TeXstudio konfigurieren -> Erzeugen -> Standardcompiler:
	%     "txs:///pdflatex|txs:///makeglossaries|txs:///pdflatex"
	\newcommand{\glopreambelEinordnung}{
		Die Einordnung von einem Substantiv mit Adjektiven erfolgt stets unter dem Substantiv.
		Ist das Substantiv und \textggf\ ein oder mehrere Adjektive schon vorher aufgelistet worden, werden diese Worte durch je ein "`---"' ersetzt.
		\par
	}
	\newcommand{\glopreambelSchriftarten}{
		Mit Seitenzahlen \likeHyperDef{in dieser} Schriftart wird auf die Definition oder sonstige wichtige Stellen verwiesen.
		%%%--- Aus dem Autor nicht bekannten Gründen sind die angegebenen Seitenzahlen manchmal um eins zu klein.
		\par
	}
	\newcommand{\glopreambelSymbolbeschreibung}{
		In eckigen Klammern wird wird, sofern vorhanden, die \Benennung\ für das jeweilige \Symbol\ angegeben, für \Funktionen\ und \Relationen\ auch mittels eines Aufrufs.
		Die Wörter \ManFt{in dieser} Schrift sind notwendig für die Definition, solche \OptFt{in dieser} Schrift können auch weggelassen werden.
		\par
	}
	\newcommand{\glopreambelWikipedia}{
		Vielfach ist hier der erste Abschnitt%
		\footnote{%
			Der Teil zwischen Überschrift und Inhaltsverzeichnis.
		} aus dem entsprechenden \Wikipedia-Artikel zitiert, manchmal gekürzt und immer ohne die originalen Fußnoten und ohne Verweise auf andere \Wikipedia-Artikel.
		Letztere werden allerdings noch, wie im Original, in \wikiLinkFt{blau} angegeben.
		\par
	}
	\iftestFlg \glsaddallunused[symbols,main]\else\fi% nicht verwendete Glossareinträge hinzufügen
	% --------------------------------------------------------------------------
	\renewcommand{\glossarypreamble}{%  am Beginn des Index
		\beginchapter[Verzeichnisse]{Index}
		\label                  {dic:Index}
		\glopreambelEinordnung
	}
	\renewcommand*{\glsnamefont}[1]{\textmd{#1}}
	\printindex[style=mcolindexspannav]
	\Chead{Index}%                      Kopfzeile Mitte
	\Endchapter
	\newpage% ------------------------------------------------------------------
	\renewcommand{\glossarypreamble}{%  am Beginn des Symbolverzeichnisses
		\beginchapter[Verzeichnisse]{Symbolverzeichnis}
		\label                  {dic:Symbolverzeichnis}
		\glopreambelSchriftarten
		\glopreambelSymbolbeschreibung
	}
	\renewcommand*{\glsnamefont}[1]{#1}
	\printglossary[type=symbols,style=list]
	\Chead{Symbolverzeichnis}%          Kopfzeile Mitte
	\Endchapter
	\newpage% ------------------------------------------------------------------
	\renewcommand{\glossarypreamble}{%  am Beginn des Glossars
		\beginchapter[Verzeichnisse]{Glossar}
		\label                  {dic:Glossar}
		\glopreambelEinordnung
		\glopreambelSchriftarten
		\glopreambelWikipedia
	}
	\renewcommand*{\glsnamefont}[1]{\DefFt{#1}}
	\printglossary[type=main,style=listhypergroup]
	\Chead{Glossar}%                    Kopfzeile Mitte
	\Endchapter
	% --------------------------------------------------------------------------

\end{document}

% Ende des Dokuments ###########################################################
