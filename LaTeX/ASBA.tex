% !TeX root = ASBA.tex
% !TeX encoding = UTF-8
% !TeX spellcheck = de_DE

%% Datei ASBA.tex zur Erzeugung des Projektdokuments von ASBA.
%%
%% Copyright (C) 2017  Winfried Teschers
%%
%% This program is free software: you can redistribute it and/or modify
%% it under the terms of the GNU Affero General Public License as published
%% by the Free Software Foundation, either version 3 of the License, or
%% (at your option) any later version.
%%
%% This program is distributed in the hope that it will be useful,
%% but WITHOUT ANY WARRANTY; without even the implied warranty of
%% MERCHANTABILITY or FITNESS FOR A PARTICULAR PURPOSE.  See the
%% GNU Affero General Public License for more details.
%%
%% You should have received a copy of the GNU Affero General Public License
%% along with this program.  If not, see <http://www.gnu.org/licenses/>.
%%
%% Dr. Winfried Teschers
%% Anton-Günther-Str. 26c
%% 91083 Baiersdorf
%% Germany
%%
%% e-mail: winfried.teschers@t-online.de

\documentclass[english,ngerman,parskip=half,headsepline,footsepline,
	fleqn,notitlepage]{scrreprt}

%%%% Pakete %%%%%%%%%%%%%%%%%%%%%%%%%%%%%%%%%%%%%%%%%%%%%%%%%%%%%%%%%%%%%%%%%%%%

% allgemein --------------------------------------------------------------------
\usepackage[utf8]{inputenc}% Input encoding specification
\usepackage[T1]{fontenc}
\usepackage{lmodern}
\usepackage{scrlayer-scrpage}
\usepackage{geometry}% Flexible and complete interface to document dimensions.
\usepackage{microtype}% Subliminal refinements towards typographical perfection.
\usepackage{graphicx}% Alternative interface to graphics functions.
\usepackage{pict2e}% New implementation of picture commands.
\usepackage{multicol}% An environment for multicolumn output
\usepackage{babel}% Multilingual support for plain TeX or LaTeX.
\usepackage[autostyle]{csquotes}% Contex sensitive quotation facilities

% mathematische Pakete ---------------------------------------------------------
\usepackage{amsmath}% Mathematical facilities for LaLeX from ASM
\usepackage{amsfonts}% TeX fonts from the American Mathematical Society.
\usepackage{amssymb}% Symbols from the American Mathematical Society.
\usepackage{mathtools}% Mathematical tools to use with asmmath.
\usepackage{mathabx}% Three series of mathematical symbols.
\usepackage{mathpazo}% Fonts to typeset mathematics to match palatino.
%\usepackage{cancel}% Place lines through maths formulae.

% Tabellen ---------------------------------------------------------------------
\usepackage[table]{xcolor}% Driver-independent color extensions - vor 'color'?
%\usepackage{ctable}% Flexible typesetting of table and figure using key/value
%% Das Paket ctable fast die Eigenschaften der Pakete
\usepackage{array}%
\usepackage{tabularx}% Erweiterung von tabular*
\usepackage{booktabs}% Nicer layout of tables
%% zusammen und lädt zusätzlich noch die Pakete
\usepackage{rotating}% Rotating tools, including rotated full page floats
\usepackage{xspace}% Behandelt Zwischenraum nach Makros
\usepackage{color}% LaTeX support for color
\usepackage{xkeyval}% Extension of the keyval package
% Ende der von 'ctable' geladenen Pakete
\usepackage{threeparttable}% Tables with captions and notes all the same width.
\usepackage{multirow}% Create tabular cell spanning multiple rows.
\usepackage{diagbox}% Table heads with diagonal lines.
\usepackage{arydshln}% Draw dash-lines in array/tabular.
\usepackage{caption}% Customizing captions in floating environments.

% Indizes ----------------------------------------------------------------------
%\usepackage{makeidx}% Indexing - Entweder 'makeidx' oder 'splitidx'
\usepackage[protected,idxcommands]{splitidx}% mehrere Indizes - statt 'makeidx'
%\usepackage{hvindex}% Support for indexing - after 'babel
%\usepackage{showidx}% Index auf Seitenrand anzeigen - Zum Testen der Indizes
% TODO Fehler: 'showindex' gibt direkt und nicht auf Rand aus %##
\usepackage{glossaries}% Create glossaries and lists of acronyms
%\usepackage{glossaries-german}% German language module for glossaries package

% Verweise ---------------------------------------------------------------------
%\usepackage[germanb]{minitoc}\dosectoc% Unterverzeichnisse erstellen
% TODO Fehler: 'minitoc' funktioniert nicht
\usepackage{varioref}
\usepackage[colorlinks,linktoc=all]{hyperref}% Extensive support for hypertext
\usepackage{glossaries}% Create glossaries and lists of acronyms
% lädt     {glossaries-german}% German language module for glossaries package

%%% Einstellung von globalen Werten und Makro-Redefinitionen %%%%%%%%%%%%%%%%%%%

\geometry{textwidth=170mm,textheight=256mm,twoside}% optional Option 'showframe'

% Kopfzeilen ===================================================================
\newcommand*{\texthead}[1]{\textnormal{\textsf{\textbf{#1}}}}% Schriftart
\newcommand*{\Lehead}[1]{\lehead{\texthead{#1}}}
\newcommand*{\Cehead}[1]{\cehead{\texthead{#1}}}
\newcommand*{\Rehead}[1]{\rehead{\texthead{#1}}}
\newcommand*{\Lohead}[1]{\lohead{\texthead{#1}}}
\newcommand*{\Cohead}[1]{\cohead{\texthead{#1}}}
\newcommand*{\Rohead}[1]{\rohead{\texthead{#1}}}
\newcommand*{\Ohead}[1]{\ohead{\texthead{#1}}}
\newcommand*{\Chead}[1]{\chead{\texthead{#1}}}
\newcommand*{\Ihead}[1]{\ihead{\texthead{#1}}}
\newcommand*{\Ofoot}[1]{\ofoot{\textnormal{\textbf{#1}}}}
\newcommand*{\Cfoot}[1]{\cfoot{\textnormal{#1}}}
\newcommand*{\Ifoot}[1]{\ifoot{\textnormal{#1}}}
\newcommand*{\Pagestyle}{\pagestyle{scrheadings}}
\newcommand*{\Thispagestyle}{\thispagestyle{scrheadings}}

% Kopfzeilen mit 'scrlayer-scrpage'
%         \Lehead \Cehead \Rehead | \Lohead \Cohead \Rohead
% \Ohead: \Lehead                                   \Rohead
% \Chead:         \Cehead                   \Cohead
% \Ihead:                 \Rehead   \Lohead
% ASBA <Chapter-Überschrift> \Chaptername~\thechapter
%                            \sectionname~\thesection <Section-Überschrift> ASBA
%Initialisierung
\Ohead{ASBA}%      bleibt unverändert
\Chead{Copyright}% wird laufend verändert
\Ihead{}%          wird laufend verändert

% Kapitel ======================================================================
\newcommand*{\Chaptername}{\chaptername}% wird mit 'Anhang' überschrieben

\newcommand*{\beforechapter}{% direkt vor \chapter (auch im Kommentar)
	\Thispagestyle%        Kopfzeile für diese Seite aktivieren - vor \clearpage
	\clearpage%            neue Seite
}
\newcommand*{\beginchapter}[1]{%        direkt nach \chapter
	\Chead{#1}%                         Kopfzeile Mitte = <Kapitelname>
	\Ihead{\Chaptername~\thechapter}%   Kopfzeile Innen = Kapitel/Anhang <Nr.>
	\Pagestyle%                         veränderte Kopfzeile aktivieren
	\Thispagestyle%                     ... auch für diese Seite, da ...
}%                                      '\chapter' Kopf-/Fußzeilen deaktiviert
\newcommand*{\likechapter}[2][chapter]{%statt \beginchapter für Inhalts-,
	%                                   Tabellen- und Abbildungsverzeichnis
	\Chead{#2}%                         Mitte in der Kopfzeile = <Kapitelname>
	\Ihead{}%                           Innen in der Kopfzeile = <leer>
	\Pagestyle%                         veränderte Kopfzeile aktivieren ...
	\Thispagestyle%                     sicherheitshalber auch für diese Seite.
	\addcontentsline{toc}{#1}{#2}%      Eintrag ins Inhaltsverzeichnis
	%TODO Fehler: 2. Seite Kopf Mitte für Lit.-Verz. = Abildungsverzeichnis %##
	%TODO Fehler: Links vom Inhaltsverz. nach Tab.-Verz. 1 Seite zu früh    %##
}
% Indizes ======================================================================
\newcommand*{\idxdictionary}[1]{%       nur für Indices
	\Thispagestyle%       Kopfzeile für diese Seite aktivieren - vor \clearpage
	\clearpage%                         neue Seite
	\extendtheindex{}{%                 aktiviert die Kopfzeile für Index-Seiten
		\Chead{#1}%                     Mitte in der Kopfzeile = <Kapitelname>
		\Ihead{}%                       Innen in der Kopfzeile = <leer>
		\Pagestyle%                     veränderte Kopfzeile aktivieren
		\Thispagestyle%                 ... auch für diese Seite
		\addcontentsline{toc}{section}{#1}% Eintrag ins Inhaltsverzeichnis
	}{}{}
}
\newcommand*{\glodictionary}[1]{%       nur für Glossary
	\Chead{#1}%                         Mitte in der Kopfzeile = <Kapitelname>
	\Ihead{}%                           Innen in der Kopfzeile = <leer>
	\Pagestyle%                         veränderte Kopfzeile aktivieren
}

% Abschnitte ===================================================================
\newcommand*{\beginsection}[1]{%        direkt nach \section
	\Cohead{#1}%                        oben rechts mittig = <Abschnittsname>
	\Lohead{\sectionname~\thesection}%  oben rechts innen = <Abschnittsnummer>
	\Pagestyle%                         veränderte Kopfzeile aktivieren
}

% Fußzeilen ====================================================================
\Ofoot{\thepage}
\Cfoot{Winfried Teschers}
\Ifoot{\today}
\Pagestyle%                                       aktiviert Kopf- und Fußzeilen

% Fußnoten ---------------------------------------------------------------------
\deffootnote[10pt]% Markenbreite
{10pt}% Einzug - für Blocksatz: Markenbreite
{0pt}% Absatzeinzug für Folgeabsätze
{\makebox[9pt][r]{\textsuperscript{\thefootnotemark} }}% Zeichen; < Markenbreite

% Vordefinierte Werte ändern ===================================================
\setcounter{tocdepth}{3}%    Tiefe des Inhaltsverzeichnisses: 2 => subsection
\setcounter{secnumdepth}{3}% Nummerierung:                    3 => subsubsection
\setlength\extrarowheight{1pt}% Tabellenzellenhöhe vergrößern
\captionsetup{labelfont=bf}%    Tabellenbeschriftung in bf = bold font

% Empfehlung aus: Herbert Voß, LaTeX Referenz, 3. Auflage, Berlin 2014; S. 37f
\renewcommand{\floatpagefraction}{0.7}% Empfehlung: 0.5-0.8 Voreinstellung: 0.9
\renewcommand{\textfraction}{0.15}%                 0.1-0.3                 0.05
\renewcommand{\topfraction}{0.8}%                   0.5-0.85                0.9
\renewcommand{\bottomfraction}{0.5}%                0.2-0.5                 0.9
\setcounter{topnumber}{3}%                                                  2
\setcounter{totalnumber}{15}%                                               3

% Neue Elemente ----------------------------------------------------------------
\newcounter{Enumi}% für unterbrochene Listennummerierung

%% Bildelemente %%%%%%%%%%%%%%%%%%%%%%%%%%%%%%%%%%%%%%%%%%%%%%%%%%%%%%%%%%%%%%%%

\newcommand*{\textbild}[1]{\textbf{\textsf{#1}}}% Textauszeichnungen für Text im Bild
\newcommand*{\Datei}[4][0.5]{% #2 x #3 = (-#2/2,-#3/2),(#2/2,#3/2)
	% [Eck-Radius], Breite, Höhe, Name
	\put(0,0){\oval[#1](#2,#3)}
	\put(0,0){\makebox(0,0){\textbild{#4}}}
}
\newcommand*{\Datenbank}[5]{% 2(#1) x 2(#2+#3) = (-#1,-#2-#3),(+#1,+#2+#3)
	% Halbmesser x, Halbmesser y, halbe Höhe, Name - Ursprung in der Mitte
	\put(0.0,-#3){
		\qbezier(-#1,0.0)(-#1,-#2)(0.0,-#2)
		\qbezier(+#1,0.0)(+#1,-#2)(0.0,-#2)
	}
	\put(0,0){\Line(-#1,-#3)(-#1, #3)}
	\put(0,0){\Line( #1,-#3)( #1, #3)}
	\put(0.0,#3){
		\qbezier(-#1,0.0)(-#1,-#2)(0.0,-#2)
		\qbezier( #1,0.0)( #1,-#2)(0.0,-#2)
		\qbezier(-#1,0.0)(-#1, #2)(0.0, #2)
		\qbezier( #1,0.0)( #1, #2)(0.0, #2)
	}
	\makebox(0,0){\textbild{#4}}
	\makebox(0,-#3){\textbild{#5}}
}
% '\Männchen' (mit 'ä') führt zu Fehler
\newcommand*{\Maennchen}{% 1x2 = (-0.5,-1.7),(+0.5,+0.3) Ursprung im Kopf
	\put(0,0){\circle{0.6}}
	\Line(0.0,-0.3)(0.0,-1.2)
	\polyline(-0.5,-0.3)(0.0,-0.6)(0.5,-0.3)
	\polyline(-0.5,-1.7)(0.0,-1.2)(0.5,-1.7)
}
\newcommand*{\Marker}[2][0.5]{% 2x#1 x 2x#1 - Kreis mit Text
	{
		\linethickness{0.5pt}
		\color{white}
		\put(0,0){\circle*{#1}}
		\color{black}
		\put(0,0){\circle{#1}}
		\put(0,0){\makebox(0,0){\small\textbild{#2}}}
	}
}
\newcommand*{\marker}[2][0.5]{% 2x#1 x 2x#1 - Kreis mit Text - grau
	{
		\linethickness{0.5pt}
		\color{white}
		\put(0,0){\circle*{#1}}
		\color{gray}
		\put(0,0){\circle{#1}}
		\put(0,0){\makebox(0,0){\small\textbild{#2}}}
	}
}
\newcommand*{\Papier}[3]{% 3x#1+#2 = (0.0,-#2),(2.1,#1) Ursprung links unten
	% Länge (Höhe), Länge Abschluss, Name
	\polyline(0.0,-0.01)(+0.0,+#1)(+2.8,+#1)(+2.8,-0.01)
	\qbezier(+2.1,+#2)(+2.6,+#2)(+2.8,0.0)
	\qbezier(+2.1,+#2)(+1.6,+#2)(+1.4,0.0)
	\qbezier(+0.7,-#2)(+1.2,-#2)(+1.4,0.0)
	\qbezier(+0.7,-#2)(+0.2,-#2)(+0.0,0.0)
	\put(0,0){\makebox(+2.8,+#1){\textbild{#3}}}
}
\newcommand*{\Terminal}[1]{% 2x2 =(-1.0,-1.4),(+1.0,+0.6)Ursprung im Monitor
	% Bildschirm
	%		\put(0,0){\polygon(-1.0,-0.6)(+1.0,-0.6)(+1.0,+0.6)(-1.0,+0.6)}
	\put(-1.0,-0.6){\framebox(2,1.2){#1}}
	\put(0,0){\oval[0.1](1.65,0.85)}
	% Hals
	\put(0,0){\Line(-0.2,-0.6)(-0.2,-1.0)}
	\put(0,0){\Line(+0.2,-0.6)(+0.2,-1.0)}
	% Tastatur
	\multiput(-1.0,-1.0)(+0.0,-0.133){4}{\line(1,0){2.0}}
	\multiput(-1.0,-1,0)(+0.2,+0.0){11}{\line(0,-1){0.4}}
}
\newcommand*{\Wolke}[1]{% 3.0x1.5 = (-1.5,-1.0),(+1.5,+0.5)
	% unterer Bogen
	\qbezier(-1.5,+0.0)(-1.5,-1.0)(-0.0,-1.0)
	\qbezier(+1.5,+0.0)(+1.5,-1.0)(+0.0,-1.0)
	% oberer Bogen rechts
	\qbezier(+1.5,+0.0)(+1.5,+0.5)(+0.8,+0.5)
	\qbezier(+0.4,+0.4)(+0.4,+0.5)(+0.8,+0.5)
	% oberer Bogen Mitte
	\qbezier(+0.5,+0.2)(+0.5,+0.5)(+0.0,+0.5)
	\qbezier(-0.4,+0.4)(-0.4,+0.5)(-0.0,+0.5)
	% oberer Bogen links
	\qbezier(-0.3,+0.2)(-0.3,+0.5)(-0.8,+0.5)
	\qbezier(-1.5,+0.0)(-1.5,+0.5)(-0.8,+0.5)
	\put(-1.5,-1.0){\makebox(3.0,1.5){\textbild{#1}}}
}

%% Metasprachliche Symbole %%%%%%%%%%%%%%%%%%%%%%%%%%%%%%%%%%%%%%%%%%%%%%%%%%%%%

\newcommand*{\metaund}{\&\&}%              Und-Symbol  für Texte
\newcommand*{\metaoder}{||}%               Oder-Symbol für Texte
\newcommand*{\metaand}{\;\metaund\;}%      Und-Symbol  für Formeln
\newcommand*{\metaor}{\;\metaoder\;}%      Oder-Symbol für Formeln
% Nur im Mathematikmodus!
\newcommand*{\metaimp}{\Rightarrow}%       aus ... folgt ...
\newcommand*{\metarep}{\Leftarrow}%        ... folgt aus ...
\newcommand*{\metaequiv}{\Leftrightarrow}% ... genau dann wenn ...
\newcommand*{\defeq}{:=}%                  definitionsgemäß gleich
\newcommand*{\eq}{=}%                      Gleichheit

%% Mathematische Symbole %%%%%%%%%%%%%%%%%%%%%%%%%%%%%%%%%%%%%%%%%%%%%%%%%%%%%%%

% Beispieloperatoren ===========================================================
% \*bsp
\newcommand*{\opbsp}{\ast}
\newcommand*{\relbsp}{\sim}
\newcommand*{\releqbsp}{\simeq}
\newcommand*{\lrelbsp}{\lhd}
\newcommand*{\rrelbsp}{\rhd}
\newcommand*{\lreleqbsp}{\unlhd}
\newcommand*{\rreleqbsp}{\unrhd}

% Definitionen für die Tabelle der Junktoren ===================================
% \l*  -           logischer Operator
% \ln* - negierter logischer Operator
% Logische Operatoren als Addition und Multiplikation
\newcommand*{\ladd}{+}
\newcommand*{\lmult}{\cdot}
% Wahrheitswerte ---------------------------------------------------------------
\newcommand*{\texttrue}{W}%  in einem Kommentar stets 'W'
\newcommand*{\textfalse}{F}% in einem Kommentar stets 'F'
% Konstante --------------------------------------------------------------------
\newcommand*{\ltrue}{\top}%      W - wahr
%\newcommand*{\lnfalse}{\notbot}% " - nicht falsch
\newcommand*{\lfalse}{\bot}%     F - falsch
%\newcommand*{\lntrue}{\nottop}%  " - nicht wahr
% unäre Operatoren -------------------------------------------------------------
%                                                 W F - Aussage A
%\newcommand*{\lutrue}{\operatorname{\top}}%      W W - wahr [unär]
%\newcommand*{\lnufalse}{\operatorname{\notbot}}% " " - nicht falsch [unär]
%                                                 W F - A
%             \lnot                               F W - nicht
%\newcommand*{\lufalse}{\operatorname{\bot}}%     F F - falsch [unär]
%\newcommand*{\lnutrue}{\operatorname{\nottop}}%  " " - nicht wahr [unär]
% binäre Operatoren ------------------------------------------------------------
%                                                    W W F F - Aussage A
%                                                    W F W F - Aussage B
%- - - - - - - - - - - - - - - - - - - - - - - - - - - - - - - - - - - - - - - -
%\newcommand*{\lbtrue}{\operatorname{\top}}%         W W W W - wahr [binär]
%\newcommand*{\lnbfalse}{\operatorname{\notbot}}%    " " " " - nicht falsch
%            \lor                                    W W W F - A oder B
\newcommand*{\lrep}{\leftarrow}%                     W W F W - A folgt aus B
\newcommand*{\lrepA}{\Leftarrow}%
\newcommand*{\lrepB}{\subset}%
\newcommand*{\lleft}{\operatorname{\rfloor}}%        W W F F - A
%- - - - - - - - - - - - - - - - - - - - - - - - - - - - - - - - - - - - - - - -
\newcommand*{\limp}{\rightarrow}%                    W F W W - aus A folgt B
\newcommand*{\limpA}{\Rightarrow}%
\newcommand*{\limpB}{\supset}%
\newcommand*{\lright}{\operatorname{\lfloor}}%       W F W F - B
\newcommand*{\lequiv}{\leftrightarrow}%              W F F W - A genau dann,
\newcommand*{\lequivA}{\Leftrightarrow}%                       wenn B
%            \lnxor                                  " " " " - nicht
%                                                         (entweder A oder B)
%            \land                                   W F F F - A und B
\newcommand*{\landA}{\&}
\newcommand*{\landB}{\lmult}
%- - - - - - - - - - - - - - - - - - - - - - - - - - - - - - - - - - - - - -
\newcommand*{\lnand}{\uparrow}%                      F W W W - nicht
\newcommand*{\lnandA}{\barwedge}%                              (A und B)
\newcommand*{\lnandB}{\mid}%
\newcommand*{\lxor}{\ladd}%                          F W W F - entweder A
\newcommand*{\lxorA}{\operatorname{\dot\lor}}%                 oder B
\newcommand*{\lxorB}{\veebar}%
\newcommand*{\lxorC}{\oplus}%
\newcommand*{\lnequiv}{\nleftrightarrow}%            " " " " - nicht
\newcommand*{\lnequivA}{\nLeftrightarrow}%            (A genau dann, wenn B)
\newcommand*{\lnequivB}{\notequiv}%
\newcommand*{\lnright}{\lceil}%                      F W F W - nicht B
\newcommand*{\lnimp}{\nrightarrow}%                  F W F F - nicht
\newcommand*{\lnimpA}{\nRightarrow}%                         (aus A folgt B)
\newcommand*{\lnimpB}{\nsupset}%
%- - - - - - - - - - - - - - - - - - - - - - - - - - - - - - - - - - - - - -
\newcommand*{\lnleft}{\rceil}%                       F F W W - nicht A
\newcommand*{\lnrep}{\nleftarrow}%                   F F W F - nicht
\newcommand*{\lnrepA}{\nLeftarrow}%                          (A folgt aus B)
\newcommand*{\lnrepB}{\nsubset}%
\newcommand*{\lnor}{\downarrow}%                     F F F W - nicht
\newcommand*{\lnorA}{\operatorname{\overline\vee}}%            (A oder B)
%\newcommand*{\lbfalse}{\operatorname{\bot}}%        F F F F - falsch[binär]
%\newcommand*{\lnbtrue}{\operatorname{\nottop}}%     " " " " - nicht wahr

% Verwendete Mengenbezeichnungen ===============================================

% \gs* = globales Symbol
\newcommand*{\gsN}{\mathbb{N}}%    Menge der natürlichen Zahlen ohne 0
\newcommand*{\gsNo}{\mathbb{N}_0}% Menge der natürlichen Zahlen einschließlich 0

% \as* = aussagenlogisches Symbol
\newcommand*{\asA}{\mathcal{A}}%       Alphabet der Sprache
\newcommand*{\asAx}{\mathcal{A}_x}%    ... davon eine Teilmenge bzgl. \asJx
\newcommand*{\asAy}{\mathcal{A}_y}%    entsprechend \asAx
\newcommand*{\asB}{\mathcal{B}}%       Menge der binären Operatoren
\newcommand*{\asC}{\mathcal{C}}%       Menge der Konstanten
\newcommand*{\asF}{\mathcal{F}}%       Menge der Formeln
\newcommand*{\asFp}{\mathcal{F}^p}%    ... in polnischer Notation
\newcommand*{\asFx}{\mathcal{F}_x}%    Teilenge der Formeln
\newcommand*{\asFxp}{\mathcal{F}_x^p}% ... bzgl. \asJx in polnischer Notation
\newcommand*{\asFy}{\mathcal{F}_y}%    entsprechend \asFx
\newcommand*{\asFyp}{\mathcal{F}_y^p}% entsprechend \asFxp
\newcommand*{\asJ}{\mathcal{J}}%       Menge der Junktoren
\newcommand*{\asJx}{\mathcal{J}_x}%    Teilmenge der Junktoren
\newcommand*{\asJy}{\mathcal{J}_y}%    entsprechend \asJx
\newcommand*{\asM}{\mathcal{M}}%       Metaoperatoren und "Gleichheiten"
\newcommand*{\asU}{\mathcal{U}}%       Menge der unären Operatoren
\newcommand*{\asV}{\mathcal{V}}%       Menge der atomaren Formeln
\newcommand*{\asX}{\mathcal{X}}%       Mengenvariable
% verschiedene Indizes für Teilmengen von \asA, \asF, \asFp und \asJ
\newcommand*{\xAnd}{\mathrm{and}}%
\newcommand*{\xBool}{\mathrm{bool}}%
\newcommand*{\xImp}{\mathrm{imp}}%
\newcommand*{\xNand}{\mathrm{nand}}%
\newcommand*{\xNor}{\mathrm{nor}}%
\newcommand*{\xOr}{\mathrm{or}}%
\newcommand*{\xRep}{\mathrm{rep}}%

%% sonstige nützliche Kommandos %%%%%%%%%%%%%%%%%%%%%%%%%%%%%%%%%%%%%%%%%%%%%%%%

% Im Parameter von '\turl' muss '\' vor jedem Zeichen aus '{}#&%$' ein '\'
% stehen und '\' / '~' durch '\textbackslash' / '\textasctilde' ersetzt werden.
\newcommand*{\tourl}[1]{$\rightarrow$~\url{#1}}
\newcommand*{\formulatoleft}{&&&&&&&&&&}%  Um Formeln nach links zu komprimieren
\newcommand*{\formulaspace}{&&&&}%         Für Platz zwischen den Formeln
\newcommand*{\todo}[1]{\textbf{>~>~>~#1~<~<~<}}% für TODOs
\newcommand*{\charqt}[1]{\enquote*{#1}}% Quotierung von Zeichen    (character)
\newcommand*{\strqt}[1]{\enquote{#1}}%   Quotierung von Zeichenketten (string)
% Nur mit Parameter im Mathematikmodus!
%TODO einzelne Symbole in `...´ einschließen!
\newcommand*{\symqt}[1]{\charqt{#1}}%    Quotierung einzelner Symbole (symbol)
\newcommand*{\forqt}[1]{\strqt{#1}}%     Quotierung von Formeln      (formula)

% Strukturbezeichnungen ergänzen
\newcommand*{\sectionname}{Abschnitt}
\newcommand*{\subsectionname}{Unterabschnitt}
\newcommand*{\subsubsectionname}{Paragraph}

% Abkürzungen mit Punkten; zur Unterscheidung vom Satzende
\newcommand*{\textbzgl}{bzgl.\@ }
\newcommand*{\textbzw}{bzw.\@ }
\newcommand*{\textdh}{d.\@\,h.\@ }
\newcommand*{\textggf}{ggf.\@ }
\newcommand*{\textiAlg}{i.\@\,Alg.\@ }
\newcommand*{\textua}{u.\@\,a.\@ }
\newcommand*{\textusw}{usw.\@ }
\newcommand*{\textzB}{z.\@\,B.\@ }
\newcommand*{\textZB}{Z.\@\,B.\@ }

% Ergebnis von Makros verschwinden lassen
\newcommand*{\hidden}[1]{}

%%%% Glossareinträge %%%%%%%%%%%%%%%%%%%%%%%%%%%%%%%%%%%%%%%%%%%%%%%%%%%%%%%%%%%

% Indices und Symbole ==========================================================
\makeindex
\newindex[Symbolverzeichnis]{sym}
\newindex[Index]{idx}
\newcommand*{\Idx}[1]{#1\idx{#1}}%   normaler Index
\newcommand*{\Sym}[1]{#1\sym{$#1$}}% Symbol - Nur im Mathematikmodus verwenden!

% Glossareinträge ==============================================================
\GlsSetQuote{+}% wegen Gebrauch von ngerman; see glossaries guide for beginners
\makeglossaries
\setacronymstyle{long-sc-short}

% ==============================================================================

% Symbole für Beispieloperatoren

\newglossaryentry{lrelbsp}{
	name={$ \lrelbsp$},
	plural={\lrelbsp},% im Mathematikmodus
	description={%
		Beispiel für eine Relation
		mit Umkehrrelation $\rreleqbsp$%
	}
}
\newglossaryentry{lreleqbsp}{
	name={$ \lreleqbsp$},
	plural={\lreleqbsp},% im Mathematikmodus
	description={%
		Beispiel für eine Relation
		mit Gleichheit und Umkehrrelation $\rreleqbsp$%
	}
}
\newglossaryentry{relbsp}{
	name={$ \relbsp$},
	plural={\relbsp},% im Mathematikmodus
	description={%
		Beispiel für eine Relation%
	}
}
\newglossaryentry{releqbsp}{
	name={$ \releqbsp$},
	plural={\releqbsp},% im Mathematikmodus
	description={%
		Beispiel für eine Relation
		mit Gleichheit%
	}
}
\newglossaryentry{rrelbsp}{
	name={$ \rrelbsp$},
	plural={\rrelbsp},% im Mathematikmodus
	description={%
		Beispiel für eine Relation
		mit Umkehrrelation $\lrelbsp$%
	}
}
\newglossaryentry{rreleqbsp}{
	name={$ \rreleqbsp$},
	plural={\rreleqbsp},% im Mathematikmodus
	description={%
		Beispiel für eine Relation
		mit Gleichheit und Umkehrrelation $\lreleqbsp$%
	}
}

% Symbole für Metaoperatoren

\newglossaryentry{defeq}{
	name={$ \defeq$},
	plural={\defeq},% im Mathematikmodus
	description={%
		Metaoperator: ... definitionsgemäß gleich ..%
	}
}
\newglossaryentry{eq}{
	name={$ \eq$},
	plural={\eq},% im Mathematikmodus
	description={%
		Metaoperator: ... gleich ..%
	}
}
\newglossaryentry{equiv}{
	name={$ \equiv$},
	plural={\equiv},% im Mathematikmodus
	description={%
		(Meta-)Operator: ... äquivalent zu ..%
	}
}
\newglossaryentry{ne}{
	name={$ \ne$},
	plural={\ne},% im Mathematikmodus
	description={%
		Metaoperator: ... ungleich ..%
	}
}
\newglossaryentry{metaand}{
	name={$ \metaund$},
	plural={\metaand},% im Mathematikmodus
	description={%
		Metaoperator: ... oder ..%
	}
}
\newglossaryentry{metaequiv}{
	name={$ \metaequiv$},
	plural={\metaequiv},% im Mathematikmodus
	description={%
		Metaoperator: ... genau dann wenn ..%
	}
}
\newglossaryentry{metaimp}{
	name={$ \metaimp$},
	plural={\metaimp},% im Mathematikmodus
	description={%
		Metaoperator: ... dann auch ..%
	}
}
\newglossaryentry{metaor}{
	name={$ \metaoder$},
	plural={\metaor},% im Mathematikmodus
	description={%
		Metaoperator: ... oder ..%
	}
}
\newglossaryentry{metarep}{
	name={$ \metarep$},
	plural={\metarep},% im Mathematikmodus
	description={%
		Metaoperator: ... sofern ..%
	}
}

% sonstige Operatoren

\newglossaryentry{lfalse}{
	name={$ \lfalse$},
	plural={\lfalse},% im Mathematikmodus
	description={%
		Aussagenlogische Konstante: Falsch%
	}
}
\newglossaryentry{ltrue}{
	name={$ \ltrue$},
	plural={\ltrue},% im Mathematikmodus
	description={%
		Aussagenlogische Konstante: Wahr%
	}
}

% Symbole für Mengen

\newglossaryentry{gsN}{
	name={$ \gsN$},
	plural={\gsN},% im Mathematikmodus
	description={%
		Menge der natürlichen Zahlen ohne 0%
	}
}
\newglossaryentry{gsNo}{
	name={$ \gsNo$},
	plural={\gsNo},% im Mathematikmodus - erfolgt im falschen Zeichensatz!
	description={%
		Menge der natürlichen Zahlen einschließlich 0%
	}
}
\newglossaryentry{asA}{
	name={$ \asA$},
	plural={\asA},% im Mathematikmodus
	description={%
		Alphabet der aussagenlogischen Sprache%
	}
}
\newglossaryentry{asAx}{
	name={$ \asAx$},
	plural={\asAx},% im Mathematikmodus
	description={%
		Teilmenge des Alphabets $\asA$ der aussagenlogischen Sprache%
	}
}
\newglossaryentry{asB}{
	name={$ \asB$},
	plural={\asB},% im Mathematikmodus
	description={%
		Menge der aussagenlogischen, binären Operatoren%
	}
}
\newglossaryentry{asC}{
	name={$ \asC$},
	plural={\asC},% im Mathematikmodus
	description={%
		Menge der aussagenlogischen Konstanten%
	}
}
\newglossaryentry{asF}{
	name={$ \asF$},
	plural={\asF},% im Mathematikmodus
	description={%
		Menge der aussagenlogischen Formeln mit Klammerung%
	}
}
\newglossaryentry{asFp}{
	name={$ \asFp$},
	plural={\asFp},% im Mathematikmodus
	description={%
		Menge der aussagenlogischen Formeln in polnischer Notation%
	}
}
\newglossaryentry{asFx}{
	name={$ \asFx$},
	plural={\asFx},% im Mathematikmodus
	description={%
		Teilmenge der Menge $\asF$
		der aussagenlogischen Formeln mit Klammerung%
	}
}
\newglossaryentry{asFxp}{
	name={$ \asFxp$},
	plural={\asFxp},% im Mathematikmodus
	description={%
		Teilmenge der Menge $\asF$
		der aussagenlogischen Formeln in polnischer Notation%
	}
}
\newglossaryentry{asJ}{
	name={$ \asJ$},
	plural={\asJ},% im Mathematikmodus
	description={%
		Menge der aussagenlogischen Operatoren%
	}
}
\newglossaryentry{asJx}{
	name={$ \asJx$},
	plural={\asJx},% im Mathematikmodus
	description={%
		Teilmenge der Menge $\asJ$ der aussagenlogischen Operatoren
	}
}
\newglossaryentry{asM}{
	name={$ \asM$},
	plural={\asM},% im Mathematikmodus
	description={%
		Menge der Metaoperatoren und der mit Gleichheit verwandten Symbole%
	}
}
\newglossaryentry{asU}{
	name={$ \asU$},
	plural={\asU},% im Mathematikmodus
	description={%
		Menge der aussagenlogischen, unären Operatoren%
	}
}
\newglossaryentry{asV}{
	name={$ \asV$},
	plural={\asV},% im Mathematikmodus
	description={%
		Menge der aussagenlogischen, atomaren Formeln%
	}
}

% keine Symbole mehr

\newglossaryentry{Aussage}{
	name={Aussage},
	plural={Aussagen},
	description={%
		Eine Aussage in natürlicher Sprache oder als Formel,
		die einen Wahrheitswert liefert%
	}
}
\newglossaryentry{Aussagenlogik}{
	name={Aussagenlogik},
	description={%
		\see \sectionname~\vref{sec:Aussagenlogik}%
	}
}
\newglossaryentry{Beweisschritt}{
	name={Beweisschritt},
	plural={Beweisschritte},
	description={%
		Eine Vorschrift, wie aus vorgegebenen Aussagen eine weitere folgt%
	}
}
\newcommand*{\ASBA}{\textsf{\glsIdx{ASBA}}}
\newacronym{ASBA}{ASBA}{
	Das zu entwickelnde Programmsystem,
	das \textbf{A}xiome, \textbf{S}ätze,
	\textbf{B}eweise und \textbf{A}uswertungen behandeln kann%
}
\newglossaryentry{Ausgabeschema}{
	name={Ausgabeschema},
	plural={Ausgabeschemata},
	description={%
		Ein Schema, mit dem bestimmte mathematische Objekte
		ausgegeben werden sollen%
	}
}
\newglossaryentry{Axiom}{
	name={Axiom},
	plural={Axiome},
	description={%
		Eine Formel, die unbewiesen als wahr angesehen wird%
	}
}
\newglossaryentry{Beweis}{
	name={Beweis},
	plural={Beweise},
	description={%
		Eine zulässige
		Ableitung einer Folgerung aus gegebenen Voraussetzungen%
	}
}
\newglossaryentry{Fachbegriff}{
	name={Fachbegriff},
	plural={Fachbegriffe},
	description={%
		Ein Name für einen mathematischen Begriff%
	}
}
\newglossaryentry{Fachgebiet}{
	name={Fachgebiet},
	plural={Fachgebiete},
	description={%
		Ein Teil der Mathematik mit einer zugehörigen Basis aus Axiomen,
		Sätzen und spezifischen Fachbegriffen und Darstellungen%
	}
}
\newglossaryentry{FormalelementV}{
	name={formales Element},   % ein   ...
	plural={formale Elemente}, % viele ...
	description={%
		Ein mathematisches Element in formaler Schreibweise.
		Bis auf wenige Aussagen kommen darin
		\glsIdxPl{MetaausdruckV} nicht mehr vor%
	}%
}
\newglossaryentry{intEigenschaftA}{
	name={interessierende Eigenschaft},      % eine ...
	plural={interessierenden Eigenschaften}, % alle ...
	description={
		Interessierende Eigenschaften sind solche,
		die \textbzgl \symqt{$=$} \textbzw \symqt{$\equiv$}
		von Interesse sind.
		\textZB ist die interessierende Eigenschaft der Operanden
		der Gleichung \forqt{$y=f(x)$}
		normalerweise der Wert (von \forqt{$y$} und \forqt{$f(x)$}),
		und nicht deren Darstellung (Zeichenketten)%
	}%
}
\newglossaryentry{MetaausdruckV}{
	name={metasprachlicher Ausdruck},   % ein   ...
	plural={metasprachliche Ausdrücke}, % viele ...
	description={%
		Eine in normaler Sprache verfasste Aussage,
		die auch zusammengesetzt sein kann%
	}%
}
\newglossaryentry{MetaaussageV}{
	name={metasprachliche Aussage},    % eine   ...
	plural={metasprachliche Aussagen}, % viele ...
	description={%
		Eine in \glsIdx{Metasprache} verfasste Aussage,
		die auch zusammengesetzt sein kann%
	}%
}
\newglossaryentry{MetaoperatorV}{
	name={metasprachlicher Operator},    % ein   ...
	plural={metasprachliche Operatoren}, % viele ...
	description={%
		Ein Operator,
		deren Operanden \glsIdxPl{MetaausdruckV} sind%
	}%
}
\newglossaryentry{Metasprache}{
	name={Metasprache},
	plural={Metasprachen},
	description={%
		Eine Sprache,
		in der Aussagen über Elemente einer anderen Sprache
		getroffen werden können%
	}%
}
\newglossaryentry{Praedikat}{
	name={Prädikat},
	plural={Prädikate},
	description={%
		Ein Element der \glsIdx{Praedikatenlogik}
		(\see \sectionname~\vref{sec:Prädikatenlogik}).
		\textZB kann man eine \forqt{$Gruppe$} als
		ein dreistelliges Prädikat \forqt{$Gruppe(G,+,\cdot)$} definieren,
		in dem $G$ eine Menge und \symqt{$+$} und \symqt{$\cdot$}
		Operationen in \forqt{$G \times G$} sind,
		so dass die Gruppenaxiome erfüllt sind%
	}
}
\newglossaryentry{Praedikatenlogik}{
	name={Prädikatenlogik},
	description={%
		\see \sectionname~\vref{sec:Prädikatenlogik}%
	}
}
\newglossaryentry{Satz}{
	name={Satz},
	plural={Sätze},
	description={%
		Eine mathematische Aussage,
		dass eine bestimmte Folgerung
		aus gegebenen Voraussetzungen abgeleitet werden kann%
	}
}
\newglossaryentry{vergleichbar}{
	name={vergleichbar},
	plural={vergleichbare},
	description={
		Zwei \glsIdxPl{MetaausdruckV} \textbzw \glsIdxPl{FormalelementV}
		heißen
		-- auf eine bestimmte Art --
		vergleichbar,
		wenn sie auf diese Art (\textzB als Zeichenketten oder als vergleichbare Ergebnisse von Formeln) verglichen werden können.
		Die Art muss implizit bekannt oder explizit angegeben sein.
		Meistens genügt es zu wissen,
		was für
		\glsIdxPl{MetaausdruckV} \textbzw \glsIdxPl{FormalelementV} es sind.
		Beide müssen dann nur von derselben Art sein%
	}%
}
\newglossaryentry{Wahrheitswert}{
	name={Wahrheitswert},
	plural={Wahrheitswerte},
	description={%
		Wahrheitswerte sind die Werte \strqt{wahr} und \strqt{falsch},
		oft auch als \strqt{true} und \strqt{false}
		oder einfach \charqt{1} und \charqt{0} bezeichnet.
	}
}

% Symbol/Index und Glossareintrag ==============================================

\newcommand*{\glsSym}[1]{\glspl{#1}\sym{\gls{#1}}}% Symbol - Im Mathematikmodus!
\newcommand*{\glsIdx}[1]{\gls{#1}\idx{\gls{#1}}}%     normal
\newcommand*{\GlsIdx}[1]{\Gls{#1}\idx{\gls{#1}}}%     groß
\newcommand*{\glsIdxBg}[2]{#2\idx{\gls{#1}}}%                  Beugung
\newcommand*{\GlsIdxBg}[2]{#2\idx{\gls{#1}}}%         groß und Beugung
\newcommand*{\glsIdxPl}[1]{\glspl{#1}\idx{\gls{#1}}}%          Plural
\newcommand*{\GlsIdxPl}[1]{\Glspl{#1}\idx{\gls{#1}}}% groß und Plural

%%%% Titelseite %%%%%%%%%%%%%%%%%%%%%%%%%%%%%%%%%%%%%%%%%%%%%%%%%%%%%%%%%%%%%%%%

\titlehead{
	{\Large Dr. Winfried Teschers}\\
	Anton-Günther-Straße 26c\\91083 Baiersdorf\\
	{\footnotesize winfried.teschers@t-online.de}
}
\subject{Projektdokument}
\title{{\Huge \ASBA}\\Axiome, Sätze, Beweise und Auswertungen}
\subtitle{Projekt zur maschinellen Überprüfung von mathematischen Beweisen
	und deren Ausgabe in lesbarer Form}
\author{Winfried Teschers}
\date{\today}
\publishers{\vspace{1cm}\normalsize
	Es wird ein Programmsystem beschrieben,
	das zu eingegebenen Axiomen, Sätzen, und Beweisen letztere prüft,
	Auswertungen generiert
	und zu gegebenen Ausgabeschemata eine Ausgabe der Elemente
	in üblicher Formelschreibweise im \LaTeX-Format erstellt.
}

%%%% Dokument %%%%%%%%%%%%%%%%%%%%%%%%%%%%%%%%%%%%%%%%%%%%%%%%%%%%%%%%%%%%%%%%%%

\begin{document}
	\maketitle

	~\vfill Copyright \copyright{} 2017 Winfried Teschers\bigskip

	Permission is granted to copy, distribute and/or modify this document under
	the terms of the GNU Free Documentation License, Version~1.3 or any later
	version published by the Free Software Foundation;
	with no Invariant Sections, no Front-Cover Texts, and no Back-Cover Texts.
	You should have received a copy of the GNU Free Documentation License along
	with this document.
	If not, see \url{http://www.gnu.org/licenses/}.

	\beforechapter% Kopfzeile aktivieren und neue Seite
	%chapter{\contentsname}% Inhaltsverzeichnis %%%%%%%%%%%%%%%%%%%%%%%%%%%%%%%%
	\likechapter{\contentsname}
	\label{cha:Inhaltsverzeichnis}

	\tableofcontents

	\beforechapter% Kopfzeile aktivieren und neue Seite
	\chapter{Analyse}%%%%%%%%%%%%%%%%%%%%%%%%%%%%%%%%%%%%%%%%%%%%%%%%%%%%%%%%%%%
	\beginchapter{Analyse}
	\label{cha:Analyse}

	In der Mathematik gibt es eine unüberschaubare Menge an
	\glsIdxPl{Axiom}n, \glsIdxPl{Satz}n, \glsIdxPl{Beweis}n, \emph{\glsIdxPl{Fachbegriff}n}%
	\footnote{%
		\emph{\glsIdxPl{Fachbegriff}} sind Namen für mathematische Elemente und
		Konstruktionen,
		\textzB \glsIdxPl{Axiom}, \glsIdxPl{Satz}, \glsIdxPl{Beweis} und \glsIdxPl{Fachgebiet}.
		Symbole können als spezielle \glsIdxPl{Fachbegriff} aufgefasst werden.%
	}
	und \emph{\glsIdxPl{Fachgebiet}n}%
	\footnote{%
		Ein \emph{\glsIdx{Fachgebiet}} ist ein Teil der Mathematik
		mit einer zugehörigen Basis an Axiomen,
		\glsIdxPl{Satz}n und spezifischen \glsIdxPl{Fachbegriff}n und Darstellungen.
		\textZB \emph{Logik}, \emph{Mengenlehre} und \emph{Gruppentheorie}.
		Ein \glsIdx{Fachgebiet} kann sehr klein sein
		und im Extremfall kein einziges Element enthalten.
		\emph{Umgebung} wäre eine bessere Bezeichnung,
		ist aber schon ein verbreiteter \glsIdx{Fachbegriff},
		so dass hier die Bezeichnung \glsIdx{Fachgebiet} verwendet wird.%
	}.
	Zu den meisten \glsIdxPl{Fachgebiet}n gibt es noch ungelöste Probleme.

	Es fehlt ein System, das einen Überblick bietet
	und die Möglichkeit, \glsIdxPl{Beweis} automatisch zu überprüfen.
	Außerdem sollte all dies in üblicher mathematischer Schreibweise
	ein- und ausgegeben werden können.
	In diesem Dokument werden
	die Grundlagen für das zu entwickelnde \ASBA\ behandelt.

	Ein Programmsystem mit ähnlicher Aufgabenstellung findet sich
	im GitHub Projekt Hilbert~II (\seename~\cite{bib:HilbertII, bib:qedeq}).
	Einige Ideen sind von dort übernommen worden.

	\section{Fragen}%===========================================================
	\beginsection{Fragen}
	\label{sec:Fragen}

	Einige der Fragen, die in diesem Zusammenhang auftauchen,
	werden hier formuliert:

	\begin{enumerate}

		\item \label{Frage:Grundlagen} \emph{Grundlagen}:
		Was sind die Grundlagen?
		\textZB welche Logik und Mengenlehre.

		\item \label{Frage:Basis} \emph{Basis}:
		Welche wichtigen \glsIdxPl{Axiom}, \glsIdxPl{Satz}, \glsIdxPl{Beweis},
		\glsIdxPl{Fachbegriff} und \glsIdxPl{Fachgebiet} gibt es?
		Welche davon sind Standard?

		\item \label{Frage:Axiome} \emph{\glsIdxPl{Axiom}}:
		Welche \glsIdxPl{Axiom} werden bei einem \glsIdx{Satz} oder \glsIdx{Beweis} vorausgesetzt?
		Allgemein anerkannte oder auch strittige,
		wie \textzB den \emph{Satz vom ausgeschlossenen Dritten}
		(\emph{tertium non datur}) oder das \emph{Auswahlaxiom}.

		\item \label{Frage:Beweis} \emph{\glsIdx{Beweis}}:
		Ist ein \glsIdx{Beweis} fehlerfrei?

		\item \label{Frage:Konstruktion} \emph{Konstruktion}:
		Gibt es einen konstruktiven \glsIdx{Beweis}?

		\item \label{Frage:Vergleiche} \emph{Vergleiche}:
		Welcher \glsIdx{Beweis} ist besser?
		Nach welchem Kriterium?
		\textZB elegant, kurz, einsichtig oder wenige \glsIdxPl{Axiom}.
		Was heißt eigentlich \emph{elegant}?

		\item \label{Frage:Definitionen} \emph{Definitionen}:
		Was ist mit einem \glsIdx{Fachbegriff} jeweils genau gemeint?
		\textZB \emph{Stetigkeit}, \emph{Integral} und \emph{Analysis}.

		\item \label{Frage:Abhängigkeiten} \emph{Abhängigkeiten}:
		Wie heißt ein \glsIdx{Fachbegriff} in einer anderen Sprache?
		Ist wirklich dasselbe gemeint?
		Was ist mit \glsIdxPl{Fachbegriff}n
		in verschiedenen \glsIdxPl{Fachgebiet}n?

		\item \label{Frage:Überblick} \emph{Überblick}:
		Ist ein \glsIdx{Axiom}, \glsIdx{Satz}, \glsIdx{Beweis} oder Fachbegriff schon einmal
		-- \textggf abweichend --
		definiert, formuliert oder bewiesen worden?

		\item \label{Frage:Darstellung} \emph{Darstellung}:
		Wie kann man einen \glsIdx{Satz} und den zugehörigen \glsIdx{Beweis}
		-- \textggf auch spezifisch für ein \glsIdx{Fachgebiet} --
		darstellen?

		\item \label{Frage:Forschung} \emph{Forschung}:
		Welche Probleme gibt es noch zu erforschen.

	\end{enumerate}

	\section{Eigenschaften}%====================================================
	\beginsection{Eigenschaften}
	\label{sec:Eigenschaften}

	Ausgehend von den Fragen in \sectionname~\vref{sec:Fragen}
	soll \ASBA\ entwickelt werden, das die folgenden Eigenschaften hat:
	\begin{enumerate}

		\item \label{Eigenschaft:Daten} \emph{Daten}:
		\glsIdxPl{Axiom}, \glsIdxPl{Satz}, \glsIdxPl{Beweis},
		\glsIdxPl{Fachbegriff} und \glsIdxPl{Fachgebiet}
		können in formaler Form gespeichert werden
		-- auch nicht oder unvollständig bewiesene \glsIdxPl{Satz}.
		Dabei soll die übliche mathematische Schreibweise
		verwendet werden können.

		\item \label{Eigenschaft:Definitionen} \emph{Definitionen}:
		Es können \glsIdxPl{Fachbegriff} für \glsIdxPl{Axiom}, \glsIdxPl{Satz}, \glsIdxPl{Beweis}
		und \glsIdxPl{Fachgebiet}
		-- letztere mit eigenen Axiomen, \glsIdxPl{Satz}n, \glsIdxPl{Beweis}n,
		\glsIdxPl{Fachbegriff}n und über- oder untergeordneten
		\glsIdxPl{Fachgebiet}n --
		definiert werden.
		Die Definitionen dürfen wiederum an dieser Stelle
		schon bekannte \glsIdxPl{Fachbegriff} und \glsIdxPl{Fachgebiet} verwenden.

		\item \label{Eigenschaft:Prüfung} \emph{Prüfung}:
		Vorhandene \glsIdxPl{Beweis} können automatisch geprüft werden.

		\item \label{Eigenschaft:Ausgaben} \emph{Ausgaben}:
		Die \glsIdxPl{Axiom}, \glsIdxPl{Satz} und \glsIdxPl{Beweis} können in üblicher Schreibweise
		-- abhängig von Sprache und \glsIdx{Fachgebiet} --
		ausgegeben werden.

		\item \label{Eigenschaft:Auswertungen} \emph{Auswertungen}:
		Zusätzlich zur Ausgabe der gespeicherten Daten
		sind verschiedene Auswertungen möglich,
		unter anderem für die meisten der unter \sectionname~\vref{sec:Fragen}
		behandelten Fragen.

		\setcounter{Enumi}{\value{enumi}}% Nummerierung wird fortgesetzt.
	\end{enumerate}
	Damit \ASBA\ nicht umsonst erstellt wird
	und möglichst breite Verwendung findet,
	werden noch zwei Punkte angefügt:
	\begin{enumerate}
		\setcounter{enumi}{\value{Enumi}}% Nummerierung wird fortgesetzt.

		\item \label{Eigenschaft:Lizenz} \emph{Lizenz}:
		Die Software ist \emph{Open Source}.

		\item \label{Eigenschaft:Akzeptanz} \emph{Akzeptanz}:
		\ASBA\ wird von Mathematikern akzeptiert und verwendet.
	\end{enumerate}
	\tablename~\vref{tab:Fragen->Eigenschaften} zeigt,
	wie sich die Eigenschaften zu den Fragen in
	\sectionname~\vref{sec:Fragen} verhalten.
	Mit einem X werden die Spalten einer Zeile markiert,
	deren zugehörige Eigenschaften zur Beantwortung
	der entsprechenden Frage beitragen sollen.
	Idealerweise sollte die Erfüllung aller angegebenen Eigenschaften
	alle gestellten Fragen beantworten, was allerdings illusorisch ist.

	% Abstände für die nächsten drei Tabellen
	\newcommand*{\vsL}{\hspace{-1.0cm}}%   für 1-stellige Zahlen
	\newcommand*{\vsl}{\hspace{-6pt}\vsL}% für 2-stellige Zahlen
	\newcommand*{\vsc}{\hspace{6pt}}%      für die gedrehten Überschriften

	\begin{table}[h]
		\begin{tabularx}{\linewidth-10.95pt}
			{@{\hspace{.5cm}}rl@{\extracolsep{\fill}}|*{7}{c}@{\hspace{1cm}}|}
			\multicolumn{2}{l|}{\diagbox[height=3.0cm,width=4.5cm]%
				{\textbf{Frage}\\~}{\\\textbf{Eigenschaft}}}
			&\rotatebox{90}{%
				\mbox{\vsL\ref{Eigenschaft:Daten}        \vsc Daten        }}
			&\rotatebox{90}{%
				\mbox{\vsL\ref{Eigenschaft:Definitionen} \vsc Definitionen }}
			&\rotatebox{90}{%
				\mbox{\vsL\ref{Eigenschaft:Prüfung}      \vsc Prüfung      }}
			&\rotatebox{90}{%
				\mbox{\vsL\ref{Eigenschaft:Ausgaben}     \vsc Ausgaben     }}
			&\rotatebox{90}{%
				\mbox{\vsL\ref{Eigenschaft:Auswertungen} \vsc Auswertungen }}
			&\rotatebox{90}{%
				\mbox{\vsL\ref{Eigenschaft:Lizenz}       \vsc Lizenz       }}
			&\rotatebox{90}{%
				\mbox{\vsL\ref{Eigenschaft:Akzeptanz}    \vsc Akzeptanz    }}
			\\\hline
			\ref{Frage:Grundlagen}      & Grundlagen
				& X & X & - & X & X & - & - \\
			\ref{Frage:Basis}           & Basis
				& X & X & - & X & X & - & - \\
			\ref{Frage:Axiome}          & \glsIdxPl{Axiom}
				& X & X & - & X & X & - & - \\
			\hdashline[2pt/2pt]
			\ref{Frage:Beweis}          & \glsIdx{Beweis}
				& X & - & X & X & - & - & - \\
			\ref{Frage:Konstruktion}    & Konstruktion
				& X & - & - & X & - & - & - \\
			\ref{Frage:Vergleiche}      & Vergleiche
				& X & - & - & - & X & - & - \\
			\hdashline[2pt/2pt]
			\ref{Frage:Definitionen}    & Definitionen
				& X & X & - & X & - & - & - \\
			\ref{Frage:Abhängigkeiten}  & Abhängigkeiten
				& X & - & - & X & - & - & - \\
			\ref{Frage:Überblick}       & Überblick
				& X & - & - & - & X & - & - \\
			\hdashline[2pt/2pt]
			\ref{Frage:Darstellung}     & Darstellung
				& - & X & - & X & - & - & - \\
			\ref{Frage:Forschung}       & Forschung
				& X & - & - & - & X & - & - \\
			\hline
		\end{tabularx}
		\caption{\ref{sec:Fragen} Fragen
			$\to$ \ref{sec:Eigenschaften} Eigenschaften}
		\label{tab:Fragen->Eigenschaften}
	\end{table}

	\section{Ziele}%============================================================
	\beginsection{Ziele}
	\label{sec:Ziele}

	Um die Eigenschaften von \sectionname~\vref{sec:Eigenschaften} zu erreichen,
	werden für \ASBA\ die folgenden Ziele%
	\footnote{%
		Es sind eigentlich Anforderungen.
		Da dieser Begriff auch im \chaptername~\vref{cha:Design} verwendet wird,
		werden die Anforderungen hier \emph{\Idx{Ziel}}\emph{e} genannt.%
	}
	gesetzt:
	\begin{enumerate}

		\item \label{Ziel:Daten} \emph{Daten}:
		Es enthält möglichst viele wichtige
		\glsIdxPl{Axiom}, \glsIdxPl{Satz}, \glsIdxPl{Beweis}, \glsIdxPl{Fachbegriff},
		\glsIdxPl{Fachgebiet} und \glsIdxPl{Ausgabeschema}%
		\footnote{%
			Um den Punkt~\ref{Eigenschaft:Ausgaben}
			von \sectionname~\vref{sec:Eigenschaften} erfüllen zu können,
			werden noch fachgebietsspezifische
			\glsIdxPl{Ausgabeschema} benötigt,
			welche die Art der Ausgaben beschreiben.%
		}.

		\item \label{Ziel:Form} \emph{Form}:
		Die Daten liegt in formaler, geprüfter Form vor.

		\item \label{Ziel:Eingaben} \emph{Eingaben}:
		Die Eingabe von Daten erfolgt in einer formalen Syntax
		unter Verwendung der üblichen mathematischen Schreibweise.

		\item \label{Ziel:Prüfung} \emph{Prüfung}:
		Vorhandene \glsIdxPl{Beweis} können automatisch geprüft werden.

		\item \label{Ziel:Ausgaben} \emph{Ausgaben}:
		Die Ausgabe kann in einer eindeutigen, formalen Syntax
		gemäß vorhandener \glsIdxPl{Ausgabeschema} erfolgen.

		\item \label{Ziel:Auswertungen} \emph{Auswertungen}:
		Zusätzlich zur Ausgabe der Daten sind verschiedene Auswertungen möglich.
		Insbesondere kann zu jedem \glsIdx{Beweis} angegeben werden,
		wie viele Beweisschritte
		und welche \glsIdxPl{Axiom} und \glsIdxPl{Satz}%
		\footnote{\glsIdxPl{Satz}, die quasi als \glsIdxPl{Axiom} verwendet werden.}
		er verwendet.

		\item \label{Ziel:Anpassbarkeit} \emph{Anpassbarkeit}:
		\glsIdxPl{Fachbegriff} und die Darstellung bei der Ausgabe können
		mit Hilfe von
		-- gegebenenfalls unbenannten --
		untergeordneten \glsIdxPl{Fachgebiet}n angepasst werden.

		\item \label{Ziel:Individualität} \emph{Individualität}:
		\glsIdxPl{Axiom} und \glsIdxPl{Satz} können
		für jeden \glsIdx{Beweis} individuell vorausgesetzt werden.
		Dabei sind fachgebietsspezifische \glsIdxPl{Fachbegriff} erlaubt.

		\item \label{Ziel:Internet} \emph{Internet}:
		Die Daten können auf mehrere Dateien verteilt sein.
		Ein Teil davon
		-- oder sogar alle --
		können im Internet liegen.

		\item \label{Ziel:Kommunikation} \emph{Kommunikation}:
		Die Kommunikation mit \ASBA\ kann mit den
		\glsIdxPl{Fachbegriff}n der einzelnen \glsIdxPl{Fachgebiet} erfolgen.

		\item \label{Ziel:Zugriff} \emph{Zugriff}:
		Der Zugriff auf \ASBA\ kann lokal und über das Internet erfolgen.

		\item \label{Ziel:Unabhängigkeit} \emph{Unabhängigkeit}:
		\ASBA\ kann online und offline arbeiten.

		\item \label{Ziel:Rekursion} \emph{Rekursion}:
		Es kann rekursiv über alle verwendeten Dateien
		-- auch solchen, die im Internet liegen --
		ausgewertet werden.

		\item \label{Ziel:Bedienbarkeit} \emph{Bedienbarkeit}:
		\ASBA\ ist einfach zu bedienen.

		\item \label{Ziel:Lizenz} \emph{Lizenz}:
		Die Software ist \emph{Open Source}.

		\setcounter{Enumi}{\value{enumi}}% Nummerierung wird fortgesetzt.
	\end{enumerate}
	Der Punkt \ref{Ziel:Zwischenspeicher} wurde noch eingefügt,
	damit \ASBA\ effizient arbeiten kann
	und um die Akzeptanz zu erhöhen:
	\begin{enumerate}
		\setcounter{enumi}{\value{Enumi}}% Nummerierung wird fortgesetzt.

		\item \label{Ziel:Zwischenspeicher} \emph{Zwischenspeicher}:
		Wichtige Auswertungen können an vorhandenen Dateien angehängt
		oder separat in eigenen Dateien gespeichert werden.

	\end{enumerate}
	\tablename~\vref{tab:Eigenschaften->Ziele} zeigt wieder,
	wie sich die Ziele zu den Eigenschaften
	in \sectionname~\vref{sec:Eigenschaften} verhalten.
	Mit einem X werden wieder die Spalten einer Zeile markiert,
	deren zugehörige Ziele zur
	Sicherstellung der entsprechenden Eigenschaft beitragen sollen.
	Idealerweise sollte durch Erreichen aller aufgestellten Ziele
	\gls{ASBA} alle angegebenen Eigenschaften aufweisen,
	was wahrscheinlich ebenfalls illusorisch ist.
	\begin{table}[ht]
		\begin{tabularx}{\linewidth-10.95pt}
			{@{\hspace{.3cm}}rl@{\extracolsep{\fill}}|*{16}{c}@{\hspace{0.4cm}}|}
			\multicolumn{2}{l|}{\diagbox[height=3.0cm,width=4.0cm]%
				{\textbf{Eigenschaft}\\~}{\\\\\\\textbf{Ziel}}}
			&\rotatebox{90}{%
				\mbox{\vsL\ref{Ziel:Daten}            \vsc Daten            }}
			&\rotatebox{90}{%
				\mbox{\vsL\ref{Ziel:Form}             \vsc Form             }}
			&\rotatebox{90}{%
				\mbox{\vsL\ref{Ziel:Eingaben}         \vsc Eingaben         }}
			&\rotatebox{90}{%
				\mbox{\vsL\ref{Ziel:Prüfung}          \vsc Prüfung          }}
			&\rotatebox{90}{%
				\mbox{\vsL\ref{Ziel:Ausgaben}         \vsc Ausgaben         }}
			&\rotatebox{90}{%
				\mbox{\vsL\ref{Ziel:Auswertungen}     \vsc Auswertungen     }}
			&\rotatebox{90}{%
				\mbox{\vsL\ref{Ziel:Anpassbarkeit}    \vsc Anpassbarkeit    }}
			&\rotatebox{90}{%
				\mbox{\vsL\ref{Ziel:Individualität}   \vsc Individualität   }}
			&\rotatebox{90}{%
				\mbox{\vsL\ref{Ziel:Internet}         \vsc Internet         }}
			&\rotatebox{90}{%
				\mbox{\vsl\ref{Ziel:Kommunikation}    \vsc Kommunikation    }}
			&\rotatebox{90}{%
				\mbox{\vsl\ref{Ziel:Zugriff}          \vsc Zugriff          }}
			&\rotatebox{90}{%
				\mbox{\vsl\ref{Ziel:Unabhängigkeit}   \vsc Unabhängigkeit   }}
			&\rotatebox{90}{%
				\mbox{\vsl\ref{Ziel:Rekursion}        \vsc Rekursion        }}
			&\rotatebox{90}{%
				\mbox{\vsl\ref{Ziel:Bedienbarkeit}    \vsc Bedienbarkeit    }}
			&\rotatebox{90}{%
				\mbox{\vsl\ref{Ziel:Lizenz}           \vsc Lizenz           }}
			&\rotatebox{90}{%
				\mbox{\vsl\ref{Ziel:Zwischenspeicher} \vsc Zwischenspeicher }}
			\\\hline
			\ref{Eigenschaft:Daten}        & Daten%
				& X & X & X & - & - & - & - & - & - & - & - & - & - & - & - &-\\
			\ref{Eigenschaft:Definitionen} & Definitionen%
				& X & - & X & - & - & - & - & - & - & - & - & - & - & - & - &-\\
			\ref{Eigenschaft:Prüfung}      & Prüfung
				& - & - & - & X & - & - & - & - & - & - & - & - & - & - & - &-\\
			\hdashline[2pt/2pt]
			\ref{Eigenschaft:Ausgaben}     & Ausgaben%
				& - & - & - & - & X & - & - & - & - & - & - & - & - & - & - &-\\
			\ref{Eigenschaft:Auswertungen} & Auswertungen%
				& - & - & - & - & - & X & - & - & - & - & - & - & - & - & - &-\\
			\ref{Eigenschaft:Lizenz}       & Lizenz%
				& - & - & - & - & - & - & - & - & - & - & - & - & - & - & X &-\\
			\hdashline[2pt/2pt]
			\ref{Eigenschaft:Akzeptanz}    & Akzeptanz%
				& X & X & X & X & X & X & X & X & X & X & X & X & X & X & X &X\\
			\hline
		\end{tabularx}
		\caption{\ref{sec:Eigenschaften}
			Eigenschaften $\to$ \ref{sec:Ziele} Ziele}
		\label{tab:Eigenschaften->Ziele}% Erst nach '\caption'!
	\end{table}

	\section{Zusammenfassung}%==================================================
	\beginsection{Zusammenfassung}
	\label{sec:Zusammenfassung}

	Die \tablename~\vref{tab:Fragen->Ziele} ist eine Kombination aus den
	\tablename n~\vref{tab:Fragen->Eigenschaften}
	und~\vref{tab:Eigenschaften->Ziele}
	und zeigt, wie sich die Ziele in \sectionname~\vref{sec:Ziele}
	zu den Fragen in \sectionname~\vref{sec:Fragen} verhalten.
	Auch hier werden mit einem X die Spalten einer Zeile markiert,
	deren zugehörige Ziele für die Beantwortung der entsprechenden Frage
	nötig sind.
	Mit einem kleinen x werden sie markiert,
	wenn sie zur Beantwortung der Fragen nicht nötig,
	aber von Interesse sind.
	Idealerweise sollte das Erreichen aller aufgestellten Ziele
	wieder alle gestellten Fragen beantworten,
	was natürlich auch illusorisch ist.
	\begin{table}[ht]
		\begin{tabularx}{\linewidth-10.95pt}
			{@{\hspace{.3cm}}rl@{\extracolsep{\fill}}|*{15}{c}@{\hspace{0.4cm}}|}
				\multicolumn{2}{l|}{\diagbox[height=3.0cm,width=4.0cm]%
					{\textbf{Frage}\\~}{\\\\\\\textbf{Ziel}}}
			&\rotatebox{90}{%
				\mbox{\vsL\ref{Ziel:Daten}            \vsc Daten            }}
			&\rotatebox{90}{%
				\mbox{\vsL\ref{Ziel:Form}             \vsc Form             }}
			&\rotatebox{90}{%
				\mbox{\vsL\ref{Ziel:Eingaben}         \vsc Eingaben         }}
			&\rotatebox{90}{%
				\mbox{\vsL\ref{Ziel:Prüfung}          \vsc Prüfung          }}
			&\rotatebox{90}{%
				\mbox{\vsL\ref{Ziel:Ausgaben}         \vsc Ausgaben         }}
			&\rotatebox{90}{%
				\mbox{\vsL\ref{Ziel:Auswertungen}     \vsc Auswertungen     }}
			&\rotatebox{90}{%
				\mbox{\vsL\ref{Ziel:Anpassbarkeit}    \vsc Anpassbarkeit    }}
			&\rotatebox{90}{%
				\mbox{\vsL\ref{Ziel:Individualität}   \vsc Individualität   }}
			&\rotatebox{90}{%
				\mbox{\vsL\ref{Ziel:Internet}         \vsc Internet         }}
			&\rotatebox{90}{%
				\mbox{\vsl\ref{Ziel:Kommunikation}    \vsc Kommunikation    }}
			&\rotatebox{90}{%
				\mbox{\vsl\ref{Ziel:Zugriff}          \vsc Zugriff          }}
			&\rotatebox{90}{%
				\mbox{\vsl\ref{Ziel:Unabhängigkeit}   \vsc Unabhängigkeit   }}
			&\rotatebox{90}{%
				\mbox{\vsl\ref{Ziel:Rekursion}        \vsc Rekursion        }}
			&\rotatebox{90}{%
				\mbox{\vsl\ref{Ziel:Bedienbarkeit}    \vsc Bedienbarkeit    }}
			&\rotatebox{90}{%
				\mbox{\vsl\ref{Ziel:Lizenz}           \vsc Lizenz           }}
			\\\hline
			\ref{Frage:Grundlagen}      & Grundlagen%
				& X & X & X & - & X & X & x & - & - & - & - & - & - & - & - \\
			\ref{Frage:Basis}           & Basis%
				& X & X & X & - & X & X & x & x & - & - & - & - & - & - & - \\
			\ref{Frage:Axiome}          & \glsIdxPl{Axiom}%
				& X & X & X & - & X & X & x & - & - & - & - & - & - & - & - \\
			\hdashline[2pt/2pt]
			\ref{Frage:Beweis}          & \glsIdx{Beweis}%
				& X & X & X & X & X & - & - & x & - & - & - & - & - & - & - \\
			\ref{Frage:Konstruktion}    & Konstruktion%
				& X & X & X & - & X & - & - & x & - & - & - & - & - & - & - \\
			\ref{Frage:Vergleiche}      & Vergleiche%
				& X & X & X & - & - & X & - & x & - & - & - & - & - & - & - \\
			\hdashline[2pt/2pt]
			\ref{Frage:Definitionen}    & Definitionen%
				& X & X & X & - & X & - & x & - & - & - & - & - & - & - & - \\
			\ref{Frage:Abhängigkeiten}  & Abhängigkeiten%
				& X & X & X & - & X & - & x & - & - & - & - & - & - & - & - \\
			\ref{Frage:Überblick}       & Überblick%
				& X & X & X & - & - & X & x & - & - & - & - & - & - & - & - \\
			\hdashline[2pt/2pt]
			\ref{Frage:Darstellung}     & Darstellung%
				& X & - & X & - & X & - & x & - & - & - & - & - & - & - & - \\
			\ref{Frage:Forschung}       & Forschung%
				& X & X & X & - & - & X & x & - & - & - & - & - & - & - & - \\
			\hline
			\multicolumn{17}{l|}{Die nächsten beiden Punkte
			 sind Eigenschaften aus \sectionname~\vref{sec:Eigenschaften}:}\\
			\hline
			\ref{Eigenschaft:Lizenz}    & Lizenz%
				& - & - & - & - & - & - & - & - & - & - & - & - & - & - & X \\
			\ref{Eigenschaft:Akzeptanz} & Akzeptanz%
				& X & X & X & X & X & X & X & X & X & X & X & X & X & X & X \\
			\hline
		\end{tabularx}
		\caption{\ref{sec:Fragen} Fragen $\to$ \ref{sec:Ziele} Ziele}
		\label{tab:Fragen->Ziele}% Erst nach '\caption'!
	\end{table}
	\clearpage

	\section{Die Umgebung von \textsf{ASBA}}%===================================
	\beginsection{Die Umgebung von \textsf{ASBA}}
	\label{sec:Umgebung}

	In der \figurename~\vref{fig:Umgebung} wird beschrieben,
	welche Interaktionen \ASBA\ mit der Umgebung hat,
	\textdh welche Ein- und Ausgaben existieren
	und woher sie kommen \textbzw wohin sie gehen.
	\begin{figure}[h]
		\setlength\unitlength{1cm}
		\begin{picture}(17.0,7.0)(-8.4,-3.5)
%			% Gitter                                            %##
%			\color{lightgray}%                                  %##
%			\multiput(-8.4,-3.0)(+0.0,1.0){ 7}{\line(1,0){17.0}}%##
%			\multiput(-7.9,-3.5)(+1.0,0.0){17}{\line(0,1){ 7.0}}%##
			\linethickness{1.5pt}
			% Hintergrund (grau) ===============================================
			\color{gray}
			% rechts: externes ASBA mit Pfeilen --------------------------------
			\put(+3.00,-0.50){\framebox(2.40,1.60){\huge\textbf{\ASBA}}}
			\put(+3.00,-0.50){\makebox(2.40,1.50)[t]{\textbild{externes}}}
			\put(+4.00,+1.12){\vector(-1,+2){0.45}}% <--- externes ASBA
			\put(+4.00,+2.05){\vector(+1,-2){0.47}}% ---> externes ASBA
			\put(+3.81,+1.57){\marker{a}}
			% rechts oben: externe Datenbank mit Pfeilen -----------------------
			\put(+7.30,+2.30){\Datenbank{1.20}{0.40}{0.80}{\small externe}{\small Datenbank}}
			\put(+5.41,+1.10){\vector(+1,+1){0.70}}% <--- externes ASBA
			\put(+6.14,+1.39){\vector(-1,-1){0.72}}% ---> externes ASBA
			\put(+5.60,+1.23){\marker{b}}
			% Verbindung Auswertungen ---> Männchen ----------------------------
			\put(+5.60,-2.30){\vector(-1,0){4.05}}%  Auswertungen ---> Männchen
			\put(+3.50,-2.30){\marker{c}}
			% Verbindung Männchen <---> Terminal -------------------------------
			\put(-2.00,-2.00){\vector(+1,0){2.45}}% Männchen <--- Terminal
			\put(+0.40,-2.30){\vector(-1,0){2.40}}% Männchen ---> Terminal
			\put(-0.75,-2.15){\marker{d}}
			% Verbindung Terminal <---> Datei ----------------------------------
			\put(-5.50,-1.50){\vector(+3,-1){1.50}}% Terminal <--- Datei
			\put(-4.01,-2.30){\vector(-3,+1){1.60}}% Terminal ---> Datei
			\put(-5.00,-1.90){\marker{e}}
			% Vordergrund (schwarz) ============================================
			\color{black}
			% rechts oben: Wolke mit Pfeilen -----------------------------------
			\put(+3.40,+3.00){\Wolke{Internet}}
			\put(+2.00,+2.54){\vector(-1,-1){0.98}}% ---> ASBA
			\put(+1.53,+1.53){\vector(+1,+1){0.75}}% <--- ASBA
			\put(+1.50,+1.95){\Marker{1}}
			% links oben: Datenbank mit Pfeilen --------------------------------
			\put(-7.00,+2.00){\Datenbank{1.50}{0.50}{1.00}{Datenbank}{}}
			\put(-5.50,+2.20){\vector(+4,-1){3.95}}% ---> ASBA
			\put(-1.51,+0.90){\vector(-4,+1){4.00}}% <--- ASBA
			\put(-3.70,+1.55){\Marker{2}}
			% links Mitte: Datei mit Pfeilen -----------------------------------
			\put(-7.00,-1.00){\Datei{3.00}{2.00}{Datei}}
			\put(-5.50,-0.80){\vector(+4,+1){3.95}}% ---> ASBA
			\put(-1.51,-0.10){\vector(-4,-1){4.00}}% <--- ASBA
			\put(-3.70,-0.45){\Marker{3}}
			%links unten: Rechner mit Pfeilen ----------------------------------
			\put(-3.00,-2.10){\Terminal{Terminal}}
			\put(-2.50,-1.48){\vector(+1,+1){0.95}}% ---> ASBA
			\put(-1.10,-0.52){\vector(-1,-1){0.95}}% <--- ASBA
			\put(-2.00,-1.00){\Marker{4}}
			% Mitte unten: Männchen mit Pfeilen --------------------------------
			\put(+1.00,-1.80){\Maennchen}
			\put(+0.85,-1.50){\vector(0,+1){1.00}}% ---> ASBA
			\put(+1.15,-0.51){\vector(0,-1){1.00}}% <--- ASBA
			\put(+0.80,-1.02){\Marker{5}}
			% rechts unten: Papier mit Pfeil -----------------------------------
			\put(+5.60,-3.00){\Papier{+2.00}{+0.30}{Ausgabe}}
			\put(+1.51,-0.55){\vector(+4,-1){+4.10}}% <--- ASBA
			\put(+3.25,-1.05){\Marker{6}}
			% Mitte: ASBA ------------------------------------------------------
			\linethickness{3pt}
			\put(-1.5,-0.5){\framebox(3.0,2.0){\Huge\textbf{\ASBA}}}
		\end{picture}
		\caption{Die Umgebung von \ASBA}
		\label{fig:Umgebung}% Erst nach '\caption'!
	\end{figure}

	In den in \figurename~\vref{fig:Umgebung} abgebildeten Datenflüssen (1) bis
	(6) und (a) bis (e) werden die folgenden Daten übertragen:
	\begin{itemize}
		\newcommand*{\vonnach}[2]{#1 $\rightarrow$ #2}
		\newcommand*{\nachvon}[2]{\vonnach{#2}{#1}}
		\newcommand*{\hinundher}[2]{#1 $\leftrightarrow$ #2}

		\item[(1)]\label{dat:Internet}
		\begin{description}
			\item[\vonnach{\ASBA}{Internet}]\label{dat:ausInternet}
			Inhalte der Datenbank.
			\item[\nachvon{\ASBA}{Internet}]\label{dat:inInternet}
			Inhalte der externen Datenbank.
		\end{description}

		\item[(2)]\label{dat:Datenbank}
		\begin{description}
			\item[\vonnach{Datenbank}{\ASBA}]\label{dat:ausDatenbank}
			Inhalte der Datenbank und Antworten auf Datenbankanweisungen.
			\item[\nachvon{Datenbank}{\ASBA}]\label{dat:inDatenbank}
			Inhalte der Datei, der externen Datenbank und Datenbankanweisungen.
		\end{description}

		\item[(3)]\label{dat:Datei}
		\begin{description}
			\item[\vonnach{Datei}{\ASBA}]\label{dat:ausDatei}
			Inhalte der Datei.
			\item[\nachvon{Datei}{\ASBA}]\label{dat:inDatei}
			Die Datei wird um zusätzliche Auswertungen ergänzt,
			\textzB ob die \glsIdxPl{Beweis} korrekt sind,
			welche \glsIdxPl{Axiom} und \glsIdxPl{Satz}
			-- auch externe aus dem Internet --
			verwendet wurden,
			Anzahl der Beweisschritte \textusw
		\end{description}

		\item[(4)]\label{dat:Terminal}
		\begin{description}
			\item[\vonnach{Terminal}{\ASBA}]\label{dat:ausTerminal}
			Anweisungen, Daten und Batchprogramme.
			\item[\nachvon{Terminal}{\ASBA}]\label{dat:inTerminal}
			Antworten auf Anweisungen, Auswertungen \textusw
		\end{description}
		Außerdem interaktive Ein- und Ausgabe durch einen Anwender,
		wie in (5) beschrieben.

		\item[(5)]\label{dat:Anwender}
		\begin{description}
			\item[\hinundher{Anwender}{\ASBA}]\label{dat:mitAnwender}
			Interaktive Ein- und Ausgaben durch einen Anwender
			mit Komponenten von (3), (4) und (6).
			-- Die Kommunikation läuft \textiAlg über ein Terminal.
		\end{description}

		\item[(6)]\label{dat:Ausgabe}
		\begin{description}
			\item[\nachvon{Ausgabe}{\ASBA}]\label{dat:inAusgabe}
			Inhalte von Datei und Datenbank in lesbarer Form,
			\textua mit Hilfe von \glsIdxPl{Ausgabeschema} auch in Formelschreibweise.
			Die Ausgabe kann auch in eine Datei erfolgen,
			\textzB im \LaTeX-Format.
		\end{description}

		\item[(a)]\label{dat:extInternet}
		Nur die für \ASBA\ relevanten Daten:
		\begin{description}
			\item[\vonnach{Internet}{externes \ASBA}]\label{dat:ausextInternet}
			Inhalte der Datenbank.
			\item[\nachvon{Internet}{externes \ASBA}]\label{dat:inextInternet}
			Inhalte der externen Datenbank.
		\end{description}

		\item[(b)]\label{dat:extDatenbank}
		Nur die für \ASBA\ relevanten Daten:
		\begin{description}
			\item[\vonnach{externe Datenbank}{externes \ASBA}]
			\label{dat:ausextDatenbank} Inhalte der externen Datenbank.
			\item[\nachvon{externe Datenbank}{externes \ASBA}]
			\label{dat:inextDatenbank} Inhalte der Datenbank.
		\end{description}

		\item[(c)]\label{dat:AusgabeAnwender}
		\begin{description}
			\item[\vonnach{Ausgabe}{Anwender}]\label{dat:Ausgabe2Anwender}
			Alle Daten der Ausgabe.
		\end{description}

		\item[(d)] \label{dat:AnwenderTerminal}
		\begin{description}
			\item[\hinundher{Anwender}{Terminal}]\label{dat:Anwender22Terminal}
			Interaktive Ein- und Ausgabe durch einen Anwender,
			wie in (5) beschrieben.
		\end{description}

		\item[(e)] \label{dat:TerminalDatei}
		\begin{description}
			\item[\hinundher{Terminal}{Datei}]\label{dat:Terminal22Datei}
			Erstellen und Bearbeiten der Datei durch einen Anwender.
			-- siehe (d)
		\end{description}

	\end{itemize}
	Die Datenflüsse (a) bis (e) erfolgen außerhalb von \ASBA
	und werden nicht weiter behandelt.

	Die Datenbank und die Datei enthalten im Prinzip die gleichen Daten,
	wobei sie in der Datei im Textformat in lesbarer Form
	und in der Datenbank in einem internen Format vorliegen.
	Zudem enthält die Datenbank \textiAlg sehr viel mehr Daten.
	Es handelt sich dabei jeweils um die folgenden Daten:
	\begin{description}

		\item[\glsIdxPl{Axiom}]         \label{Daten:Axiom}
		Ein \glsIdx{Axiom} ist eine Aussage oder Behauptung,
		die nicht aus anderen Aussagen abgeleitet werden kann.
		Es können wie bei Sätzen Voraussetzungen vorhanden sein,
		aber keine \glsIdx{Beweis}.

		\item[\glsIdxPl{Satz}]          \label{Daten:Satz}
		Ein \glsIdx{Satz} besteht aus einer Anzahl von Voraussetzungen,
		einer Behauptung und einem \glsIdx{Beweis},
		der die Behauptung aus den Voraussetzungen ableitet.
		Letztere können \glsIdxPl{Axiom} und andere Sätze sein,
		auf die dann verwiesen wird.

		\item[\glsIdxPl{Beweis}]        \label{Daten:Beweis}
		Ein \glsIdx{Beweis} besteht aus einer Folge von
		\glsIdxBg{Beweisschritt}{Beweisschritten},
		die aus gegebenen Voraussetzungen eine Behauptung ableitet.

		\item[\glsIdxPl{Fachbegriff}]   \label{Daten:Fachbegriff}
		Ein \glsIdx{Fachbegriff} ist ein Name für ein \glsIdx{Praedikat}
		in einem bestimmten \glsIdx{Fachgebiet}.

		\item[\glsIdxPl{Fachgebiet}]    \label{Daten:Fachgebiet}
		Ein Fachgebiet ist ein Teil der Mathematik mit einer zugehörigen Basis
		von \glsIdxPl{Axiom}, \glsIdxPl{Satz}, \glsIdxPl{Fachbegriff} und
		\glsIdxPl{Ausgabeschema},
		quasi eine untergeordnete Datenbank.

		\item[\glsIdxPl{Ausgabeschema}] \label{Daten:Ausgabeschema}
		Eine Beschreibung,
		wie ein bestimmtes mathematisches Objekt ausgegeben werden soll.
		Dies kann \textzB ein Stück
		\LaTeX-Code mit entsprechenden Parametern sein.

		\item[Auswertungen]             \label{Daten:Auswertung}
		Statistische und andere Auswertungen,
		die bestimmten Elementen der Datei \textbzw Datenbank zugeordnet sind.
		\textZB können zu einem \glsIdx{Satz}
		alle für einen Beweis notwendigen \glsIdxPl{Axiom} angegeben werden
		-- als Verweise.

	\end{description}
	Die Daten können interne und externe Verweise enthalten.

	\beforechapter% Kopfzeile aktivieren und neue Seite
	\chapter{Mathematische Grundlagen}%%%%%%%%%%%%%%%%%%%%%%%%%%%%%%%%%%%%%%%%%%
	\beginchapter{Mathematische Grundlagen}
	\label{cha:Grundlagen}

	\section{Metasprache}%======================================================
	\beginsection{Metasprache}
	\label{sec:Metasprache}

	Wenn man über eine Sprache spricht, braucht man auch eine Sprache,
	in der Aussagen über die erstere getroffen werden können.
	Wenn die zuerst genannte Sprache die der Mathematik ist,
	nimmt man üblicherweise die natürliche Sprache als \glsIdx{Metasprache}.
	Leider ist diese oft ungenau,
	nicht immer eindeutig und abhängig vom Zusammenhang,
	in dem sie gesprochen wird%
	\footnote{%
		Man betrachte die beiden Aussagen
		\enquote{Studenten und Rentner zahlen die Hälfte.}
		und
		\enquote{Studenten oder Rentner zahlen die Hälfte.},
		die beide das gleiche meinen.
		-- Entnommen aus \cite{bib:Rautenberg} \sectionname~1.2 Bemerkung 1.

		Ein weiteres Problem ist,
		dass man unauflösbare Widersprüche formulieren kann,
		\textzB \enquote{%
			Der Barbier ist der Mann im Ort,
			der genau die Männer im Ort rasiert,
			die sich nicht selbst rasieren.%
		}.
		Und der Barbier?
		Wenn er sich selbst rasiert,
		dann rasiert er sich nicht selbst,
		und wenn er sich nicht selbst rasiert,
		dann rasiert er sich selbst.
		Was denn nun?
		-- Quelle unbekannt) --
		Das Problem ist verwandt mit dem Problem der
		\enquote{Menge aller Mengen, die sich nicht selbst enthalten}.%
	}.
	Um diese Probleme in den Griff zu bekommen,
	wird die \glsIdx{Metasprache} teilweise formalisiert.
	Durch diese Formalisierung
	erinnert sie dann teilweise schon an mathematische Formeln.
	Die Sprachebenen müssen aber sorgfältig unterschieden werden.

	\subsection{Metasprachliche Ausdrücke}%=====================================
	\beginsection{\glsIdxPl{MetaausdruckV}}
	\label{sub:Metaausdruck}

	Ein \emph{\glsIdx{MetaausdruckV}}
	ist eine in normaler Sprache verfasste Aussage,
	wie \textzB
	(a) \strqt{Morgen scheint die Sonne.},
	(b) \strqt{Ich bin 1,83\,m groß.},
	(c) \strqt{Ich habe ein rotes Auto und das kann 200\,km/h schnell fahren.},
	\textusw
	In einem erweiterten Sinne gehören auch Relationen
	einschließlich ihrer Operanden dazu%
	\footnote{%
		Wird statt des Symbols der Name der zugehörigen Relation verwendet,
		ist dies unmittelbar einleuchtend.
		So wird \textzB aus der Formel \forqt{$A<B$}
		die \glsIdx{MetaaussageV} \strqt{$A$ ist kleiner als $B$}.%
	},
	wie \textzB \forqt{$A=A$}, \forqt{$A \equiv B$}, \forqt{$A<B$}, \textusw

	Während die Beispiele (a) und (b) einfache,
	nicht mehr zerlegbare \glsIdxPl{MetaausdruckV} sind,
	ist Beispiel (c) zusammengesetzt.
	Für alle drei Aussagen lässt sich feststellen,
	ob sie richtig sind oder nicht.
	Das kann man für den zweiten Teil von (c) allein aber nicht,
	wenn man nicht weiss worauf sich \strqt{das} bezieht.
	Natürlich muss auch der Zusammenhang,
	in dem ein \glsIdx{MetaausdruckV} formuliert wird,
	bekannt sein,
	denn \textzB ist die Bedeutung von \strqt{Ich} nur dann bekannt,
	wenn man weiß von wem die Aussage ist.
	Auf eine exakte Definition von \strqt{\glsIdx{MetaausdruckV}}
	wird verzichtet, weil das intuitive Verständnis hier ausreicht.
	In erster Näherung können aber alle sprachlichen Ausdrücke,
	die im Prinzip überprüft werden können,
	als \glsIdxPl{MetaausdruckV} betrachtet werden.

	Zusammengesetzte \glsIdxPl{MetaausdruckV} wie (c)
	können zum Teil formalisiert werden.
	Dies wird mit den folgenden Definitionen erreicht:

	Sind $A$ und $B$ \glsIdxPl{MetaausdruckV}, dann wird definiert:
	\begin{itemize}

		\item \forqt{$A \glsSym{metaimp}   B$} steht für
		\strqt{\emph{Wenn} $A$ [gilt] \emph{dann} [gilt] [auch] $B$}.

		\item \forqt{$A \glsSym{metarep}   B$} steht für
		\strqt{$A$ [gilt] \emph{sofern} $B$ [gilt]}.

		\item \forqt{$A \glsSym{metaequiv} B$} steht für
		\strqt{$A$ [gilt] \emph{genau dann wenn} $B$ [gilt]}.

		\item \forqt{$A \glsSym{metaand}   B$} steht für
		\strqt{[Es gilt] $A$ \emph{und} $B$}.

		\item \forqt{$A \glsSym{metaor}    B$} steht für
		\strqt{[Es gilt] $A$ \emph{oder} $B$}.

	\end{itemize}

	Offensichtlich sind das alles ebenfalls \glsIdxPl{MetaausdruckV},
	jetzt aber zumindest teilweise formalisiert.
	(c) lässt sich dann ausdrücken als
	\strqt{\strqt{Ich habe ein rotes Auto} $\metaund$
		\strqt{das kann 200\,km/h schnell fahren.}}.

	Um Verwechslungen mit den logischen Symbolen zu vermeiden,
	werden für \strqt{und} und \strqt{oder}
	die Symbole \symqt{$\metaund$} und \symqt{$\metaoder$} verwendet.
	Ein Symbol für \strqt{nicht} wird hier nicht gebraucht.

	\GlsIdxPl{MetaausdruckV} können auch geklammert werden,
	um die Reihenfolge der Auswertung eindeutig zu machen.
	\symqt{$\metaimp$}, \symqt{$\metarep$}, \symqt{$\metaequiv$},
	\symqt{$\metaand$} und \symqt{$\metaoder$}
	heißen \emph{\glsIdxPl{MetaoperatorV}}.
	Ihre Prioritäten werden im \subsectionname~\vref{sub:Klammerregeln}
	zusammen mit anderen Operatoren definiert.

	Sollen zwei \glsIdxPl{MetaausdruckV} miteinander verglichen werden,
	muss klar sein auf welche Art;
	ob \textzB als Zeichenfolgen
	-- mit oder ohne Wertung der Zwischenräume --,
	als Wahrheitswerte
	oder auf sonstige Art.
	Wenn die Art des Vergleichs implizit oder explizit klar ist
	und sich die beiden Ausdrücke damit vergleichen lassen,
	heißen sie \emph{\glsIdx{vergleichbar}}.

	\subsection{Mit Gleichheit verwandte Symbole}%------------------------------
	\label{sub:Gleichheit}

	In diesem \sectionname{} wird vorausgesetzt:
	\begin{itemize}

		\item {$=$}, $\ne$, $\equiv$ und $\defeq$
		werden im selben Zusammenhang verwendet.

		\item $A$, $B$, $P$ und $Q$ sind \glsIdx{vergleichbar},
		\textdh Ausdrücke derselben Art.

	\end{itemize}
	Dann werden folgende Symbole definiert:
	\begin{itemize}

		\item $\glsSym{eq}$~~(\emph{\Idx{Gleichheit}}):
		\forqt{$A = B$} heißt,
		dass $A$ und $B$ sich in den \glsIdxPl{intEigenschaftA}
		nicht unterscheiden.
		Welche das sind,
		ergibt sich normalerweise aus dem Zusammenhang%
		\footnote{%
			Statt von einem \emph{Zusammenhang}
			könnte man auch von einer \emph{Umgebung} sprechen.
			Diese Bezeichnung ist aber auch ein verbreiteter Fachbegriff.
			Die Exaktheit der Begriffe in diesem Dokument
			soll für Erstellung von \ASBA\ ausreichen;
			was darüber hinausgeht,
			ist nicht Inhalt dieses Dokuments.%
		}
		oder muss explizit angegeben werden.
		\textZB sind zwei Operatoren gleich,
		wenn sie stets denselben \emph{\gls{Wahrheitswert}} liefern.

		Inwieweit die Begriffe \emph{Gleichheit} und \emph{Identität}
		korrelieren,
		wird hier nicht erörtert.
		\seename~\cite{bib:Identitaet}

		\item $\glsSym{ne}$~~(\emph{\Idx{Ungleichheit}}):
		\forqt{$A \ne B$} heißt,
		dass $A$ und $B$ sich in mindestens einer der
		\glsIdxPl{intEigenschaftA} unterscheiden.

		\item $\glsSym{equiv}$~~(\emph{\Idx{Äquivalenz}}):
		\forqt{$A \equiv B$} heißt,
		dass $A$ und $B$ sich in den
		\glsIdxPl{intEigenschaftA} nicht unterscheiden.
		Welche das sind,
		ergibt sich wie bei \symqt{$=$} aus dem Zusammenhang
		oder wird explizit angegeben.
		Werden \symqt{$=$} und \symqt{$\equiv$} im selben Zusammenhang verwendet,
		muss mit \forqt{$A=B$} stets auch \forqt{$A \equiv B$} gelten,
		\textdh alle \glsIdxPl{intEigenschaftA}
		von \symqt{$\equiv$}
		müssen auch \glsIdxBg{intEigenschaftA}{interessierende Eigenschaften}
		von \forqt{$A=B$} sein.

		\item $\glsSym{defeq}$~~(\emph{\Idx{Definition}}):
		\forqt{$A \defeq B$} bedeutet,
		dass $A$ definitionsgemäß gleich $B$ ist.
		Gewissermaßen ist $A$
		nur eine andere Schreibweise für $B$.
		$A$ und $B$ können sich gegenseitig ersetzten.

		Nach dieser Definition sind $P$ und $Q$ schon dann gleich,
		wenn nach der Ersetzung aller Vorkommen von $A$ in $P$ und $Q$ durch $B$
		die resultierenden Ausdrücke $\overline{P}$ und $\overline{Q}$
		gleich sind.

	\end{itemize}

	Es gilt dann:
	\begin{itemize}

		\item Die interessierenden Eigenschaften für \symqt{$\equiv$}
		sind auch solche für \symqt{$=$}.

		\item Für \symqt{$=$} können
		-- müssen aber nicht --
		mehr Eigenschaften von Interesse sein als für \symqt{$\equiv$}.

		\item $(A = B) \metaor (A \ne B)$

		\item $(A = B) \metaimp (A \equiv B)$

	\end{itemize}

	Und unter den Voraussetzungen:
	\begin{itemize}

		\item $A \defeq B$

		\item $\overline{P}$ ergibt sich aus $P$ durch
		Ersetzung aller Vorkommen (Teilausdrücke) $A$ durch $B$.

		\item $\overline{Q}$ ergibt sich aus $Q$ durch
		Ersetzung aller Vorkommen (Teilausdrücke) $A$ durch $B$.

	\end{itemize}

	gilt schließlich:
	\begin{itemize}

		\item $(\overline{P} = \overline{Q}) \metaequiv (P = Q)$

	\end{itemize}

	Schließlich sei noch
	\begin{equation}
		\glsSym{asM} \defeq \{
		\glsIdxBg{metaand}{\metaund}, \glsIdxBg{metaor}{\metaoder},
		\glsSym{metaimp}, \glsSym{metaequiv}, \glsSym{metarep},
		\glsSym{eq}, \glsSym{ne}, \glsSym{equiv}, \glsSym{defeq} \}
		\label{for:asM}
	\end{equation}
	die Menge der Metaoperatoren und der mit Gleichheit verwandten Symbole

	\section{Formale Elemente}%=================================================
	\beginsection{Formale Elemente}
	\label{sec:Formalelement}

	Ein \emph{\glsIdx{FormalelementV}} kann \textzB
	eine Menge, Zeichenfolge, Zahl, Formel, \textusw sein.
	Zwei \glsIdxPl{FormalelementV} $A$ und $B$ sind \emph{\glsIdx{vergleichbar}},
	wenn beide von derselben Art sind,
	\textdh wenn \textzB jeweils beide
	Mengen, Zeichenfolgen, Zahlen oder Formeln
	-- die vergleichbare Ergebnisse liefern --
	sind.

	Intuitiv scheint klar zu sein,
	was damit  gemeint ist.
	Wenn aber entschieden werden muss,
	ob \textzB (a) \strqt{1+1} gleich \strqt{2}
	oder (b) \strqt{1+1} gleich \strqt{1 + 1} ist,
	muss man erst entscheiden,
	von welcher Art die beiden zu vergleichenden Ausdrücke sind,
	\textdh \emph{wie} verglichen wird.
	Wenn sie als jeweiliges Ergebnis der beiden Formeln verglichen werden,
	dann ist (a) richtig.
	Wenn sie als Formeln,
	\textdh als Zeichenfolgen,
	verglichen werden ist (a) falsch.
	Wenn die Ausdrücke in (b) als Zeichenfolgen verglichen werden,
	ist (b) dann richtig,
	wenn der Zwischenraum zwischen den einzelnen Zeichen nicht zählt.
	Wenn er aber zählt, ist (b) falsch.

%##	Bei der Entwicklung von ASBA
%##	müssen diese Feinheiten stets berücksichtigt werden.

	Im Zusammenhang mit binären Relationen werden noch einige Verabredungen
	getroffen.
	Dazu seien
	\symqt{$\glsSym{relbsp}$},       \symqt{$\glsSym{releqbsp}$},
	\symqt{$\glsSym{lrelbsp}$},      \symqt{$\glsSym{rrelbsp}$},
	\symqt{$\glsSym{lreleqbsp}$} und \symqt{$\glsSym{rreleqbsp}$}
	Beispielsymbole für Relationen und
	\symqt{$\glsSym{eq}$}        und \symqt{$\glsSym{ne}$} die in \sectionname~\vref{sub:Gleichheit}
	definierten Symbole für Gleichheit und Ungleichheit.
	Wenn dann nichts anderes gesagt wird,
	gelte stets:
	\begin{align}
		& ((A \relbsp   B) \metaor (A = B)) & \metaequiv &&& (A \releqbsp  B)
		\label{eq:coreleq}   \\
		& (A \lrelbsp   B)                  & \metaequiv &&& (B \rrelbsp   A)
		\label{eq:colrrel}   \\
		& (A \lreleqbsp B)                  & \metaequiv &&& (B \rreleqbsp A)
		\label{eq:colrreleq} \formulatoleft
	\end{align}
	Mit der Definition einer Relation der einen Seite
	ist damit automatisch auch die der anderen Seite erfolgt,
	mit der Ausnahme,
	dass man \forqt{$A \relbsp B$} so nicht mit Hilfe von \forqt{$A \releqbsp B$}
	definieren kann.
	Dies könnte man zwar mit Hilfe des Ansatzes
	\begin{align}
		& (A \relbsp B) &\formulaspace \metaequiv &&&
		(A \releqbsp B) \metaand (A \ne B) \label{eq:corel} \formulatoleft
	\end{align}
	versuchen,
	aber die so definierte Relation \symqt{$\relbsp$} kann,
	muss aber nicht mit der in \vref{eq:coreleq} übereinstimmen.
	Allerdings lässt sich \vref{eq:coreleq} aus \vref{eq:corel} ableiten
	und wenn \forqt{$(A = B) \metaimp (A \releqbsp B)$} gilt,
	auch \vref{eq:corel} aus \vref{eq:coreleq}.
	-- Auf einen Beweis verzichten wir hier.

	Es sei noch angemerkt, dass wegen \vref{eq:corel}
	die Definition von \symqt{$\metarep$} in \sectionname~\vref{sub:Metaausdruck}
	überflüssig ist
	und wegen der Klammerregeln
	(\seename \subsectionname~\vref{sub:Klammerregeln})
	auch alle Klammern in diesem \sectionname~\ref{sec:Formalelement}.
	Die Prioritäten der Operatoren \symqt{$\lrelbsp$}, \symqt{$\rrelbsp$},
	\symqt{$\lreleqbsp$} und \symqt{$\rreleqbsp$}
	unterscheiden sich normalerweise nicht;
	ebensowenig die der Operatoren \symqt{$\relbsp$} und \symqt{$\releqbsp$},
	die aber durchaus verschieden von den Prioritäten
	von \symqt{$=$} und \symqt{$\ne$} sein können.

	Als Beispielsymbol für binäre Operatoren wird \symqt{$\opbsp$} verwendet.
	Damit zusammenhängende Verabredungen werden hier nicht getroffen.

	\section{Aussagenlogik}%====================================================
	\beginsection{Aussagenlogik}
	\label{sec:Aussagenlogik}
	\hidden{\glsIdx{Aussagenlogik}}

	\subsection{Konstante und Operatoren}%--------------------------------------
	\label{sub:Operatoren}

	Die \tablename~\vref{tab:Symbole}
	\footnote{%
		Die \tablename{} basiert auf den Wahrheitstafeln in~\cite{bib:Junktor}
		Kapitel~2.2 und~\cite{bib:Rautenberg} Kapitel~1.1 Seite~3.%
	}
	definiert für die zweiwertige Logik Konstanten- und Operatorsymbole
	über die Wahrheitswerte ihrer Anwendung.
	So ergeben sich,
	abhängig von den Wahrheitswerten der Operanden $A$ und $B$
	\footnote{%
		Im Gegensatz zu \subsubsectionname~\vref{subsub:Bausteine}
		können A und B hier beliebige Aussagen -- auch Formeln -- sein.%
	},
	die in der \tablename{} angegebenen Wahrheitswerte für die Operationen.
	Die mit 0, 1 und 2 benannten Spalten werden jeweils
	nur für die 0-, 1- und 2-stelligen Operatoren, \textdh
	für die Konstanten, die unären und die binären Operatoren ausgefüllt.
	Dabei werden die Konstanten als 0-stellige Operatoren angesehen.
	Hat der Inhalt einer Zelle keine Relevanz, steht dort ein Minuszeichen,
	ist kein Wert bekannt, so bleibt sie leer.

	% ==========================================================================
	% Definitionen für die folgende Tabelle; siehe auch die Symbole im Vorspann
	% Prioritäten - jeweils Prio p* für Symbol \l* -----------------------------
	\newcounter{prio}                                        \stepcounter{prio}
	\newcounter{pnequiv} \setcounter{pnequiv} {\value{prio}}
	\newcounter{pequiv}  \setcounter{pequiv}  {\value{prio}} \stepcounter{prio}
	\newcounter{pnrep}   \setcounter{pnrep}   {\value{prio}}
	\newcounter{prep}    \setcounter{prep}    {\value{prio}}
	\newcounter{pnimp}   \setcounter{pnimp}   {\value{prio}}
	\newcounter{pimp}    \setcounter{pimp}    {\value{prio}} \stepcounter{prio}
%	\newcounter{pnleft}  \setcounter{pnleft}  {\value{prio}}
%	\newcounter{pleft}   \setcounter{pleft}   {\value{prio}}
%	\newcounter{pnright} \setcounter{pnright} {\value{prio}}
%	\newcounter{pright}  \setcounter{pright}  {\value{prio}} \stepcounter{prio}
	\newcounter{padd}    \setcounter{padd}    {\value{prio}}
%	\newcounter{pnxor}   \setcounter{pnxor}   {\value{prio}}
	\newcounter{pxor}    \setcounter{pxor}    {\value{prio}}
	\newcounter{pnor}    \setcounter{pnor}    {\value{prio}}
	\newcounter{por}     \setcounter{por}     {\value{prio}} \stepcounter{prio}
	\newcounter{pmult}   \setcounter{pmult}   {\value{prio}}
	\newcounter{pnand}   \setcounter{pnand}   {\value{prio}}
	\newcounter{pand}    \setcounter{pand}    {\value{prio}} \stepcounter{prio}
	\newcounter{pnot}    \setcounter{pnot}    {\value{prio}}
	% Farben
	\definecolor{cNormalUse}{rgb}{.80,.80,.80}
	\definecolor{cRareUse}{rgb}{.90,.90,.99}
	% ==========================================================================

	\begin{table}
		\newcommand*{\tablegroup}{\hdashline[6pt/3pt]}
		\newcommand*{\tableline}{\hdashline[3pt/3pt]}
		\newcommand*{\gapline}{%
			\cdashline{1-1}[1pt/3pt]\cdashline{9-11}[1pt/3pt]}
		\setlength\tabcolsep{3pt}
		\setlength\extrarowheight{1.5pt}
		\begin{threeparttable}
			\begin{tabularx}{\linewidth-10.95pt}{c||c:cc:cccc|X:X|c|}

				A & - & \texttrue & \textfalse &%
				\texttrue  & \texttrue  & \textfalse & \textfalse &
				- & Aussage A & - \\

				\tableline%.................................................
				B & - & -       & -        &%
				\texttrue  & \textfalse & \texttrue  & \textfalse &
				- & Aussage B & - \\

				\hline%--- Überschrift -----------------------------------------

				\textbf{Junktor}\tnote{1}&\textbf{0}&\multicolumn{2}{c:}{%
				\textbf{1}}&\multicolumn{4}{c|}{\textbf{2}}& \textbf{%
				Name}&\textbf{Sprechweise}\tnote{2}&\textbf{Prio}\\%Kein Fehler!

				\hline\hline%=== Konstante =====================================

				\rowcolor{cRareUse}
				$\glsSym{ltrue}$
				& \texttrue  & - & - & - & - & - & - & Verum  & Wahr   & - \\

				\tableline%.................................................

				\rowcolor{cRareUse}

				$\glsSym{lfalse}$
				& \textfalse & - & - & - & - & - & - & Falsum & Falsch & - \\

				\hline%--- unäre Operatoren ------------------------------------

				& - & \texttrue  & \texttrue  & - & - & - & -
				&                     &                  & -                 \\

				\tableline%.................................................

				\rowcolor{cNormalUse}

				$\Sym{(\dots)}$
				& - & \texttrue  & \textfalse & - & - & - & -
				& Klammerung\tnote{3} & A ist geklammert & 6\tnote{4}        \\

				\tableline%.................................................

				\rowcolor{cNormalUse}
				$\Sym{\lnot}$
				& - & \textfalse & \texttrue  & - & - & - & -
				& Negation            & Nicht A          & \thepnot\tnote{5} \\

				\tableline%.................................................

				& - & \textfalse & \textfalse & - & - & - & -
				&                     &                  & -                 \\

				\hline%--- binäre Operatoren -----------------------------------

				~ & - & - & - &\texttrue&\texttrue&\texttrue&\texttrue
				& Tautologie
				&
				& - \\

				\tableline%.................................................

				\rowcolor{cNormalUse}

				$\Sym{\lor}$
				& - & - & - &\texttrue&\texttrue&\texttrue&\textfalse
				& Disjunktion; Adjunktion;\newline Alternative
				& A oder B
				& \thepor \\

				\tableline%.................................................

				\rowcolor{cRareUse}
				$\Sym{\lrep}$ $\lrepA$ $\lrepB$
				& - & - & - &\texttrue&\texttrue&\textfalse&\texttrue
				& Replikation; Konversion;\newline konverse Implikation
				& A folgt aus B
				& \theprep \\

				\tableline%.................................................

				$\lleft$
				& - & - & - &\texttrue&\texttrue&\textfalse&\textfalse
				& Präpendenz
				& Identität von A
				& - \\

				\tablegroup%------------------------------------------------

				\rowcolor{cNormalUse}
				$\Sym{\limp}$ $\limpA$ $\limpB$
				& - & - & - &\texttrue&\textfalse&\texttrue&\texttrue
				& Implikation; Subjunktion;\newline Konditional
				& Wenn A so B; Aus A folgt B; A nur dann wenn B
				& \thepimp \\

				\tableline%.................................................

				$\lright$
				& - & - & - &\texttrue&\textfalse&\texttrue&\textfalse
				& Postpendenz
				& Identität von B
				& - \\

				\tableline%.................................................

				\rowcolor{cNormalUse}
				$\Sym{\lequiv}$ $\lequivA$
				& - & - & - &\texttrue&\textfalse&\textfalse&\texttrue
				& Äquivalenz; Bijunktion;\newline Bikonditional
				& A genau dann wenn B; A dann und nur dann wenn B
				& \thepequiv \\

				\tableline%.................................................

				\rowcolor{cNormalUse}
				$\Sym{\land}$ $\landA$ $\landB$
				& - & - & - &\texttrue&\textfalse&\textfalse&\textfalse
				& Konjunktion
				& {\small A und B; Sowohl A als auch B}
				& \thepand \\

				\tablegroup%------------------------------------------------

				\rowcolor{cRareUse}
				$\Sym{\lnand}$ $\lnandA$ $\lnandB$
				& - & - & - &\textfalse&\texttrue&\texttrue&\texttrue
				& NAND; Unverträglichkeit;\newline Sheffer-Funktion
				& Nicht zugleich A und B
				& \thepnand \\

				\tableline%.................................................

				\rowcolor{cRareUse}
				$\Sym{\lxor}$ $\lxorA$ $\lxorB$ $\lxorC$
				& - & - & - &\textfalse&\texttrue&\texttrue&\textfalse
				& XOR; Antivalenz;\newline ausschließende Disjunktion
				& Entweder A oder B
				& \thepxor \\

				\gapline%. . . . . . . . . . . . . . . . . . . . . . . . . .

				$\lnequiv$ $\lnequivA$ $\lnequivB$
				& - & - & - &"&"&"&"
				& Kontravalenz
				&
				& - \\

				\tableline%.................................................

				$\lnright$
				& - & - & - &\textfalse&\texttrue&\textfalse&\texttrue
				& Postnonpendenz
				& Negation von B
				& - \\

				\tableline%.................................................

				$\lnimp$ $\lnimpA$ $\lnimpB$
				& - & - & - &\textfalse&\texttrue&\textfalse&\textfalse
				& Postsektion
				&
				& - \\

				\tablegroup%------------------------------------------------

				$\lnleft$
				& - & - & - &\textfalse&\textfalse&\texttrue&\texttrue
				& Pränonpendenz
				& Negation von A
				& - \\

				\tableline%.................................................

				$\lnrep$ $\lnrepA$ $\lnrepB$
				& - & - & - &\textfalse&\textfalse&\texttrue&\textfalse
				& Präsektion
				&
				& - \\

				\tableline%.................................................

				\rowcolor{cRareUse}
				$\Sym{\lnor}$ $\lnorA$
				& - & - & - &\textfalse&\textfalse&\textfalse&\texttrue
				& NOR; Nihilation;\newline Peirce-Funktion
				& Weder A noch B
				& \thepnor \\

				\tableline%.................................................

				~
				& - & - & - &\textfalse&\textfalse&\textfalse&\textfalse
				& Kontradiktion
				&
				& - \\

				\hline%_________________________________________________________
			\end{tabularx}
			\begin{tablenotes}
				\footnotesize

				\item[1] \emph{Operatorsymbole}. Sie stehen meistens für die Operatoren selbst.
				Der Einfachheit halber bezeichnen wir auch die beiden Konstanten
				$\ltrue$ und $\lfalse$ als Junktoren \textbzw Operatorsymbole.

				Die Operatoren \symqt{$\subset$}, \symqt{$\supset$},
				\symqt{$\nsubset$} und \symqt{$\nsupset$} haben hier nicht die
				Bedeutung der entsprechenden Operatoren der Mengenlehre
				und dürfen nicht damit verwechselt werden;
				entsprechendes gilt für \symqt{$+$} und \symqt{$\cdot$}
				mit Addition und Multiplikation.

				\item[2] Ist eine Zelle in dieser Spalte leer,
				so ist die zugehörige Zeile nur vorhanden,
				um alle binären Operationen aufzuführen.

				\item[3] Klammerung ist genau genommen kein Operator
				und wird nicht nur bei logischen,
				sondern auch bei anderen Ausdrücken verwendet.

				\item[4] Die Priorität der Klammern
				ist größer als die aller Operatoren.

				\item[5] Die Priorität der unären Operatoren
				muss größer sein als die aller mehrwertigen,
				also auch der binären Operatoren.
				Wenn alle unären Operatoren
				auf derselben Seite des Operanden stehen,
				brauchen sie eigentlich keine Priorität,
				da die Auswertung nur von innen (dem Operanden)
				nach außen erfolgen kann.
				Nur wenn es sowohl links-, als auch rechtsseitige
				unäre Operatoren gibt,
				muss man für diese Prioritäten definieren.

			\end{tablenotes}
			\caption{Definition von aussagenlogischen Symbolen.}
			\label{tab:Symbole}% Erst nach '\caption'!
		\end{threeparttable}
	\end{table}

	Für einige Junktoren, Namen und Sprechweisen
	sind auch Alternativen angegeben.
	Die durchgestrichenen (\textdh negierten) Symbole sind ungebräuchlich
	und nur aus formalen Gründen aufgeführt.
	Wenn für eine bestimmte Kombination von Wahrheitswerten
	mehr als eine Zeile angegeben ist,
	so sind die zugehörigen Operationen
	in der zweiwertigen Aussagenlogik alle gleich.
	Bei der formalen Definition setzen wir aber keine Zweiwertigkeit voraus,
	so dass je nach Definition
	die Operationen verschiedene Ergebnisse liefern können.

	Um vollständig zu sein,
	\textdh alle 22 möglichen Kombinationen von Wahrheitswerten
	für höchstens zwei Variable zu berücksichtigen,
	enthält die \tablename{}
	auch viele ungebräuchliche Junktoren und Operationen.
	Die Zeilen mit den Klammern und den gebräuchlichsten Junktoren
	sind in der \tablename{} grau hinterlegt.
	Hellgrau hinterlegt sind Zeilen mit weniger gebräuchlichen Junktoren.
	Die restlichen Operationen sind uninteressant
	und brauchen daher keine Priorität.

	\subsection{Klammerregeln}%-------------------------------------------------
	\label{sub:Klammerregeln}

	Zur Klammerersparnis werden die üblichen Regeln verwendet,
	\textdh dass Operatoren mit höherer Priorität stärker binden,
	als solche mit niedrigerer Priorität.

	Für die Operatoren derselben Priorität gilt Rechtsklammerung%
	\footnote{%
		Unäre Operatoren stehen hier stets links \emph{vor} dem Operanden,
		so dass es nur Rechtsklammerung geben kann.
		Zur Rechtsklammerung bei binären Operationen ein Zitat
		aus~\cite{bib:Rautenberg} Kapitel~1.1 Seite~5:
		\enquote{Diese hat gegenüber Linksklammerung Vorteile
		bei der Niederschrift von Tautologien in $\limp$, [...]}%
	}.
	Im Folgenden wird nur noch ein Teil der logischen Operatoren aus der
	\tablename~\vref{tab:Symbole} und
	der \glsIdxBg{MetaoperatorV}{metasprachlichen Operatoren}
	aus \subsectionname~\vref{sub:Metaausdruck}
	berücksichtigt.
	Diese werden in der \tablename~\vref{tab:Prioritaeten}
	mit abnehmender Priorität aufgelistet.
	\begin{table}
		\begin{center}
			\begin{tabular}[c]{|l|l|}
				\hline
				Klammern
				& $      (      \quad      )                          $ \\
				\hline
				Unäre logische Operatoren
				& $ \Sym{\lnot}                                       $ \\
				\hdashline
				& $ \Sym{\land} \quad \Sym{\lmult} \quad \Sym{\lnand} $ \\
				Binäre logische Operatoren
				& $ \Sym{\lor}  \quad \Sym{\ladd}  \quad \Sym{\lnor}  $ \\
				& $ \Sym{\lrep} \quad \Sym{\limp}                     $ \\
				& $ \Sym{\lequiv}                                     $ \\
				\hline
				\parbox[][1.5cm][c]{6.2cm}{%
					Mit Gleichheit verwandte Symbole;
					\small ihre Prioritäten untereinander sind nicht eindeutig
					und bleiben daher undefiniert.
				}
				& $ \glsSym{eq} \quad \glsSym{ne} \quad
				\glsSym{equiv}  \quad \glsSym{defeq}        $ \\
				\hline
				& $ \glsSym{metaor}                         $ \\
				\GlsIdxPl{MetaoperatorV}
				& $ \glsSym{metaand}                        $ \\
				& $ \glsSym{metarep} \quad \glsSym{metaimp} $ \\
				& $ \glsSym{metaequiv}                      $ \\
				\hline
			\end{tabular}
		\end{center}
		\vspace{-0.5cm}
		\caption{Prioritäten von Operatoren in abnehmender Reihenfolge}
		\label{tab:Prioritaeten}% Erst nach '\caption'!
	\end{table}
	Die Prioritäten der logischen Operatoren wurden aus~\cite{bib:Rautenberg}
	Kapitel~1.1 Seite~5 entnommen und ergänzt und die der
	\glsIdxBg{MetaoperatorV}{metasprachlichen Operatoren} daran angeglichen.

	\subsection{Formalisierung}%------------------------------------------------
	\label{sub:Formalisierung}

	Da sie die Grundlage
	-- quasi das Fundament --
	des mathematischen Inhalts von \ASBA\ sind,
	müssen die \glsIdxPl{Axiom}, \glsIdxPl{Satz}, \glsIdxPl{Beweis}, \textusw der Aussagenlogik
	in streng formaler Form vorliegen.
	Die Formalisierung stützt sich auf~\cite{bib:Aussagenlogik};
	\alsoname~\cite{bib:LogikDe, bib:LogikEn}.
	Da Computerprogramme mit der
	\emph{Polnischen Notation}\idx{Polnische Notation}%
	\footnote{%
		Bei der \emph{Polnischen Notation} wird eine zweistellige
		Operation $(A\circ B)$ dargestellt als $\circ A B$.
		Eine Zwischenstufe ist $\circ(A,B)$,
		bei der noch die redundanten Gliederungszeichen Komma und Klammern
		-- auch andere als die runden --
		hinzukommen, so dass die Operationen optisch besser getrennt
		und dadurch für Menschen besser lesbar werden.
		Durch einfaches Weglassen der Gliederungszeichen
		ergibt sich dann die Polnische Notation.%
	}
	besser umgehen können und Klammern dort überflüssig sind,
	werden viele Formeln auch in die Polnische Notation überführt.

	\subsubsection{Bausteine der aussagenlogischen Sprache}% - - - - - - - - - -
	\label{subsub:Bausteine}

	Zur Einteilung der aussagenlogischen Junktoren
	werden die folgenden Mengen definiert:
	\begin{align}
		%
		& \gsNo\hidden{\gls{gsNo}}  & & \defeq & &
		& & \text{Menge der \emph{natürlichen Zahlen} einschließlich 0}
		\idx{natürlichen Zahlen, Menge der} \label{def:N} \\
		%
		& \glsSym{asC}  & & \defeq & & \{ \ltrue, \lfalse \}
		& & \text{Menge der \emph{aussagenlogischen Konstanten}}
		\idx{Konstanten, Menge der}         \label{def:C} \\
		%
		& \glsSym{asU}  & & \defeq & & \{ \lnot \}
		& & \text{Menge der \emph{unären aussagenlogischen Operatoren}}
		\idx{unären Operatoren, Menge der}  \label{def:U} \\
		%
		& \glsSym{asB}  & & \defeq & &
		\{ \land, \lor, \limp, \lequiv, \lrep, \lnand, \lnor, \lmult, \ladd \}
		& & \text{Menge der \emph{binären aussagenlogischen Operatoren}}
		\idx{binären Operatoren, Menge der} \label{def:B} \formulatoleft
		%
	\end{align}

	Damit sind alle in der \tablename~\vref{tab:Symbole} verwendeten
	wesentlichen Konstanten und Operatoren%
	\footnote{%
		Jeweils nur die ersten der grau hinterlegten Zeilen sowie \symqt{$\lmult$}.%
	}
	erfasst und es können die folgende Mengen definiert werden:
	\begin{align}
		%
		& \glsSym{asV}  & & \defeq    & & \{ p_n | n \in \gsNo \}
		& & \text{Menge der \emph{atomaren Formeln}}
		\idx{atomaren Formeln, Menge der}         \label{def:V}  \\
		%
		& \glsSym{asJ}  & & \defeq    & &\asC \cup \asU \cup \asB
		& & \text{Menge der \emph{Junktoren}, \textbzw \emph{Operatoren}}
		\idx{Junktoren, Menge der}                \label{def:J}  \\
		%
		& \glsSym{asA}  & & \defeq    & & \asV \cup \asJ
		& & \text{\emph{Alphabet der aussagenlogischen Sprache (für }} \asJ
		\text{\emph{)}}
		\idx{Alphabet der logischen Sprache}      \label{def:A}  \\
		%
		& \glsSym{asJx} & & \subseteq & & \asJ
		& & \text{eine Teilmenge von } \asJ \text{ für eine Indexvariable }
		x                                         \label{def:Bx} \\
		%
		& \glsSym{asAx} & & \defeq    & & \asV \cup \asJx \quad
		& & \text{Alphabet der aussagenlogischen Sprache \emph{für} } \asJx
		\idx{Teil-Alphabet der aussagenlogischen Sprache} \label{def:Ax}
		\formulatoleft
		%
	\end{align}
	Für Elemente aus $\asV$ werden hier normalerweise
	die großen lateinischen Buchstaben $A$, $B$, $C$, $\dots$ verwendet.
	Die Elemente aus $\asV$ (atomaren Formeln) werden auch \emph{\Idx{Satzbuchstabe}}\emph{n}
	oder kurz \emph{\Idx{Atom}}\emph{e}. genannt.

	\subsubsection{Aussagenlogische Formeln}%- - - - - - - - - - - - - - - - - -
	\label{subsub:Formeln}

	Neben dem Alphabet $\asA$
	werden noch Klammern als Gliederungszeichen verwendet.
	Damit können nun rekursiv für jede Teilmenge $\asJx$ von $\asJ$
	zwei Mengen von Formeln definiert werden:

	$\glsSym{asFx}$ sei die Menge der auf folgende Weise definierten
	\emph{aussagenlogischen Formeln mit Klammerung}%
	\idx{aussagenlogische Formel mit Klammerung}:
	\begin{align}
		\asV            \subset    \asFx
		\\
		\asJx \cap \asC \subset    \asFx
		\\
		A                      \in \asFx
		& \metaimp (\circ A)   \in \asFx
		& , \quad \text{für} \quad \circ \in \asU \cap \asJx
		\\
		A, B            \in        \asFx
		& \metaimp (A \circ B) \in \asFx
		& , \quad \text{für} \quad \circ \in \asB \cap \asJx
	\end{align}
	Nur die auf diese Weise konstruierten Formeln sind Elemente von $\asFx$.
	\\Für $\asJ = \asJx$ sei noch $\asF \defeq \glsSym{asFx}$.

	$\glsSym{asFxp}$ sei die Menge der auf folgende Weise definierten
	aussagenlogischen Formeln in \emph{Polnischer Notation}%
	\idx{aussagenlogische Formel in Polnischer Notation}:
	\begin{align}
		\asV            \subset  \asFxp
		\\
		\asJx \cap \asC \subset  \asFxp
		\\
		A                    \in \asFxp
		& \metaimp \circ A   \in \asFxp
		& , \quad \text{für} \quad \circ \in \asU \cap \asJx
		\\
		A, B                 \in \asFxp
		& \metaimp \circ A B \in \asFxp
		& , \quad \text{für} \quad \circ \in \asB \cap \asJx
	\end{align}
	Nur die auf diese Weise konstruierten Formeln sind Elemente von $\asFxp$.
	\\Für $\asJ = \asJx$ sei noch $\asFp \defeq \glsSym{asFxp}$.

	Wie man leicht sieht, gilt:
	\begin{equation}
		\asJx \subset \asJy \subseteq \asJ \metaimp
		\begin{cases}
			\asAx  \subset \asAy  \subseteq \asA \\
			\asFx  \subset \asFy  \subseteq \asF \\
			\asFxp \subset \asFyp \subseteq \asFp
		\end{cases}
	\end{equation}

	Durch Anwendung der Klammerregeln von
	\subsubsectionname~\vref{subsub:Bausteine}
	lassen sich in der Regel noch viele Klammern
	der Formeln aus $\asFx$ einsparen.
	Die Formeln aus $\asFxp$ sind frei von Klammern.
	Die Namen der Operatoren finden sich in der \tablename~\vref{tab:Symbole}.
	Für aussagenlogische Formeln,
	\textdh von Elementen aus $\asF$ \textbzw $\asFp$,
	werden hier normalerweise die kleinen griechischen Buchstaben
	$\alpha$, $\beta$, $\gamma$, $\dots$ verwendet.
	Sie können dabei auch atomare Formeln bezeichnen (\seename \eqref{def:V}).

	\subsection{Aussagenlogische Axiome}%---------------------------------------
	\label{sub:logAxiome}

%TODO Signatur definieren

	Es werden noch bestimmte Teilmengen von $\asJ$ eingeführt,
	die jeweils ausreichen alle anderen Elemente aus $\asJ$ zu definieren:
	\begin{align}
		& \asJ_\xBool &\defeq & & & \{ \lnot, \land, \lor \} \label{def:Jbool}\\
		& \asJ_\xAnd  &\defeq & & & \{ \lnot, \land       \} \label{def:Jand} \\
		& \asJ_\xOr   &\defeq & & & \{ \lnot, \lor        \} \label{def:Jor}  \\
		& \asJ_\xImp  &\defeq & & & \{ \lnot, \limp       \} \label{def:Jimp} \\
		& \asJ_\xRep  &\defeq & & & \{ \lnot, \lrep       \} \label{def:Jrep} \\
		& \asJ_\xNand &\defeq & & & \{ \lnand             \} \label{def:Jnand}\\
		& \asJ_\xNor  &\defeq & & & \{ \lnor              \} \label{def:Jnor}
		\formulatoleft\formulatoleft\formulatoleft\formulatoleft\formulatoleft
	\end{align}

	Im Folgenden stehen jeweils
	links die Formeln in üblicher Schreibweise mit Klammern
	und rechts in Polnischer Notation (ohne Klammern).

	Ausgehend von den Junktoren \textbzw Operatoren aus $\asJ_\xBool$
	werden die restlichen Operatoren aus $\asJ$ definiert.
	Die Definitionen sind in zwei Gruppen eingeteilt,
	und zwar die mit den Operatoren aus $\asJ_\xAnd$:
	\begin{align}
		% folgt ------------------------
		&                (A \limp B)  &\defeq & & & (\lnot (A \land (\lnot B)))
		& \formulaspace &   \limp A B &\defeq & & & \lnot \land A \lnot B
		\label{def:imp}
		\\
		% sofern -----------------------
		&                (A \lrep B)  &\defeq & & & (\lnot (B \land (\lnot A)))
		& \formulaspace &   \lrep B A &\defeq & & & \lnot \land B \lnot A
		\label{def:rep}
		\\
		% genau dann -------------------
		&              (A \lequiv B)  &\defeq & & & ((A\limp B)\land(A\lrep B))
		& \formulaspace & \lequiv A B &\defeq & & & \land \limp A B \lrep A B
		\label{def:equiv}
		\\
		%falsch ------------------------
		&                     \lfalse & \defeq & & & (p_0 \land (\lnot p_0))
		& \formulaspace &     \lfalse & \defeq & & & \land p_0 \lnot p_0
		\label{def:false}
		\\
		% mal --------------------------
		&               (A \lmult B)  & \defeq & & & (A \land B)
		& \formulaspace &  \lmult A B & \defeq & & & \land A B
		\label{def:mult}
		\\
		% NAND -------------------------
		&               (A \lnand B)  &\defeq & & & (\lnot (A \land B ))
		& \formulaspace &  \lnand A B &\defeq & & & \lnot \land A B
		\label{def:nand}
		\formulatoleft
	\end{align}
	und die mit den Operatoren  aus $\asJ_\xOr$:
	\begin{align}
		% NOR --------------------------
		&                (A \lnor B)  & \defeq & & & (\lnot (A \lor B))
		& \formulaspace &   \lnor A B & \defeq & & & \lnot \lor A B
		\label{def:nor}
		\\
		% plus -------------------------
		& (A \ladd B) & \defeq & & & ((A\lor B)\land(\lnot(A\land B)))
		& \formulaspace &   \ladd A B & \defeq & & & \land\lor A B\lnot\land A B
		\label{def:add}
		\\
		% wahr -------------------------
		&                      \ltrue & \defeq & & & (p_0 \lor (\lnot p_0))
		& \formulaspace &      \ltrue & \defeq & & & \lor p_0 \lnot p_0
		\label{def:true}
		\formulatoleft
	\end{align}
	Ist \symqt{$\lor$} oder \symqt{$\land$} nicht vorgegeben,
	\textdh wird von den Elementen aus $\asJ_\xAnd$ \textbzw $\asJ_\xOr$
	statt von denen von $\asJ_\xBool$ ausgegengen,
	so muss man den fehlenden Operator mittels
	der passenden der beiden folgenden Definitionen einführen:
	\begin{align}
		% und --------------------------
		&         (A \land B) & \defeq & & & (\lnot ((\lnot A) \lor (\lnot B)))
		& \formulaspace &    \land A B &\defeq & & & \lnot \lor \lnot A \lnot B
		\label{def:and}
		\\
		% oder -------------------------
		&          (A \lor B) & \defeq & & & (\lnot ((\lnot A) \land (\lnot B)))
		& \formulaspace &    \lor A B &\defeq & & & \lnot \land \lnot A \lnot B
		\label{def:or}
		\formulatoleft
	\end{align}
	Nun sind wieder alle Operatoren definiert:

\todo{Hier weitermachen.}% TODO >>> Hier weitermachen <<< restliche Signaturen

	Entsprechend wählt man bei Vorgabe von $\asJ_\xImp$ \textbzw $\asJ_\xRep$
	eine von den beiden folgenden Definitionen aus:
	\begin{align}
		% und --------------------------
		&             (A \land B) & \defeq & & & (???)
		& \formulaspace &    \land A B &\defeq & & & ???
		\label{def:and2}
		\\
		% oder -------------------------
		&              (A \lor B) & \defeq & & & (???)
		& \formulaspace &    \land A B &\defeq & & & ???
		\label{def:or2}
		\formulatoleft
	\end{align}
	und bei Vorgabe von $\asJ_\xNand$ \textbzw $\asJ_\xNor$
	eine von:
	\begin{align}
	% und --------------------------
	&             (A \land B) & \defeq & & & (???)
	& \formulaspace &    \land A B &\defeq & & & ???
	\label{def:and3}
	\\
	% oder -------------------------
	&              (A \lor B) & \defeq & & & (???)
	& \formulaspace &    \land A B &\defeq & & & ???
	\label{def:or3}
	\formulatoleft
	\end{align}

	Aus \ref{def:rep} folgt durch Vertauschung der Variablen unmittelbar
	\begin{align}
		%
		&                 (A \limp B) & \equiv & & & (B \lrep A)
		& \formulaspace &   \limp A B & \equiv & & & \lrep B A
		\label{eq:imp}   \formulatoleft
		%
	\end{align}

	Es werden nun die aussagenlogischen \glsIdxPl{Axiom}
	-- ohne Quantoren --
	mit Hilfe der Operatoren aus $\asJ_\xAnd$ definiert
	und zusätzlich alle Elemente von $\asJ$ (\seename~\vref{def:J}),
	\textdh alle aussagenlogischen Operatoren und Konstanten aus $\asJ$.

%##	Die in den \subsubsectionname en~\vref{subsub:OperatorenAnfang}
%##	bis \vref{subsub:OperatorenEnde} fehlenden Definitionen von Operatoren
%##	sollen mit denen aus \subsubsectionname~\vref{subsub:AussagenlogischeAxiome}
%##	übereinstimmen.

\todo{Junktoren definieren und Übereinstimmung ableiten.}%##
%TODO Junktoren definieren und Übereinstimmung ableiten. %##

	\subsubsection{Aussagenlogisches Axiomensystem}% - - - - - - - - - - - - - -
	\label{subsub:AussagenlogischeAxiome}
	Gegebene Operatoren: $\lnot, \land, \limp$\par
	\glsIdxPl{Axiom}:
	\begin{align}
		%
		&(\alpha\limp\beta\limp\gamma)\limp(\alpha\limp\beta)%
			\limp(\alpha\limp\gamma)
		\formulaspace&&%
		\limp\limp\alpha\limp\beta\gamma\limp\limp\alpha\beta%
			\limp\alpha\gamma\\
		%
		&\alpha\limp\beta\limp\alpha\land\beta
		\formulaspace&&%
		\limp\alpha\limp\beta\land\alpha\beta\\
		%
		&\alpha\land\beta\limp\alpha ;\quad\alpha\land\beta\limp\beta
		\formulaspace&&%
		\limp\land\alpha\beta\alpha ;\quad\limp\land\alpha\beta\beta\\
		%
		&(\alpha\limp\lnot\beta)\limp(\beta\limp\lnot\alpha)
		\formulaspace&&%
		\limp\limp\alpha\lnot\beta\limp\beta\lnot\alpha
		\formulatoleft
		%
	\end{align}
	Definierte Operatoren: $\lor, \lequiv, \lmult, \ladd,
		\lnand, \lnor, \lrep, \lfalse, \ltrue$
	\begin{align}
		(\alpha\lor\beta)&\defeq\lnot(\lnot\alpha\limp\beta)
		\formulaspace&%
		\lor\alpha\beta&\defeq\lnot\limp\lnot\alpha\beta\\
		%
		(\alpha\lequiv\beta)&\defeq
		((\alpha\limp\beta)\land(\beta\limp\alpha))
		\formulaspace&%
		(\alpha\lequiv\beta)&\defeq((\alpha\limp\beta)%
			\land(\beta\limp\alpha))\\
		%
		(\alpha\lmult\beta)&\defeq(\alpha\land\beta)
		\formulaspace&%
		(\alpha\lmult\beta)&\defeq(\alpha\land\beta)\\
		%
		(\alpha\ladd\beta)&\defeq((\alpha\lor\beta)%
			\land\lnot(\alpha\land\beta))
		\formulaspace&%
		(\alpha\ladd\beta)&\defeq((\alpha\lor\beta)%
			\land\lnot(\alpha\land\beta))\\
		%
		(\alpha\lnand\beta)&\defeq\lnot(\alpha\land\beta)
		\formulaspace&%
		(\alpha\lnand\beta)&\defeq\lnot(\alpha\land\beta)\\
		%
		(\alpha\lnor\beta)&\defeq\lnot(\alpha\lor\beta)
		\formulaspace&%
		(\alpha\lnor\beta)&\defeq\lnot(\alpha\lor\beta)\\
		%
		(\alpha\lrep\beta)&\defeq(\beta\limp\alpha)
		\formulaspace&%
		(\alpha\lrep\beta)&\defeq(\beta\limp\alpha)\\
		%
		\lfalse&\defeq(p_0\land\lnot p_0)
		\formulaspace&%
		\lfalse&\defeq(p_0\land\lnot p_0)\\
		%
		\ltrue&\defeq\lnot\lfalse
		\formulaspace&%
		\ltrue&\defeq\lnot\lfalse
		\formulatoleft
		%
	\end{align}
	Zu zeigen
	\begin{align}
		(\alpha\limp\beta)&\equiv\lnot(\alpha\land\lnot\beta)
		\formulaspace
		&\limp\alpha\beta&\equiv\lnot\land\alpha\lnot\beta
		\formulatoleft
	\end{align}

	\subsection{Definition von Junktoren durch andere}%-------------------------
	\label{sub:Junktordefinitionen}

	\subsubsection{nicht, und, oder}%- - - - - - - - - - - - - - - - - - - - - -
	\label{subsub:Standard}
	\label{subsub:OperatorenAnfang}
	Gegebene Operatoren: $\lnot, \land, \lor$\par
	Definierte Operatoren:
	$\limp, \lequiv, \lmult, \ladd, \lnand, \lnor, \lrep, \lfalse, \ltrue$
	\begin{align}
	\end{align}
	Zu zeigen:
	\begin{align}
		(\alpha\lor\beta)&\defeq\lnot(\lnot\alpha\land\lnot\beta)
	\end{align}

	\subsubsection{nicht, und}%- - - - - - - - - - - - - - - - - - - - - - - - -
	Gegebene Operatoren: $\lnot, \land$\par
	Definierter Operator: $\lor$
	\begin{align}
		(\alpha\lor\beta)&\defeq\lnot(\lnot\alpha\land\lnot\beta)
	\end{align}
	Zu zeigen:
	\begin{align}
		(\alpha\lor\beta)&\equiv\lnot(\lnot\alpha\land\lnot\beta)
	\end{align}
	Zur Definition der Operatoren $\limp, \lequiv, \lmult, \ladd, \lnand, \lnor,
	\lrep, \lfalse, \ltrue$ siehe \subsubsectionname~\vref{subsub:Standard}

	\subsubsection{nicht, oder}% - - - - - - - - - - - - - - - - - - - - - - - -
	Gegebene Operatoren: $\lnot, \lor$\par
	Definierter Operator: $\land$
	\begin{align}
		(\alpha\land\beta)&\defeq\lnot(\lnot\alpha\lor\lnot\beta))
	\end{align}

	\subsubsection{nicht, impliziert}% - - - - - - - - - - - - - - - - - - - - -
	Gegebene Operatoren: $\lnot, \limp$\par
	Definierte Operatoren: $\land, \lor$
	\begin{align}
		(\alpha\land\beta)&\defeq\dots\\
		(\alpha\lor\beta)&\defeq\dots
	\end{align}

	\subsubsection{NAND}%- - - - - - - - - - - - - - - - - - - - - - - - - - - -
	Gegebener Operator: $\lnand$\par
	Definierte Operatoren: $\lnot, \land, \lor$
	\begin{align}
		\lnot\alpha&\defeq\dots\\
		(\alpha\land\beta)&\defeq\dots\\
		(\alpha\lor\beta)&\defeq\dots
	\end{align}

	\subsubsection{NOR}% - - - - - - - - - - - - - - - - - - - - - - - - - - - -
	\label{subsub:OperatorenEnde}
	Gegebener Operator: $\lnor$\par
	Definierte Operatoren: $\lnot, \land, \lor$
	\begin{align}
		\lnot\alpha&\defeq\dots\\
		(\alpha\land\beta)&\defeq\dots\\
		(\alpha\lor\beta)&\defeq\dots
	\end{align}

	\subsection{Aussagenlogische Axiome}%---------------------------------------
	\label{sub:ausAxiome}

\todo{Aussagenlogik weiter bearbeiten.}%##
%TODO Aussagenlogik weiter bearbeiten. %##

	\section{Prädikatenlogik}%==================================================
	\beginsection{Prädikatenlogik}
	\label{sec:Prädikatenlogik}
	\hidden{\glsIdx{Praedikatenlogik}}

\todo{Prädikatenlogik bearbeiten.}%##
%TODO Prädikatenlogik bearbeiten. %##

	\section{Mengenlehre}%======================================================
	\beginsection{Mengenlehre}
	\label{sec:Mengenlehre}

\todo{Mengenlehre bearbeiten.}%##
%TODO Mengenlehre bearbeiten. %##

	\beforechapter% Kopfzeile aktivieren und neue Seite
	\chapter{Design}%%%%%%%%%%%%%%%%%%%%%%%%%%%%%%%%%%%%%%%%%%%%%%%%%%%%%%%%%%%%
	\beginchapter{Design}
	\label{cha:Design}

	Diese Projekt soll Open Source sein.
	Daher gilt für die Dokumente die \emph{GNU Free Documentation License (FDL)}
	und für die Software die \emph{GNU Affero General Public License (APGL)}.
	Die \emph{GNU General Public License (GPL)} reicht für die Software nicht,
	da das Programm auch mittels eines Servers betrieben werden kann und soll.
	Damit das Projekt gegebenenfalls durch verschiedene Entwickler gleichzeitig
	bearbeitet werden kann und wegen des Konfigurationsmanagements
	wurde es als ein GitHub Projekt erstellt (\seename~\cite{bib:ASBA}).

	Wenn die Lizenzen nicht mitgeliefert wurden,
	können sie unter \url{http://www.gnu.org/licenses/} gefunden werden.

	\section{Anforderungen}%====================================================
	\beginsection{Anforderungen}
	\label{sec:Anforderungen}

	Die Anforderungen ergeben sich zunächst
	aus den Zielen in \sectionname~\vref{sec:Ziele}.
	Die beiden Ziele \ref{Ziel:Daten}~\emph{Daten}
	und \ref{Ziel:Lizenz}~\emph{Lizenz}
	sind für die Entwicklung von \ASBA\ von sekundärer Bedeutung
	und werden daher in diesen \sectionname{} nicht übernommen.
	Die anderen Ziele werden noch verfeinert.

%\todo{Ziele aus Abschnitt "Ziele" in Anforderungen umwandeln.}%##
% TODO Ziele aus Abschnitt "Ziele" in Anforderungen umwandeln. %##
	\begin{enumerate}

		\item \label{Anforderung:Form} \emph{Form}:
		Die Daten liegt in formaler, geprüfter Form vor.
		(\seename{} Ziel~\vref{Ziel:Form})

		\item \label{Anforderung:Eingaben} \emph{Eingaben}:
		Die Eingabe von Daten erfolgt in einer formalen Syntax
		unter Verwendung der üblichen mathematischen Schreibweise.
		Folgende Daten können eingegeben werden:
		\begin{enumerate}
			\item \glsIdxPl{Axiom}
			\item \glsIdxPl{Satz}
			\item \glsIdxPl{Beweis}
			\item \glsIdxPl{Fachbegriff}
			\item \glsIdxPl{Fachgebiet}
			\item \glsIdxPl{Ausgabeschema}
		\end{enumerate}
		Dabei sind alle Begriffe nur innerhalb eines \glsIdxPl{Fachgebiet}s
		und seiner untergeordneten \glsIdxPl{Fachgebiet} gültig,
		solange sie nicht umdefiniert werden.
		Das oberste \glsIdx{Fachgebiet} ist die ganze Mathematik.
		(\seename{} Ziel~\vref{Ziel:Eingaben})

		\item \label{Anforderung:Prüfung} \emph{Prüfung}:
		Vorhandene \glsIdxPl{Beweis} können automatisch geprüft werden.
		(\seename{} Ziel~\vref{Ziel:Prüfung})

		\item \label{Anforderung:Ausgaben} \emph{Ausgaben}:
		Die Ausgabe kann in einer eindeutigen, formalen Syntax
		gemäß vorhandener \glsIdxPl{Ausgabeschema} erfolgen.
		(\seename{} Ziel~\vref{Ziel:Ausgaben})

		\item \label{Anforderung:Auswertungen} \emph{Auswertungen}:
		Zusätzlich zur Ausgabe der Daten sind verschiedene Auswertungen möglich.
		Insbesondere kann zu jedem \glsIdx{Beweis} angegeben werden,
		wie viele Beweisschritte und welche \glsIdxPl{Axiom} und \glsIdxPl{Satz}%
		\footnote{\glsIdxPl{Satz}, die quasi als \glsIdxPl{Axiom} verwendet werden.}
		er verwendet.
		(\seename{} Ziel~\vref{Ziel:Auswertungen})

		\item \label{Anforderung:Anpassbarkeit} \emph{Anpassbarkeit}:
		\glsIdxPl{Fachbegriff}
		und die Darstellung bei der Ausgabe können mit Hilfe von
		-- gegebenenfalls unbenannten --
		untergeordneten \glsIdxPl{Fachgebiet}n angepasst werden.
		(\seename{} Ziel~\vref{Ziel:Anpassbarkeit})

		\item \label{Anforderung:Individualität} \emph{Individualität}:
		\glsIdxPl{Axiom} und \glsIdxPl{Satz} können
		für jeden \glsIdx{Beweis} individuell vorausgesetzt werden.
		Dabei sind fachgebietsspezifische \glsIdxPl{Fachbegriff} erlaubt.
		(\seename{} Ziel~\vref{Ziel:Individualität})

		\item \label{Anforderung:Internet} \emph{Internet}:
		Die Daten können auf mehrere Dateien verteilt sein.
		Ein Teil davon --
		oder sogar alle --
		können im Internet liegen.
		(\seename{} Ziel~\vref{Ziel:Internet})

		\item \label{Anforderung:Kommunikation} \emph{Kommunikation}:
		Die Kommunikation mit \ASBA\ kann mit den
		\glsIdxPl{Fachbegriff}n der einzelnen \glsIdxPl{Fachgebiet} erfolgen.
		(\seename{} Ziel~\vref{Ziel:Kommunikation})

		\item \label{Anforderung:Zugriff} \emph{Zugriff}:
		Der Zugriff auf \ASBA\ kann lokal und über das Internet erfolgen.
		(\seename{} Ziel~\vref{Ziel:Zugriff})

		\item \label{Anforderung:Unabhängigkeit} \emph{Unabhängigkeit}:
		\ASBA\ kann offline und online arbeiten.
		(\seename{} Ziel~\vref{Ziel:Unabhängigkeit})

		\item \label{Anforderung:Rekursion} \emph{Rekursion}:
		Es kann rekursiv über alle verwendeten Dateien
		-- auch solchen, die im Internet liegen --
		ausgewertet werden.
		(\seename{} Ziel~\vref{Ziel:Rekursion})

		\item \label{Anforderung:Bedienbarkeit} \emph{Bedienbarkeit}:
		\ASBA\ ist einfach zu bedienen.
		(\seename{} Ziel~\vref{Ziel:Bedienbarkeit})

	\end{enumerate}

	\section{Axiome}%===========================================================
	\beginsection{\glsIdxPl{Axiom}}
	\label{sec:Axiome}
\todo{Axiome auswählen und definieren.}%##
%TODO Axiome auswählen und definieren. %##

	\section{Beweise}%==========================================================
	\beginsection{\glsIdxPl{Beweis}}
	\label{sec:Beweise}
\todo{Schlussregeln auswählen und Beweise definieren.}%##
%TODO Schlussregeln auswählen und Beweise definieren. %##

	\section{Datenstruktur}%====================================================
	\beginsection{Datenstruktur}
	\label{sec:Datenstruktur}
\todo{Datenstruktur abstrakt und in XML definieren.}%##
%TODO Datenstruktur abstrakt und in XML definieren. %##

	\section{Bausteine}%========================================================
	\beginsection{Bausteine}
	\label{sec:Bausteine}
\todo{Bausteine? definieren.}%##
%TODO Bausteine? definieren. %##

	\appendix
	\renewcommand*{\Chaptername}{\appendixname}

	\beforechapter% Kopfzeile aktivieren und neue Seite
	\chapter{Anhang}%%%%%%%%%%%%%%%%%%%%%%%%%%%%%%%%%%%%%%%%%%%%%%%%%%%%%%%%%%%%
	\beginchapter{Anhang}
	\label{cha:Anhang}

	\section{Werkzeuge}%========================================================
	\beginsection{Werkzeuge}
	\label{sec:Werkzeuge}

	Da dies ein Open Source Projekt sein soll,
	müssen alle Werkzeuge,
	die zum Ablauf der Software erforderlich sind,
	ebenfalls Open Source sein.
	Für die reine Entwicklung sollte das auch gelten, muss es aber nicht.

	\paragraph{Werkzeuge zur Übersetzung der Quelldateien}%---------------------

	\begin{enumerate}

		\item\label{Werkzeug:LaTeX}
		Ein Übersetzer für \LaTeX Quellcode (*.tex).
		-- Verwendet wird \emph{MiK\TeX}.

		\item\label{Werkzeug:C++}
		Ein Übersetzer für C++ Quellcode (*.c, *.cpp, *.h, *.hpp).
		-- Verwendet wird \emph{Visual Studio Community 2017}.

		\setcounter{Enumi}{\value{enumi}}% Nummerierung wird fortgesetzt.
	\end{enumerate}
	Nicht unbedingt nötig, aber sinnvoll:
	\begin{enumerate}
		\setcounter{enumi}{\value{Enumi}}% Nummerierung wird fortgesetzt.

	%TODO Dateien für Doxygen aufführen
	\item\label{Werkzeug:Dokumentation}
	Ein Dokumentationssystem für in C++ Quellcode und darin enthaltene Doxygen Kommentare (*.c, *.cpp, *.h, *.hpp).
	-- Verwendet wird \emph{Doxygen}.

	\item\label{Werkzeug:Konfigurationsmanagement}
	Ein Konfigurationsmanagementsystem zur Verwaltung der Quelldateien.
	-- Verwendet wird \emph{GitHub}.

		\setcounter{Enumi}{\value{enumi}}% Nummerierung wird fortgesetzt.
	\end{enumerate}

	\paragraph{Werkzeuge für die Entwicklung}%----------------------------------

	\begin{enumerate}
		\setcounter{enumi}{\value{Enumi}}% Nummerierung wird fortgesetzt.

		\item\label{Werkzeug:GitHub}\emph{GitHub}
		als Online Konfigurationsmanagementsystem
		zur Zusammenarbeit verschiedener Entwickler.
		\tourl{https://github.com/}
		-- Lizenz \seename~\cite{bib:GPLii}

		\item\label{Werkzeug:Git}GitHub benötigt
		\emph{Git} als Konfigurationsmanagementsystem.
		\tourl{https://git-scm.com/}
		-- Lizenz \seename~\cite{bib:GPLii}

		\item\label{Werkzeug:MiKTeX}\emph{MiK\TeX}
		für Dokumentation und Ausgaben in \LaTeX.
		\tourl{https://miktex.org/}
		-- Lizenz \seename~\cite{bib:MiKTeX}

		\item\label{Werkzeug:VSC}angedacht: \emph{Visual Studio Community 2017}%
		\footnote{%
			Visual Studio Community ist zwar nicht Open Source,
			darf aber zur Entwicklung von Open Source Software
			unentgeltlich verwendet werden.%
		}
		(\emph{VS}) als Entwicklungsumgebung für C++.
		\tourl{https://www.visualstudio.com/downloads/}
		-- Lizenz \seename~\cite{bib:EULA}

		\item\label{Werkzeug:VSC DB}angedacht: In \emph{Visual Studio Community 2015}
		integrierte Datenbank für \glsIdxPl{Axiom}, \glsIdxPl{Satz}, \glsIdxPl{Beweis},
		\glsIdxPl{Fachbegriff} und \glsIdxPl{Fachgebiet}.
		-- Lizenz \seename~\cite{bib:EULA}

		\item\label{Werkzeug:RapidXml}angedacht: \emph{RapidXml}
		für Ein- und Ausgabe in XML.
		\tourl{http://rapidxml.sourceforge.net/index.htm}
		-- Lizenz \seename{} wahlweise~\cite{bib:BSLi} oder~\cite{bib:MIT}
		\footnote{%
			RapidXml stellt eine C++ Header-Datei zur Verfügung.
			Wenn diese im Quellcode eines Programms enthalten ist,
			gilt das ganze Programm als Open Source.
			Wenn diese Header-Datei nur
			in einer Bibliothek innerhalb eines Projekts verwendet wird,
			so gilt nur diese Bibliothek als Open Source.%
		}

		\item\label{Werkzeug:Doxygen}angedacht: \emph{Doxygen}
		als Dokumentationssystem für C++.
		\tourl{http://www.stack.nl/~dimitri/doxygen/}
		-- Lizenz \seename~\cite{bib:GPLii}

		\item\label{Werkzeug:Ghostscript}angedacht: Doxygen benötigt \emph{Ghostscript}
		als Interpreter für Postscript und PDF.
		\tourl{http://ghostscript.com/}
		-- Lizenz \seename~\cite{bib:AGPL}

		\item\label{Werkzeug:Graphviz}angedacht: Doxygen
		benötigt \emph{Graphviz} mit \emph{Dot}
		zur Erzeugung und Visualisierung von Graphen.
		\tourl{http://www.graphviz.org/Home.php}
		-- Lizenz \seename~\cite{bib:EPL}

		\setcounter{Enumi}{\value{enumi}}% Nummerierung wird fortgesetzt.
	\end{enumerate}

	\paragraph{Werkzeuge zur Bearbeitung der Quelldateien}%---------------------

	\begin{enumerate}
		\setcounter{enumi}{\value{Enumi}}% Nummerierung wird fortgesetzt.

		\item\label{Werkzeug:TeXstudio}\emph{\TeX studio} als Editor für \LaTeX.
		\tourl{http://www.texstudio.org/}
		-- Lizenz \seename~\cite{bib:GPLii}\\
		\TeX studio benötigt einen Interpreter für Perl:

		\item\label{Werkzeug:Perl}\emph{Strawberry Perl}
		als Interpreter für Perl.
		\tourl{http://strawberryperl.com/}
		-- Lizenz:
		Various OSI-compatible Open Source licenses,
		or given to the public domain

		\item\label{Werkzeug:Notepadpp}\emph{Notepad++} als Text-Editor.
		\tourl{https://notepad-plus-plus.org/}
		-- Lizenz \seename~\cite{bib:GPLi}

		\item\label{Werkzeug:WinMerge}\emph{WinMerge}
		zum Vergleich von Dateien und Verzeichnissen.
		\tourl{http://winmerge.org/}
		-- Lizenz \seename~\cite{bib:GPLi}

		\setcounter{Enumi}{\value{enumi}}% Nummerierung wird fortgesetzt.
	\end{enumerate}

%##	\paragraph{Im Projekt \emph{qedeq} verwendete Werkzeuge}%-------------------
%##
%##	\begin{itemize}
%##		\setcounter{enumi}{\value{Enumi}}% Nummerierung wird fortgesetzt.
%##
%##		\item\label{Werkzeug:Java}\emph{Java}
%##		als Programmiersprache und Laufzeitumgebung.
%##		\tourl{https://www.java.com/de/download/win10.jsp}
%##		-- Lizenz \seename~\cite{bib:JavaSE}
%##
%##		\item\label{Werkzeug:Apache Ant}\emph{Apache Ant}
%##		als Java Bibliothek und Kommandozeilen-Werkzeug
%##		um Java Programme zu erzeugen.
%##		\tourl{http://ant.apache.org/}
%##		-- Lizenz \seename~\cite{bib:Apacheii}
%##
%##		\item\label{Werkzeug:Checkstyle}\emph{Checkstyle}
%##		zur statischen Code-Analyse für Java.
%##		\tourl{http://checkstyle.sourceforge.net/}
%##		-- Lizenz \seename~\cite{bib:LGPLii}
%##
%##		\item\label{Werkzeug:Clover}\emph{Clover}%
%##		\footnote{%
%##			Clover ist proprietäre Software, aber auf Anfrage frei für 30 Tage.
%##			Danach ist eine einmalige Lizenzgebühr fällig.%
%##		}
%##		als Testwerkzeug zur Analyse der Code-Abdeckung.
%##		\tourl{https://www.atlassian.com/software/clover/}
%##		-- Lizenz \seename~\cite{bib:Clover}
%##
%##		\item\label{Werkzeug:Eclipse Java}\emph{Eclipse IDE for Java Developers}
%##		als Entwicklungsumgebung für Java.
%##		\tourl{http://www.eclipse.org/downloads/packages/eclipse-ide-java-developers/neon1a/}
%##		-- Lizenz \seename~\cite{bib:OSI}
%##
%##		\item\label{Werkzeug:JUnit}\emph{JUnit}
%##		zur Erzeugung von wiederholbaren Tests.
%##		\tourl{http://junit.org/junit4/}
%##		-- Lizenz \seename~\cite{bib:EPL}
%##
%##		\item\label{Werkzeug:Xerces2}\emph{Xerces2} als XML-Parser in Java.
%##		\tourl{http://xerces.apache.org/xerces2-j/}
%##		-- Lizenzen \seename~\cite{bib:Apacheii, bib:SAX, bib:WDCDL, bib:WDCSNL}
%##
%##		\setcounter{Enumi}{\value{enumi}}% Nummerierung wird fortgesetzt.
%##	\end{itemize}

	\section{Offene Aufgaben}%==================================================
	\beginsection{Offene Aufgaben}
	\label{sec:Offene Aufgaben}

	\begin{enumerate}
		\item TODOs bearbeiten
		\item Eingabeprogramm erstellen (liest XML)
		\item Prüfprogramm erstellen
		\item Ausgabeprogramm erstellen (schreibt XML)
		\item Formelausgabe erstellen (erzeugt \LaTeX{} aus XML)
		\item \glsIdxPl{Axiom} sammeln und eingeben
		\item \glsIdxPl{Satz} sammeln und eingeben
		\item \glsIdxPl{Beweis} sammeln und eingeben
		\item \glsIdxPl{Fachbegriff} und Symbole sammeln und eingeben
		\item \glsIdxPl{Fachgebiet} sammeln und eingeben
		\item \glsIdxPl{Ausgabeschema} sammeln und eingeben
	\end{enumerate}

	\beforechapter% Kopfzeile aktivieren und neue Seite
	\chapter{Ideen}%%%%%%%%%%%%%%%%%%%%%%%%%%%%%%%%%%%%%%%%%%%%%%%%%%%%%%%%%%%%%
	\beginchapter{Ideen}
	\label{cha:Ideen}

	\section{Test der Referenzen}                                  %##
	\beginsection{Test der Referenzen}                             %##
	\label{sec:Referenzen}                                         %##

	cha:Inhaltsverzeichnis:    \vref{cha:Inhaltsverzeichnis}       %##
	Fehler: Link 1 Seite zu früh                                   %##

	dic:Tabellenverzeichnis:   \vref{dic:Tabellenverzeichnis}      %##
	Fehler: Link 1 Seite zu früh                                   %##

	dic:Abbildungsverzeichnis: \vref{dic:Abbildungsverzeichnis} ok %##

	dic:Literaturverzeichnis:  \vref{dic:Literaturverzeichnis}  ok %##

	\beforechapter% Kopfzeile aktivieren und neue Seite
	%chapter{Tabellenverzeichnis}%%%%%%%%%%%%%%%%%%%%%%%%%%%%%%%%%%%%%%%%%%%%%%%
	\likechapter[section]{\listtablename}
%##	\addcontentsline{lot}{chapter}{Test der 2. Seite Tabellenverzeichnis}%##
	\label{dic:Tabellenverzeichnis}
	\listoftables
	\Thispagestyle%              veränderte Kopfzeile für diese Seite aktivieren

	%section{Abbildungsverzeichnis}=============================================
	\likechapter[section]{\listfigurename}
%##	\addcontentsline{lof}{chapter}{Test der 2. Seite Abbildungsverzeichnis}%##
	\begin{minipage}{\linewidth-10.95pt}
		\label{dic:Abbildungsverzeichnis}
		\listoffigures
	\end{minipage}
	\Thispagestyle%              veränderte Kopfzeile für diese Seite aktivieren

	%chapter{Literaturverzeichnis}%%%%%%%%%%%%%%%%%%%%%%%%%%%%%%%%%%%%%%%%%%%%%%
	\begin{flushleft}
		\begin{thebibliography}{12}
			\likechapter[section]{\bibname}  % erst hier!
			\label{dic:Literaturverzeichnis} % erst hier!

			\bibitem{bib:Rautenberg}Wolfgang Rautenberg,
			\emph{Einführung in die Mathematische Logik}:
			Ein Lehrbuch, 3.\@ Auflage, Vieweg+Teubner 2008

			\bibitem{bib:Apacheii}\emph{Apache License}, Version 2.0
			-- 02.01.2004
			\tourl{http://www.apache.org/licenses/LICENSE-2.0} (09.03.2017)%
			\footnote{%
				Der Pfeil~($\rightarrow$)
				verweist stets auf einen Link zu einer Seite im Internet.
				Das geklammerte Datum hinter dem Link gibt den Zeitpunkt an,
				zu dem die Seite im Rahmen der Erstellung dieses Dokuments
				zum letzten Mal angeschaut wurde.
				Das nicht geklammerte Datum gibt,
				je nachdem welches Datum in der Seite angegeben ist,
				den Stand der Seite
				\textbzw den Zeitpunkt der letzten Änderung an.
				-- Dies gilt für alle hier aufgelisteten Seiten im Internet.%
			}

			\bibitem{bib:BSLi}\emph{Boost Software License} 1.0 -- 17.08.2003
			\tourl{http://www.boost.org/users/license.html}
			(09.03.2017)

			\bibitem{bib:EPL}\emph{Eclipse Public License} Version 1.0
			\tourl{http://www.eclipse.org/org/documents/epl-v10.php}
			(09.03.2017)

			\bibitem{bib:AGPL}\emph{GNU Affero General Public License}
			-- 19.11.2007
			\tourl{http://www.gnu.org/licenses/agpl}
			(09.02.2017)

			\bibitem{bib:GPLi}\emph{GNU General Public License}
			\tourl{http://www.gnu.org/licenses/old-licenses/gpl-1.0}
			-- 02.1989 (09.03.2017)

			\bibitem{bib:GPLii}\emph{GNU General Public License}, Version 2
			-- 06.1991
			\tourl{http://www.gnu.org/licenses/old-licenses/gpl-2.0}
			(09.03.2017)

			\bibitem{bib:LGPLii}\emph{GNU Lesser General Public License},
			Version 2.1 -- 02.1999
			\tourl{http://www.gnu.org/licenses/old-licenses/lgpl-2.1}
			(09.03.2017)

			\bibitem{bib:Clover}Lizenz für \emph{Clover}
			-- 2017
			\tourl{https://www.atlassian.com/software/clover}
			(09.03.2017)

			\bibitem{bib:EULA}Lizenz
			für \emph{Microsoft Visual Studio Express 2015}
			-- 2017
			\tourl{https://www.visualstudio.com/de/license-terms/mt171551/}
			(09.03.2017)

			\bibitem{bib:MiKTeX}Lizenz für \emph{MikTeX}
			-- 14.01.2014
			\tourl{https://miktex.org/kb/copying}
			(09.03.2017)

			\bibitem{bib:SAX}Lizenz für \emph{SAX}
			\tourl{http://www.saxproject.org/copying.html}
			-- 05.05.2000 (09.03.2017)

			\bibitem{bib:MIT}\emph{MIT License}
			\tourl{https://opensource.org/licenses/MIT/}
			-- (09.03.2017)

			\bibitem{bib:JavaSE}\emph{Oracle Binary Code License Agreement}
			-- 02.04.2013
			\tourl{http://java.com/license}
			(09.03.2017)

			\bibitem{bib:OSI}\emph{OSI Certified Open Source Software}
			-- 16.06.1999
			\tourl{https://opensource.org/pressreleases/certified-open-source.php}
			(09.03.2017)

			\bibitem{bib:WDCDL}\emph{W3C Document License} -- 01.02.2015
			\tourl{http://www.w3.org/Consortium/Legal/2015/doc-license}
			(09.03.2017)

			\bibitem{bib:WDCSNL}\emph{W3C Software Notice and License}
			-- 13.05.2015
			\tourl{http://www.w3.org/Consortium/Legal/2002/copyright-software-20021231.html}
			(09.03.2017)

			\bibitem{bib:HilbertII}\emph{Hilbert II -- Introduction}
			-- 20.01.2014
			\tourl{http://www.qedeq.org/}
			(09.03.2017)

			\bibitem{bib:qedeq}\emph{Formal Correct Mathematical Knowledge}:
			GitHub Repository vom Projekt Hilbert II
			-- 04.08.2016
			\tourl{https://github.com/m-31/qedeq/}
			(09.03.2017)

			\bibitem{bib:ASBA}\emph{ASBA
			-- Axiome, Sätze, Beweise und Auswertungen}.
			Projekt zur maschinellen Überprüfung von mathematischen Beweisen
			und deren Ausgabe in lesbarer Form:
			GitHub Repository vom Projekt ASBA
			-- in Bearbeitung
			\tourl{https://github.com/Dr-Winfried/ASBA}

			\bibitem{bib:LogikDe}Meyling, Michael:
			\emph{Anfangsgründe der mathematischen Logik}
			-- 24.~Mai~2013 (in Bearbeitung)
			\tourl{http://www.qedeq.org/current/doc/math/qedeq\_logic\_v1\_de.pdf}
			-- (09.03.2017)

			\bibitem{bib:PraedikatenlogikDe}Meyling, Michael:
			\emph{Formale Prädikatenlogik}
			-- 24.~Mai~2013 (in Bearbeitung)
			\tourl{http://www.qedeq.org/current/doc/math/qedeq\_formal\_logic\_v1\_de.pdf}
			-- (09.03.2017)

			\bibitem{bib:MengenlehreDe}Meyling, Michael:
			\emph{Axiomatische Mengenlehre}
			-- 24.~Mai~2013 (in Bearbeitung)
			\tourl{http://www.qedeq.org/current/doc/math/qedeq\_set\_theory\_v1\_de.pdf}
			-- (09.03.2017)

			\bibitem{bib:LogikEn}Meyling, Michael:
			\emph{Elements of Mathematical Logic}
			-- May~24,~2013 (in Bearbeitung)
			\tourl{http://www.qedeq.org/current/doc/math/qedeq\_logic\_v1\_en.pdf}
			-- (09.03.2017)

			\bibitem{bib:PraedikatenlogikEn}Meyling, Michael:
			\emph{Formal Predicate Calculus}
			-- May~24,~2013 (in Bearbeitung)
			\tourl{http://www.qedeq.org/current/doc/math/qedeq\_formal\_logic\_v1\_en.pdf}
			-- (09.03.2017)

			\bibitem{bib:MengenlehreEn}Meyling, Michael:
			\emph{Axiomatic Set Theory}
			-- May~24,~2013 (in Bearbeitung)
			\tourl{http://www.qedeq.org/current/doc/math/qedeq\_set\_theory\_v1\_en.pdf}
			-- (09.03.2017)

			\bibitem{bib:Junktor}Wikipedia:
			\emph{Aussagenlogik} \chaptername~2.2 \emph{Mögliche Junktoren}
			-- 20.01.2016
			\tourl{https://de.wikipedia.org/wiki/Junktor\#M.C3.B6gliche\_Junktoren}
			(09.03.2017)

			\bibitem{bib:Aussagenlogik}Wikipedia:
			\emph{Aussagenlogik} \chaptername~4 \emph{Formaler Zugang}
			-- 13.02.2017
			\tourl{https://de.wikipedia.org/wiki/Aussagenlogik\#Formaler\_Zugang}
			(09.03.2017)

			\bibitem{bib:Identitaet}Wikipedia:
			\emph{Identität (Logik)} \chaptername~2.3
			\emph{Identität in der Informatik}
			-- 24.02.2017
			\tourl{https://de.wikipedia.org/wiki/Identit\%C3\%A4t\_(Logik)\#Identit.C3.A4t\_in\_der\_Informatik}
			-- 03.03.2017 (17.04.2017)

			\bibitem{bib:Mengenlehre}Wikipedia:
			\emph{Mengenlehre}
			-- 03.03.2017
			\tourl{https://de.wikipedia.org/wiki/Mengenlehre}
			(09.03.2017)

			\bibitem{bib:Praedikatenlogik}Wikipedia:
			\emph{Prädikatenlogik erster Stufe}
			-- 24.02.2017
			\tourl{https://de.wikipedia.org/wiki/Pr\%C3\%A4dikatenlogik\_erster\_Stufe}
			(09.03.2017)

			\bibitem{bib:Schlussregel}Wikipedia:
			\emph{Schlussregel}
			-- 01.05.2017
			\tourl{https://de.wikipedia.org/wiki/Schlussregel}
			(29.03.2017)

			\bibitem{bib:NatuerlichesSchliessen}Wikipedia:
			\emph{Natürliches Schließen}
			-- 01.05.2017
			\tourl{https:https://de.wikipedia.org/wiki/Systeme\_nat\%C3\%BCrlichen\_Schlie\%C3\%9Fens}
			(26.10.2016)

		\end{thebibliography}
	\end{flushleft}
	\Thispagestyle%              veränderte Kopfzeile für diese Seite aktivieren

% Um Kopf- und Fußzeilen zu erhalten und korrekte Referenzen zu erhalten, müssen
% die Programme 'splitindices.pl', 'makeglossaries.exe' und
% 'ErzeugeVerzeichnisse.bat' aufgerufen werden.  Letzteres Programm fügt in die
% Dateien 'ASBA-*.ind' und 'ASBA.gls' nach der ersten bzw. zweiten Zeile je eine
% Zeile mit "  \insert*" bzw. "\insertglo" und dem anschließenden Kommentar
% "% -- Eingefuegt von 'ErzeugeVerzeichnisse.bat'" ein.
%
% Die Kommandos '\insert*' und '\insertglo' müssen in dieser Datei definiert
% werden.

	\beforechapter% Kopfzeile aktivieren und neue Seite
	\idxdictionary{Index}% Index ===============================================
	\newcommand*{\insertidx}{%
		\label{idx:Index}
	}
	\printindex[idx]

	\beforechapter% Kopfzeile aktivieren und neue Seite
	\idxdictionary{Symbolverzeichnis}% =========================================
	\newcommand*{\insertsym}{%
		\label{idx:Symbolverzeichnis}
	}
	\printindex[sym]


	\beforechapter% Kopfzeile aktivieren und neue Seite
	\glodictionary{\glossaryname}% Index =======================================
	\newcommand*{\insertglo}{%
		\label{glo:Glossar}
		\Thispagestyle
		\addcontentsline{toc}{section}{\glossaryname}% Eintrag ins Inhaltsverzeichnis
	}
	\printglossaries

%%%%%%%%%%%%%%%%%%%%%%%%%%%%%%%%%%%%%%%%%%%%%%%%%%%%%%%%%%%%%%%%%%%%%%%%%%%%%%%%
% Damit Änderungen in den Verzeichnissen im PDF-Dokument erscheinen, müssen
% "splitindex.pl" und "makeglossaries" aufgerufen
% und dann das TeX-Dokument erneut übersetzt werden:
%     <MiKTeXhome>\scripts\splitindex\splitindex.pl        ASBA.idx
%     <MiKTeXhome>\scriptsmiktex\bin\x64\makeglossaries.ex ASBA.glo
%%%%%%%%%%%%%%%%%%%%%%%%%%%%%%%%%%%%%%%%%%%%%%%%%%%%%%%%%%%%%%%%%%%%%%%%%%%%%%%%

\end{document}

%%%% Ende des Dokuments %%%%%%%%%%%%%%%%%%%%%%%%%%%%%%%%%%%%%%%%%%%%%%%%%%%%%%%%