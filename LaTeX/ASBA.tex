% !TeX root = ASBA.tex
% !TeX encoding = UTF-8
% !TeX spellcheck = de_DE

%% Datei ASBA.tex zur Erzeugung des Projektdokuments von ASBA.
%%
%% Copyright (C) 2017  Winfried Teschers
%%
%% This program is free software: you can redistribute it and/or modify
%% it under the terms of the GNU Affero General Public License as published
%% by the Free Software Foundation, either version 3 of the License, or
%% (at your option) any later version.
%%
%% This program is distributed in the hope that it will be useful,
%% but WITHOUT ANY WARRANTY; without even the implied warranty of
%% MERCHANTABILITY or FITNESS FOR A PARTICULAR PURPOSE.  See the
%% GNU Affero General Public License for more details.
%%
%% You should have received a copy of the GNU Affero General Public License
%% along with this program.  If not, see <http://www.gnu.org/licenses/>.
%%
%% Dr. Winfried Teschers
%% Anton-Günther-Str. 26c
%% 91083 Baiersdorf
%% Germany
%%
%% e-mail: winfried.teschers@t-online.de

\documentclass[english,ngerman,parskip=half,headsepline,footsepline]{scrreprt}

%%%% Pakete %%%%%%%%%%%%%%%%%%%%%%%%%%%%%%%%%%%%%%%%%%%%%%%%%%%%%%%%%%%%%%%%%%%%%

% allgemein
\usepackage[utf8]{inputenc} % Input encoding specification
\usepackage[T1]{fontenc}
\usepackage{lmodern}
\usepackage{scrlayer-scrpage}
\usepackage{geometry} % Flexible and complete interface to document dimensions.
\usepackage{microtype} % Subliminal refinements towards typographical perfection.
\usepackage{graphicx}
\usepackage[table]{xcolor} % Driver-independent color extensions for LaTeX and pdfLaTeX - vor "color"?
\usepackage{babel} % Multilingual support for plain TeX or LaTeX.
% mathematische Pakete
\usepackage{amsmath} %% ASM mathematical facilities for LaLeX.
\usepackage{amsfonts} % TeX fonts from the American Mathematical Society.
\usepackage{amssymb} % Symbols from the American Mathematical Society.
\usepackage{mathtools} % Mathematical tools to use with asmmath.
\usepackage{mathabx} % Three series of mathematical symbols.
\usepackage{mathpazo} % Fonts to typeset mathematics to match palatino.
\usepackage{cancel} % Place lines through maths formulae.
% Tabellen
\usepackage{ctable} % Flexible typesetting of table and figure floats using key/value directives
%% Das Paket ctable fast die Eigenschaften der Pakete
% \usepackage{array} %
% \usepackage{tabularx} % Erweiterung von tabular*
% \usepackage{booktabs} % Nicer layout of tables
%% zusammen und lädt zusätzlich noch die Pakete
% \usepackage{rotating} % Rotating tools, including rotated full page floats
% \usepackage{xspace} % Behandelt Zwischenraum nach Makros
% \usepackage{color} % LaTeX upport for color, for those dvi drivers that can produce coloured text
% \usepackage{xkeyval} % Extension of the keyval package
\usepackage{threeparttable} % Tables with captions and notes all the same width.
\usepackage{multirow} % Create tabular cell spanning multiple rows.
\usepackage{caption} % Customizing captions in floating environments.
\usepackage{diagbox} % Table heads with diagonal lines.
\usepackage{arydshln} % Draw dash-lines in array/tabular.
% Verweise
\usepackage{varioref}
\usepackage[colorlinks,linktoc=all]{hyperref} % Extensive support for hypertext in LaTeX - muss als letztes Paket angegeben werden.

%%% Einstellung von globalen Werten %%%%%%%%%%%%%%%%%%%%%%%%%%%%%%%%%%%%%%%%%%%%%

\geometry{textwidth=170mm,textheight=256mm,twoside}

% Kopfzeilen
\newcommand*{\texthead}[1]{\textnormal{\textsf{\textbf{#1}}}}
\newcommand*{\newchapter}[1]{
	\markboth{#1}{}
	\ohead{\texthead{ASBA}}
	\chead{\texthead{#1}}
	\ihead{\texthead{\chaptername~\thechapter}}
	\thispagestyle{scrheadings}
}
\newcommand*{\newsection}[1]{
	\markright{#1}
	\lehead{\texthead{ASBA}}
	\cehead{\texthead{\leftmark}}
	\rehead{\texthead{\chaptername~\thechapter}}
	\lohead{\texthead{\sectionname~\thesection}}
	\cohead{\texthead{#1}}
	\rohead{\texthead{ASBA}}
	\thispagestyle{scrheadings}
}

% Fußzeilen
\ifoot{\textnormal{\today}}
\cfoot{\textnormal{Winfried Teschers}}
\ofoot{\textnormal{\textbf{\thepage}}}
\deffootnote[10pt] % Markenbreite
{10pt} % Einzug - für Blocksatz: Markenbreite
{0pt} % Absatzeinzug für Folgeabsätze
{\makebox[9pt][r]{\textsuperscript{\thefootnotemark} }} % Fußnotenzeichen; < Markenbreite

% Vordefinierte Werte ändern
\setcounter{tocdepth}{2}
\setcounter{secnumdepth}{3}
\setlength\extrarowheight{1pt}
\captionsetup{labelfont=bf}
% Empfehlung aus: Herbert Voß, LaTeX Referenz, 3. Auflage, Berlin 2014; Seite 37f
\renewcommand{\floatpagefraction}{0.7} % Empfehlung: 0.5-0.8 Voreinstellung: 0.9
\renewcommand{\textfraction}{0.15}     %             0.1-0.3                 0.05
\renewcommand{\topfraction}{0.8}       %             0.5-0.85                0.9
\renewcommand{\bottomfraction}{0.5}    %             0.2-0.5                 0.9
\setcounter{topnumber}{3}              %                                     2
\setcounter{totalnumber}{15}           %                                     3

% Neue Elemente
\newcounter{Enumi}

%%%% Definitionen für den Mathematiksatz %%%%%%%%%%%%%%%%%%%%%%%%%%%%%%%%%%%%%%%%

%% logische Symbole - Siehe Anhang, Definition von aussagenlogischen Symbolen %%%

%% sonstige nützliche Kommandos %%%%%%%%%%%%%%%%%%%%%%%%%%%%%%%%%%%%%%%%%%%%%%%%%

\newcommand*{\sectionname}{Abschnitt}
\newcommand*{\subsectionname}{Unterabschnitt}
\newcommand*{\subsubsectionname}{Paragraph}
\newcommand*{\textbzw}{bzw.\@ }
\newcommand*{\textdh}{d.\@ h.\@}
\newcommand*{\textetc}{etc.\@}
\newcommand*{\textggf}{ggf.\@}
\newcommand*{\textusw}{usw.\@}
\newcommand*{\textzB}{z.\@ B.\@}
\newcommand*{\textZB}{Z.\@ B.\@}
\newcommand*{\clq}{'} % linker Anfang für Zeichen (character left quote)
\newcommand*{\crq}{'} % rechtes Ende für Zeichen (character right quote)
\newcommand*{\cse}{, } % Zeichen zwischen zwei Zeichen (character seperator)
\newcommand*{\csp}{\crqt\cset\clqt} % \clq für Tabellen
\newcommand*{\clqt}{} % \clq für Tabellen
\newcommand*{\crqt}{} % \crq für Tabellen
\newcommand*{\cset}{~} % \cse für Tabellen
\newcommand*{\cspt}{\crqt\cset\clqt} % \csp für Tabellen

%% Farben %%%%%%%%%%%%%%%%%%%%%%%%%%%%%%%%%%%%%%%%%%%%%%%%%%%%%%%%%%%%%%%%%%%%%%%

\definecolor{cNormalUse}{rgb}{.80,.80,.80}
\definecolor{cRareUse}{rgb}{.90,.90,.99}

%%%% Titelseite %%%%%%%%%%%%%%%%%%%%%%%%%%%%%%%%%%%%%%%%%%%%%%%%%%%%%%%%%%%%%%%%%

\titlehead{{\Large Dr. Winfried Teschers}\\Anton-Günther-Straße 26c\\91083 Baiersdorf\\{\footnotesize winfried.teschers@t-online.de}}
\subject{Projektdokument}
\title{{\Huge ASBA}\\Axiome, Sätze, Beweise und Auswertungen}
\subtitle{Projekt zur maschinellen Überprüfung von mathematischen Beweisen und deren Ausgabe in lesbarer Form}
\author{Winfried Teschers}
\date{\today}
\publishers{Es wird ein System beschrieben, das zu eingegebenen Axiomen, Sätzen, und Beweisen letztere prüft, Auswertungen generiert und zu gegebenen Ausgabeschemata eine Ausgabe der Elemente in üblicher Formelschreibweise im \LaTeX-Format erstellt.}

%%%% Dokument %%%%%%%%%%%%%%%%%%%%%%%%%%%%%%%%%%%%%%%%%%%%%%%%%%%%%%%%%%%%%%%%%%%

\begin{document}
	\maketitle
	\tableofcontents
	% Kopfzeilen
	\ohead{\texthead{ASBA}}
	\chead{\texthead{\contentsname}}
	\ihead{}
	\thispagestyle{scrheadings}
	\vfill
	Copyright \copyright\ 2017 Winfried Teschers
	\bigskip
	\selectlanguage{english}
	Permission is granted to copy, distribute and/or modify this document under the terms of the GNU Free Documentation License, Version~1.3 or any later version published by the Free Software Foundation; with no Invariant Sections, no Front-Cover Texts, and no Back-Cover Texts. You should have received a copy of the GNU Free Documentation License along with this document.  If not, see \url{http://www.gnu.org/licenses/}.
	\selectlanguage{ngerman}

	\cleardoublepage
	\chapter{Analyse} %%%%%%%%%%%%%%%%%%%%%%%%%%%%%%%%%%%%%%%%%%%%%%%%%%%%%%%%%%%
	\label{cha:Analyse}
	\newchapter{Analyse}

	In der Mathematik gibt es eine unüberschaubare Menge an Axiomen, Sätzen, Beweisen, Fachbegriffen\footnote{\emph{Fachbegriffe} sind Namen für Axiome, Sätze, Beweise und Fachgebiete. Symbole können als spezielle Fachbegriffe aufgefasst werden.} und Fachgebieten. Dabei soll ein \emph{Fachgebiet} einen Teil der Mathematik mit einer zugehörigen Basis von Axiomen, Sätzen und spezifischen Fachbegriffen sein, zum Beispiel \emph{Logik}, \emph{Mengenlehre} und \emph{Gruppentheorie}\footnote{Ein Fachgebiet kann hier sehr klein sein und im Extremfall kein einziges Element enthalten. \emph{Umgebung} wäre in diesem Projekt eine bessere Bezeichnung, könnte aber zu Verwechslungen führen, da dies schon ein verbreiteter Fachbegriff ist.}. Zu den meisten Fachgebieten gibt es auch noch ungelöste Probleme.

	Es fehlt ein System, das einen Überblick bietet und die Möglichkeit, Beweise automatisch zu überprüfen. Außerdem sollte all dies in üblicher mathematischer Schreibweise ein- und ausgegeben werden können.

	Ein System mit ähnlicher Aufgabenstellung findet sich im GitHub Projekt Hilbert~II (\seename~\cite{bib:HilbertII, bib:qedeq}). Einige Ideen sind von dort übernommen worden.

	{% Die drei jetzt definierten Kommandos sollen nur lokal gelten.
		\newcommand*{\vsll}{\hspace{-1.0cm}}
		\newcommand*{\vsl}{\hspace{-6pt}\vsll}
		\newcommand*{\vsc}{\hspace{6pt}}

		\section{Fragen} %===========================================================
		\label{sec:Fragen}
		\newsection{Fragen}

		Einige der Fragen, die in diesem Zusammenhang auftauchen, werden hier formuliert:

		\begin{enumerate}

			\item \label{Frage:Grundlagen} \emph{Grundlagen}: Was sind die Grundlagen? Zum Beispiel welche Logik und Mengenlehre.

			\item \label{Frage:Basis} \emph{Basis}: Welche wichtigen Axiome, Sätze, Beweise, Fachbegriffe und Fachgebiete gibt es? Welche davon sind Standard?

			\item \label{Frage:Axiome} \emph{Axiome}: Welche Axiome werden bei einem Satz oder Beweis vorausgesetzt? Allgemein anerkannte oder auch strittige, wie zum Beispiel den \emph{Satz vom ausgeschlossenen Dritten} (\emph{tertium non datur}) oder das \emph{Auswahlaxiom}.

			\item \label{Frage:Beweis} \emph{Beweis}: Ist ein Beweis fehlerfrei?

			\item \label{Frage:Konstruktion} \emph{Konstruktion}: Gibt es einen konstruktiven Beweis?

			\item \label{Frage:Vergleiche} \emph{Vergleiche}: Welcher Beweis ist besser? Nach welchem Kriterium? Zum Beispiel elegant, kurz, einsichtig oder wenige Axiome. Was heißt eigentlich \emph{elegant}?

			\item \label{Frage:Definitionen} \emph{Definitionen}: Was ist mit einem Fachbegriff oder Fachgebiet jeweils genau gemeint? Zum Beispiel \emph{Stetigkeit}, \emph{Integral} und \emph{Analysis}.

			\item \label{Frage:Abhängigkeiten} \emph{Abhängigkeiten}: Wie heißt ein Fachbegriff oder Fachgebiet in einer anderen Sprache? Ist wirklich dasselbe gemeint? Was ist mit Fachbegriffen in verschiedenen Fachgebieten?

			\item \label{Frage:Überblick} \emph{Überblick}: Ist ein Axiom, Satz, Beweis, Fachbegriff oder Fachgebiet schon einmal \textendash\ \textggf\ abweichend \textendash\ definiert, formuliert oder bewiesen worden?

			\item \label{Frage:Darstellung} \emph{Darstellung}: Wie kann man einen Satz und den zugehörigen Beweis \textendash\ \textggf\ auch spezifisch für ein Fachgebiet \textendash\ darstellen?

			\item \label{Frage:Forschung} \emph{Forschung}: Welche Probleme gibt es noch zu erforschen.

		\end{enumerate}

		\section{Eigenschaften} %======================================================
		\label{sec:Eigenschaften}
		\newsection{Eigenschaften}

		Ausgehend von den Fragen in \sectionname~\vref{sec:Fragen} soll ein System entwickelt werden, das die folgenden Eigenschaften hat:

		\begin{enumerate}

			\item \label{Eigenschaft:Daten} \emph{Daten}: Axiome, Sätze, Beweise, Fachbegriffe und Fachgebiete können in formaler Form gespeichert werden \textendash\ auch nicht oder unvollständig bewiesene Sätze. Dabei soll die übliche mathematische Schreibweise verwendet werden können.

			\item \label{Eigenschaft:Definitionen} \emph{Definitionen}: Es können Fachbegriffe für Axiome, Sätze, Beweise und Fachgebiete \textendash\ letztere mit eigenen Axiomen, Sätzen, Beweisen, Fachbegriffen und über- oder untergeordneten Fachgebieten \textendash\ definiert werden. Die Definitionen dürfen wiederum an dieser Stelle schon bekannte Fachbegriffe und Fachgebiete verwenden.

			\item \label{Eigenschaft:Prüfung} \emph{Prüfung}: Vorhandene Beweise können automatisch geprüft werden.

			\item \label{Eigenschaft:Ausgaben} \emph{Ausgaben}: Die Axiome, Sätze und Beweise können in üblicher Schreibweise \textendash\ abhängig von Sprache und Fachgebiet \textendash\ ausgegeben werden.

			\item \label{Eigenschaft:Auswertungen} \emph{Auswertungen}: Zusätzlich zur Ausgabe der gespeicherten Daten sind verschiedene Auswertungen möglich, unter anderem für die meisten der unter \sectionname~\vref{sec:Fragen} behandelten Fragen.

			\setcounter{Enumi}{\value{enumi}}% Die Nummerierung soll fortgeführt werden.
		\end{enumerate}

		Damit das System nicht umsonst erstellt wird und möglichst breite Verwendung findet, werden noch zwei Punkte angefügt:

		\begin{enumerate}
			\setcounter{enumi}{\value{Enumi}}

			\item \label{Eigenschaft:Lizenz} \emph{Lizenz}: Die Software ist \emph{Open Source}.

			\item \label{Eigenschaft:Akzeptanz} \emph{Akzeptanz}: Das System wird von Mathematikern akzeptiert und verwendet.

		\end{enumerate}

		\tablename~\vref{tab:Fragen->Eigenschaften} zeigt, wie sich die Eigenschaften zu den Fragen in \sectionname~\vref{sec:Fragen} verhalten. Mit einem X werden die Spalten einer Zeile markiert, deren zugehörige Eigenschaften zur Beantwortung der entsprechenden Frage beitragen sollen. Idealerweise sollte die Erfüllung aller angegebenen Eigenschaften alle gestellten Fragen beantworten, was allerdings illusorisch ist.

		\begin{table}
			\begin{tabularx}{\linewidth-10.95pt}{@{\hspace{.5cm}}rl@{\extracolsep{\fill}}|*{7}{c}@{\hspace{1cm}}|}
				\multicolumn{2}{l|}{\diagbox[height=3.0cm,width=4.5cm]{\textbf{Frage}\\~}{\\\textbf{Eigenschaft}}}
				&\rotatebox{90}{\mbox{\vsl\ref{Eigenschaft:Daten}\vsc Daten}}
				&\rotatebox{90}{\mbox{\vsl\ref{Eigenschaft:Definitionen}\vsc Definitionen}}
				&\rotatebox{90}{\mbox{\vsl\ref{Eigenschaft:Prüfung}\vsc Prüfung}}
				&\rotatebox{90}{\mbox{\vsl\ref{Eigenschaft:Ausgaben}\vsc Ausgaben}}
				&\rotatebox{90}{\mbox{\vsl\ref{Eigenschaft:Auswertungen}\vsc Auswertungen }}
				&\rotatebox{90}{\mbox{\vsl\ref{Eigenschaft:Lizenz}\vsc Lizenz}}
				&\rotatebox{90}{\mbox{\vsl\ref{Eigenschaft:Akzeptanz}\vsc Akzeptanz}}
				\\\hline
				\ref{Frage:Grundlagen}&Grundlagen&X&X&-&X&X&-&-\\
				\ref{Frage:Basis}&Basis&X&X&-&X&X&-&-\\
				\ref{Frage:Axiome}&Axiome&X&X&-&X&X&-&-\\
				\hdashline[2pt/2pt]
				\ref{Frage:Beweis}&Beweis&X&-&X&X&-&-&-\\
				\ref{Frage:Konstruktion}&Konstruktion&X&-&-&X&-&-&-\\
				\ref{Frage:Vergleiche}&Vergleiche&X&-&-&-&X&-&-\\
				\hdashline[2pt/2pt]
				\ref{Frage:Definitionen}&Definitionen&X&X&-&X&-&-&-\\
				\ref{Frage:Abhängigkeiten}&Abhängigkeiten&X&-&-&X&-&-&-\\
				\ref{Frage:Überblick}&Überblick&X&-&-&-&X&-&-\\
				\hdashline[2pt/2pt]
				\ref{Frage:Darstellung}&Darstellung&-&X&-&X&-&-&-\\
				\ref{Frage:Forschung}&Forschung&X&-&-&-&X&-&-\\
				\hline
			\end{tabularx}
			\caption{Fragen $\to$ Eigenschaften}
			\label{tab:Fragen->Eigenschaften}
		\end{table}

		\section{Ziele} %========================================================
		\label{sec:Ziele}
		\newsection{Ziele}

		Um die Eigenschaften von \sectionname~\vref{sec:Eigenschaften} zu erreichen, werden für das System die folgenden Ziele\footnote{ Es sind eigentlich Anforderungen. Da dieser Begriff aber auch im \chaptername~\vref{cha:Design} verwendet wird, werden die Anforderungen hier \emph{Ziele} genannt.} gesetzt:

		\begin{enumerate}

			\item \label{Ziel:Daten} \emph{Daten}: Es enthält möglichst viele wichtige Axiome, Sätze, Beweise, Fachbegriffe, Fachgebiete und Ausgabeschemata\footnote{Um den Punkt~\ref{Eigenschaft:Ausgaben} von \sectionname~\vref{sec:Eigenschaften} erfüllen zu können, werden noch fachgebietsspezifische Ausgabeschemata benötigt, welche die Art der Ausgaben beschreiben.}.

			\item \label{Ziel:Form} \emph{Form}: Die Daten liegt in formaler, geprüfter Form vor.

			\item \label{Ziel:Eingaben} \emph{Eingaben}: Die Eingabe von Daten erfolgt in einer formalen Syntax unter Verwendung der üblichen mathematischen Schreibweise.

			\item \label{Ziel:Prüfung} \emph{Prüfung}: Vorhandene Beweise können automatisch geprüft werden.

			\item \label{Ziel:Ausgaben} \emph{Ausgaben}: Die Ausgabe kann in einer eindeutigen, formalen Syntax gemäß vorhandener Ausgabeschemata erfolgen.

			\item \label{Ziel:Auswertungen} \emph{Auswertungen}: Zusätzlich zur Ausgabe der Daten sind verschiedene Auswertungen möglich. Insbesondere kann zu jedem Beweis angegeben werden, wie viele Beweisschritte und welche Axiome und Sätze\footnote{Sätze, die quasi als Axiome verwendet werden.} er verwendet.

			\item \label{Ziel:Anpassbarkeit} \emph{Anpassbarkeit}: Fachbegriffe und die Darstellung bei der Ausgabe können mit Hilfe von \textendash\ gegebenenfalls unbenannten \textendash\ untergeordneten Fachgebieten angepasst werden.

			\item \label{Ziel:Individualität} \emph{Individualität}: Axiome und Sätze können für jeden Beweis individuell vorausgesetzt werden. Dabei sind fachgebietsspezifische Fachbegriffe erlaubt.

			\item \label{Ziel:Internet} \emph{Internet}: Die Daten können auf mehrere Dateien verteilt sein. Ein Teil davon \textendash\ oder sogar alle \textendash\ können im Internet liegen.

			\item \label{Ziel:Kommunikation} \emph{Kommunikation}: Die Kommunikation mit dem System kann mit den Fachbegriffen der einzelnen Fachgebiete erfolgen.

			\item \label{Ziel:Zugriff} \emph{Zugriff}: Der Zugriff auf das System kann lokal und über das Internet erfolgen.

			\item \label{Ziel:Unabhängigkeit} \emph{Unabhängigkeit}: Das System kann online und offline arbeiten.

			\item \label{Ziel:Rekursion} \emph{Rekursion}: Es kann rekursiv über alle verwendeten Dateien \textendash\ auch solchen, die im Internet liegen \textendash\ ausgewertet werden.

			\item \label{Ziel:Bedienbarkeit} \emph{Bedienbarkeit}: Das System ist einfach zu bedienen.

			\item \label{Ziel:Lizenz} \emph{Lizenz}: Die Software ist \emph{Open Source}.

			\setcounter{Enumi}{\value{enumi}}% Die Nummerierung soll fortgeführt werden.
		\end{enumerate}

		Der Punkt \ref{Ziel:Zwischenspeicher} wurde noch eingefügt, damit das System effizient arbeiten kann und um die Akzeptanz zu erhöhen:

		\begin{enumerate}
			\setcounter{enumi}{\value{Enumi}}

			\item \label{Ziel:Zwischenspeicher} \emph{Zwischenspeicher}: Wichtige Auswertungen können angehängt an vorhandene oder separat in eigenen Dateien zwischengespeichert werden.

		\end{enumerate}

		\tablename~\vref{tab:Eigenschaften->Ziele} zeigt wieder, wie sich die Ziele zu den Eigenschaften in \sectionname~\vref{sec:Eigenschaften} verhalten. Mit einem X werden wieder die Spalten einer Zeile markiert, deren zugehörige Ziele zur Sicherstellung der entsprechenden Eigenschaft beitragen sollen. Idealerweise sollte durch Erreichen aller aufgestellten Ziele das System alle angegebenen Eigenschaften aufweisen, was wahrscheinlich ebenfalls illusorisch ist.

		\begin{table}
			\begin{tabularx}{\linewidth-10.95pt}{@{\hspace{.3cm}}rl@{\extracolsep{\fill}}|*{16}{c}@{\hspace{0.4cm}}|}
				\multicolumn{2}{l|}{\diagbox[height=3.0cm,width=4.0cm]{\textbf{Eigenschaft}\\~}{\\\\\\\textbf{Ziel}}}
				&\rotatebox{90}{\mbox{\vsll\ref{Ziel:Daten}\vsc Daten}}
				&\rotatebox{90}{\mbox{\vsll\ref{Ziel:Form}\vsc Form}}
				&\rotatebox{90}{\mbox{\vsll\ref{Ziel:Eingaben}\vsc Eingaben}}
				&\rotatebox{90}{\mbox{\vsll\ref{Ziel:Prüfung}\vsc Prüfung}}
				&\rotatebox{90}{\mbox{\vsll\ref{Ziel:Ausgaben}\vsc Ausgaben}}
				&\rotatebox{90}{\mbox{\vsll\ref{Ziel:Auswertungen}\vsc Auswertungen}}
				&\rotatebox{90}{\mbox{\vsll\ref{Ziel:Anpassbarkeit}\vsc Anpassbarkeit}}
				&\rotatebox{90}{\mbox{\vsll\ref{Ziel:Individualität}\vsc Individualität}}
				&\rotatebox{90}{\mbox{\vsll\ref{Ziel:Internet}\vsc Internet}}
				&\rotatebox{90}{\mbox{\vsl\ref{Ziel:Kommunikation}\vsc Kommunikation}}
				&\rotatebox{90}{\mbox{\vsl\ref{Ziel:Zugriff}\vsc Zugriff}}
				&\rotatebox{90}{\mbox{\vsl\ref{Ziel:Unabhängigkeit}\vsc Unabhängigkeit}}
				&\rotatebox{90}{\mbox{\vsl\ref{Ziel:Rekursion}\vsc Rekursion}}
				&\rotatebox{90}{\mbox{\vsl\ref{Ziel:Bedienbarkeit}\vsc Bedienbarkeit}}
				&\rotatebox{90}{\mbox{\vsl\ref{Ziel:Lizenz}\vsc Lizenz}}
				&\rotatebox{90}{\mbox{\vsl\ref{Ziel:Zwischenspeicher}\vsc Zwischenspeicher}}
				\\\hline
				\ref{Eigenschaft:Daten}&Daten&X&X&X&-&-&-&-&-&-&-&-&-&-&-&-&-\\
				\ref{Eigenschaft:Definitionen}&Definitionen&X&-&X&-&-&-&-&-&-&-&-&-&-&-&-&-\\
				\ref{Eigenschaft:Prüfung}&Prüfung&-&-&-&X&-&-&-&-&-&-&-&-&-&-&-&-\\
				\hdashline[2pt/2pt]
				\ref{Eigenschaft:Ausgaben}&Ausgaben&-&-&-&-&X&-&-&-&-&-&-&-&-&-&-&-\\
				\ref{Eigenschaft:Auswertungen}&Auswertungen&-&-&-&-&-&X&-&-&-&-&-&-&-&-&-&-\\
				\ref{Eigenschaft:Lizenz}&Lizenz&-&-&-&-&-&-&-&-&-&-&-&-&-&-&X&-\\
				\hdashline[2pt/2pt]
				\ref{Eigenschaft:Akzeptanz}&Akzeptanz&X&X&X&X&X&X&X&X&X&X&X&X&X&X&X&X\\
				\hline
			\end{tabularx}
			\caption{Eigenschaften $\to$ Ziele}
			\label{tab:Eigenschaften->Ziele}
		\end{table}

		\section{Zusammenfassung} %========================================================
		\label{sec:Zusammenfassung}
		\newsection{Zusammenfassung}

		Die \tablename~\vref{tab:Fragen->Ziele} ist eine Kombination aus den \tablename n~\vref{tab:Fragen->Eigenschaften} und~\vref{tab:Eigenschaften->Ziele} und zeigt, wie sich die Ziele in \sectionname~\vref{sec:Ziele} zu den Fragen in \sectionname~\vref{sec:Fragen} verhalten. Auch hier werden mit einem X die Spalten einer Zeile markiert, deren zugehörige Ziele für die Beantwortung der entsprechenden Frage nötig sind. Mit einem kleinen x werden sie markiert, wenn sie zur Beantwortung der Fragen nicht nötig, aber von Interesse sind. Idealerweise sollte das Erreichen aller aufgestellten Ziele wieder alle gestellten Fragen beantworten, was natürlich auch illusorisch ist.

		\begin{table}
			\begin{tabularx}{\linewidth-10.95pt}{@{\hspace{.3cm}}rl@{\extracolsep{\fill}}|*{15}{c}@{\hspace{0.4cm}}|}
				\multicolumn{2}{l|}{\diagbox[height=3.0cm,width=4.0cm]{\textbf{Frage}\\~}{\\\\\\\textbf{Ziel}}}
				&\rotatebox{90}{\mbox{\vsll\ref{Ziel:Daten}\vsc Daten}}
				&\rotatebox{90}{\mbox{\vsll\ref{Ziel:Form}\vsc Form}}
				&\rotatebox{90}{\mbox{\vsll\ref{Ziel:Eingaben}\vsc Eingaben}}
				&\rotatebox{90}{\mbox{\vsll\ref{Ziel:Prüfung}\vsc Prüfung}}
				&\rotatebox{90}{\mbox{\vsll\ref{Ziel:Ausgaben}\vsc Ausgaben}}
				&\rotatebox{90}{\mbox{\vsll\ref{Ziel:Auswertungen}\vsc Auswertungen}}
				&\rotatebox{90}{\mbox{\vsll\ref{Ziel:Anpassbarkeit}\vsc Anpassbarkeit}}
				&\rotatebox{90}{\mbox{\vsll\ref{Ziel:Individualität}\vsc Individualität}}
				&\rotatebox{90}{\mbox{\vsll\ref{Ziel:Internet}\vsc Internet}}
				&\rotatebox{90}{\mbox{\vsl\ref{Ziel:Kommunikation}\vsc Kommunikation}}
				&\rotatebox{90}{\mbox{\vsl\ref{Ziel:Zugriff}\vsc Zugriff}}
				&\rotatebox{90}{\mbox{\vsl\ref{Ziel:Unabhängigkeit}\vsc Unabhängigkeit}}
				&\rotatebox{90}{\mbox{\vsl\ref{Ziel:Rekursion}\vsc Rekursion}}
				&\rotatebox{90}{\mbox{\vsl\ref{Ziel:Bedienbarkeit}\vsc Bedienbarkeit}}
				&\rotatebox{90}{\mbox{\vsl\ref{Ziel:Lizenz}\vsc Lizenz}}
				\\\hline
				\ref{Frage:Grundlagen}&Grundlagen&X&X&X&-&X&X&x&-&-&-&-&-&-&-&-\\
				\ref{Frage:Basis}&Basis&X&X&X&-&X&X&x&x&-&-&-&-&-&-&-\\
				\ref{Frage:Axiome}&Axiome&X&X&X&-&X&X&x&-&-&-&-&-&-&-&-\\
				\hdashline[2pt/2pt]
				\ref{Frage:Beweis}&Beweis&X&X&X&X&X&-&-&x&-&-&-&-&-&-&-\\
				\ref{Frage:Konstruktion}&Konstruktion&X&X&X&-&X&-&-&x&-&-&-&-&-&-&-\\
				\ref{Frage:Vergleiche}&Vergleiche&X&X&X&-&-&X&-&x&-&-&-&-&-&-&-\\
				\hdashline[2pt/2pt]
				\ref{Frage:Definitionen}&Definitionen&X&X&X&-&X&-&x&-&-&-&-&-&-&-&-\\
				\ref{Frage:Abhängigkeiten}&Abhängigkeiten&X&X&X&-&X&-&x&-&-&-&-&-&-&-&-\\
				\ref{Frage:Überblick}&Überblick&X&X&X&-&-&X&x&-&-&-&-&-&-&-&-\\
				\hdashline[2pt/2pt]
				\ref{Frage:Darstellung}&Darstellung&X&-&X&-&X&-&x&-&-&-&-&-&-&-&-\\
				\ref{Frage:Forschung}&Forschung&X&X&X&-&-&X&x&-&-&-&-&-&-&-&-\\
				\hline
				\multicolumn{17}{l|}{Die nächsten beiden Punkte sind Eigenschaften aus \sectionname~\vref{sec:Eigenschaften}:}\\
				\hline
				\ref{Eigenschaft:Lizenz}&Lizenz&-&-&-&-&-&-&-&-&-&-&-&-&-&-&X\\
				\ref{Eigenschaft:Akzeptanz}&Akzeptanz&X&X&X&X&X&X&X&X&X&X&X&X&X&X&X\\
				\hline
			\end{tabularx}
			\caption{Fragen $\to$ Ziele}
			\label{tab:Fragen->Ziele}
		\end{table}
	}

	\cleardoublepage
	\chapter{Design} %%%%%%%%%%%%%%%%%%%%%%%%%%%%%%%%%%%%%%%%%%%%%%%%%%%%%%%%%%%%
	\label{cha:Design}
	\newchapter{Design}

	Diese Projekt soll Open Source sein. Daher gilt für die Dokumente die \emph{GNU Free Documentation License (FDL)} und für die Software die \emph{GNU Affero General Public License (APGL)}. Die \emph{GNU General Public License (GPL)} reicht für die Software nicht, da das Programm auch mittels eines Servers betrieben werden kann und soll. Damit das Projekt gegebenenfalls durch verschiedene Entwickler gleichzeitig bearbeitet werden kann und wegen des Konfigurationsmanagements wurde es als ein GitHub Projekt erstellt (\seename~\cite{bib:ASBA}).

	Wenn die Lizenzen nicht mitgeliefert wurden, können sie unter \url{http://www.gnu.org/licenses/} gefunden werden.

	\section{Anforderungen}
	\label{Anforderungen}
	\newsection{Anforderungen}

	Die Anforderungen ergeben sich zunächst aus den Zielen in \sectionname~\vref{sec:Ziele}. Die beiden Ziele \ref{Ziel:Daten}~\emph{Daten} und \ref{Ziel:Lizenz}~\emph{Lizenz} sind für die Entwicklung des Systems von sekundärer Bedeutung und werden daher in diesen \sectionname\ nicht übernommen. Die anderen Ziele werden noch verfeinert.

	\textbf{> > > ZIELE in Anforderungen umwandeln. < < <} % ZIELE in Anforderungen umwandeln.
	\begin{enumerate}

		\item \label{Anforderung:Form} \emph{Form}: Die Daten liegt in formaler, geprüfter Form vor. (\seename\ Ziel~\vref{Ziel:Form})

		\item \label{Anforderung:Eingaben} \emph{Eingaben}: Die Eingabe von Daten erfolgt in einer formalen Syntax unter Verwendung der üblichen mathematischen Schreibweise. Folgende Daten können eingegeben werden:
		\begin{enumerate}
			\item Axiome
			\item Sätze
			\item Beweise
			\item Fachbegriffe
			\item Fachgebiete
			\item Ausgabeschemata
		\end{enumerate}
		Dabei sind alle Begriffe nur innerhalb eines Fachgebietes und seiner untergeordneten Fachgebiete gültig, solange sie nicht umdefiniert werden. Das oberste Fachgebiet ist die ganze Mathematik. (\seename\ Ziel~\vref{Ziel:Eingaben})

		\item \label{Anforderung:Prüfung} \emph{Prüfung}: Vorhandene Beweise können automatisch geprüft werden. (\seename\ Ziel~\vref{Ziel:Prüfung})

		\item \label{Anforderung:Ausgaben} \emph{Ausgaben}: Die Ausgabe kann in einer eindeutigen, formalen Syntax gemäß vorhandener Ausgabeschemata erfolgen. (\seename\ Ziel~\vref{Ziel:Ausgaben})

		\item \label{Anforderung:Auswertungen} \emph{Auswertungen}: Zusätzlich zur Ausgabe der Daten sind verschiedene Auswertungen möglich. Insbesondere kann zu jedem Beweis angegeben werden, wie viele Beweisschritte und welche Axiome und Sätze\footnote{Sätze, die quasi als Axiome verwendet werden.} er verwendet. (\seename\ Ziel~\vref{Ziel:Auswertungen})

		\item \label{Anforderung:Anpassbarkeit} \emph{Anpassbarkeit}: Fachbegriffe und die Darstellung bei der Ausgabe können mit Hilfe von \textendash\ gegebenenfalls unbenannten \textendash\ untergeordneten Fachgebieten angepasst werden. (\seename\ Ziel~\vref{Ziel:Anpassbarkeit})

		\item \label{Anforderung:Individualität} \emph{Individualität}: Axiome und Sätze können für jeden Beweis individuell vorausgesetzt werden. Dabei sind fachgebietsspezifische Fachbegriffe erlaubt. (\seename\ Ziel~\vref{Ziel:Individualität})

		\item \label{Anforderung:Internet} \emph{Internet}: Die Daten können auf mehrere Dateien verteilt sein. Ein Teil davon \textendash\ oder sogar alle \textendash\ können im Internet liegen. (\seename\ Ziel~\vref{Ziel:Internet})

		\item \label{Anforderung:Kommunikation} \emph{Kommunikation}: Die Kommunikation mit dem System kann mit den Fachbegriffen der einzelnen Fachgebiete erfolgen. (\seename\ Ziel~\vref{Ziel:Kommunikation})

		\item \label{Anforderung:Zugriff} \emph{Zugriff}: Der Zugriff auf das System kann lokal und über das Internet erfolgen. (\seename\ Ziel~\vref{Ziel:Zugriff})

		\item \label{Anforderung:Unabhängigkeit} \emph{Unabhängigkeit}: Das System kann offline und online arbeiten. (\seename\ Ziel~\vref{Ziel:Unabhängigkeit})

		\item \label{Anforderung:Rekursion} \emph{Rekursion}: Es kann rekursiv über alle verwendeten Dateien \textendash\ auch solchen, die im Internet liegen \textendash\ ausgewertet werden. (\seename\ Ziel~\vref{Ziel:Rekursion})

		\item \label{Anforderung:Bedienbarkeit} \emph{Bedienbarkeit}: Das System ist einfach zu bedienen. (\seename\ Ziel~\vref{Ziel:Bedienbarkeit})

	\end{enumerate}

	\textbf{> > > ANFORDERUNGEN bearbeiten. < < <} % ANFORDERUNGEN bearbeiten.

	\section{Datenstruktur}
	\label{Datenstruktur}
	\newsection{Datenstruktur}

	\textbf{> > > DATENSTRUKTUR bearbeiten. < < <} % DATENSTRUKTUR bearbeiten.

	\section{Bausteine}
	\label{Bausteine}
	\newsection{Bausteine}

	\textbf{> > > BAUSTEINE bearbeiten. < < <} % BAUSTEINE bearbeiten.

	\appendix
	\chapter{\appendixname} %%%%%%%%%%%%%%%%%%%%%%%%%%%%%%%%%%%%%%%%%%%%%%%%%%%%%
	\label{cha:\appendixname}
	\newchapter{\appendixname}

	\section{Werkzeuge} %========================================================
	\label{sec:Werkzeuge}
	\newsection{Werkzeuge}

	Da dies ein Open Source Projekt sein soll, müssen alle Werkzeuge, die zum Ablauf der Software erforderlich sind, ebenfalls Open Source sein. Für die reine Entwicklung sollte das auch gelten.

	Im Folgenden verweist der Pfeil~($\rightarrow$) stets auf einen Link ins Internet.

	\paragraph{Werkzeuge, die zum Ablauf der Software erforderlich sind}
	\begin{itemize}

		\item\label{Werkzeug:MiKTeX}\emph{MiK\TeX} für Dokumentation und Ausgaben in \LaTeX. $\rightarrow$~\url{https://miktex.org/} \textendash\ Lizenz \seename~\cite{bib:MiKTeX}

		\setcounter{Enumi}{\value{enumi}}
	\end{itemize}

	\paragraph{Werkzeuge, die für die Entwicklung verwendet werden}
	\begin{itemize}
		\setcounter{enumi}{\value{Enumi}}

		\item\label{Werkzeug:GitHub}\emph{GitHub} als Online Konfigurationsmanagementsystem zur Zusammenarbeit verschiedener Entwickler. $\rightarrow$~\url{https://github.com/} \textendash\ Lizenz \seename~\cite{bib:GPLii}

		\item\label{Werkzeug:Git}GitHub benötigt \emph{Git} als Konfigurationsmanagementsystem. $\rightarrow$~\url{https://git-scm.com/} \textendash\ Lizenz \seename~\cite{bib:GPLii}

		\item\label{Werkzeug:VSC}\emph{Visual Studio Community 2017}\footnote{Visual Studio Community ist zwar nicht Open Source, darf aber zur Entwicklung von Open Source Software unentgeltlich verwendet werden.} (\emph{VS}) als Entwicklungsumgebung für C++. $\rightarrow$~\url{https://www.visualstudio.com/downloads/} \textendash\ Lizenz \seename~\cite{bib:EULA}

		\item\label{Werkzeug:Doxygen}\emph{Doxygen} als Dokumentationssystem für C++. $\rightarrow$~\url{http://www.stack.nl/~dimitri/doxygen/} \textendash\ Lizenz \seename~\cite{bib:GPLii}

		\item\label{Werkzeug:Ghostscript}Doxygen benötigt \emph{Ghostscript} als Interpreter für Postscript und PDF. $\rightarrow$~\url{http://ghostscript.com/} \textendash\ Lizenz \seename~\cite{bib:AGPL}

		\item\label{Werkzeug:Graphviz}Doxygen benötigt \emph{Graphviz} mit \emph{Dot} zur Erzeugung und Visualisierung von Graphen. $\rightarrow$~\url{http://www.graphviz.org/Home.php} \textendash\ Lizenz \seename~\cite{bib:EPL}

		\setcounter{Enumi}{\value{enumi}}
	\end{itemize}

	\paragraph{Werkzeuge für die Entwicklung, die jeder Entwickler individuell durch andere ersetzten kann}
	\begin{itemize}
		\setcounter{enumi}{\value{Enumi}}

		\item\label{Werkzeug:TeXstudio}\emph{\TeX studio} als Editor für \LaTeX. $\rightarrow$~\url{http://www.texstudio.org/} \textendash\ Lizenz \seename~\cite{bib:GPLii}

		\item\label{Werkzeug:Notepadpp}\emph{Notepad++} als Text-Editor. $\rightarrow$~\url{https://notepad-plus-plus.org/} \textendash\ Lizenz \seename~\cite{bib:GPLi}

		\item\label{Werkzeug:WinMerge}\emph{WinMerge} zum Vergleich von Dateien und Verzeichnissen. $\rightarrow$~\url{http://winmerge.org/} \textendash\ Lizenz \seename~\cite{bib:GPLi}

		\setcounter{Enumi}{\value{enumi}}
	\end{itemize}

	\paragraph{Angedachte Werkzeuge}
	\begin{itemize}
		\setcounter{enumi}{\value{Enumi}}

		\item\label{Werkzeug:VSC DB}In \emph{Visual Studio Community 2015} integrierte Datenbank für Axiome, Sätze, Beweise, Fachbegriffe und Fachgebiete. \textendash\ Lizenz \seename~\cite{bib:EULA}

		\item\label{Werkzeug:RapidXml}\emph{RapidXml} für Ein- und Ausgabe in XML. $\rightarrow$~\url{http://rapidxml.sourceforge.net/index.htm} \textendash\ Lizenz \seename\ wahlweise~\cite{bib:BSLi} oder~\cite{bib:MIT}

	\end{itemize}

	\paragraph{Im Projekt \emph{qedeq} verwendete Werkzeuge}

	\textbf{> > > QEDEQ Werkzeuge auflisten? < < <} % QEDEQ Werkzeuge auflisten?

	\begin{itemize}
		\setcounter{enumi}{\value{Enumi}}

		\item\label{Werkzeug:Java}\emph{Java} als Programmiersprache \textendash\ Laufzeitumgebung. $\rightarrow$~\url{https://www.java.com/de/download/win10.jsp} \textendash\ Lizenz \seename~\cite{bib:JavaSE}

		\item\label{Werkzeug:Apache Ant}\emph{Apache Ant} als Java Bibliothek und Kommandozeilen-Werkzeug um Java Programme zu erzeugen. $\rightarrow$~\url{http://ant.apache.org/} \textendash\ Lizenz \seename~\cite{bib:Apacheii}

		\item\label{Werkzeug:Checkstyle}\emph{Checkstyle} zur statischen Code-Analyse für Java. $\rightarrow$~\url{http://checkstyle.sourceforge.net/} \textendash\ Lizenz \seename~\cite{bib:LGPLii}

		\item\label{Werkzeug:Clover}\emph{Clover}\footnote{Clover ist proprietäre Software, aber auf Anfrage frei für 30 Tage. Danach ist eine einmalige Lizenzgebühr fällig.} als Testwerkzeug zur Analyse der Code-Abdeckung. $\rightarrow$~\url{https://www.atlassian.com/software/clover/} \textendash\ Lizenz \seename~\cite{bib:Clover}

		\item\label{Werkzeug:Eclipse Java}\emph{Eclipse IDE for Java Developers} als Entwicklungsumgebung für Java. $\rightarrow$~\url{http://www.eclipse.org/downloads/packages/eclipse-ide-java-developers/neon1a/} \textendash\ Lizenz \seename~\cite{bib:OSI}

		\item\label{Werkzeug:JUnit}\emph{JUnit} zur Erzeugung von wiederholbaren Tests. $\rightarrow$~\url{http://junit.org/junit4/} \textendash\ Lizenz \seename~\cite{bib:EPL}

		\item\label{Werkzeug:Xerces2}\emph{Xerces2} als XML-Parser in Java. $\rightarrow$~\url{http://xerces.apache.org/xerces2-j/} \textendash\ Lizenzen \seename~\cite{bib:Apacheii, bib:SAX, bib:WDCDL, bib:WDCSNL}

		\setcounter{Enumi}{\value{enumi}}
	\end{itemize}

	\section{Aussagenlogik} %====================================================
	\label{sec:Aussagenlogik}
	\newsection{Aussagenlogik}

	\subsection{Konstante und Operatoren} %--------------------------------------
	\label{sub:Operatoren}

	Die \tablename~\vref{tab:Symbole} \footnote{Die \tablename\ basiert auf den Wahrheitstafeln in~\cite{bib:Junktor} Kapitel~2.2 und~\cite{bib:Rautenberg} Kapitel~1.1 Seite~3.} definiert für die zweiwertige Logik Konstanten- und Operatorsymbole über die Wahrheitswerte ihrer Anwendung. So ergeben sich, abhängig von den Wahrheitswerten der Operanden A und B \footnote{Im Gegensatz zu \subsubsectionname~\vref{subsub:Bausteine} können A und B hier beliebige Aussagen \textendash\ auch Formeln \textendash\ sein, nicht nur Atome.}, die in der \tablename\ angegebenen Wahrheitswerte für die Operationen. Die mit 0, 1 und 2 benannten Spalten werden jeweils nur für die 0-, 1- und 2-stelligen Operatoren, \textdh\ für die Konstanten, die unären und die binären Operatoren ausgefüllt. Dabei werden die Konstanten als 0-stellige Operatoren angesehen. Hat der Inhalt einer Zelle keine Relevanz, steht dort ein Minuszeichen, ist kein Wert bekannt, so bleibt sie leer.

	% Logische Operatoren als Addition und Multiplikation
	\newcommand*{\add}{+}
	\newcommand*{\mult}{\cdot}
	% Wahrheitswerte ------------------------------------------------------------
	\newcommand*{\texttrue}{W}  % in einem Kommentar stets 'W'
	\newcommand*{\textfalse}{F} % in einem Kommentar stets 'F'
	% Konstante -----------------------------------------------------------------
	\newcommand*{\ltrue}{\top}                           %       W       - wahr
%	\newcommand*{\lnfalse}{\notbot}                      %       "       - nicht falsch
	\newcommand*{\lfalse}{\bot}                          %       F       - falsch
%	\newcommand*{\lntrue}{\nottop}                       %       "       - nicht wahr
	% unäre Operatoren ----------------------------------------------------------
	%                                                       A  - W F     - Aussage A
%	\newcommand*{\lutrue}{\operatorname{\top}}           % *A  - W W     - wahr [unär]
%	\newcommand*{\lnufalse}{\operatorname{\notbot}}      % *A  - " "     - nicht falsch [unär]
%	             Klammern                                % (A) - W F     - geklammert
%	             \lnot                                   % *A  - F W     - nicht
%	\newcommand*{\lufalse}{\operatorname{\bot}}          % *A  - F F     - falsch [unär]
%	\newcommand*{\lnutrue}{\operatorname{\nottop}}       % *A  - " "     - nicht wahr [unär]
	% binäre Operatoren ---------------------------------------------------------
	%                                                       A  - W W F F - Aussage A
	%                                                       B  - W F W F - Aussage B
	% - - - - - - - - - - - - - - - - - - - - - - - - - - - - - - - - - - - - - -
%	\newcommand*{\lbtrue}{\operatorname{\top}}           % A*B - W W W W - wahr [binär]
%	\newcommand*{\lnbfalse}{\operatorname{\notbot}}      % A*B - " " " " - nicht falsch [binär]
	%            \lor                                    % A*B - W W W F - A oder B
	\newcommand*{\lnxor}{\operatorname{\cancel{\lxor}}}  % A*B - " " " " - nicht (entweder A oder B)
	\newcommand*{\lleftimp}{\leftarrow}                  % A*B - W W F W - A folgt aus B
	\newcommand*{\lleftimpA}{\Leftarrow}
	\newcommand*{\lleftimpB}{\subset}
	\newcommand*{\lleft}{\operatorname{\rfloor}}         % A*B - W W F F - A
	% - - - - - - - - - - - - - - - - - - - - - - - - - - - - - - - - - - - - - -
	\newcommand*{\lrightimp}{\rightarrow}                % A*B - W F W W - aus A folgt B
	\newcommand*{\lrightimpA}{\Rightarrow}
	\newcommand*{\lrightimpB}{\supset}
	\newcommand*{\limp}{\lrightimp}
	\newcommand*{\lright}{\operatorname{\lfloor}}        % A*B - W F W F - B
	\newcommand*{\lequiv}{\leftrightarrow}               % A*B - W F F W - A genau dann, wenn B
	\newcommand*{\lequivA}{\Leftrightarrow}
	%            \land                                   % A*B - W F F F - A und B
	\newcommand*{\landA}{\&}
	\newcommand*{\landB}{\mult}
	% - - - - - - - - - - - - - - - - - - - - - - - - - - - - - - - - - - - - - -
	\newcommand*{\lnand}{\uparrow}                       % A*B - F W W W - nicht (A und B)
	\newcommand*{\lnandA}{\barwedge}
	\newcommand*{\lnandB}{\mid}
	\newcommand*{\lxor}{\add}                            % A*B - F W W F - entweder A oder B
	\newcommand*{\lxorA}{\operatorname{\dot\lor}}
	\newcommand*{\lxorB}{\veebar}
	\newcommand*{\lxorC}{\oplus}
	\newcommand*{\lnequiv}{\nleftrightarrow}             % A*B - " " " " - nicht (A genau dann, wenn B)
	\newcommand*{\lnequivA}{\nLeftrightarrow}
	\newcommand*{\lnequivB}{\notequiv}
	\newcommand*{\lnright}{\lceil}                       % A*B - F W F W - nicht B
	\newcommand*{\lnrightimp}{\nrightarrow}              % A*B - F W F F - nicht (aus A folgt B)
	\newcommand*{\lnrightimpA}{\nRightarrow}
	\newcommand*{\lnrightimpB}{\nsupset}
	\newcommand*{\lnimp}{\lnrightimp}
	% - - - - - - - - - - - - - - - - - - - - - - - - - - - - - - - - - - - - - -
	\newcommand*{\lnleft}{\rceil}                        % A*B - F F W W - nicht A
	\newcommand*{\lnleftimp}{\nleftarrow}                % A*B - F F W F - nicht (A folgt aus B)
	\newcommand*{\lnleftimpA}{\nLeftarrow}
	\newcommand*{\lnleftimpB}{\nsubset}
	\newcommand*{\lnor}{\downarrow}                      % A*B - F F F W - nicht (A oder B)
	\newcommand*{\lnorA}{\operatorname{\overline\vee}}
%	\newcommand*{\lbfalse}{\operatorname{\bot}}          % A*B - F F F F - falsch [binär]
%	\newcommand*{\lnbtrue}{\operatorname{\nottop}}       % A*B - " " " " - nicht wahr [binär]
	% Gleichheit ----------------------------------------------------------------
	%              =
	%             \ne
	\newcommand*{\defeq}{\coloneqq}
	% Prioritäten ---------------------------------------------------------------
	\newcounter{prio}                                            \stepcounter{prio}
	\newcounter{pnequiv}   \setcounter{pnequiv}   {\value{prio}}
	\newcounter{pequiv}    \setcounter{pequiv}    {\value{prio}} \stepcounter{prio}
	\newcounter{pnleftimp} \setcounter{pnleftimp} {\value{prio}}
	\newcounter{pleftimp}  \setcounter{pleftimp}  {\value{prio}}
	\newcounter{pnrightimp}\setcounter{pnrightimp}{\value{prio}}
	\newcounter{prightimp} \setcounter{prightimp} {\value{prio}}
	\newcounter{pnimp}     \setcounter{pnimp}     {\value{prio}}
	\newcounter{pimp}      \setcounter{pimp}      {\value{prio}} \stepcounter{prio}
%	\newcounter{pnleft}    \setcounter{pnleft}    {\value{prio}}
%	\newcounter{pleft}     \setcounter{pleft}     {\value{prio}}
%	\newcounter{pnright}   \setcounter{pnright}   {\value{prio}}
%	\newcounter{pright}    \setcounter{pright}    {\value{prio}} \stepcounter{prio}
	\newcounter{padd}      \setcounter{padd}      {\value{prio}}
%	\newcounter{pnxor}     \setcounter{pnxor}     {\value{prio}}
	\newcounter{pxor}      \setcounter{pxor}      {\value{prio}}
	\newcounter{pnor}      \setcounter{pnor}      {\value{prio}}
	\newcounter{por}       \setcounter{por}       {\value{prio}} \stepcounter{prio}
	\newcounter{pmult}     \setcounter{pmult}     {\value{prio}}
	\newcounter{pnand}     \setcounter{pnand}     {\value{prio}}
	\newcounter{pand}      \setcounter{pand}      {\value{prio}} \stepcounter{prio}
	\newcounter{pnot}      \setcounter{pnot}      {\value{prio}}

	% ZELLEN vertikal zentrieren
	\begin{table}
		\renewcommand*{\clq}{}
		\renewcommand*{\crq}{}
		\renewcommand*{\cse}{~}
		\newcommand*{\tablegroup}{\hdashline[6pt/3pt]}
		\newcommand*{\tableline}{\hdashline[3pt/3pt]}
		\newcommand*{\gapline}{\cdashline{1-1}[1pt/3pt]\cdashline{9-11}[1pt/3pt]}
		\setlength\tabcolsep{3pt}
		\setlength\extrarowheight{1.5pt}
		\begin{threeparttable}
			\begin{tabularx}{\linewidth-10.95pt}{c||c:cc:cccc|X:X|c|}

				\textbf{Junktor}\tnote{1}&\textbf{0}&\multicolumn{2}{c:}{\textbf{1}}&\multicolumn{4}{c|}{\textbf{2}}& \textbf{Name}& \textbf{Sprechweise}\tnote{2}&\textbf{Prio}\\
				\hline %-------------------------------------------------------------
				A &-&\texttrue&\textfalse&\texttrue&\texttrue&\textfalse&\textfalse& -&Aussage A&-\\
				\tableline %...............................................
				B &-&-&-&\texttrue&\textfalse&\texttrue&\textfalse& -&Aussage B&-\\
				\hline\hline %===================================================

				\rowcolor{cRareUse}
				\clqt$\ltrue$\crqt &\texttrue&-&-&-&-&-&-& ~&Wahr&-\\
				\tableline %...............................................
				\rowcolor{cRareUse}
				\clqt$\lfalse$\crqt &\textfalse&-&-&-&-&-&-& ~&Falsch&-\\
				\hline %-------------------------------------------------------------

				~ &-&\texttrue&\texttrue&-&-&-&-& ~&&-\tnote{4}\\
				\tableline %...............................................
				\rowcolor{cNormalUse}
				\clqt$(\dots)$\crqt &-&\texttrue&\textfalse&-&-&-&-& Klammerung\tnote{3}&A ist geklammert&-\tnote{5}\\
				\tableline %...............................................
				\rowcolor{cNormalUse}
				\clqt$\lnot$\crqt &-&\textfalse&\texttrue&-&-&-&-& Negation&Nicht A&\thepnot\tnote{6}\\
				\tableline %...............................................
				~ &-&\textfalse&\textfalse&-&-&-&-& ~&&-\tnote{4}\\
				\hline %-------------------------------------------------------------

				~ &-&-&-&\texttrue&\texttrue&\texttrue&\texttrue& Tautologie&&-\tnote{4}\\
				\tableline %...............................................
				\rowcolor{cNormalUse}
				\clqt$\lor$\crqt &-&-&-&\texttrue&\texttrue&\texttrue&\textfalse& Disjunktion; Adjunktion;\newline Alternative&A oder B&\thepor\\
				\tableline %...............................................
				\rowcolor{cRareUse}
				\clqt$\lleftimp$\cspt$\lleftimpA$\cspt$\lleftimpB$\crqt &-&-&-&\texttrue&\texttrue&\textfalse&\texttrue& Replikation; Konversion&A folgt aus B&\thepleftimp\\
				\tableline %...............................................
				\clqt$\lleft$\crqt &-&-&-&\texttrue&\texttrue&\textfalse&\textfalse& Präpendenz&Identität von A&-\tnote{4}\\
				\tablegroup %----------------------------------------------

				\rowcolor{cNormalUse}
				\clqt$\lrightimp$\cspt$\lrightimpA$\cspt$\lrightimpB$\crqt &-&-&-&\texttrue&\textfalse&\texttrue&\texttrue& Implikation; Subjunktion;\newline Konditional&Wenn A so B; Aus A folgt B;\newline A nur dann wenn B&\theprightimp\\
				\tableline %...............................................
				\clqt$\lright$\crqt &-&-&-&\texttrue&\textfalse&\texttrue&\textfalse& Postpendenz&Identität von B&-\tnote{4}\\
				\tableline %...............................................
				\rowcolor{cNormalUse}
				\clqt$\lequiv$\cspt$\lequivA$\crqt &-&-&-&\texttrue&\textfalse&\textfalse&\texttrue& Äquivalenz \tnote{7}; Bijunktion;\newline Bikonditional&A genau dann wenn B;\newline A dann und nur dann wenn B&\thepequiv\\
				\gapline %. . . . . . . . . . . . . . . . . . . . . . . . .
				\clqt$\lnxor$\crqt &-&-&-&"&"&"&"& ~&&\\
				\tableline %...............................................
				\rowcolor{cNormalUse}
				\clqt$\land$\cspt$\landA$\cspt$\landB$\crqt &-&-&-&\texttrue&\textfalse&\textfalse&\textfalse& Konjunktion&{\small A und B; Sowohl A als auch B}&\thepand\\
				\tablegroup %----------------------------------------------

				\rowcolor{cRareUse}
				\clqt$\lnand$\cspt$\lnandA$\cspt$\lnandB$\crqt &-&-&-&\textfalse&\texttrue&\texttrue&\texttrue& NAND; Unverträglichkeit;\newline Sheffer-Funktion&Nicht zugleich A und B&\thepnand\\
				\tableline %...............................................
				\rowcolor{cRareUse}
				\clqt$\lxor$\cspt$\lxorA$\cspt$\lxorB$\cspt$\lxorC$\crqt &-&-&-&\textfalse&\texttrue&\texttrue&\textfalse& XOR; Antivalenz;\newline ausschließende Disjunktion&Entweder A oder B&\thepxor\\
				\gapline %. . . . . . . . . . . . . . . . . . . . . . . . .
				\clqt$\lnequiv$\cspt$\lnequivA$\cspt$\lnequivB$\crqt &-&-&-&"&"&"&"& Kontravalenz&&\thepnequiv\\
				\tableline %...............................................
				$\lnright$ &-&-&-&\textfalse&\texttrue&\textfalse&\texttrue& Postnonpendenz&Negation von B&-\tnote{4}\\
				\tableline %...............................................
				\clqt$\lnrightimp$\cspt$\lnrightimpA$\cspt$\lnrightimpB$\crqt &-&-&-&\textfalse&\texttrue&\textfalse&\textfalse& Postsektion&&\thepnrightimp\\
				\tablegroup %----------------------------------------------

				\clqt$\lnleft$\crqt &-&-&-&\textfalse&\textfalse&\texttrue&\texttrue& Pränonpendenz&Negation von A&-\tnote{4}\\
				\tableline %...............................................
				\clqt$\lnleftimp$\cspt$\lnleftimpA$\cspt$\lnleftimpB$\crqt &-&-&-&\textfalse&\textfalse&\texttrue&\textfalse& Präsektion&&\thepnleftimp\\
				\tableline %...............................................
				\rowcolor{cRareUse}
				\clqt$\lnor$\cspt$\lnorA$\crqt &-&-&-&\textfalse&\textfalse&\textfalse&\texttrue& NOR; Nihilation;\newline Peirce-Funktion&Weder A noch B&\thepnor\\
				\tableline %...............................................
				~ &-&-&-&\textfalse&\textfalse&\textfalse&\textfalse& Kontradiktion&&-\tnote{4}\\
				\hline\hline %===================================================

				\rowcolor{cNormalUse}
				$=$ &-&-&-&-&-&-&-& Identität; Gleichheit\tnote{7}&A ist gleich B&-\tnote{8}\\
				\tableline %...............................................
				\rowcolor{cNormalUse}
				$\defeq$ &-&-&-&-&-&-&-& Definition\tnote{7}&A ist definiert als B&-\tnote{8}\\
				\hline %_____________________________________________________________
			\end{tabularx}
			\begin{tablenotes}
				\footnotesize
				\item[1] Operatorsymbole. \clq$\subset$\crq, \clq$\supset$\crq, \clq$\nsubset$\crq\ und \clq$\nsupset$\crq, dürfen nicht mit den entsprechenden Mengensymbolen verwechselt werden; gleiches gilt für \clq$+$\crq\ und \clq$\cdot$\crq\ mit Addition und Multiplikation.
				\item[2] Ist eine Zelle in dieser Spalte leer, so sind die angegebenen Junktoren nur aus Symmetriegründen definiert.
				\item[3] Klammerung wird nicht nur bei logischen, sondern auch bei anderen Ausdrücken verwendet.
				\item[4] Wegen Nichtgebrauch wird keine Priorität angegeben.
				\item[5] Die Priorität der Klammern ist größer als die aller Operatoren.
				\item[6] Die Priorität der unären Operatoren muss größer sein als die aller mehrwertigen (\textzB\  binären) Operatoren. Wenn alle unären Operatoren auf derselben Seite des Operanden stehen, brauchen sie eigentlich keine Priorität, da die Auswertung nur von innen (dem Operanden) nach außen erfolgen kann.
				\item[7] Identität und Definition sind keine aussagenlogischen Operatoren. Daher werden keine Wahrheitswerte angegeben. Zudem dürfen sie nicht mit der (aussagenlogischen) Äquivalenz verwechselt werden.
				\item[8] Die Priorität von \clq$=$\crq\ und \clq$\defeq$\crq\ ist kleiner als die aller logischen Operatoren.
			\end{tablenotes}
			\caption{Definition von aussagenlogischen Symbolen.}
			\label{tab:Symbole}
		\end{threeparttable}
	\end{table}

	Für einige Junktoren sowie Namen und Sprechweisen sind auch Alternativen angegeben. Die meisten durchgestrichenen (\textdh\ negierten) Symbole sind ungebräuchlich und nur aus formalen Gründen aufgeführt. Normalerweise werden im Folgenden nur die in der Spalte \emph{Junktor} jeweils zuerst angegebenen Symbole verwendet.

	Um vollständig zu sein, \textdh\ alle 22 möglichen Kombinationen von Wahrheitswerten für höchstens zwei Variable zu berücksichtigen, enthält die \tablename\ auch viele ungebräuchliche Junktoren und Operationen. Die Zeilen mit den Klammern, den verbreitetsten Junktoren und \clq$=$\crq\ und \clq$\defeq$\crq\ sind in der \tablename\ grau hinterlegt. Hellgrau hinterlegt sind Zeilen mit weniger gebräuchlichen Junktoren. Die vier Operationen ohne Angabe eines Junktors sind uninteressant und brauchen keine Priorität.

	Wenn für eine bestimmte Kombination von Wahrheitswerten mehr als eine Zeile angegeben ist, so sind die zugehörigen Operationen in der zweiwertigen Aussagenlogik alle gleich. Bei der formalen Definition setzen wir aber keine Zweiwertigkeit voraus, so dass je nach Definition die Operationen verschiedene Ergebnisse liefern können.

	\subsection{Klammerregeln} %-------------------------------------------------
	\label{sub:Klammerregeln}

	Zur Klammerersparnis werden die üblichen Regeln verwendet, \textdh\ dass Operatoren mit höherer Priorität stärker binden, als solche mit niedrigerer Priorität, so dass redundante Klammern weggelassen werden können. Bei gleicher Priorität binden Klammern von innen nach außen. Für die Operatoren gilt Rechtsklammerung\footnote{Unäre Operatoren \textendash\ außer Klammern \textendash\ stehen hier stets links \emph{vor} dem Operanden, so dass es nur Rechtsklammerung geben kann. Zitat aus~\cite{bib:Rautenberg} Kapitel~1.1 Seite~5: \emph{Diese hat gegenüber Linksklammerung Vorteile bei der Niederschrift von Tautologien in $\limp$, [...]}}. Es gilt also mit abnehmender Priorität:

	{
		\renewcommand*{\clq}{}
		\renewcommand*{\crqt}{}
		\renewcommand*{\cse}{~}

		Klammern
		\begin{itemize}
			\item \clqt$(\dots)$\crqt
		\end{itemize}

		Unäre logische Operatoren
		\begin{itemize}
			\item \clqt$\lnot$\crqt
			%       \thepnot
		\end{itemize}

		Binäre logische Operatoren
		\begin{itemize}
			\item \clqt$\land$\cspt$\mult$\cspt$\lnand$\crqt
			%       \thepand, \thepmult,  \thepnand

			\item \clqt$\lor$\cspt$\add$\cspt$\lnor$\crqt
			%       \thepor, \thepadd,  \thepnor

			%\item \clqt$\lleft$\cspt$\lright$\cspt$\lnleft$\cspt$\lnright$\crqt
	        %       \thepleft,  \thepright,  \thepnleft,  \thepnright

			\item \clqt$\lleftimp$\cspt$\lrightimp$\cspt$\lnleftimp$\cspt$\lnrightimp$\crqt
			%       \thepleftimp,  \theprightimp,  \thepnleftimp,  \thepnrightimp

			\item \clqt$\lequiv$\cspt$\lnequiv$\crqt
			%       \thepequiv,  \thepnequiv
		\end{itemize}

		Nichtlogische Operatoren
		\begin{itemize}
			\item \clqt$=$\cspt$\defeq$\crqt
		\end{itemize}
	}

	Die Prioritäten der logischen Operatoren wurden aus~\cite{bib:Rautenberg} Kapitel~1.1 Seite~5 entnommen und ergänzt.

	\subsection{Formalisierung} %------------------------------------------------
	\label{sub:Formalisierung}

	Da Computerprogramme verwendet werden, müssen die Axiome, Sätze, Beweise, \textetc\ in streng formaler Form vorliegen. Die Formalisierung stützt sich auf~\cite{bib:Aussagenlogik} mit Auslassungen, Umsortierungen und Änderungen; \alsoname~\cite{bib:LogikDe, bib:LogikEn}.

	{
		\subsubsection{Bausteine der aussagenlogischen Sprache}
		\label{subsub:Bausteine}

		\newcommand*{\ItemB}[2]{\item[]\makebox[0.7cm][l]{#1}\makebox[4.0cm][l]{#2}}
		\newcommand*{\ItemF}[2]{\item[]\makebox[2.0cm][l]{#1}\makebox[4.5cm][l]{#2}}

		\newcommand*{\asN}{\mathbb{N}_0}
		\newcommand*{\asK}{\mathcal{K}}
		\newcommand*{\asV}{\mathcal{V}}
		\newcommand*{\asU}{\mathcal{U}}
		\newcommand*{\asB}{\mathcal{B}}
		\newcommand*{\asG}{\mathcal{G}}
		\newcommand*{\asS}{\mathcal{J}}
		\newcommand*{\asA}{\mathcal{A}}
		\newcommand*{\asF}{\mathcal{F}}
		\newcommand*{\asX}{\mathcal{X}}
		\newcommand*{\ase}{_\mathrm{e}}
		\newcommand*{\asBe}{\asB\ase}
		\newcommand*{\asSe}{\asS\ase}
		\newcommand*{\asAe}{\asA\ase}
		\newcommand*{\asFe}{\asF\ase}
		\newcommand*{\asXe}{\asX\ase}

		Es werden zur Erfassung der Symbole die folgenden Mengen\footnote{Hier wird die naive Mengenlehre vorausgesetzt.} definiert:
		\begin{itemize}

			\ItemB{$\asK$}{$\defeq\{\ltrue,\lfalse\}$} Menge der \emph{Konstanten}.

			\ItemB{$\asU$}{$\defeq\{\lnot\}$} Menge der \emph{unären Operatoren}.

			\ItemB{$\asB$}{$\defeq\{\land,\lor,\limp,\lequiv\}$} Menge der \emph{binären Operatoren}.

			\ItemB{$\asBe$}{$\defeq\asB\cup\{\mult,\add,\lnand,\lnor,\lleftimp\}$} \emph{Erweiterte} Menge der binären Operatoren.

			\ItemB{$\asG$}{$\defeq\{~'('~,~')'~\}$} Menge der \emph{Gliederungszeichen}.

		\end{itemize}

		Damit sind alle in der \tablename~\vref{tab:Symbole} verwendeten logischen Konstanten und Operatoren\footnote{Jeweils nur die ersten der grau hinterlegten Zeilen sowie \clq$\mult$\crq. Man beachte, dass \clq$=$\crq, $\ne$ und \clq$\defeq$\crq\ hier keine logischen Operatoren sind \textemdash\ \seename~\vref{sub:Klammerregeln}.} sowie die Klammern erfasst und es können die folgende Mengen definiert werden:
		\begin{itemize}

			\ItemB{$\asN$}{$\defeq$} Menge der \emph{natürlichen Zahlen} einschließlich $0$.\footnote{Für alle endlichen Formeln und endlichen Mengen davon \textendash\ und nur solche betrachten wir \textendash\ wird jeweils nur eine endliche Teilmenge aus $\asN$ gebraucht. Somit gibt es keine Schwierigkeiten mit unendlichen Mengen.}

			\ItemB{$\asV$}{$\defeq\{P_n|n\in \mathbb{N}_0\}$} Menge der \emph{atomaren Formeln} (Satzbuchstaben), kurz: \emph{Atome}.

			\ItemB{$\asS$}{$\defeq\asU\cup\asB\cup\asG$} Menge der \emph{Symbole}.\footnote{Junktoren und Gliederungszeichen}

			\ItemB{$\asSe$}{$\defeq\asU\cup\asBe\cup\asG\cup\asK$} \emph{Erweiterte} Menge der Symbole.\footnote{Konstanten, Junktoren und Gliederungszeichen}

			\ItemB{$\asA$}{$\defeq\asV\cup\asS$} \emph{Alphabet} der logischen Sprache.

			\ItemB{$\asAe$}{$\defeq\asV\cup\asSe$} \emph{Erweitertes} Alphabet der logischen Sprache.

		\end{itemize}
		Für Elemente von $\asV$ werden auch die großen lateinischen Buchstaben $A$, $B$, $C$, $\dots$ verwendet.

		\subsubsection{Formationsregeln}
		\label{subsub:Formeln}

		Es werden nun rekursiv noch zwei Mengen von Formeln definiert:
		\begin{itemize}

			\ItemF{$\asF$}{\clq$\defeq$\crqt}{Menge der \emph{aussagenlogischen Formeln}.}

			\item[]$\asV\subset\asF$

			\ItemF{$A\in\asF$ }{dann auch $(\circ A)\in\asF$}{für~$\circ\in\asU$}

			\ItemF{$A,B\in\asF$}{dann auch $(A\circ B)\in\asF$}{für~$\circ\in\asB$}

			\item[] Nur die auf diese Weise konstruierten Formeln sind Elemente von $\asF$, \textdh\ aussagenlogische Formeln.

		\end{itemize}
		und
		\begin{itemize}

			\ItemF{$\asFe$}{\clq$\defeq$\crqt}{Menge der \emph{erweiterten} aussagenlogischen Formeln.}

			\item[]$(\asV\cup\asK)\subset\asFe$

			\ItemF{$A\in\asFe$}{dann auch $(\circ A)\in\asFe$}{für~$\circ\in\asU$}

			\ItemF{$A,B\in\asFe$}{dann auch $(A\circ B)\in\asFe$}{für~$\circ\in\asBe$}

			\item[] Nur die auf diese Weise konstruierten Formeln sind Elemente von $\asFe$, \textdh\ erweiterte aussagenlogische Formeln.

		\end{itemize}
		Wie man leicht sieht, ist stets $\asX\subset\asXe$ für $\asX\in\{\asB,\asS,\asA,\asF\}$. Durch Anwendung der Klammerregeln von \subsubsectionname~\vref{subsub:Bausteine} lassen sich in der Regel noch die meisten Klammern einsparen. Die Namen der Operationen finden sich in der \tablename~\vref{tab:Symbole}. Für Elemente von $\asF$ und $\asFe$ werden auch die kleinen griechischen Buchstaben $\varphi$, $\psi$, $\dots$ verwendet.
	}

	\subsubsection{Aussagenlogische Axiome}
	\label{sec:Axiome}

	\textbf{> > > AUSSAGENLOGIK weiter bearbeiten. < < <} % AUSSAGENLOGIK weiter bearbeiten.

	\section{Prädikatenlogik} %==================================================
	\label{Prädikatenlogik}
	\newsection{Prädikatenlogik}

	\textbf{> > > PRÄDIKATENLOGIK bearbeiten. < < <} % PRÄDIKATENLOGIK bearbeiten.

	\section{Mengenlehre} %======================================================
	\label{Mengenlehre}
	\newsection{Mengenlehre}

	\textbf{> > > PRÄDIKATENLOGIK bearbeiten. < < <} % MENGENLEHRE bearbeiten.

	\section{Offene Aufgaben} %==================================================
	\label{Offene Aufgaben}
	\newsection{Offene Aufgaben}

	\begin{enumerate}
		\item TODOs bearbeiten
		\item Datenstruktur definieren
		\item Prüfung der Beweise definieren
		\item Axiome für das System bestimmen
		\item Eingabeprogramm erstellen (liest XML)
		\item Prüfprogramm erstellen
		\item Ausgabeprogramm erstellen (schreibt XML)
		\item Formelausgabe erstellen (erzeugt \LaTeX\ aus XML)
		\item Axiome sammeln und eingeben
		\item Sätze sammeln und eingeben
		\item Beweise sammeln und eingeben
		\item Fachbegriffe und Symbole sammeln und eingeben
		\item Fachgebiete sammeln und eingeben
		\item Ausgabeschemata sammeln und eingeben
	\end{enumerate}

	%%%% Verzeichnisse %%%%%%%%%%%%%%%%%%%%%%%%%%%%%%%%%%%%%%%%%%%%%%%%%%%%%%%%%%
	\clearpage

	%=== Tabellenverzeichnis ====================================================

	\ihead{\textnormal{\textsf{\textbf{\listtablename}}}}
	\begin{minipage}{\textwidth-10.95pt}
		\listoftables
		\addcontentsline{toc}{chapter}{\listtablename}
	\end{minipage}\par
	\thispagestyle{scrheadings}

	%=== Abbildungsverzeichnis ==================================================

	\ihead{\textnormal{\textsf{\textbf{\listfigurename}}}}
	\begin{minipage}{\textwidth-10.95pt}
		\listoffigures
		\addcontentsline{toc}{chapter}{\listfigurename}
		\addcontentsline{lof}{section}{<Noch keine \figurename en vorhanden.>}
	\end{minipage}\par
	\thispagestyle{scrheadings}

	%=== Literaturverzeichnis ===================================================

	\begin{flushleft}
		\begin{thebibliography}{12}
			\ihead{\textnormal{\textsf{\textbf{\bibname}}}}
			\thispagestyle{scrheadings}
			\addcontentsline{toc}{chapter}{\bibname}

			\bibitem{bib:Rautenberg}Wolfgang Rautenberg, \emph{Einführung in die Mathematische Logik}: Ein Lehrbuch, 3. Auflage, Vieweg+Teubner 2008

			\bibitem{bib:Apacheii}\emph{Apache License}, Version 2.0 $\rightarrow$~\url{http://www.apache.org/licenses/LICENSE-2.0} \textendash\ 02.01.2004 (09.03.2017)
			\footnote{Der Pfeil~($\rightarrow$) verweist stets auf einen Link ins Internet. Das Datum hinter dem Link \textendash\ sofern vorhanden \textendash\ gibt die letzte Änderung des Dokuments, der Lizenz oder der entsprechenden Seite an. Das kann vom Datum der Seite oder der Copyright-Angabe abweichen. Das geklammerte Datum gibt den Zeitpunkt an, als diese Seite im Rahmen der Erstellung dieses Dokuments zum letzten Mal angeschaut wurde. Dies gilt für alle hier aufgelisteten Literaturangaben.}

			\bibitem{bib:BSLi}\emph{Boost Software License} 1.0 $\rightarrow$~\url{http://www.boost.org/users/license.html} \textendash\ 17.08.2003 (09.03.2017)

			\bibitem{bib:EPL}\emph{Eclipse Public License} Version 1.0 $\rightarrow$~\url{http://www.eclipse.org/org/documents/epl-v10.php} \textendash\ (09.03.2017)

			\bibitem{bib:AGPL}\emph{GNU Affero General Public License} \textendash\ $\rightarrow$~\url{http://www.gnu.org/licenses/agpl} 19.11.2007 (09.02.2017)

			\bibitem{bib:GPLi}\emph{GNU General Public License} $\rightarrow$~\url{http://www.gnu.org/licenses/old-licenses/gpl-1.0} \textendash\ 02.1989 (09.03.2017)

			\bibitem{bib:GPLii}\emph{GNU General Public License}, Version 2 $\rightarrow$~\url{http://www.gnu.org/licenses/old-licenses/gpl-2.0} \textendash\ 06.1991 (09.03.2017)

			\bibitem{bib:LGPLii}\emph{GNU Lesser General Public License}, Version 2.1 $\rightarrow$~\url{http://www.gnu.org/licenses/old-licenses/lgpl-2.1} \textendash\ 02.1999 (09.03.2017)

			\bibitem{bib:Clover}Lizenz für \emph{Clover} $\rightarrow$~\url{https://www.atlassian.com/software/clover} \textendash\ 2017 (09.03.2017)

			\bibitem{bib:EULA}Lizenz für \emph{Microsoft Visual Studio Express 2015} $\rightarrow$~\url{https://www.visualstudio.com/de/license-terms/mt171551/} \textendash\ 2017 (09.03.2017)

			\bibitem{bib:MiKTeX}Lizenz für \emph{MikTeX} $\rightarrow$~\url{https://miktex.org/kb/copying} \textendash\ 14.01.2014 (09.03.2017)

			\bibitem{bib:SAX}Lizenz für \emph{SAX} $\rightarrow$~\url{http://www.saxproject.org/copying.html} \textendash\ 05.05.2000 (09.03.2017)

			\bibitem{bib:MIT}\emph{MIT License} $\rightarrow$~\url{https://opensource.org/licenses/MIT/} \textendash\ (09.03.2017)

			\bibitem{bib:JavaSE}\emph{Oracle Binary Code License Agreement} $\rightarrow$~\url{http://java.com/license} \textendash\ 02.04.2013 (09.03.2017)

			\bibitem{bib:OSI}\emph{OSI Certified Open Source Software} $\rightarrow$~\url{https://opensource.org/pressreleases/certified-open-source.php} \textendash\ 16.06.1999 (09.03.2017)

			\bibitem{bib:WDCDL}\emph{W3C Document License} $\rightarrow$~\url{http://www.w3.org/Consortium/Legal/2015/doc-license} \textendash\ 01.02.2015 (09.03.2017)

			\bibitem{bib:WDCSNL}\emph{W3C Software Notice and License} $\rightarrow$~\url{http://www.w3.org/Consortium/Legal/2002/copyright-software-20021231.html} \textendash\ 13.05.2015 (09.03.2017)

			\bibitem{bib:HilbertII}\emph{Hilbert II \textendash\ Introduction} $\rightarrow$~\url{http://www.qedeq.org/} \textendash\ 20.01.2014 (09.03.2017)

			\bibitem{bib:qedeq}\emph{Formal Correct Mathematical Knowledge}: GitHub Repository von Projekt Hilbert II $\rightarrow$~\url{https://github.com/m-31/qedeq/} \textendash\ 04.08.2016 (09.03.2017)

			\bibitem{bib:ASBA}\emph{ASBA \textendash\ Axiome, Sätze, Beweise und Auswertungen}. Projekt zur maschinellen Überprüfung von mathematischen Beweisen und deren Ausgabe in lesbarer Form: GitHub Repository von Projekt ASBA $\rightarrow$~\url{https://github.com/Dr-Winfried/ASBA} \textendash\ laufend (laufend)

			\bibitem{bib:LogikDe}Meyling, Michael: \emph{Anfangsgründe der mathematischen Logik} \textendash\ 24. Mai 2013 (in Bearbeitung) $\rightarrow$~\url{http://www.qedeq.org/current/doc/math/qedeq_logic_v1_de.pdf} \textendash\ (09.03.2017)

			\bibitem{bib:PrädikatenlogikDe}Meyling, Michael: \emph{Formale Prädikatenlogik} \textendash\ 24. Mai 2013 (in Bearbeitung) $\rightarrow$~\url{http://www.qedeq.org/current/doc/math/qedeq_formal_logic_v1_de.pdf} \textendash\ (09.03.2017)

			\bibitem{bib:MengenlehreDe}Meyling, Michael: \emph{Axiomatische Mengenlehre} \textendash\ 24. Mai 2013 (in Bearbeitung) $\rightarrow$~\url{http://www.qedeq.org/current/doc/math/qedeq_set_theory_v1_de.pdf} \textendash\ (09.03.2017)

			\bibitem{bib:LogikEn}Meyling, Michael: \emph{Elements of Mathematical Logic} \textendash\ May 24, 2013 (in Bearbeitung) $\rightarrow$~\url{http://www.qedeq.org/current/doc/math/qedeq_logic_v1_en.pdf} \textendash\ (09.03.2017)

			\bibitem{bib:PrädikatenlogikEn}Meyling, Michael: \emph{Formal Predicate Calculus} \textendash\ May 24, 2013 (in Bearbeitung) $\rightarrow$~\url{http://www.qedeq.org/current/doc/math/qedeq_formal_logic_v1_en.pdf} \textendash\ (09.03.2017)

			\bibitem{bib:MengenlehreEn}Meyling, Michael: \emph{Axiomatic Set Theory} \textendash\ May 24, 2013 (in Bearbeitung) $\rightarrow$~\url{http://www.qedeq.org/current/doc/math/qedeq_set_theory_v1_en.pdf} \textendash\ (09.03.2017)

			\bibitem{bib:Junktor}Wikipedia: \emph{Aussagenlogik} \chaptername~2.2 \emph{Mögliche Junktoren} \textendash\ 02.03.2017 $\rightarrow$~\url{https://de.wikipedia.org/wiki/Junktor#M.C3.B6gliche_Junktoren} \textendash\ 20.01.2016 (09.03.2017)

			\bibitem{bib:Aussagenlogik}Wikipedia: \emph{Aussagenlogik} \chaptername~4 \emph{Formaler Zugang} \textendash\ 24.02.2017 $\rightarrow$~\url{https://de.wikipedia.org/wiki/Aussagenlogik#Formaler_Zugang} \textendash\ 13.02.2017 (09.03.2017)

			\bibitem{bib:Prädikatenlogik}Wikipedia: \emph{Prädikatenlogik erster Stufe} \textendash\ 24.02.2017 $\rightarrow$~\url{https://de.wikipedia.org/wiki/Pr%C3%A4dikatenlogik_erster_Stufe} \textendash\ 17.07.2016 (09.03.2017)

			\bibitem{bib:Mengenlehre}Wikipedia: \emph{Mengenlehre} \textendash\ 24.02.2017 $\rightarrow$~\url{https://de.wikipedia.org/wiki/Mengenlehre} \textendash\ 03.03.2017 (09.03.2017)

		\end{thebibliography}
	\end{flushleft}
	\thispagestyle{scrheadings}

\end{document}

%%%% Ende des Dokuments %%%%%%%%%%%%%%%%%%%%%%%%%%%%%%%%%%%%%%%%%%%%%%%%%%%%%%%%%