%%############################################################################%%
%%                                                                            %%
%% Datei:  ASBA.tex                                                           %%
%% Inhalt: Erzeugung des Projektdokuments von ASBA.                           %%
%%                                                                            %%
%% Copyright (C) 2017  Winfried Teschers                                      %%
%%                                                                            %%
%% This program is free software: you can redistribute it and/or modify       %%
%% it under the terms of the GNU Affero General Public License as published   %%
%% by the Free Software Foundation, either version 3 of the License, or       %%
%% (at your option) any later version.                                        %%
%%                                                                            %%
%% This program is distributed in the hope that it will be useful,            %%
%% but WITHOUT ANY WARRANTY; without even the implied warranty of             %%
%% MERCHANTABILITY or FITNESS FOR A PARTICULAR PURPOSE.  See the              %%
%% GNU Affero General Public License for more details.                        %%
%%                                                                            %%
%% You should have received a copy of the GNU Affero General Public License   %%
%% along with this program.  If not, see <http://www.gnu.org/licenses/>.      %%
%%                                                                            %%
%% Dr. Winfried Teschers                                                      %%
%% Anton-Günther-Straße 26c                                                   %%
%% 91083 Baiersdorf                                                           %%
%% Germany                                                                    %%
%%                                                                            %%
%% e-mail: winfried.teschers@t-online.de                                      %%
%%                                                                            %%
%%############################################################################%%

% !TeX root = ASBA.tex
% !TeX encoding = UTF-8
% !TeX spellcheck = de_DE

%%############################################################################%%
%%                                                                            %%
%% Datei:  ASBA-Vorspann.tex                                                  %%
%% Inhalt: Vorspann für die Datei ASBA.txt                                    %%
%%                                                                            %%
%% Copyright (C) 2017  Winfried Teschers                                      %%
%%                                                                            %%
%% This program is free software: you can redistribute it and/or modify       %%
%% it under the terms of the GNU Affero General Public License as published   %%
%% by the Free Software Foundation, either version 3 of the License, or       %%
%% (at your option) any later version.                                        %%
%%                                                                            %%
%% This program is distributed in the hope that it will be useful,            %%
%% but WITHOUT ANY WARRANTY; without even the implied warranty of             %%
%% MERCHANTABILITY or FITNESS FOR A PARTICULAR PURPOSE.  See the              %%
%% GNU Affero General Public License for more details.                        %%
%%                                                                            %%
%% You should have received a copy of the GNU Affero General Public License   %%
%% along with this program.  If not, see <http://www.gnu.org/licenses/>.      %%
%%                                                                            %%
%% Dr. Winfried Teschers                                                      %%
%% Anton-Günther-Straße 26c                                                   %%
%% 91083 Baiersdorf                                                           %%
%% Germany                                                                    %%
%%                                                                            %%
%% e-mail: winfried.teschers@t-online.de                                      %%
%%                                                                            %%
%%############################################################################%%

% !TeX root = ASBA.tex
% !TeX encoding = UTF-8
% !TeX spellcheck = de_DE

\documentclass[english, ngerman, parskip=half, headsepline, footsepline, fleqn, notitlepage]{scrreprt}

% Pakete #######################################################################

% allgemein --------------------------------------------------------------------
\usepackage[utf8]{inputenc}% Input encoding specification
\usepackage[T1]{fontenc}
\usepackage{lmodern}
\usepackage{scrlayer-scrpage}
\usepackage{geometry}% Flexible and complete interface to document dimensions.
\usepackage{microtype}% Subliminal refinements towards typographical perfection.
\usepackage{graphicx}% Alternative interface to graphics functions.
\usepackage{pict2e}% New implementation of picture commands.
\usepackage{multicol}% An environment for multicolumn output
\usepackage{babel}% Multilingual support for plain TeX or LaTeX.
\usepackage[autostyle]{csquotes}% Contex sensitive quotation facilities

% mathematische Pakete ---------------------------------------------------------
\usepackage{amsmath}% Mathematical facilities for LaLeX from ASM
\usepackage{amsfonts}% TeX fonts from the American Mathematical Society.
\usepackage{amssymb}% Symbols from the American Mathematical Society.
\usepackage{mathtools}% Mathematical tools to use with asmmath.
\usepackage{mathabx}% Three series of mathematical symbols.
\usepackage{mathpazo}% Fonts to typeset mathematics to match palatino.
%\usepackage{cancel}% Place lines through mathematical formulae.

% Tabellen ---------------------------------------------------------------------
\usepackage[table]{xcolor}% Driver-independent color extensions - vor 'color'?
%\usepackage{ctable}% Flexible typesetting of table and figure using key/value
%% Das Paket ctable fast die Eigenschaften der Pakete
\usepackage{array}%
\usepackage{tabularx}% Erweiterung von tabular*
\usepackage{booktabs}% Nicer layout of tables
%% zusammen und lädt zusätzlich noch die Pakete
\usepackage{rotating}% Rotating tools, including rotated full page floats
\usepackage{xspace}% Behandelt Zwischenraum nach Makros
\usepackage{color}% LaTeX support for color
\usepackage{xkeyval}% Extension of the keyval package
% Ende der von 'ctable' geladenen Pakete
\usepackage{threeparttable}% Tables with captions and notes all the same width.
\usepackage{multirow}% Create tabular cell spanning multiple rows.
\usepackage{diagbox}% Table heads with diagonal lines.
\usepackage{arydshln}% Draw dash-lines in array/tabular.
\usepackage{caption}% Customizing captions in floating environments.

% Indizes ----------------------------------------------------------------------
%\usepackage{makeidx}% Indexing - Entweder 'makeidx' oder 'splitidx'
\usepackage[protected]{splitidx}% mehrere Indizes - statt 'makeidx'
%\usepackage{hvindex}% Support for indexing - after 'babel
%\usepackage{showidx}% Index auf Seitenrand anzeigen - Zum Testen der Indizes
%TODO Fehler: 'showindex' gibt direkt und nicht auf Rand aus
\usepackage{glossaries}% Create glossaries and lists of acronyms
%TODO Fehler: 'hyperfirst' hat keine Wirkung'.
%\usepackage{glossaries-german}% German language module for glossaries package

% Verweise ---------------------------------------------------------------------
%\usepackage[germanb]{minitoc}\dosectoc% Unterverzeichnisse erstellen
%TODO Fehler: 'minitoc' funktioniert nicht
\usepackage{varioref}
\usepackage[colorlinks,linktoc=all]{hyperref}% Extensive support for hypertext
\usepackage{glossaries}% Create glossaries and lists of acronyms
% lädt     {glossaries-german}% German language module for glossaries package

% Einstellung von globalen Werten und Makro-Redefinitionen #####################

\geometry{textwidth=170mm,textheight=256mm,twoside}% optional Option 'showframe'

% Kopfzeilen ===================================================================
\newcommand*{\texthead}[1]{\textnormal{\textsf{\textbf{#1}}}}% Schriftart
\newcommand*{\Lehead}[1]{\lehead{\texthead{#1}}}
\newcommand*{\Cehead}[1]{\cehead{\texthead{#1}}}
\newcommand*{\Rehead}[1]{\rehead{\texthead{#1}}}
\newcommand*{\Lohead}[1]{\lohead{\texthead{#1}}}
\newcommand*{\Cohead}[1]{\cohead{\texthead{#1}}}
\newcommand*{\Rohead}[1]{\rohead{\texthead{#1}}}
\newcommand*{\Ohead}[1]{\ohead{\texthead{#1}}}
\newcommand*{\Chead}[1]{\chead{\texthead{#1}}}
\newcommand*{\Ihead}[1]{\ihead{\texthead{#1}}}
\newcommand*{\Ofoot}[1]{\ofoot{\textnormal{\textbf{#1}}}}
\newcommand*{\Cfoot}[1]{\cfoot{\textnormal{#1}}}
\newcommand*{\Ifoot}[1]{\ifoot{\textnormal{#1}}}
\newcommand*{\Pagestyle}{\pagestyle{scrheadings}}
\newcommand*{\Thispagestyle}{\thispagestyle{scrheadings}}

% Kopfzeilen mit 'scrlayer-scrpage'
%         \Lehead \Cehead \Rehead | \Lohead \Cohead \Rohead
% \Ohead: \Lehead                                   \Rohead
% \Chead:         \Cehead                   \Cohead
% \Ihead:                 \Rehead   \Lohead
% ASBA <Chapter-Überschrift> \Chaptername~\thechapter
%                            \sectionname~\thesection <Section-Überschrift> ASBA
%Initialisierung
\Ohead{ASBA}%               bleibt unverändert
\Chead{Inhaltsverzeichnis}% wird laufend verändert
\Ihead{}%                   wird laufend verändert

% Kapitel ======================================================================
\newcommand*{\Chaptername}{\chaptername}% wird mit 'Anhang' überschrieben

\newcommand*{\beforechapter}{% direkt vor \chapter (auch im Kommentar)
	\Thispagestyle%        Kopfzeile für diese Seite aktivieren - vor \clearpage
	\clearpage%            neue Seite
}
\newcommand*{\beginchapter}[1]{%        direkt nach \chapter
	\Chead{#1}%                         Kopfzeile Mitte = <Kapitelname>
	\Ihead{\Chaptername~\thechapter}%   Kopfzeile Innen = Kapitel/Anhang <Nr.>
	\Pagestyle%                         veränderte Kopfzeile aktivieren
	\Thispagestyle%                     ... auch für diese Seite, da ...
}%                                      '\chapter' Kopf-/Fußzeilen deaktiviert
\newcommand*{\likechapter}[2][chapter]{%statt \beginchapter für Inhalts-,
	%                                   Tabellen- und Abbildungsverzeichnis
	\Chead{#2}%                         Mitte in der Kopfzeile = <Kapitelname>
	\Ihead{}%                           Innen in der Kopfzeile = <leer>
	\Pagestyle%                         veränderte Kopfzeile aktivieren ...
	\Thispagestyle%                     sicherheitshalber auch für diese Seite.
	\addcontentsline{toc}{#1}{#2}%      Eintrag ins Inhaltsverzeichnis
	%TODO Fehler: 2. Seite Kopf Mitte für Literaturverzeichnis = Abbildungsverzeichnis
}
\newcommand*{\Endchapter}{% am Ende eines Kapitels
	\Thispagestyle% sicherheitshalber Kopfzeile für diese Seite aktivieren
}
% Indizes ======================================================================
\newcommand*{\idxdictionary}[1]{%       nur für Indices
	\Thispagestyle%       Kopfzeile für diese Seite aktivieren - vor \clearpage
	\clearpage%                         neue Seite
	\extendtheindex{}{%                 aktiviert die Kopfzeile für Index-Seiten
		\Chead{#1}%                     Mitte in der Kopfzeile = <Kapitelname>
		\Ihead{}%                       Innen in der Kopfzeile = <leer>
		\Pagestyle%                     veränderte Kopfzeile aktivieren
		\Thispagestyle%                 ... auch für diese Seite
		\addcontentsline{toc}{section}{#1}% Eintrag ins Inhaltsverzeichnis
	}{}{}
}
\newcommand*{\glodictionary}[1]{%       nur für Glossary
	\Chead{#1}%                         Mitte in der Kopfzeile = <Kapitelname>
	\Ihead{}%                           Innen in der Kopfzeile = <leer>
	\Pagestyle%                         veränderte Kopfzeile aktivieren
}

% Abschnitte ===================================================================
\newcommand*{\beginsection}[1]{%        direkt nach \section
	\Cohead{#1}%                        oben rechts mittig = <Abschnittsname>
	\Lohead{\sectionname~\thesection}%  oben rechts innen = <Abschnittsnummer>
	\Pagestyle%                         veränderte Kopfzeile aktivieren
}

% Fußzeilen ====================================================================
\Ofoot{\thepage}
\Cfoot{Winfried Teschers}
\Ifoot{\today}
\Pagestyle%                                       aktiviert Kopf- und Fußzeilen

% Fußnoten ---------------------------------------------------------------------
\deffootnote[10pt]% Markenbreite
{10pt}% Einzug - für Blocksatz: Markenbreite
{0pt}% Absatzeinzug für Folgeabsätze
{\makebox[9pt][r]{\textsuperscript{\thefootnotemark)} }}% Zeichen; < Markenbreite
\deffootnotemark {\textsuperscript{\thefootnotemark)}}

% Vordefinierte Werte ändern ===================================================
\setcounter{tocdepth}{3}%    Tiefe des Inhaltsverzeichnisses: 2 => subsection
\setcounter{secnumdepth}{3}% Nummerierung:                    3 => subsubsection
\setlength\extrarowheight{1pt}% Tabellenzellenhöhe vergrößern
\captionsetup{labelfont=bf}%    Tabellenbeschriftung in bf = bold font

% Empfehlung aus: Herbert Voß, LaTeX Referenz, 3. Auflage, Berlin 2014; S. 37f
\renewcommand{\floatpagefraction}{0.7}% Empfehlung: 0.5-0.8 Voreinstellung: 0.9
\renewcommand{\textfraction}{0.15}%                 0.1-0.3                 0.05
\renewcommand{\topfraction}{0.8}%                   0.5-0.85                0.9
\renewcommand{\bottomfraction}{0.5}%                0.2-0.5                 0.9
\setcounter{topnumber}{3}%                                                  2
\setcounter{totalnumber}{15}%                                               3

% Neue Elemente ----------------------------------------------------------------
\newcounter{Enumi}% für unterbrochene Listennummerierung

% Bildelemente #################################################################

\newcommand*{\textbild}[1]{\textbf{\textsf{#1}}}% Textauszeichnungen für Text im Bild
\newcommand*{\Datei}[4][0.5]{% #2 x #3 = (-#2/2,-#3/2),(#2/2,#3/2)
	% [Eck-Radius], Breite, Höhe, Name
	\put(0,0){\oval[#1](#2,#3)}
	\put(0,0){\makebox(0,0){\textbild{#4}}}
}
\newcommand*{\Datenbank}[5]{% 2(#1) x 2(#2+#3) = (-#1,-#2-#3),(+#1,+#2+#3)
	% Halbmesser x, Halbmesser y, halbe Höhe, Name - Ursprung in der Mitte
	\put(0.0,-#3){
		\qbezier(-#1,0.0)(-#1,-#2)(0.0,-#2)
		\qbezier(+#1,0.0)(+#1,-#2)(0.0,-#2)
	}
	\put(0,0){\Line(-#1,-#3)(-#1, #3)}
	\put(0,0){\Line( #1,-#3)( #1, #3)}
	\put(0.0,#3){
		\qbezier(-#1,0.0)(-#1,-#2)(0.0,-#2)
		\qbezier( #1,0.0)( #1,-#2)(0.0,-#2)
		\qbezier(-#1,0.0)(-#1, #2)(0.0, #2)
		\qbezier( #1,0.0)( #1, #2)(0.0, #2)
	}
	\makebox(0,0){\textbild{#4}}
	\makebox(0,-#3){\textbild{#5}}
}
% '\Männchen' (mit 'ä') führt zu Fehler
\newcommand*{\Maennchen}{% 1x2 = (-0.5,-1.7),(+0.5,+0.3) Ursprung im Kopf
	\put(0,0){\circle{0.6}}
	\Line(0.0,-0.3)(0.0,-1.2)
	\polyline(-0.5,-0.3)(0.0,-0.6)(0.5,-0.3)
	\polyline(-0.5,-1.7)(0.0,-1.2)(0.5,-1.7)
}
\newcommand*{\Marker}[2][0.5]{% 2x#1 x 2x#1 - Kreis mit Text
	{
		\linethickness{0.5pt}
		\color{white}
		\put(0,0){\circle*{#1}}
		\color{black}
		\put(0,0){\circle{#1}}
		\put(0,0){\makebox(0,0){\small\textbild{#2}}}
	}
}
\newcommand*{\marker}[2][0.5]{% 2x#1 x 2x#1 - Kreis mit Text - grau
	{
		\linethickness{0.5pt}
		\color{white}
		\put(0,0){\circle*{#1}}
		\color{gray}
		\put(0,0){\circle{#1}}
		\put(0,0){\makebox(0,0){\small\textbild{#2}}}
	}
}
\newcommand*{\Papier}[3]{% 3x#1+#2 = (0.0,-#2),(2.1,#1) Ursprung links unten
	% Länge (Höhe), Länge Abschluss, Name
	\polyline(0.0,-0.01)(+0.0,+#1)(+2.8,+#1)(+2.8,-0.01)
	\qbezier(+2.1,+#2)(+2.6,+#2)(+2.8,0.0)
	\qbezier(+2.1,+#2)(+1.6,+#2)(+1.4,0.0)
	\qbezier(+0.7,-#2)(+1.2,-#2)(+1.4,0.0)
	\qbezier(+0.7,-#2)(+0.2,-#2)(+0.0,0.0)
	\put(0,0){\makebox(+2.8,+#1){\textbild{#3}}}
}
\newcommand*{\Terminal}[1]{% 2x2 =(-1.0,-1.4),(+1.0,+0.6)Ursprung im Monitor
	% Bildschirm
	%		\put(0,0){\polygon(-1.0,-0.6)(+1.0,-0.6)(+1.0,+0.6)(-1.0,+0.6)}
	\put(-1.0,-0.6){\framebox(2,1.2){#1}}
	\put(0,0){\oval[0.1](1.65,0.85)}
	% Hals
	\put(0,0){\Line(-0.2,-0.6)(-0.2,-1.0)}
	\put(0,0){\Line(+0.2,-0.6)(+0.2,-1.0)}
	% Tastatur
	\multiput(-1.0,-1.0)(+0.0,-0.133){4}{\line(1,0){2.0}}
	\multiput(-1.0,-1,0)(+0.2,+0.0){11}{\line(0,-1){0.4}}
}
\newcommand*{\Wolke}[1]{% 3.0x1.5 = (-1.5,-1.0),(+1.5,+0.5)
	% unterer Bogen
	\qbezier(-1.5,+0.0)(-1.5,-1.0)(-0.0,-1.0)
	\qbezier(+1.5,+0.0)(+1.5,-1.0)(+0.0,-1.0)
	% oberer Bogen rechts
	\qbezier(+1.5,+0.0)(+1.5,+0.5)(+0.8,+0.5)
	\qbezier(+0.4,+0.4)(+0.4,+0.5)(+0.8,+0.5)
	% oberer Bogen Mitte
	\qbezier(+0.5,+0.2)(+0.5,+0.5)(+0.0,+0.5)
	\qbezier(-0.4,+0.4)(-0.4,+0.5)(-0.0,+0.5)
	% oberer Bogen links
	\qbezier(-0.3,+0.2)(-0.3,+0.5)(-0.8,+0.5)
	\qbezier(-1.5,+0.0)(-1.5,+0.5)(-0.8,+0.5)
	\put(-1.5,-1.0){\makebox(3.0,1.5){\textbild{#1}}}
}

% Metasprachliche Symbole ######################################################

\newcommand*{\metaund}{\&\&}%               Und-Symbol  für Texte
\newcommand*{\metaoder}{||}%                Oder-Symbol für Texte
\newcommand*{\metaand}{\;\metaund\;}%       Und-Symbol  für Formeln
\newcommand*{\metaor}{\;\metaoder\;}%       Oder-Symbol für Formeln
% Nur im Mathematikmodus!
\newcommand*{\metaimp}{\Rightarrow}%        aus ... folgt ...
\newcommand*{\metarep}{\Leftarrow}%         ... folgt aus ...
\newcommand*{\metaequiv}{\Leftrightarrow}%  ... genau dann wenn ...
\newcommand*{\metadefeq}{:\Leftrightarrow}% ... definitionsgemäß genau dann wenn
\newcommand*{\defeq}{:=}%                   ... definitionsgemäß gleich ...
\newcommand*{\eq}{=}%                       ... gleich ...

% Mathematische Symbole ########################################################

% Beispieloperatoren ===========================================================
% \*bsp
\newcommand*{\opbsp}{\circ}
\newcommand*{\relbsp}{\sim}
\newcommand*{\releqbsp}{\simeq}
\newcommand*{\lrelbsp}{\lhd}
\newcommand*{\rrelbsp}{\rhd}
\newcommand*{\lreleqbsp}{\unlhd}
\newcommand*{\rreleqbsp}{\unrhd}

% Definitionen für die Tabelle der Junktoren ===================================
% \l*  -           logischer Operator
% \ln* - negierter logischer Operator
% Logische Operatoren als Addition und Multiplikation
\newcommand*{\ladd}{+}
\newcommand*{\lmult}{\cdot}
% Wahrheitswerte ---------------------------------------------------------------
\newcommand*{\texttrue}{W}%  in einem Kommentar stets 'W'
\newcommand*{\textfalse}{F}% in einem Kommentar stets 'F'
% Konstante --------------------------------------------------------------------
\newcommand*{\ltrue}{\top}%      W - wahr
%\newcommand*{\lnfalse}{\notbot}% " - nicht falsch
\newcommand*{\lfalse}{\bot}%     F - falsch
%\newcommand*{\lntrue}{\nottop}%  " - nicht wahr
% unäre Operatoren -------------------------------------------------------------
%                                                 W F - Aussage A
%\newcommand*{\lutrue}{\operatorname{\top}}%      W W - wahr [unär]
%\newcommand*{\lnufalse}{\operatorname{\notbot}}% " " - nicht falsch [unär]
%                                                 W F - A
%             \lnot                               F W - nicht
%\newcommand*{\lufalse}{\operatorname{\bot}}%     F F - falsch [unär]
%\newcommand*{\lnutrue}{\operatorname{\nottop}}%  " " - nicht wahr [unär]
% binäre Operatoren ------------------------------------------------------------
%                                                    W W F F - Aussage A
%                                                    W F W F - Aussage B
%  - - - - - - - - - - - - - - - - - - - - - - - - - - - - - - - - - - - - - - -
%\newcommand*{\lbtrue}{\operatorname{\top}}%         W W W W - wahr [binär]
%\newcommand*{\lnbfalse}{\operatorname{\notbot}}%    " " " " - nicht falsch
%            \lor                                    W W W F - A oder B
\newcommand*{\lrep}{\leftarrow}%                     W W F W - A folgt aus B
\newcommand*{\lrepA}{\Leftarrow}%
\newcommand*{\lrepB}{\subset}%
\newcommand*{\lleft}{\operatorname{\rfloor}}%        W W F F - A
%  - - - - - - - - - - - - - - - - - - - - - - - - - - - - - - - - - - - - - - -
\newcommand*{\limp}{\rightarrow}%                    W F W W - aus A folgt B
\newcommand*{\limpA}{\Rightarrow}%
\newcommand*{\limpB}{\supset}%
\newcommand*{\lright}{\operatorname{\lfloor}}%       W F W F - B
\newcommand*{\lequiv}{\leftrightarrow}%              W F F W - A genau dann,
\newcommand*{\lequivA}{\Leftrightarrow}%                       wenn B
%            \lnxor                                  " " " " - nicht
%                                                         (entweder A oder B)
%            \land                                   W F F F - A und B
\newcommand*{\landA}{\&}
\newcommand*{\landB}{\lmult}
%  - - - - - - - - - - - - - - - - - - - - - - - - - - - - - - - - - - - - -
\newcommand*{\lnand}{\uparrow}%                      F W W W - nicht
\newcommand*{\lnandA}{\barwedge}%                              (A und B)
\newcommand*{\lnandB}{\mid}%
\newcommand*{\lxor}{\ladd}%                          F W W F - entweder A
\newcommand*{\lxorA}{\operatorname{\dot\lor}}%                 oder B
\newcommand*{\lxorB}{\veebar}%
\newcommand*{\lxorC}{\oplus}%
\newcommand*{\lnequiv}{\nleftrightarrow}%            " " " " - nicht
\newcommand*{\lnequivA}{\nLeftrightarrow}%            (A genau dann, wenn B)
\newcommand*{\lnequivB}{\notequiv}%
\newcommand*{\lnright}{\lceil}%                      F W F W - nicht B
\newcommand*{\lnimp}{\nrightarrow}%                  F W F F - nicht
\newcommand*{\lnimpA}{\nRightarrow}%                         (aus A folgt B)
\newcommand*{\lnimpB}{\nsupset}%
%  - - - - - - - - - - - - - - - - - - - - - - - - - - - - - - - - - - - - -
\newcommand*{\lnleft}{\rceil}%                       F F W W - nicht A
\newcommand*{\lnrep}{\nleftarrow}%                   F F W F - nicht
\newcommand*{\lnrepA}{\nLeftarrow}%                          (A folgt aus B)
\newcommand*{\lnrepB}{\nsubset}%
\newcommand*{\lnor}{\downarrow}%                     F F F W - nicht
\newcommand*{\lnorA}{\operatorname{\overline\vee}}%            (A oder B)
%\newcommand*{\lbfalse}{\operatorname{\bot}}%        F F F F - falsch[binär]
%\newcommand*{\lnbtrue}{\operatorname{\nottop}}%     " " " " - nicht wahr

% Verwendete Mengenbezeichnungen ===============================================

% \gs* = globales Symbol
\newcommand*{\gsN}{\mathbb{N}}%    Menge der natürlichen Zahlen ohne 0
\newcommand*{\gsNo}{\mathbb{N}_0}% Menge der natürlichen Zahlen einschließlich 0

% \as* = aussagenlogisches Symbol
\newcommand*{\asA}{\mathcal{A}}%       Alphabet der Sprache
\newcommand*{\asAx}{\mathcal{A}_x}%    ... davon eine Teilmenge bzgl. \asJx
\newcommand*{\asAy}{\mathcal{A}_y}%    entsprechend \asAx
\newcommand*{\asB}{\mathcal{B}}%       Menge der binären Operatoren
\newcommand*{\asC}{\mathcal{C}}%       Menge der Konstanten
\newcommand*{\asF}{\mathcal{F}}%       Menge der Formeln
\newcommand*{\asFp}{\mathcal{F}^p}%    ... in polnischer Notation
\newcommand*{\asFx}{\mathcal{F}_x}%    Teilenge der Formeln
\newcommand*{\asFxp}{\mathcal{F}_x^p}% ... bzgl. \asJx in polnischer Notation
\newcommand*{\asFy}{\mathcal{F}_y}%    entsprechend \asFx
\newcommand*{\asFyp}{\mathcal{F}_y^p}% entsprechend \asFxp
\newcommand*{\asJ}{\mathcal{J}}%       Menge der Junktoren
\newcommand*{\asJx}{\mathcal{J}_x}%    Teilmenge der Junktoren
\newcommand*{\asJy}{\mathcal{J}_y}%    entsprechend \asJx
\newcommand*{\asM}{\mathcal{M}}%       Metaoperatoren und "Gleichheiten"
\newcommand*{\asU}{\mathcal{U}}%       Menge der unären Operatoren
\newcommand*{\asV}{\mathcal{V}}%       Menge der atomaren Formeln
\newcommand*{\asX}{\mathcal{X}}%       Mengenvariable
% verschiedene Indizes für Teilmengen von \asA, \asF, \asFp und \asJ
\newcommand*{\xAnd}{\mathrm{and}}%
\newcommand*{\xBool}{\mathrm{bool}}%
\newcommand*{\xImp}{\mathrm{imp}}%
\newcommand*{\xNand}{\mathrm{nand}}%
\newcommand*{\xNor}{\mathrm{nor}}%
\newcommand*{\xOr}{\mathrm{or}}%
\newcommand*{\xRep}{\mathrm{rep}}%

% sonstige mathematische Zeichen ===============================================

\newcommand*{\abltb}{\vdash}%           ... ableitbar ...
\newcommand*{\subst}{\curvearrowright}% ... substituiert durch ...

% sonstige Kommandos für den Mathematiksatz ####################################

\mathtoolsset{showonlyrefs,showmanualtags}% Nur mit \ref referenzierte Gleichungen, aber alle manuellen Tags

% sonstige nützliche Kommandos #################################################

% Im Parameter von '\turl' muss '\' vor jedem Zeichen aus '{}#&%$' ein '\'
% stehen und '\' / '~' durch '\textbackslash' / '\textasctilde' ersetzt werden.
\newcommand*{\tourl}[1]{$\rightarrow$~\url{#1}}
\newcommand*{\formulatoleft}{&&&&&&&&&&}%  Um Formeln nach links zu komprimieren
\newcommand*{\formulaspace}{&&&&}%         Für Platz zwischen den Formeln
\newcommand*{\todo}[1]{\textbf{>~>~>~#1~<~<~<}}% für TODOs
\newcommand*{\charqt}[1]{'#1'}%          Quotierung von Zeichen    (character) %\newcommand*{\charqt}[1]{\guilsinglleft#1\guilsinglright}%  ... Alternative
\newcommand*{\strqt}[1]{\enquote{#1}}%   Quotierung von Zeichenketten (string)
%\newcommand*{\strqt}[1]{\guillemotleft#1\guillemotright}%     ... Alternative
% Nur mit Parameter im Mathematikmodus!
\newcommand*{\symqt}[1]{\charqt{#1}}%    Quotierung einzelner Symbole (symbol)
\newcommand*{\forqt}[1]{\strqt{#1}}%     Quotierung von Formeln      (formula)

% Strukturbezeichnungen ergänzen
\newcommand*{\sectionname}{Abschnitt}
\newcommand*{\subsectionname}{Unterabschnitt}
\newcommand*{\subsubsectionname}{Paragraph}

% Abkürzungen mit Punkten; zur Unterscheidung vom Satzende
\newcommand*{\textbzgl}{bzgl.\@ }
\newcommand*{\textbzw}{bzw.\@ }
\newcommand*{\textdh}{d.\@\,h.\@ }
\newcommand*{\textggf}{ggf.\@ }
\newcommand*{\textiAlg}{i.\@\,Alg.\@ }
\newcommand*{\textua}{u.\@\,a.\@ }
\newcommand*{\textusw}{usw.\@ }
\newcommand*{\textzB}{z.\@\,B.\@ }
\newcommand*{\textZB}{Z.\@\,B.\@ }

% Ergebnis von Makros verschwinden lassen
\newcommand*{\hidden}[1]{}

% Glossareinträge ##############################################################

% Indices und Symbole ==========================================================
\makeindex
\newindex[Symbolverzeichnis]{sym}
\newindex[Index]{idx}
\newcommand*{\Idx}[1]{#1\idx{#1}}%   normaler Index
\newcommand*{\Sym}[1]{#1\sym{$#1$}}% Symbol - Nur im Mathematikmodus verwenden!

% Glossareinträge ==============================================================
\GlsSetQuote{+}% wegen Gebrauch von ngerman; see glossaries guide for beginners
\makeglossaries
\setacronymstyle{long-sc-short}

% Symbol/Index und Glossareintrag ==============================================

\newcommand*{\idx}[1]{\sindex[idx]{#1}}
\newcommand*{\sym}[1]{\sindex[sym]{#1}}
\newcommand*{\glsSym}[1]{\glspl{#1}\sym{\gls{#1}}}% Symbol - Im Mathematikmodus!
\newcommand*{\glsIdx}[1]{\gls{#1}\idx{\gls{#1}}}%        normal
\newcommand*{\GlsIdx}[1]{\Gls{#1}\idx{\gls{#1}}}%        groß
\newcommand*{\glsIdxBg}[2]{#2\idx{\gls{#1}}}%                     Beugung
\newcommand*{\GlsIdxBg}[2]{#2\idx{\gls{#1}}}%            groß und Beugung
\newcommand*{\glsIdxPl}[1]{\glspl{#1}\idx{\gls{#1}}}%             Plural
\newcommand*{\GlsIdxPl}[1]{\Glspl{#1}\idx{\gls{#1}}}%    groß und Plural
\newcommand*{\glsIdxX}[1]{\glsIdxPl{#1}}% Sonderfall
\newcommand*{\GlsIdxX}[1]{\GlsIdxPl{#1}}% Sonderfall und groß
\newcommand*{\Tag}[1]{\tag{\glsPl{#1}}\sym{\gls{#1}}}% Glossary, Symbol, Tag in Formel

% ==============================================================================

% Symbole für Beispieloperatoren -----------------------------------------------

\newglossaryentry{lrelbsp}{
	name={$ \lrelbsp$},
	plural={\lrelbsp},% im Mathematikmodus
	description={%
		Ein Beispielsymbol für eine Relation mit Umkehrrelation $\rrelbsp$%
	}
}
\newglossaryentry{lreleqbsp}{
	name={$ \lreleqbsp$},
	plural={\lreleqbsp},% im Mathematikmodus
	description={%
		Ein Beispielsymbol für eine Relation mit Gleichheit und Umkehrrelation $\rreleqbsp$%
	}
}
\newglossaryentry{relbsp}{
	name={$ \relbsp$},
	plural={\relbsp},% im Mathematikmodus
	description={%
		Ein Beispielsymbol für eine Relation%
	}
}
\newglossaryentry{releqbsp}{
	name={$ \releqbsp$},
	plural={\releqbsp},% im Mathematikmodus
	description={%
		Ein Beispielsymbol für eine Relation mit Gleichheit%
	}
}
\newglossaryentry{rrelbsp}{
	name={$ \rrelbsp$},
	plural={\rrelbsp},% im Mathematikmodus
	description={%
		Ein Beispielsymbol für eine Relation mit Umkehrrelation $\lrelbsp$%
	}
}
\newglossaryentry{rreleqbsp}{
	name={$ \rreleqbsp$},
	plural={\rreleqbsp},% im Mathematikmodus
	description={%
		Ein Beispielsymbol für eine Relation mit Gleichheit und Umkehrrelation $\lreleqbsp$%
	}
}

% Symbole für Metaoperatoren ---------------------------------------------------

\newglossaryentry{defeq}{
	name={$ \defeq$},
	plural={\defeq},% im Mathematikmodus
	description={%
		Ein \glsIdx{MetaoperatorV}: ... definitionsgemäß gleich ..%
	}
}
\newglossaryentry{eq}{
	name={$ \eq$},
	plural={\eq},% im Mathematikmodus
	description={%
		Ein \glsIdx{MetaoperatorV}: ... gleich (ist dasselbe wie, ist identisch zu) ..%
	}
}
\newglossaryentry{equiv}{
	name={$ \equiv$},
	plural={\equiv},% im Mathematikmodus
	description={%
		Ein (Meta-)Operator: ... äquivalent (ist das gleiche wie, ist so wie) zu ..%
	}
}
\newglossaryentry{metaand}{
	name={$ \metaund$},
	plural={\metaand},% im Mathematikmodus
	description={%
		Ein \glsIdx{MetaoperatorV}: ... oder ... (\seealso~\glsSym{mid})%
	}
}
\newglossaryentry{metadefeq}{
	name={$ \metadefeq$},
	plural={\metadefeq},% im Mathematikmodus
	description={%
		Ein \glsIdx{MetaoperatorV}: ... definitionsgemäß gleich (definitionsgemäß genau dann, wenn) ..%
	}
}
\newglossaryentry{metaequiv}{
	name={$ \metaequiv$},
	plural={\metaequiv},% im Mathematikmodus
	description={%
		Ein \glsIdx{MetaoperatorV}: ... genau dann wenn ..%
	}
}
\newglossaryentry{metaimp}{
	name={$ \metaimp$},
	plural={\metaimp},% im Mathematikmodus
	description={%
		Ein \glsIdx{MetaoperatorV}: ... dann auch ..%
	}
}
\newglossaryentry{metaor}{
	name={$ \metaoder$},
	plural={\metaor},% im Mathematikmodus
	description={%
		Ein \glsIdx{MetaoperatorV}: ... oder ..%
	}
}
\newglossaryentry{metarep}{
	name={$ \metarep$},
	plural={\metarep},% im Mathematikmodus
	description={%
		Ein \glsIdx{MetaoperatorV}: ... sofern ..%
	}
}
\newglossaryentry{mid}{
	name={$ \mid$},
	plural={\mid},% im Mathematikmodus
	description={%
		Ein \glsIdx{MetaoperatorV}: ... und ... (\seealso~\glsSym{metaand})%
	}
}
\newglossaryentry{ne}{
	name={$ \ne$},
	plural={\ne},% im Mathematikmodus
	description={%
		Ein \glsIdx{MetaoperatorV}: ... ungleich (nicht dasselbe wie, nicht identisch zu) ..%
	}
}
\newglossaryentry{notequiv}{
	name={$ \notequiv$},
	plural={\notequiv},% im Mathematikmodus
	description={%
		Ein (Meta-)Operator: ... nicht äquivalent (ist nicht das gleiche wie, ist nicht so wie) ..%
	}
}

% sonstige mathematische Symbole -----------------------------------------------

\newglossaryentry{abltb}{
	name={$ \abltb$},
	plural={\abltb},% im Mathematikmodus
	description={%
		Ableitungsrelation: ... ableitbar ...
		(\seename\ \emph{\glsIdx{ableitbar}})%
	}
}
\newglossaryentry{lfalse}{
	name={$ \lfalse$},
	plural={\lfalse},% im Mathematikmodus
	description={%
		Eine Aussagenlogische Konstante: Falsch%
	}
}
\newglossaryentry{ltrue}{
	name={$ \ltrue$},
	plural={\ltrue},% im Mathematikmodus
	description={%
		Eine Aussagenlogische Konstante: Wahr%
	}
}
\newglossaryentry{subst}{
	name={$ \subst$},
	plural={\subst},% im Mathematikmodus
	description={%
		Substitution: ... substituiert durch ...
		(siehe die Definition in \subsectionname~\vref{sub:Basisregeln})%
	}
}

% Symbole für Mengen -----------------------------------------------------------

\newglossaryentry{gsN}{
	name={$ \gsN$},
	plural={\gsN},% im Mathematikmodus
	description={%
		Die Menge der natürlichen Zahlen ohne 0%
	}
}
\newglossaryentry{gsNo}{
	name={$ \gsNo$},
	plural={\gsNo},% im Mathematikmodus - erfolgt im falschen Zeichensatz!
	description={%
		Die Menge der natürlichen Zahlen einschließlich 0%
	}
}
\newglossaryentry{asA}{
	name={$ \asA$},
	plural={\asA},% im Mathematikmodus
	description={%
		Das Alphabet der aussagenlogischen Sprache%
	}
}
\newglossaryentry{asAx}{
	name={$ \asAx$},
	plural={\asAx},% im Mathematikmodus
	description={%
		Eine Teilmenge des Alphabets $\asA$ der aussagenlogischen Sprache%
	}
}
\newglossaryentry{asB}{
	name={$ \asB$},
	plural={\asB},% im Mathematikmodus
	description={%
		Die Menge der aussagenlogischen, binären Operatoren%
	}
}
\newglossaryentry{asC}{
	name={$ \asC$},
	plural={\asC},% im Mathematikmodus
	description={%
		Die Menge der aussagenlogischen Konstanten%
	}
}
\newglossaryentry{asF}{
	name={$ \asF$},
	plural={\asF},% im Mathematikmodus
	description={%
		Die Menge der aussagenlogischen Formeln mit Klammerung%
	}
}
\newglossaryentry{asFp}{
	name={$ \asFp$},
	plural={\asFp},% im Mathematikmodus
	description={%
		Die Menge der aussagenlogischen Formeln in polnischer Notation%
	}
}
\newglossaryentry{asFx}{
	name={$ \asFx$},
	plural={\asFx},% im Mathematikmodus
	description={%
		Eine Teilmenge der Menge $\asF$ der aussagenlogischen Formeln mit Klammerung%
	}
}
\newglossaryentry{asFxp}{
	name={$ \asFxp$},
	plural={\asFxp},% im Mathematikmodus
	description={%
		Eine Teilmenge der Menge $\asF$ der aussagenlogischen Formeln in polnischer Notation%
	}
}
\newglossaryentry{asJ}{
	name={$ \asJ$},
	plural={\asJ},% im Mathematikmodus
	description={%
		Die Menge der aussagenlogischen Operatoren%
	}
}
\newglossaryentry{asJx}{
	name={$ \asJx$},
	plural={\asJx},% im Mathematikmodus
	description={%
		Eine Teilmenge der Menge $\asJ$ der aussagenlogischen Operatoren
	}
}
\newglossaryentry{asM}{
	name={$ \asM$},
	plural={\asM},% im Mathematikmodus
	description={%
		Die Menge der \glsIdxPl{MetaoperatorV} und der mit Gleichheit verwandten Symbole%
	}
}
\newglossaryentry{asU}{
	name={$ \asU$},
	plural={\asU},% im Mathematikmodus
	description={%
		Die Menge der aussagenlogischen unären Operatoren%
	}
}
\newglossaryentry{asV}{
	name={$ \asV$},
	plural={\asV},% im Mathematikmodus
	description={%
		Die Menge der aussagenlogischen \glsIdxPl{atomareFormelA}%
	}
}

% Schlussregeln ----------------------------------------------------------------

\newcommand*{\tagAR}{AR}% Argument für \tag
\newglossaryentry{AR}{
	name={(AR)},
	plural={(\text{AR})},% im Mathematikmodus
	description={%
		\glsIdx{Anfangsregel}%
	}
}
\newcommand*{\tagMR}{MR}% Argument für \tag
\newglossaryentry{MR}{
	name={(MR)},
	plural={(\text{MR})},% im Mathematikmodus
	description={%
		\glsIdx{Monotonieregel}%
	}
}
\newcommand*{\tagSR}{SR}% Argument für \tag
\newglossaryentry{SR}{
	name={(SR)},
	plural={(\text{SR})},% im Mathematikmodus
	description={%
		\glsIdx{Schnittregel} (Modus ponens)%
	}
}
\newcommand*{\tagTR}{TR}% Argument für \tag
\newglossaryentry{TR}{
	name={(TR)},
	plural={(\text{TR})},% im Mathematikmodus
	description={%
		\glsIdx{Abtrennungsregel}%
	}
}
\newcommand*{\tagandB}{$\land$B}% Argument für \tag
\newglossaryentry{andB}{
	name={($ \land     $B)},
	plural={(\land\text{B})},% im Mathematikmodus
	description={%
		Beseitigung von \symqt{$\land$}%
	}
}
\newcommand*{\tagandE}{$\land$E}% Argument für \tag
\newglossaryentry{andE}{
	name={($ \land     $E)},
	plural={(\land\text{E})},% im Mathematikmodus
	description={%
		Einführung von \symqt{$\land$}%
	}
}
\newcommand*{\tagimpB}{$\limp$B}% Argument für \tag
\newglossaryentry{impB}{
	name={($ \limp     $B)},
	plural={(\limp\text{B})},% im Mathematikmodus
	description={%
		Beseitigung von \symqt{$\limp$}%
	}
}
\newcommand*{\tagimpE}{$\limp$E}% Argument für \tag
\newglossaryentry{impE}{
	name={($ \limp     $E)},
	plural={(\limp\text{E})},% im Mathematikmodus
	description={%
		Einführung von \symqt{$\limp$}%
	}
}
\newcommand*{\tagnota}{$\lnot$1}% Argument für \tag
\newglossaryentry{nota}{
	name={($ \lnot     $1)},
	plural={(\lnot\text{1})},% im Mathematikmodus
	description={%
		Einführung/Beseitigung von \symqt{$\lnot$} Teil 1%
	}
}
\newcommand*{\tagnotb}{$\lnot$2}% Argument für \tag
\newglossaryentry{notb}{
	name={($ \lnot     $2)},
	plural={(\lnot\text{2})},% im Mathematikmodus
	description={%
		Einführung/Beseitigung von \symqt{$\lnot$} Teil 2%
	}
}
\newcommand*{\tagnotc}{$\lnot$3}% Argument für \tag
\newglossaryentry{notc}{
	name={($ \lnot     $3)},
	plural={(\lnot\text{3})},% im Mathematikmodus
	description={%
		Beweistechnik \strqt{Indirekter Beweis}%
	}
}
\newcommand*{\tagnotd}{$\lnot$4}% Argument für \tag
\newglossaryentry{notd}{% statt "notE"
	name={($ \lnot     $4)},
	plural={(\lnot\text{4})},% im Mathematikmodus
	description={%
		Reductio ad absurdum (indirekter Beweis)%
	}
}
%%%\newglossaryentry{orB}{
%%%	name={($ \lor     $B)},
%%%	plural={(\lor\text{B})},% im Mathematikmodus
%%%	description={%
%%%		Beseitigung von \symqt{$\lor$}%
%%%	}
%%%}
%%%\newglossaryentry{orE}{
%%%	name={($ \lor     $E)},
%%%	plural={(\lor\text{E})},% im Mathematikmodus
%%%	description={%
%%%		Einführung von \symqt{$\lor$}%
%%%	}
%%%}

% keine Symbole mehr -----------------------------------------------------------

\newglossaryentry{ableitbar}{
	name={ableitbar},
	plural={ableitbare},
	description={%
		Wenn sich eine Formel $\beta$ aus einer Formel $\alpha$ mittels zulässiger Transaktionen ableiten lässt, heißt $\beta$ ableitbar aus $\alpha$.
		Sprechweise: \forqt{$\alpha$ ableitbar $\beta$}.
		Eine oder beide Formeln $\alpha$ \textbzw $\beta$ dürfen dabei durch Formelmengen ersetzt werden.
		(\seealso\ \symqt{\glsIdx{abltb}} und \glsIdx{Ableitungsrelation})
		-- Synonym: \glsIdx{beweisbar}%
	}
}
\newglossaryentry{Ableitungsrelation}{
	name={Ableitungsrelation},
	plural={Ableitungsrelationen},
	description={%
		Die Relation \symqt{\glsIdx{abltb}}%
	}
}
\newglossaryentry{Abtrennungsregel}{
	name={Abtrennungsregel},
	plural={Abtrennungsregeln},
	description={%
		Eine \emph{\glsIdx{Schlussregel}} - siehe~\glsIdx{TR}%
	}
}
\newglossaryentry{Anfangsregel}{
	name={Anfangsregel},
	plural={Anfangsregeln},
	description={%
		Eine \emph{\glsIdx{Schlussregel}} um beginnen zu können - siehe~\glsIdx{AR}%
	}
}
\newcommand*{\ASBA}{\glsIdx{ASBA}}
\newacronym{ASBA}{ASBA}{
	Programmsystem, das \textbf{A}xiome, \textbf{S}ätze, \textbf{B}eweise und \textbf{A}uswertungen behandeln kann%
}
\newglossaryentry{atomareFormelA}{
	name={atomare Formel},     % eine ...
	plural={atomaren Formeln}, % alle ...
	description={%
		Eine Formel, die sich nicht weiter zerlegen lässt%
	}%
}
\newglossaryentry{Ausgabeschema}{
	name={Ausgabeschema},
	plural={Ausgabeschemata},
	description={%
		Ein Schema, mit dem bestimmte mathematische Objekte ausgegeben werden sollen%
	}
}
\newglossaryentry{Aussage}{
	name={Aussage},
	plural={Aussagen},
	description={%
		Eine Aussage in natürlicher Sprache oder als Formel, die einen Wahrheitswert liefert%
	}
}
\newglossaryentry{Aussagenlogik}{
	name={Aussagenlogik},
	description={%
		\seename\ \sectionname~\vref{sec:Aussagenlogik}%
	}
}
\newglossaryentry{Axiom}{
	name={Axiom},
	plural={Axiome},
	description={%
		Eine Formel, die unbewiesen als wahr angesehen wird%
	}
}
\newglossaryentry{Basisregel}{
	name={Basisregel},
	plural={Basisregeln},
	description={%
		Eine \emph{\glsIdx{Schlussregel}}, die nicht mehr auf andere zurückgeführt wird%
	}
}
\newglossaryentry{Beweis}{
	name={Beweis},
	plural={Beweise},
	description={%
		Eine zulässige Ableitung von Folgerungen aus gegebenen Voraussetzungen%
	}
}
\newglossaryentry{beweisbar}{
	name={beweisbar},
	plural={beweisbare},
	description={%
		Synonym zu \emph{ableitbar}%
	}
}

\newglossaryentry{Beweisschritt}{
	name={Beweisschritt},
	plural={Beweisschritte},
	description={%
		Eine Vorschrift, wie aus vorgegebenen Aussagen eine weitere folgt%
	}
}
\newglossaryentry{BoolscheSignatur}{
	name={Boolsche Signatur},
	plural={Boolschen Signatur},% Dativ
	description={%
		Die \glsIdx{logischeSignaturV} $\{\lnot, \land, \lor\}$%
	}
}
\newglossaryentry{Fachbegriff}{
	name={Fachbegriff},
	plural={Fachbegriffe},
	description={%
		Ein Name für einen mathematischen Begriff%
	}
}
\newglossaryentry{Fachgebiet}{
	name={Fachgebiet},
	plural={Fachgebiete},
	description={%
		Ein Teil der Mathematik mit einer zugehörigen Basis aus Axiomen, Sätzen und spezifischen Fachbegriffen und Darstellungen%
	}
}
\newglossaryentry{FormalelementV}{
	name={formales Element},   % ein   ...
	plural={formale Elemente}, % viele ...
	description={%
		Ein mathematisches Element in formaler Schreibweise.
		Bis auf wenige Aussagen kommen darin \emph{\glsIdxPl{MetaausdruckV}} nicht mehr vor%
	}%
}
\newglossaryentry{intEigenschaftA}{
	name={interessierende Eigenschaft},      % eine ...
	plural={interessierenden Eigenschaften}, % alle ...
	description={%
		Solche Eigenschaften von Ausdrücken, die im aktuellen Zusammenhang von Interesse sind.%
	}%
}
\newglossaryentry{logischeSignaturV}{
	name={logische Signatur},     % eine  ...
	plural={logische Signaturen}, % viele ...
	description={%
		Eine in \emph{\glsIdx{Metasprache}} verfasste Aussage, die auch zusammengesetzt sein kann%
	}%
}
\newglossaryentry{MetaausdruckV}{
	name={metasprachlicher Ausdruck},   % ein   ...
	plural={metasprachliche Ausdrücke}, % viele ...
	description={%
		Eine in normaler Sprache verfasste Aussage, die auch zusammengesetzt sein kann%
	}%
}
\newglossaryentry{MetaaussageV}{
	name={metasprachliche Aussage},    % eine   ...
	plural={metasprachliche Aussagen}, % viele ...
	description={%
		Eine in \emph{\glsIdx{Metasprache}} verfasste Aussage, die auch zusammengesetzt sein kann%
	}%
}
\newglossaryentry{MetaoperatorV}{
	name={metasprachlicher Operator},    % ein   ...
	plural={metasprachliche Operatoren}, % viele ...
	description={%
		Ein Operator, dessen Operanden \emph{\glsIdxPl{MetaausdruckV}} sind%
	}%
}
\newglossaryentry{Metasprache}{
	name={Metasprache},
	plural={Metasprachen},
	description={%
		Eine Sprache, in der Aussagen über Elemente einer anderen Sprache getroffen werden können%
	}%
}
\newglossaryentry{Monotonieregel}{
	name={Monotonieregel},
	plural={Monotonieregeln},
	description={%
		Eine \emph{\glsIdx{Schlussregel}} - siehe~\glsIdx{MR}%
	}
}
\newglossaryentry{Praedikat}{
	name={Prädikat},
	plural={Prädikate},
	description={%
		Ein Element der \emph{\glsIdx{Praedikatenlogik}} (\see \sectionname~\vref{sec:Praedikatenlogik}).
		\textZB kann man eine \forqt{$Gruppe$} als ein zweistelliges Prädikat \forqt{$Gruppe(G,+)$} definieren, in dem $G$ eine Menge und \symqt{$+$} eine Operation, \textdh eine (zweistellige) Funktion \forqt{$+: G \times G \rightarrow G$} ist, so dass die Gruppenaxiome erfüllt sind%
	}
}
\newglossaryentry{Praedikatenlogik}{
	name={Prädikatenlogik},
	description={%
		\seename\ \sectionname~\vref{sec:Praedikatenlogik}%
	}
}
\newglossaryentry{Satz}{
	name={Satz},
	plural={Sätze},
	description={%
		Eine mathematische Aussage, dass eine bestimmte Folgerung aus gegebenen Voraussetzungen abgeleitet werden kann%
	}
}
\newglossaryentry{Schlussregel}{
	name={Schlussregel},
	plural={Schlussregeln},
	description={%
		Eine Regel für eine (zulässige) Umwandlung von Formeln%
	}
}
\newglossaryentry{Schnittregel}{
	name={Schnittregel},
	plural={Schnittregeln},
	description={%
		Eine \emph{\glsIdx{Schlussregel}} - siehe~\glsIdx{SR}%
	}
}
\newglossaryentry{vergleichbar}{
	name={vergleichbar},
	plural={vergleichbare},
	description={
		Zwei \emph{\glsIdxPl{MetaausdruckV}} \textbzw \emph{\glsIdxPl{FormalelementV}} heißen -- auf eine bestimmte Art -- vergleichbar, wenn sie auf diese Art (\textzB als Zeichenketten oder als vergleichbare Ergebnisse von Formeln) verglichen werden können.
		Die Art muss implizit bekannt oder explizit angegeben sein.
		Meistens genügt es zu wissen, was für \glsIdxPl{MetaausdruckV} \textbzw \glsIdxPl{FormalelementV} es sind.
		Sie müssen dann nur von derselben Art sein%
	}%
}
\newglossaryentry{Wahrheitswert}{
	name={Wahrheitswert},
	plural={Wahrheitswerte},
	description={%
		Wahrheitswerte sind die Werte \strqt{wahr} und \strqt{falsch}, oft auch als \strqt{true} und \strqt{false} oder einfach \charqt{1} und \charqt{0} bezeichnet%
	}
}
\newglossaryentry{zulaessigeTransformation}{
	name={zulässige Transformation},
	plural={zulässige Transformation},
	description={%
		Die Anwendung einer \glsIdx{Basisregel} oder einer davon abgeleiteten sonstigen \glsIdx{Schlussregel}%
	}
}

%%############################################################################%%
%%                                                                            %%
%% Datei:  ASBA-Vorspann.tex                                                  %%
%% Inhalt: Vorspann für die Datei ASBA.txt                                    %%
%%                                                                            %%
%% Copyright (C) 2017  Winfried Teschers                                      %%
%%                                                                            %%
%% This program is free software: you can redistribute it and/or modify       %%
%% it under the terms of the GNU Affero General Public License as published   %%
%% by the Free Software Foundation, either version 3 of the License, or       %%
%% (at your option) any later version.                                        %%
%%                                                                            %%
%% This program is distributed in the hope that it will be useful,            %%
%% but WITHOUT ANY WARRANTY; without even the implied warranty of             %%
%% MERCHANTABILITY or FITNESS FOR A PARTICULAR PURPOSE.  See the              %%
%% GNU Affero General Public License for more details.                        %%
%%                                                                            %%
%% You should have received a copy of the GNU Affero General Public License   %%
%% along with this program.  If not, see <http://www.gnu.org/licenses/>.      %%
%%                                                                            %%
%% Dr. Winfried Teschers                                                      %%
%% Anton-Günther-Straße 26c                                                   %%
%% 91083 Baiersdorf                                                           %%
%% Germany                                                                    %%
%%                                                                            %%
%% e-mail: winfried.teschers@t-online.de                                      %%
%%                                                                            %%
%%############################################################################%%

% !TeX root = ASBA.tex
% !TeX encoding = UTF-8
% !TeX spellcheck = de_DE

% Elemente, die in anderen Dateien als "ASBA-Mathematik" verwendet werden, werden in "ASBA-Vorspann" definiert.

% Metasprachliche Symbole ######################################################

\newcommand*{\srand}{\mid}% in formalen Sätzen und Schlussregeln:   ... und ...
% Nur im Mathematikmodus!
\DeclareMathOperator{\metaand}{\&}%   ... und  ...
\DeclareMathOperator{\metaor}{||}%    ... oder ...
\DeclareMathOperator{\metaorr}{||\,}% ... oder ... (besserer Abstand)
\DeclareMathOperator{\metaimp}{\Rightarrow}%   aus ... folgt              ...
\DeclareMathOperator{\metarep}{\Leftarrow}%        ... folgt aus          ...
\DeclareMathOperator{\metaequiv}{\Leftrightarrow}% ... genau dann wenn    ...
\DeclareMathOperator{\metadefeq}{:\!\metaequiv}%   ... definitionsgemäß " ...
\DeclareMathOperator{\eq}{=}%                      ... gleich             ...
\DeclareMathOperator{\defeq}{:\!\eq}%              ... definitionsgemäß " ...

% Mathematische Symbole ########################################################

% Beispieloperatoren ===========================================================
% \*bsp
\DeclareMathOperator{\opbsp}{\bullet}
\DeclareMathOperator{\relbsp}{\sim}
\DeclareMathOperator{\relBsp}{\relbsp_1}
\DeclareMathOperator{\relnbsp}{\nsim}
\DeclareMathOperator{\releqbsp}{\simeq}
\DeclareMathOperator{\lrelbsp}{\lhd}
\DeclareMathOperator{\rrelbsp}{\rhd}
\DeclareMathOperator{\lreleqbsp}{\unlhd}
\DeclareMathOperator{\rreleqbsp}{\unrhd}

% Definitionen für die Tabelle der Junktoren ===================================
% \l*  -           logischer Operator
% \ln* - negierter logischer Operator
% Logische Operatoren als Addition und Multiplikation
\DeclareMathOperator{\ladd}{+}
\DeclareMathOperator{\lmult}{\cdot}
% Wahrheitswerte ---------------------------------------------------------------
\newcommand*{\texttrue}{W}%  in einem Kommentar stets 'W'
\newcommand*{\textfalse}{F}% in einem Kommentar stets 'F'
% Konstante --------------------------------------------------------------------
\newcommand*{\ltrue}{\top}%      W - wahr
%\newcommand*{\lnfalse}{\notbot}% " - nicht falsch
\newcommand*{\lfalse}{\bot}%     F - falsch
%\newcommand*{\lntrue}{\nottop}%  " - nicht wahr
% unäre Operatoren -------------------------------------------------------------
%                                                        W F - Aussage A
%\DeclareMathOperator{\lutrue}{\operatorname{\top}}%     W W - wahr
%\DeclareMathOperator{\lnufalse}{\operatorname{\notbot}}%" " - nicht falsch
%                                                        W F - A
%             \lnot                                      F W - nicht
%\DeclareMathOperator{\lufalse}{\operatorname{\bot}}%    F F - falsch
%\DeclareMathOperator{\lnutrue}{\operatorname{\nottop}}% " " - nicht wahr
% binäre Operatoren ------------------------------------------------------------
%                                                        W W F F - Aussage A
%                                                        W F W F - Aussage B
%  - - - - - - - - - - - - - - - - - - - - - - - - - - - - - - - - - - - - - - -
%\DeclareMathOperator{\lbtrue}{\operatorname{\top}}%     W W W W - wahr [binär]
%\DeclareMathOperator{\lnbfalse}{\operatorname{\notbot}}%" " " " - nicht falsch
%            \lor                                        W W W F - A oder B
\DeclareMathOperator{\lrep}{\leftarrow}%                 W W F W - A folgt aus B
\DeclareMathOperator{\lrepA}{\Leftarrow}%
\DeclareMathOperator{\lrepB}{\subset}%
\DeclareMathOperator{\lleft}{\operatorname{\rfloor}}%    W W F F - A
%  - - - - - - - - - - - - - - - - - - - - - - - - - - - - - - - - - - - - - - -
\DeclareMathOperator{\limp}{\rightarrow}%                W F W W - aus A folgt B
\DeclareMathOperator{\limpA}{\Rightarrow}%
\DeclareMathOperator{\limpB}{\supset}%
\DeclareMathOperator{\lright}{\operatorname{\lfloor}}%   W F W F - B
\DeclareMathOperator{\lequiv}{\leftrightarrow}%          W F F W - A genau dann,
\DeclareMathOperator{\lequivA}{\Leftrightarrow}%                    wenn B
%            \lnxor                                      " " " " - nicht
%                                                            (entweder A oder B)
%            \land                                       W F F F - A und B
\DeclareMathOperator{\landA}{\&}
\DeclareMathOperator{\landB}{\lmult}
%  - - - - - - - - - - - - - - - - - - - - - - - - - - - - - - - - - - - - - - -
\DeclareMathOperator{\lnand}{\uparrow}%                  F W W W - nicht
\DeclareMathOperator{\lnandA}{\barwedge}%                          (A und B)
\DeclareMathOperator{\lnandB}{\mid}%
\DeclareMathOperator{\lxor}{\ladd}%                      F W W F - entweder A
\DeclareMathOperator{\lxorA}{\operatorname{\dot\lor}}%                 oder B
\DeclareMathOperator{\lxorB}{\veebar}%
\DeclareMathOperator{\lxorC}{\oplus}%
\DeclareMathOperator{\lnequiv}{\nleftrightarrow}%        " " " " - nicht
\DeclareMathOperator{\lnequivA}{\nLeftrightarrow}%        (A genau dann, wenn B)
\DeclareMathOperator{\lnequivB}{\nequiv}%
\DeclareMathOperator{\lnright}{\lceil}%                  F W F W - nicht B
\DeclareMathOperator{\lnimp}{\nrightarrow}%              F W F F - nicht
\DeclareMathOperator{\lnimpA}{\nRightarrow}%                     (aus A folgt B)
\DeclareMathOperator{\lnimpB}{\nsupset}%
%  - - - - - - - - - - - - - - - - - - - - - - - - - - - - - - - - - - - - - - -
\DeclareMathOperator{\lnleft}{\rceil}%                   F F W W - nicht A
\DeclareMathOperator{\lnrep}{\nleftarrow}%               F F W F - nicht
\DeclareMathOperator{\lnrepA}{\nLeftarrow}%                      (A folgt aus B)
\DeclareMathOperator{\lnrepB}{\nsubset}%
\DeclareMathOperator{\lnor}{\downarrow}%                 F F F W - nicht
\DeclareMathOperator{\lnorA}{\operatorname{\overline\vee}}%        (A oder B)
%\DeclareMathOperator{\lbfalse}{\operatorname{\bot}}%    F F F F - falsch[binär]
%\DeclareMathOperator{\lnbtrue}{\operatorname{\nottop}}% " " " " - nicht wahr

% Verwendete Mengenbezeichnungen ===============================================

% \gs* = globales Symbol
\newcommand*{\gsN}{\mathbb{N}}%    Menge der natürlichen Zahlen ohne 0
\newcommand*{\gsNo}{\mathbb{N}_0}% Menge der natürlichen Zahlen einschließlich 0

% Elemente und Mengen für Beweise
\newcommand*{\Voraussetzung}  {V}
\newcommand*{\Beweisschritt}  {S}
\newcommand*{\Folgerung}      {F}
\newcommand*{\Voraussetzungen}{\mathcal{\Voraussetzung}}
\newcommand*{\Beweisschritte} {\mathcal{\Beweisschritt}}
\newcommand*{\Folgerungen}    {\mathcal{\Folgerung}}

% \al* = aussagenlogische
\newcommand*{\alABC}  {\mathcal{A}}%    Alphabet der Sprache
\newcommand*{\alABCx} {\alABC_x}%   ... davon eine Teilmenge bzgl. \alJunx
\newcommand*{\alABCy} {\mathcal{A}_y}%  entsprechend \alABCx
\newcommand*{\alBin}  {\mathcal{B}}%    Menge der binären Operatoren
\newcommand*{\alCon}  {\mathcal{C}}%    Menge der Konstanten
\newcommand*{\alFor}  {\mathcal{L}}%    Menge der Formeln
\newcommand*{\alForp} {\alFor^p}%   ... in polnischer Notation
\newcommand*{\alForx} {\alFor_x}%       Teilenge der Formeln
\newcommand*{\alForxp}{\alFor_x^p}% ... bzgl. \alJunx in polnischer Notation
\newcommand*{\alFory} {\alFor_y}%       entsprechend \alForx
\newcommand*{\alForyp}{\alFor_y^p}%     entsprechend \alForxp
\newcommand*{\alJun}  {\mathcal{J}}%    Menge der Junktoren
\newcommand*{\alJunx} {\alJun_x}%       Teilmenge der Junktoren
\newcommand*{\alJuny} {\alJun_y}%       entsprechend \alJunx
\newcommand*{\alMet}  {\mathcal{M}}%    Metaoperatoren und "Gleichheiten"
\newcommand*{\alUna}  {\mathcal{U}}%    Menge der unären Operatoren
\newcommand*{\alVar}  {\mathcal{Q}}%    Menge der Variablen
\newcommand*{\alvar}  {q}%              Name einer Variablen
% verschiedene Indizes für Teilmengen von \alABC, \alFor, \alForp und \alJun
\newcommand*{\xAnd} {\mathrm{and}}%
\newcommand*{\xBool}{\mathrm{bool}}%
\newcommand*{\xImp} {\mathrm{imp}}%
\newcommand*{\xNand}{\mathrm{nand}}%
\newcommand*{\xNor} {\mathrm{nor}}%
\newcommand*{\xOr}  {\mathrm{or}}%
\newcommand*{\xRep} {\mathrm{rep}}%

% sonstige mathematische Zeichen ===============================================

\newcommand*{\textderive}{$\vdash$}%               im Textmodus
\DeclareMathOperator{\derive}   {\;\vdash\;}%      ... ableitbar ...
\DeclareMathOperator{\derivegls}{\vdash}%          -"- für Glossar
\DeclareMathOperator{\swap}     {\leftrightarrows}%... vertauscht mit ...
\DeclareMathOperator{\subst}    {\leftarrowtail}%  ... substituiert durch ...
%%%\DeclareMathOperator{\subst}    {\curvearrowright}%... substituiert durch ...

% sonstige Kommandos für den Mathematiksatz ####################################

\mathtoolsset{showonlyrefs,showmanualtags}% Nur mit \ref referenzierte Gleichungen, aber alle manuellen Tags

% Glossareinträge ##############################################################

% Symbole für Beispieloperatoren -----------------------------------------------

\newglossaryentry{lrelbsp}{
	name={$ \lrelbsp$},
	plural={\lrelbsp},% im Mathematikmodus
	description={%
		Ein Beispielsymbol für eine Relation mit Umkehrrelation $\rrelbsp$%
	}
}
\newglossaryentry{lreleqbsp}{
	name={$ \lreleqbsp$},
	plural={\lreleqbsp},% im Mathematikmodus
	description={%
		Ein Beispielsymbol für eine Relation mit Gleichheit und Umkehrrelation $\rreleqbsp$%
	}
}
\newglossaryentry{relbsp}{
	name={$ \relbsp$},
	plural={\relbsp},% im Mathematikmodus
	description={%
		Ein Beispielsymbol für eine Relation%
	}
}
\newglossaryentry{releqbsp}{
	name={$ \releqbsp$},
	plural={\releqbsp},% im Mathematikmodus
	description={%
		Ein Beispielsymbol für eine Relation mit Gleichheit%
	}
}
\newglossaryentry{relnbsp}{
	name={$ \relnbsp$},
	plural={\relnbsp},% im Mathematikmodus
	description={%
		Verneinung von $\relbsp$%
	}
}
\newglossaryentry{rrelbsp}{
	name={$ \rrelbsp$},
	plural={\rrelbsp},% im Mathematikmodus
	description={%
		Ein Beispielsymbol für eine Relation mit Umkehrrelation $\lrelbsp$%
	}
}
\newglossaryentry{rreleqbsp}{
	name={$ \rreleqbsp$},
	plural={\rreleqbsp},% im Mathematikmodus
	description={%
		Ein Beispielsymbol für eine Relation mit Gleichheit und Umkehrrelation $\lreleqbsp$%
	}
}

% Symbole für Metaoperatoren ---------------------------------------------------

\newglossaryentry{defeq}{
	name={$ \defeq$},
	plural={\defeq},% im Mathematikmodus
	description={%
		Ein \emph{Metaoperator}: ... definitionsgemäß gleich ..%
	}
}
\newglossaryentry{eq}{
	name={$ \eq$},
	plural={\eq},% im Mathematikmodus
	description={%
		Ein \emph{Metaoperator}: ... gleich (ist dasselbe wie, ist identisch zu) ..%
	}
}
\newglossaryentry{equiv}{
	name={$ \equiv$},
	plural={\equiv},% im Mathematikmodus
	description={%
		Ein (Meta-)Operator: ... äquivalent (ist das gleiche wie, ist so wie) zu ..%
	}
}
\newglossaryentry{metaand}{
	name={$ \metaand$},
	plural={\metaand},% im Mathematikmodus
	description={%
		Ein \emph{Metaoperator}: ... und ...
		Die Priorität ist höher als die von \symqt{$\srand$}%
	}
}
\newglossaryentry{metadefeq}{
	name={$ \metadefeq$},
	plural={\metadefeq},% im Mathematikmodus
	description={%
		Ein \emph{Metaoperator}: ... definitionsgemäß gleich (definitionsgemäß genau dann, wenn) ..%
	}
}
\newglossaryentry{metaequiv}{
	name={$ \metaequiv$},
	plural={\metaequiv},% im Mathematikmodus
	description={%
		Ein \emph{Metaoperator}: ... genau dann wenn ..%
	}
}
\newglossaryentry{metaimp}{
	name={$ \metaimp$},
	plural={\metaimp},% im Mathematikmodus
	description={%
		Ein \emph{Metaoperator}: ... dann auch ..%
	}
}
\newglossaryentry{metaor}{
	name={$ \metaor$},
	plural={\metaor},% im Mathematikmodus
	description={%
		Ein \emph{Metaoperator}: ... oder ..%
	}
}
\newglossaryentry{metarep}{
	name={$ \metarep$},
	plural={\metarep},% im Mathematikmodus
	description={%
		Ein \emph{Metaoperator}: ... sofern ..%
	}
}
\newglossaryentry{ne}{
	name={$ \ne$},
	plural={\ne},% im Mathematikmodus
	description={%
		Ein \emph{Metaoperator}: ... ungleich (nicht dasselbe wie, nicht identisch zu) ..%
	}
}
\newglossaryentry{nequiv}{
	name={$ \nequiv$},
	plural={\nequiv},% im Mathematikmodus
	description={%
		Ein (Meta-)Operator: ... nicht äquivalent (ist nicht das gleiche wie, ist nicht so wie) ..%
	}
}
\newglossaryentry{subst}{
	name={$ \subst$},
	plural={\subst},% im Mathematikmodus
	description={%
		\emph{Substitution}: ... substituiert durch ...
		-- siehe die Definition \vrefinsub{sub:Indentitaetsregeln}%
	}
}
\newglossaryentry{swap}{
	name={$ \swap$},
	plural={\swap},% im Mathematikmodus
	description={%
		\emph{Vertauschung}: ... vertauscht mit ...
		-- siehe die Definition \vrefinsub{sub:Indentitaetsregeln}%
	}
}
\newglossaryentry{srand}{
	name={$ \srand$},
	plural={\srand},% im Mathematikmodus
	description={%
		Ein \emph{Metaoperator}: ... und ...
		Die Priorität ist niedriger als die von \symqt{$\metaand$}%
	}
}

% sonstige mathematische Symbole -----------------------------------------------

\newglossaryentry{derive}{
	name={$ \derivegls$},
	plural={\derivegls},% im Mathematikmodus
	description={%
		Ableitungsrelation: ... ableitbar ...
		-- siehe \emph{ableitbar}%
	}
}
\newglossaryentry{lfalse}{
	name={$ \lfalse$},
	plural={\lfalse},% im Mathematikmodus
	description={%
		Eine Aussagenlogische Konstante: Falsch%
	}
}
\newglossaryentry{ltrue}{
	name={$ \ltrue$},
	plural={\ltrue},% im Mathematikmodus
	description={%
		Eine Aussagenlogische Konstante: Wahr%
	}
}

% Symbole für Mengen -----------------------------------------------------------

\newglossaryentry{gsN}{
	name={$ \gsN$},
	plural={\gsN},% im Mathematikmodus
	description={%
		Die Menge der natürlichen Zahlen ohne 0%
	}
}
\newglossaryentry{gsNo}{
	name={$ \gsNo$},
	plural={\gsNo},% im Mathematikmodus - erfolgt im falschen Zeichensatz! ???
	description={%
		Die Menge der natürlichen Zahlen einschließlich 0%
	}
}
\newglossaryentry{alABC}{
	name={$ \alABC$},
	plural={\alABC},% im Mathematikmodus
	description={%
		Das Alphabet der aussagenlogischen Sprache%
	}
}
\newglossaryentry{alABCx}{
	name={$ \alABCx$},
	plural={\alABCx},% im Mathematikmodus
	description={%
		Eine Teilmenge des Alphabets $\alABC$ der aussagenlogischen Sprache%
	}
}
\newglossaryentry{alBin}{
	name={$ \alBin$},
	plural={\alBin},% im Mathematikmodus
	description={%
		Die Menge der aussagenlogischen, binären Operatoren%
	}
}
\newglossaryentry{alCon}{
	name={$ \alCon$},
	plural={\alCon},% im Mathematikmodus
	description={%
		Die Menge der aussagenlogischen Konstanten%
	}
}
\newglossaryentry{alFor}{
	name={$ \alFor$},
	plural={\alFor},% im Mathematikmodus
	description={%
		Die Menge der aussagenlogischen Formeln mit Klammerung%
	}
}
\newglossaryentry{alForp}{
	name={$ \alForp$},
	plural={\alForp},% im Mathematikmodus
	description={%
		Die Menge der aussagenlogischen Formeln in polnischer Notation%
	}
}
\newglossaryentry{alForx}{
	name={$ \alForx$},
	plural={\alForx},% im Mathematikmodus
	description={%
		Eine Teilmenge der Menge $\alFor$ der aussagenlogischen Formeln mit Klammerung%
	}
}
\newglossaryentry{alForxp}{
	name={$ \alForxp$},
	plural={\alForxp},% im Mathematikmodus
	description={%
		Eine Teilmenge der Menge $\alFor$ der aussagenlogischen Formeln in polnischer Notation%
	}
}
\newglossaryentry{alJun}{
	name={$ \alJun$},
	plural={\alJun},% im Mathematikmodus
	description={%
		Die Menge der aussagenlogischen Junktoren (Operatorsymbole)%
	}
}
\newglossaryentry{alJunx}{
	name={$ \alJunx$},
	plural={\alJunx},% im Mathematikmodus
	description={%
		Eine Teilmenge der Menge $\alJun$ der aussagenlogischen Operatoren
	}
}
\newglossaryentry{alMet}{
	name={$ \alMet$},
	plural={\alMet},% im Mathematikmodus
	description={%
		Die Menge der \emph{Metaoperatoren} und der mit Gleichheit verwandten Symbole%
	}
}
\newglossaryentry{alUna}{
	name={$ \alUna$},
	plural={\alUna},% im Mathematikmodus
	description={%
		Die Menge der aussagenlogischen unären Operatoren%
	}
}
\newglossaryentry{alVar}{
	name={$ \alVar$},
	plural={\alVar},% im Mathematikmodus
	description={%
		Die Menge der aussagenlogischen \emph{Variablen}%
	}
}

% Schlussregeln ----------------------------------------------------------------

\newcommand*{\tagAR}{AR}% Argument für \tag - im Textmodus
\newcommand*{\AR}{(\text{AR})}% im Mathematikmodus
\newglossaryentry{AR}{
	name={$ \AR$},
	plural={\AR},% im Mathematikmodus
	description={%
		\emph{Anfangsregel}%
	}
}
\newcommand*{\tagFS}{FS}% Argument für \tag - im Textmodus
\newcommand*{\FS}{(\text{FS})}% im Mathematikmodus
\newglossaryentry{FS}{
	name={$ \FS$},
	plural={\FS},% im Mathematikmodus
	description={%
		\emph{formaler Satz}%
	}
}
\newcommand*{\tagMR}{MR}% Argument für \tag - im Textmodus
\newcommand*{\MR}{(\text{MR})}% im Mathematikmodus
\newglossaryentry{MR}{
	name={$ \MR$},
	plural={\MR},% im Mathematikmodus
	description={%
		\emph{Monotonieregel}%
	}
}
\newcommand*{\tagSR}{SR}% Argument für \tag - im Textmodus
\newcommand*{\SR}{(\text{SR})}% im Mathematikmodus
\newglossaryentry{SR}{
	name={$ \SR$},
	plural={\SR},% im Mathematikmodus
	description={%
		\emph{Schnittregel} (Modus ponens)%
	}
}
\newcommand*{\tagTR}{TR}% Argument für \tag - im Textmodus
\newcommand*{\TR}{(\text{TR})}% im Mathematikmodus
\newglossaryentry{TR}{
	name={$ \TR$},
	plural={\TR},% im Mathematikmodus
	description={%
		\emph{Abtrennungsregel}%
	}
}
\newcommand*{\tageqB}{$\eq$B}% Argument für \tag - im Textmodus
\newcommand*{\eqB}{(\eq\text{B})}% im Mathematikmodus
\newglossaryentry{eqB}{
	name={($\eqB$)},
	plural={\eqB},% im Mathematikmodus
	description={%
		Beseitigung von \symqt{$\eq$}%
	}
}
\newcommand*{\tageqE}{$\eq$E}% Argument für \tag - im Textmodus
\newcommand*{\eqE}{(\eq\text{E})}% im Mathematikmodus
\newglossaryentry{eqE}{
	name={($\eqE$)},
	plural={\eqE},% im Mathematikmodus
	description={%
		Einführung von \symqt{$\eq$}%
	}
}
\newcommand*{\tagandB}{$\land$B}% Argument für \tag - im Textmodus
\newcommand*{\andB}{(\land\text{B})}% im Mathematikmodus
\newglossaryentry{andB}{
	name={($\andB$)},
	plural={\andB},% im Mathematikmodus
	description={%
		Beseitigung von \symqt{$\andB$}%
	}
}
\newcommand*{\tagandE}{$\land$E}% Argument für \tag - im Textmodus
\newcommand*{\andE}{(\land\text{E})}% im Mathematikmodus
\newglossaryentry{andE}{
	name={($\andE$)},
	plural={\andE},% im Mathematikmodus
	description={%
		Beseitigung von \symqt{$\andE$}%
	}
}
\newcommand*{\tagimpB}{$\limp$B}% Argument für \tag - im Textmodus
\newcommand*{\impB}{(\limp\text{B})}% im Mathematikmodus
\newglossaryentry{impB}{
	name={($\impB$)},
	plural={\impB},% im Mathematikmodus
	description={%
		Beseitigung von \symqt{$\impB$}%
	}
}
\newcommand*{\tagimpE}{$\limp$E}% Argument für \tag - im Textmodus
\newcommand*{\impE}{(\limp\text{E})}% im Mathematikmodus
\newglossaryentry{impE}{
	name={($\impE$)},
	plural={\impE},% im Mathematikmodus
	description={%
		Beseitigung von \symqt{$\impE$}%
	}
}
\newcommand*{\tagnota}{$\lnot$1}% Argument für \tag - im Textmodus
\newcommand*{\nota}{(\lnot\text{1})}% im Mathematikmodus
\newglossaryentry{nota}{
	name={($\nota$)},
	plural={\nota},% im Mathematikmodus
	description={%
		Einführung/Beseitigung von \symqt{$\lnot$} Teil 1%
	}
}
\newcommand*{\tagnotb}{$\lnot$2}% Argument für \tag - im Textmodus
\newcommand*{\notb}{(\lnot\text{2})}% im Mathematikmodus
\newglossaryentry{notb}{
	name={($\notb$)},
	plural={\notb},% im Mathematikmodus
	description={%
		Einführung/Beseitigung von \symqt{$\lnot$} Teil 1%
	}
}
\newcommand*{\tagnotc}{$\lnot$3}% Argument für \tag - im Textmodus
\newcommand*{\notc}{(\lnot\text{3})}% im Mathematikmodus
\newglossaryentry{notc}{
	name={($\notc$)},
	plural={\notc},% im Mathematikmodus
	description={%
		Beweistechnik \strqt{Indirekter Beweis}%
	}
}
\newcommand*{\tagnotd}{$\lnot$4}% Argument für \tag - im Textmodus
\newcommand*{\notd}{(\lnot\text{4})}% im Mathematikmodus
\newglossaryentry{notd}{% statt "notE"
	name={($\notd$)},
	plural={\notd},% im Mathematikmodus
	description={%
		Reductio ad absurdum (indirekter Beweis)%
	}
}
%%%\newcommand*{\tagorB}{$\lor$B}% Argument für \tag - im Textmodus
%%%\newcommand*{\orB}{(\lor\text{B})}% im Mathematikmodus
%%%\newglossaryentry{orB}{
%%%	name={($\obB$)},
%%%	plural={\orB},% im Mathematikmodus
%%%	description={%
%%%		Beseitigung von \symqt{$\lor$}%
%%%	}
%%%}
%%%\newcommand*{\tagorE}{$\lor$E}% Argument für \tag - im Textmodus
%%%\newcommand*{\orE}{(\lor\text{E})}% im Mathematikmodus
%%%\newglossaryentry{orE}{
%%%	name={($\obE$)},
%%%	plural={\orE},% im Mathematikmodus
%%%	description={%
%%%		Beseitigung von \symqt{$\lor$}%
%%%	}
%%%}

% Fachbegriffe -----------------------------------------------------------------

\newglossaryentry{ableitbar}{
	name={ableitbar},
	plural={ableitbare},
	description={%
		Wenn sich eine Formel $\beta$ aus einer Formel $\alpha$ mittels zulässiger Transaktionen ableiten lässt, heißt $\beta$ ableitbar aus $\alpha$.
		Sprechweise: \forqt{$\alpha$ ableitbar $\beta$}.
		Eine oder beide Formeln $\alpha$ \textbzw\ $\beta$ dürfen dabei durch Formelmengen ersetzt werden.
		-- siehe \emph{Ableitungsrelation} und \symqt{$\derive$}
		\newline
		Synonym: \emph{beweisbar}%
	}
}
\newglossaryentry{Ableitungsrelation}{
	name={Ableitungsrelation},
	plural={Ableitungsrelationen},
	description={%
		Die Relation \symqt{$\derive$}%
	}
}
\newglossaryentry{Abtrennungsregel}{
	name={Abtrennungsregel},
	plural={Abtrennungsregeln},
	description={%
		Eine \emph{Schlussregel} -- siehe~\emph{TR}%
	}
}
\newglossaryentry{allgemeingueltigeSchlussregelV}{
	name={allgemeingültige Schlussregel},    % eine  ...
	plural={allgemeingültige Schlussregeln}, % viele ..
	description={%
		Eine \emph{Schlussregel} die aus den \emph{Basisregeln} und den schon bekannten allgemeingültigen Schlussregeln abgeleitet werden kann%
	}
}
\newglossaryentry{Anfangsregel}{
	name={Anfangsregel},
	plural={Anfangsregeln},
	description={%
		Eine \emph{Schlussregel} um beginnen zu können -- siehe~\emph{AR}%
	}
}
\newglossaryentry{atomareFormelA}{
	name={atomare Formel},     % eine ...
	plural={atomaren Formeln}, % alle ...
	description={%
		Eine Formel, die sich nicht weiter zerlegen lässt%
	}%
}
\newglossaryentry{Aussage}{
	name={Aussage},
	plural={Aussagen},
	description={%
		Eine Aussage in natürlicher Sprache oder als Formel, die einen \emph{Wahrheitswert} liefert%
	}
}
\newglossaryentry{Aussagenlogik}{
	name={Aussagenlogik},
	description={%
		\vrefseesec{sec:Aussagenlogik}%
	}
}
\newglossaryentry{Basisregel}{
	name={Basisregel},
	plural={Basisregeln},
	description={%
		Eine \emph{Schlussregel}, die nicht mehr auf andere zurückgeführt wird.
		Obwohl das auch auf die \emph{Identitätsregeln} zutrifft, werden diese hier aber nicht dazu gezählt%
	}
}
\newglossaryentry{beweisbar}{
	name={beweisbar},
	plural={beweisbare},
	description={%
		Synonym zu \emph{ableitbar}%
	}
}
\newglossaryentry{Beweisschritt}{
	name={Beweisschritt},
	plural={Beweisschritte},
	description={%
		Eine Vorschrift, wie aus vorgegebenen Aussagen eine weitere folgt%
	}
}
\newglossaryentry{BoolscheSignatur}{
	name={Boolsche Signatur},
	plural={Boolschen Signatur},% Dativ
	description={%
		Die \emph{logische Signatur} $\{\lnot, \land, \lor\}$%
	}
}
\newglossaryentry{Folgerung}{
	name={Folgerung},
	plural={Folgerungen},
	description={%
		Die Folgerungen einer \emph{Schlussregel} sind die \emph{Aussagen} über ihrem Querstrich.
	}
}
\newglossaryentry{formalerSatzV}{
	name={formaler Satz},   % ein   ...
	plural={formale Sätze}, % viele ...
	description={%
		Formale Darstellung eines mathematischen Satzes -- siehe~\emph{FS}%
	}
}
\newglossaryentry{formalesElementV}{
	name={formales Element},   % ein   ...
	plural={formale Elemente}, % viele ...
	description={%
		Ein mathematisches Element in formaler Schreibweise.
		Bis auf wenige Aussagen kommen darin \emph{Metaausdrücke} nicht mehr vor%
	}%
}
\newglossaryentry{Identitaetsregel}{
	name={Identitätsregel},
	plural={Identitätsregeln},
	description={%
		Eigentlich eine \emph{Basisregel} zur Identität.
		Da die Identitätsregeln nur zur Rechtfertigung der \emph{Substitution} verwendet werden, werden sie hier nicht zu den \emph{Basisregeln} gezählt%
	}
}
\newglossaryentry{intEigenschaftA}{
	name={interessierende Eigenschaft},      % eine ...
	plural={interessierenden Eigenschaften}, % alle ...
	description={%
		Solche Eigenschaften von Ausdrücken, die im aktuellen Zusammenhang von Interesse sind%
	}%
}
\newglossaryentry{Kontraposition}{
	name={Kontraposition},
	plural={Kontraposition},
	description={%
		Die allgemeingültige Aussage $(\alpha \limp \beta) \limp (\lnot\beta \limp \lnot\alpha)$%
	}%
}
\newglossaryentry{logischeSignaturV}{
	name={logische Signatur},     % eine  ...
	plural={logische Signaturen}, % viele ...
	description={%
		Eine in \emph{Metasprache} verfasste Aussage, die auch zusammengesetzt sein kann%
	}%
}
\newglossaryentry{MetaausdruckV}{
	name={metasprachlicher Ausdruck},   % ein   ...
	plural={metasprachliche Ausdrücke}, % viele ...
	description={%
		Eine in normaler Sprache verfasste Aussage, die auch zusammengesetzt sein kann%
	}%
}
\newglossaryentry{MetaaussageV}{
	name={metasprachliche Aussage},    % eine   ...
	plural={metasprachliche Aussagen}, % viele ...
	description={%
		Eine in \emph{Metasprache} verfasste Aussage, die auch zusammengesetzt sein kann%
	}%
}
\newglossaryentry{MetaoperatorV}{
	name={metasprachlicher Operator},    % ein   ...
	plural={metasprachliche Operatoren}, % viele ...
	description={%
		Ein Operator, dessen Operanden \emph{Metaausdrücke} sind%
	}%
}
\newglossaryentry{Metasprache}{
	name={Metasprache},
	plural={Metasprachen},
	description={%
		Eine Sprache, in der Aussagen über Elemente einer anderen Sprache getroffen werden können%
	}%
}
\newglossaryentry{Monotonieregel}{
	name={Monotonieregel},
	plural={Monotonieregeln},
	description={%
		Eine \emph{Schlussregel} -- siehe~\emph{MR}%
	}
}
\newglossaryentry{Praedikat}{
	name={Prädikat},
	plural={Prädikate},
	description={%
		Ein Element der \emph{Prädikatenlogik} (\vrefseesec{sec:Praedikatenlogik}).
		\textZB\ kann man eine \forqt{$Gruppe$} als ein zweistelliges Prädikat \forqt{$Gruppe(G,+)$} definieren, in dem $G$ eine Menge und \symqt{$+$} eine Operation, \textdh\ eine (zweistellige) Funktion \forqt{$+: G \times G \rightarrow G$} ist, so dass die Gruppenaxiome erfüllt sind%
	}
}
\newglossaryentry{Praedikatenlogik}{
	name={Prädikatenlogik},
	description={%
		\vrefseesec{sec:Praedikatenlogik}%
	}
}
\newglossaryentry{Schlussregel}{
	name={Schlussregel},
	plural={Schlussregeln},
	description={%
		Eine Regel für eine (zulässige) Umwandlung von Formeln%
	}
}
\newglossaryentry{Schnittregel}{
	name={Schnittregel},
	plural={Schnittregeln},
	description={%
		Eine \emph{allgemeingültige Schlussregel} -- siehe~\emph{SR}%
	}
}
\newglossaryentry{Substitution}{
	name={Substitution},
	plural={Substitutionen},
	description={%
		Die Ersetzung von einem, mehreren oder allen \emph{formalen Elementen} ($\alpha$) in einem anderen \emph{formalen Element} ($\gamma$) durch ein drittes \emph{formales Element} ($\beta$)
		-- formal: $\gamma(\alpha\subst\beta)$.
		Wenn alle $\alpha$ in $\gamma$ durch $\beta$ ersetzt werden, ist die Substitution \emph{vollständig}.
		(\vrefseesub{sub:Indentitaetsregeln})%
	}
}
\newglossaryentry{vergleichbar}{
	name={vergleichbar},
	plural={vergleichbare},
	description={
		Zwei \emph{Metaausdrücke} \textbzw\ \emph{formale Elemente} heißen -- auf eine bestimmte Art -- \emph{vergleichbar}, wenn sie auf diese Art (\textzB\ als Zeichenketten oder als vergleichbare Ergebnisse von Formeln) verglichen werden können.
		Die Art muss implizit bekannt oder explizit angegeben sein.
		Meistens genügt es zu wissen, was für \emph{Metaausdrücke} \textbzw\ \emph{formale Elemente} es sind.
		Sie müssen dann nur von derselben Art sein%
	}
}
\newglossaryentry{Vertauschung}{
	name={Vertauschung},
	plural={Vertauschungen},
	description={%
		Die Vertauschung von zwei unabhängigen \emph{formalen Elementen} ($\alpha$ und $\beta$) in einem anderen \emph{formalen Element} ($\gamma$)
		-- formal: $\gamma(\alpha\swap\beta)$.
		Die Vertauschung ist eine spezielle Form der \emph{Substitution}.
		-- siehe die Definition~\eqref{def:Vertauschung} \vrefinsub{sub:Indentitaetsregeln}%
	}
}
\newglossaryentry{Voraussetzung}{
	name={Voraussetzung},
	plural={Voraussetzungen},
	description={%
		Die Voraussetzungen einer \emph{Schlussregel} sind die \emph{Aussagen} über ihrem Querstrich.
	}
}
\newglossaryentry{Wahrheitswert}{
	name={Wahrheitswert},
	plural={Wahrheitswerte},
	description={%
		Wahrheitswerte sind die Werte \strqt{wahr} und \strqt{falsch}, oft auch als \strqt{true} und \strqt{false} oder einfach \charqt{1} und \charqt{0} bezeichnet%
	}
}
\newglossaryentry{zulaessigeTransformationA}{
	name={zulässige Transformation},      % eine ...
	plural={zulässigen Transformationen}, % alle ...
	description={%
		Eine zulässige Umformung oder Erzeugung einer Formel aus einer vorgegebenen Menge von Formeln,
		\textdh\ die Anwendung einer \emph{allgemeingültigen Schlussregel}%
	}
}

%%############################################################################%%
%%                                                                            %%
%% Datei:  ASBA-Vorspann-Glossary.tex                                         %%
%% Inhalt: Vorspann Glossareinträge für ASBA                                  %%
%%                                                                            %%
%% Copyright (C) 2017  Winfried Teschers                                      %%
%%                                                                            %%
%% This program is free software: you can redistribute it and/or modify       %%
%% it under the terms of the GNU Affero General Public License as published   %%
%% by the Free Software Foundation, either version 3 of the License, or       %%
%% (at your option) any later version.                                        %%
%%                                                                            %%
%% This program is distributed in the hope that it will be useful,            %%
%% but WITHOUT ANY WARRANTY; without even the implied warranty of             %%
%% MERCHANTABILITY or FITNESS FOR A PARTICULAR PURPOSE.  See the              %%
%% GNU Affero General Public License for more details.                        %%
%%                                                                            %%
%% You should have received a copy of the GNU Affero General Public License   %%
%% along with this program.  If not, see <http://www.gnu.org/licenses/>.      %%
%%                                                                            %%
%% Dr. Winfried Teschers                                                      %%
%% Anton-Günther-Straße 26c                                                   %%
%% 91083 Baiersdorf                                                           %%
%% Germany                                                                    %%
%%                                                                            %%
%% e-mail: winfried.teschers@t-online.de                                      %%
%%                                                                            %%
%%############################################################################%%

% !TeX root = ASBA.tex
% !TeX encoding = UTF-8
% !TeX spellcheck = de_DE

% Elemente, die keine Glossareinträge sind, werden in "ASBA-Vorspann.tex" und "ASBA-Mathematik-Vorspann.tex" definiert.
%TODO Symbole in  Glossar und Index eintragen

% Symbole für Mengen -----------------------------------------------------------

\newglossaryentry{gsN}{
	name      ={$\gsN$},
	plural     ={\gsN},% im Mathematikmodus
	description ={Die Menge der natürlichen Zahlen ohne 0.}
}
\newglossaryentry{gsNo}{
	name      ={$\gsNo$},
	%TODO falscher Zeichensatz?
	plural     ={\gsNo},% im Mathematikmodus
	description ={Die Menge der natürlichen Zahlen einschließlich 0.}
}
\newglossaryentry{alABC}{
	name      ={$\alABC$},
	plural     ={\alABC},% im Mathematikmodus
	description ={Das Alphabet der aussagenlogischen Sprache.}
}
\newglossaryentry{alABCx}{
	name      ={$\alABC_x$},
	plural     ={\alABC_x},% im Mathematikmodus
	description ={
		Eine Teilmenge des Alphabets $\alABC$ der aussagenlogischen Sprache.
	}
}
\newglossaryentry{alBin}{
	name      ={$\alBin$},
	plural     ={\alBin},% im Mathematikmodus
	description ={Die Menge der aussagenlogischen, binären Operatoren.}
}
\newglossaryentry{alCon}{
	name      ={$\alCon$},
	plural     ={\alCon},% im Mathematikmodus
	description ={Die Menge der aussagenlogischen Konstanten.}
}
\newglossaryentry{alFor}{
	name      ={$\alFor$},
	plural     ={\alFor},% im Mathematikmodus
	description ={Die Menge der aussagenlogischen \emph{Formeln} mit Klammerung.}
}
\newglossaryentry{alForp}{
	name      ={$\alForp$},
	plural     ={\alForp},% im Mathematikmodus
	description ={
		Die Menge der aussagenlogischen \emph{Formeln} in polnischer Notation.
	}
}
\newglossaryentry{alForx}{
	name      ={$\alFor_x$},
	plural     ={\alFor_x},% im Mathematikmodus
	description ={
		Eine Teilmenge der Menge $\alFor$ der aussagenlogischen \emph{Formeln} mit Klammerung.
	}
}
\newglossaryentry{alForxp}{
	name      ={$\alForp_x$},
	plural     ={\alForp_x},% im Mathematikmodus
	description ={
		Eine Teilmenge der Menge $\alFor$ der aussagenlogischen \emph{Formeln} in polnischer Notation.
	}
}
\newglossaryentry{alJun}{
	name      ={$\alJun$},
	plural     ={\alJun},% im Mathematikmodus
	description ={Die Menge der aussagenlogischen \emph{Junktoren}.}
}
\newglossaryentry{alJunx}{
	name      ={$\alJun_x$},
	plural     ={\alJun_x},% im Mathematikmodus
	description ={
		Eine Teilmenge der Menge $\alJun$ der aussagenlogischen Operatoren.
	}
}
\newglossaryentry{alUna}{
	name      ={$\alUna$},
	plural     ={\alUna},% im Mathematikmodus
	description ={Die Menge der aussagenlogischen unären Operatoren.}
}
\newglossaryentry{alVar}{
	name      ={$\alVar$},
	plural     ={\alVar},% im Mathematikmodus
	description ={Die Menge der aussagenlogischen Variablen.}
}

% Symbole für Beispieloperatoren -----------------------------------------------

\newglossaryentry{lrelbsp}{
	name      ={$\lrelbsp$},
	plural     ={\lrelbsp},% im Mathematikmodus
	description ={
		Ein Beispielsymbol für eine Relation mit Umkehrrelation $\rrelbsp$.
	}
}
\newglossaryentry{lreleqbsp}{
	name      ={$\lreleqbsp$},
	plural     ={\lreleqbsp},% im Mathematikmodus
	description ={
		Ein Beispielsymbol für eine Relation mit Gleichheit und Umkehrrelation $\rreleqbsp$.
	}
}
\newglossaryentry{relbsp}{
	name      ={$\relbsp$},
	plural     ={\relbsp},% im Mathematikmodus
	description ={Ein Beispielsymbol für eine Relation.}
}
\newglossaryentry{releqbsp}{
	name      ={$\releqbsp$},
	plural     ={\releqbsp},% im Mathematikmodus
	description ={Ein Beispielsymbol für eine Relation mit Gleichheit.}
}
\newglossaryentry{relnbsp}{
	name      ={$\relnbsp$},
	plural     ={\relnbsp},% im Mathematikmodus
	description ={Verneinung von $\relbsp$.}
}
\newglossaryentry{rrelbsp}{
	name      ={$\rrelbsp$},
	plural     ={\rrelbsp},% im Mathematikmodus
	description ={
		Ein Beispielsymbol für eine Relation mit Umkehrrelation $\lrelbsp$.
	}
}
\newglossaryentry{rreleqbsp}{
	name      ={$\rreleqbsp$},
	plural     ={\rreleqbsp},% im Mathematikmodus
	description ={
		Ein Beispielsymbol für eine Relation mit Gleichheit und Umkehrrelation $\lreleqbsp$.
	}
}

% Meta-Symbole -----------------------------------------------------------------

\newglossaryentry{defeq}{
	name      ={$:=$},
	plural      ={:=},% im Mathematikmodus
	description ={Definition: \textdots\ definitionsgemäß gleich \textdots}
}
\newglossaryentry{derive}{
	name      ={$\derivesym$},
	plural     ={\derive},% im Mathematikmodus
	description ={
		\emph{Ableitungsrelation}: \textdots\ ableitbar \textdots
		-- siehe \emph{ableitbar}.
	}
}
\newglossaryentry{eq}{
	name      ={$\eq$},
	plural     ={\eq},% im Mathematikmodus
	description ={
		Eine \emph{Metarelation}: \textdots\ gleich (ist dasselbe wie, ist identisch zu) \textdots
	}
}
\newglossaryentry{equiv}{
	name      ={$\equiv$},
	plural     ={\equiv},% im Mathematikmodus
	description ={
		Eine \emph{Metarelation}: \textdots\ äquivalent zu (ist das gleiche wie, ist so wie) \textdots
	}
}
\newglossaryentry{metaand}{
	name      ={$\metaandsym$},
	plural     ={\metaand},% im Mathematikmodus
	description ={Ein \emph{Metaoperator}: \textdots\ und \textdots}
}
\newglossaryentry{metadefeq}{
	name      ={$\metadefeq$},
	plural     ={\metadefeq},% im Mathematikmodus
	description ={
		\emph{Metadefinition}: \textdots\ definitionsgemäß gleich (definitionsgemäß genau dann, wenn) \textdots
	}
}
\newglossaryentry{metaequiv}{
	name      ={$\metaequiv$},
	plural     ={\metaequiv},% im Mathematikmodus
	description ={Eine \emph{Metarelation}: \textdots\ genau dann wenn \textdots}
}
\newglossaryentry{metaimp}{
	name      ={$\metaimp$},
	plural     ={\metaimp},% im Mathematikmodus
	description ={Eine \emph{Metarelation}: \textdots\ dann auch \textdots}
}
\newglossaryentry{metaor}{
	name      ={$\metaorsym$},
	plural     ={\metaor},% im Mathematikmodus
	description ={Ein \emph{Metaoperator}: \textdots\ oder \textdots}
}
\newglossaryentry{metarep}{
	name      ={$\metarep$},
	plural     ={\metarep},% im Mathematikmodus
	description ={Eine \emph{Metarelation}: \textdots\ sofern \textdots}
}
\newglossaryentry{ne}{
	name      ={$\ne$},
	plural     ={\ne},% im Mathematikmodus
	description ={
		Eine (Meta-)Operator: \textdots\ ungleich (nicht dasselbe wie, nicht identisch zu) \textdots
	}
}
\newglossaryentry{nequiv}{
	name      ={$\nequiv$},
	plural     ={\nequiv},% im Mathematikmodus
	description ={
		Eine \emph{Metarelation}: \textdots\ nicht äquivalent (ist nicht das gleiche wie, ist nicht so wie) \textdots
	}
}
\newglossaryentry{subst}{
	name      ={$\subst$},
	plural     ={\subst},% im Mathematikmodus
	description ={
		\emph{Substitution}: \textdots\ substituiert durch \textdots\
		-- siehe die Definition \vrefinsub{sub:Identitätsregeln}.
	}
}
\newglossaryentry{swap}{
	name      ={$\swap$},
	plural     ={\swap},% im Mathematikmodus
	description ={
		\emph{Vertauschung}: \textdots\ vertauscht mit \textdots\
		-- siehe die Definition \vrefinsub{sub:Identitätsregeln}.
	}
}
\newglossaryentry{srand}{
	name      ={$\srand$},
	plural     ={\srand},% im Mathematikmodus
	description ={
		Ein \emph{Metaoperator}: \textdots\ und \textdots\
		-- wird nur bei den \emph{Schlussregeln} verwendet.
	}
}

% sonstige mathematische Symbole -----------------------------------------------

\newglossaryentry{lfalse}{
	name      ={$\lfalse$},
	plural     ={\lfalse},% im Mathematikmodus
	description ={Eine aussagenlogische Konstante (\emph{Wahrheitswert}): Falsch.}
}
\newglossaryentry{ltrue}{
	name      ={$\ltrue$},
	plural     ={\ltrue},% im Mathematikmodus
	description ={Eine aussagenlogische Konstante (\emph{Wahrheitswert}): Wahr.}
}

% Schlussregeln ----------------------------------------------------------------

\newcommand* {\tagAR}{AR}% Argument für \tag - im Textmodus
\newcommand*    {\AR}{(\text{AR})}% im Mathematikmodus
\newglossaryentry{AR}{
	name      ={$\AR$},
	plural     ={\AR},% im Mathematikmodus
	description ={\emph{Anfangsregel}}
}
\newcommand* {\tagFS}{FS}% Argument für \tag - im Textmodus
\newcommand*    {\FS}{(\text{FS})}% im Mathematikmodus
\newglossaryentry{FS}{
	name      ={$\FS$},
	plural     ={\FS},% im Mathematikmodus
	description ={\emph{formaler Satz}}
}
\newcommand* {\tagMR}{MR}% Argument für \tag - im Textmodus
\newcommand*    {\MR}{(\text{MR})}% im Mathematikmodus
\newglossaryentry{MR}{
	name      ={$\MR$},
	plural     ={\MR},% im Mathematikmodus
	description ={\emph{Monotonieregel}}
}
\newcommand* {\tagSR}{SR}% Argument für \tag - im Textmodus
\newcommand*    {\SR}{(\text{SR})}% im Mathematikmodus
\newglossaryentry{SR}{
	name      ={$\SR$},
	plural     ={\SR},% im Mathematikmodus
	description ={\emph{Schnittregel} (Modus ponens)}
}
\newcommand* {\tagTR}{TR}% Argument für \tag - im Textmodus
\newcommand*    {\TR}{(\text{TR})}% im Mathematikmodus
\newglossaryentry{TR}{
	name      ={$\TR$},
	plural     ={\TR},% im Mathematikmodus
	description ={\emph{Abtrennungsregel}}
}
\newcommand* {\tageqB}{$\eq$B}% Argument für \tag - im Textmodus
\newcommand*    {\eqB}{(\eq\text{B})}% im Mathematikmodus
\newglossaryentry{eqB}{
	name      ={$\eqB$},
	plural     ={\eqB},% im Mathematikmodus
	description ={Beseitigung von \chrqt{$\eq$}}
}
\newcommand* {\tageqE}{$\eq$E}% Argument für \tag - im Textmodus
\newcommand*    {\eqE}{(\eq\text{E})}% im Mathematikmodus
\newglossaryentry{eqE}{
	name      ={$\eqE$},
	plural     ={\eqE},% im Mathematikmodus
	description ={Einführung von \chrqt{$\eq$}}
}
\newcommand* {\tagandB}{$\land$B}% Argument für \tag - im Textmodus
\newcommand*    {\andB}{(\land\text{B})}% im Mathematikmodus
\newglossaryentry{andB}{
	name      ={$\andB$},
	plural     ={\andB},% im Mathematikmodus
	description ={Beseitigung von \chrqt{$\land$}}
}
\newcommand* {\tagandE}{$\land$E}% Argument für \tag - im Textmodus
\newcommand*    {\andE}{(\land\text{E})}% im Mathematikmodus
\newglossaryentry{andE}{
	name      ={$\andE$},
	plural     ={\andE},% im Mathematikmodus
	description ={Einführung von \chrqt{$\land$}}
}
\newcommand* {\tagimpB}{$\limp$B}% Argument für \tag - im Textmodus
\newcommand*    {\impB}{(\limp\text{B})}% im Mathematikmodus
\newglossaryentry{impB}{
	name      ={$\impB$},
	plural     ={\impB},% im Mathematikmodus
	description ={Beseitigung von \chrqt{$\limp$}}
}
\newcommand* {\tagimpE}{$\limp$E}% Argument für \tag - im Textmodus
\newcommand*    {\impE}{(\limp\text{E})}% im Mathematikmodus
\newglossaryentry{impE}{
	name      ={$\impE$},
	plural     ={\impE},% im Mathematikmodus
	description ={Einführung von \chrqt{$\limp$}}
}
\newcommand* {\tagnota}{$\lnot$1}% Argument für \tag - im Textmodus
\newcommand*    {\nota}{(\lnot\text{1})}% im Mathematikmodus
\newglossaryentry{nota}{
	name={ $\nota$},
	plural     ={\nota},% im Mathematikmodus
	description ={Einführung/Beseitigung von \chrqt{$\lnot$} Teil 1}
}
\newcommand* {\tagnotb}{$\lnot$2}% Argument für \tag - im Textmodus
\newcommand*    {\notb}{(\lnot\text{2})}% im Mathematikmodus
\newglossaryentry{notb}{
	name      ={$\notb$},
	plural     ={\notb},% im Mathematikmodus
	description ={Einführung/Beseitigung von \chrqt{$\lnot$} Teil 2}
}
\newcommand* {\tagnotc}{$\lnot$3}% Argument für \tag - im Textmodus
\newcommand*    {\notc}{(\lnot\text{3})}% im Mathematikmodus
\newglossaryentry{notc}{
	name      ={$\notc$},
	plural     ={\notc},% im Mathematikmodus
	description ={Beweistechnik \enquote{Indirekter \emph{Beweis}}}
}
\newcommand* {\tagnotd}{$\lnot$4}% Argument für \tag - im Textmodus
\newcommand*    {\notd}{(\lnot\text{4})}% im Mathematikmodus
\newglossaryentry{notd}{% statt "notE"
	name      ={$\notd$},
	plural     ={\notd},% im Mathematikmodus
	description ={Reductio ad absurdum (Indirekter \emph{Beweis})}
}
%%%\newcommand* {\tagorB}{$\lor$B}% Argument für \tag - im Textmodus
%%%\newcommand*    {\orB}{(\lor\text{B})}% im Mathematikmodus
%%%\newglossaryentry{orB}{
%%%	name      ={$\obB$},
%%%	plural     ={\orB},% im Mathematikmodus
%%%	description ={Beseitigung von \chrqt{$\lor$}}
%%%}
%%%\newcommand* {\tagorE}{$\lor$E}% Argument für \tag - im Textmodus
%%%\newcommand*    {\orE}{(\lor\text{E})}% im Mathematikmodus
%%%\newglossaryentry{orE}{
%%%	name      ={$\obE$},
%%%	plural     ={\orE},% im Mathematikmodus
%%%	description ={Beseitigung von \chrqt{$\lor$}}
%%%}

% Fachbegriffe #################################################################
%TODO Alle Fachbegriffe in Glossar und Index eintragen

%A === A === A === A === A === A === A === A === A === A === A === A === A === A

\newcommand*{\ASBA}{\glsIdx{ASBA}}
\newacronym{ASBA}{ASBA}{
	Programmsystem, das \textbf{A}xiome, \textbf{S}ätze, \textbf{B}eweise und \textbf{A}uswertungen behandeln kann.
}
\newcommand*    {\ableitbar} {\glsIdx  {ableitbar}}
\newcommand*    {\ableitbare}{\glsIdxPl{ableitbar}}
\newglossaryentry{ableitbar}{
	name        ={ableitbar},
	plural      ={ableitbare},
	description ={
		Wenn sich eine \emph{Formel} $\beta$ aus einer anderen \emph{Formel} $\alpha$ mittels zulässiger Transaktionen ableiten lässt, heißt $\beta$ \emph{ableitbar} aus $\alpha$.
		Sprechweise: \seqqt{$ \alpha \text{ableitbar} \beta $}.
		Eine oder beide \emph{Formeln} $\alpha$ \textbzw\ $\beta$ dürfen dabei durch \emph{Formelmengen} ersetzt werden.
		-- siehe \emph{Ableitungsrelation} und \chrqt{$\derivesym$}.
		\newline
		Synonym: \emph{beweisbar}.%
	}
}
\newcommand*    {\Ableitungsrelation}{\glsIdx{Ableitungsrelation}}
\newglossaryentry{Ableitungsrelation}{
	name        ={Ableitungsrelation},
	description ={Die Relation \chrqt{$\derivesym$}.}
}
\newcommand*    {\Abtrennungsregel}{\glsIdx{Abtrennungsregel}}
\newglossaryentry{Abtrennungsregel}{
	name        ={Abtrennungsregel},
	description ={Eine \emph{Schlussregel} -- siehe~\emph{TR}}
}
\newcommand*    {\allgemeingueltigeSchlussregel}  {\glsIdx  {allgemeingueltige-Schlussregel}}
\newcommand*    {\allgemeingueltigenSchlussregel} {\glsIdxBg{allgemeingueltige-Schlussregel}{allgemeingültigen Schlussregel}}
\newcommand*    {\allgemeingueltigeSchlussregeln} {\glsIdxPl{allgemeingueltige-Schlussregel}}
\newcommand*    {\allgemeingueltigenSchlussregeln}{\glsIdxBg{allgemeingueltige-Schlussregel}{allgemeingültigen Schlussregeln}}
\newglossaryentry{allgemeingueltige-Schlussregel}{
	name        ={allgemeingültige Schlussregel},
	plural      ={allgemeingültige Schlussregeln},
	description ={
		Eine \emph{Schlussregel} die aus den \emph{Basisregeln} und den schon bekannten \emph{allgemeingültigen Schlussregeln} abgeleitet werden kann.
	}
}
\newcommand*    {\Anfangsregel}{\glsIdx{Anfangsregel}}
\newglossaryentry{Anfangsregel}{
	name        ={Anfangsregel},
	description ={
		Eine \emph{Schlussregel} um beginnen zu können -- siehe~\emph{AR}.}
}
\newcommand*    {\atomareFormel} {\glsIdx  {atomare-Formel}}
\newcommand*    {\atomareFormeln}{\glsIdxPl{atomare-Formel}}
\newglossaryentry{atomare-Formel}{
	name        ={atomare Formel},
	plural      ={atomare Formeln},
	description ={Eine \emph{Formel}, die sich nicht weiter zerlegen lässt.}
}
\newcommand*    {\Ausgabeschema}  {\glsIdx  {Ausgabeschema}}
\newcommand*    {\Ausgabeschemata}{\glsIdxPl{Ausgabeschema}}
\newglossaryentry{Ausgabeschema}{
	name        ={Ausgabeschema},
	plural      ={Ausgabeschemata},
	description ={
		Ein Schema, mit dem bestimmte mathematische \emph{Objekte} ausgegeben werden sollen.
	}
}
\newcommand*    {\Aussage} {\glsIdx  {Aussage}}
\newcommand*    {\Aussagen}{\glsIdxPl{Aussage}}
\newglossaryentry{Aussage}{
	name        ={Aussage},
	plural      ={Aussagen},
	description ={
		Eine \emph{Aussage} in natürlicher Sprache oder als \emph{Formel}, die einen \emph{Wahrheitswert} liefert.
	}
}
\newcommand*    {\Aussagenlogik}{\glsIdx{Aussagenlogik}}
\newglossaryentry{Aussagenlogik}{
	name        ={Aussagenlogik},
	description ={-- \vrefseesec{sec:Aussagenlogik}.}
}
\newcommand*    {\Axiom}  {\glsIdx  {Axiom}}
\newcommand*    {\Axiome} {\glsIdxPl{Axiom}}
\newcommand*    {\Axiomen}{\glsIdxBg{Axiom}{Axiomen}}
\newglossaryentry{Axiom}{
	name        ={Axiom},
	plural      ={Axiome},
	description ={Eine Formel, die unbewiesen als wahr angesehen wird.}
}
\newcommand*    {\Axiomensystem} {\glsIdx  {Axiomensystem}}
\newcommand*    {\Axiomensysteme}{\glsIdxPl{Axiomensystem}}
\newglossaryentry{Axiomensystem}{
	name        ={Axiomensystem},
	plural      ={Axiomensysteme},
	description ={Eine Menge von \emph{Axiomen}.}
}

%B === B === B === B === B === B === B === B === B === B === B === B === B === B

\newcommand*    {\Basisregel} {\glsIdx  {Basisregel}}
\newcommand*    {\Basisregeln}{\glsIdxPl{Basisregel}}
\newglossaryentry{Basisregel}{
	name        ={Basisregel},
	plural      ={Basisregeln},
	description ={
		Eine \emph{Schlussregel}, die nicht mehr auf andere zurückgeführt wird.
		Obwohl das auch auf die \emph{Identitätsregeln} zutrifft, werden diese hier aber nicht dazu gezählt.
	}
}
\newcommand*    {\Beweis}  {\glsIdx  {Beweis}}
\newcommand*    {\Beweises}{\glsIdxBg{Beweis}{Beweises}}
\newcommand*    {\Beweise} {\glsIdxPl{Beweis}}
\newcommand*    {\Beweisen}{\glsIdxBg{Beweis}{Beweisen}}
\newglossaryentry{Beweis}{
	name        ={Beweis},
	plural      ={Beweise},
	description ={
		Eine zulässige Ableitung von Folgerungen aus gegebenen Voraussetzungen.
		-- 	\newline\vrefseesec{sec:BeweiseASBA}.
	}
}
%TODO === hier weitermachen
\newcommand*    {\beweisbar} {\glsIdx  {beweisbar}}
\newcommand*    {\beweisbare}{\glsIdxPl{beweisbar}}
\newglossaryentry{beweisbar}{
	name        ={beweisbar},
	plural      ={beweisbare},
	description ={Synonym zu \emph{ableitbar}.}
}
\newcommand*    {\Beweisschritt}  {\glsIdx  {Beweisschritt}}
\newcommand*    {\Beweisschritte} {\glsIdxPl{Beweisschritt}}
\newcommand*    {\Beweisschritten}{\glsIdxBg{Beweisschritt}{Beweisschritten}}
\newglossaryentry{Beweisschritt}{
	name        ={Beweisschritt},
	plural      ={Beweisschritte},
	description ={
		Eine Vorschrift, wie aus vorgegebenen \emph{Aussagen} (den \emph{Voraussetzungen}) eine weitere (die \emph{Folgerung}) folgt.
	}
}
\newglossaryentry{Beweisschrittfolge}{
	name        ={Beweisschrittfolge},
	plural      ={Beweisschrittfolgen},
	description ={Eine Folge von \emph{Beweisschritten}.}
}
\newglossaryentry{Beweisschrittmenge}{
	name        ={Beweisschrittmenge},
	plural      ={Beweisschrittmengen},
	description ={
		Die Menge der \emph{Beweisschritte}, \textdh\ der Glieder der \emph{Beweisschrittfolge} eines \emph{Beweises}.
	}
}
\newglossaryentry{Boolsche-Signatur}{
	name        ={Boolsche Signatur},
	plural      ={Boolsche Signaturen},
	description ={Die \emph{logische Signatur} $\{\lnot, \land, \lor\}$.}
}

%F === F === F === F === F === F === F === F === F === F === F === F === F === F

\newcommand*    {\Fachbegriff}  {\glsIdx  {Fachbegriff}}
\newcommand*    {\Fachbegriffe} {\glsIdxPl{Fachbegriff}}
\newcommand*    {\Fachbegriffen}{\glsIdxBg{Fachbegriff}{Fachbegriffen}}
\newglossaryentry{Fachbegriff}{
	name        ={Fachbegriff},
	plural      ={Fachbegriffe},
	description ={Ein Name für einen mathematischen Begriff.}
}
\newcommand*    {\Fachgebiet}  {\glsIdx  {Fachgebiet}}
\newcommand*    {\Fachgebiete} {\glsIdxPl{Fachgebiet}}
\newcommand*    {\Fachgebieten}{\glsIdxBg{Fachgebiet}{Fachgebieten}}
\newglossaryentry{Fachgebiet}{
	name        ={Fachgebiet},
	plural      ={Fachgebiete},
	description ={
		Ein Teil der Mathematik mit einer zugehörigen Basis aus \Axiomen, Sätzen und spezifischen \Fachbegriffen\ und Darstellungen.
	}
}
\newglossaryentry{Folgerung}{
	name        ={Folgerung},
	plural      ={Folgerungen},
	description ={
		Die Folgerungen einer \emph{Schlussregel} $\frac{\prerequisiteset}{\conclusionset}$ sind die Elemente von $\conclusionset$.
	}
}
%TODO Folgerungsmenge
\newglossaryentry{formaler-Satz}{
	name        ={formaler Satz},
	plural      ={formale  Sätze},
	description ={
		Formale Darstellung eines mathematischen Satzes. -- siehe~\emph{FS}.
	}
}
\newglossaryentry{Formel}{
	name        ={Formel},
	plural      ={Formeln},
	description ={
		Unter einer \emph{Formel} verstehen wir in diesem Dokument stets eine mathematische \emph{Formel}.
		Diese kann auch mehrdimensional sein, lässt sich aber mittels geeigneter Definitionen immer eindeutig als eine \emph{Zeichenfolge} schreiben.
		\emph{Schlussregeln} betrachten wir \emph{nicht} als \emph{Formeln}.
	}
}
\newglossaryentry{Formelmenge}{
	name        ={Formelmenge},
	plural      ={Formelmengen},
	description ={
		\textIAlg\ eine Menge von Formeln $\formula$ \textbzw\ Worten.
		Man nennt $\formulaset$ auch \emph{Sprache}.
	}
}

%G === G === G === G === G === G === G === G === G === G === G === G === G === G

\newglossaryentry{Gleichheitsrelation}{
	name        ={Gleichheitsrelation},
	plural      ={Gleichheitsrelationen},
	description ={
		Eine mit der Gleichheit verwandte Relation: $\eq$, $\ne$, $\equiv$ und $\nequiv$.
	}
}

%I === I === I === I === I === I === I === I === I === I === I === I === I === I

%TODO Identitätsregel nötig?
\newglossaryentry{Identitaetsregel}{
	name        ={Identitätsregel},
	plural      ={Identitätsregeln},
	description ={
		Eigentlich eine \emph{Basisregel} zur Identität.
		Da die \emph{Identitätsregeln} nur zur Rechtfertigung der \emph{Substitution} verwendet werden, werden sie hier nicht zu den \emph{Basisregeln} gezählt.
	}
}
\newglossaryentry{interessierende-Eigenschaft}{
	name        ={interessierende Eigenschaft},
	plural      ={interessierende Eigenschaften},
	description ={
		Solche Eigenschaften von \emph{Objekten}, die im aktuellen Zusammenhang von Interesse sind.
	}
}

%J === J === J === J === J === J === J === J === J === J === J === J === J === J

\newglossaryentry{Junktor}{
	name        ={Junktor},
	plural      ={Junktoren},
	description ={Ein Operatorsymbol, \textdh\ ein Symbol für einen Operator.}
}

%K === K === K === K === K === K === K === K === K === K === K === K === K === K

\newglossaryentry{Kontraposition}{
	name        ={Kontraposition},
	plural      ={Kontraposition},
	description ={
		Die allgemeingültige \emph{Aussage}: $ (\alpha \limp \beta) \limp (\lnot\beta \limp \lnot\alpha) $.
	}
}

%L === L === L === L === L === L === L === L === L === L === L === L === L === L

\newglossaryentry{logische-Signatur}{
	name        ={logische Signatur},
	plural      ={logische Signaturen},
	description ={
		Eine Teilmenge von $\alJun$, die ausreicht, alle anderen Elemente von $\alJun$ zu definieren.
	}
}

%M === M === M === M === M === M === M === M === M === M === M === M === M === M

\newglossaryentry{Mengenlehre}{
	name={Mengenlehre},
	description ={-- \vrefseesec{sec:Mengenlehre}.}
}
\newglossaryentry{Metaoperator}{
	name        ={Metaoperator},
	plural      ={Metaoperatoren},
	description ={
		Ein Operator der \emph{Metasprache}: $\metaandsym$, $\metaorsym$ und $\srand$.
	}
}
\newglossaryentry{Metarelation}{
	name        ={Metarelation},
	plural      ={Metarelationen},
	description ={
		Eine Relation der \emph{Metasprache}: $\metaimp$, $\metarep$ und $\metaequiv$.
	}
}
\newglossaryentry{Metasprache}{
	name        ={Metasprache},
	plural      ={Metasprachen},
	description ={
		Eine Sprache, in der \emph{Aussagen} über Elemente einer anderen Sprache getroffen werden können.
		In diesem Dokument ist dies immer die normale Sprache.
		-- \vrefseesec{sec:Metasprache}.
	}
}
\newglossaryentry{Monotonieregel}{
	name        ={Monotonieregel},
	plural      ={Monotonieregeln},
	description ={
		Eine \emph{Schlussregel}. -- siehe~\emph{MR}.
	}
}

%O === O === O === O === O === O === O === O === O === O === O === O === O === O

\newglossaryentry{Objekt}{
	name        ={Objekt},
	plural      ={Objekte},
	description ={
		Symbole, \emph{Formeln} und \emph{Aussagen} sowie Mengen, \emph{Zeichenfolgen}, Zahlen; ganz allgemein reale oder gedachte Dinge an sich.
	}
}

%P === P === P === P === P === P === P === P === P === P === P === P === P === P

\newglossaryentry{Praedikat}{
	name        ={Prädikat},
	plural      ={Prädikate},
	description ={
		Ein Element der \emph{Prädikatenlogik} -- \vrefseesec{sec:Prädikatenlogik}.\\
		\textZB\ kann man eine Gruppe als ein zweistelliges Prädikat $\mathrm{Gruppe}(G,+)$ definieren, in dem $G$ eine Menge und $+$ eine Operation, \textdh\ eine (zweistellige) Funktion $ +: G \times G \rightarrow G $ ist, so dass die Gruppenaxiome erfüllt sind.
	}
}
\newcommand*    {\Praedikatenlogik}{\glsIdx{Praedikatenlogik}}
\newglossaryentry{Praedikatenlogik}{
	name={Prädikatenlogik},
	description ={-- \vrefseesec{sec:Prädikatenlogik}.}
}

%S === S === S === S === S === S === S === S === S === S === S === S === S === S

\newcommand*    {\Satz}   {\glsIdx  {Satz}}
\newcommand*    {\Satzes} {\glsIdxBg{Satz}{Satzes}}
\newcommand*    {\Saetze} {\glsIdxPl{Satz}}
\newcommand*    {\Saetzen}{\glsIdxBg{Satz}{Sätzen}}
\newglossaryentry{Satz}{
	name        ={Satz},
	plural      ={Sätze},
	description ={
		Eine mathematische \emph{Aussage}, dass bestimmte Folgerungen aus gegebenen Voraussetzungen abgeleitet werden können.
	}
}
\newglossaryentry{Schlussregel}{
	name        ={Schlussregel},
	plural      ={Schlussregeln},
	description ={
		Eine \emph{Schlussregel} $\frac{\prerequisiteset}{\conclusionset}$ entspricht der \emph{Aussage}:
		Wenn alle \emph{Voraussetzungen} $\prerequisite$ aus $\prerequisiteset$ zutreffen, dann auch alle \emph{Folgerungen} $\conclusion$ aus $\conclusionset$.
		Wenn diese \emph{Aussage} zutrifft, kann die Schlussregel zur zulässigen Umwandlung von \emph{Formeln} dienen.
	}
}
\newglossaryentry{Schlussregelmenge}{
	name        ={Schlussregelmenge},
	plural      ={Schlussregelmengen},
	description ={
		Eine Menge von \emph{Schlussregeln}, meistens mit $\conclusionruleset$ bezeichnet.
	}
}
\newglossaryentry{Schnittregel}{
	name        ={Schnittregel},
	plural      ={Schnittregeln},
	description ={Eine \emph{allgemeingültige Schlussregel}. -- siehe~\emph{SR}}.
}
\newglossaryentry{Sprache}{
	name        ={Sprache},
	plural      ={Sprachen},
	description ={-- siehe \emph{Formelmenge}.}
}
\newglossaryentry{Substitution}{ %TODO ggf. überarbeiten
	name        ={Substitution},
	plural      ={Substitutionen},
	description ={
		Die Ersetzung von einem, mehreren oder allen \emph{formalen Elementen} ($\alpha$) in einem anderen \emph{formalen Element} ($\gamma$) durch ein drittes \emph{formales Element} ($\beta$)
		-- formal: $\gamma(\alpha\subst\beta)$.
		Wenn alle $\alpha$ in $\gamma$ durch $\beta$ ersetzt werden, ist die \emph{Substitution vollständig}.
		-- \vrefseesub{sub:Identitätsregeln}.
	}
}
%TODO Substitutionsmenge
\newglossaryentry{Symbol}{
	name        ={Symbol},
	plural      ={Symbole},
	description ={
		Ein einfaches Symbol ist ein druckbares typographisches Zeichen.
		Ein zusammengesetztes Symbol besteht aus mehreren einfachen Symbolen.
		In beiden Fällen wird ein Symbol als unzerlegbar angesehen.
		\vrefseesec{sec:Notationen}.
	}
}

%T === T === T === T === T === T === T === T === T === T === T === T === T === T

\newglossaryentry{Transformation}{
	name        ={Transformation},
	plural      ={Transformationen},
	description ={
		Eine Umformung oder Erzeugung einer \emph{Formel} aus einer vorgegebenen Menge von \emph{Formeln},
		\textdh\ die Anwendung einer \emph{Schlussregel}.
	}
}
\newglossaryentry{Transformationsmenge}{
	name        ={Transformationsmenge},
	plural      ={Transformationsmenge},
	description ={
		Eine Menge von \emph{Transformationen}.
	}
}

%V === V === V === V === V === V === V === V === V === V === V === V === V === V

\newglossaryentry{vergleichbar}{
	name        ={vergleichbar},
	plural      ={vergleichbare},
	description ={
		Zwei \emph{Objekte} $A$ und $B$ sind vergleichbar, wenn beide von derselben Art sind, \textdh\ wenn \textzB\ jeweils beide Mengen, \emph{Zeichenfolgen}, Zahlen, \textusw\ sind.
		Dabei muss bei \emph{Formeln} zwischen der Formel an sich und dem Ergebnis der Formel unterschieden werden.
		-- \vrefseesec{subsub:Vergleichbar}.
	}
}
\newglossaryentry{Vertauschung}{ %TODO ggf. überarbeiten
	name        ={Vertauschung},
	plural      ={Vertauschungen},
	description ={
		Die \emph{Vertauschung} von zwei unabhängigen \emph{formalen Elementen} ($\alpha$ und $\beta$) in einem anderen formalen Element ($\gamma$)
		-- formal: $\gamma(\alpha\swap\beta)$.
		Die Vertauschung ist eine spezielle Form der \emph{Substitution}.
		-- siehe die Definition~\eqref{def:Vertauschung} \vrefinsub{sub:Identitätsregeln}.
	}
}
\newglossaryentry{Voraussetzung}{
	name        ={Voraussetzung},
	plural      ={Voraussetzungen},
	description ={
		Die Voraussetzungen einer \emph{Schlussregel} $\frac{\prerequisiteset}{\conclusionset}$ sind die Elemente von $\prerequisiteset$.
	}
}
%TODO Voraussetzungsmenge

%W === W === W === W === W === W === W === W === W === W === W === W === W === W

\newglossaryentry{Wahrheitswert}{
	name        ={Wahrheitswert},
	plural      ={Wahrheitswerte},
	description ={
		Wahrheitswerte sind die Werte \chrqt{$\ltrue$} und \chrqt{$\lfalse$}, oft auch mit \chrqt{$\mathrm{wahr}$} und \chrqt{$\mathrm{falsch}$}, \chrqt{$\mathrm{true}$} und \chrqt{$\mathrm{false}$} oder einfach \chrqt{$1$} und \chrqt{$0$} bezeichnet.
	}
}

%Z === Z === Z === Z === Z === Z === Z === Z === Z === Z === Z === Z === Z === Z

\newglossaryentry{Zeichenfolge}{
	name        ={Zeichenfolge},
	plural      ={Zeichenfolgen},
	description ={
		Folgen von unzerlegbaren Zeichen und Symbolen, wobei Leerstellen und sonstiger Zwischenraum nicht zählen und nur zur besseren Darstellung dienen.
		Dabei sind als spezielle Symbole auch \emph{Zeichenketten} erlaubt, solange die Zerlegung eindeutig bleibt.
		\textZB\ kann \chrqt{$sin$} als ein einzelnes Symbol -- für die Sinusfunktion -- aufgefasst werden, aber auch als Folge der Buchstaben \chrqt{s}, \chrqt{i} und \chrqt{b}.
		\emph{Formeln} werden immer als Zeichenfolgen aufgefasst.
	}
}
\newglossaryentry{Zeichenkette}{
	name        ={Zeichenkette},
	plural      ={Zeichenketten},
	description ={
		Folgen von unzerlegbaren Zeichen, auch Leerstellen und sonstigem Zwischenraum.
		-- siehe auch \emph{Zeichenfolge}.
	}
}
\newglossaryentry{zulaessige-Transformation}{%TODO ggf. überarbeiten
	name        ={zulässige Transformation},
	plural      ={zulässige Transformationen},
	description ={
		Eine \emph{Transformation} aus einer vorgegebenen Menge von Transformationen oder eine daraus zulässiger weise abgeleitete Transformation.
	}
}


% Titelseite ###################################################################

\titlehead{
	{\Large Dr. Winfried Teschers}\\
	Anton-Günther-Straße 26c\\91083 Baiersdorf\\
	{\footnotesize winfried.teschers@t-online.de}
}
\subject{Projektdokument}
\title{{\Huge ASBA}\\\Axiome, Sätze, \Beweise\ und Auswertungen}
\subtitle{Projekt zur maschinellen Überprüfung von mathematischen \Beweisen\ und deren Ausgabe in lesbarer Form}
\author{Winfried Teschers}
\date{\today}
\publishers{\vspace{1cm}\normalsize
	Es wird ein Programmsystem beschrieben, das zu eingegebenen \Axiomen, Sätzen, und \Beweisen\ letztere prüft, Auswertungen generiert und zu gegebenen \Ausgabeschemata\ eine Ausgabe der Elemente in üblicher Formelschreibweise im \LaTeX-Format erstellt.
}

% Dokument #####################################################################

\begin{document}
	\maketitle

	~\vfill Copyright \copyright\ 2017 Winfried Teschers\bigskip

	Permission is granted to copy, distribute and/or modify this document under the terms of the GNU Free Documentation License, Version~1.3 or any later version published by the Free Software Foundation; with no Invariant Sections, no Front-Cover Texts, and no Back-Cover Texts.
	You should have received a copy of the GNU Free Documentation License along with this document.
	If not, see \url{http://www.gnu.org/licenses/}.

	%chapter{Inhaltsverzeichnis}% ##############################################
	\tableofcontents
	\Endchapter

	%%############################################################################%%
%%                                                                            %%
%% Datei:  ASBA-Analyse.tex                                                   %%
%% Inhalt: Kapitel "Analyse"                                                  %%
%%                                                                            %%
%% Copyright (C) 2017  Winfried Teschers                                      %%
%%                                                                            %%
%% This program is free software: you can redistribute it and/or modify       %%
%% it under the terms of the GNU Affero General Public License as published   %%
%% by the Free Software Foundation, either version 3 of the License, or       %%
%% (at your option) any later version.                                        %%
%%                                                                            %%
%% This program is distributed in the hope that it will be useful,            %%
%% but WITHOUT ANY WARRANTY; without even the implied warranty of             %%
%% MERCHANTABILITY or FITNESS FOR A PARTICULAR PURPOSE.  See the              %%
%% GNU Affero General Public License for more details.                        %%
%%                                                                            %%
%% You should have received a copy of the GNU Affero General Public License   %%
%% along with this program.  If not, see <http://www.gnu.org/licenses/>.      %%
%%                                                                            %%
%% Dr. Winfried Teschers                                                      %%
%% Anton-Günther-Straße 26c                                                   %%
%% 91083 Baiersdorf                                                           %%
%% Germany                                                                    %%
%%                                                                            %%
%% e-mail: winfried.teschers@t-online.de                                      %%
%%                                                                            %%
%%############################################################################%%

% !TeX root = ASBA.tex
% !TeX encoding = UTF-8
% !TeX spellcheck = de_DE

\chapter     {Analyse}% ########################################################
\beginchapter{Analyse}
\label   {cha:Analyse}

In der Mathematik gibt es eine unüberschaubare Menge an \Axiomen, \Saetzen, \Beweisen, \Fachbegriffen\ und \Fachgebieten.
Zu den meisten \Fachgebieten\ gibt es noch ungelöste Probleme.

Es fehlt ein System, das einen Überblick bietet und die Möglichkeit, \Beweise\ automatisch zu überprüfen.
Außerdem sollte all dies in üblicher mathematischer Schreibweise ein- und ausgegeben werden können.
In diesem Dokument werden die Grundlagen für das zu entwickelnde Programmsystem \defTxt{\ASBA} (ein Akronym für "`\DefFt{A}xiome, \DefFt{S}ätze, \DefFt{B}eweise und \DefFt{A}uswertungen"') behandelt.

Ein Programmsystem mit ähnlicher Aufgabenstellung findet sich im GitHub Projekt \emph{Hilbert~II} (\cite{bib:HilbertII, bib:qedeq}).
Einige Ideen sind von dort übernommen worden.

\section     {Fragen}% =========================================================
\beginsection{Fragen}
\label   {sec:Fragen}

Einige der Fragen, die in diesem Zusammenhang auftauchen,
werden nun formuliert:
\begin{enumerate}
	%
	\item \label{Frage:Grundlagen} \DefFt{Grundlagen}:
	Was sind die Grundlagen?
	\textZB\ welche \Logik\ und welche \Mengenlehre.
	%
	\item \label{Frage:Basis} \DefFt{Basis}:
	Welche wichtigen \Axiome, \Saetze, \Beweise, \Fachbegriffe\ und \Fachgebiete\ gibt es?
	Welche davon sind Standard?
	%
	\item \label{Frage:Axiome} \defTxt{\Axiome}:
	Welche \Axiome\ werden bei einem \Satz\ oder \Beweis\ vorausgesetzt?
	Allgemein anerkannte oder auch strittige, wie \textzB\ den \emph{\Satz\ vom ausgeschlossenen Dritten} (\emph{tertium non datur}) oder das \emph{Auswahlaxiom}.
	%
	\item \label{Frage:Beweis} \defTxt{\Beweis}:
	Ist ein \Beweis\ fehlerfrei?
	%
	\item \label{Frage:Konstruktion} \DefFt{Konstruktion}:
	Gibt es einen konstruktiven \Beweis?
	%
	\item \label{Frage:Vergleiche} \DefFt{Vergleiche}:
	Welcher \Beweis\ ist besser?
	Nach welchem Kriterium?
	\textZB\ elegant, kurz, einsichtig oder wenige \Axiome.
	Was heißt eigentlich \emph{elegant}?
	%
	\item \label{Frage:Definitionen} \DefFt{Definitionen}:
	Was ist mit einem \Fachbegriff\ jeweils genau gemeint?
	\textZB\ \emph{Stetigkeit}, \emph{Integral} und \emph{Analysis}.
	%
	\item \label{Frage:Abhaengigkeiten} \DefFt{Abhängigkeiten}:
	Wie heißt ein \Fachbegriff\ in einer anderen Sprache?
	Ist wirklich dasselbe gemeint?
	Was ist mit \Fachbegriffen\ in verschiedenen \Fachgebieten?
	%
	\item \label{Frage:Ueberblick} \DefFt{Überblick}:
	Ist ein \Axiom, \Satz, \Beweis\ oder \Fachbegriff\ schon einmal --- \textggf\ abweichend --- definiert, formuliert oder bewiesen worden?
	%
	\item \label{Frage:Darstellung} \defTxt{\Darstellung}:
	Wie kann man einen \Satz\ und den zugehörigen \Beweis\ --- \textggf\ auch spezifisch für ein \Fachgebiet\ --- darstellen?
	%
	\item \label{Frage:Forschung} \DefFt{Forschung}:
	Welche Probleme gibt es noch zu erforschen.
	%
\end{enumerate}

\section     {Eigenschaften}% ==================================================
\beginsection{Eigenschaften}
\label   {sec:Eigenschaften}

\ASBA\ soll ausgehend von den Fragen in \vrefsec{sec:Fragen} entwickelt werden, und die folgenden Eigenschaften haben:
\begin{enumerate}
	%
	\item \label{Eigenschaft:Daten} \DefFt{Daten}:
	\Axiome, \Saetze, \Beweise, \Fachbegriffe\ und \Fachgebiete\ können in formaler Form gespeichert werden --- auch (noch) nicht oder unvollständig bewiesene \Saetze.
	Dabei soll die übliche mathematische Schreibweise verwendet werden können.
	%
	\item \label{Eigenschaft:Definitionen} \DefFt{Definitionen}:
	Es können \Fachbegriffe\ für \Axiome, \Saetze, \Beweise\ und \Fachgebiete\ --- letztere mit eigenen \Axiomen, \Saetzen, \Beweisen, \Fachbegriffen\ und über- oder untergeordneten \Fachgebieten\ --- definiert werden.
	Die Definitionen dürfen an anderer Stelle definierte \Fachbegriffe\ und \Fachgebiete\ verwenden.%
	\footnote{Rekursive Definitionen sollten ebenfalls möglich sein.}
	%
	\item \label{Eigenschaft:Pruefung} \DefFt{Prüfung}:
	Vorhandene \Beweise\ können automatisch geprüft werden.
	%
	\item \label{Eigenschaft:Ausgaben} \DefFt{Ausgaben}:
	Die \Axiome, \Saetze\ und \Beweise\ können in üblicher Schreibweise --- abhängig von Sprache und \Fachgebiet\ --- ausgegeben werden.
	%
	\item \label{Eigenschaft:Auswertungen} \DefFt{Auswertungen}:
	Zusätzlich zur Ausgabe der gespeicherten Daten sind verschiedene Auswertungen möglich, unter anderem für die meisten der unter \vrefsec{sec:Fragen} behandelten Fragen.
	%
	\setcounter{Enumi}{\value{enumi}}% Nummerierung wird fortgesetzt.
\end{enumerate}
%
Damit \ASBA\ nicht umsonst erstellt wird und möglichst breite Verwendung findet, werden noch zwei Punkte angefügt:
\begin{enumerate}
	\setcounter{enumi}{\value{Enumi}}% Nummerierung wird fortgesetzt.
	%
	\item \label{Eigenschaft:Lizenz} \DefFt{Lizenz}:
	Die Software ist \emph{Open Source}.
	%
	\item \label{Eigenschaft:Akzeptanz} \DefFt{Akzeptanz}:
	\ASBA\ wird von Mathematikern akzeptiert und verwendet.
\end{enumerate}
%
\vreftab{tab:Fragen2Eigenschaften} zeigt, wie sich die Eigenschaften zu den Fragen \vrefinsec{sec:Fragen} verhalten.
Mit einem X werden die Spalten einer Zeile markiert, deren zugehörige Eigenschaften zur Beantwortung der entsprechenden Frage beitragen sollen.
Idealerweise sollte die Erfüllung aller angegebenen Eigenschaften alle gestellten Fragen beantworten, was allerdings illusorisch ist.
%
% Abstände für die nächsten drei Tabellen
\newcommand*{\vsL}{\hspace{-1.0cm}}  % für 1-stellige Zahlen
\newcommand*{\vsl}{\hspace{-6pt}\vsL}% für 2-stellige Zahlen
\newcommand*{\vsc}{\hspace{6pt}}     % für die gedrehten Überschriften
%
\begin{table}[H]
	\begin{tabularx}{\linewidth}
		{@{\hspace{.5cm}}rl@{\extracolsep{\fill}}|*{7}{c}@{\hspace{1cm}}|}
		\multicolumn{2}{l|}{\diagbox[height=3.0cm,width=4.5cm]%
			{\TabFt{Frage}\\~}{\\\TabFt{Eigenschaft}}}
		&\rotatebox{90}{%
			\mbox{\vsL\ref{Eigenschaft:Daten}        \vsc Daten        }}
		&\rotatebox{90}{%
			\mbox{\vsL\ref{Eigenschaft:Definitionen} \vsc Definitionen }}
		&\rotatebox{90}{%
			\mbox{\vsL\ref{Eigenschaft:Pruefung}     \vsc Prüfung      }}
		&\rotatebox{90}{%
			\mbox{\vsL\ref{Eigenschaft:Ausgaben}     \vsc Ausgaben     }}
		&\rotatebox{90}{%
			\mbox{\vsL\ref{Eigenschaft:Auswertungen} \vsc Auswertungen }}
		&\rotatebox{90}{%
			\mbox{\vsL\ref{Eigenschaft:Lizenz}       \vsc Lizenz       }}
		&\rotatebox{90}{%
			\mbox{\vsL\ref{Eigenschaft:Akzeptanz}    \vsc Akzeptanz    }}
		\\\hline
		\ref{Frage:Grundlagen}      & Grundlagen
		& X & X & - & X & X & - & - \\
		\ref{Frage:Basis}           & Basis
		& X & X & - & X & X & - & - \\
		\ref{Frage:Axiome}          & \Axiome
		& X & X & - & X & X & - & - \\
		\hdashline[2pt/2pt]
		\ref{Frage:Beweis}          & \Beweis
		& X & - & X & X & - & - & - \\
		\ref{Frage:Konstruktion}    & Konstruktion
		& X & - & - & X & - & - & - \\
		\ref{Frage:Vergleiche}      & Vergleiche
		& X & - & - & - & X & - & - \\
		\hdashline[2pt/2pt]
		\ref{Frage:Definitionen}    & Definitionen
		& X & X & - & X & - & - & - \\
		\ref{Frage:Abhaengigkeiten} & Abhängigkeiten
		& X & - & - & X & - & - & - \\
		\ref{Frage:Ueberblick}      & Überblick
		& X & - & - & - & X & - & - \\
		\hdashline[2pt/2pt]
		\ref{Frage:Darstellung}     & \Darstellung
		& - & X & - & X & - & - & - \\
		\ref{Frage:Forschung}       & Forschung
		& X & - & - & - & X & - & - \\
		\hline
	\end{tabularx}
	\caption{%
		Fragen (\ref{sec:Fragen}) $\to$ Eigenschaften (\ref{sec:Eigenschaften})
	}
	\label{tab:Fragen2Eigenschaften}% Erst nach '\caption'!
\end{table}

\section[Ziele]{\Ziele}% =======================================================
\beginsection  {\Ziele}
\label      {sec:Ziele}

Um die Eigenschaften von \vrefsec{sec:Eigenschaften} zu erreichen, werden für \ASBA\ die folgenden \Ziele%
\footnote{%
	Es sind eigentlich Anforderungen.
	Diese \Bezeichnung\ wird aber schon \vrefincha{cha:Design} verwendet.
}
gesetzt:
\begin{enumerate}
	%
	\item \label{Ziel:Daten} \DefFt{Daten}:
	Die verteilte Datenbank von \ASBA\ enthält möglichst viele wichtige \Axiome, \Saetze, \Beweise, \Fachbegriffe, \Fachgebiete\ und \Ausgabeschemata%
	\footnote{%
		Um den Punkt~\ref{Eigenschaft:Ausgaben} \vrefvonsec{sec:Eigenschaften} erfüllen zu können, werden noch fachgebietsspezifische \Ausgabeschemata\ benötigt, welche die Art der Ausgaben beschreiben.
	}.
	%
	\item \label{Ziel:Form} \DefFt{Form}:
	Die Daten liegen in formaler, geprüfter Form vor.
	%
	\item \label{Ziel:Eingaben} \DefFt{Eingaben}:
	Die Eingabe von Daten erfolgt in einer formalen \Syntax\ unter Verwendung der üblichen mathematischen Schreibweise.
	%
	\item \label{Ziel:Pruefung} \DefFt{Prüfung}:
	\Beweise\ können automatisch geprüft\footnote{%
		An dieser Stelle soll \ASBA\ soll keine \Beweise\ finden --- das ist \Ziel\ von Punkt \ref{Ziel:Beweisunterstuetzung}, sondern nur vorhandene prüfen.
	}
	werden.
	%
	\item \label{Ziel:Ausgaben} \DefFt{Ausgaben}:
	Die Ausgabe kann in einer eindeutigen, formalen \Syntax\ gemäß vorhandener \Ausgabeschemata\ erfolgen.
	%
	\item \label{Ziel:Auswertungen} \DefFt{Auswertungen}:
	Zusätzlich zur Ausgabe der Daten sind verschiedene Auswertungen möglich.
	Insbesondere kann zu jedem \Beweis\ angegeben werden, wie lang er ist und welche \Axiome\ und \Saetze%
	\footnote{%
		\Saetze, die quasi als \Axiome\ verwendet werden.
	}
	er benötigt.
	%
	\item \label{Ziel:Anpassbarkeit} \DefFt{Anpassbarkeit}:
	\Fachbegriffe\ und die \Darstellung\ bei der Ausgabe können mit Hilfe von --- gegebenenfalls unbenannten --- untergeordneten \Fachgebieten\ angepasst werden.
	%
	\item \label{Ziel:Individualitaet} \DefFt{Individualität}:
	\Axiome\ und \Saetze\ können für jeden \Beweis\ individuell vorausgesetzt werden.
	Dabei sind fachgebietsspezifische \Fachbegriffe\ erlaubt.
	%
	\item \label{Ziel:Internet} \DefFt{Internet}:
	Die Daten können auf mehrere Dateien verteilt sein.
	Ein Teil davon --- oder sogar alle --- können im Internet liegen.
	%
	\item \label{Ziel:Kommunikation} \DefFt{Kommunikation}:
	Die Kommunikation mit \ASBA\ kann mit den \Fachbegriffen\ der einzelnen \Fachgebiete\ erfolgen.
	%
	\item \label{Ziel:Zugriff} \DefFt{Zugriff}:
	Der Zugriff auf \ASBA\ kann lokal und über das Internet erfolgen.
	%
	\item \label{Ziel:Unabhaengigkeit} \DefFt{Unabhängigkeit}:
	\ASBA\ kann online und offline arbeiten.
	%
	\item \label{Ziel:Rekursion} \DefFt{Rekursion}:
	Es kann rekursiv über alle verwendeten Dateien --- auch solchen, die im Internet liegen --- ausgewertet werden.
	%
	\item \label{Ziel:Bedienbarkeit} \DefFt{Bedienbarkeit}:
	\ASBA\ ist einfach zu bedienen.
	%
	\item \label{Ziel:Lizenz} \DefFt{Lizenz}:
	Die Software ist \emph{Open Source}.
	%
	\item \label{Ziel:Zwischenspeicher} \DefFt{Zwischenspeicher}:
	Wichtige Auswertungen können an vorhandenen Dateien angehängt oder separat in eigenen Dateien gespeichert werden.
	%
	\item \label{Ziel:Beweisunterstuetzung} \DefFt{Beweisunterstützung}:
	\ASBA\ hilft bei der Erstellung von \Beweisen.
	%
\end{enumerate}
%
Punkt \ref{Ziel:Zwischenspeicher} wurde noch angefügt, damit \ASBA\ effizient arbeiten kann und um die Akzeptanz zu erhöhen.
Um letzteres zu erreichen, dafür ist auch Punkt \ref{Ziel:Beweisunterstuetzung} nützlich.
Es bietet sich ja auch an, die Fähigkeiten, die \ASBA\ mit der Prüfung von Beweisen haben wird, auch auf die Erstellung von Beweisen anzuwenden.
Die Reihenfolge der \Ziele\ stellt noch keine Priorisierung fest.

\vrefDtab{tab:Eigenschaften2Ziele} zeigt wieder, wie sich die Ziele zu den Eigenschaften \vrefinsec{sec:Eigenschaften} verhalten.
Mit einem X werden wieder die Spalten einer Zeile markiert, deren zugehörige Ziele zur Sicherstellung der entsprechenden Eigenschaft beitragen sollen.
Idealerweise sollte durch Erreichen aller aufgestellten Ziele \ASBA\ alle angegebenen Eigenschaften aufweisen, was wahrscheinlich ebenfalls illusorisch ist.
%
\begin{table}[H]
	\begin{tabularx}{\linewidth}
		{@{\hspace{.2cm}}rl@{\extracolsep{\fill}}|*{17}{c}@{\hspace{0.2cm}}|}
		\multicolumn{2}{l|}{\diagbox[height=3.0cm,width=3.6cm]%
			{\TabFt{Eigenschaft}\\~}{\\\\\TabFt{Ziel}}}
		&\rotatebox{90}{%
			\mbox{\vsL\ref{Ziel:Daten}                \vsc Daten              }}
		&\rotatebox{90}{%
			\mbox{\vsL\ref{Ziel:Form}                 \vsc Form               }}
		&\rotatebox{90}{%
			\mbox{\vsL\ref{Ziel:Eingaben}             \vsc Eingaben           }}
		&\rotatebox{90}{%
			\mbox{\vsL\ref{Ziel:Pruefung}             \vsc Prüfung            }}
		&\rotatebox{90}{%
			\mbox{\vsL\ref{Ziel:Ausgaben}             \vsc Ausgaben           }}
		&\rotatebox{90}{%
			\mbox{\vsL\ref{Ziel:Auswertungen}         \vsc Auswertungen       }}
		&\rotatebox{90}{%
			\mbox{\vsL\ref{Ziel:Anpassbarkeit}        \vsc Anpassbarkeit      }}
		&\rotatebox{90}{%
			\mbox{\vsL\ref{Ziel:Individualitaet}      \vsc Individualität     }}
		&\rotatebox{90}{%
			\mbox{\vsL\ref{Ziel:Internet}             \vsc Internet           }}
		&\rotatebox{90}{%
			\mbox{\vsl\ref{Ziel:Kommunikation}        \vsc Kommunikation      }}
		&\rotatebox{90}{%
			\mbox{\vsl\ref{Ziel:Zugriff}              \vsc Zugriff            }}
		&\rotatebox{90}{%
			\mbox{\vsl\ref{Ziel:Unabhaengigkeit}      \vsc Unabhängigkeit     }}
		&\rotatebox{90}{%
			\mbox{\vsl\ref{Ziel:Rekursion}            \vsc Rekursion          }}
		&\rotatebox{90}{%
			\mbox{\vsl\ref{Ziel:Bedienbarkeit}        \vsc Bedienbarkeit      }}
		&\rotatebox{90}{%
			\mbox{\vsl\ref{Ziel:Lizenz}               \vsc Lizenz             }}
		&\rotatebox{90}{%
			\mbox{\vsl\ref{Ziel:Zwischenspeicher}     \vsc Zwischenspeicher   }}
		&\rotatebox{90}{%
			\mbox{\vsl\ref{Ziel:Beweisunterstuetzung} \vsc Beweisunterstützung}}
		\\\hline
		\ref{Eigenschaft:Daten}         & Daten%
		& X & X & X & - & - & - & - & - & - & - & - & - & - & - & - & - & - \\
		\ref{Eigenschaft:Definitionen}  & Definitionen%
		& X & - & X & - & - & - & - & - & - & - & - & - & - & - & - & - & - \\
		\ref{Eigenschaft:Pruefung}      & Prüfung
		& - & - & - & X & - & - & - & - & - & - & - & - & - & - & - & - & - \\
		\hdashline[2pt/2pt]
		\ref{Eigenschaft:Ausgaben}      & Ausgaben%
		& - & - & - & - & X & - & - & - & - & - & - & - & - & - & - & - & - \\
		\ref{Eigenschaft:Auswertungen}  & Auswertungen%
		& - & - & - & - & - & X & - & - & - & - & - & - & - & - & - & - & - \\
		\ref{Eigenschaft:Lizenz}        & Lizenz%
		& - & - & - & - & - & - & - & - & - & - & - & - & - & - & X & - & - \\
		\hdashline[2pt/2pt]
		\ref{Eigenschaft:Akzeptanz}     & Akzeptanz%
		& X & X & X & X & X & X & X & X & X & X & X & X & X & X & X & X & X \\
		\hline
	\end{tabularx}
	\caption{%
		Eigenschaften (\ref{sec:Eigenschaften}) $\to$ Ziele (\ref{sec:Ziele})
	}
	\label{tab:Eigenschaften2Ziele}% Erst nach '\caption'!
\end{table}

\section     {Zusammenfassung}% ================================================
\beginsection{Zusammenfassung}
\label   {sec:Zusammenfassung}

\begin{table}[H]
	\begin{tabularx}{\linewidth}
		{@{\hspace{.2cm}}rl@{\extracolsep{\fill}}|*{17}{c}@{\hspace{0.2cm}}|}
		\multicolumn{2}{l|}{\diagbox[height=3.0cm,width=4.0cm]%
			{\TabFt{Frage}\\~}{\\\\\TabFt{Ziel}}}
		&\rotatebox{90}{%
			\mbox{\vsL\ref{Ziel:Daten}                \vsc Daten              }}
		&\rotatebox{90}{%
			\mbox{\vsL\ref{Ziel:Form}                 \vsc Form               }}
		&\rotatebox{90}{%
			\mbox{\vsL\ref{Ziel:Eingaben}             \vsc Eingaben           }}
		&\rotatebox{90}{%
			\mbox{\vsL\ref{Ziel:Pruefung}             \vsc Prüfung            }}
		&\rotatebox{90}{%
			\mbox{\vsL\ref{Ziel:Ausgaben}             \vsc Ausgaben           }}
		&\rotatebox{90}{%
			\mbox{\vsL\ref{Ziel:Auswertungen}         \vsc Auswertungen       }}
		&\rotatebox{90}{%
			\mbox{\vsL\ref{Ziel:Anpassbarkeit}        \vsc Anpassbarkeit      }}
		&\rotatebox{90}{%
			\mbox{\vsL\ref{Ziel:Individualitaet}      \vsc Individualität     }}
		&\rotatebox{90}{%
			\mbox{\vsL\ref{Ziel:Internet}             \vsc Internet           }}
		&\rotatebox{90}{%
			\mbox{\vsl\ref{Ziel:Kommunikation}        \vsc Kommunikation      }}
		&\rotatebox{90}{%
			\mbox{\vsl\ref{Ziel:Zugriff}              \vsc Zugriff            }}
		&\rotatebox{90}{%
			\mbox{\vsl\ref{Ziel:Unabhaengigkeit}      \vsc Unabhängigkeit     }}
		&\rotatebox{90}{%
			\mbox{\vsl\ref{Ziel:Rekursion}            \vsc Rekursion          }}
		&\rotatebox{90}{%
			\mbox{\vsl\ref{Ziel:Bedienbarkeit}        \vsc Bedienbarkeit      }}
		&\rotatebox{90}{%
			\mbox{\vsl\ref{Ziel:Lizenz}               \vsc Lizenz             }}
		&\rotatebox{90}{%
			\mbox{\vsl\ref{Ziel:Zwischenspeicher}     \vsc Zwischenspeicher   }}
		&\rotatebox{90}{%
			\mbox{\vsl\ref{Ziel:Beweisunterstuetzung} \vsc Beweisunterstützung}}
		\\\hline
		\ref{Frage:Grundlagen}      & Grundlagen%
		& X & X & X & - & X & X & x & - & - & - & - & - & - & - & - & - & - \\
		\ref{Frage:Basis}           & Basis%
		& X & X & X & - & X & X & x & x & - & - & - & - & - & - & - & - & - \\
		\ref{Frage:Axiome}          & \Axiome%
		& X & X & X & - & X & X & x & - & - & - & - & - & - & - & - & - & - \\
		\hdashline[2pt/2pt]
		\ref{Frage:Beweis}          & \Beweis%
		& X & X & X & X & X & - & - & x & - & - & - & - & - & - & - & - & - \\
		\ref{Frage:Konstruktion}    & Konstruktion%
		& X & X & X & - & X & - & - & x & - & - & - & - & - & - & - & - & - \\
		\ref{Frage:Vergleiche}      & Vergleiche%
		& X & X & X & - & - & X & - & x & - & - & - & - & - & - & - & - & - \\
		\hdashline[2pt/2pt]
		\ref{Frage:Definitionen}    & Definitionen%
		& X & X & X & - & X & - & x & - & - & - & - & - & - & - & - & - & - \\
		\ref{Frage:Abhaengigkeiten} & Abhängigkeiten%
		& X & X & X & - & X & - & x & - & - & - & - & - & - & - & - & - & - \\
		\ref{Frage:Ueberblick}      & Überblick%
		& X & X & X & - & - & X & x & - & - & - & - & - & - & - & - & - & - \\
		\hdashline[2pt/2pt]
		\ref{Frage:Darstellung}     & \Darstellung%
		& X & - & X & - & X & - & x & - & - & - & - & - & - & - & - & - & - \\
		\ref{Frage:Forschung}       & Forschung%
		& X & X & X & - & - & X & x & - & - & - & - & - & - & - & - & - & - \\
		\hline
		\multicolumn{19}{l|}{Die nächsten beiden Punkte
			sind Eigenschaften aus \vrefsec{sec:Eigenschaften}:}\\
		\hline
		\ref{Eigenschaft:Lizenz}    & Lizenz%
		& - & - & - & - & - & - & - & - & - & - & - & - & - & - & X & - & - \\
		\ref{Eigenschaft:Akzeptanz} & Akzeptanz%
		& X & X & X & X & X & X & X & X & X & X & X & X & X & X & X & X & X \\
		\hline
	\end{tabularx}
	\caption{Fragen (\ref{sec:Fragen}) $\to$ Ziele (\ref{sec:Ziele})}
	\label{tab:Fragen2Ziele}% Erst nach '\caption'!
\end{table}
%
\vrefDtab{tab:Fragen2Ziele} ist eine Kombination der Tabellen~ \ref{tab:Fragen2Eigenschaften} und~\ref{tab:Eigenschaften2Ziele} und zeigt, wie sich die Ziele \vrefinsec{sec:Ziele} zu den Fragen \vrefinsec{sec:Fragen} verhalten.
Auch in dieser Tabelle werden mit einem X die Spalten einer Zeile markiert, deren zugehörige Ziele für die Beantwortung der entsprechenden Frage nötig sind.
Mit einem kleinen x werden sie markiert, wenn sie zur Beantwortung der Fragen nicht nötig, aber von Interesse sind.
Idealerweise sollte das Erreichen aller aufgestellten Ziele alle gestellten Fragen beantworten, was natürlich auch illusorisch ist.

\clearpage

\section[Die Umgebung von \glsentrytext{ASBA}]{Die Umgebung von \ASBA}%
\beginsection                                 {Die Umgebung von \ASBA}
\label                                        {sec:Umgebung}

\vrefInfig{fig:Umgebung} wird beschrieben, welche Interaktionen \ASBA\ mit der Umgebung hat, \textdh\ welche Ein- und Ausgaben existieren und woher sie kommen \textbzw\ wohin sie gehen.

\begin{figure}[H]
	\setlength\unitlength{1cm}
	\begin{picture}(17.0,9.5)(-8.4,-4.5)
		% Hilfsgitter während der Bildbearbeitung
		%\color{lightgray}
		%\multiput(-8.4,-4.5)(+0.0,1.0){10}{\line(1,0){17.0}}
		%\multiput(-7.9,-4.5)(+1.0,0.0){17}{\line(0,1){ 9.5}}
		\linethickness{1.5pt}
		% Hintergrund (grau) ===============================================
		\color{gray}
		% rechts: externes ASBA mit Pfeilen --------------------------------
		\put(+3.00,+0.50){\framebox(2.40,1.60){\huge\ImageFt{\ASBA}}}
		\put(+3.00,+0.50){\makebox(2.40,1.50)[t]{\ImageFt{externes}}}
		\put(+4.00,+2.12){\vector(-1,+4){0.35}}% <--- externes ASBA
		\put(+4.00,+3.55){\vector(+1,-4){0.36}}% ---> externes ASBA
		\put(+3.81,+2.82){\marker{a}}
		% rechts oben: externe Datenbank mit Pfeilen -----------------------
		\put(+7.30,+3.80){\Datenbank{1.20}{0.40}{0.80}{\small externe}{\small Datenbank}}
		\put(+5.41,+2.10){\vector(+1,+2){0.70}}% <--- externes ASBA
		\put(+6.14,+2.89){\vector(-1,-2){0.72}}% ---> externes ASBA
		\put(+5.60,+2.48){\marker{b}}
		% Verbindung Auswertungen ---> Männchen ----------------------------
		\put(+5.60,-3.30){\vector(-1,0){4.05}}% Auswertungen ---> Männchen
		\put(+3.50,-3.30){\marker{c}}
		% Verbindung Männchen <---> Terminal -------------------------------
		\put(-2.00,-3.00){\vector(+1,0){2.45}}% Männchen <--- Terminal
		\put(+0.40,-3.30){\vector(-1,0){2.40}}% Männchen ---> Terminal
		\put(-0.75,-3.15){\marker{d}}
		% Verbindung Terminal <---> Datei ----------------------------------
		\put(-5.50,-1.50){\vector(+3,-2){1.50}}% Terminal <--- Datei
		\put(-4.01,-2.80){\vector(-3,+2){1.60}}% Terminal ---> Datei
		\put(-5.00,-2.10){\marker{e}}
		% Vordergrund (schwarz) ============================================
		\color{black}
		% rechts oben: Wolke mit Pfeilen -----------------------------------
		\put(+3.40,+4.50){\Wolke{Internet}}
		\put(+2.00,+4.04){\vector(-1,-3){0.82}}% ---> ASBA
		\put(+1.53,+1.53){\vector(+1,+3){0.75}}% <--- ASBA
		\put(+1.55,+2.70){\Marker{1}}
		% links oben: Datenbank mit Pfeilen --------------------------------
		\put(-7.00,+3.50){\Datenbank{1.50}{0.50}{1.00}{\large \ASBA}{\large Datenbank}}
		\put(-5.50,+3.75){\vector(+7,-4){3.95}}% ---> ASBA
		\put(-1.51,+1.10){\vector(-7,+4){4.00}}% <--- ASBA
		\put(-3.70,+2.40){\Marker{2}}
		% links Mitte: Datei mit Pfeilen -----------------------------------
		\put(-7.00,-1.00){\Datei{3.00}{2.00}{\ASBA}{Datei}}
		\put(-5.50,-0.80){\vector(+4,+1){3.95}}% ---> ASBA
		\put(-1.51,-0.10){\vector(-4,-1){4.00}}% <--- ASBA
		\put(-3.70,-0.45){\Marker{3}}
		%links unten: Rechner mit Pfeilen ----------------------------------
		\put(-3.00,-3.10){\Terminal{Terminal}}
		\put(-2.50,-2.48){\vector(+1,+2){0.98}}% ---> ASBA
		\put(-1.09,-0.52){\vector(-1,-2){0.98}}% <--- ASBA
		\put(-2.00,-1.50){\Marker{4}}
		% Mitte unten: Männchen mit Pfeilen --------------------------------
		\put(+1.00,-2.80){\Maennchen}
		\put(+0.85,-2.50){\vector(0,+1){2.00}}% ---> ASBA
		\put(+1.15,-0.51){\vector(0,-1){2.00}}% <--- ASBA
		\put(+0.80,-1.52){\Marker{5}}
		% rechts unten: Papier mit Pfeil -----------------------------------
		\put(+5.60,-4.20){\Papier{+2.00}{+0.30}{\ASBA}{Ausgabe}}
		\put(+1.51,-0.55){\vector(+2,-1){+4.10}}% <--- ASBA
		\put(+3.25,-1.55){\Marker{6}}
		% Mitte: ASBA ------------------------------------------------------
		\linethickness{3pt}
		\put(-1.5,-0.5){\framebox(3.0,2.0){\Huge\ImageFt{\ASBA}}}
	\end{picture}
	\caption{Die Umgebung von \ASBA}
	\label{fig:Umgebung}% Erst nach '\caption'!
\end{figure}

In den \vrefinfig{fig:Umgebung} abgebildeten Datenflüssen (1) bis (6) und (a) bis (e) werden die folgenden Daten übertragen:
\begin{itemize}
	\newcommand*{\vonnach}  [2]{#1 $\rightarrow$ #2}
	\newcommand*{\nachvon}  [2]{\vonnach{#2}{#1}}
	\newcommand*{\hinundher}[2]{#1 $\leftrightarrow$ #2}
	%
	\item[(1)]\label{dat:Internet}
	\begin{description}
		\item[\vonnach{\ASBA}{Internet}]\label{dat:ausInternet}
		Inhalte der Datenbank.
		\item[\nachvon{\ASBA}{Internet}]\label{dat:inInternet}
		Inhalte der externen Datenbank.
	\end{description}
	%
	\item[(2)]\label{dat:Datenbank}
	\begin{description}
		\item[\vonnach{Datenbank}{\ASBA}]\label{dat:ausDatenbank}
		Inhalte der Datenbank und Antworten auf Datenbankanweisungen.
		\item[\nachvon{Datenbank}{\ASBA}]\label{dat:inDatenbank}
		Inhalte der Datei, der externen Datenbank und Datenbankanweisungen.
	\end{description}
	%
	\item[(3)]\label{dat:Datei}
	\begin{description}
		\item[\vonnach{Datei}{\ASBA}]\label{dat:ausDatei}
		Inhalte der Datei.
		\item[\nachvon{Datei}{\ASBA}]\label{dat:inDatei}
		Die Datei wird um zusätzliche Auswertungen ergänzt, \textzB\ ob die \Beweise\ korrekt sind, welche \Axiome\ und \Saetze\ --- auch externe aus dem Internet --- verwendet wurden, Länge des \Beweises\ usw.
	\end{description}
	%
	\item[(4)]\label{dat:Terminal}
	\begin{description}
		\item[\vonnach{Terminal}{\ASBA}]\label{dat:ausTerminal}
		Anweisungen, Daten und Batchprogramme.
		\item[\nachvon{Terminal}{\ASBA}]\label{dat:inTerminal}
		Antworten auf Anweisungen, Auswertungen usw.
	\end{description}
	Außerdem interaktive Ein- und Ausgabe durch einen Anwender, wie in (5) beschrieben.
	%
	\item[(5)]\label{dat:Anwender}
	\begin{description}
		\item[\hinundher{Anwender}{\ASBA}]\label{dat:mitAnwender}
		Interaktive Ein- und Ausgaben durch einen Anwender mit Komponenten von (3), (4) und (6).
		--- Die Kommunikation läuft \textiAlg\ über ein Terminal.
	\end{description}
	%
	\item[(6)]\label{dat:Ausgabe}
	\begin{description}
		\item[\nachvon{Ausgabe}{\ASBA}]\label{dat:inAusgabe}
		Inhalte von Datei und Datenbank in lesbarer Form, \textua\ mit Hilfe von \Ausgabeschemata\ auch mit \Formeln.
		Die Ausgabe kann auch in eine Datei erfolgen,
		\textzB\ im \LaTeX-Format.
	\end{description}
	%
	\item[(a)]\label{dat:extInternet}
	\begin{description}
		\item[\vonnach{Internet}{externes \ASBA}]\label{dat:ausextInternet}
		Inhalte der Datenbank.
		\item[\nachvon{Internet}{externes \ASBA}]\label{dat:inextInternet}
		Inhalte der externen Datenbank.
	\end{description}
	%
	\item[(b)]\label{dat:extDatenbank}
	\begin{description}
		\item[\vonnach{externe Datenbank}{externes \ASBA}]
		\label{dat:ausextDatenbank} Inhalte der externen Datenbank.
		\item[\nachvon{externe Datenbank}{externes \ASBA}]
		\label{dat:inextDatenbank} Inhalte der Datenbank.
	\end{description}
	%
	\item[(c)]\label{dat:AusgabeAnwender}
	\begin{description}
		\item[\vonnach{Ausgabe}{Anwender}]\label{dat:Ausgabe2Anwender}
		Alle Daten der Ausgabe.
	\end{description}
	%
	\item[(d)] \label{dat:AnwenderTerminal}
	\begin{description}
		\item[\hinundher{Anwender}{Terminal}]\label{dat:Anwender22Terminal}
		Interaktive Ein- und Ausgabe durch einen Anwender, wie in (5) beschrieben.
	\end{description}
	%
	\item[(e)] \label{dat:TerminalDatei}
	\begin{description}
		\item[\hinundher{Terminal}{Datei}]\label{dat:Terminal22Datei}
		Erstellen und Bearbeiten der Datei durch einen Anwender.
		--- siehe (d)
	\end{description}
	%
\end{itemize}
Die Datenflüsse (a) bis (e) erfolgen außerhalb von \ASBA\ und werden nicht weiter behandelt.

Die Datenbank und die Datei enthalten im Prinzip die gleichen Daten, wobei sie in der Datei im Textformat in lesbarer Form und in der Datenbank in einem internen Format vorliegen.
Zudem enthält die Datenbank \textiAlg\ sehr viel mehr Daten. Es handelt sich dabei jeweils um die folgenden Daten:
\begin{description}
	\item[\defTxt{\Axiome}]         \label{Daten:Axiom}         \glsBeschreibung{Axiom}
	\item[\defTxt{\Saetze}]         \label{Daten:Satz}          \glsBeschreibung{Satz}
	\item[\defTxt{\Beweise}]        \label{Daten:Beweis}        \glsBeschreibung{Beweis}
	\item[\defTxt{\Fachbegriffe}]   \label{Daten:Fachbegriff}   \glsBeschreibung{Fachbegriff}
	\item[\defTxt{\Fachgebiete}]    \label{Daten:Fachgebiet}    \glsBeschreibung{Fachgebiet}
	\item[\defTxt{\Ausgabeschemata}]\label{Daten:Ausgabeschema} \glsBeschreibung{Ausgabeschema}
	\item[\defTxt{\Auswertungen}]   \label{Daten:Auswertung}    \glsBeschreibung{Auswertung}
\end{description}
Alle Daten können interne und externe Verweise enthalten.

\begin{offen}%%%

\section[Basis von Beweisen]{Basis von \Beweisen}% =============================
\beginsection               {Basis von \Beweisen}
\label                             {sec:BeweisBasis}

Da ein Computerprogramm erstellt werden soll, muss die Grundstruktur des Vorgehens bei \Beweisen\ definiert werden.%
\footnote{\seename~\cite{bib:Kalkuel}}

\begin{description}
	%
	\item[Die \logischeDarstellung] von mathematischen \Aussagen, wozu auch \Axiome\ und \Saetze\ gehören, erfolgt, da es sich immer um \Formeln\ handelt, an besten mit \Symbolfolgen%
	\footnote{%
		Die \interneDarstellung\ der \Symbolfolgen\ kann zur Optimierung von \ASBA\ von der \logischenD\ abweichen.
	},
	\textdh\ Folgen von Zeichen und Symbolen, in denen Zwischenraum --- insbesondere Leerstellen --- nicht zählen.
	Mehrdimensionale \Formeln, wie \textzB\ Matrizen, Baumstrukturen, Funktionsschemata und anderes, können auch als (eindimensionale) Symbolfolgen dargestellt werden.%
	\footnote{%
		\textZB\ könnte man eine 2$\times$2-Matrix
		$\begin{bmatrix} a & b \\ c & d \end{bmatrix}$
		auch darstellen als Folge von Zeilen: \seqqt{$[(a,b),(c,d)]$}, oder noch einfacher: \seqqt{$[a,b;c,d]$}.
		In \ASBA\ wird die \LaTeX-Syntax verwendet.
		\\Damit wird die soeben angegebene Matrix codiert durch \seqqt{\$\textbackslash begin\{bmatrix\}a\&b\textbackslash\textbackslash c\&d\textbackslash end\{bmatrix\}\$}.
	}
	\Beweise\ sind letztendlich nichts anderes, als erlaubte \Transformationen\ dieser \Symbolfolgen.
	%
	\item[\Bausteine] sind Grundelemente, auch \DefFt{Zeichen} oder \DefFt{(Satz-)Buchstaben} genannt, aus denen die Symbolfolgen bestehen dürfen, und müssen definiert werden.
	%
	\item[\Formationsregeln] dienen zur Festlegung, wie man aus den Bausteinen Ausdrücke erzeugen kann, und müssen ebenfalls definiert werden.
	%
	\item[\Saetze] lassen sich als eine \Menge\ von \Formeln, den \Praemissen, wozu auch \Axiome\ und andere \Saetze\ gehören können, einer weiteren \Menge\ von \Formeln\ (\Symbolfolgen), den \Konklusionen, und der Angabe eines \Beweises\ darstellen.
	%
	\item[\Beweise] zu gegebenen \Praemissen\ und \Konklusionen\ lassen sich als \Folge\ von \Transformationen, beginnend mit den \Praemissen\ und endend mit den \Konklusionen, darstellen.
	%
	\item[\Transformationsregeln] definieren, welche \Transformationen\ mit gegebenen \Formelmengen\ zulässig sind.%
	\footnote{\seename~\cite{bib:Rautenberg,bib:Schlussregel,bib:NatuerlichesSchliessen}}
	%
\end{description}

\end{offen}%%%

\Endchapter

	%%############################################################################%%
%%                                                                            %%
%% Datei:  ASBA-Mathematik.tex                                                %%
%% Inhalt: Kapitel "Mathematische Grundlagen"                                 %%
%%                                                                            %%
%% Copyright (C) 2017  Winfried Teschers                                      %%
%%                                                                            %%
%% This program is free software: you can redistribute it and/or modify       %%
%% it under the terms of the GNU Affero General Public License as published   %%
%% by the Free Software Foundation, either version 3 of the License, or       %%
%% (at your option) any later version.                                        %%
%%                                                                            %%
%% This program is distributed in the hope that it will be useful,            %%
%% but WITHOUT ANY WARRANTY; without even the implied warranty of             %%
%% MERCHANTABILITY or FITNESS FOR A PARTICULAR PURPOSE.  See the              %%
%% GNU Affero General Public License for more details.                        %%
%%                                                                            %%
%% You should have received a copy of the GNU Affero General Public License   %%
%% along with this program.  If not, see <http://www.gnu.org/licenses/>.      %%
%%                                                                            %%
%% Dr. Winfried Teschers                                                      %%
%% Anton-Günther-Str. 26c                                                     %%
%% 91083 Baiersdorf                                                           %%
%% Germany                                                                    %%
%%                                                                            %%
%% e-mail: winfried.teschers@t-online.de                                      %%
%%                                                                            %%
%%############################################################################%%

% !TeX root = ASBA.tex
% !TeX encoding = UTF-8
% !TeX spellcheck = de_DE

\chapter{Mathematische Grundlagen}% ############################################
\beginchapter{Mathematische Grundlagen}
\label{cha:Grundlagen}

Die mathematischen Grundlagen werden einerseits gebraucht um die erlaubten Beweisschritte zu definieren (\seename~\vref{sec:Schlussregeln}), andererseits dienen sie auch zum Testen von \ASBA. Alle hier aufgeführten Axiome, Sätze und Beweise sollen dazu kodiert und von \ASBA ausgewertet werden.

\section{Metasprache}% =========================================================
\beginsection{Metasprache}
\label{sec:Metasprache}

Wenn man über eine Sprache spricht, braucht man auch eine Sprache, in der Aussagen über die erstere getroffen werden können.
Wenn die zuerst genannte Sprache die der Mathematik ist, nimmt man üblicherweise die natürliche Sprache als \glsIdx{Metasprache}.
Leider ist diese oft ungenau, nicht immer eindeutig und abhängig vom Zusammenhang, in dem sie gesprochen wird%
\footnote{%
	Man betrachte die beiden Aussagen \enquote{Studenten und Rentner zahlen die Hälfte.} und \enquote{Studenten oder Rentner zahlen die Hälfte.}, die beide das gleiche meinen.
	-- Entnommen aus \cite{bib:Rautenberg} \sectionname~1.2 Bemerkung 1.
	
	Ein weiteres Problem ist, dass man unauflösbare Widersprüche formulieren kann, \textzB \enquote{%
		Der Barbier ist der Mann im Ort, der genau die Männer im Ort rasiert, die sich nicht selbst rasieren.%
	}.
	Und der Barbier?
	Wenn er sich selbst rasiert, dann rasiert er sich nicht selbst, und wenn er sich nicht selbst rasiert, dann rasiert er sich selbst.
	Was denn nun?
	-- Quelle unbekannt) --
	Das Problem ist verwandt mit dem Problem der \enquote{Menge aller Mengen, die sich nicht selbst enthalten}.%
}.
Um diese Probleme in den Griff zu bekommen, wird die \glsIdx{Metasprache} teilweise formalisiert.
Durch diese Formalisierung erinnert sie dann teilweise schon an mathematische Formeln.
Die Sprachebenen sollten aber sorgfältig unterschieden werden.

\subsection{Metasprachliche Ausdrücke}% ========================================
\beginsection{\glsIdxPl{MetaausdruckV}}
\label{sub:Metaausdruck}

Ein \emph{\glsIdx{MetaausdruckV}} ist eine in normaler Sprache verfasste Aussage, wie \textzB
(a) \strqt{Morgen scheint die Sonne.},
(b) \strqt{Ich bin 1,83\,m groß.},
(c) \strqt{Ich habe ein rotes Auto und das kann 200\,km/h schnell fahren.},
\textusw
In einem erweiterten Sinne gehören auch Relationen einschließlich ihrer Operanden dazu%
\footnote{%
	Wird statt des Symbols der Name der zugehörigen Relation verwendet, ist dies unmittelbar einleuchtend.
	So wird \textzB aus der Formel \forqt{$A<B$} die \glsIdx{MetaaussageV} \strqt{$A$ ist kleiner als $B$}.%
},
wie \textzB \forqt{$A=A$}, \forqt{$A \equiv B$}, \forqt{$A<B$}, \textusw

Während die Beispiele (a) und (b) einfache, nicht mehr zerlegbare \glsIdxPl{MetaausdruckV} sind, ist Beispiel (c) zusammengesetzt.
Für alle drei Aussagen lässt sich feststellen, ob sie richtig sind oder nicht.
Das kann man für den zweiten Teil von (c) allein aber nicht, wenn man nicht weiss worauf sich \strqt{das} bezieht.
Natürlich muss auch der Zusammenhang, in dem ein \glsIdx{MetaausdruckV} formuliert wird, bekannt sein, denn \textzB ist die Bedeutung von \strqt{Ich} nur dann bekannt, wenn man weiß von wem die Aussage ist.
Auf eine exakte Definition von \strqt{\glsIdx{MetaausdruckV}} wird verzichtet, weil das intuitive Verständnis hier ausreicht.
In erster Näherung können aber alle sprachlichen Ausdrücke, die im Prinzip überprüft werden können, als \glsIdxPl{MetaausdruckV} betrachtet werden.

Zusammengesetzte \glsIdxPl{MetaausdruckV} wie (c) können zum Teil formalisiert werden.
Dies wird mit den folgenden Definitionen erreicht:

Sind $A$ und $B$ \glsIdxPl{MetaausdruckV}, dann wird definiert:
\begin{itemize}
	
	\item \forqt{$A \glsSym{metaimp}   B$} steht für
	\strqt{\emph{Wenn} $A$ [gilt] \emph{dann} [gilt] [auch] $B$}.
	
	\item \forqt{$A \glsSym{metarep}   B$} steht für
	\strqt{$A$ [gilt] \emph{sofern} $B$ [gilt]}.
	
	\item \forqt{$A \glsSym{metaequiv} B$} steht für
	\strqt{$A$ [gilt] \emph{genau dann wenn} $B$ [gilt]}.
	
	\item \forqt{$A \glsSym{metaand}   B$} steht für
	\strqt{[Es gilt] $A$ \emph{und} $B$}.
	
	\item \forqt{$A \glsSym{metaor}    B$} steht für
	\strqt{[Es gilt] $A$ \emph{oder} $B$}.
	
\end{itemize}

Offensichtlich sind das alles ebenfalls \glsIdxPl{MetaausdruckV}, jetzt aber zumindest teilweise formalisiert.
(c) lässt sich dann ausdrücken als \strqt{\strqt{Ich habe ein rotes Auto} $\metaund$ \strqt{das kann 200\,km/h schnell fahren.}}.

Um Verwechslungen mit den logischen Symbolen zu vermeiden, werden für \strqt{und} und \strqt{oder} die Symbole \symqt{$\metaund$} und \symqt{$\metaoder$} verwendet.
Ein Symbol für \strqt{nicht} wird hier nicht gebraucht.

\GlsIdxPl{MetaausdruckV} können auch geklammert werden, um die Reihenfolge der Auswertung eindeutig zu machen.
\symqt{$\metaimp$}, \symqt{$\metarep$}, \symqt{$\metaequiv$}, \symqt{$\metaand$} und \symqt{$\metaoder$} heißen \emph{\glsIdxPl{MetaoperatorV}}.
Ihre Prioritäten werden im \subsectionname~\vref{sub:Klammerregeln} zusammen mit anderen Operatoren definiert.

Sollen zwei \glsIdxPl{MetaausdruckV} miteinander verglichen werden, muss klar sein auf welche Art; ob \textzB als Zeichenfolgen -- mit oder ohne Wertung der Zwischenräume --, als Wahrheitswerte oder auf sonstige Art.
Wenn die Art des Vergleichs implizit oder explizit klar ist und sich die beiden Ausdrücke damit vergleichen lassen, heißen sie \emph{\glsIdx{vergleichbar}}.

\subsection{Mit Gleichheit verwandte Symbole}% ---------------------------------
\label{sub:Gleichheit}

In diesem \sectionname{} wird vorausgesetzt:
\begin{itemize}
	
	\item {$=$}, $\ne$, $\equiv$, $\defeq$ und $\metadefeq$%
	werden im selben Zusammenhang verwendet.
	
	\item $A$, $B$, $P$ und $Q$ sind \glsIdx{vergleichbar},
	\textdh Ausdrücke derselben Art.
	
\end{itemize}
Dann werden folgende Symbole definiert:
\begin{itemize}
	
	\item $\glsSym{eq}$~~(\emph{\Idx{Gleichheit}}):
	\forqt{$A = B$} heißt, dass $A$ und $B$ sich in den \glsIdxPl{intEigenschaftA} nicht unterscheiden.
	Welche das sind, ergibt sich normalerweise aus dem Zusammenhang%
	\footnote{%
		Statt von einem \emph{Zusammenhang} könnte man auch von einer \emph{Umgebung} sprechen.
		Diese Bezeichnung ist aber auch ein verbreiteter Fachbegriff, so dass auf seine Verwendung verzichtet wird.
		Die Exaktheit der Begriffe in diesem Dokument soll für Erstellung von \ASBA\ ausreichen; was darüber hinausgeht, ist nicht Inhalt dieses Dokuments.%
	}
	oder muss explizit angegeben werden.
	\textZB sind zwei Operatoren gleich, wenn sie stets denselben \emph{\gls{Wahrheitswert}} liefern.
	
	Inwieweit die Begriffe \emph{Gleichheit} und \emph{Identität} korrelieren, wird hier nicht erörtert.
	\seename~\cite{bib:Identitaet}
	
	\item $\glsSym{ne}$~~(\emph{\Idx{Ungleichheit}}):
	\forqt{$A \ne B$} heißt, dass $A$ und $B$ sich in mindestens einer der \glsIdxPl{intEigenschaftA} unterscheiden.
	
	\item $\glsSym{equiv}$~~(\emph{\Idx{Äquivalenz}}):
	\forqt{$A \equiv B$} heißt, dass $A$ und $B$ sich in den \glsIdxPl{intEigenschaftA} nicht unterscheiden.
	Welche das sind, ergibt sich wie bei \symqt{$=$} aus dem Zusammenhang oder wird explizit angegeben.
	Werden \symqt{$=$} und \symqt{$\equiv$} im selben Zusammenhang verwendet, muss mit \forqt{$A=B$} stets auch \forqt{$A \equiv B$} gelten, \textdh alle \glsIdxPl{intEigenschaftA} von \symqt{$\equiv$} müssen auch \glsIdxBg{intEigenschaftA}{interessierende Eigenschaften} von \forqt{$A=B$} sein.
	
	\item $\glsSym{defeq}$~~(\emph{\Idx{Definition}}):
	\forqt{$A \defeq B$} heißt, dass $A$ definitionsgemäß gleich $B$ ist.
	Gewissermaßen ist $A$ nur eine andere Schreibweise für $B$.
	$A$ und $B$ können sich gegenseitig ersetzten.
	
	Nach dieser Definition sind $P$ und $Q$ schon dann gleich, wenn nach der Ersetzung aller Vorkommen von $A$ in $P$ und $Q$ durch $B$ die resultierenden Ausdrücke $\overline{P}$ und $\overline{Q}$ gleich sind.
	
	\item $\glsSym{metadefeq}$~~(\emph{\Idx{Metadefinition}}):
	\forqt{$A \metadefeq B$} heißt, dass der Metaausdruck $A$ definitionsgemäß gleich dem Metaausdruck $B$ ist, wobei $B$ auch eine Definition in natürlicher Sprache sein kann.
	Oft wird damit Gleichheit (\symqt{$=$}), Äquivalenz (\symqt{$\equiv$}) oder eine andere Relation definiert.
	
\end{itemize}

Es gilt dann:
\begin{itemize}
	
	\item Die interessierenden Eigenschaften für \symqt{$\equiv$} sind auch solche für \symqt{$=$}.
	
	\item Für \symqt{$=$} können -- müssen aber nicht -- mehr Eigenschaften von Interesse sein als für \symqt{$\equiv$}.
	
	\item $(A = B) \metaor (A \ne B)$
	
	\item $(A = B) \metaimp (A \equiv B)$
	
\end{itemize}

Und unter den Voraussetzungen:
\begin{itemize}
	
	\item $A \defeq B$
	
	\item $\overline{P}$ ergibt sich aus $P$ durch Ersetzung aller Vorkommen (Teilausdrücke) $A$ durch $B$.
	
	\item $\overline{Q}$ ergibt sich aus $Q$ durch Ersetzung aller Vorkommen (Teilausdrücke) $A$ durch $B$.
	
\end{itemize}

gilt schließlich:
\begin{itemize}
	\item $(\overline{P} = \overline{Q}) \metaequiv (P = Q)$
\end{itemize}

Schließlich sei noch
\begin{equation}
\glsSym{asM} \defeq \{
\glsIdxBg{metaand}{\metaund}, \glsIdxBg{metaor}{\metaoder},
\glsSym{metaimp}, \glsSym{metaequiv}, \glsSym{metarep},
\glsSym{eq}, \glsSym{ne}, \glsSym{equiv}, \glsSym{defeq} \}
\label{for:asM}
\end{equation}
die Menge der Metaoperatoren und der mit Gleichheit verwandten Symbole

\section{Formale Elemente}% ====================================================
\beginsection{Formale Elemente}
\label{sec:Formalelement}

Ein \emph{\glsIdx{FormalelementV}} kann \textzB eine Menge, Zeichenfolge, Zahl, Formel, \textusw sein.
Zwei \glsIdxPl{FormalelementV} $A$ und $B$ sind \emph{\glsIdx{vergleichbar}}, wenn beide von derselben Art sind, \textdh wenn \textzB jeweils beide Mengen, Zeichenfolgen, Zahlen oder Formeln -- die vergleichbare Ergebnisse liefern -- sind.

Intuitiv scheint klar zu sein, was damit  gemeint ist.
Wenn aber entschieden werden muss, ob \textzB (a) \strqt{1+1} gleich \strqt{2} oder (b) \strqt{1+1} gleich \strqt{1 + 1} ist, muss man erst entscheiden, von welcher Art die beiden zu vergleichenden Ausdrücke sind, \textdh \emph{wie} verglichen wird.
Wenn sie als jeweiliges Ergebnis der beiden Formeln verglichen werden, dann ist (a) richtig.
Wenn sie als Formeln, \textdh als Zeichenfolgen, verglichen werden ist (a) falsch.
Wenn die Ausdrücke in (b) als Zeichenfolgen verglichen werden, ist (b) dann richtig, wenn der Zwischenraum zwischen den einzelnen Zeichen nicht zählt.
Wenn er aber zählt, ist (b) falsch.

Im Zusammenhang mit binären Relationen werden noch einige Verabredungen getroffen.
Dazu seien \symqt{$\glsSym{relbsp}$}, \symqt{$\glsSym{releqbsp}$}, \symqt{$\glsSym{lrelbsp}$}, \symqt{$\glsSym{rrelbsp}$}, 	\symqt{$\glsSym{lreleqbsp}$} und \symqt{$\glsSym{rreleqbsp}$} Beispielsymbole für Relationen und \symqt{$\glsSym{eq}$} und \symqt{$\glsSym{ne}$} die in \sectionname~\vref{sub:Gleichheit} definierten Symbole für Gleichheit und Ungleichheit.
Wenn dann nichts anderes gesagt wird gelte stets:
\begin{align}
& ((A \relbsp   B) \metaor (A = B)) & \metaequiv &&& (A \releqbsp  B)
\label{eq:coreleq}   \\
& (A \lrelbsp   B)                  & \metaequiv &&& (B \rrelbsp   A)
\label{eq:colrrel}   \\
& (A \lreleqbsp B)                  & \metaequiv &&& (B \rreleqbsp A)
\label{eq:colrreleq} \formulatoleft
\end{align}

Mit der Definition einer Relation der einen Seite ist damit automatisch auch die der anderen Seite erfolgt, mit der Ausnahme, dass man \forqt{$A \relbsp B$} so nicht mit Hilfe von \forqt{$A \releqbsp B$} definieren kann.
Dies könnte man zwar mit Hilfe des Ansatzes
\begin{align}
& (A \relbsp B) &\formulaspace \metaequiv &&&
(A \releqbsp B) \metaand (A \ne B) \label{eq:corel} \formulatoleft
\end{align}
versuchen, aber die so definierte Relation \symqt{$\relbsp$} kann, muss aber nicht mit der in \ref{eq:coreleq} übereinstimmen.
Allerdings lässt sich \ref{eq:coreleq} aus \ref{eq:corel} ableiten und wenn \forqt{$(A \eq B) \metaimp (A \releqbsp B)$} gilt, auch \ref{eq:corel} aus \ref{eq:coreleq}.
-- Auf einen Beweis wird hier verzichtet.

Es sei noch angemerkt, dass wegen \vref{eq:corel} die Definition von \symqt{$\metarep$} in \sectionname~\vref{sub:Metaausdruck} überflüssig ist und wegen der Klammerregeln (\seename \subsectionname~\vref{sub:Klammerregeln}) auch alle Klammern in diesem \sectionname~\ref{sec:Formalelement}.
Die Prioritäten der Operatoren \symqt{$\lrelbsp$}, \symqt{$\rrelbsp$}, \symqt{$\lreleqbsp$} und \symqt{$\rreleqbsp$} unterscheiden sich normalerweise nicht; ebensowenig die der Operatoren \symqt{$\relbsp$} und \symqt{$\releqbsp$}, die aber durchaus verschieden von den Prioritäten von \symqt{$=$} und \symqt{$\ne$} sein können.

Als Beispielsymbol für binäre Operatoren wird \symqt{$\opbsp$} verwendet.
Damit zusammenhängende Verabredungen werden hier nicht getroffen.

\section{Schlussregeln}% =======================================================
\beginsection{Schlussregeln}
\label{sec:Schlussregeln}
\hidden{\glsIdx{Schlussregel}}

\todo{Schlussregeln bearbeiten.}%%%
%TODO Schlussregeln bearbeiten. %%%

\section{Aussagenlogik}% =======================================================
\beginsection{Aussagenlogik}
\label{sec:Aussagenlogik}
\hidden{\glsIdx{Aussagenlogik}}

\subsection{Konstante und Operatoren}% -----------------------------------------
\label{sub:Operatoren}

Die \tablename~\vref{tab:Symbole}
\footnote{%
	Die \tablename{} basiert auf den Wahrheitstafeln in~\cite{bib:Junktor} Kapitel~2.2 und~\cite{bib:Rautenberg} Kapitel~1.1 Seite~3.%
}
definiert für die zweiwertige Logik Konstanten- und Operatorsymbole über die Wahrheitswerte ihrer Anwendung.
So ergeben sich, abhängig von den Wahrheitswerten der Operanden $A$ und $B$
\footnote{%
	Im Gegensatz zu \subsubsectionname~\vref{subsub:Bausteine} können A und B hier beliebige Aussagen -- auch Formeln -- sein.%
},
die in der \tablename{} angegebenen Wahrheitswerte für die Operationen.
Die mit 0, 1 und 2 benannten Spalten werden jeweils nur für die 0-, 1- und 2-stelligen Operatoren, \textdh für die Konstanten, die unären und die binären Operatoren ausgefüllt.
Dabei werden die Konstanten als 0-stellige Operatoren angesehen.
Hat der Inhalt einer Zelle keine Relevanz, steht dort ein Minuszeichen, ist kein Wert bekannt, so bleibt sie leer.

% ==============================================================================
% Definitionen für die folgende Tabelle; siehe auch die Symbole im Vorspann
% Prioritäten - jeweils Prio p* für Symbol \l* ---------------------------------
\newcounter{prio}                                        \stepcounter{prio}
\newcounter{pnequiv} \setcounter{pnequiv} {\value{prio}}
\newcounter{pequiv}  \setcounter{pequiv}  {\value{prio}} \stepcounter{prio}
\newcounter{pnrep}   \setcounter{pnrep}   {\value{prio}}
\newcounter{prep}    \setcounter{prep}    {\value{prio}}
\newcounter{pnimp}   \setcounter{pnimp}   {\value{prio}}
\newcounter{pimp}    \setcounter{pimp}    {\value{prio}} \stepcounter{prio}
%	\newcounter{pnleft}  \setcounter{pnleft}  {\value{prio}}
%	\newcounter{pleft}   \setcounter{pleft}   {\value{prio}}
%	\newcounter{pnright} \setcounter{pnright} {\value{prio}}
%	\newcounter{pright}  \setcounter{pright}  {\value{prio}} \stepcounter{prio}
\newcounter{padd}    \setcounter{padd}    {\value{prio}}
%	\newcounter{pnxor}   \setcounter{pnxor}   {\value{prio}}
\newcounter{pxor}    \setcounter{pxor}    {\value{prio}}
\newcounter{pnor}    \setcounter{pnor}    {\value{prio}}
\newcounter{por}     \setcounter{por}     {\value{prio}} \stepcounter{prio}
\newcounter{pmult}   \setcounter{pmult}   {\value{prio}}
\newcounter{pnand}   \setcounter{pnand}   {\value{prio}}
\newcounter{pand}    \setcounter{pand}    {\value{prio}} \stepcounter{prio}
\newcounter{pnot}    \setcounter{pnot}    {\value{prio}}
% Farben
\definecolor{cNormalUse}{rgb}{.80,.80,.80}
\definecolor{cRareUse}{rgb}{.90,.90,.99}
% ==============================================================================

\begin{table}
	\newcommand*{\tablegroup}{\hdashline[6pt/3pt]}
	\newcommand*{\tableline}{\hdashline[3pt/3pt]}
	\newcommand*{\gapline}{%
		\cdashline{1-1}[1pt/3pt]\cdashline{9-11}[1pt/3pt]}
	\setlength\tabcolsep{3pt}
	\setlength\extrarowheight{1.5pt}
	\begin{threeparttable}
		\begin{tabularx}{\linewidth-10.95pt}{c||c:cc:cccc|X:X|c|}
			
			A & - & \texttrue & \textfalse &%
			\texttrue  & \texttrue  & \textfalse & \textfalse &
			- & Aussage A & - \\
			
			\tableline%.................................................
			B & - & -       & -        &%
			\texttrue  & \textfalse & \texttrue  & \textfalse &
			- & Aussage B & - \\
			
			\hline% -- Überschrift -----------------------------------------
			
			\textbf{Junktor}\tnote{1}&\textbf{0}&\multicolumn{2}{c:}{%
				\textbf{1}}&\multicolumn{4}{c|}{\textbf{2}}& \textbf{%
				Name}&\textbf{Sprechweise}\tnote{2}&\textbf{Prio}\\%Kein Fehler!
			
			\hline\hline% == Konstante =====================================
			
			\rowcolor{cRareUse}
			$\glsSym{ltrue}$
			& \texttrue  & - & - & - & - & - & - & Verum  & Wahr   & - \\
			
			\tableline%.................................................
			
			\rowcolor{cRareUse}
			
			$\glsSym{lfalse}$
			& \textfalse & - & - & - & - & - & - & Falsum & Falsch & - \\
			
			\hline% -- unäre Operatoren ------------------------------------
			
			& - & \texttrue  & \texttrue  & - & - & - & -
			&                     &                  & -                 \\
			
			\tableline%.................................................
			
			\rowcolor{cNormalUse}
			
			$\Sym{(\dots)}$
			& - & \texttrue  & \textfalse & - & - & - & -
			& Klammerung\tnote{3} & A ist geklammert & 6\tnote{4}        \\
			
			\tableline%.................................................
			
			\rowcolor{cNormalUse}
			$\Sym{\lnot}$
			& - & \textfalse & \texttrue  & - & - & - & -
			& Negation            & Nicht A          & \thepnot\tnote{5} \\
			
			\tableline%.................................................
			
			& - & \textfalse & \textfalse & - & - & - & -
			&                     &                  & -                 \\
			
			\hline% -- binäre Operatoren -----------------------------------
			
			~ & - & - & - &\texttrue&\texttrue&\texttrue&\texttrue
			& Tautologie
			&
			& - \\
			
			\tableline%.................................................
			
			\rowcolor{cNormalUse}
			
			$\Sym{\lor}$
			& - & - & - &\texttrue&\texttrue&\texttrue&\textfalse
			& Disjunktion; Adjunktion;\newline Alternative
			& A oder B
			& \thepor \\
			
			\tableline%.................................................
			
			\rowcolor{cRareUse}
			$\Sym{\lrep}$ $\lrepA$ $\lrepB$
			& - & - & - &\texttrue&\texttrue&\textfalse&\texttrue
			& Replikation; Konversion;\newline konverse Implikation
			& A folgt aus B
			& \theprep \\
			
			\tableline%.................................................
			
			$\lleft$
			& - & - & - &\texttrue&\texttrue&\textfalse&\textfalse
			& Präpendenz
			& Identität von A
			& - \\
			
			\tablegroup% -----------------------------------------------
			
			\rowcolor{cNormalUse}
			$\Sym{\limp}$ $\limpA$ $\limpB$
			& - & - & - &\texttrue&\textfalse&\texttrue&\texttrue
			& Implikation; Subjunktion;\newline Konditional
			& Wenn A so B; Aus A folgt B; A nur dann wenn B
			& \thepimp \\
			
			\tableline%.................................................
			
			$\lright$
			& - & - & - &\texttrue&\textfalse&\texttrue&\textfalse
			& Postpendenz
			& Identität von B
			& - \\
			
			\tableline%.................................................
			
			\rowcolor{cNormalUse}
			$\Sym{\lequiv}$ $\lequivA$
			& - & - & - &\texttrue&\textfalse&\textfalse&\texttrue
			& Äquivalenz; Bijunktion;\newline Bikonditional
			& A genau dann wenn B; A dann und nur dann wenn B
			& \thepequiv \\
			
			\tableline%.................................................
			
			\rowcolor{cNormalUse}
			$\Sym{\land}$ $\landA$ $\landB$
			& - & - & - &\texttrue&\textfalse&\textfalse&\textfalse
			& Konjunktion
			& {\small A und B; Sowohl A als auch B}
			& \thepand \\
			
			\tablegroup% -----------------------------------------------
			
			\rowcolor{cRareUse}
			$\Sym{\lnand}$ $\lnandA$ $\lnandB$
			& - & - & - &\textfalse&\texttrue&\texttrue&\texttrue
			& NAND; Unverträglichkeit;\newline Sheffer-Funktion
			& Nicht zugleich A und B
			& \thepnand \\
			
			\tableline%.................................................
			
			\rowcolor{cRareUse}
			$\Sym{\lxor}$ $\lxorA$ $\lxorB$ $\lxorC$
			& - & - & - &\textfalse&\texttrue&\texttrue&\textfalse
			& XOR; Antivalenz;\newline ausschließende Disjunktion
			& Entweder A oder B
			& \thepxor \\
			
			\gapline%. . . . . . . . . . . . . . . . . . . . . . . . . .
			
			$\lnequiv$ $\lnequivA$ $\lnequivB$
			& - & - & - &"&"&"&"
			& Kontravalenz
			&
			& - \\
			
			\tableline%.................................................
			
			$\lnright$
			& - & - & - &\textfalse&\texttrue&\textfalse&\texttrue
			& Postnonpendenz
			& Negation von B
			& - \\
			
			\tableline%.................................................
			
			$\lnimp$ $\lnimpA$ $\lnimpB$
			& - & - & - &\textfalse&\texttrue&\textfalse&\textfalse
			& Postsektion
			&
			& - \\
			
			\tablegroup% -----------------------------------------------
			
			$\lnleft$
			& - & - & - &\textfalse&\textfalse&\texttrue&\texttrue
			& Pränonpendenz
			& Negation von A
			& - \\
			
			\tableline%.................................................
			
			$\lnrep$ $\lnrepA$ $\lnrepB$
			& - & - & - &\textfalse&\textfalse&\texttrue&\textfalse
			& Präsektion
			&
			& - \\
			
			\tableline%.................................................
			
			\rowcolor{cRareUse}
			$\Sym{\lnor}$ $\lnorA$
			& - & - & - &\textfalse&\textfalse&\textfalse&\texttrue
			& NOR; Nihilation;\newline Peirce-Funktion
			& Weder A noch B
			& \thepnor \\
			
			\tableline%.................................................
			
			~
			& - & - & - &\textfalse&\textfalse&\textfalse&\textfalse
			& Kontradiktion
			&
			& - \\
			
			\hline%_____________________________________________________________
		\end{tabularx}
		\begin{tablenotes}
			\footnotesize
			
			\item[1] \emph{Operatorsymbole}.
			Sie stehen meistens für die Operatoren selbst.
			Der Einfachheit halber bezeichnen wir auch die beiden Konstanten $\ltrue$ und $\lfalse$ als Junktoren \textbzw Operatorsymbole.
			
			Die Operatoren \symqt{$\subset$}, \symqt{$\supset$}, \symqt{$\nsubset$} und \symqt{$\nsupset$} haben hier nicht die Bedeutung der entsprechenden Operatoren der Mengenlehre und dürfen nicht damit verwechselt werden; entsprechendes gilt für \symqt{$+$} und \symqt{$\cdot$} mit Addition und Multiplikation.
			
			\item[2] Ist eine Zelle in dieser Spalte leer, so ist die zugehörige Zeile nur vorhanden, um alle binären Operationen aufzuführen.
			
			\item[3] Klammerung ist genau genommen kein Operator und wird nicht nur bei logischen, sondern auch bei anderen Ausdrücken verwendet.
			
			\item[4] Die Priorität der Klammern ist größer als die aller Operatoren.
			
			\item[5] Die Priorität der unären Operatoren muss größer sein als die aller mehrwertigen, also auch der binären Operatoren.
			Wenn alle unären Operatoren auf derselben Seite des Operanden stehen, brauchen sie eigentlich keine Priorität, da die Auswertung nur von innen (dem Operanden) nach außen erfolgen kann.
			Nur wenn es sowohl links-, als auch rechtsseitige unäre Operatoren gibt, muss man für diese Prioritäten definieren.
			
		\end{tablenotes}
		\caption{Definition von aussagenlogischen Symbolen.}
		\label{tab:Symbole}% Erst nach '\caption'!
	\end{threeparttable}
\end{table}

Für einige Junktoren, Namen und Sprechweisen sind auch Alternativen angegeben.
Die durchgestrichenen (\textdh negierten) Symbole sind ungebräuchlich und nur aus formalen Gründen aufgeführt.
Wenn für eine bestimmte Kombination von Wahrheitswerten mehr als eine Zeile angegeben ist, so sind die zugehörigen Operationen in der zweiwertigen Aussagenlogik alle gleich.
Bei der formalen Definition setzen wir aber keine Zweiwertigkeit voraus, so dass je nach Definition die Operationen verschiedene Ergebnisse liefern können.

Um vollständig zu sein, \textdh alle 22 möglichen Kombinationen von Wahrheitswerten für höchstens zwei Variable zu berücksichtigen, enthält die \tablename{} auch viele ungebräuchliche Junktoren und Operationen.
Die Zeilen mit den Klammern und den gebräuchlichsten Junktoren sind in der \tablename{} grau hinterlegt.
Hellgrau hinterlegt sind Zeilen mit weniger gebräuchlichen Junktoren.
Die restlichen Operationen sind uninteressant und brauchen daher keine Priorität.

\subsection{Klammerregeln}% ----------------------------------------------------
\label{sub:Klammerregeln}

Zur Klammerersparnis werden die üblichen Regeln verwendet, \textdh dass Operatoren mit höherer Priorität stärker binden, als solche mit niedrigerer Priorität.

Für die Operatoren derselben Priorität gilt Rechtsklammerung%
\footnote{%
	Unäre Operatoren stehen hier stets links \emph{vor} dem Operanden, so dass es nur Rechtsklammerung geben kann.
	Zur Rechtsklammerung bei binären Operationen ein Zitat aus~\cite{bib:Rautenberg} Kapitel~1.1 Seite~5:
	\enquote{Diese hat gegenüber Linksklammerung Vorteile
		bei der Niederschrift von Tautologien in $\limp$, [...]}%
}.
Im Folgenden wird nur noch ein Teil der logischen Operatoren aus der \tablename~\vref{tab:Symbole} und der \glsIdxBg{MetaoperatorV}{metasprachlichen Operatoren} aus \subsectionname~\vref{sub:Metaausdruck} berücksichtigt.
Diese werden in der \tablename~\vref{tab:Prioritaeten} mit abnehmender Priorität aufgelistet.
\begin{table}
	\begin{center}
		\begin{tabular}[c]{|l|l|}
			\hline
			Klammern
			& $      (      \quad      )                          $ \\
			\hline
			Unäre logische Operatoren
			& $ \Sym{\lnot}                                       $ \\
			\hdashline
			& $ \Sym{\land} \quad \Sym{\lmult} \quad \Sym{\lnand} $ \\
			Binäre logische Operatoren
			& $ \Sym{\lor}  \quad \Sym{\ladd}  \quad \Sym{\lnor}  $ \\
			& $ \Sym{\lrep} \quad \Sym{\limp}                     $ \\
			& $ \Sym{\lequiv}                                     $ \\
			\hline
			\parbox[][1.5cm][c]{6.2cm}{%
				Mit Gleichheit verwandte Symbole;
				\small ihre Prioritäten untereinander sind nicht eindeutig
				und bleiben daher undefiniert.
			}
			& $ \glsSym{eq} \quad \glsSym{ne} \quad
			\glsSym{equiv}  \quad \glsSym{defeq}        $ \\
			\hline
			& $ \glsSym{metaor}                         $ \\
			\GlsIdxPl{MetaoperatorV}
			& $ \glsSym{metaand}                        $ \\
			& $ \glsSym{metarep} \quad \glsSym{metaimp} $ \\
			& $ \glsSym{metaequiv}                      $ \\
			\hline
			Metasprachliche Definition
			& $ \glsSym{metadefeq}                      $ \\
			\hline
			\parbox[][1.1cm][c]{5.9cm}{%
				Strukturelemente der natürlichen Sprache, \textzB Satzzeichen
			}
			& . \quad , \quad ; \quad \textusw \\
			\hline
		\end{tabular}
	\end{center}
	\vspace{-0.5cm}
	\caption{Prioritäten von Operatoren in abnehmender Reihenfolge}
	\label{tab:Prioritaeten}% Erst nach '\caption'!
\end{table}

Die Prioritäten der logischen Operatoren wurden aus~\cite{bib:Rautenberg} Kapitel~1.1 Seite~5 entnommen und ergänzt und die der \glsIdxBg{MetaoperatorV}{metasprachlichen Operatoren} daran angeglichen.

\subsection{Formalisierung}% ---------------------------------------------------
\label{sub:Formalisierung}

Da sie die Grundlage -- quasi das Fundament -- des mathematischen Inhalts von \ASBA\ sind, müssen die \glsIdxPl{Axiom}, \glsIdxPl{Satz}, \glsIdxPl{Beweis}, \textusw der Aussagenlogik in streng formaler Form vorliegen.
Die Formalisierung stützt sich auf~\cite{bib:Aussagenlogik}; \alsoname~\cite{bib:LogikDe, bib:LogikEn}.
Da Computerprogramme mit der \emph{Polnischen Notation}\idx{Polnische Notation}%
\footnote{%
	Bei der \emph{Polnischen Notation} wird eine zweistellige Operation $(A\circ B)$ dargestellt als $\circ A B$.
	Eine Zwischenstufe ist $\circ(A,B)$, bei der noch die redundanten Gliederungszeichen Komma und Klammern -- auch andere als die runden -- hinzukommen, so dass die Operationen optisch besser getrennt und dadurch für Menschen besser lesbar werden.
	Durch einfaches Weglassen der Gliederungszeichen ergibt sich dann die Polnische Notation.%
}
besser umgehen können und Klammern dort überflüssig sind, werden viele Formeln auch in die Polnische Notation überführt.

\subsubsection{Bausteine der aussagenlogischen Sprache}% - - - - - - - - - - - -
\label{subsub:Bausteine}

Zur Einteilung der aussagenlogischen Junktoren werden die folgenden Mengen definiert:
\begin{align}
%
& \gsNo\hidden{\gls{gsNo}}  & & \defeq & &
&     & \text{Menge der \emph{natürlichen Zahlen} einschließlich 0}
\idx{natürlichen Zahlen, Menge der} \label{def:N} \\
%
& \glsSym{asC}  & & \defeq & & \{ \ltrue, \lfalse \}
& ,\; & \text{Menge der \emph{aussagenlogischen Konstanten}}
\idx{Konstanten, Menge der}         \label{def:C} \\
%
& \glsSym{asU}  & & \defeq & & \{ \lnot \}
& ,\; & \text{Menge der \emph{unären aussagenlogischen Operatoren}}
\idx{unären Operatoren, Menge der}  \label{def:U} \\
%
& \glsSym{asB}  & & \defeq & &
\{ \land, \lor, \limp, \lequiv, \lrep, \lnand, \lnor, \lmult, \ladd \}
& ,\; & \text{Menge der \emph{binären aussagenlogischen Operatoren}}
\idx{binären Operatoren, Menge der} \label{def:B}
%
\end{align}

Damit sind alle in der \tablename~\vref{tab:Symbole} verwendeten wesentlichen Konstanten und Operatoren%
\footnote{%
	Jeweils nur die ersten der grau hinterlegten Zeilen sowie \symqt{$\lmult$}.%
}
erfasst und es können die folgende Mengen definiert werden:
\begin{align}
%
& \glsSym{asV}  & & \defeq    & & \{ p_n | n \in \gsNo \}
& ,\; & \text{Menge der \emph{atomaren Formeln}}
\idx{atomaren Formeln, Menge der}         \label{def:V}  \\
%
& \glsSym{asJ}  & & \defeq    & &\asC \cup \asU \cup \asB
& ,\; & \text{Menge der \emph{Junktoren}, \textbzw \emph{Operatoren}}
\idx{Junktoren, Menge der}                \label{def:J}  \\
%
& \glsSym{asA}  & & \defeq    & & \asV \cup \asJ
& ,\; & \text{\emph{Alphabet der aussagenlogischen Sprache (für }} \asJ
\text{\emph{)}}
\idx{Alphabet der logischen Sprache}      \label{def:A}  \\
%
& \glsSym{asJx} & & \subseteq & & \asJ
& ,\; & \text{eine Teilmenge von } \asJ \text{ für eine Indexvariable }
x                                         \label{def:Bx} \\
%
& \glsSym{asAx} & & \defeq    & & \asV \cup \asJx \quad
& ,\; & \text{Alphabet der aussagenlogischen Sprache \emph{für} } \asJx
\idx{Teil-Alphabet der aussagenlogischen Sprache} \label{def:Ax}
\formulatoleft
%
\end{align}
Für Elemente aus $\asV$ werden hier normalerweise die großen lateinischen Buchstaben $A$, $B$, $C$, \textbzw verwendet.
Die Elemente aus $\asV$ (atomaren Formeln) werden auch \emph{\Idx{Satzbuchstabe}}\emph{n} oder kurz \emph{\Idx{Atom}}\emph{e}. genannt.

\subsubsection{Aussagenlogische Formeln}%  - - - - - - - - - - - - - - - - -
\label{subsub:Formeln}

Neben dem Alphabet $\asA$ \textbzw. $\asAx$ werden noch Klammern als Gliederungszeichen verwendet.
Damit können nun rekursiv für jede Teilmenge $\asJx$ von $\asJ$ zwei Mengen von Formeln definiert werden:

$\glsSym{asFx}$ sei die Menge der auf folgende Weise definierten \emph{aussagenlogischen Formeln mit Klammerung}%
\idx{aussagenlogische Formel mit Klammerung}:
\begin{align}
&                & \asV            \subset \asFx
\\
&                & \asJx \cap \asC \subset \asFx
\\
A                                \in \asFx
& \quad \metaimp &  (\circ A)    \in \asFx
& & , \quad \text{für} \quad \circ \in \asU \cap \asJx
\\
A, B                             \in \asFx
& \quad \metaimp & (A \circ B)   \in \asFx
& & , \quad \text{für} \quad \circ \in \asB \cap \asJx
\formulatoleft
\end{align}
Nur die auf diese Weise konstruierten Formeln sind Elemente von $\asFx$.
\\Für $\asJ = \asJx$ sei noch $\asF \defeq \glsSym{asFx}$.

$\glsSym{asFxp}$ sei die Menge der auf folgende Weise definierten aussagenlogischen Formeln in \emph{Polnischer Notation}%
\idx{aussagenlogische Formel in Polnischer Notation}:
\begin{align}
&                & \asV            \subset \asFxp
\\
&                & \asJx \cap \asC \subset \asFxp
\\
A                            \in \asFxp
& \quad \metaimp & \circ A   \in \asFxp
& & , \quad \text{für} \quad \circ \in \asU \cap \asJx
\\
A, B                         \in \asFxp
& \quad \metaimp & \circ A B \in \asFxp
& & , \quad \text{für} \quad \circ \in \asB \cap \asJx
\formulatoleft
\end{align}
Nur die auf diese Weise konstruierten Formeln sind Elemente von $\asFxp$.
\\Für $\asJ = \asJx$ sei noch $\asFp \defeq \glsSym{asFxp}$.

Wie man leicht sieht, gilt:
\begin{equation}
\asJx \subset \asJy \subseteq \asJ \metaimp
\begin{cases}
\asAx  \subset \asAy  \subseteq \asA \\
\asFx  \subset \asFy  \subseteq \asF \\
\asFxp \subset \asFyp \subseteq \asFp
\end{cases}
\end{equation}

Durch Anwendung der Klammerregeln von \subsubsectionname~\vref{subsub:Bausteine} lassen sich in der Regel noch viele Klammern der Formeln aus $\asFx$ einsparen.
Die Formeln aus $\asFxp$ sind frei von Klammern.
Die Namen der Operatoren finden sich in der \tablename~\vref{tab:Symbole}.
Für aussagenlogische Formeln, \textdh von Elementen aus $\asF$ \textbzw $\asFp$, werden hier normalerweise die kleinen griechischen Buchstaben $\alpha$, $\beta$, $\gamma$, \textusw verwendet.
Sie können dabei auch atomare Formeln bezeichnen (\seename \eqref{def:V}).

\subsection{Definition aussagenlogische Operatoren durch andere}% ----------
\label{sub:ausOperatorDef}

Im folgenden gelte für zwei aussagenlogische Formeln $\alpha$ und $\beta$:
\begin{itemize}
	
	\item $\alpha  \eq   \beta \quad \metadefeq$ \quad $\alpha$ und $\beta$ stimmen -- \textggf nach vollständiger Klammerung -- als Zeichenkette überein.
	
	\item $\alpha \equiv \beta \quad \metadefeq$ \quad $\alpha$ und $\beta$ können mit Hilfe erlaubter Substitutionen und geltender Axiome -- \seename~\vref{sub:ausAxiome} -- ineinander überführt werden.
	
\end{itemize}

%TODO Signatur definieren

Es werden verschiedene Teilmengen von $\asJ$ eingeführt, die jeweils ausreichen alle anderen Elemente aus $\asJ$ zu definieren:
\begin{align}
& \asJ_\xBool &\defeq & & & \{ \lnot, \land, \lor \} \label{def:Jbool}\\
& \asJ_\xAnd  &\defeq & & & \{ \lnot, \land       \} \label{def:Jand} \\
& \asJ_\xOr   &\defeq & & & \{ \lnot, \lor        \} \label{def:Jor}  \\
& \asJ_\xImp  &\defeq & & & \{ \lnot, \limp       \} \label{def:Jimp} \\
& \asJ_\xRep  &\defeq & & & \{ \lnot, \lrep       \} \label{def:Jrep} \\
& \asJ_\xNand &\defeq & & & \{ \lnand             \} \label{def:Jnand}\\
& \asJ_\xNor  &\defeq & & & \{ \lnor              \} \label{def:Jnor}
\formulatoleft\formulatoleft\formulatoleft\formulatoleft\formulatoleft
\end{align}

Im Folgenden stehen jeweils links die Formeln in üblicher Schreibweise mit Klammern und rechts in Polnischer Notation (ohne Klammern).
Ferner seien $A$ und $B$ beliebige, nicht notwendig verschiedene Formeln aus der passenden Menge $\asFx$ \textbzw der um die mit Hilfe der Definitionen erweiterten Formelmenge.

Ausgehend von den Junktoren \textbzw Operatoren aus $\asJ_\xBool$ werden die restlichen Operatoren aus $\asJ$ definiert. Die Definitionen sind in zwei Gruppen eingeteilt, und zwar die mit den Operatoren aus $\asJ_\xAnd$:
\begin{align}
% folgt ------------------------
&                (A \limp B)  &\defeq & & & (\lnot (A \land (\lnot B)))
& \formulaspace &   \limp A B &\defeq & & & \lnot \land A \lnot B
\label{def:imp}
\\
% sofern -----------------------
&                (A \lrep B)  &\defeq & & & (\lnot (B \land (\lnot A)))
& \formulaspace &   \lrep B A &\defeq & & & \lnot \land B \lnot A
\label{def:rep}
\\
% genau dann -------------------
&              (A \lequiv B)  &\defeq & & & ((A\limp B)\land(A\lrep B))
& \formulaspace & \lequiv A B &\defeq & & & \land \limp A B \lrep A B
\label{def:equiv}
\\
%falsch ------------------------
&                     \lfalse & \defeq & & & (p_0 \land (\lnot p_0))
& \formulaspace &     \lfalse & \defeq & & & \land p_0 \lnot p_0
\label{def:false}
\\
% mal --------------------------
&               (A \lmult B)  & \defeq & & & (A \land B)
& \formulaspace &  \lmult A B & \defeq & & & \land A B
\label{def:mult}
\\
% NAND -------------------------
&               (A \lnand B)  &\defeq & & & (\lnot (A \land B ))
& \formulaspace &  \lnand A B &\defeq & & & \lnot \land A B
\label{def:nand}
\formulatoleft
\end{align}
und die mit den Operatoren aus $\asJ_\xOr$:
\begin{align}
% NOR --------------------------
&                (A \lnor B)  & \defeq & & & (\lnot (A \lor B))
& \formulaspace &   \lnor A B & \defeq & & & \lnot \lor A B
\label{def:nor}
\\
% plus -------------------------
& (A \ladd B) & \defeq & & & ((A\lor B)\land(\lnot(A\land B)))
& \formulaspace &   \ladd A B & \defeq & & & \land\lor A B\lnot\land A B
\label{def:add}
\\
% wahr -------------------------
&                      \ltrue & \defeq & & & (p_0 \lor (\lnot p_0))
& \formulaspace &      \ltrue & \defeq & & & \lor p_0 \lnot p_0
\label{def:true}
\formulatoleft
\end{align}

Ist \symqt{$\lor$} oder \symqt{$\land$} nicht vorgegeben, \textdh wird von den Elementen aus $\asJ_\xAnd$ \textbzw $\asJ_\xOr$ statt von denen aus $\asJ_\xBool$ ausgegengen, so muss man den fehlenden Operator mittels der passenden der beiden folgenden Definitionen einführen:
\begin{align}
% oder aus und -----------------
&          (A \lor B) & \defeq & & & (\lnot ((\lnot A) \land (\lnot B)))
& \formulaspace & \lor  A B & \defeq & & & \lnot \land \lnot A \lnot B
\label{def:orand}
\\
% und aus oder -----------------
&          (A \land B) & \defeq & & & (\lnot ((\lnot A) \lor (\lnot B)))
& \formulaspace & \land A B & \defeq & & & \lnot \lor \lnot A \lnot B
\label{def:andor}
\formulatoleft
\end{align}
Nun sind wieder alle Operatoren definiert.

Entsprechend wird bei Vorgabe von $\asJ_\xImp$ \textbzw $\asJ_\xRep$ die passende der beiden folgenden Definitionen ausgewählt:
\begin{align}
% oder aus imp -----------------
&              (A \lor  B)  & \defeq & & & (imp)
& \formulaspace & \lor  A B & \defeq & & &  imp
\label{def:orrep}
\\
% und aus rep ------------------
&              (A \land B)  & \defeq & & & (rep)
& \formulaspace & \land A B & \defeq & & &  rep
\label{def:andrep}
\formulatoleft
\end{align}
woraufhin dann \ref{def:imp} \textbzw \ref{def:rep} als Gleichung nachzuweisen ist.
Da aus \ref{def:rep} durch Vertauschung der Variablen unmittelbar
\begin{align}
&              (A \lrep B)  & \equiv & & & (B \limp A)
& \formulaspace & \lrep A B & \equiv & & &    \limp B A
\label{eq:repimp}
\formulatoleft
\end{align}
folgt, vermindert sich der Aufwand dazu erheblich.

Bei Vorgabe von $\asJ_\xNand$ \textbzw $\asJ_\xNor$ schließlich
wird die passende Definition aus
\begin{align}
% oder aus nor -----------------
&              (A \lor  B)  & \defeq & & & (nor)
& \formulaspace & \lor  A B & \defeq & & &  nor
\label{def:ornor}
\\
% und aus nand -----------------
&              (A \land B)  & \defeq & & & (nand)
& \formulaspace & \land A B & \defeq & & &  nand
\label{def:andnand}
\formulatoleft
\end{align}
und, da \symqt{$\lnot$} weder in $\asJ_\xNand$ noch in $\asJ_\xNor$ vorhanden ist, aus
\begin{align}
% nicht aus nor ----------------
&                (\lnot A) & \defeq & & & (nor)
& \formulaspace & \lnot A  & \defeq & & &  nor
\label{def:notnor}
\\
% nicht aus nand ---------------
&                (\lnot A) & \defeq & & & (nand)
& \formulaspace & \lnot A  & \defeq & & &  nand
\label{def:notnand}
\formulatoleft
\end{align}
ausgewählt und es ist \ref{def:nand} \textbzw \ref{def:nor} als Gleichung nachzuweisen.

Abschließend ist noch nachzuweisen, dass mit Hilfe der jeweils passenden Definitionen \ref{def:imp} bis \ref{def:notnor} ausgehend vom jeweils passenden $\asFx$ genau die gesamte Formelmenge $\asF$ erzeugt werden kann.

\subsubsection{Aussagenlogisches Axiomensystem}% - - - - - - - - - - - - - - - -
\label{subsub:ausAxiome}

Gegebene Operatoren: $\lnot, \land, \limp$\par
\glsIdxPl{Axiom}:
\begin{align}
%
&(\alpha\limp\beta\limp\gamma)\limp(\alpha\limp\beta)%
\limp(\alpha\limp\gamma)
\formulaspace&&%
\limp\limp\alpha\limp\beta\gamma\limp\limp\alpha\beta%
\limp\alpha\gamma\\
%
&\alpha\limp\beta\limp\alpha\land\beta
\formulaspace&&%
\limp\alpha\limp\beta\land\alpha\beta\\
%
&\alpha\land\beta\limp\alpha ;\quad\alpha\land\beta\limp\beta
\formulaspace&&%
\limp\land\alpha\beta\alpha ;\quad\limp\land\alpha\beta\beta\\
%
&(\alpha\limp\lnot\beta)\limp(\beta\limp\lnot\alpha)
\formulaspace&&%
\limp\limp\alpha\lnot\beta\limp\beta\lnot\alpha
\formulatoleft
%
\end{align}
Definierte Operatoren: $\lor, \lequiv, \lmult, \ladd,
\lnand, \lnor, \lrep, \lfalse, \ltrue$
\begin{align}
(\alpha\lor\beta)&\defeq\lnot(\lnot\alpha\limp\beta)
\formulaspace&%
\lor\alpha\beta&\defeq\lnot\limp\lnot\alpha\beta\\
%
(\alpha\lequiv\beta)&\defeq
((\alpha\limp\beta)\land(\beta\limp\alpha))
\formulaspace&%
(\alpha\lequiv\beta)&\defeq((\alpha\limp\beta)%
\land(\beta\limp\alpha))\\
%
(\alpha\lmult\beta)&\defeq(\alpha\land\beta)
\formulaspace&%
(\alpha\lmult\beta)&\defeq(\alpha\land\beta)\\
%
(\alpha\ladd\beta)&\defeq((\alpha\lor\beta)%
\land\lnot(\alpha\land\beta))
\formulaspace&%
(\alpha\ladd\beta)&\defeq((\alpha\lor\beta)%
\land\lnot(\alpha\land\beta))\\
%
(\alpha\lnand\beta)&\defeq\lnot(\alpha\land\beta)
\formulaspace&%
(\alpha\lnand\beta)&\defeq\lnot(\alpha\land\beta)\\
%
(\alpha\lnor\beta)&\defeq\lnot(\alpha\lor\beta)
\formulaspace&%
(\alpha\lnor\beta)&\defeq\lnot(\alpha\lor\beta)\\
%
(\alpha\lrep\beta)&\defeq(\beta\limp\alpha)
\formulaspace&%
(\alpha\lrep\beta)&\defeq(\beta\limp\alpha)\\
%
\lfalse&\defeq(p_0\land\lnot p_0)
\formulaspace&%
\lfalse&\defeq(p_0\land\lnot p_0)\\
%
\ltrue&\defeq\lnot\lfalse
\formulaspace&%
\ltrue&\defeq\lnot\lfalse
\formulatoleft
%
\end{align}
Zu zeigen
\begin{align}
(\alpha\limp\beta)&\equiv\lnot(\alpha\land\lnot\beta)
\formulaspace
&\limp\alpha\beta&\equiv\lnot\land\alpha\lnot\beta
\formulatoleft
\end{align}

%%%	\subsection{Definition von Junktoren durch andere}% ------------------------
%%%	\label{sub:Junktordefinitionen}
%%%
%%%	\subsubsection{nicht, und, oder}%  - - - - - - - - - - - - - - - - - - - - -
%%%	\label{subsub:Standard}
%%%	\label{subsub:OperatorenAnfang}
%%%	Gegebene Operatoren: $\lnot, \land, \lor$\par
%%%	Definierte Operatoren:
%%%	$\limp, \lequiv, \lmult, \ladd, \lnand, \lnor, \lrep, \lfalse, \ltrue$
%%%	\begin{align}
%%%	\end{align}
%%%	Zu zeigen:
%%%	\begin{align}
%%%		(\alpha\lor\beta)&\defeq\lnot(\lnot\alpha\land\lnot\beta)
%%%	\end{align}
%%%
%%%	\subsubsection{nicht, und}%  - - - - - - - - - - - - - - - - - - - - - - - -
%%%	Gegebene Operatoren: $\lnot, \land$\par
%%%	Definierter Operator: $\lor$
%%%	\begin{align}
%%%		(\alpha\lor\beta)&\defeq\lnot(\lnot\alpha\land\lnot\beta)
%%%	\end{align}
%%%	Zu zeigen:
%%%	\begin{align}
%%%		(\alpha\lor\beta)&\equiv\lnot(\lnot\alpha\land\lnot\beta)
%%%	\end{align}
%%%	Zur Definition der Operatoren $\limp, \lequiv, \lmult, \ladd, \lnand, \lnor,
%%%	\lrep, \lfalse, \ltrue$ siehe \subsubsectionname~\vref{subsub:Standard}
%%%
%%%	\subsubsection{nicht, oder}% - - - - - - - - - - - - - - - - - - - - - - - -
%%%	Gegebene Operatoren: $\lnot, \lor$\par
%%%	Definierter Operator: $\land$
%%%	\begin{align}
%%%		(\alpha\land\beta)&\defeq\lnot(\lnot\alpha\lor\lnot\beta))
%%%	\end{align}
%%%
%%%	\subsubsection{nicht, impliziert}% - - - - - - - - - - - - - - - - - - - - -
%%%	Gegebene Operatoren: $\lnot, \limp$\par
%%%	Definierte Operatoren: $\land, \lor$
%%%	\begin{align}
%%%		(\alpha\land\beta)&\defeq\dots\\
%%%		(\alpha\lor\beta)&\defeq\dots
%%%	\end{align}
%%%
%%%	\subsubsection{NAND}%  - - - - - - - - - - - - - - - - - - - - - - - - - - -
%%%	Gegebener Operator: $\lnand$\par
%%%	Definierte Operatoren: $\lnot, \land, \lor$
%%%	\begin{align}
%%%		\lnot\alpha&\defeq\dots\\
%%%		(\alpha\land\beta)&\defeq\dots\\
%%%		(\alpha\lor\beta)&\defeq\dots
%%%	\end{align}
%%%
%%%	\subsubsection{NOR}% - - - - - - - - - - - - - - - - - - - - - - - - - - - -
%%%	\label{subsub:OperatorenEnde}
%%%	Gegebener Operator: $\lnor$\par
%%%	Definierte Operatoren: $\lnot, \land, \lor$
%%%	\begin{align}
%%%		\lnot\alpha&\defeq\dots\\
%%%		(\alpha\land\beta)&\defeq\dots\\
%%%		(\alpha\lor\beta)&\defeq\dots
%%%	\end{align}

\subsection{Aussagenlogische Axiome}% ------------------------------------------
\label{sub:ausAxiome}

\todo{Aussagenlogik weiter bearbeiten.}%%%
%TODO Aussagenlogik weiter bearbeiten. %%%

\section{Prädikatenlogik}% =====================================================
\beginsection{Prädikatenlogik}
\label{sec:Prädikatenlogik}
\hidden{\glsIdx{Praedikatenlogik}}

\todo{Prädikatenlogik bearbeiten.}%%%
%TODO Prädikatenlogik bearbeiten. %%%

\section{Mengenlehre}% =========================================================
\beginsection{Mengenlehre}
\label{sec:Mengenlehre}

\todo{Mengenlehre bearbeiten.}%%%
%TODO Mengenlehre bearbeiten. %%%

\Endchapter

	%%############################################################################%%
%%                                                                            %%
%% Datei:  ASBA-Ideen.tex                                                     %%
%% Inhalt: Kapitel "Ideen" --- Nur vorübergehend ---                          %%
%%                                                                            %%
%% Copyright (C) 2017  Winfried Teschers                                      %%
%%                                                                            %%
%% This program is free software: you can redistribute it and/or modify       %%
%% it under the terms of the GNU Affero General Public License as published   %%
%% by the Free Software Foundation, either version 3 of the License, or       %%
%% (at your option) any later version.                                        %%
%%                                                                            %%
%% This program is distributed in the hope that it will be useful,            %%
%% but WITHOUT ANY WARRANTY; without even the implied warranty of             %%
%% MERCHANTABILITY or FITNESS FOR A PARTICULAR PURPOSE.  See the              %%
%% GNU Affero General Public License for more details.                        %%
%%                                                                            %%
%% You should have received a copy of the GNU Affero General Public License   %%
%% along with this program.  If not, see <http://www.gnu.org/licenses/>.      %%
%%                                                                            %%
%% Dr. Winfried Teschers                                                      %%
%% Anton-Günther-Straße 26c                                                   %%
%% 91083 Baiersdorf                                                           %%
%% Germany                                                                    %%
%%                                                                            %%
%% e-mail: winfried.teschers@t-online.de                                      %%
%%                                                                            %%
%%############################################################################%%

% !TeX root = ASBA.tex
% !TeX encoding = UTF-8
% !TeX spellcheck = de_DE

\chapter{Ideen}% ###############################################################
\beginchapter{Ideen}
\label{cha:Ideen}

\section{Schlussregeln}% =======================================================
\beginsection{Schlussregeln}
\label{sec:Schlussregeln}
\hidden{\glsIdx{Schlussregel}}

In diesem \sectionname\ geht es um \glsIdxPl{zulaessige-Transformation}, \textdh\ \glsIdx{allgemeingueltige-Schlussregel}.
Dazu gehören zunächst die \glsIdxPl{Basisregel}.
Dann aber auch alle aus den \glsIdxPl{Basisregel} und den bis dahin \glsIdxBg{allgemeingueltige-Schlussregel}{allgemeingültigen Schlussregeln} korrekt abgeleiteten neuen \glsIdxPl{Schlussregel}.
Die \glsIdxPl{Schlussregel} haben die Form eines Formalen \glsIdx{Satz}es.

\subsection{Basisregeln}% ------------------------------------------------------
\label{sub:Basisregeln}
\hidden{\glsIdxPl{Basisregel}}

Gemäß \cite{bib:Rautenberg} Kapitel~1.4 \emph{Ein vollständiger aussagenlogischer Kalkül} werden sechs \glsIdxPl{Basisregel} definiert. Zuvor werden aber noch einige Definition gebraucht. Dazu seien $n$, $m$, $k$ und $l$ natürliche Zahlen (auch~0), $\alpha$, $\alpha_i$, $\beta$ und $\beta_j$ \glsIdxPl{Formel}, $X$, $X_i$, $Y$ und $Y_j$ Mengen von \glsIdxBg{Formel}{formalen Elementen} und
\begin{align}
	%
	&X&&\defeq&&X_1\cup X_2\cup...\cup X_n\cup\{\alpha_1,\alpha_2,...,\alpha_m\}
	\\
	&Y&&\defeq&&Y_1\cup Y_2\cup...\cup Y_k\cup\{\beta_1, \beta_2, ...,\beta_l \}
	\formulatoleft\formulatoleft
\end{align}
%
$X$ und $Y$ können auch die leere Menge sein. Damit wird definiert:
\begin{align}
	& \alpha \derive \beta \quad \metadefeq \quad
	\parbox[t]{10.5cm}{%
	$\beta$ ist mittels schrittweiser Anwendung \emph{\glsIdxBg{zulaessige-Transformation}{zulässiger Transformationen}} (siehe weiter unten) aus $\alpha$ ableitbar.
	Sprechweise: Aus $\alpha$ ist $\beta$ \emph{ableitbar} oder \emph{beweisbar};
	kurz: \enquote{$\alpha$ \emph{\glsIdx{ableitbar}} $\beta$} \textbzgl\ \enquote{$\alpha$ \emph{\glsIdx{beweisbar}} $\beta$}
	-- Es kann auch \chrqt{$\alpha$} durch \chrqt{$X$} und/oder \chrqt{$\beta$} durch \chrqt{$Y$} ersetzt werden.
	}
	\label{def:ableitbar}
	\\
	& \derive \beta \quad \metadefeq \quad \emptyset \derive \beta \qquad \text{(\chrqt{$\derivesym$} kann dann auch ganz entfallen)}
	\\
	&             X_1, X_2, ...,X_n, \alpha_1, \alpha_2, ..., \alpha_m \quad
	\derive \quad Y_1, Y_2, ...,Y_n,  \beta_1,  \beta_2,  ..., \beta_m \quad
	\metadefeq \quad X \derive Y
	\label{def:ableitbarKurz}
	\formulatoleft
\end{align}
%
Eine \emph{\glsIdx{zulaessige-Transformation}} ist die Anwendung einer \emph{\glsIdx{Substitution}}{\vrefnotesub{sub:Identitätsregeln} (siehe unten), einer \emph{\glsIdx{Basisregel}} (siehe unten) oder einer davon abgeleiteten sonstigen \emph{\glsIdx{Schlussregel}}, \textzB\ aus \vrefsub{sub:Schlussregeln}.
Bei den \glsIdxPl{Schlussregel} und der \glsIdx{Substitution} ($\subst$) soll das Komma stärker binden als \chrqt{$\derivesym$}, \chrqt{$\subst$} und \chrqt{$\srand$},
 wobei \chrqt{$\srand$} für \enquote{und} \textbzgl\ \chrqt{\glsIdx{metaand}}\vrefnotesub{sub:Aussagen} steht und schwächer bindet als \chrqt{$\derivesym$} und \chrqt{$\subst$}.%
\footnote{siehe Fußnote~3 \vrefvontab{tab:Prio-Aussagenlogik}} %TODO Kommentar zu \srand prüfen

Zur der Auswahl der \glsIdxPl{Basisregel}, der Formulierung und der Bezeichnungen wird auf~\cite{bib:Rautenberg,bib:NatuerlichesSchliessen} zurückgegriffen.
Wie in~\cite{bib:NatuerlichesSchliessen} steht \chrqt{$E$} für \enquote{-Einführung} und \chrqt{$B$} für \enquote{-Beseitigung} (oder \enquote{-Elimination}) von Operatoren.%
\footnote{%
	In der \glsIdx{Monotonieregel} wird hier, anders als in~\cite{bib:Rautenberg}, \seqqt{$X,Y$} statt \seqqt{$ Y \text{ , für } Y \supseteq X $} genommen. Das ist gleichwertig, vermeidet aber den Zusatz \seqqt{$ \text{ , für } Y \supseteq X $}.
	Außerdem werden bei den Bezeichnungen \seqqt{$(\land 1)$} und \seqqt{$(\land 2)$} gemäß~\cite{bib:NatuerlichesSchliessen} durch \seqqt{$\andE$} \textbzw\ \seqqt{$\andB$} ersetzt.
}

Im Folgenden seien $\alpha$ und $\beta$ wieder stets \glsIdxPl{Formel} und $X$ und $Y$ Mengen von \glsIdxBg{Formel}{formalen Elementen}.
Für die sechs \glsIdxPl{Basisregel} werden dann nur noch die logischen Operatoren \chrqt{$\lnot$} und \chrqt{$\land$} benötigt.
Bei den weiteren \glsIdxPl{Schlussregel} wird noch \chrqt{$\limp$} gemäß der Definition~\vref{def:imp} verwendet.
%
\begin{align}
	& \frac{}{\alpha\derive\alpha}
	& & (\text{\glsIdx{Anfangsregel}})
	\tag{\tagAR} \sym{\gls{AR}} \label{def:Anfangsregel}
	\\\\
	& \frac{X\derive\alpha}{X,Y\derive\alpha}
	& & (\text{\glsIdx{Monotonieregel}})
	\tag{\tagMR} \sym{\gls{MR}} \label{def:Monotonieregel}
	\\\\
	& \frac{X\derive\alpha,\lnot\alpha}{X\derive\beta}
	& & (\text{Einführung/Beseitigung der Negation Teil 1})
	\tag{\tagnota} \sym{\gls{nota}} \label{def:nota}
	\\\\
	& \frac{X,\alpha\derive\beta \srand X,\lnot\alpha\derive\beta}{X\derive\beta}
	& & (\text{Einführung/Beseitigung der Negation Teil 2})
	\tag{\tagnotb} \sym{\gls{notb}} \label{def:notb}
	\\\\
	& \frac{X\derive\alpha,\beta}{X\derive\alpha\land\beta}
	& & (\text{Einführung der Konjunktion})
	\tag{\tagandE} \sym{\gls{andE}} \label{def:andE}
	\\\\
	& \frac{X\derive\alpha\land\beta}{X\derive\alpha,\beta}
	& & (\text{Beseitigung der Konjunktion})
	\tag{\tagandB} \sym{\gls{andB}} \label{def:andB}
	\formulatoleft
\end{align}
%
In einer \glsIdx{Schlussregel} werden die \glsIdxBg{Formel}{formalen Elemente}%
\footnote{hier: \glsIdxPl{Aussage} in einer formalen Form.}
über dem Querstrich als \emph{\glsIdxPl{Voraussetzung}} und die unter dem Querstrich als \emph{\glsIdx{Folgerung}} der Regel bezeichnet.
Eine \glsIdx{Schlussregel} steht für die \glsIdx{Aussage}, dass mit ihren \glsIdxPl{Voraussetzung} auch auch ihre \glsIdxPl{Folgerung} gelten.
-- Im Gegensatz zu den weiteren \glsIdxPl{Schlussregel} werden die oben aufgelisteten Basisregeln nicht weiter hinterfragt, \textdh\ sie gelten quasi als \glsIdxPl{Axiom}.

\subsection{Identitätsregeln}% --------------------------------------------------------
\label{sub:Identitätsregeln}

Die zulässigen Transformationen, \textdh\ die Anwendung der \glsIdxPl{Schlussregel}, erfordern zulässige \glsIdxPl{Substitution}.
Damit wird dem Gleichheits- oder Identitätszeichen \chrqt{$\eq$} dann mittels Einführungs- und Beseitigungsregel eine Bedeutung verliehen.%
\footnote{siehe~\cite{bib:NatuerlichesSchliessen}}
Dazu seien $\alpha$, $\beta$ und $\gamma$ \glsIdxPl{vergleichbar}%
\footnote{siehe Ende \vrefvonsub{sub:Aussagen}}
\glsIdxPl{Formel}.
Zunächst wird definiert:
\begin{align}
	\gamma(\alpha \subst \beta) \quad \defeq \quad
	\parbox[t]{11cm}{%
		Das \glsIdxBg{Formel}{formale Element}, dass man erhält, wenn in $\gamma$ alle oder nur einige Vorkommen von $\alpha$ durch $\beta$ ersetzt werden.
		-- Gegebenenfalls muss noch die Auswahl der Ersetzungen angegeben werden, andernfalls werden alle Vorkommen ersetzt.
		Letzteres heißt dann \emph{vollständige} \glsIdx{Substitution}.
	} \label{def:Substitution}\\
	\gamma(\alpha \swap \beta) \quad \defeq \quad
	\parbox[t]{11cm}{%
		Das \glsIdxBg{Formel}{formale Element}, dass man erhält, wenn in $\gamma$ alle $\alpha$ und $\beta$ miteinander vertauscht werden.
		Dazu ist es nötig, das $\alpha$ und $\beta$ voneinander unabhängig sind, vorzugsweise zwei verschiedene Variable.
	} \label{def:Vertauschung}
\end{align}
\seqqt{$ \alpha \subst \beta $} heißt \emph{\glsIdx{Substitution}} und \seqqt{$ \alpha \swap \beta $} \emph{\glsIdx{Vertauschung}} oder kurz \emph{Tausch}.
-- Sei noch $S = (s_1, s_2, ...)$ eine endliche Folge von \glsIdxPl{Substitution}, die auch \glsIdxPl{Vertauschung} enthalten und auch leer sein kann. Dann wird definiert:
\begin{align}
	\gamma(S) & \quad \defeq \quad \gamma(s_1)(s_2)... \label{def:Substitutionen}\\
	\gamma(\emptyset) & \quad \; = \quad \gamma & \text{(nur zur Verdeutlichung)}\\
	\gamma(s_1,s_2,...) & \quad \defeq \quad \gamma(S)
\end{align}
%
Die \glsIdx{Vertauschung} ist eine spezielle Form der \glsIdx{Substitution}.
Wenn $x$ und $y$ zwei verschiedene Variable, die in $\alpha$, $\beta$ und $\gamma$ nicht vorkommen, gilt:
\[
	\gamma(\alpha \swap \beta) = \gamma(\alpha\subst x, \beta\subst y,  y\subst\alpha, x \subst\beta)
\]

Sei zusätzlich noch $s$ eine \glsIdx{Substitution}.
Folgende Sprechweisen werden verwendet:
\begin{itemize}
	\renewcommand*{\itemindent}{1,5cm}
	\renewcommand*{\labelsep}{5pt}
	\item [$\gamma(\alpha \subst \beta)$ :] In $\gamma$ wird $\alpha$ (vollständig) \emph{durch $\beta$ substituiert}.
	\item [$\gamma(\alpha \swap \beta)$ :] In $\gamma$ werden $\alpha$ und $\beta$ \emph{vertauscht}.
	\item [$\gamma(s)$ :] $s$ wird auf $\gamma$ \emph{angewendet}.
	\item [$\gamma(S)$ :] Die \glsIdxPl{Substitution} aus S werden in der angegebenen Reihenfolge auf $\gamma$ angewendet.
	\item [$\gamma(S)$ :] $S$ wird auf $\gamma$ angewendet.
\end{itemize}
%
Bei obiger Definition der \glsIdx{Substitution} bleibt noch offen, unter welchen \glsIdxPl{Voraussetzung} sie angewendet werden darf. Das soll hier nicht untersucht werden. In diesem \sectionname\ genügt es, das nur \glsIdx{Vertauschung} und vollständige \glsIdx{Substitution} verwendet werden.
In diesen Fällen sind beliebige \glsIdxPl{Substitution} von Variablen durch \glsIdxPl{Formel} erlaubt.

Ist $\gamma$ wie oben und $S$ eine Menge von \glsIdxPl{Substitution}.

Nun können die beiden \glsIdxPl{Identitaetsregel} definiert werden:
\begin{align}
	& \frac{}{\alpha\eq\alpha}
	& & (\text{Einführung der Identität})
	\tag{\tageqE} \sym{\gls{eqE}} \label{def:eqE}
	\\\\
	& \frac{\alpha\eq\beta \srand \gamma}{\gamma(\alpha\subst\beta)}
	& & (\text{Beseitigung der Identität})
	\tag{\tageqB} \sym{\gls{eqB}} \label{def:eqB}
	\formulatoleft
\end{align}
%
Die \glsIdxPl{Identitaetsregel} werden hier eingeführt, um die \glsIdx{Substitution} zu rechtfertigen.
Wie die \glsIdxPl{Basisregel} gelten sie als \glsIdxPl{Axiom}, würden also eigentlich dazu gehören.
Da sie aber nicht weiter verwendet werden, werden sie hier nicht zu den \glsIdxPl{Basisregel} gezählt.

\subsection{Weitere Schlussregeln}% --------------------------------------------
\label{sub:weitereSchlussregeln}

In~\cite{bib:Rautenberg} werden aus den Basisregeln mittels \glsIdxBg{zulaessige-Transformation}{zulässiger Transformationen} weitere \glsIdxPl{Schlussregel} abgeleitet.%
%TODO Identitätsregeln kommen bei Rautenberg später vor. ???
\footnote{%
	In~\cite{bib:Rautenberg} werden die \glsIdxPl{Identitaetsregel} zwar weder aufgeführt noch angewandt, ohne \glsIdx{Substitution} geht es aber nicht.
}
Man vergleiche auch mit~\cite{bib:NatuerlichesSchliessen}.
%
\begin{align}
	& \frac{X,\lnot\alpha\derive\alpha}{X\derive\alpha}
	& & (\text{Beseitigung der Negation; Indirekter \glsIdx{Beweis}})
	\tag{\tagnotc} \sym{\gls{notc}} \label{def:notc}
	\\\\
	& \frac{X,\lnot\alpha\derive\beta,\lnot\beta}{X\derive\alpha}
	& & (\text{Reductio ad absurdum})
	\tag{\tagnotd} \sym{\gls{notd}} \label{def:notd}
	\\\\
	& \frac{X,\alpha\derive\beta}{X\derive\alpha\limp\beta}
	& & (\text{Einführung der Implikation})
	\tag{\tagimpE} \sym{\gls{impE}} \label{def:impE}
	\\\\
	& \frac{X\derive\alpha\limp\beta}{X,\alpha\derive\beta}
	& & (\text{Beseitigung der Implikation})
	\tag{\tagimpB} \sym{\gls{impB}} \label{def:impB}
	\\\\
	& \frac{X\derive\alpha \srand X,\alpha\derive\beta}{X\derive\beta}
	& & (\text{\glsIdx{Schnittregel}})
	\tag{\tagSR} \sym{\gls{SR}} \label{def:SR}
	\\\\
	& \frac{X\derive\alpha \srand \alpha\limp\beta}{X\derive\beta}
	& & (\text{\glsIdx{Abtrennungsregel}--\emph{Modus ponens}})
	\tag{\tagTR} \sym{\gls{TR}} \label{def:TR}
	\formulatoleft
\end{align}
%
Dabei werden zum \glsIdx{Beweis} der \glsIdxPl{Schlussregel} in~\cite{bib:Rautenberg} folgende Basisregeln verwendet:
\begin{itemize}
	\renewcommand*{\itemindent}{1cm}
	\renewcommand*{\labelsep}{5pt}
	\item[\tagnotc~:] \tagAR, \tagMR,           \tagnotb
	\item[\tagnotd~:] \tagAR, \tagMR, \tagnota, \tagnotb
	\item[\tagimpE~:] \tagAR, \tagMR, \tagnota, \tagnotb, \tagandE
	\item[\tagimpB~:] \tagAR, \tagMR, \tagnota, \tagnotb          , \tagandB
	\item[\tagSR  ~:] \tagAR, \tagMR, \tagnota, \tagnotb
	\item[\tagTR  ~:] \tagAR, \tagMR, \tagnota, \tagnotb, \tagandE
\end{itemize}
%
\subsection{Beispiel einer Ableitung}% -----------------------------------------
\label{sub:BeispielAbleitung}

Als Beispiel wird hier die \glsIdx{Schnittregel} aus den Basisregeln abgeleitet.%
\footnote{%
	Die Form der Tabelle ist angelehnt an~\cite{bib:NatuerlichesSchliessen} Kapitel~2.2.4 \emph{Eine Beispielableitung}.
}
Dazu wird verabredet, dass \vrefintab{tab:AbleitungSchnittregel} der Inhalt der Zelle in der Zeile $i$ und der Spalte $(X_n)$ mit $X_i$ bezeichnet wird.
Zur kürzeren Darstellung wird statt auf die vollständigen Spaltenüberschriften nur auf die dort notierten $(X_n)$ verwiesen. Dass in der Spalte $(n)$ stets die Zeilennummer steht, wird im folgenden nicht mehr extra erwähnt.
-- Für die ausgefüllten Felder wird nun definiert:%
\footnote{%
	Eigentlich müsste man für jede \glsIdx{Substitution} aus $S_i$ eine eigene Zeile vorsehen.
	Um die Tabellen für die \glsIdxPl{Beweis} kürzer zu halten, werden aufeinanderfolgende \glsIdxPl{Substitution} zusammengefasst.
}
\begin{align}
	R_i & \defeq
	\left\{
		\begin{array}{l}
			\text{- \enquote{\glsIdx{Voraussetzung}} = Die \glsIdx{Aussage} $A_i$ ist eine \glsIdx{Voraussetzung}.}\\
			\text{- \enquote{\glsIdx{Folgerung}} = Die \glsIdx{Aussage} $A_i$ ist eine \glsIdx{Folgerung}.}\\
			\text{- \enquote{Annahme} = Die \glsIdx{Aussage} $A_i$ wird vorübergehend als zutreffend angenommen.}\\
			\text{- $j$ = Verweis auf die \glsIdx{Schlussregel} $\overline{R}_j$ für ein $j < i$.}\\
			\text{- Verweis (ohne Klammern) auf eine \glsIdx{allgemeingueltige-Schlussregel}.}
		\end{array}
	\right.
	\\
	S_i & \defeq \text{Die Reihe der anzuwendenden \glsIdxPl{Substitution}.}
	\\
	\overline{R}_i & \defeq \text{Das Ergebnis der in der angegebenen Reihenfolge angewendeten}\\
	& \quad\;\; \text{\glsIdxPl{Substitution} aus $S_i$ auf die \glsIdx{Schlussregel} $R_i$}
	\\
	Z_i & \defeq \text{Die Indizes $j$ (mit $j < i$) als Verweise auf eine oder mehrere \glsIdxPl{Aussage} $A_j$,}\\
	& \quad\;\; \text{welche zusammen genau die \glsIdxPl{Voraussetzung} der \glsIdx{Schnittregel} } \overline{R}_i \text{ erfüllen.}
	\\
	A_i & \defeq \text{\glsIdx{Folgerung}(en) der \glsIdx{Schlussregel} $\overline{R}_i$ --}\\
	& \quad\;\; \text{auch in Form der Indizes von einem oder mehreren von $Aj$ (mit $j < i$).}\\
	& \quad\;\; \text{In der Ergebniszeile kann hier auch die bewiesene \glsIdx{Aussage} als Schlussregel stehen.}
	\\
	D_i & \defeq \text{die Indizes der $A_j$, von denen $A_i$ abhängig ist.}
\end{align}
Bis zur Zeile $i$ hat man die folgende \glsIdx{Schlussregel} bewiesen:
\[ \frac{A_{i_1} \srand A_{i_2} ...}{A_i} \quad \text{, für alle } i_j \in D_i \]
Sei nun
\[
	\Gamma_i \defeq
	\left\{
		\begin{array}{ll}
			\text{leer}    & \text{ für } R_i = \text{\enquote{\glsIdx{Voraussetzung}}} \\
			\text{leer}    & \text{ für } R_i = \text{\enquote{\glsIdx{Folgerung}}}     \\
			\text{leer}    & \text{ für } R_i = \text{\enquote{Annahme}}       \\
			\overline{R_j} & \text{ für } R_i = j \quad \text{(eine \emph{interne} \glsIdx{Schlussregel})} \\
			\text{die \glsIdx{Schlussregel}} & \text{ für } R_i = \text{Verweis auf eine \emph{externe} \glsIdx{Schlussregel}}
		\end{array}
	\right.
\]
Damit gilt für die Einträge in einer Zeile $i$:
\begin{itemize}
	\item Wenn $\Gamma_i$ nicht leer ist, ist $R_i$ eine \glsIdx{Schlussregel} mit $R_i = \Gamma_i(S_i)$%
	\footnote{%
		siehe Definition~\eqref{def:Substitutionen} \vrefvonsub{sub:Identitätsregeln}
	}.
	\item Wenn $A_i$ nicht leer ist, ist $R_i = \dfrac{A_{z_1} \srand A_{z_2} \srand ...}{A_i}$ (alle $z_j \in Z_i$).
	\item Wenn $A_i$ nicht leer ist, ist bis jetzt die \glsIdx{Schlussregel} $\dfrac{A_{d_1} \srand A_{d_2} \srand ...}{A_i}$ (alle $d_j \in D_i$) schon bewiesen.
\end{itemize}
$S_i$, $Z_i$ und $D_i$ dürfen dabei auch leer sein.

\begin{table}[!htb]
	\setlength\tabcolsep{1pt}
	\setlength\extrarowheight{7pt}
	\newcommand*{\centerParbox}[2]{\parbox{#1}{\centering #2}}
	\newcommand*{\titleCell}[3]{\centerParbox{#1}{\textbf{#2} (#3)}}
	\newcommand*{\SnCell}[1]{\centerParbox{1.85cm}{#1}}
	\newcommand*{\DnCell}[1]{\centerParbox{1.95cm}{#1}}
	\begin{tabular}{|c||c|c|c|c|c|c|}
		\hline
		\titleCell{0.95cm}{Zeile}                       {$n$} &
		\titleCell{1.05cm}{Regel}                     {$R_n$} &
		\titleCell{1.85cm}{Substitu"=tionen}          {$S_n$} &
		\titleCell{1.80cm}{erzeugte Regel} {$\overline{R}_n$} &
		\titleCell{2.15cm}{angewendet auf ...}        {$Z_n$} &
		\titleCell{1.65cm}{\glsIdx{Aussage}}          {$A_n$} &
		\titleCell{1.95cm}{Abhängig"=keiten}          {$D_n$}
		\\\hline\hline
		1 & \centerParbox{1.35cm}{Voraus"=setzung} & & & & $X \derive \alpha$ & 1
		\\\hline
		2 & \centerParbox{1.35cm}{Voraus"=setzung} & & & & $X,\alpha \derive \beta$ & 2
		\\\hline
		3 & \centerParbox{1.00cm}{Folge"=rung} & & & & $X \derive \beta$ & 3
		\\\hline
		4 & \tagMR & & $\dfrac{X \derive \alpha}{X, Y \derive \alpha}$ & & &
		\\\hline
		5 & 4 & $Y \subst \lnot\alpha$ & $\dfrac{X \derive \alpha}{X, \lnot\alpha \derive \alpha}$ & 1 & $X, \lnot\alpha \derive \alpha$ & 1
		\\\hline
		6 & \tagAR & & $ \dfrac{}{\alpha \derive \alpha} $ & & &
		\\\hline
		7 & 6 & $\alpha \subst \lnot\alpha$ & $\dfrac{}{\lnot\alpha \derive \lnot\alpha}$ & & $\lnot\alpha \derive \lnot\alpha$ &
		\\\hline
		8 & 4 & \SnCell{%
			$\alpha \subst \lnot\alpha$\\
			$X \subst \lnot\alpha$\\
			$Y \subst X$
		} & $\dfrac{\lnot\alpha \derive \lnot\alpha}{X,\lnot\alpha \derive \lnot\alpha}$ & 7 & $X,\lnot\alpha \derive \lnot\alpha$ &
		\\\hline
		9 & \tagnota & & $\dfrac{X \derive \alpha, \lnot\alpha}{X \derive \beta}$ & & &
		\\\hline
		10 & 9 & $X \subst X, \lnot\alpha$ & $\dfrac{X,\lnot\alpha \derive \alpha, \lnot\alpha}{X,\lnot\alpha \derive \beta}$ & 5, 8 & $X,\lnot\alpha \derive \beta$ & 1
		\\\hline
		11 & \tagnotb & & $\dfrac{X,\alpha \derive \beta \srand X,\lnot\alpha \derive \beta}{X \derive \beta}$ & 2, 10 & 3 & 1, 2
		\\\hline\hline
		12 & \centerParbox{1.4cm}{\tagAR, \tagMR, \tagnota, \tagnotb} & & $\dfrac{A_1 \srand A_2}{A_3}$ & & $\dfrac{X \derive \alpha \srand X, \alpha \derive \beta}{X \derive \beta}$ &
		\\\hline
	\end{tabular}
	\caption{Ableitung der \glsIdx{Schnittregel} aus den \glsIdxPl{Basisregel}}
	\label{tab:AbleitungSchnittregel}
\end{table}

Die Erzeugung einer Tabelle analog zu~\vref{tab:AbleitungSchnittregel} wird im folgenden beschrieben.
Zellen, für die kein Inhalt angegeben wird, bleiben leer.
Rückwärts-Referenzen auf schon ausgefüllte Zellinhalte sind jederzeit möglich.
Das Eintragen der Zeilennummer $i$ wird nicht extra erwähnt.
-- Die Tabelle und die Beschreibung sind so ausführlich, damit man daraus leicht ein Computerprogramm erstellen kann.
%
\begin{enumerate}
	%
	\item Am Anfang der Tabelle werden zuerst \glsIdxPl{Voraussetzung}, dann zu beweisende \glsIdxPl{Folgerung} und schließlich Annahmen aufgeführt.%
	\footnote{%
		Die Angabe ist dann erforderlich, wenn darauf verwiesen wird.
		Durch die Auflistung hat man aber einen vollständigen Überblick über die \glsIdxPl{Voraussetzung} und \glsIdxPl{Folgerung} eines \glsIdx{Beweis}es und die Zwischenannahmen.
		Auf jede nötige \glsIdx{Voraussetzung} und jede verwendete Zwischenannahme wird in der Spalte $(Z_n$) mindestens einmal verwiesen, so dass sie auch aufgeführt werden müssen.
		Die Angabe der \glsIdxPl{Folgerung} erleichtert die Erstellung einer \emph{Ergebniszeile} (siehe Punkt~\ref{item:Ergebniszeile}).
	}
	Jede der drei Gruppen kann auch leer sein und es ist auch möglich, die Zeilen an anderen Stellen der Tabelle anzugeben, spätestens aber, wenn darauf verwiesen wird.
	Für jede \glsIdx{Voraussetzung}, \glsIdx{Folgerung} und Annahme gibt es eine Zeile:
	\begin{enumerate}
		\item $R_i =$ \enquote{\glsIdx{Voraussetzung}}, \enquote{\glsIdx{Folgerung}} oder \enquote{Annahme}.
		\item $A_i =$ Die aktuelle \glsIdx{Voraussetzung}, \glsIdx{Folgerung} oder Annahme.
		\item $D_i =$ $i$ \quad (ein Verweis auf $A_i$).
	\end{enumerate}
	%
	\item In den nächsten Zeilen werden die \glsIdxPl{Beweisschritt} aufgeführt, für jeden Schritt eine Zeile.

	Zunächst kann $R_i$ kann auf zwei Arten erzeugt werden:
	\begin{enumerate}
		\setcounter{enumii}{\value{Enumii}}% Nummerierung wird fortgesetzt.
		\item
		\begin{enumerate}
			\item $R_i =$ Verweis auf eine \glsIdx{allgemeingueltige-Schlussregel}.
			\item $\overline{R}_i =$ Die \glsIdx{Schlussregel}, auf die verwiesen wird.
		\end{enumerate}
		\setcounter{Enumii}{\value{enumii}}% Nummerierung wird fortgesetzt.
	\end{enumerate}
	oder
	\begin{enumerate}
		\item
		\begin{enumerate}
			\item $R_i = j$, wenn die schon bewiesene \glsIdx{Schlussregel} $\overline{R}_j$ (mit $j < i$) angewendet werden soll.
			\item $S_i =$ Die auf die \glsIdx{Schlussregel} $R_i$ anzuwendende \glsIdx{Substitution}.
			\item $\overline{R}_i =$ Das Ergebnis der \glsIdx{Substitution} $S_i$ auf die \glsIdx{Schlussregel} $R_i$.
		\end{enumerate}
		\setcounter{Enumii}{\value{enumii}}% Nummerierung wird fortgesetzt.
	\end{enumerate}
	Man beachte, dass die \glsIdx{Schlussregel} $\overline{R}_i$, stets allgemeingültig ist, da sie ausschließlich aus \glsIdxBg{allgemeingueltige-Schlussregel}{allgemeingültigen Schlussregeln} mittels \glsIdxPl{Substitution} abgeleitet worden ist.
	Daher gibt es auch keine Beschränkung weiterer \glsIdxPl{Substitution} durch irgendwelche Abhängigkeiten.

	Nun kann die Zeile beendet werden, oder es geht weiter mit:
	\begin{enumerate}
		\setcounter{enumii}{\value{Enumii}}% Nummerierung wird fortgesetzt.
		\item \label{item:Anwendung} $Z_n =$ Die Indizes aller $A_j$ (mit $j < i$), die eine \glsIdx{Voraussetzung} der \glsIdx{Schlussregel} $\overline{R}_i$ sind, möglichst in der verwendeten Reihenfolge.
		-- Für jedes angegebene $j$ werden noch die Abhängigkeiten $D_j$ den Abhängigkeiten $D_i$ hinzugefügt.
		%
		\item $A_i =$ \glsIdx{Folgerung}(en) der \glsIdx{Schlussregel} $\overline{R}_i$.
		-- Wenn diese \glsIdxPl{Folgerung} schon als \glsIdxPl{Aussage} $A_j$ (mit $j < i$) vorhanden sind, können auch einfach deren Indizes eingetragen werden.
		Damit werden die Zusammenhänge und der Abschluss des \glsIdx{Beweis}es besser ersichtlich.
		%
		\item $D_i =$ Die Verweise wurden schon in (\ref{item:Anwendung}) eingetragen.%
		\footnote{Wenn $D_n$ leer ist, dann ist $A_n$ allgemeingültig.}
		%
	\end{enumerate}
	Der \glsIdx{Beweis} muss so lange fortgeführt werden, bis alle \glsIdxPl{Folgerung} als \glsIdxPl{Aussage} in der Spalte $(A_n)$ erschienen und dort jeweils nur von den gegebenen \glsIdxPl{Voraussetzung} abhängig sind.
	%
	\item \label{item:Ergebniszeile} In einer \emph{Ergebniszeile}, die dann die letzte ist, kann noch die bewiesene Behauptung in Form einer \glsIdx{Schlussregel} formuliert und in einer passenden Spalte notiert werden.
	Zusätzlich können dort auch noch alle verwendeten \glsIdxPl{Schlussregel} gesammelt werden.
	Dies kann \textzB\ folgendermaßen geschehen:
	\begin{enumerate}
		%
		\item $(R_n) =$ Verweise auf alle verwendeten externen \glsIdxPl{Schlussregel}.
		%
		\item $(\overline{R}_n) =$ Die bewiesene Behauptung als \glsIdxPl{Schlussregel}, wobei alle $A_i$, die \glsIdxPl{Voraussetzung} sind, als \glsIdx{Voraussetzung} und alle $A_j$, die \glsIdxPl{Folgerung} sind, als \glsIdx{Folgerung} eingesetzt werden, jeweils in der Form \enquote{$A_i$} \textbzgl\ \enquote{$A_j$}.
		Das ergibt dann:
		\[ \frac{A_{i_1} \srand A_{i_2} \srand ...}{A_{j_1} \srand A_{j_2} \srand ...} \]
		%
		\item $(A_n) =$ $\overline{R}_i$, wobei die \glsIdxPl{Voraussetzung} und \glsIdxPl{Folgerung} aufgelöst werden.
		%
		\item $(D_n) =$ Die Vereinigung aller Abhängigkeiten der \glsIdxPl{Folgerung}, vermindert um die \glsIdxPl{Voraussetzung}.
		-- Wenn das Feld dabei nicht leer bleibt, ist der \glsIdx{Beweis} missglückt!
		%
	\end{enumerate}
	%
\end{enumerate}
%
Ein weiteres Beispiel \vrefintab{tab:AbleitungKontraposition} soll verdeutlichen, wie Abhängigkeiten von Zwischenannahmen wieder beseitigt werden können.%
\footnote{siehe~\cite{bib:NatuerlichesSchliessen},
Kapitel 2.2.4 \emph{Eine Beispielableitung}}

\begin{table}[!htb]
	\setlength\tabcolsep{1pt}
	\setlength\extrarowheight{7pt}
	\newcommand*{\centerParbox}[2]{\parbox{#1}{\centering #2}}
	\newcommand*{\titleCell}[3]{\centerParbox{#1}{\textbf{#2} (#3)}}
	\newcommand*{\SnCell}[1]{\centerParbox{2.30cm}{#1}}
	\newcommand*{\DnCell}[1]{\centerParbox{1.95cm}{#1}}
	\begin{tabular}{|c||c|c|c|c|c|c|}
		\hline
		\titleCell{0.95cm}{Zeile}                       {$n$} &
		\titleCell{1.05cm}{Regel}                     {$R_n$} &
		\titleCell{1.85cm}{Substitu"=tionen}          {$S_n$} &
		\titleCell{1.80cm}{erzeugte Regel} {$\overline{R}_n$} &
		\titleCell{2.15cm}{angewendet auf ...}        {$Z_n$} &
		\titleCell{1.65cm}{\glsIdx{Aussage}}          {$A_n$} &
		\titleCell{1.95cm}{Abhängig"=keiten}          {$D_n$}
		\\\hline \hline
		1 & \centerParbox{1.00cm}{Folge"=rung} & & & & $(\alpha\limp\beta)\limp(\lnot\beta\limp\lnot\alpha)$ & 1
		\\\hline
		2 & \centerParbox{1.20cm}{An"=nahme} & & & & $\alpha\limp\beta$ & 2
		\\\hline
		3 & \centerParbox{1.20cm}{An"=nahme} & & & & $\lnot\beta$ & 3
		\\\hline
		4 & \centerParbox{1.20cm}{An"=nahme} & & & & $\alpha$ & 4
		\\\hline
		5 & \tagimpB & & $\dfrac{X \derive \alpha\limp\beta}{X,\alpha \derive \beta}$ & & &
		\\\hline
		6 & -1 & $X \subst \emptyset$ & $\dfrac{\alpha\limp\beta}{\alpha \derive \beta}$ & 2 & $\alpha \derive \beta $ & 2
		\\\hline
		7 & \tagSR & & $\dfrac{X \derive \alpha \srand X,\alpha \derive \beta}{X \derive \beta}$ & & &
		\\\hline
		8 & -1 & $X \subst \emptyset$ & $\dfrac{\alpha \srand \alpha \derive \beta}{\beta}$ & 4, 6 & $\beta $ & 4, 6
		\\\hline
		9' & \tagandE & & $\dfrac{X \derive \alpha, \beta}{X \derive \alpha \land \beta}$ & & &
		\\\hline
		10' & -1 & $X \subst \emptyset$ & $\dfrac{\alpha \srand \beta}{\alpha \land \beta}$ & & &
		\\\hline
		11' & -1 &\SnCell{
			$\alpha \swap \beta$\\
			$\alpha \subst \lnot\beta$
		}  & $\dfrac{\beta \srand \lnot\beta}{\beta \land \lnot\beta}$ & 8, 3 & $\beta \land \lnot\beta$ &
		\\\hline
		9 & \tagnota & & $\dfrac{X \derive \alpha, \lnot\alpha}{X \derive \beta}$ & & &
		\\\hline
		10 & -1 & $X \subst \emptyset$ & $\dfrac{\alpha \srand \lnot\alpha}{\beta}$ & & &
		\\\hline
		11 & -1 & \SnCell{
			$\alpha \swap \beta$\\
			$\alpha \subst \lnot\alpha$
		} & $\dfrac{\beta \srand \lnot\beta}{\lnot\alpha}$ & 8, 3 & $\lnot\alpha$ & 2, 3, 4
		\\\hline
		12 & \tagimpE & & $\dfrac{X, \alpha \derive \beta}{X \derive \alpha\limp\beta}$ & & &
		\\\hline
		13 & -1 & $X \subst \emptyset$ & $\dfrac{\alpha \derive \beta}{\alpha\limp\beta}$ & & &
		\\\hline
		14 & -1 & \SnCell{
			$\alpha \swap \beta$\\
			$\alpha \subst \lnot\alpha$\\
			$\beta \subst \lnot\beta$
		} & $\dfrac{\lnot\beta \derive \lnot\alpha}{\lnot\beta\limp\lnot\alpha}$ & 3, 11, ??? & $\lnot\beta\limp\lnot\alpha$ & 2, 3, 4, ???
		\\\hline
		15 & \tagimpE+1 & \SnCell{
			$\alpha \subst \gamma$\\
			$\beta \subst \delta$\\
			$\gamma \subst \alpha\limp\beta$\\
			$\delta \subst \lnot\beta\limp\lnot\alpha$
		} & $\dfrac{\alpha\limp\beta \derive \lnot\beta\limp\lnot\alpha}
		{(\alpha\limp\beta)\limp(\lnot\beta\limp\lnot\alpha)}$ & 2, 14 &
		$(\alpha\limp\beta)\limp(\lnot\beta\limp\lnot\alpha)$ & 2, 3, 4, ???
		\\\hline\hline
		16 & \centerParbox{1.5cm}{\tagimpE, \tagimpB, \tagSR} & & $\dfrac{}{A_1}$ & & $\dfrac{}{(\alpha\limp\beta)\limp(\lnot\beta\limp\lnot\alpha)}$ &
		\\\hline
	\end{tabular}
	\caption{Ableitung der \glsIdx{Kontraposition} aus \glsIdxBg{allgemeingueltige-Schlussregel}{allgemeingültigen Schlussregeln}}
	\label{tab:AbleitungKontraposition}
\end{table}

\todo{Beispielableitung der Kontraposition vervollständigen}%%%
%TODO Beispielableitung der Kontraposition vervollständigen %%%

\Endchapter

	%%############################################################################%%
%%                                                                            %%
%% Datei:  ASBA-Design.tex                                                    %%
%% Inhalt: Kapitel "Design"                                                   %%
%%                                                                            %%
%% Copyright (C) 2017  Winfried Teschers                                      %%
%%                                                                            %%
%% This program is free software: you can redistribute it and/or modify       %%
%% it under the terms of the GNU Affero General Public License as published   %%
%% by the Free Software Foundation, either version 3 of the License, or       %%
%% (at your option) any later version.                                        %%
%%                                                                            %%
%% This program is distributed in the hope that it will be useful,            %%
%% but WITHOUT ANY WARRANTY; without even the implied warranty of             %%
%% MERCHANTABILITY or FITNESS FOR A PARTICULAR PURPOSE.  See the              %%
%% GNU Affero General Public License for more details.                        %%
%%                                                                            %%
%% You should have received a copy of the GNU Affero General Public License   %%
%% along with this program.  If not, see <http://www.gnu.org/licenses/>.      %%
%%                                                                            %%
%% Dr. Winfried Teschers                                                      %%
%% Anton-Günther-Straße 26c                                                   %%
%% 91083 Baiersdorf                                                           %%
%% Germany                                                                    %%
%%                                                                            %%
%% e-mail: winfried.teschers@t-online.de                                      %%
%%                                                                            %%
%%############################################################################%%

% !TeX root = ASBA.tex
% !TeX encoding = UTF-8
% !TeX spellcheck = de_DE

\chapter{Design}% ##############################################################
\beginchapter{Design}
\label{cha:Design}

Dieses Projekt soll Open Source sein.
Daher gilt für die Dokumente die \emph{GNU Free Documentation License (FDL)} und für die Software die \emph{GNU Affero General Public License (APGL)}.
Die \emph{GNU General Public License (GPL)} reicht für die Software nicht aus, da das Programm auch mittels eines Servers betrieben werden kann und soll.
Damit das Projekt gegebenenfalls durch verschiedene Entwickler gleichzeitig bearbeitet werden kann und wegen des Konfigurationsmanagements wurde es als ein GitHub Projekt erstellt (siehe~\cite{bib:ASBA}).

Wenn die Lizenzen nicht mitgeliefert wurden, können sie unter \url{http://www.gnu.org/licenses/} gefunden werden.

\section{Anforderungen}% =======================================================
\beginsection{Anforderungen}
\label{sec:Anforderungen}

Die Anforderungen ergeben sich zunächst aus den Zielen \vrefinsec{sec:Ziele}.
Die beiden Ziele~\ref{Ziel:Daten}~\emph{Daten} und~\ref{Ziel:Lizenz}~\emph{Lizenz} sind für die Entwicklung von \ASBA\ von sekundärer Bedeutung und werden daher in diesem \sectionname\ nicht übernommen.
Die anderen Ziele werden noch verfeinert.

\todo{Ziele aus Abschnitt "'Ziele"' in Anforderungen umwandeln.}%%%
%TODO Ziele aus Abschnitt "'Ziele"' in Anforderungen umwandeln. %%%
%
\begin{enumerate}

	\item \label{Anforderung:Form} \emph{Form}:
	Die Daten liegt in formaler, geprüfter Form vor.
	(\vrefseeziel{Ziel:Form})

	\item \label{Anforderung:Eingaben} \emph{Eingaben}:
	Die Eingabe von Daten erfolgt in einer formalen Syntax unter Verwendung der üblichen mathematischen Schreibweise.
	Folgende Daten können eingegeben werden:
	\begin{enumerate}
		\item \Axiome
		\item \Saetze
		\item \Beweise
		\item \Fachbegriffe
		\item \Fachgebiete
		\item \Ausgabeschemata
	\end{enumerate}
	Dabei sind alle Begriffe nur innerhalb eines Fachgebiets und seiner untergeordneten \Fachgebiete\ gültig, solange sie nicht umdefiniert werden.
	Das oberste \Fachgebiet\ ist die ganze Mathematik.
	-- \vrefseeziel{Ziel:Eingaben}

	\item \label{Anforderung:Prüfung} \emph{Prüfung}:
	Vorhandene \Beweise\ können automatisch geprüft werden.
	-- \vrefseeziel{Ziel:Prüfung}

	\item \label{Anforderung:Ausgaben} \emph{Ausgaben}:
	Die Ausgabe kann in einer eindeutigen, formalen Syntax gemäß vorhandener \Ausgabeschemata\ erfolgen.
	-- \vrefseeziel{Ziel:Ausgaben}

	\item \label{Anforderung:Auswertungen} \emph{Auswertungen}:
	Zusätzlich zur Ausgabe der Daten sind verschiedene Auswertungen möglich.
	Insbesondere kann zu jedem \Beweis\ angegeben werden, wie lang er ist und welche \Axiome\ und Sätze%
	\footnote{Sätze, die quasi als \Axiome\ verwendet werden.}
	er benötigt.
	-- \vrefseeziel{Ziel:Auswertungen}

	\item \label{Anforderung:Anpassbarkeit} \emph{Anpassbarkeit}:
	\Fachbegriffe\ und die Darstellung bei der Ausgabe können mit Hilfe von -- gegebenenfalls unbenannten -- untergeordneten \Fachgebieten\ angepasst werden.
	-- \vrefseeziel{Ziel:Anpassbarkeit}

	\item \label{Anforderung:Individualität} \emph{Individualität}:
	\Axiome\ und Sätze können für jeden \Beweis\ individuell vorausgesetzt werden.
	Dabei sind fachgebietsspezifische \Fachbegriffe\ erlaubt.
	-- \vrefseeziel{Ziel:Individualität})

	\item \label{Anforderung:Internet} \emph{Internet}:
	Die Daten können auf mehrere Dateien verteilt sein.
	Ein Teil davon -- oder sogar alle -- können im Internet liegen.
	-- \vrefseeziel{Ziel:Internet}

	\item \label{Anforderung:Kommunikation} \emph{Kommunikation}:
	Die Kommunikation mit \ASBA\ kann mit den \Fachbegriffen\ der einzelnen \Fachgebiete\ erfolgen.
	-- \vrefseeziel{Ziel:Kommunikation}

	\item \label{Anforderung:Zugriff} \emph{Zugriff}:
	Der Zugriff auf \ASBA\ kann lokal und über das Internet erfolgen.
	-- \vrefseeziel{Ziel:Zugriff}

	\item \label{Anforderung:Unabhängigkeit} \emph{Unabhängigkeit}:
	\ASBA\ kann offline und online arbeiten.
	-- \vrefseeziel{Ziel:Unabhängigkeit}

	\item \label{Anforderung:Rekursion} \emph{Rekursion}:
	Es kann rekursiv über alle verwendeten Dateien -- auch solchen, die im Internet liegen -- ausgewertet werden.
	-- \vrefseeziel{Ziel:Rekursion}

	\item \label{Anforderung:Bedienbarkeit} \emph{Bedienbarkeit}:
	\ASBA\ ist einfach zu bedienen.
	-- \vrefseeziel{Ziel:Bedienbarkeit}

\end{enumerate}

\section{Axiome}% ==============================================================
\beginsection{\Axiome}
\label{sec:Axiome}
\todo{Axiome auswählen und definieren.}%%%
%TODO Axiome auswählen und definieren. %%%

\section{Beweise}% =============================================================
\beginsection{\Beweise}
\label{sec:Beweise}
\todo{Schlussregeln auswählen und Beweise definieren.}%%%
%TODO Schlussregeln auswählen und Beweise definieren. %%%

\section{Datenstruktur}% =======================================================
\beginsection{Datenstruktur}
\label{sec:Datenstruktur}
\todo{Datenstruktur abstrakt und in XML definieren.}%%%
%TODO Datenstruktur abstrakt und in XML definieren. %%%

\section{Bausteine}% ===========================================================
\beginsection{Bausteine}
\label{sec:Bausteine}
\todo{Bausteine? definieren.}%%%
%TODO Bausteine? definieren. %%%

\Endchapter

	%%############################################################################%%
%%                                                                            %%
%% Datei:  ASBA-Anhang.tex                                                    %%
%% Inhalt: Anhang                                                             %%
%%                                                                            %%
%% Copyright (C) 2017  Winfried Teschers                                      %%
%%                                                                            %%
%% This program is free software: you can redistribute it and/or modify       %%
%% it under the terms of the GNU Affero General Public License as published   %%
%% by the Free Software Foundation, either version 3 of the License, or       %%
%% (at your option) any later version.                                        %%
%%                                                                            %%
%% This program is distributed in the hope that it will be useful,            %%
%% but WITHOUT ANY WARRANTY; without even the implied warranty of             %%
%% MERCHANTABILITY or FITNESS FOR A PARTICULAR PURPOSE.  See the              %%
%% GNU Affero General Public License for more details.                        %%
%%                                                                            %%
%% You should have received a copy of the GNU Affero General Public License   %%
%% along with this program.  If not, see <http://www.gnu.org/licenses/>.      %%
%%                                                                            %%
%% Dr. Winfried Teschers                                                      %%
%% Anton-Günther-Straße 26c                                                   %%
%% 91083 Baiersdorf                                                           %%
%% Germany                                                                    %%
%%                                                                            %%
%% e-mail: winfried.teschers@t-online.de                                      %%
%%                                                                            %%
%%############################################################################%%

% !TeX root = ASBA.tex
% !TeX encoding = UTF-8
% !TeX spellcheck = de_DE

\appendix
\renewcommand*{\Chaptername}{\appendixname}

\chapter     {Anhang}% #########################################################
\beginchapter{Anhang}
\label   {cha:Anhang}

\section     {Werkzeuge}% ======================================================
\beginsection{Werkzeuge}
\label   {sec:Werkzeuge}

Da dies ein Open Source Projekt sein soll, müssen alle Werkzeuge, die zum Ablauf der Software erforderlich sind, ebenfalls Open Source sein.
Für die reine Entwicklung sollte das auch gelten, muss es aber nicht.

\paragraph{Werkzeuge zur Übersetzung der Quelldateien}% --------------------

\begin{enumerate}

	\item\label{Werkzeug:LaTeX}
	Ein Übersetzer für \LaTeX\ Quellcode (*.tex).
	--- Verwendet wird \emph{MiK\TeX}.

	\item\label{Werkzeug:Cpp}
	Ein Übersetzer für C++ Quellcode (*.c, *.cpp, *.h, *.hpp).
	--- Verwendet wird \emph{Visual Studio Community 2017}.

	\setcounter{Enumi}{\value{enumi}}% Nummerierung wird fortgesetzt.
\end{enumerate}
%
Nicht unbedingt nötig, aber sinnvoll:
\begin{enumerate}
	\setcounter{enumi}{\value{Enumi}}% Nummerierung wird fortgesetzt.

	\item\label{Werkzeug:Dokumentation}
	Ein Dokumentationssystem für in C++ Quellcode und darin enthaltene Doxygen Kommentare (*.c, *.cpp, *.h, *.hpp).
	--- Verwendet wird \emph{Doxygen} mit Konfigurationsdatei "`Doxyfile"'.

	\item\label{Werkzeug:Konfigurationsmanagement}
	Ein Konfigurationsmanagementsystem zur Verwaltung der Quelldateien.
	--- Verwendet wird \emph{GitHub}.

	\setcounter{Enumi}{\value{enumi}}% Nummerierung wird fortgesetzt.
\end{enumerate}

\paragraph{Werkzeuge für die Entwicklung}% -------------------------------------

\begin{enumerate}
	\setcounter{enumi}{\value{Enumi}}% Nummerierung wird fortgesetzt.

	\item\label{Werkzeug:GitHub}\emph{GitHub} als Online Konfigurationsmanagementsystem zur Zusammenarbeit verschiedener Entwickler.
	\\\tourl{https://github.com/}
	--- Lizenz \citesee{bib:GPLii}

	\item\label{Werkzeug:Git}GitHub benötigt \emph{Git} als Konfigurationsmanagementsystem.
	\\\tourl{https://git-scm.com/}
	--- Lizenz \citesee{bib:GPLii}

	\item\label{Werkzeug:MiKTeX}\emph{MiK\TeX} für Dokumentation und Ausgaben in \LaTeX.
	\\\tourl{https://miktex.org/}
	--- Lizenz \citesee{bib:MiKTeX}

	\item\label{Werkzeug:VSC}angedacht: \emph{Visual Studio Community 2017}%
	\footnote{%
		Visual Studio Community ist zwar nicht Open Source, darf aber zur Entwicklung von Open Source Software
		unentgeltlich verwendet werden.
	}
	(\emph{VS}) als Entwicklungsumgebung für C++.
	\\\tourl{https://www.visualstudio.com/downloads/}
	--- Lizenz \citesee{bib:EULA}

	\item\label{Werkzeug:VSC DB}angedacht: In \emph{Visual Studio Community 2015} integrierte Datenbank für \Ausgabeschemata, \Saetze, \Beweise, \Fachbegriffe\ und \Fachgebiete.
	--- Lizenz \citesee{bib:EULA}

	\item\label{Werkzeug:RapidXml}angedacht: \emph{RapidXml} für Ein- und Ausgabe in XML.
	\\\tourl{http://rapidxml.sourceforge.net/index.htm}
	--- Lizenz \citesee{bib:BSLi} oder wahlweise~\cite{bib:MIT}
	\footnote{%
		RapidXml stellt eine C++ Header-Datei zur Verfügung.
		Wenn diese im Quellcode eines Programms enthalten ist, gilt das ganze Programm als Open Source.
		Wenn diese Header-Datei nur in einer Bibliothek innerhalb eines Projekts verwendet wird, so gilt nur diese Bibliothek als Open Source.
	}

	\item\label{Werkzeug:Doxygen}angedacht: \emph{Doxygen} als Dokumentationssystem für C++.
	\\\tourl{http://www.stack.nl/~dimitri/doxygen/}
	--- Lizenz \citesee{bib:GPLii}

	\item\label{Werkzeug:Ghostscript}angedacht: Doxygen benötigt \emph{Ghostscript} als Interpreter für Postscript und PDF.
	\\\tourl{http://ghostscript.com/}
	--- Lizenz \citesee{bib:AGPL}

	\item\label{Werkzeug:Graphviz}angedacht: Doxygen benötigt \emph{Graphviz} mit \emph{Dot} zur Erzeugung und Visualisierung von Graphen.
	\\\tourl{http://www.graphviz.org/Home.php}
	--- Lizenz \citesee{bib:EPL}

	\setcounter{Enumi}{\value{enumi}}% Nummerierung wird fortgesetzt.
\end{enumerate}

\paragraph{Werkzeuge zur Bearbeitung der Quelldateien}% ------------------------

\begin{enumerate}
	\setcounter{enumi}{\value{Enumi}}% Nummerierung wird fortgesetzt.

	\item\label{Werkzeug:TeXstudio}\emph{\TeX studio} als Editor für \LaTeX.
	\\\tourl{http://www.texstudio.org/}
	--- Lizenz \citesee{bib:GPLii}
	\\\TeX studio benötigt einen Interpreter für Perl:

	\item\label{Werkzeug:Perl}\emph{Strawberry Perl} als Interpreter für Perl.
	\\\tourl{http://strawberryperl.com/}
	--- Lizenz: Various OSI-compatible Open Source licenses, or given to the public domain

	\item\label{Werkzeug:Notepadpp}\emph{Notepad++} als Text-Editor.
	\\\tourl{https://notepad-plus-plus.org/}
	--- Lizenz \citesee{bib:GPLi}

	\item\label{Werkzeug:WinMerge}\emph{WinMerge} zum Vergleich von Dateien und Verzeichnissen.
	\\\tourl{http://winmerge.org/}
	--- Lizenz \citesee{bib:GPLi}

	\setcounter{Enumi}{\value{enumi}}% Nummerierung wird fortgesetzt.
\end{enumerate}

\color{gray}%%% Anfang grauer Text ---------------------------------------------
\paragraph{Im Projekt \emph{qedeq} verwendete Werkzeuge}% ----------------------

\begin{itemize}
	\setcounter{enumi}{\value{Enumi}}% Nummerierung wird fortgesetzt.

	\item\label{Werkzeug:Java}\emph{Java} als Programmiersprache und Laufzeitumgebung.
	\\\tourl{https://www.java.com/de/download/win10.jsp}
	--- Lizenz \citesee{bib:JavaSE}

	\item\label{Werkzeug:Apache Ant}\emph{Apache Ant} als Java Bibliothek und Kommandozeilen-Werkzeug
	um Java Programme zu erzeugen.
	\\\tourl{http://ant.apache.org/}
	--- Lizenz \citesee{bib:Apacheii}

	\item\label{Werkzeug:Checkstyle}\emph{Checkstyle} zur statischen Code-Analyse für Java.
	\\\tourl{http://checkstyle.sourceforge.net/}
	--- Lizenz \citesee{bib:LGPLii}

	\item\label{Werkzeug:Clover}\emph{Clover}%
	\footnote{%
		Clover ist proprietäre Software, aber auf Anfrage frei für 30 Tage.
		Danach ist eine einmalige Lizenzgebühr fällig.
	}
	als Testwerkzeug zur Analyse der Code-Abdeckung.
	\\\tourl{https://www.atlassian.com/software/clover/}
	--- Lizenz \citesee{bib:Clover}

	\item\label{Werkzeug:Eclipse Java}\emph{Eclipse IDE for Java Developers} als Entwicklungsumgebung für Java.
	\\\tourl{http://www.eclipse.org/downloads/packages/eclipse-ide-java-developers/neon1a/}
	--- Lizenz \citesee{bib:OSI}

	\item\label{Werkzeug:JUnit}\emph{JUnit} zur Erzeugung von wiederholbaren Tests.
	\\\tourl{http://junit.org/junit4/}
	--- Lizenz \citesee{bib:EPL}

	\item\label{Werkzeug:Xerces2}\emph{Xerces2} als XML-Parser in Java.
	\\\tourl{http://xerces.apache.org/xerces2-j/}
	--- Lizenzen \citesee{bib:Apacheii, bib:SAX, bib:WDCDL, bib:WDCSNL}
\end{itemize}
\color{black}%%% Ende  grauer Text ---------------------------------------------

\section     [Die Struktur ausgewählter Begriffe]{Die Struktur ausgewählter \Begriffe}
\beginsection{Die Struktur ausgewählter Begriffe}
\label   {sec:Begriffsstruktur}

\begin{table}[H]
	\centering
	\begin{threeparttable}
		\setlength\extrarowheight{3pt}
		\begin{tabularx}{\linewidth}{c@{\extracolsep{\fill}}|c|c|c|c|}
			\hline% ------------------------------------------------------------
			\multicolumn{5}{c|}{\textbf{\Objekt}}\Tnote{1}
			\\
			\hline% ------------------------------------------------------------
		\end{tabularx}
		\begin{tablenotes}
			\footnotesize
			\item[1] Fußnote zur Tabelle
		\end{tablenotes}
	\end{threeparttable}
	\caption{\Bezeichnungen}
	\label{tab:Objekte}% Erst nach '\caption'!
\end{table}

\begin{table}[H]
	\begin{threeparttable}
		\setlength\extrarowheight{3pt}
		\begin{tabularx}{\linewidth}{c@{\extracolsep{\fill}}|c|c|c|c|}
			\hline% ------------------------------------------------------------
			\multicolumn{3}{c|}{\textbf{\Metasprache}}&
			\multicolumn{2}{c|}{\textbf{\Objektsprache}}
			\\
			\textbf{natürliche Sprache} & \multicolumn{2}{c|}{\textbf{\formaleMetasprache}}
			& \textbf{\Aussagenlogik} & \textbf{\Praedikatenlogik}
			\\
			\hline% ------------------------------------------------------------
			& \multicolumn{4}{c|}{\Symbole}
			\\
			& \multicolumn{2}{c|}{\Metasymbol}
			& \multicolumn{2}{c|}{\Objektsymbol}
			\\
			\hline% ------------------------------------------------------------
			& \multicolumn{4}{c|}{Beispielsymbole}
			\\
			\unaere\  \Operation
			& \multicolumn{4}{c|}{\BspOpU}
			\\
			\binaere\ \Operation
			& \multicolumn{4}{c|}{\BspOpB}
			\\
			\binaere\ \Relationen
			& \multicolumn{4}{c|}{\BspRel  \quad \BspRelEq  \quad \BspRelBck  \quad \BspRelBckEq \quad \BspRelN \quad \BspRelEqN \quad \BspRelBckN \quad \BspRelBckEqN}
			\\
			\hline% ------------------------------------------------------------
			\multicolumn{5}{c|}{\Wahrheitswerte}
			\\
			~                    \TxtTrue \quad \TxtFalse
			&\multicolumn{2}{c|}{\MtsTrue \quad \MtsFalse}
			&\multicolumn{2}{c|}{\OjkTrue \quad \OjkFalse}
			\\
			\hline% ------------------------------------------------------------
			& \multicolumn{4}{c|}{\Operation \quad \Relation \quad \Umkehrrelation \quad \Negation}
			\\
			& \multicolumn{2}{c|}{\Metaoperation \quad \Metarelation}
			& \multicolumn{2}{c|}{\Junktor}
			\\
			\hline% ------------------------------------------------------------
			~                       nicht
			& \multicolumn{2}{c|}{\MtsNot}
			& \multicolumn{2}{c|}{\OjkNot}
			\\
			~                         und \quad   oder \quad    dann
			& \multicolumn{2}{c|}{\MtsAnd \quad \MtsOr \quad \MtsImp}
			& \multicolumn{2}{c|}{\OjkAnd \quad \OjkOr \quad \OjkImp}
			\\
			~                     dann wenn \quad    wenn
			& \multicolumn{2}{c|}{\MtsEquiv \quad \MtsRep}
			& \multicolumn{2}{c|}{\OjkEquiv \quad \OjkRep}
			\\
			~                         und\Tnote{1} \quad entweder oder
			& \multicolumn{2}{c|}{\MtsUnd}
			& \multicolumn{2}{c|}{                             \OjkXor}
			\\
			~                    nicht und \quad nicht oder
			& \multicolumn{2}{c|}{ }
			& \multicolumn{2}{c|}{\OjkNand \quad \OjkNor}
			\\
			\hline% ------------------------------------------------------------
			~                     gleich \quad ungleich
			& \multicolumn{2}{c|}{\MtsEq \quad \MtsEqN}
			& \multicolumn{2}{c|}{\OjkEq \quad \OjkEqN}
			\\
			definitionsgemäß            gleich
			& \multicolumn{2}{c|}{\MtsDefEquiv}
			& \multicolumn{2}{c|}{ }
			\\
			definitionsgemäß         gleich
			& \multicolumn{2}{c|}{\MtsDefEq}
			& \multicolumn{2}{c|}{ }
			\\
			\hline% ------------------------------------------------------------
			\Quantoren
			& \multicolumn{2}{c|}{$\MtsForall \quad \MtsExists \quad \MtsExione$}
			&                   & $\OjkForall \quad \OjkExists \quad \OjkExione$
			\\
			\hline% ------------------------------------------------------------
			\Ersetzung \quad \Vertauschung
			& \multicolumn{2}{c|}{\MtsSubst \quad \MtsSwap}
			& \multicolumn{2}{c|}{ }
			\\
			\Ableitungsrelationen:
			& \multicolumn{2}{c|}{\MtsDerive \quad \MtsDeriveR \quad \MtsPraemisseRel \quad \MtsKonklusionRel \quad \MtsErgebnisRel}
			& \multicolumn{2}{c|}{ }
			\\
			\hline% ------------------------------------------------------------
			\Elementrelationen:
			& \multicolumn{2}{c|}{\MtsIn \quad \MtsNi \quad \MtsInN \quad \MtsNiN}
			& \multicolumn{2}{c|}{ }
			\\
			\Mengenrelationen:
			& \multicolumn{2}{c|}{\MtsSubset \quad \MtsSubsetEq \quad \MtsSubsetN \quad \MtsSubsetEqN \quad \MtsSupset \quad \MtsSupsetEq \quad \MtsSupsetN \quad \MtsSupsetEqN}
			& \multicolumn{2}{c|}{ }
			\\
			\Komponentenrelationen:
			& \multicolumn{2}{c|}{\MtsSeqIn \quad \MtsSeqNi \quad \MtsSeqInN \quad \MtsSeqNiN}
			& \multicolumn{2}{c|}{ }
			\\
			\Folgenrelationen:
			& \multicolumn{2}{c|}{\MtsSubseq \quad \MtsSubseqEq \quad \MtsSubseqN \quad \MtsSubseqEqN \quad \MtsSupseq \quad \MtsSupseqEq \quad \MtsSupseqN \quad \MtsSupseqEqN}
			& \multicolumn{2}{c|}{ }
			\\
			ausgewählte Mengen
			& \multicolumn{2}{c|}{\MtsIN \quad \MtsINo \quad \MtsUniversum \quad \Sprache }
			& \multicolumn{2}{c|}{ }
			\\
			\hline% ------------------------------------------------------------
			& \textbf{\unaer} & \textbf{\binaer}
			& \multicolumn{2}{c|}{ }
			\\
			\Mengenoperationen
			& \MtsPot \quad \MtsPotf \quad \MtsRel \quad \MtsRelf & \MtsCap \quad \MtsCup \quad \MtsSetminus \quad \MtsTimes
			& \multicolumn{2}{c|}{ }
			\\
			& \MtsFol \quad \MtsFolf \quad \MtsTup &
			& \multicolumn{2}{c|}{ }
			\\
			\hline% ------------------------------------------------------------
			\unaere\ \Operationen\ auf:
			& \textbf{\Relationen} & \textbf{\Funktionen}
			& \multicolumn{2}{c|}{ }
			\\
			& \MtsStelR            & \MtsStelF
			& \multicolumn{2}{c|}{ }
			\\
			\DefinitionsB- \quad \Zielbereich
			&                      & \MtsDb \quad \MtsZb
			& \multicolumn{2}{c|}{ }
			\\
			\QuellB- \quad \Wertebereich
			&                      & \MtsQb \quad \MtsWb
			& \multicolumn{2}{c|}{ }
			\\
			\Traegermenge
			& \multicolumn{2}{c|}{$\MtsTraeger \quad \MtsTraeger_i$}
			& \multicolumn{2}{c|}{ }
			\\
			\Graph
			& \multicolumn{2}{c|}{ \MtsGraph }
			& \multicolumn{2}{c|}{ }
			\\
			\hline% ------------------------------------------------------------
			\unaere\ \Operationen\ auf:
			& \multicolumn{2}{c|}{ \Folgen \quad \Tupel }
			& \multicolumn{2}{c|}{ }
			\\
			& \multicolumn{2}{c|}{ \MtsLen \quad \MtsSet }
			& \multicolumn{2}{c|}{ }
			\\
			\hline% ------------------------------------------------------------
		\end{tabularx}
		\begin{tablenotes}
			\footnotesize
			\item[] Die erste Spalte beschreibt die anderen Spalten.
			Die \textbf{fettgedruckten} Teile, und nur diese, gelten als Überschriften.
			\item[1] nur in Schlussregeln
		\end{tablenotes}
	\end{threeparttable}
	\caption{Ausgewählte \Bezeichnungen}
	\label{tab:Benennungen}% Erst nach '\caption'!
\end{table}


\section     {Offene Aufgaben}% ================================================
\beginsection{Offene Aufgaben}
\label   {sec:OffeneAufgaben}

\begin{enumerate}
	\item TODOs bearbeiten.
	\item Eingabeprogramm erstellen (liest XML).
	\item Prüfprogramm erstellen.
	\item Ausgabeprogramm erstellen (schreibt XML).
	\item Formelausgabe erstellen (erzeugt \LaTeX{} aus XML).
	\item \Axiome\ sammeln und eingeben.
	\item \Saetze\ sammeln und eingeben.
	\item \Beweise\ sammeln und eingeben.
	\item \Fachbegriffe\ und Symbole sammeln und eingeben.
	\item \Fachgebiete\ sammeln und eingeben.
	\item \Ausgabeschemata\ sammeln und eingeben.
\end{enumerate}

\Endchapter


	\chapter{Verzeichnisse}% ###################################################
	\beginchapter{Verzeichnisse}
	\label{cha:Verzeichnisse}

	%section{Tabellenverzeichnis}% =============================================
	\likechapter[section]{\listtablename}
	\begin{minipage}{\linewidth-10.95pt}
		\label{dic:Tabellenverzeichnis}
		\listoftables
	\end{minipage}
	\Endchapter

	%section{Abbildungsverzeichnis}% ===========================================
	\likechapter[section]{\listfigurename}
	\begin{minipage}{\linewidth-10.95pt}
		\label{dic:Abbildungsverzeichnis}
		\listoffigures
	\end{minipage}
	\Endchapter

	%%############################################################################%%
%%                                                                            %%
%% Datei:  ASBA-Literaturverzeichnis.tex                                      %%
%% Inhalt: Literaturverzeichnis                                               %%
%%                                                                            %%
%% Copyright (C) 2017  Winfried Teschers                                      %%
%%                                                                            %%
%% This program is free software: you can redistribute it and/or modify       %%
%% it under the terms of the GNU Affero General Public License as published   %%
%% by the Free Software Foundation, either version 3 of the License, or       %%
%% (at your option) any later version.                                        %%
%%                                                                            %%
%% This program is distributed in the hope that it will be useful,            %%
%% but WITHOUT ANY WARRANTY; without even the implied warranty of             %%
%% MERCHANTABILITY or FITNESS FOR A PARTICULAR PURPOSE.  See the              %%
%% GNU Affero General Public License for more details.                        %%
%%                                                                            %%
%% You should have received a copy of the GNU Affero General Public License   %%
%% along with this program.  If not, see <http://www.gnu.org/licenses/>.      %%
%%                                                                            %%
%% Dr. Winfried Teschers                                                      %%
%% Anton-Günther-Straße 26c                                                   %%
%% 91083 Baiersdorf                                                           %%
%% Germany                                                                    %%
%%                                                                            %%
%% e-mail: winfried.teschers@t-online.de                                      %%
%%                                                                            %%
%%############################################################################%%

% !TeX root = ASBA.tex
% !TeX encoding = UTF-8
% !TeX spellcheck = de_DE

%chapter{Literaturverzeichnis}% ################################################

\begin{flushleft}
	\begin{thebibliography}{12}
		\likechapter[section]{\bibname}  % erst hier!
		\label{dic:Literaturverzeichnis} % erst hier!

		\bibitem{bib:Rautenberg}Wolfgang Rautenberg,
		\emph{Einführung in die Mathematische Logik}:
		Ein Lehrbuch, 3.\@ Auflage, Vieweg+Teubner 2008

		\bibitem{bib:Apacheii}\emph{Apache License}, Version 2.0
		$\rightarrow$%
		\footnote{%
			Der Pfeil~($\rightarrow$) verweist stets auf einen Link zu einer Seite im Internet.
		}
		\url{http://www.apache.org/licenses/LICENSE-2.0}
		01.2004%
		\footnote{%
			Das Datum hinter dem Link gibt -- je nachdem welches bekannt ist -- das Datum der letzten Änderung, den Stand der Seite oder das Datum, an dem die Seite angeschaut wurde an.
			Sind mehrere Daten vorhanden, wird das erste vorhandene in der angegebenen Reihenfolge genommen.
			-- Dies gilt für alle hier aufgelisteten Seiten im Internet.
		}

		\bibitem{bib:BSLi}\emph{Boost Software License} 1.0
		\tourl{http://www.boost.org/users/license.html}
		17.08.2003

		\bibitem{bib:EPL}\emph{Eclipse Public License} Version 1.0
		\tourl{http://www.eclipse.org/org/documents/epl-v10.php}
		09.03.2017

		\bibitem{bib:AGPL}\emph{GNU Affero General Public License}
		\tourl{http://www.gnu.org/licenses/agpl}
		19.11.2007

		\bibitem{bib:GPLi}\emph{GNU General Public License}
		\tourl{http://www.gnu.org/licenses/old-licenses/gpl-1.0}
		02.1989

		\bibitem{bib:GPLii}\emph{GNU General Public License}, Version 2
		\tourl{http://www.gnu.org/licenses/old-licenses/gpl-2.0}
		06.1991

		\bibitem{bib:LGPLii}\emph{GNU Lesser General Public License},
		Version 2.1
		\tourl{http://www.gnu.org/licenses/old-licenses/lgpl-2.1}
		02.1999

		\bibitem{bib:Clover}Lizenz für \emph{Clover}
		\tourl{https://www.atlassian.com/software/clover}
		2017

		\bibitem{bib:EULA}Lizenz
		für \emph{Microsoft Visual Studio Express 2015}
		\tourl{https://www.visualstudio.com/de/license-terms/mt171551/}
		2017

		\bibitem{bib:MiKTeX}Lizenz für \emph{MikTeX}
		\tourl{https://miktex.org/kb/copying}
		13.04.2017

		\bibitem{bib:SAX}Lizenz für \emph{SAX}
		\tourl{http://www.saxproject.org/copying.html}
		05.05.2000

		\bibitem{bib:MIT}\emph{MIT License}
		\tourl{https://opensource.org/licenses/MIT/}
		09.03.2017

		\bibitem{bib:JavaSE}\emph{Oracle Binary Code License Agreement}
		\tourl{http://java.com/license}
		02.04.2013

		\bibitem{bib:OSI}\emph{OSI Certified Open Source Software}
		\tourl{https://opensource.org/pressreleases/certified-open-source.php}
		16.06.1999

		\bibitem{bib:WDCDL}\emph{W3C Document License}
		\tourl{http://www.w3.org/Consortium/Legal/2015/doc-license}
		01.02.2015

		\bibitem{bib:WDCSNL}\emph{W3C Software Notice and License}
		\tourl{http://www.w3.org/Consortium/Legal/2002/copyright-software-20021231.html}
		13.05.2015

		\bibitem{bib:HilbertII}\emph{Hilbert II -- Introduction}
		\tourl{http://www.qedeq.org/}
		20.01.2014

		\bibitem{bib:qedeq}\emph{Formal Correct Mathematical Knowledge}:
		GitHub Repository vom Projekt Hilbert II
		\tourl{https://github.com/m-31/qedeq/}
		18.03.2017

		\bibitem{bib:ASBA}\emph{ASBA -- Axiome, Sätze, Beweise und Auswertungen}.
		Projekt zur maschinellen Überprüfung von mathematischen Beweisen
		und deren Ausgabe in lesbarer Form:
		GitHub Repository vom Projekt ASBA
		-- in Bearbeitung
		\tourl{https://github.com/Dr-Winfried/ASBA}

		\bibitem{bib:LogikDe}Meyling, Michael:
		\emph{Anfangsgründe der mathematischen Logik}
		\tourl{http://www.qedeq.org/current/doc/math/qedeq\_logic\_v1\_de.pdf}
		24.~Mai~2013 (in Bearbeitung)

		\bibitem{bib:PraedikatenlogikDe}Meyling, Michael:
		\emph{Formale Prädikatenlogik}
		\tourl{http://www.qedeq.org/current/doc/math/qedeq\_formal\_logic\_v1\_de.pdf}
		24.~Mai~2013 (in Bearbeitung)

		\bibitem{bib:MengenlehreDe}Meyling, Michael:
		\emph{Axiomatische Mengenlehre}
		\tourl{http://www.qedeq.org/current/doc/math/qedeq\_set\_theory\_v1\_de.pdf}
		24.~Mai~2013 (in Bearbeitung)

		\bibitem{bib:LogikEn}Meyling, Michael:
		\emph{Elements of Mathematical Logic}
		\tourl{http://www.qedeq.org/current/doc/math/qedeq\_logic\_v1\_en.pdf}
		24.~Mai~2013 (in Bearbeitung)

		\bibitem{bib:PraedikatenlogikEn}Meyling, Michael:
		\emph{Formal Predicate Calculus}
		\tourl{http://www.qedeq.org/current/doc/math/qedeq\_formal\_logic\_v1\_en.pdf}
		24.~Mai~2013 (in Bearbeitung)

		\bibitem{bib:MengenlehreEn}Meyling, Michael:
		\emph{Axiomatic Set Theory}
		\tourl{http://www.qedeq.org/current/doc/math/qedeq\_set\_theory\_v1\_en.pdf}
		24.~Mai~2013 (in Bearbeitung)

		\bibitem{bib:Aussagenlogik}Wikipedia:
		\emph{Aussagenlogik} \chaptername~4 \emph{Formaler Zugang}
		\tourl{https://de.wikipedia.org/wiki/Aussagenlogik\#Formaler\_Zugang}
		18.01.2018

		\bibitem{bib:Funktion}Wikipedia:
		\emph{Funktion (Mathematik)} \chaptername~2.1 \emph{Mengentheoretische Definition}
		\tourl{https://de.wikipedia.org/wiki/Funktion\_(Mathematik)\#Mengentheoretische\_Definition}
		27.01.2018

		\bibitem{bib:HilbertKalkuel}Wikipedia:
		\emph{Hilbert-Kalkül} \chaptername~1.4 \emph{Modus (ponendo) ponens}
		\tourl{https://de.wikipedia.org/wiki/Hilbert-Kalk\%C3\%BCl\#Modus\_(ponendo)\_ponens}
		18.06.16

		\bibitem{bib:Identitaet}Wikipedia:
		\emph{Identität (Logik)} \chaptername~2.3 \emph{Identität in der Informatik}
		\tourl{https://de.wikipedia.org/wiki/Identit\%C3\%A4t\_(Logik)\#Identit.C3.A4t\_in\_der\_Informatik}
		18.05.2017

		\bibitem{bib:Junktor}Wikipedia:
		\emph{Junktor} \chaptername~2.2 \emph{Mögliche Junktoren}
		\tourl{https://de.wikipedia.org/wiki/Junktor\#M.C3.B6gliche\_Junktoren}
		21.10.2017

		\bibitem{bib:Kalkuel}Wikipedia:
		\emph{Kalkül}
		\tourl{https://de.wikipedia.org/wiki/Kalk\%C3\%BCl}
		26.02.2017

		\bibitem{bib:Mengenlehre}Wikipedia:
		\emph{Mengenlehre}
		\tourl{https://de.wikipedia.org/wiki/Mengenlehre}
		17.01.2018

		\bibitem{bib:Praedikatenlogik}Wikipedia:
		\emph{Prädikatenlogik erster Stufe}
		\tourl{https://de.wikipedia.org/wiki/Pr\%C3\%A4dikatenlogik\_erster\_Stufe}
		26.11.2017

		\bibitem{bib:Relation}Wikipedia:
		\emph{Relation (Mathematik)} \chaptername~1.1 \emph{Mehrstellige Relation}
		\tourl{https://de.wikipedia.org/wiki/Relation\_(Mathematik)\#Mehrstellige\_Relation}
		27.01.2018

		\bibitem{bib:Schlussregel}Wikipedia:
		\emph{Schlussregel}
		\tourl{https://de.wikipedia.org/wiki/Schlussregel}
		29.03.2015

		\bibitem{bib:NatuerlichesSchliessen}Wikipedia:
		\emph{Systeme natürlichen Schließens}
		\tourl{https://de.wikipedia.org/wiki/Systeme\_nat\%C3\%BCrlichen\_Schlie\%C3\%9Fens}
		25.10.2017

		\clearpage % schon hier!
	\end{thebibliography}
\end{flushleft}



	% Indizes und Glossars #####################################################

	% Um Kopf- und Fußzeilen und korrekte Referenzen zu erhalten, müssen die
	% Programme 'splitindices.pl' und 'makeglossaries.exe' aufgerufen und die
	% erzeugten Dateien 'ASBA-*.ind' und 'ASBA.gls' modifiziert werden.
	% Dies tut das Batchprogramm 'erzeugeIndizes.bat' und fügt mit Hilfe des
	% Batchprogramms 'einfuegen.bat' in die Dateien 'ASBA-*.ind' und 'ASBA.gls'
	% nach der ersten bzw. zweiten Zeile je eine Zeile mit "  \insert*" bzw.
	% "\insertglo" und dem anschließenden Kommentar "% -- Eingefuegt" ein.
	%
	% Nach der ersten Übersetzung dieser Datei muss das Programm 'einfuegen.bat'
	% manuell aufgerufen werden. Danach ist noch eine Übersetzung notwendig.
	%
	% Die Kommandos '\insert*' und '\insertglo' müssen vor Aufruf von
	% '\printindex' bzw. '\printglossaries' mit '\newcommand' definiert werden.
	%
	% '*' steht für die vorhandenen Indizes 'idx' und 'sym'.

	\beforechapter% Kopfzeile aktivieren und neue Seite
	\idxdictionary{Index}% Index ===============================================
	\newcommand*{\insertidx}{\label{idx:Index}}
	\printindex[idx]

	\beforechapter% Kopfzeile aktivieren und neue Seite
	\idxdictionary{Symbolverzeichnis}% =========================================
	\newcommand*{\insertsym}{\label{idx:Symbolverzeichnis}}
	\printindex[sym]

	\beforechapter% Kopfzeile aktivieren und neue Seite
	\glodictionary{\glossaryname}% Index =======================================
	\newcommand*{\insertglo}{\label{glo:Glossar}
		\Thispagestyle
		\addcontentsline{toc}{section}{\glossaryname}% Eintrag ins Inhaltsverzeichnis
	}
	\printglossaries

\end{document}

% Ende des Dokuments ###########################################################
