%%############################################################################%%
%%                                                                            %%
%% Datei:  ASBA-Mathematik.tex                                                %%
%% Inhalt: Kapitel "Mathematische Grundlagen"                                 %%
%%                                                                            %%
%% Copyright (C) 2017  Winfried Teschers                                      %%
%%                                                                            %%
%% This program is free software: you can redistribute it and/or modify       %%
%% it under the terms of the GNU Affero General Public License as published   %%
%% by the Free Software Foundation, either version 3 of the License, or       %%
%% (at your option) any later version.                                        %%
%%                                                                            %%
%% This program is distributed in the hope that it will be useful,            %%
%% but WITHOUT ANY WARRANTY; without even the implied warranty of             %%
%% MERCHANTABILITY or FITNESS FOR A PARTICULAR PURPOSE.  See the              %%
%% GNU Affero General Public License for more details.                        %%
%%                                                                            %%
%% You should have received a copy of the GNU Affero General Public License   %%
%% along with this program.  If not, see <http://www.gnu.org/licenses/>.      %%
%%                                                                            %%
%% Dr. Winfried Teschers                                                      %%
%% Anton-Günther-Straße 26c                                                   %%
%% 91083 Baiersdorf                                                           %%
%% Germany                                                                    %%
%%                                                                            %%
%% e-mail: winfried.teschers@t-online.de                                      %%
%%                                                                            %%
%%############################################################################%%

% !TeX root = ASBA.tex
% !TeX encoding = UTF-8
% !TeX spellcheck = de_DE

\chapter{Mathematische Grundlagen}% ############################################
\beginchapter{Mathematische Grundlagen}
\label{cha:Grundlagen}

Die mathematischen Grundlagen werden einerseits gebraucht, um die erlaubten \glsIdxPl{Beweisschritt} zu definieren (\seename\ \sectionname~\vref{sec:Schlussregeln}), andererseits dienen sie auch zum Testen von \ASBA. Daher behandelt dieses Kapitel die mathematischen Grundlagen viel ausführlicher, als für die Erstellung von \ASBA\ erforderlich ist. Alle hier aufgeführten Axiome, Sätze und Beweise sollen dazu kodiert und die Beweise von \ASBA\ verifiziert werden.

\section{Metasprache}% =========================================================
\beginsection{Metasprache}
\label{sec:Metasprache}

Wenn man über eine Sprache spricht, braucht man auch eine Sprache, in der Aussagen über die erstere getroffen werden können.
Wenn die zuerst genannte Sprache die der Mathematik ist, nimmt man üblicherweise die natürliche Sprache als \glsIdx{Metasprache}.
Leider ist diese oft ungenau, nicht immer eindeutig und abhängig vom Zusammenhang, in dem sie gesprochen wird%
\footnote{%
	Man betrachte die beiden Aussagen \enquote{Studenten und Rentner zahlen die Hälfte.} und \enquote{Studenten oder Rentner zahlen die Hälfte.}, die beide das gleiche meinen.
	-- Entnommen aus \cite{bib:Rautenberg} \sectionname~1.2 Bemerkung 1.

	Ein weiteres Problem ist, dass man unauflösbare Widersprüche formulieren kann, \textzB \enquote{%
		Der Barbier ist der Mann im Ort, der genau die Männer im Ort rasiert, die sich nicht selbst rasieren.%
	}.
	Und der Barbier?
	Wenn er sich selbst rasiert, dann rasiert er sich nicht selbst, und wenn er sich nicht selbst rasiert, dann rasiert er sich selbst.
	Was denn nun?
	-- Quelle unbekannt) --
	Das Problem ist verwandt mit dem Problem der \enquote{Menge aller Mengen, die sich nicht selbst enthalten}.%
}.
Um diese Probleme in den Griff zu bekommen, wird die \glsIdx{Metasprache} zum Teil formalisiert.
Durch diese Formalisierung erinnert sie dann teilweise schon an mathematische Formeln.
Die Sprachebenen sollten aber sorgfältig unterschieden werden.

\subsection{Metasprachliche Ausdrücke}% ========================================
\beginsection{\glsIdxPl{MetaausdruckV}}
\label{sub:Metaausdruck}

Ein \emph{\glsIdx{MetaausdruckV}} ist eine in normaler Sprache verfasste Aussage, wie \textzB
(a) \strqt{Morgen scheint die Sonne.},
(b) \strqt{Ich bin 1,83\,m groß.},
(c) \strqt{Ich habe ein rotes Auto und das kann 200\,km/h schnell fahren.},
\textusw
In einem erweiterten Sinne gehören auch Relationen einschließlich ihrer Operanden dazu%
\footnote{%
	Wird statt des Symbols der Name der zugehörigen Relation verwendet, ist dies unmittelbar einleuchtend.
	So wird \textzB aus der Formel \forqt{$A<B$} die \glsIdx{MetaaussageV} \strqt{$A$ ist kleiner als $B$}.%
},
wie \textzB \forqt{$A=A$}, \forqt{$A \equiv B$}, \forqt{$A<B$}, \textusw

Während die Beispiele (a) und (b) einfache, nicht mehr zerlegbare \glsIdxPl{MetaausdruckV} sind, ist Beispiel (c) zusammengesetzt.
Für alle drei Aussagen lässt sich feststellen, ob sie richtig sind oder nicht.
Das kann man für den zweiten Teil von (c) aber nicht, wenn man nicht weiss worauf sich \strqt{das} bezieht.
Natürlich muss auch der Zusammenhang, in dem ein \glsIdx{MetaausdruckV} formuliert wird, bekannt sein, denn \textzB ist die Bedeutung von \strqt{Ich} nur dann bekannt, wenn man weiss von wem die Aussage ist.
Auf eine exakte Definition von \strqt{\glsIdx{MetaausdruckV}} wird verzichtet, weil das intuitive Verständnis hier ausreicht.
In erster Näherung können aber alle sprachlichen Ausdrücke, die im Prinzip überprüft werden können, als \glsIdxPl{MetaausdruckV} betrachtet werden.

Zusammengesetzte \glsIdxPl{MetaausdruckV} wie (c) können zum Teil formalisiert werden.
Dies wird mit den folgenden Definitionen erreicht:

\begin{align}
	%
	& A \glsSym{metaimp}   B & \text{steht für }
	& \text{\strqt{\emph{Wenn} $A$ [gilt] \emph{dann} [gilt] [auch] $B$}.}
	\\
	& A \glsSym{metarep}   B & \text{steht für }
	& \text{\strqt{$A$ [gilt] \emph{sofern} $B$ [gilt]}.}
	\\
	& A \glsSym{metaequiv} B & \text{steht für }
	& \text{\strqt{$A$ [gilt] \emph{genau dann wenn} $B$ [gilt]}.}
	\\
	& A \glsSym{metaand}   B & \text{steht für }
	& \text{\strqt{[Es gilt] $A$ \emph{und} $B$}.}
	\\
	& A \glsSym{metaor}    B & \text{steht für }
	& \text{\strqt{[Es gilt] $A$ \emph{oder} $B$}.}
	\formulatoleft
\end{align}

Offensichtlich sind das alles ebenfalls \glsIdxPl{MetaausdruckV}, jetzt aber teilweise formalisiert.
(c) lässt sich dann ausdrücken als \strqt{\strqt{Ich habe ein rotes Auto} $\metaund$ \strqt{das kann 200\,km/h schnell fahren.}}.

Um Verwechslungen mit den logischen Symbolen zu vermeiden, werden für \strqt{und} und \strqt{oder} die Symbole \symqt{$\metaund$} und \symqt{$\metaoder$} verwendet.
$A$ und $B$ können dabei ohne Bedeutungsveränderung vertauscht werden.
-- Ein Symbol für \strqt{nicht} wird hier nicht gebraucht.

Bei den Schlussregeln (\seename\ \sectionname~\vref{sec:Schlussregeln}) wird \symqt{$\mid$} statt \symqt{$\metaund$} bei unterschiedlicher Priorität verwendet.

\GlsIdxPl{MetaausdruckV} können auch geklammert werden, um die Reihenfolge der Auswertung eindeutig zu machen.
\symqt{$\metaimp$}, \symqt{$\metarep$}, \symqt{$\metaequiv$}, \symqt{$\metaund$} und \symqt{$\metaoder$} heißen \emph{\glsIdxPl{MetaoperatorV}}.
Ihre Prioritäten werden im \subsectionname~\vref{sub:Klammerregeln} zusammen mit anderen Operatoren definiert.

Sollen zwei \glsIdxPl{MetaausdruckV} miteinander verglichen werden, muss klar sein auf welche Art; ob \textzB als Zeichenfolgen -- mit oder ohne Wertung der Zwischenräume --, als \glsIdxPl{Wahrheitswert} oder auf sonstige Art.
Wenn die Art des Vergleichs implizit oder explizit klar ist und sich die beiden Ausdrücke auf diese Art vergleichen lassen, heißen sie \emph{\glsIdx{vergleichbar}}.

\subsection{Mit Gleichheit verwandte Symbole}% ---------------------------------
\label{sub:Gleichheit}

\subsubsection{Allgemeine Voraussetzungen}
\label{subsub:Voraussetzungen}

In diesem und allen weiteren \sectionname{}en wird vorausgesetzt:
\begin{itemize}

	\item Wenn mehrere der im Folgenden definierten Operatoren \symqt{$\metadefeq$}, \symqt{$\defeq$}, \symqt{$=$}, \symqt{$\ne$} und \symqt{$\equiv$} verwendet werden, dann im selben Zusammenhang%
	\footnote{%
		Statt von einem \emph{Zusammenhang} könnte man auch von einer \emph{Umgebung} sprechen.
		Diese Bezeichnung ist aber auch ein verbreiteter Fachbegriff, so dass auf seine Verwendung verzichtet wird.
		Die Exaktheit der Begriffe in diesem Dokument soll für Erstellung von \ASBA\ ausreichen; was darüber hinausgeht, ist nicht Inhalt dieses Dokuments.%
	}.

	\item Soweit verwendet sind die \emph{\glsIdxPl{intEigenschaftA}} für \symqt{$=$} und \symqt{$\equiv$} bekannt.
	Dabei muss jede \glsIdx{intEigenschaftA} für \symqt{$\equiv$} auch eine für \symqt{$=$} sein.

	\item Soweit verwendet sind die jeweiligen Operanden von \symqt{$=$} und \symqt{$\equiv$} \glsIdx{vergleichbar}.

\end{itemize}

\subsubsection{Definition der mit Gleichheit verwandten Symbole}
\label{subsub:DefinitionGleichheit}

Unter den Voraussetzungen von \subsubsectionname~\vref{subsub:Voraussetzungen} werden die folgenden (metasprachlichen) Operatoren definiert:
\begin{description}

	\item[$\glsSym{eq}$~~\emph{\Idx{Gleichheit}}]\label{def:Gleichheit}
	\forqt{$A = B$} heißt, dass $A$ und $B$ sich in den \glsIdxPl{intEigenschaftA} für \symqt{$=$} nicht unterscheiden.%
	\footnote{%
		\textZB sind zwei logische Operatoren gleich, wenn sie stets denselben \emph{\gls{Wahrheitswert}} liefern.
	}
	-- \strqt{$A$ ist \emph{dasselbe} wie $B$} oder \strqt{$A$ ist \emph{identisch} zu $B$}
	-- Inwieweit die Begriffe \emph{Gleichheit} und \emph{Identität} korrelieren, wird hier nicht erörtert. \seename~\cite{bib:Identitaet}

	\item[$\glsSym{ne}$~~\emph{\Idx{Ungleichheit}}]\label{def:Ungleichheit}
	\forqt{$A \ne B$} heißt, dass $A$ und $B$ sich in mindestens einer der \glsIdxPl{intEigenschaftA} für \symqt{$=$} unterscheiden. \strqt{$A$ ist \emph{nicht dasselbe} wie $B$} (aber vielleicht das gleiche) oder \strqt{$A$ ist \emph{nicht identisch} zu $B$}.

	\item[$\glsSym{equiv}$~~\emph{\Idx{Äquivalenz}}]\label{def:Äquivalenz}
	\forqt{$A \equiv B$} heißt, dass $A$ und $B$ sich in den \glsIdxPl{intEigenschaftA} für \symqt{$\equiv$} nicht unterscheiden.
	-- \strqt{$A$ ist \emph{das gleiche} wie $B$} oder \strqt{$A$ ist \emph{so wie} $B$}.

	\item[$\glsSym{notequiv}$~~\emph{\Idx{Kontravalenz}}]\label{def:Kontravalenz}
	\forqt{$A \notequiv B$} heißt, dass $A$ und $B$ sich in mindestens einer der \glsIdxPl{intEigenschaftA} für \symqt{$\notequiv$} unterscheiden.
	-- \strqt{$A$ ist \emph{nicht das gleiche} wie $B$} oder \strqt{$A$ ist \emph{nicht so wie} $B$}.

	\item[$\glsSym{metadefeq}$~\emph{\Idx{Metadefinition}}]\label{def:Metadefinition}
	\forqt{$A \metadefeq B$} heißt, dass der Metaausdruck $A$ \emph{definitionsgemäß gleich} dem Metaausdruck $B$ ist, wobei $B$ auch eine Definition in natürlicher Sprache sein kann.
	$A$ und $B$ können sich gegenseitig ersetzten.

	Üblicherweise sind $A$ und $B$ hier Aussagen, so dass man \forqt{$A \metadefeq B$} auch als \strqt{$A$ gilt \emph{definitionsgemäß genau dann, wenn} $B$ gilt} lesen kann.

	Oft wird damit Gleichheit (\symqt{$=$}), Äquivalenz (\symqt{$\equiv$}) oder eine andere Relation definiert.

	\item[$\glsSym{defeq}$~\emph{\Idx{Definition}}]
	\forqt{$A \defeq B$} heißt, dass der Ausdruck $A$ \emph{definitionsgemäß gleich} dem Ausdruck $B$ ist.
	Gewissermaßen ist $A$ nur eine andere Schreibweise für $B$.
	$A$ und $B$ können sich gegenseitig ersetzten.%
	\footnote{%
		Nach den Definitionen von \symqt{$\metadefeq$} und \symqt{$\defeq$} sind zwei Ausdrücke $P$ und $Q$ schon dann gleich, wenn nach der Ersetzung aller Vorkommen von $A$ durch $B$ sowohl in $P$ als auch in $Q$ die resultierenden Ausdrücke $\overline{P}$ und $\overline{Q}$ gleich sind.%
	}
	-- Man beachte, dass dies \symqt{$\metadefeq$} und \symqt{$\defeq$} verschiedene Sprachebenen sind.

\end{description}
Es sei noch
\begin{equation}
	\label{eq:asM}
	\glsSym{asM} \defeq \{
		\glsIdxBg{metaand}{\metaund}, \glsIdxBg{metaor}{\metaoder}, \glsSym{metaimp}, \glsSym{metaequiv}, \glsSym{metarep}, \glsSym{eq}, \glsSym{ne}, \glsSym{equiv}, \glsSym{notequiv}, \glsSym{metadefeq}, \glsSym{defeq}
	\}
\end{equation}
die Menge der \glsIdxBg{MetaoperatorV}{metasprachlichen Operatoren} und der mit Gleichheit verwandten Symbole.

\section{Formale Elemente}% ====================================================
\beginsection{Formale Elemente}
\label{sec:Formalelement}

Ein \emph{\glsIdx{FormalelementV}} kann \textzB eine Menge, Zeichenfolge, Zahl, Formel, \textusw sein.
Zwei \glsIdxPl{FormalelementV} $A$ und $B$ sind \emph{\glsIdx{vergleichbar}}, wenn beide von derselben Art sind, \textdh wenn \textzB jeweils beide Mengen, Zeichenfolgen, Zahlen oder Formeln -- die vergleichbare Ergebnisse liefern -- sind.

Intuitiv scheint klar zu sein, was damit  gemeint ist.
Wenn aber entschieden werden muss, ob \textzB (a) \strqt{1+1} gleich \strqt{2} oder (b) \strqt{1+1} gleich \strqt{1 + 1} ist, muss man erst entscheiden, von welcher Art die beiden zu vergleichenden Ausdrücke sind, \textdh \emph{wie} verglichen wird.
Wenn sie als jeweiliges Ergebnis der beiden Formeln verglichen werden, dann ist (a) richtig.
Wenn sie als Formeln, \textdh als Zeichenfolgen, verglichen werden ist (a) falsch.
Wenn die Ausdrücke in (b) als Zeichenfolgen verglichen werden, ist (b) dann richtig, wenn der Zwischenraum zwischen den einzelnen Zeichen nicht zählt.
Wenn er aber zählt, ist (b) falsch.

Im Zusammenhang mit binären Relationen werden noch einige Verabredungen getroffen.
Dazu seien \symqt{$\glsSym{relbsp}$}, \symqt{$\glsSym{releqbsp}$}, \symqt{$\glsSym{lrelbsp}$}, \symqt{$\glsSym{rrelbsp}$}, \symqt{$\glsSym{lreleqbsp}$} und \symqt{$\glsSym{rreleqbsp}$} Beispielsymbole für Relationen und \symqt{$\glsSym{eq}$} und \symqt{$\glsSym{ne}$} die in \sectionname~\vref{sub:Gleichheit} definierten Symbole für Gleichheit und Ungleichheit.
Wenn dann nichts anderes gesagt wird gelte stets:
\begin{align}
	& ((A \relbsp   B) \metaor (A = B)) & \metaequiv &&& (A \releqbsp  B)
	\label{eq:coreleq}   \\
	& (A \lrelbsp   B)                  & \metaequiv &&& (B \rrelbsp   A)
	\label{eq:colrrel}   \\
	& (A \lreleqbsp B)                  & \metaequiv &&& (B \rreleqbsp A)
	\label{eq:colrreleq} \formulatoleft
\end{align}

Mit der Definition einer Relation der einen Seite ist damit automatisch auch die der anderen Seite erfolgt, mit der Ausnahme, dass man \forqt{$A \relbsp B$} so nicht mit Hilfe von \forqt{$A \releqbsp B$} definieren kann.
Dies könnte man zwar mit Hilfe des Ansatzes
\begin{align}
	& (A \relbsp B) &\formulaspace \metaequiv &&&
	(A \releqbsp B) \metaand (A \ne B) \label{eq:corel} \formulatoleft
\end{align}
versuchen, aber die so definierte Relation \symqt{$\relbsp$} kann, muss aber nicht mit der in \eqref{eq:coreleq} übereinstimmen.
Allerdings lässt sich \eqref{eq:coreleq} aus \eqref{eq:corel} ableiten und wenn \forqt{$(A = B) \metaimp (A \releqbsp B)$} gilt, auch \eqref{eq:corel} aus \eqref{eq:coreleq}.
-- Auf einen Beweis wird hier verzichtet.

Es sei noch angemerkt, dass wegen \eqref{eq:corel} die Definition von \symqt{$\metarep$} in \sectionname~\vref{sub:Metaausdruck} überflüssig ist und wegen der Klammerregeln (\seename\ \subsectionname~\vref{sub:Klammerregeln}) auch alle Klammern in diesem \sectionname~\ref{sec:Formalelement}.
Die Prioritäten der Operatoren \symqt{$\lrelbsp$}, \symqt{$\rrelbsp$}, \symqt{$\lreleqbsp$} und \symqt{$\rreleqbsp$} unterscheiden sich normalerweise nicht; ebenso wenig die der Operatoren \symqt{$\relbsp$} und \symqt{$\releqbsp$}, die aber durchaus verschieden von den Prioritäten von \symqt{$=$} und \symqt{$\ne$} sein können.

Als Beispielsymbol für binäre Operatoren wird \symqt{$\opbsp$} verwendet.
Mit \symqt{$\opbsp$} zusammenhängende Verabredungen werden hier nicht getroffen.

\section{Aussagenlogik}% =======================================================
\beginsection{Aussagenlogik}
\label{sec:Aussagenlogik}
\hidden{\glsIdx{Aussagenlogik}}

\subsection{Konstante und Operatoren}% -----------------------------------------
\label{sub:Operatoren}

Die \tablename~\vref{tab:Symbole}
\footnote{%
	Die \tablename{} basiert auf den Wahrheitstafeln in~\cite{bib:Junktor} Kapitel~2.2 und~\cite{bib:Rautenberg} Kapitel~1.1 Seite~3.%
}
definiert für die zweiwertige Logik Konstanten- und Operatorsymbole über die Wahrheitswerte ihrer Anwendung.
So ergeben sich, abhängig von den Wahrheitswerten der Operanden $A$ und $B$
\footnote{%
	Im Gegensatz zu \subsubsectionname~\vref{subsub:Bausteine} können A und B hier beliebige Aussagen -- auch Formeln -- sein.%
},
die in der \tablename{} angegebenen Wahrheitswerte für die Operationen.
Die mit 0, 1 und 2 benannten Spalten werden jeweils nur für die 0-, 1- und 2-stelligen Operatoren, \textdh für die Konstanten, die unären und die binären Operatoren ausgefüllt.
Dabei werden die Konstanten als 0-stellige Operatoren angesehen.
Hat der Inhalt einer Zelle keine Relevanz, steht dort ein Minuszeichen, ist kein Wert bekannt, so bleibt sie leer.

% ==============================================================================
% Definitionen für die folgende Tabelle; siehe auch die Symbole im Vorspann
% Prioritäten - jeweils Prio p* für Symbol \l* ---------------------------------
\newcounter{prio}                                        \stepcounter{prio}
\newcounter{pnequiv} \setcounter{pnequiv} {\value{prio}}
\newcounter{pequiv}  \setcounter{pequiv}  {\value{prio}} \stepcounter{prio}
\newcounter{pnrep}   \setcounter{pnrep}   {\value{prio}}
\newcounter{prep}    \setcounter{prep}    {\value{prio}}
\newcounter{pnimp}   \setcounter{pnimp}   {\value{prio}}
\newcounter{pimp}    \setcounter{pimp}    {\value{prio}} \stepcounter{prio}
%	\newcounter{pnleft}  \setcounter{pnleft}  {\value{prio}}
%	\newcounter{pleft}   \setcounter{pleft}   {\value{prio}}
%	\newcounter{pnright} \setcounter{pnright} {\value{prio}}
%	\newcounter{pright}  \setcounter{pright}  {\value{prio}} \stepcounter{prio}
\newcounter{padd}    \setcounter{padd}    {\value{prio}}
%	\newcounter{pnxor}   \setcounter{pnxor}   {\value{prio}}
\newcounter{pxor}    \setcounter{pxor}    {\value{prio}}
\newcounter{pnor}    \setcounter{pnor}    {\value{prio}}
\newcounter{por}     \setcounter{por}     {\value{prio}} \stepcounter{prio}
\newcounter{pmult}   \setcounter{pmult}   {\value{prio}}
\newcounter{pnand}   \setcounter{pnand}   {\value{prio}}
\newcounter{pand}    \setcounter{pand}    {\value{prio}} \stepcounter{prio}
\newcounter{pnot}    \setcounter{pnot}    {\value{prio}}
% Farben
\definecolor{cNormalUse}{rgb}{.80,.80,.80}
\definecolor{cRareUse}{rgb}{.90,.90,.99}
% ==============================================================================

\begin{table}
	\newcommand*{\tablegroup}{\hdashline[6pt/3pt]}
	\newcommand*{\tableline}{\hdashline[3pt/3pt]}
	\newcommand*{\gapline}{%
		\cdashline{1-1}[1pt/3pt]\cdashline{9-11}[1pt/3pt]}
	\setlength\tabcolsep{3pt}
	\setlength\extrarowheight{1.5pt}
	\begin{threeparttable}
		\begin{tabularx}{\linewidth-10.95pt}{c||c:cc:cccc|X:X|c|}

			A & - & \texttrue & \textfalse &%
			\texttrue  & \texttrue  & \textfalse & \textfalse &
			- & Aussage A & - \\

			\tableline%.................................................
			B & - & -       & -        &%
			\texttrue  & \textfalse & \texttrue  & \textfalse &
			- & Aussage B & - \\

			\hline% -- Überschrift -----------------------------------------

			\textbf{Junktor}\tnote{1}&\textbf{0}&\multicolumn{2}{c:}{%
				\textbf{1}}&\multicolumn{4}{c|}{\textbf{2}}& \textbf{%
				Name}&\textbf{Sprechweise}\tnote{2}&\textbf{Prio}\\%Kein Fehler!

			\hline\hline% == Konstante =====================================

			\rowcolor{cRareUse}
			$\glsSym{ltrue}$
			& \texttrue  & - & - & - & - & - & - & Verum  & Wahr   & - \\

			\tableline%.................................................

			\rowcolor{cRareUse}

			$\glsSym{lfalse}$
			& \textfalse & - & - & - & - & - & - & Falsum & Falsch & - \\

			\hline% -- unäre Operatoren ------------------------------------

			& - & \texttrue  & \texttrue  & - & - & - & -
			&                     &                  & -                 \\

			\tableline%.................................................

			\rowcolor{cNormalUse}

			$\Sym{(\dots)}$
			& - & \texttrue  & \textfalse & - & - & - & -
			& Klammerung\tnote{3} & A ist geklammert & 6\tnote{4}        \\

			\tableline%.................................................

			\rowcolor{cNormalUse}
			$\Sym{\lnot}$
			& - & \textfalse & \texttrue  & - & - & - & -
			& Negation            & Nicht A          & \thepnot\tnote{5} \\

			\tableline%.................................................

			& - & \textfalse & \textfalse & - & - & - & -
			&                     &                  & -                 \\

			\hline% -- binäre Operatoren -----------------------------------

			~ & - & - & - &\texttrue&\texttrue&\texttrue&\texttrue
			& Tautologie
			&
			& - \\

			\tableline%.................................................

			\rowcolor{cNormalUse}

			$\Sym{\lor}$
			& - & - & - &\texttrue&\texttrue&\texttrue&\textfalse
			& Disjunktion; Adjunktion;\newline Alternative
			& A oder B
			& \thepor \\

			\tableline%.................................................

			\rowcolor{cRareUse}
			$\Sym{\lrep}$ $\lrepA$ $\lrepB$
			& - & - & - &\texttrue&\texttrue&\textfalse&\texttrue
			& Replikation; Konversion;\newline konverse Implikation
			& A folgt aus B
			& \theprep \\

			\tableline%.................................................

			$\lleft$
			& - & - & - &\texttrue&\texttrue&\textfalse&\textfalse
			& Präpendenz
			& Identität von A
			& - \\

			\tablegroup% -----------------------------------------------

			\rowcolor{cNormalUse}
			$\Sym{\limp}$ $\limpA$ $\limpB$
			& - & - & - &\texttrue&\textfalse&\texttrue&\texttrue
			& Implikation; Subjunktion;\newline Konditional
			& Wenn A so B; Aus A folgt B; A nur dann wenn B
			& \thepimp \\

			\tableline%.................................................

			$\lright$
			& - & - & - &\texttrue&\textfalse&\texttrue&\textfalse
			& Postpendenz
			& Identität von B
			& - \\

			\tableline%.................................................

			\rowcolor{cNormalUse}
			$\Sym{\lequiv}$ $\lequivA$
			& - & - & - &\texttrue&\textfalse&\textfalse&\texttrue
			& Äquivalenz; Bijunktion;\newline Bikonditional
			& A genau dann wenn B; A dann und nur dann wenn B
			& \thepequiv \\

			\tableline%.................................................

			\rowcolor{cNormalUse}
			$\Sym{\land}$ $\landA$ $\landB$
			& - & - & - &\texttrue&\textfalse&\textfalse&\textfalse
			& Konjunktion
			& {\small A und B; Sowohl A als auch B}
			& \thepand \\

			\tablegroup% -----------------------------------------------

			\rowcolor{cRareUse}
			$\Sym{\lnand}$ $\lnandA$ $\lnandB$
			& - & - & - &\textfalse&\texttrue&\texttrue&\texttrue
			& NAND; Unverträglichkeit;\newline Sheffer-Funktion
			& Nicht zugleich A und B
			& \thepnand \\

			\tableline%.................................................

			\rowcolor{cRareUse}
			$\Sym{\lxor}$ $\lxorA$ $\lxorB$ $\lxorC$
			& - & - & - &\textfalse&\texttrue&\texttrue&\textfalse
			& XOR; Antivalenz;\newline ausschließende Disjunktion
			& Entweder A oder B
			& \thepxor \\

			\gapline%. . . . . . . . . . . . . . . . . . . . . . . . . .

			$\lnequiv$ $\lnequivA$ $\lnequivB$
			& - & - & - &"&"&"&"
			& Kontravalenz
			&
			& - \\

			\tableline%.................................................

			$\lnright$
			& - & - & - &\textfalse&\texttrue&\textfalse&\texttrue
			& Postnonpendenz
			& Negation von B
			& - \\

			\tableline%.................................................

			$\lnimp$ $\lnimpA$ $\lnimpB$
			& - & - & - &\textfalse&\texttrue&\textfalse&\textfalse
			& Postsektion
			&
			& - \\

			\tablegroup% -----------------------------------------------

			$\lnleft$
			& - & - & - &\textfalse&\textfalse&\texttrue&\texttrue
			& Pränonpendenz
			& Negation von A
			& - \\

			\tableline%.................................................

			$\lnrep$ $\lnrepA$ $\lnrepB$
			& - & - & - &\textfalse&\textfalse&\texttrue&\textfalse
			& Präsektion
			&
			& - \\

			\tableline%.................................................

			\rowcolor{cRareUse}
			$\Sym{\lnor}$ $\lnorA$
			& - & - & - &\textfalse&\textfalse&\textfalse&\texttrue
			& NOR; Nihilation;\newline Peirce-Funktion
			& Weder A noch B
			& \thepnor \\

			\tableline%.................................................

			~
			& - & - & - &\textfalse&\textfalse&\textfalse&\textfalse
			& Kontradiktion
			&
			& - \\

			\hline%_____________________________________________________________
		\end{tabularx}
		\begin{tablenotes}
			\footnotesize

			\item[1] \emph{Operatorsymbole}.
			Sie stehen meistens für die Operatoren selbst.
			Der Einfachheit halber werden auch die beiden Konstanten $\ltrue$ und $\lfalse$ als Junktoren \textbzw Operatoren bezeichnet.

			Die Operatoren \symqt{$\subset$}, \symqt{$\supset$}, \symqt{$\nsubset$} und \symqt{$\nsupset$} haben hier nicht die Bedeutung der entsprechenden Operatoren der Mengenlehre und dürfen nicht damit verwechselt werden; entsprechendes gilt für \symqt{$+$} und \symqt{$\cdot$} mit Addition und Multiplikation.

			\item[2] Ist eine Zelle in dieser Spalte leer, so ist die zugehörige Zeile nur vorhanden, um alle binären Operationen aufzuführen.

			\item[3] Klammerung ist genau genommen kein Operator und wird nicht nur bei logischen, sondern auch bei anderen Ausdrücken verwendet.

			\item[4] Die Priorität der Klammern ist größer als die aller Operatoren.

			\item[5] Die Priorität der unären Operatoren muss größer sein als die aller mehrwertigen, also auch der binären Operatoren.
			Wenn alle unären Operatoren auf derselben Seite des Operanden stehen, brauchen sie eigentlich keine Priorität, da die Auswertung nur von innen (dem Operanden) nach außen erfolgen kann.
			Nur wenn es sowohl links-, als auch rechtsseitige unäre Operatoren gibt, muss man für diese Prioritäten definieren.

		\end{tablenotes}
	\end{threeparttable}
	\caption{Definition von aussagenlogischen Symbolen.}
	\label{tab:Symbole}% Erst nach '\caption'!
\end{table}

Für einige Junktoren, Namen und Sprechweisen sind auch Alternativen angegeben.
Die durchgestrichenen (\textdh negierten) Symbole sind ungebräuchlich und nur aus formalen Gründen aufgeführt.
Wenn für eine bestimmte Kombination von Wahrheitswerten mehr als eine Zeile angegeben ist, so sind die zugehörigen Operationen in der zweiwertigen Aussagenlogik alle gleich.
Bei der formalen Definition wird aber keine Zweiwertigkeit vorausgesetzt, so dass je nach Definition die Operationen verschiedene Ergebnisse liefern können.

Um vollständig zu sein, \textdh alle 22 möglichen Kombinationen von Wahrheitswerten für höchstens zwei Variable zu berücksichtigen, enthält die \tablename{} auch viele ungebräuchliche Junktoren und Operationen.
Die Zeilen mit den Klammern und den gebräuchlichsten Junktoren sind in der \tablename{} grau hinterlegt.
Hellgrau hinterlegt sind Zeilen mit weniger gebräuchlichen Junktoren.
Die restlichen Operationen sind uninteressant und brauchen daher keine Priorität.

\subsection{Klammerregeln}% ----------------------------------------------------
\label{sub:Klammerregeln}

Zur Klammerersparnis werden die üblichen Regeln verwendet, \textdh dass Operatoren mit höherer Priorität stärker binden, als solche mit niedrigerer Priorität.

Für die Operatoren derselben Priorität gilt Rechtsklammerung%
\footnote{%
	Unäre Operatoren stehen hier stets links \emph{vor} dem Operanden, so dass es nur Rechtsklammerung geben kann.
	Zur Rechtsklammerung bei binären Operationen ein Zitat aus~\cite{bib:Rautenberg} Kapitel~1.1 Seite~5:
	\enquote{Diese hat gegenüber Linksklammerung Vorteile
		bei der Niederschrift von Tautologien in $\limp$, [...]}%
}.
Im Folgenden wird nur noch ein Teil der logischen Operatoren aus der \tablename~\vref{tab:Symbole} und der \glsIdxBg{MetaoperatorV}{metasprachlichen Operatoren} aus \subsectionname~\vref{sub:Metaausdruck} berücksichtigt.
Diese werden in der \tablename~\vref{tab:Prioritaeten} mit abnehmender Priorität aufgelistet.

\begin{table}[!htb]
	\setlength\extrarowheight{1.5pt}
	\begin{center}
		\begin{threeparttable}
			\begin{tabularx}{12cm}{|@{~~}l|@{\extracolsep{\fill}}l|}
				\hline
				Klammern
				& $      (      \quad      )                          $ \\
				\hline
				Unäre logische Operatoren
				& $ \Sym{\lnot}                                       $ \\
				\hdashline
				& $ \Sym{\land} \quad \Sym{\lmult} \quad \Sym{\lnand} $ \\
				Binäre logische Operatoren
				& $ \Sym{\lor}  \quad \Sym{\ladd}  \quad \Sym{\lnor}  $ \\
				& $ \Sym{\lrep} \quad \Sym{\limp}                     $ \\
				& $ \Sym{\lequiv}                                     $ \\
				\hline
				\parbox[][1.5cm][c]{8.6cm}{%
					Mit Gleichheit verwandte Symbole;\newline
					\small ihre Prioritäten untereinander sind nicht eindeutig
					und bleiben daher undefiniert.
				}
				& $ \glsSym{eq} \quad \glsSym{ne} \quad
				\glsSym{equiv}  \quad \glsSym{notequiv}     $ \\
				\hline
				\glsIdx{Ableitungsrelation}\tnote{1}
				& $ \glsSym{abltb}                          $ \\
				Substitution\tnote{1}
				& $ \glsSym{subst}                          $ \\
				Definition
				& $ \glsSym{defeq}                          $ \\
				\hline
				& $ \glsSym{metaand}                        $ \\
				& $ \glsSym{metaor}                         $ \\
				\GlsIdxPl{MetaoperatorV}
				& $ \glsSym{metarep} \quad \glsSym{metaimp} $ \\
				& $ \glsSym{metaequiv}                      $ \\
				& $ \mid                                    $ \\
				\hline
				Metadefinition
				& $ \glsSym{metadefeq}                      $ \\
				\hline
				\parbox[][1.1cm][c]{5.9cm}{%
					Strukturelemente der natürlichen Sprache, \textzB Satzzeichen\tnote{2}
				}
				& . \quad , \quad ; \quad \textusw \\
				\hline
			\end{tabularx}
			\begin{tablenotes}
				\footnotesize
				\item[1] \seename\ \subsectionname~\vref{sub:Basisregeln}
				\item[2] Innerhalb von Formeln können Satzzeichen eine andere Priorität haben.
				Siehe \textzB \subsectionname~\vref{sub:Basisregeln}.
			\end{tablenotes}
		\end{threeparttable}
		\caption{Prioritäten von Operatoren in abnehmender Reihenfolge}
		\label{tab:Prioritaeten}% Erst nach '\caption'!
	\end{center}
\end{table}

Die Prioritäten der logischen Operatoren wurden aus~\cite{bib:Rautenberg} Kapitel~1.1 Seite~5 entnommen und ergänzt und die der \glsIdxBg{MetaoperatorV}{metasprachlichen Operatoren} daran angeglichen.
Wie üblich bindet ein Operator \emph{stärker} als jeder andere mit einer niedrigeren Priorität und \emph{schwächer} als jeder andere mit höherer Priorität.

\subsection{Formalisierung}% ---------------------------------------------------
\label{sub:Formalisierung}

Da sie die Grundlage -- quasi das Fundament -- des mathematischen Inhalts von \ASBA\ sind, müssen die \glsIdxPl{Axiom}, \glsIdxPl{Satz}, \glsIdxPl{Beweis}, \textusw der Aussagenlogik in streng formaler Form vorliegen.
Die Formalisierung stützt sich auf~\cite{bib:Aussagenlogik}; \alsoname~\cite{bib:LogikDe, bib:LogikEn}.
Da Computerprogramme mit der \emph{Polnischen Notation}\idx{Polnische Notation}%
\footnote{%
	Bei der \emph{Polnischen Notation} wird eine zweistellige Operation $(A\circ B)$ dargestellt als $\circ A B$.
	Eine Zwischenstufe ist $\circ(A,B)$, bei der noch die redundanten Gliederungszeichen Komma und Klammern -- auch andere als die runden -- hinzukommen, so dass die Operationen optisch besser getrennt und dadurch für Menschen besser lesbar werden.
	Durch einfaches Weglassen der Gliederungszeichen ergibt sich dann die Polnische Notation.%
}
besser umgehen können und Klammern dort überflüssig sind, werden viele Formeln auch in die Polnische Notation überführt.

\subsubsection{Bausteine der aussagenlogischen Sprache}% - - - - - - - - - - - -
\label{subsub:Bausteine}

Zur Einteilung der aussagenlogischen Junktoren werden die folgenden Mengen definiert:
\begin{align}
	%
	& \gsNo\hidden{\gls{gsNo}}  & & \defeq & &
	&     & \text{Menge der \emph{natürlichen Zahlen} einschließlich 0}
	\idx{natürlichen Zahlen, Menge der} \label{def:N} \\
	%
	& \glsSym{asC}  & & \defeq & & \{ \ltrue, \lfalse \}
	& ,\; & \text{Menge der \emph{aussagenlogischen Konstanten}}
	\idx{Konstanten, Menge der}         \label{def:C} \\
	%
	& \glsSym{asU}  & & \defeq & & \{ \lnot \}
	& ,\; & \text{Menge der \emph{unären aussagenlogischen Operatoren}}
	\idx{unären Operatoren, Menge der}  \label{def:U} \\
	%
	& \glsSym{asB}  & & \defeq & &
	\{ \land, \lor, \limp, \lequiv, \lrep, \lnand, \lnor, \lmult, \ladd \}
	& ,\; & \text{Menge der \emph{binären aussagenlogischen Operatoren}}
	\idx{binären Operatoren, Menge der} \label{def:B}
	%
\end{align}

Damit sind alle in der \tablename~\vref{tab:Symbole} verwendeten wesentlichen Konstanten und Operatoren%
\footnote{%
	Jeweils nur die ersten der grau hinterlegten Zeilen sowie \symqt{$\lmult$}.%
}
erfasst und es können die folgende Mengen definiert werden:
\begin{align}
	%
	& \glsSym{asV}  & & \defeq    & & \{ p_n \mid n \in \gsNo \}
	& ,\; & \text{Menge der \emph{\glsIdxPl{atomareFormelA}}}
	\idx{atomare Formeln, Menge der}          \label{def:V}  \\
	%
	& \glsSym{asJ}  & & \defeq    & &\asC \cup \asU \cup \asB
	& ,\; & \text{Menge der \emph{Junktoren} \textbzw \emph{Operatoren};
	auch \emph{\glsIdx{logischeSignaturV}}}
	\idx{Junktoren, Menge der}                \label{def:J}  \\
	%
	& \glsSym{asA}  & & \defeq    & & \asV \cup \asJ
	& ,\; & \text{\emph{Alphabet der aussagenlogischen Sprache (für }} \asJ
	\text{\emph{)}}
	\idx{Alphabet der logischen Sprache}      \label{def:A}  \\
	%
	& \glsSym{asJx} & & \subseteq & & \asJ
	& ,\; & \text{eine Teilmenge von } \asJ \text{ für eine Indexvariable }
	x                                         \label{def:Bx} \\
	%
	& \glsSym{asAx} & & \defeq    & & \asV \cup \asJx \quad
	& ,\; & \text{Alphabet der aussagenlogischen Sprache \emph{für} } \asJx
	\idx{Teil-Alphabet der aussagenlogischen Sprache} \label{def:Ax}
	%
\end{align}
Für Elemente aus $\asV$ werden hier normalerweise die großen lateinischen Buchstaben $A$, $B$, $C$, \textusw verwendet.
Die Elemente aus $\asV$ (\glsIdx{atomareFormelA}n) werden auch \emph{\Idx{Satzbuchstabe}n} oder kurz \emph{\Idx{Atom}e} genannt.

\subsubsection{Aussagenlogische Formeln}%  - - - - - - - - - - - - - - - - -
\label{subsub:Formeln}

Neben dem Alphabet $\asA$ \textbzw. $\asAx$ werden noch Klammern als Gliederungszeichen verwendet.
Damit können nun rekursiv für jede Teilmenge $\asJx$ von $\asJ$ zwei Mengen von Formeln definiert werden:

$\glsSym{asFx}$ sei die Menge der auf folgende Weise definierten \emph{aussagenlogischen Formeln mit Klammerung}%
\idx{aussagenlogische Formel mit Klammerung}:
\begin{align}
	&                & \asV            \subset \asFx
	\\
	&                & \asJx \cap \asC \subset \asFx
	\\
	A                                \in \asFx
	& \quad \metaimp &  (\circ A)    \in \asFx
	& & , \quad \text{für} \quad \circ \in \asU \cap \asJx
	\\
	A, B                             \in \asFx
	& \quad \metaimp & (A \circ B)   \in \asFx
	& & , \quad \text{für} \quad \circ \in \asB \cap \asJx
	\formulatoleft
\end{align}
Nur die auf diese Weise konstruierten Formeln sind Elemente von $\asFx$.
\\Für $\asJ = \asJx$ sei noch $\asF \defeq \glsSym{asFx}$.

$\glsSym{asFxp}$ sei die Menge der auf folgende Weise definierten aussagenlogischen Formeln in \emph{Polnischer Notation}%
\idx{aussagenlogische Formel in Polnischer Notation}:
\begin{align}
	&                & \asV            \subset \asFxp
	\\
	&                & \asJx \cap \asC \subset \asFxp
	\\
	A                            \in \asFxp
	& \quad \metaimp & \circ A   \in \asFxp
	& & , \quad \text{für} \quad \circ \in \asU \cap \asJx
	\\
	A, B                         \in \asFxp
	& \quad \metaimp & \circ A B \in \asFxp
	& & , \quad \text{für} \quad \circ \in \asB \cap \asJx
	\formulatoleft
\end{align}
Nur die auf diese Weise konstruierten Formeln sind Elemente von $\asFxp$.
\\Für $\asJ = \asJx$ sei noch $\asFp \defeq \glsSym{asFxp}$.

Wie man leicht sieht, gilt:
\begin{equation}
	\asJx \subset \asJy \subseteq \asJ \metaimp
	\begin{cases}
		\asAx  \subset \asAy  \subseteq \asA \\
		\asFx  \subset \asFy  \subseteq \asF \\
		\asFxp \subset \asFyp \subseteq \asFp
	\end{cases}
\end{equation}

Durch Anwendung der Klammerregeln von \subsubsectionname~\vref{subsub:Bausteine} lassen sich in der Regel noch viele Klammern der Formeln aus $\asFx$ einsparen.
Die Formeln aus $\asFxp$ sind frei von Klammern.
Die Namen der Operatoren finden sich in der \tablename~\vref{tab:Symbole}.
Für aussagenlogische Formeln, \textdh von Elementen aus $\asF$ \textbzw $\asFp$, werden hier normalerweise die kleinen griechischen Buchstaben $\alpha$, $\beta$, $\gamma$, \textusw verwendet.
Sie können dabei auch \glsIdx{atomareFormelA}n bezeichnen (\seename~\eqref{def:V}).

\subsection{Definition aussagenlogische Operatoren durch andere}% ----------
\label{sub:ausOperatorDef}

Im folgenden gelte für zwei aussagenlogische Formeln $\alpha$ und $\beta$:

\begin{itemize}

	\item[] $\alpha   =    \beta \quad \metadefeq$ \quad $\alpha$ und $\beta$
	stimmen als Zeichenkette überein.

	\item[] $\alpha \equiv \beta \quad \metadefeq$ \quad $\alpha$ und $\beta$
	\parbox[t]{11cm}{können mit Hilfe erlaubter Substitutionen und geltender Axiome -- \seename\ \subsectionname~\vref{sub:ausAxiome} -- ineinander überführt werden.}

\end{itemize}

Es werden verschiedene Teilmengen von $\asJ$ -- \glsIdxPl{logischeSignaturV} -- eingeführt, die jeweils ausreichen alle anderen Elemente aus $\asJ$ zu definieren:
\begin{align}
	& \asJ_\xBool &\defeq & & & \{ \lnot, \land, \lor \} \label{def:Jbool}
	& \formulaspace       & \text{, \glsIdx{BoolscheSignatur}}            \\
	& \asJ_\xAnd  &\defeq & & & \{ \lnot, \land       \} \label{def:Jand} \\
	& \asJ_\xOr   &\defeq & & & \{ \lnot, \lor        \} \label{def:Jor}  \\
	& \asJ_\xImp  &\defeq & & & \{ \lnot, \limp       \} \label{def:Jimp} \\
	& \asJ_\xRep  &\defeq & & & \{ \lnot, \lrep       \} \label{def:Jrep} \\
	& \asJ_\xNand &\defeq & & & \{ \lnand             \} \label{def:Jnand}\\
	& \asJ_\xNor  &\defeq & & & \{ \lnor              \} \label{def:Jnor}
	\formulatoleft
\end{align}

Im Folgenden stehen jeweils links die Formeln in üblicher Schreibweise vollständig geklammert und rechts in Polnischer Notation (ohne Klammern).
Ferner seien $\alpha$ und $\beta$ beliebige, nicht notwendig verschiedene Formeln aus der passenden Menge $\asFx$ \textbzw der um die mit Hilfe der Definitionen erweiterten Formelmenge.

Ausgehend von den Operatoren aus der \glsIdxX{BoolscheSignatur} $\asJ_\xBool$ werden die restlichen Operatoren aus $\asJ$ definiert. Die Definitionen sind in zwei Gruppen eingeteilt, und zwar die mit den Operatoren aus $\asJ_\xAnd$:
\begin{align}
	% folgt ------------------------
	& (\alpha \limp \beta) & \defeq & & & (\lnot (\alpha \land  (\lnot \beta)))
	& \formulaspace
	&  \limp \alpha \beta  & \defeq & & &  \lnot    \land \alpha \lnot \beta
	\label{def:imp}
	\\
	% sofern -----------------------
	& (\alpha \lrep \beta) & \defeq & & & (\lnot (\beta \land  (\lnot \alpha)))
	& \formulaspace
	&   \lrep \beta \alpha & \defeq & & &  \lnot    \land \beta \lnot \alpha
	\label{def:rep}
	\\
	% genau dann -------------------
	& (\alpha\lequiv\beta)&\defeq &&&((\alpha\limp\beta)\land(\alpha\lrep\beta))
	& \formulaspace
	& \lequiv\alpha \beta &\defeq &&&\land \limp \alpha \beta \lrep \alpha \beta
	\label{def:equiv}
	\\
	%falsch ------------------------
	&              \lfalse & \defeq & & & (p_0 \land    (\lnot p_0))
	& \formulaspace
	&              \lfalse & \defeq & & &      \land p_0 \lnot p_0
	\label{def:false}
	\\
	% mal --------------------------
	& (\alpha \lmult \beta) & \defeq & & & (\alpha \land \beta)
	& \formulaspace
	&  \lmult \alpha \beta  & \defeq & & &  \land \alpha \beta
	\label{def:mult}
	\\
	% NAND -------------------------
	& (\alpha \lnand \beta)  & \defeq & & & (\lnot (\alpha \land \beta ))
	& \formulaspace
	&  \lnand \alpha \beta   & \defeq & & &  \lnot  \land \alpha \beta
	\label{def:nand}
	\formulatoleft
\end{align}
und die mit den Operatoren aus $\asJ_\xOr$:
\begin{align}
	% NOR --------------------------
	& (\alpha \lnor \beta) & \defeq & & & (\lnot (\alpha \lor \beta))
	& \formulaspace
	&  \lnor \alpha \beta  & \defeq & & &  \lnot  \lor \alpha \beta
	\label{def:nor}
	\\
	% plus -------------------------
	&(\alpha\ladd\beta)&\defeq&&&((\alpha\lor\beta)\land(\lnot(\alpha\land\beta)))
	& \formulaspace
	& \ladd\alpha\beta &\defeq&&&  \land \lor \alpha\beta \lnot \land \alpha\beta
	\label{def:add}
	\\
	% wahr -------------------------
	& \ltrue & \defeq & & & (p_0 \lor    (\lnot p_0))
	& \formulaspace
	& \ltrue & \defeq & & &      \lor p_0 \lnot p_0
	\label{def:true}
	\formulatoleft
\end{align}

Ist \symqt{$\lor$} oder \symqt{$\land$} nicht vorgegeben, \textdh wird von den Elementen aus $\asJ_\xAnd$ \textbzw $\asJ_\xOr$ statt von denen aus $\asJ_\xBool$ ausgegangen, so muss man den fehlenden Operator mittels der passenden der beiden folgenden Definitionen einführen:
\begin{align}
	% oder aus und -----------------
	& (\alpha \lor \beta) & \defeq & & & (\lnot((\lnot\alpha)\land(\lnot\beta)))
	& \formulaspace
	&  \lor \alpha \beta  & \defeq & & &  \lnot \land \lnot \alpha \lnot \beta
	\label{def:orand}
	\\
	% und aus oder -----------------
	& (\alpha \land \beta) & \defeq & & & (\lnot((\lnot\alpha)\lor(\lnot\beta)))
	& \formulaspace
	& \land \alpha \beta   & \defeq & & &  \lnot \lor \lnot \alpha \lnot \beta
	\label{def:andor}
	\formulatoleft
\end{align}
Nun sind wieder alle Operatoren definiert.

Entsprechend wird bei Vorgabe von $\asJ_\xImp$ \textbzw $\asJ_\xRep$ die passende der beiden folgenden Definitionen ausgewählt:
\begin{align}
	% oder aus imp -----------------
	& (\alpha \lor  \beta) & \defeq & & & ((\lnot \alpha) \limp \beta)
	& \formulaspace
	&  \lor  \alpha \beta  & \defeq & & &   \limp \lnot \alpha \beta
	\label{def:orrep}
	\\
	% und aus rep ------------------
	& (\alpha \land \beta) & \defeq & & & (\lnot ((\lnot \beta) \lrep \alpha))
	& \formulaspace
	&  \land \alpha \beta  & \defeq & & &  \lnot \lrep \lnot \beta \alpha
	\label{def:andrep}
	\formulatoleft
\end{align}
woraufhin dann \eqref{def:imp} \textbzw \eqref{def:rep} als Gleichung nachzuweisen ist.
Da aus \eqref{def:rep} durch Vertauschung der Variablen unmittelbar
\begin{align}
	& (\alpha \lrep \beta) & \equiv & & & (\beta \limp \alpha)
	& \formulaspace
	&  \lrep \alpha \beta  & \equiv & & &  \limp \beta \alpha
	\label{eq:repimp}
	\formulatoleft
\end{align}
folgt, vermindert sich der Aufwand dazu erheblich.

Bei Vorgabe von $\asJ_\xNand$ \textbzw $\asJ_\xNor$ schließlich werden die passenden Definition aus
\begin{align}
	% nicht aus nor ----------------
	& (\lnot \alpha) & \defeq & & & (\alpha \lnor \alpha)
	& \formulaspace
	&  \lnot \alpha  & \defeq & & &  \lnor \alpha \alpha
	\label{def:notnor}
	\\
	% nicht aus nand ---------------
	& (\lnot \alpha) & \defeq & & & (\alpha \lnand \alpha)
	& \formulaspace
	&  \lnot \alpha  & \defeq & & &  \lnand \alpha \alpha
	\label{def:notnand}
	\formulatoleft
\end{align}
und, da \symqt{$\lnot$} jetzt definiert ist, aus
\begin{align}
	% oder aus nor -----------------
	& (\alpha \lor \beta) & \defeq & & & (\lnot(\alpha \lnor \beta))
	& \formulaspace
	&  \lor \alpha \beta  & \defeq & & &  \lnot \lnor \alpha \beta
	\label{def:ornor}
	\\
	% und aus nand -----------------
	& (\alpha \land \beta) & \defeq & & & (\lnot(\alpha \lnand \beta))
	& \formulaspace
	&  \land \alpha \beta  & \defeq & & &  \lnot \lnand \alpha \beta
	\label{def:andnand}
	\formulatoleft
\end{align}
ausgewählt und es ist \eqref{def:nand} \textbzw \eqref{def:nor} als Gleichung nachzuweisen.

Abschließend ist noch nachzuweisen, dass mit Hilfe der jeweils passenden der Definitionen \eqref{def:imp} bis \eqref{def:andnand}, ausgehend vom jeweils passenden $\asFx$, genau die gesamte Formelmenge $\asF$ erzeugt werden kann.

\section{Schlussregeln}% =======================================================
\beginsection{Schlussregeln}
\label{sec:Schlussregeln}
\hidden{\glsIdx{Schlussregel}}

Die Regeln zur Formulierung und Prüfung der Beweise müssen fest codiert werden.
Sie sind quasi die Axiome und sollten daher möglichst wenig voraussetzen.
Für \ASBA\ wird ein \emph{Genzen-Kalkül} verwendet, so wie er in~\cite{bib:Rautenberg} Kapitel~1.4 beschrieben ist (siehe auch~\cite{bib:NatuerlichesSchliessen,bib:Schlussregel}).

Ein Beweis in \ASBA\ besteht aus einer Menge von Zeichenketten (die Voraussetzungen) und einer Folge von erlaubte Transformationen, die sogenannten \emph{\glsIdxPl{Schlussregel}}, mit der neue (Mengen von) Zeichenketten generiert werden, bis eine gewünschte Menge von Zeichenketten (die Folgerungen) abgeleitet sind.

In diesem \sectionname\ geht es um die erlaubten Transformationen. Dazu gehören zunächst die \glsIdxPl{Basisregel}.

\subsection{Basisregeln}%  - - - - - - - - - - - - - - - - - - - - - - - - - - -
\label{sub:Basisregeln}
\hidden{\glsIdxPl{Basisregel}}

Gemäß \cite{bib:Rautenberg} Kapitel~1.4 \emph{Ein vollständiger aussagenlogischer Kalkül} werden sechs \glsIdxPl{Basisregel} definiert. Zuvor werden aber noch einige Definition gebraucht. Dazu seien $n$, $m$, $k$ und $l$ natürliche Zahlen (auch 0), $\alpha$, $\alpha_i$, $\beta$ und $\beta_j$ Formeln, $X$, $X_i$, $Y$ und $Y_j$ Mengen von Formeln und
\begin{align}
	%
	&X&&\defeq&&X_1\cup X_2\cup...\cup X_n\cup\{\alpha_1,\alpha_2,...,\alpha_m\}
	\\
	&Y&&\defeq&&Y_1\cup Y_2\cup...\cup Y_k\cup\{\beta_1, \beta_2, ...,\beta_l \}
	\formulatoleft\formulatoleft
\end{align}
Damit wird definiert:
\begin{align}
	%
	& \alpha \glsSym{abltb} \beta \quad \metadefeq \quad
	\parbox[t]{10.5cm}{%
		$\beta$ ist mittels schrittweiser Anwendung \glsIdxBg{zulaessigeTransformation}{zulässiger Transformationen} (siehe unten) aus $\alpha$ ableitbar.
		Sprechweise: Aus $\alpha$ ist $\beta$ \emph{ableitbar} oder \emph{beweisbar};
		kurz: \strqt{$\alpha$ \emph{\glsIdx{ableitbar}} $\beta$} \textbzw \strqt{$\alpha$ \emph{\glsIdx{beweisbar}} $\beta$}
		-- Es kann auch \symqt{$\alpha$} durch \symqt{$X$} und/oder \symqt{$\beta$} durch \symqt{$Y$} ersetzt werden.
	}
	\label{def:ableitbar} \\
	%
	&            X_1, X_2, ...,X_n, \alpha_1, \alpha_2, ..., \alpha_m \quad
	\abltb \quad Y_1, Y_2, ...,Y_n,  \beta_1,  \beta_2,  ..., \beta_m \quad
	\metadefeq \quad X \abltb Y
	\label{def:ableitbarKurz} \formulatoleft
\end{align}

Eine \emph{\glsIdx{zulaessigeTransformation}} ist die Anwendung einer \glsIdx{Basisregel} aus \subsectionname~\vref{sub:Basisregeln} oder einer davon abgeleiteten sonstigen \glsIdx{Schlussregel}, \textzB aus \subsectionname~\vref{sub:Schlussregeln}.

Seien wieder $\alpha$ und $\beta$ Formeln und $X$ und $Y$ Mengen von Formeln.
Für die sechs Basisregeln werden dann nur noch die logischen Operatoren \symqt{$\lnot$} und \symqt{$\land$} aus der logischen \glsIdx{logischeSignaturV} $\asJ_\xAnd$ benötigt.
Bei den weiteren \glsIdxPl{Schlussregel} wird noch \symqt{$\limp$} gemäß der Definition~\vref{def:imp} verwendet.

Bei den \glsIdxPl{Schlussregel} und der Substitution soll das Komma stärker binden als \symqt{$\abltb$} und \symqt{$\subst$}.
Nach \subsectionname~\vref{sub:Metaausdruck} steht \symqt{$\mid$} für \strqt{und} \textbzw \symqt{\glsIdx{metaand}}
und bindet nach \subsectionname~\vref{sub:Klammerregeln} schwächer als \symqt{$\abltb$} und \symqt{$\subst$}.

Bei der Auswahl der \glsIdxPl{Basisregel}, der Formulierung und der Bezeichnungen wird auf~\cite{bib:Rautenberg,bib:NatuerlichesSchliessen} zurückgegriffen.
Wie in~\cite{bib:NatuerlichesSchliessen} steht \charqt{E} für \strqt{-Einführung} und \charqt{B} für \strqt{-Beseitigung} (oder \strqt{-Elimination}) von Operatoren.

\begin{align}
	& & & \frac{}{\alpha\abltb\alpha}
	& & (\glsIdxPl{Anfangsregel})
	\tag{\tagAR} \sym{\gls{AR}} \label{def:Anfangsregel}
	\\\\
	& & & \frac{X\abltb\alpha}{Y\abltb\alpha}
	\quad \text{, für } Y \supseteq X
	& & (\glsIdxPl{Monotonieregel})
	\tag{\tagMR} \sym{\gls{MR}} \label{def:Monotonieregel}
	\\\\
	& & & \frac{X\abltb\alpha,\lnot\alpha}{X\abltb\beta}
	& & (\text{Einführung/Beseitigung der Negation Teil 1})
	\tag{\tagnota} \sym{\gls{nota}} \label{def:nota}
	\\\\
	& & & \frac{X,\alpha\abltb\beta \mid X,\lnot\alpha\abltb\beta} {X\abltb\beta}
	& & (\text{Einführung/Beseitigung der Negation Teil 2})
	\tag{\tagnotb} \sym{\gls{notb}} \label{def:notb}
	\\\\
	& & & \frac{X\abltb\alpha,\beta}{X\abltb\alpha\land\beta}
	& & (\text{Einführung der Konjunktion})
	\tag{\tagandE} \sym{\gls{andE}} \label{def:andE}
	\\\\
	& & & \frac{X\abltb\alpha\land\beta}{X\abltb\alpha,\beta}
	& & (\text{Beseitigung der Konjunktion})
	\tag{\tagandB} \sym{\gls{andB}} \label{def:andB}
	\formulatoleft
\end{align}

Eine Schlussregel steht für die Aussage, dass mit den Formeln über dem Strich auch die unter dem Strich gelten.
Im Gegensatz zu den weiteren Schlussregeln werden die Basisregeln nicht weiter hinterfragt, \textdh sie gelten als Axiome.

\subsection{Weitere Schlussregeln}%  - - - - - - - - - - - - - - - - - - - - - - - - - - -
\label{sub:Schlussregeln}

In~\cite{bib:Rautenberg} werden aus den Basisregeln mittels \glsIdxBg{zulaessigeTransformation}{zulässiger Transformationen} weitere Schlussregeln abgeleitet. Man vergleiche auch mit~\cite{bib:NatuerlichesSchliessen}.

\begin{align}
	& & & \frac{X,\lnot\alpha\abltb\alpha}{X\abltb\alpha}
	& & (\text{Indirekter Beweis})
	\tag{\tagnotc} \sym{\gls{notc}} \label{def:notc}
	\\\\
	& & & \frac{X,\lnot\alpha\abltb\beta,\lnot\beta}{X\abltb\alpha}
	& & (\text{Reductio ad absurdum})
	\tag{\tagnotd} \sym{\gls{notd}} \label{def:notd}
	\\\\
	& & & \frac{X,\alpha\abltb\beta}{X\abltb\alpha\limp\beta}
	& & (\text{Einführung der Implikation})
	\tag{\tagimpE} \sym{\gls{impE}} \label{def:impE}
	\\\\
	& & & \frac{X\abltb\alpha\limp\beta}{X,\alpha\abltb\beta}
	& & (\text{Beseitigung der Implikation})
	\tag{\tagimpB} \sym{\gls{impB}} \label{def:impB}
	\\\\
	& & & \frac{X\abltb\alpha \mid X,\alpha\abltb\beta}{X\abltb\beta}
	& & (\glsIdxPl{Schnittregel}\text{ (Modus ponens)})
	\tag{\tagSR} \sym{\gls{SR}} \label{def:SR}
	\\\\
	& & & \frac{X\abltb\alpha,\alpha\limp\beta}{X\abltb\beta}
	& & (\glsIdxPl{Abtrennungsregel})
	\tag{\tagTR} \sym{\gls{TR}} \label{def:TR}
	\formulatoleft
\end{align}

\subsection{Beispiel einer Ableitung}%  - - - - - - - - - - - - - - - - - - - - - - - - - - -
\label{sub:BeispielAbleitung}

Als Beispiel wird hier die Schnittregel aus den Basisregeln abgeleitet.\footnote{%
	Die Form der Tabelle ist angelehnt an~\cite{bib:NatuerlichesSchliessen} Kapitel~2.2.4 \emph{Eine Beispielableitung}.%
}
Dazu müssen Substitutionen verwendet werden.
Dazu sei $\alpha$ eine Teilformel der Ausgangsformel $\gamma$.
Wenn nun alle oder einige Vorkommen von $\alpha$ in $\gamma$ durch die Teilformel $\beta$ ersetzt werden, dann wurde $\alpha$ durch $\beta$ \emph{substituiert}; in Formelschreibweise: \forqt{$\alpha\subst\beta$}.
Eine der beiden Teilformeln sollte dabei eine Variable sein, die man per Definition mit der anderen Seite gleichsetzen muss; zumindest sollte das ohne Widerspruch möglich sein.
%TODO Erläutern, wann eine Substitution zulässig ist.

\begin{table}[!htb]
	\begin{tabular}{|c|l|l|l|l|l|}
		\hline
		\rule[-1ex]{0pt}{2.5ex} \textbf{Zeile} & \textbf{Abhängigkeit(en)} & \textbf{Aussage} & \textbf{Regel} & \textbf{angewendet auf} & \textbf{Substitutionen} \\
		\hline
		\rule[-1ex]{0pt}{2.5ex} 1 & 1 & $ X      \abltb\alpha $ & fixe Annahme && \\
		\hline
		\rule[-1ex]{0pt}{2.5ex} 2 & 2 & $ X,\alpha\abltb\beta $ & fixe Annahme && \\
		\hline
		\rule[-1ex]{0pt}{2.5ex} 3 & 3 & $ X       \abltb\beta $ & Annahme      && \\
		\hline
		\rule[-1ex]{0pt}{2.5ex} 4 & 1 & $ X,\lnot\alpha\abltb\alpha $ & \glsIdxPl{MR} & 1 & $ Y \subst X,\lnot\alpha $ \\
		\hline
	%TODO Beispiel vollenden
		\rule[-1ex]{0pt}{2.5ex} 5 &  &  &  &  &  \\
		\hline
		\rule[-1ex]{0pt}{2.5ex} 6 &  &  &  &  &  \\
		\hline
		\rule[-1ex]{0pt}{2.5ex} 7 &  &  &  &  &  \\
		\hline
	\end{tabular}
	\caption{Ableitung der \glsIdx{Schnittregel} aus den \glsIdxPl{Basisregel}}
\end{table}

\subsubsection{Aussagenlogisches Axiomensystem}% - - - - - - - - - - - - - - - -
\label{subsub:ausAxiome}

Ausgehend von der \glsIdxBg{logischeSignaturV}{logischen Signatur} $\asJ_\xAnd = \{\lnot, \land\}$ und den Definitionen \eqref{def:imp} bis \eqref{def:orand} werden die folgenden vier logischen Axiome definiert:
\begin{align}
	&
	(\alpha\limp\beta\limp\gamma)\limp(\alpha\limp\beta)\limp(\alpha\limp\gamma)
	\formulaspace &
	& \limp\limp\alpha\limp\beta\gamma\limp\limp\alpha\beta\limp\alpha\gamma \\
	%
	& \alpha \limp \beta \limp \alpha \land \beta
	\formulaspace &
	& \limp \alpha \limp \beta \land \alpha \beta \\
	%
	& \alpha \land \beta \limp \alpha \;; \quad \alpha \land \beta \limp \beta
	\formulaspace &
	& \limp \land \alpha \beta \alpha \;; \quad \limp \land \alpha \beta \beta\\
	%
	&(\alpha \limp \lnot \beta) \limp (\beta \limp \lnot \alpha)
	\formulaspace &
	& \limp \limp \alpha \lnot \beta \limp \beta \lnot \alpha
	\formulatoleft
	%
\end{align}

\subsection{Aussagenlogische Axiome}% ------------------------------------------
\label{sub:ausAxiome}

\todo{Aussagenlogik weiter bearbeiten.}%%%
%TODO Aussagenlogik weiter bearbeiten. %%%

\section{Prädikatenlogik}% =====================================================
\beginsection{Prädikatenlogik}
\label{sec:Praedikatenlogik}
\hidden{\glsIdx{Praedikatenlogik}}
%TODO Fehler: Bei Prädikatenlogik im Glossar fehlt Verweis auf diese Stelle

\todo{Prädikatenlogik bearbeiten.}%%%
%TODO Prädikatenlogik bearbeiten. %%%

\section{Mengenlehre}% =========================================================
\beginsection{Mengenlehre}
\label{sec:Mengenlehre}

\todo{Mengenlehre bearbeiten.}%%%
%TODO Mengenlehre bearbeiten. %%%

\Endchapter
