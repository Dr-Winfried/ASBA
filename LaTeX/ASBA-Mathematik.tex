%%############################################################################%%
%%                                                                            %%
%% Datei:  ASBA-Mathematik.tex                                                %%
%% Inhalt: Kapitel "Mathematische Grundlagen"                                 %%
%%                                                                            %%
%% Copyright (C) 2017  Winfried Teschers                                      %%
%%                                                                            %%
%% This program is free software: you can redistribute it and/or modify       %%
%% it under the terms of the GNU Affero General Public License as published   %%
%% by the Free Software Foundation, either version 3 of the License, or       %%
%% (at your option) any later version.                                        %%
%%                                                                            %%
%% This program is distributed in the hope that it will be useful,            %%
%% but WITHOUT ANY WARRANTY; without even the implied warranty of             %%
%% MERCHANTABILITY or FITNESS FOR A PARTICULAR PURPOSE.  See the              %%
%% GNU Affero General Public License for more details.                        %%
%%                                                                            %%
%% You should have received a copy of the GNU Affero General Public License   %%
%% along with this program.  If not, see <http://www.gnu.org/licenses/>.      %%
%%                                                                            %%
%% Dr. Winfried Teschers                                                      %%
%% Anton-Günther-Straße 26c                                                   %%
%% 91083 Baiersdorf                                                           %%
%% Germany                                                                    %%
%%                                                                            %%
%% e-mail: winfried.teschers@t-online.de                                      %%
%%                                                                            %%
%%############################################################################%%

% !TeX root = ASBA.tex
% !TeX encoding = UTF-8
% !TeX spellcheck = de_DE

\chapter{Mathematische Grundlagen}% ############################################
\beginchapter{Mathematische Grundlagen}
\label{cha:Grundlagen}

Die mathematischen Grundlagen werden einerseits gebraucht, um die erlaubten \glsIdxPl{Beweisschritt} zu definieren (\vrefseesec{sec:Schlussregeln}), andererseits dienen sie auch zum Testen von \ASBA.
Daher werden sie in diesem Kapitel ausführlicher behandelt, als für die Erstellung von \ASBA\ erforderlich ist.
Alle hier aufgeführten \glsIdxPl{Axiom}, \glsIdxPl{Satz} und \glsIdxPl{Beweis} sollen dazu kodiert und die \glsIdxPl{Beweis} dann von \ASBA\ verifiziert werden.

\section{Notationen}% ==========================================================
\beginsection{Notationen}
\label{sec:Notationen}

Die \indsec{sec:Notationen} aufgeführten Notationen werden \indcha{cha:Grundlagen} verwendet, ohne nochmals erläutert zu werden. Abweichungen davon werden jeweils gesondert angegeben.

Sätze mit \enquote{wir} bestimmen Elemente, die nur für dieses Dokument gelten.
Bei allgemeingültigen Elementen wird \enquote{wir} nicht verwendet.
Die Verwendung von \enquote{wir} ist allerdings nicht konsistent und soll nur als Hinweis dienen.

\subsection{Symbole und Bezeichnungen}% ----------------------------------------
\label{sub:Bezeichnungen}

\begin{description}
	\item[$\gsN$]   Menge der natürlichen Zahlen ohne           0.
	\item[$\gsNo$]  Menge der natürlichen Zahlen einschließlich 0.
\end{description}

\subsection{Quotierung}% -------------------------------------------------------
\label{sub:Quotierung}

Ein Zeichen oder Wort der natürlichen Sprache kann verschiedene Bedeutungen haben.
\textZB\ steht das Wort \enquote{sin} für das \glsIdx{Objekt} Sinusfunktion, für den Namen, mit dem die Sinusfunktion bezeichnet wird, oder für eine Folge von Buchstaben.
Letztere bezeichnen wir als \emph{Zeichenfolge}, wenn Zwischenraum nicht zählt, sonst als \emph{Zeichenkette}.
Unter einer \emph{\glsIdx{Formel}}\footnote{%
	Eine \glsIdx{Formel} kann auch mehrdimensional sein, lässt sich aber mittels geeigneter Definitionen immer eindeutig als eine Zeichenfolge schreiben.
} verstehen wir in diesem Dokument stets eine mathematische \glsIdx{Formel}.
Zur Unterscheidung verwenden wir verschiedene Quotierungen und Schriftarten:

\begin{tabular}{llll}
	& \objqt{\sin} & Ein \glsIdx{Objekt}
	& die Sinusfunktion.                                                 \\
	& \symqt{\sin} & Ein Symbol (Zeichen, Buchstabe)
	& für das \glsIdx{Objekt}.                                           \\
	& \forqt{\sin} & Eine \glsIdx{Formel} (Zeichenfolge)
	& bestehend aus dem einem Symbol \symqt{\sin}                        \\
	& \forqt {sin} & Eine \glsIdx{Formel} (Zeichenfolge)
	& bestehend aus den drei Symbolen \symqt{s}, \symqt{i} und \symqt{n} \\
	& \strqt {sin} & Zeichenkette
	& bestehend aus den drei Zeichen \symqt{\text{s}}, \symqt{\text{i}} und \symqt{\text{n}}
\end{tabular}

Die Bezeichnung eines \glsIdx{Objekt}s kann auch aus mehreren Symbolen bestehen, \textdh\ einer Zeichenfolge oder sogar einer ganzen \glsIdx{Formel}; \textzB\ ist die Bezeichnung für das indizierte \glsIdx{Objekt} \objqt{a_i} gleich \forqt{a_i}.
Die Bezeichnung für die Sinusfunktion hingegen wird als einzelnes Symbol behandelt.

Man beachte, dass in \glsIdxPl{Formel} Worte in normaler (lateinischer) Schrift jeweils genau ein Symbol repräsentieren sollen.
Die kursive Schrift ist Variablen vorbehalten und ein kursiv geschriebener Buchstabe ist immer ein einzelnes Symbol.

Auch Symbole, \glsIdxPl{Formel}, Mengen, Zeichenfolgen, Zahlen, \textusw\ werden wir als \glsIdxPl{Objekt} betrachten, \glsIdxPl{Aussage} \textiAlg\ jedoch nicht.

\subsection{Binäre Relationen und Operatoren}% ---------------------------------
\label{sub:binaer}

Seien \symqt{\glsSym{lrelbsp}}, \symqt{\glsSym{rrelbsp}}, \symqt{\glsSym{relbsp}}, \symqt{\glsSym{releqbsp}} und \symqt{\glsSym{relnbsp}} Beispielsymbole für Relationen.\footnote{%
	Die Relationen müssen weder Ordnungen noch Äquivalenzrelationen sein, auch wenn die angegebenen Symbole das nahe legen.%
}
Wenn nichts anderes gesagt wird, gelte stets:
%
\begin{align}
	& (A \lrelbsp   B) & \metadefeq & \quad  (B \rrelbsp   A)
	\label{eq:lrrelbsp}   \\
	& (A \relnbsp  B)  & \metadefeq & \quad [(A \relbsp   B) \text{ gilt nicht}]
	\label{eq:relnbsp}    \\
	& (A \releqbsp  B) & \metadefeq & \quad ((A \relbsp   B) \metaor  (A \eq B))
	\label{eq:releqbsp}   \\
	& (A \relbsp B)    & \metadefeq & \quad ((A \releqbsp B) \metaand (A \ne B))
	\label{eq:relbsp}    \formulatoleft\formulatoleft\formulatoleft
\end{align}
%
In \eqref{eq:lrrelbsp} lassen sich \symqt{\lrelbsp} und \symqt{\lrelbsp} auch vertauschen.
Man beachte, dass weder \eqref{eq:releqbsp} aus \eqref{eq:relbsp} folgt noch umgekehrt.
Beispiele dazu sind \vrefintab{tab:Gegenbeispiel} angegeben.

\begin{table}[!htb]
	\setlength\extrarowheight{1.5pt}
	\begin{center}
		\begin{tabularx}{8,6cm}{|@{\extracolsep{\fill}}c|cccc|l|}
			\hline
			~           &$a,\;       a$&$a,\;       b$&$b,\;a$&$b,\;       b$&\\
			\hline
			~$\eq      $&$a=         a$&              &       &$b=         b$&\\
			\hline
			~$\releqbsp$&$a\releqbsp a$&$a\releqbsp b$&       &$b\releqbsp b$&
			\text{Es gilt \eqref{eq:releqbsp}}                                \\
			~$\relbsp  $&              &$a\relbsp   b$&       &              &
			\text{und \eqref{eq:relbsp}}                                      \\
			\hline
			~$\releqbsp$&$a\releqbsp a$&$a\releqbsp b$&       &$b\releqbsp b$&
			\text{Es gilt \eqref{eq:releqbsp}}                                \\
			~$\relbsp  $&              &$a\relbsp   b$&       &$b\relbsp   b$&
			\text{aber nicht \eqref{eq:relbsp}}                               \\
			\hline
			~$\releqbsp$&$a\releqbsp a$&$a\releqbsp b$&       &              &
			\text{Es gilt \eqref{eq:relbsp}}                                  \\
			~$\relbsp  $&              &$a\relbsp   b$&       &              &
			\text{aber nicht \eqref{eq:releqbsp}}                             \\
			\hline
		\end{tabularx}
		\caption{Beispiele}
		\label{tab:Gegenbeispiel}% Erst nach '\caption'!
	\end{center}
\end{table}

Als Beispielsymbol für binäre Operatoren wird \symqt{\opbsp} verwendet.
Mit \symqt{\opbsp} zusammenhängende Verabredungen werden hier nicht getroffen.

Es sei noch angemerkt, dass wegen \eqref{eq:lrrelbsp} die Definition von \symqt{\metarep} \vrefinsub{sub:Aussagen} überflüssig ist.
Für den Fall fehlender Klammern sind die Prioritäten \vrefintab{tab:Prio-Metasprache} angegeben.
Damit sind auch alle Klammern \indsub{sub:binaer} überflüssig.

\section{Metasprache}% =========================================================
\beginsection{Metasprache}
\label{sec:Metasprache}

Wenn man über eine Sprache spricht, braucht man eine zweite Sprache, mit \emph{\glsIdx{Metasprache}} bezeichnet, in der \glsIdxPl{Aussage} über die erstere getroffen werden können.\footnote{%
	Die beiden Sprachen können auch übereinstimmen, \textzB\ wenn man über die natürliche Sprache spricht.
}
Wenn die zuerst genannte Sprache die der Mathematik ist, wählt man üblicherweise die natürliche Sprache als Metasprache.
Leider ist diese oft ungenau, nicht immer eindeutig und abhängig vom Zusammenhang, in dem sie gesprochen wird.%
\footnote{%
	Man betrachte die beiden \glsIdxPl{Aussage} \enquote{Studenten und Rentner zahlen die Hälfte.} und \enquote{Studenten oder Rentner zahlen die Hälfte.}, die beide das gleiche meinen.
	-- Entnommen aus \cite{bib:Rautenberg} \sectionname~1.2 Bemerkung 1.

	Ein weiteres Problem ist, dass man unauflösbare Widersprüche formulieren kann, \textzB\ \enquote{Der Barbier ist der Mann im Ort, der genau die Männer im Ort rasiert, die sich nicht selbst rasieren.}.
	Und der Barbier?
	Wenn er sich selbst rasiert, dann rasiert er sich nicht selbst, und wenn er sich nicht selbst rasiert, dann rasiert er sich selbst.
	Was denn nun?
	-- Quelle unbekannt) --
	Das Problem ist verwandt mit dem Problem der \enquote{Menge aller Mengen, die sich nicht selbst enthalten}.%
}
Um diese Probleme in den Griff zu bekommen, kann die Metasprache teilweise formalisiert werden.
Durch diese Formalisierung erinnert sie dann schon an mathematische \glsIdxPl{Formel}.
Die Sprachebenen sollten aber sorgfältig unterschieden werden.

\subsection{Aussagen und Metaoperatoren}% --------------------------------------
\beginsection{Aussagen}
\label{sub:Aussagen}

Beispiele für \glsIdxPl{Aussage} in Metasprache sind
(a) \enquote{Morgen scheint die Sonne.},
(b) \enquote{Ich bin 1,83\,m groß.},
(c) \enquote{Ich habe ein rotes Auto und das kann 200\,km/h schnell fahren.}, \textusw\
Wie Beispiel (c) zeigt, kann eine \glsIdx{Aussage} auch aus anderen \glsIdxPl{Aussage} zusammengesetzt sein.
In diesem Fall bezeichnen wir sie als \emph{\idx{zerlegbar}}, ansonsten als \emph{\idx{unzerlegbar}}.
-- Wir betrachten auch Relationen einschließlich ihrer Operanden als \glsIdxPl{Aussage}.%
\footnote{%
	Wird statt des Symbols der Name der zugehörigen Relation verwendet, ist dies unmittelbar einleuchtend.
	So wird \textzB\ aus der \glsIdx{Formel} \forqt{A<B} die \glsIdx{Aussage} \enquote{\objqt{A} ist kleiner als \objqt{B}}.%
}

Während die Beispiele (a) und (b) unzerlegbare \glsIdxPl{Aussage} sind, ist Beispiel (c) zerlegbar.
Für alle drei \glsIdxPl{Aussage} lässt sich feststellen, ob sie richtig sind oder nicht;
für (a) allerdings nur im Nachhinein und für den zweiten Teil von (c) nur weil klar ist, worauf sich \enquote{das} bezieht.
Natürlich muss auch der Zusammenhang, in dem eine \glsIdx{Aussage} formuliert wird, bekannt sein.
\textZB\ ist die Bedeutung von \enquote{Ich} nur dann bekannt, wenn man weiss, von wem die \glsIdx{Aussage} ist.
Auf eine exakte Definition von \glsIdx{Aussage} wird verzichtet, weil das intuitive Verständnis hier ausreicht.

Zerlegbare \glsIdxPl{Aussage} wie (c) können zum Teil formalisiert werden.
Dies wird mit den folgenden Definitionen erreicht:
\begin{align}
	%
	& A \glsSym{metaimp}   B & \text{steht für }
	& \text{\enquote{\emph{Wenn} \objqt{A} [gilt] \emph{dann} [gilt] [auch] \objqt{B}}.}
	\\
	& A \glsSym{metarep}   B & \text{steht für }
	& \text{\enquote{\objqt{A} [gilt] \emph{sofern}          \objqt{B} [gilt]}.}
	\\
	& A \glsSym{metaequiv} B & \text{steht für }
	& \text{\enquote{\objqt{A} [gilt] \emph{genau dann wenn} \objqt{B} [gilt]}.}
	\\
	& A \glsSym{metaand}   B & \text{steht für }
	& \text{\enquote{\objqt{A} \emph{und}  \objqt{B}}.}
	\\
	& A \glsSym{metaor}    B & \text{steht für }
	& \text{\enquote{\objqt{A} \emph{oder} \objqt{B}}.}
%%%	\\
%%%	& A \glsSym{srand}     B & \text{steht für }
%%%	& \text{\enquote{\objqt{A} \emph{und}  \objqt{B}}.}
	\formulatoleft
\end{align}
%%%\footnote{%
%%%	Insbesondere bei den \glsIdxPl{Schlussregel} (\vrefseesec{sec:Schlussregeln}) ist es üblich, das Symbol \symqt{\srand} statt \symqt{\metaandsym} (bei niedrigerer Priorität (\vrefseetab{tab:Prio-Aussagenlogik})) zu verwenden.%
%%%}
Offensichtlich sind das alles ebenfalls \glsIdxPl{Aussage}, jetzt aber teilweise formalisiert.
(c) lässt sich dann ausdrücken als \enquote{\enquote{Ich habe ein rotes Auto} $\metaandsym$ \enquote{das kann 200\,km/h schnell fahren.}}.
\enquote{$A \metarep B$} ist nur eine andere Schreibweise für \enquote{$B \metaimp A$}.
-- Ein Symbol für \enquote{nicht} wird hier nicht gebraucht.

\glsIdxPl{Aussage} können auch geklammert werden, um die Reihenfolge der Auswertung eindeutig zu machen.
\objqt{\metaimp}, \objqt{\metarep}, \objqt{\metaequiv}, \objqt{\metaandsym} und \objqt{\metaorsym} heißen \emph{\glsIdxPl{Metaoperator}}. % und \objqt{\srand}
Für den Fall fehlender Klammern sind ihre Prioritäten \vrefintab{tab:Prio-Metasprache} angegeben.

Um Verwechslungen mit den logischen Operatoren zu vermeiden, verwenden wir für die metasprachlichen Operatoren \enquote{und} und \enquote{oder} die Symbole \symqt{\metaandsym} und \symqt{\metaorsym}.
\objqt{A} und \objqt{B} können als Operanden von \objqt{\metaequiv}, \objqt{\metaandsym} und \objqt{\metaorsym} vertauscht werden, ohne das Ergebnis zu ändern.\footnote{%
	\textDh\ die Operatoren \objqt{\metaequiv}, \objqt{\metaandsym} und \objqt{\metaorsym} sind \emph{kommutativ}.%
} %%% und \objqt{\srand}
Wird in einer (Teil"~)\glsIdx{Aussage} nur einer der Operatoren \objqt{\metaandsym} oder \objqt{\metaorsym} verwendet, können die Klammern dort weggelassen und die Operationen in beliebiger Reihenfolge ausgewertet werden, wiederum ohne das Ergebnis zu ändern.\footnote{%
	\textDh\ die Operatoren \objqt{\metaandsym} und \objqt{\metaorsym} sind \emph{assozativ}.%
} %%% oder \objqt{\srand}
Zusammengefasst ist die Reihenfolge der Operatoren und der Auswertung dort beliebig.

\subsection{Mit Gleichheit verwandte Relationen}% ------------------------------
\label{sub:Gleichheit}

\subsubsection{Vergleichbar}%- - - - - - - - - - - - - - - - - - - - - - - - - -
\label{subsub:Vergleichbar}

Zwei \glsIdxPl{Objekt} \objqt{A} und \objqt{B} sind \emph{\glsIdx{vergleichbar}}, wenn beide von derselben Art sind, \textdh\ wenn \textzB\ jeweils beide Mengen, Zeichenfolgen, Zahlen, \textusw\ sind.
Dabei muss bei \glsIdxPl{Formel} zwischen der \glsIdx{Formel} an sich und dem Ergebnis der \glsIdx{Formel} unterschieden werden. Siehe Beispiel (a).

Intuitiv scheint klar zu sein, was damit  gemeint ist.
Wenn aber entschieden werden muss, ob \textzB\ (a) \enquote{1+1} gleich \enquote{2} oder (b) \enquote{1+1} gleich \enquote{1 + 1} ist, muss man erst entscheiden, von welcher Art die beiden zu vergleichenden Ausdrücke sind, \textdh\ \emph{wie} verglichen wird.
Wenn sie als jeweiliges Ergebnis der beiden \glsIdxPl{Formel} verglichen werden, dann ist (a) richtig.
Wenn sie als \glsIdxPl{Formel}, \textdh\ als Zeichenfolgen, verglichen werden, ist (a) falsch.
Wenn die Ausdrücke in (b) als Zeichenfolgen verglichen werden, dann ist (b) richtig.
Wenn sie als Zeichenketten\footnote{%
	In Zeichenketten zählen Leerstellen, in Zeichenfolgen nicht.%
} verglichen werden, ist (b) falsch.

Bei mathematischen \glsIdxPl{Formel} sind Zwischenräume \textiAlg\ ohne Bedeutung.
Daher berücksichtigen wir beim Vergleich zweier \glsIdxPl{Formel} keine Zwischenräume, \textdh\ sie gelten als Zeichenfolgen.
Die Zwischenräume dienen dann nur der besseren Lesbarkeit.
Bei den Zeichenketten werden Leerstellen berücksichtigt; mehrere hintereinander sind dabei auch mehrere Zeichen.

Die folgende Tabelle fasst dass zusammen:

\begin{center}
	\begin{tabular}{|c|c|c|}
		\hline
		\objqt{A}   & \objqt{B}       & \objqt{A} gleich \objqt{B} \\
		\hline
		\objqt{1+1} & \objqt{2}       & richtig                    \\
		\forqt{1+1} & \forqt{2}       & falsch                     \\
		\forqt{1+1} & \forqt{1\;+\;1} & richtig                    \\
		\strqt{1+1} & \strqt{1 + 1}   & falsch                     \\
		\hline
	\end{tabular}
\end{center}

\subsubsection{Vergleiche}%- - - - - - - - - - - - - - - - - - - - - - - - - - -
\label{subsub:Vergleiche}

\objqt{A} und \objqt{B} seien \glsIdxPl{Objekt}\vrefnotesub{sub:Quotierung}.
Dann definieren wir:

\begin{description}
	%
	\item[\objqt{\glsSym{eq}}~~\emph{\Idx{Gleichheit}}]\label{def:Gleichheit}
	\forqt{A = B} heißt, dass \objqt{A} und \objqt{B} sich in den \glsIdxPl{intEigenschaftA} für \objqt{=} nicht unterscheiden.%
	\footnote{%
		\textZB\ sind zwei logische Operatoren üblicherweise dann gleich, wenn sie stets denselben \emph{\glsIdx{Wahrheitswert}} liefern.
		Ihre Bezeichnungen oder Symbole können dabei durchaus verschieden sein, interessieren bei der Feststellung der Gleichheit aber nicht.
		Andernfalls wären sie nicht gleich.
	}
	-- \enquote{\objqt{A} ist \emph{dasselbe} wie \objqt{B}} oder \enquote{\objqt{A} ist \emph{identisch} zu \objqt{B}}
	-- Inwieweit die Begriffe \emph{Gleichheit} und \emph{Identität} korrelieren, wird hier nicht erörtert. (siehe~\cite{bib:Identitaet})

	Gleichheit ist eine Äquivalenzrelation.%
	\footnote{%
		Eine Relation \objqt{\sim} ist eine \emph{Äquivalenzrelation}, wenn sie \emph{reflexiv} ($A \sim A$), \emph{transitiv} ($((A \sim B) \metaand (B \sim C)) \metaimp (A \sim C)$) und \emph{symmetrisch} ($(A \sim B) \metaimp (B \sim A)$) ist -- jeweils für alle \objqt{A}, \objqt{B} und \objqt{C}.%
	}
	%
	\item[\objqt{\glsSym{ne}}~~\emph{\Idx{Ungleichheit}}]\label{def:Ungleichheit}
	\forqt{A \ne B} heißt, dass \objqt{A} und \objqt{B} sich in mindestens einer der \glsIdxPl{intEigenschaftA} für \objqt{=} unterscheiden. \enquote{\objqt{A} ist \emph{nicht dasselbe} wie \objqt{B}} (aber vielleicht das gleiche; siehe \objqt{\equiv}) oder \enquote{\objqt{A} ist \emph{nicht identisch} zu \objqt{B}}.
	%
	\item[\objqt{\glsSym{equiv}}~~\emph{\Idx{Äquivalenz}}(relation)]\label{def:Äquivalenz}
	\forqt{A \equiv B} heißt, dass \objqt{A} und \objqt{B} sich in den \glsIdxPl{intEigenschaftA} für \objqt{\equiv} nicht unterscheiden.
	-- \enquote{\objqt{A} ist \emph{das gleiche} wie \objqt{B}} (aber nicht unbedingt dasselbe; siehe \objqt{\eq}) oder \enquote{\objqt{A} ist \emph{so wie} \objqt{B}}.

	Es kann auch mehrere Äquivalenzrelationen geben, für die dann verschiedene Bezeichnungen verwendet werden.
	%
	\item[\objqt{\glsSym{nequiv}}~~\emph{\Idx{Kontravalenz}}]\label{def:Kontravalenz}
	\forqt{A \nequiv B} heißt, dass \objqt{A} und \objqt{B} sich in mindestens einer der \glsIdxPl{intEigenschaftA} für \objqt{\nequiv} unterscheiden.
	-- \enquote{\objqt{A} ist \emph{nicht das gleiche} wie \objqt{B}} oder \enquote{\objqt{A} ist \emph{nicht so wie} \objqt{B}}.
	%
\end{description}

\objqt{=}, \objqt{\ne}, \objqt{\equiv} und \objqt{\nequiv} bezeichnen wir als  \emph{\glsIdxPl{Vergleichsoperator}}.

Jede \glsIdx{intEigenschaftA} für \objqt{\equiv} oder eine andere Äquivalenzrelation muss auch eine für \objqt{\eq} sein.
Daraus folgt insbesondere, dass mit \forqt{(A \eq B)} auch \forqt{(A \equiv B)} und mit \forqt{(A \nequiv B)} auch \forqt{(A \ne B)} gilt.

\subsubsection{Definitionen}%- - - - - - - - - - - - - - - - - - - - - - - - - -
\label{subsub:Definitionen}

{
	\newcommand{\A}{\overline{A}}
	\newcommand{\B}{\overline{B}}
	Seien \objqt{\A} und \objqt{\B} \glsIdxPl{Aussage} und \objqt{A} und \objqt{B} \glsIdxPl{Objekt}\footnote{%
		Die Anforderungen an die \glsIdx{Aussage} \objqt{\A} und das \glsIdx{Objekt} \objqt{A} sind intuitiv klar.
		Insbesondere darf \objqt{\B} \textbzw\ \objqt{B} nicht von dem bisher undefinierten Teil von \objqt{\A} \textbzw\ \objqt{A} abhängig sein.%
	}.
	\begin{description}
		%
		\item[\objqt{\glsSym{metadefeq}}~\emph{\Idx{Metadefinition}}]\label{def:Metadefinition}
		\forqt{\A \metadefeq \B} heißt, dass die \glsIdx{Aussage} \objqt{\A} \emph{definitionsgemäß gleich} der \glsIdx{Aussage} \objqt{\B} ist.
		Gewissermaßen ist \objqt{\A} nur eine andere Schreibweise für \objqt{\B}.
		\enquote{\objqt{\A} \emph{steht für} \objqt{\B}}.
		\objqt{\A} und \objqt{\B} können sich gegenseitig ersetzten.
		%
		\item[\objqt{\glsSym{defeq}}~\emph{\Idx{Definition}}]\label{def:Definition}
		\forqt{A \defeq B} heißt, dass das Objekt \objqt{A} \emph{definitionsgemäß gleich} dem Objekt \objqt{B} ist.
		Gewissermaßen ist \objqt{A} nur eine andere Schreibweise für \objqt{B}.
		\enquote{\objqt{A} \emph{steht für} \objqt{B}}.
		\objqt{A} und \objqt{B} können sich gegenseitig ersetzten.%
		\footnote{%
			Nach den Definitionen von \objqt{\metadefeq} und \objqt{\defeq} sind zwei Ausdrücke \objqt{P} und \objqt{Q} schon dann gleich, wenn nach der Ersetzung aller Vorkommen von \objqt{A} durch \objqt{B} sowohl in \objqt{P} als auch in \objqt{Q} die resultierenden Ausdrücke $\overline{P}$ und $\overline{Q}$ gleich sind.%
		}

	\end{description}
	Man beachte, dass \objqt{\metadefeq} und \objqt{\defeq} verschiedene Sprachebenen sind.
}

\subsubsection{Prioritäten}% - - - - - - - - - - - - - - - - - - - - - - - - - -
\label{subsub:Prioritaeten}

In \vrefintab{tab:Prio-Metasprache} sind für fehlende Klammern die Prioritäten der bisher behandelten Elemente angegeben. Wenn von diesen Prioritäten abgewichen wird, wird dies eigens erwähnt.

\begin{table}[!htb]
	\setlength\extrarowheight{1.5pt}
	\begin{center}
		\begin{threeparttable}
			\begin{tabularx}{11.5cm}{|@{~~}l|@{\extracolsep{\fill}}l|}
				\hline
				Klammern & $ ( \quad ) $ \quad \objqt{\quad} \quad \symqt{\quad} \quad \forqt{\quad} \\
				\hline
				Binäre Operatoren \tnote{1} & $ \opbsp $ \\
				\hline
				Binäre Relationen \tnote{1}
				& $\lrelbsp\quad\rrelbsp\quad\lreleqbsp\quad\rreleqbsp$ \\
				& $\relbsp\quad\releqbsp\quad\relnbsp$ \\
				\hline
				\glsIdxPl{Vergleichsoperator} \tnote{2}
				& $ \eq \quad \ne \quad \equiv \quad \nequiv $ \\
				\hline
				Definitionen \tnote{3} & $ \defeq $ \\
				\hline
				& $ \metaandsym             $ \\
				\GlsIdxPl{Metaoperator} \tnote{4}
				& $ \metaorsym              $ \\
				& $ \metarep \quad \metaimp $ \\
				& $ \metaequiv              $ \\
%%%				& $ \srand                  $ \\
				\hline
				Metadefinition \tnote{3} & $ \metadefeq $ \\
				\hline
				\parbox[][1.1cm][c]{6.5cm}{%
					Innerhalb natürlicher Sprache deren Strukturelemente, \textzB\ Satzzeichen\tnote{5}%
				}
				& . \quad , \quad ; \quad \textusw \\
				\hline
			\end{tabularx}
			\begin{tablenotes}
				\footnotesize
				\item[1] \vrefseesub{sub:binaer}
				\item[2] \vrefseesubsub{subsub:Vergleiche}
				\item[3] \vrefseesubsub{subsub:Definitionen}
				\item[4] \vrefseesub{sub:Aussagen}
				\item[5] Innerhalb von \glsIdxPl{Formel} können Satzzeichen eine andere Bedeutung und Priorität haben.
			\end{tablenotes}
		\end{threeparttable}
		\caption{Prioritäten in abnehmender Reihenfolge}
		\label{tab:Prio-Metasprache}% Erst nach '\caption'!
	\end{center}
\end{table}

\section{Beweise in ASBA}% =====================================================
\beginsection{Beweise in \ASBA}
\label{sec:BeweiseASBA}

Die Regeln zur Formulierung und Prüfung der \glsIdxPl{Beweis} müssen in \ASBA\ fest codiert werden.
Sie sind quasi die \glsIdxPl{Axiom} von \ASBA\ und sollten daher möglichst wenig voraussetzen.
Dazu wird ein \emph{Genzen-Kalkül}\footnote{%
	siehe~\cite{bib:Rautenberg} Kapitel~1.4 und vergleiche~\cite{bib:Schlussregel,bib:NatuerlichesSchliessen}%
} verwendet.

Ein \glsIdx{Beweis}\footnote{%
	siehe~\cite{bib:Rautenberg} Kapitel~1.6 und~3.6%
} in \ASBA\ besteht aus
\begin{align}
	& \text{einer Menge} && \Voraussetzungen = \{\Voraussetzung_n | 0 < n \le N \}
	&& \text{von \emph{Voraussetzungen} } \Voraussetzung_n
	\label{def:Voraussetzungen}
	\\
	& \text{einer Menge} && \Folgerungen = \{\Folgerung_m | 0 < m \le M  \}
	&& \text{von \emph{\glsIdxPl{Folgerung}} } \Folgerung_m
	\label{def:Folgerungen}
	\\
	& \text{einer Folge} && \Beweisschritte = (\Beweisschritt_0, \Beweisschritt_1, ..., \Beweisschritt_K)
	&& \text{von \emph{Beweisschritten} } \Beweisschritt_k
	\label{def:Beweisfolge}
	\formulatoleft
\end{align}
wobei \objqt{N}, \objqt{M} und \objqt{K} Elemente von \objqt{\gsNo}, die \glsIdxPl{Voraussetzung} und \glsIdxPl{Folgerung} \glsIdxPl{Aussage} und die \glsIdxPl{Beweisschritt} \emph{\glsIdxPl{Schlussregel}}\footnote{%
	\glsIdxPl{Schlussregel} werden später behandelt.%TODO Schlussregeln, wo?
} sind. Mit
\begin{align}
	& \Beweisschrittmenge_k & \defeq \quad & \{\Beweisschritt_0, \Beweisschritt_1, ..., \Beweisschritt_k\} & \quad \text{für~~} -1 \le k \le K
	\label{def:Beweisschrittebis} \\
	& \Beweisschrittmenge   & \defeq \quad & \Beweisschrittmenge_K \label{def:Beweisschrittmenge}
	\formulatoleft\formulatoleft\formulatoleft
\end{align}\footnote{\objqt{\Beweisschrittmenge_{-1}} ist die leere Menge \objqt{\emptyset}}
muss jeder \glsIdx{Beweisschritt} \objqt{\Beweisschritt_k \text{ für } 0 \le k \le K} entweder

%TODO Elimination von Voraussetzungen behandeln!
\begin{itemize}
	\item[] eine \glsIdx{Voraussetzung} aus \objqt{\Voraussetzungen} oder
	\item[] ein Ergebnis\footnote{eine weitere zulässige Schlussregel oder eine Formel} der Anwendung einer \emph{zulässigen \glsIdx{Schlussregel}} auf eine Teilmenge von \objqt{\Beweisschrittmenge_{k-1}} sein.
\end{itemize}
sein.
Schließlich muss noch gelten:
\[
	\Folgerungen \subseteq \Beweisschritte
\]

Die damit bewiesene \glsIdx{Aussage} (\textzB\ ein mathematischer \glsIdx{Satz}) kann dann folgendermaßen formuliert werden:%
\footnote{%
	$\srand$ steht für \enquote{und}, bindet aber wesentlich schwächer als $\metaand$.% (\vrefseesub{sub:Aussagen})%
}
\[
	\frac{\Voraussetzungen}{\Folgerungen} \quad \text{\textbzw} \quad
	\frac{\Voraussetzung_1 \srand \Voraussetzung_2 \srand ... \srand \Voraussetzung_n}{\Folgerung_1 \srand \Folgerung_2 \srand ... \srand \Folgerung_m}
	\qquad \text{(\glsIdx{formalerSatzV})}
	\tag{\tagFS} \sym{\gls{FS}}
	\label{def:formalerSatz}
\]
Ein \glsIdx{formalerSatzV} ist gleichzeitig eine Schlussregel.
Bevor die \glsIdxPl{Schlussregel} behandelt werden, werden noch Elemente der \emph{\glsIdx{Aussagenlogik}} und der \emph{\glsIdx{Praedikatenlogik}} behandelt.

\section{Aussagenlogik}% =======================================================
\beginsection{Aussagenlogik}
\label{sec:Aussagenlogik}
\hidden{\glsIdx{Aussagenlogik}}

%%%Zunächst wird dazu eine Kurzfassung der \emph{\glsIdx{Aussagenlogik}} und der \emph{\glsIdx{Prädikatenlogik}} - weitgehend nach~\cite{bib:Rautenberg} - angegeben.
%%%Dabei werden Bezeichnungen und \glsIdxPl{Satz} teilweise wörtlich aus~\cite{bib:Rautenberg} \textiAlg\ übernommen.
%%%Die beiden folgenden \sectionnames\ sind aber \emph{kein} Ersatz für das Studium von~\cite{bib:Rautenberg}!

\subsection{Konstante und Operatoren}% -----------------------------------------
\label{sub:Operatoren}

%TODO *** hier weitermachen
\begingroup% Ab hier vorübergehende Neudefinition der Quotierung von Objekten%%%
\renewcommand*{\objqt}[1]{<$#1$>}%%%

\vrefDtab{tab:Symbole}%
\footnote{%
	Die \tablename\ basiert auf den Wahrheitstafeln in~\cite{bib:Junktor} Kapitel~2.2 und~\cite{bib:Rautenberg} Kapitel~1.1 Seite~3.%
}
definiert für die zweiwertige Logik Konstanten- und Operatorsymbole über die \glsIdxPl{Wahrheitswert} ihrer Anwendung.
So ergeben sich, abhängig von den \glsIdxPl{Wahrheitswert}n der Operanden \objqt{A} und \objqt{B}%
\footnote{%
	\objqt{A} und \objqt{B} können hier beliebige \glsIdxPl{Aussage} sein -- auch \glsIdxPl{Formel} --, die jeweils genau einen \glsIdx{Wahrheitswert} repräsentieren.%
},
die in der \tablename\ angegebenen \glsIdxPl{Wahrheitswert} für die Operationen.
Die mit 0, 1 und 2 benannten Spalten werden jeweils nur für die 0-, 1- und 2-stelligen Operatoren, \textdh\ für die Konstanten, die unären und die binären Operatoren ausgefüllt.
Dabei werden die Konstanten als 0-stellige Operatoren angesehen.
Hat der Inhalt einer Zelle keine Relevanz, steht dort ein Minuszeichen, ist kein Wert bekannt, so bleibt sie leer.
%
% ==============================================================================
% Definitionen für die folgende Tabelle; siehe auch die Symbole im Vorspann
% Prioritäten - jeweils Zähler p* für das Symbol \l* und \thep* für den Wert.
\newcounter{prio}    \setcounter{prio}    {1}
\newcounter{pnot}    \setcounter{pnot}    {\value{prio}}
\stepcounter{prio}% - - - - - - - - - - - - - - - - - -
\newcounter{pand}    \setcounter{pand}    {\value{prio}}
\newcounter{pnand}   \setcounter{pnand}   {\value{prio}}
\newcounter{pmult}   \setcounter{pmult}   {\value{prio}}
\stepcounter{prio}% - - - - - - - - - - - - - - - - - -
\newcounter{por}     \setcounter{por}     {\value{prio}}
\newcounter{pnor}    \setcounter{pnor}    {\value{prio}}
\newcounter{pxor}    \setcounter{pxor}    {\value{prio}}
%\newcounter{pnxor}   \setcounter{pnxor}   {\value{prio}}
\newcounter{padd}    \setcounter{padd}    {\value{prio}}
\stepcounter{prio}% - - - - - - - - - - - - - - - - - -
%\newcounter{pright}  \setcounter{pright}  {\value{prio}}
%\newcounter{pnright} \setcounter{pnright} {\value{prio}}
%\newcounter{pleft}   \setcounter{pleft}   {\value{prio}}
%\newcounter{pnleft}  \setcounter{pnleft}  {\value{prio}}
%\stepcounter{prio}% - - - - - - - - - - - - - - - - - -
\newcounter{pimp}    \setcounter{pimp}    {\value{prio}}
\newcounter{pnimp}   \setcounter{pnimp}   {\value{prio}}
\newcounter{prep}    \setcounter{prep}    {\value{prio}}
\newcounter{pnrep}   \setcounter{pnrep}   {\value{prio}}
\stepcounter{prio}% - - - - - - - - - - - - - - - - - -
\newcounter{pequiv}  \setcounter{pequiv}  {\value{prio}}
\newcounter{pnequiv} \setcounter{pnequiv} {\value{prio}}
% Farben
\definecolor{cNormalUse}{rgb}{.80,.80,.80}
\definecolor{cRareUse}{rgb}{.90,.90,.99}
% ==============================================================================

\begin{table}
	\newcommand*{\tablegroup}{\hdashline[6pt/3pt]}
	\newcommand*{\tableline}{\hdashline[3pt/3pt]}
	\newcommand*{\gapline}{\cdashline{1-1}[1pt/3pt]\cdashline{9-11}[1pt/3pt]}
	\setlength\tabcolsep{3pt}
	\setlength\extrarowheight{1.5pt}
	\begin{threeparttable}
		\begin{tabularx}{\linewidth-10.95pt}{c||c:cc:cccc|X:X|c|}

			A & - & \texttrue & \textfalse &%
			\texttrue  & \texttrue  & \textfalse & \textfalse &
			- & \glsIdx{Aussage} A & - \\

			\tableline%.................................................
			B & - & -       & -        &%
			\texttrue  & \textfalse & \texttrue  & \textfalse &
			- & \glsIdx{Aussage} B & - \\

			\hline% -- Überschrift -----------------------------------------

			\textbf{Junktor}\tnote{1}&\textbf{0}&\multicolumn{2}{c:}{%
				\textbf{1}}&\multicolumn{4}{c|}{\textbf{2}}& \textbf{%
				Name}&\textbf{Sprechweise}\tnote{2}&\textbf{Prio}\tnote{3} \\%Kein Fehler!
			\hline\hline% == Konstante =====================================

			\rowcolor{cRareUse}
			$\glsSym{ltrue}$
			& \texttrue  & - & - & - & - & - & - & Verum  & Wahr   & - \\

			\tableline%.................................................

			\rowcolor{cRareUse}

			$\glsSym{lfalse}$
			& \textfalse & - & - & - & - & - & - & Falsum & Falsch & - \\

			\hline% -- unäre Operatoren ------------------------------------

			& - & \texttrue  & \texttrue  & - & - & - & -
			&                     &                  & -                 \\

			\tableline%.................................................

			\rowcolor{cNormalUse}

			$\Sym{(\dots)}$
			& - & \texttrue  & \textfalse & - & - & - & -
			& Klammerung          & A ist geklammert & -\tnote{4} \\

			\tableline%.................................................

			\rowcolor{cNormalUse}
			$\Sym{\lnot}$
			& - & \textfalse & \texttrue  & - & - & - & -
			& Negation            & Nicht A          & \thepnot\tnote{5} \\

			\tableline%.................................................

			& - & \textfalse & \textfalse & - & - & - & -
			&                     &                  & -                 \\

			\hline% -- binäre Operatoren -----------------------------------

			~ & - & - & - &\texttrue&\texttrue&\texttrue&\texttrue
			& Tautologie
			&
			& - \\

			\tableline%.................................................

			\rowcolor{cNormalUse}

			$\Sym{\lor}$
			& - & - & - &\texttrue&\texttrue&\texttrue&\textfalse
			& Disjunktion; Adjunktion;\newline Alternative
			& A oder B
			& \thepor \\

			\tableline%.................................................

			\rowcolor{cRareUse}
			$\Sym{\lrep}$ $\lrepA$ $\lrepB$
			& - & - & - &\texttrue&\texttrue&\textfalse&\texttrue
			& Replikation; Konversion;\newline konverse Implikation
			& A folgt aus B
			& \theprep \\

			\tableline%.................................................

			$\lleft$
			& - & - & - &\texttrue&\texttrue&\textfalse&\textfalse
			& Präpendenz
			& Identität von A
			& - \\

			\tablegroup% -----------------------------------------------

			\rowcolor{cNormalUse}
			$\Sym{\limp}$ $\limpA$ $\limpB$
			& - & - & - &\texttrue&\textfalse&\texttrue&\texttrue
			& Implikation; Subjunktion;\newline Konditional
			& Wenn A so B; Aus A folgt B; A nur dann wenn B
			& \thepimp \\

			\tableline%.................................................

			$\lright$
			& - & - & - &\texttrue&\textfalse&\texttrue&\textfalse
			& Postpendenz
			& Identität von B
			& - \\

			\tableline%.................................................

			\rowcolor{cNormalUse}
			$\Sym{\lequiv}$ $\lequivA$
			& - & - & - &\texttrue&\textfalse&\textfalse&\texttrue
			& Äquivalenz; Bijunktion;\newline Bikonditional
			& A genau dann wenn B; A dann und nur dann wenn B
			& \thepequiv \\

			\tableline%.................................................

			\rowcolor{cNormalUse}
			$\Sym{\land}$ $\landA$ $\landB$
			& - & - & - &\texttrue&\textfalse&\textfalse&\textfalse
			& Konjunktion
			& {\small A und B; Sowohl A als auch B}
			& \thepand \\

			\tablegroup% -----------------------------------------------

			\rowcolor{cRareUse}
			$\Sym{\lnand}$ $\lnandA$ $\lnandB$
			& - & - & - &\textfalse&\texttrue&\texttrue&\texttrue
			& NAND; Unverträglichkeit;\newline Sheffer-Funktion
			& Nicht zugleich A und B
			& \thepnand \\

			\tableline%.................................................

			\rowcolor{cRareUse}
			$\Sym{\lxor}$ $\lxorA$ $\lxorB$ $\lxorC$
			& - & - & - &\textfalse&\texttrue&\texttrue&\textfalse
			& XOR; Antivalenz;\newline ausschließende Disjunktion
			& Entweder A oder B
			& \thepxor \\

			\gapline%. . . . . . . . . . . . . . . . . . . . . . . . . .

			$\lnequiv$ $\lnequivA$ $\lnequivB$
			& - & - & - &"&"&"&"
			& Kontravalenz
			&
			& - \\

			\tableline%.................................................

			$\lnright$
			& - & - & - &\textfalse&\texttrue&\textfalse&\texttrue
			& Postnonpendenz
			& Negation von B
			& - \\

			\tableline%.................................................

			$\lnimp$ $\lnimpA$ $\lnimpB$
			& - & - & - &\textfalse&\texttrue&\textfalse&\textfalse
			& Postsektion
			&
			& - \\

			\tablegroup% -----------------------------------------------

			$\lnleft$
			& - & - & - &\textfalse&\textfalse&\texttrue&\texttrue
			& Pränonpendenz
			& Negation von A
			& - \\

			\tableline%.................................................

			$\lnrep$ $\lnrepA$ $\lnrepB$
			& - & - & - &\textfalse&\textfalse&\texttrue&\textfalse
			& Präsektion
			&
			& - \\

			\tableline%.................................................

			\rowcolor{cRareUse}
			$\Sym{\lnor}$ $\lnorA$
			& - & - & - &\textfalse&\textfalse&\textfalse&\texttrue
			& NOR; Nihilation;\newline Peirce-Funktion
			& Weder A noch B
			& \thepnor \\

			\tableline%.................................................

			~
			& - & - & - &\textfalse&\textfalse&\textfalse&\textfalse
			& Kontradiktion
			&
			& - \\

			\hline%_____________________________________________________________
		\end{tabularx}
		\begin{tablenotes}
			\footnotesize

			\item[1] \emph{Operatorsymbole}.
			Sie stehen meistens für die Operatoren selbst.
			Der Einfachheit halber werden auch die beiden Konstanten \textbzw\ ihre Symbole \symqt{\ltrue} und \symqt{\lfalse} als Operatoren \textbzw\ \emph{Junktoren} bezeichnet.

			Die Operatoren \symqt{\subset}, \symqt{\supset}, \symqt{\nsubset} und \symqt{\nsupset} haben hier nicht die Bedeutung der entsprechenden Operatoren der Mengenlehre und dürfen nicht damit verwechselt werden; entsprechendes gilt für \symqt{+} und \symqt{\cdot} mit Addition und Multiplikation.

			\item[2] Ist eine Zelle in dieser Spalte leer, so ist die zugehörige Zeile nur vorhanden, um alle binären Operatoren aufzuführen.

			\item[3] Je kleiner die Zahl, je höher die Priorität.

			\item[4] Klammerung ist genau genommen keine Operation und wird nicht nur bei logischen, sondern auch bei anderen Ausdrücken verwendet. Ihre Priorität - sofern man überhaupt davon sprechen kann - kann nur höher als die aller (anderen) Operatoren sein.

			\item[5] Die Priorität der unären Operatoren muss höher sein als die aller mehrwertigen, also auch der binären Operatoren.
			Wenn alle unären Operatoren auf derselben Seite des Operanden stehen, brauchen sie eigentlich keine Priorität, da die Auswertung nur von innen (dem Operanden) nach außen erfolgen kann.
			Nur wenn es sowohl links-, als auch rechtsseitige unäre Operatoren gibt, muss man für diese Prioritäten definieren.

		\end{tablenotes}
	\end{threeparttable}
	\caption{Definition von aussagenlogischen Symbolen.}
	\label{tab:Symbole}% Erst nach '\caption'!
\end{table}

Für einige \emph{Junktoren}\footnote{Ein \emph{Junktor} oder \emph{Operatorsymbol} ist ein Symbol, dass für einen bestimmten Operator verwendet wird.}, Namen und Sprechweisen sind auch Alternativen angegeben.
Die durchgestrichenen (\textdh\ negierten) Symbole sind ungebräuchlich und nur aus formalen Gründen aufgeführt.
Wenn für eine bestimmte Kombination von \glsIdxPl{Wahrheitswert}n mehr als eine Zeile angegeben ist, so sind die zugehörigen Operationen in der zweiwertigen \glsIdx{Aussagenlogik} alle gleich.
Bei der formalen Definition wird aber keine Zweiwertigkeit vorausgesetzt, so dass je nach Definition die Operationen verschiedene Ergebnisse liefern können.

Um vollständig zu sein, \textdh\ alle 22 möglichen Kombinationen von \glsIdxPl{Wahrheitswert}n für höchstens zwei Variable zu berücksichtigen, enthält die \tablename\ auch viele ungebräuchliche Junktoren und Operationen.
Die Zeilen mit den Klammern und den gebräuchlichsten Operatoren sind in der \tablename\ grau hinterlegt.
Hellgrau hinterlegt sind Zeilen mit weniger gebräuchlichen Operatoren.
Die restlichen Operatoren sind uninteressant und brauchen daher keine Priorität.

\subsection{Klammerregeln}% ----------------------------------------------------
\label{sub:Klammerregeln}

Zur Klammerersparnis werden die üblichen Regeln verwendet, \textdh\ dass Operatoren mit höherer Priorität stärker binden, als solche mit niedrigerer Priorität.

Für Operatoren derselben Priorität gilt Rechtsklammerung%
\footnote{%
	Unäre Operatoren stehen hier stets links \emph{vor} dem Operanden, so dass es nur Rechtsklammerung geben kann.
	Zur Rechtsklammerung bei binären Operationen ein Zitat aus~\cite{bib:Rautenberg} Kapitel~1.1 Seite~5:
	\enquote{Diese hat gegenüber Linksklammerung Vorteile
		bei der Niederschrift von Tautologien in $\limp$, [...]}%
}.
Im Folgenden wird nur noch ein Teil der logischen Operatoren \vrefaustab{tab:Symbole} und der \glsIdxPl{Metaoperator} \vrefaussub{sub:Aussagen} berücksichtigt.
Diese werden \vrefintab{tab:Prio-Aussagenlogik} mit abnehmender Priorität aufgelistet.

\begin{table}[!htb]
	\setlength\extrarowheight{1.5pt}
	\begin{center}
		\begin{threeparttable}
			\begin{tabularx}{11.4cm}{|@{~~}l|@{\extracolsep{\fill}}l|}
				\hline
				Klammern
				& $ ( \quad ) $ \\
				\hline
				Unäre logische Operatoren
				& $ \lnot                           $ \\
				\hdashline
				& $ \land \quad \lmult \quad \lnand $ \\
				Binäre logische Operatoren
				& $ \lor  \quad \ladd  \quad \lnor  $ \\
				& $ \lrep \quad \limp               $ \\
				& $ \lequiv                         $ \\
				\hline
				\parbox[][1.5cm][c]{8.0cm}{%
					Mit Gleichheit verwandte Operatoren;\newline
					\small ihre Prioritäten untereinander sind nicht eindeutig und bleiben daher undefiniert.%
				}
				& $ \eq \quad \ne \quad \equiv \quad \nequiv $ \\
				\hline
				\glsIdx{Ableitungsrelation}\tnote{1}
				& $ \derive          $ \\
				\glsIdx{Substitution}\tnote{1}
				& $ \subst           $ \\
				Definition
				& $ \defeq           $ \\
				\hline
				& $ \metaandsym                  $ \\
				\GlsIdxPl{Metaoperator}\tnote{2}
				& $ \metaorsym                   $ \\
				& $ \metarep \quad \metaimp      $ \\
				& $ \metaequiv                   $ \\
%%%				& $ \srand                       $ \\
				\hline
				Metadefinition & $ \metadefeq $ \\
				\hline
				\parbox[][1.1cm][c]{8.0cm}{%
					Innerhalb natürlicher Sprache deren Strukturelemente, \textzB\ Satzzeichen\tnote{3}%
				}
				& . \quad , \quad ; \quad \textusw \\
				\hline
			\end{tabularx}
			\begin{tablenotes}
				\footnotesize
				\item[1] \vrefseesub{sub:Basisregeln}
				\item[2] \vrefseesub{sub:Aussagen}
				\item[3] Innerhalb von \glsIdxPl{Formel} können Satzzeichen eine andere Bedeutung und Priorität haben.
				% Nummer 3 wird in "Basisregeln" verwendet
			\end{tablenotes}
		\end{threeparttable}
		\caption{Prioritäten von Metaoperatoren in abnehmender Reihenfolge}
		\label{tab:Prio-Aussagenlogik}% Erst nach '\caption'!
	\end{center}
\end{table}

Die Prioritäten der logischen Operatoren wurden aus~\cite{bib:Rautenberg} Kapitel~1.1 Seite~5 entnommen und ergänzt und die der \glsIdxPl{Metaoperator} daran angeglichen.
Wie üblich bindet ein Operator \emph{stärker} als jeder andere mit einer niedrigeren Priorität und \emph{schwächer} als jeder andere mit höherer Priorität.

\subsection{Formalisierung}% ---------------------------------------------------
\label{sub:Formalisierung}

Da sie die Grundlage -- quasi das Fundament -- des mathematischen Inhalts von \ASBA\ sind, müssen die \glsIdxPl{Axiom}, \glsIdxPl{Satz}, \glsIdxPl{Beweis}, \textusw\ der \glsIdx{Aussagenlogik} (und später der \glsIdx{Praedikatenlogik}) in streng formaler Form vorliegen.
Die Formalisierung stützt sich auf~\cite{bib:Aussagenlogik}; \alsoname~\cite{bib:LogikDe, bib:LogikEn}.
Da Computerprogramme mit der \emph{Polnischen Notation}\idx{Polnische Notation}%
\footnote{%
	Bei der \emph{Polnischen Notation} wird eine zweistellige Operation $(A\opbsp B)$ dargestellt als $\opbsp A B$.
	Eine Zwischenstufe ist $\opbsp(A,B)$, bei der noch die redundanten Gliederungszeichen Komma und Klammern -- auch andere als die runden -- hinzukommen, so dass die Operationen optisch besser getrennt und dadurch für Menschen besser lesbar werden.
	Durch einfaches Weglassen der Gliederungszeichen ergibt sich dann die Polnische Notation.%
}
besser umgehen können und Klammern dort überflüssig sind, werden viele \glsIdxPl{Formel} auch in die Polnische Notation überführt. Dies wird auch in \ASBA\ so gehandhabt.

\subsubsection{Bausteine der aussagenlogischen Sprache}% - - - - - - - - - - - -
\label{subsub:Bausteine}

Zur Einteilung der aussagenlogischen Junktoren werden die folgenden Mengen definiert:
\begin{align}
	& \glsSym{alCon}              & \defeq\quad & \{ \ltrue, \lfalse \}
	&& \text {, Menge der \emph{aussagenlogischen Konstanten}}
	\idx{Konstanten, Menge der}         \label{def:C}
	\\
	& \glsSym{alUna}              & \defeq\quad & \{ \lnot \}
	&& \text{, Menge der \emph{unären aussagenlogischen Operatoren}}
	\idx{unären Operatoren, Menge der}  \label{def:U}
	\\
	& \glsSym{alBin}              & \defeq\quad &
	\{ \land, \lor, \limp, \lequiv, \lrep, \lnand, \lnor, \lmult, \ladd \}
	&& \text{, Menge der \emph{binären aussagenlogischen Operatoren}}
	\idx{binären Operatoren, Menge der} \label{def:B}
\end{align}
%
Um damit \glsIdxPl{Formel} zu bilden, werden noch Variable gebraucht:
\begin{align}
	& \glsSym{alVar}  & \defeq     \quad & \{ \alvar_n \mid n \in \gsNo \}
	&&&&
	&& \text{, Menge der aussagenlogischen \emph{Variablen}} \label{def:alVar}
	&&
\end{align}
%
Die Mengen $\alCon$, $\alUna$, $\alBin$ und $\alVar$ müssen paarweise disjunkt sein. --
Damit sind alle \vrefintab{tab:Symbole} verwendeten wesentlichen Konstanten und Operatoren%
\footnote{%
	Jeweils nur die ersten der grau hinterlegten Zeilen sowie \symqt{\lmult}.%
}
und die Variablen erfasst und es können die folgende Mengen definiert werden:
\begin{align}
	& \glsSym{alJun}  & \defeq     \quad & \alCon \cup \alUna \cup \alBin
	&& \text{, Menge der aussagenlogischen \emph{Junktoren}}
	\idx{Junktoren, Menge der}                              \label{def:alJun}
	\\
	& \glsSym{alABC}  & \defeq     \quad & \alVar \cup \alJun
	&& \text{, \emph{Alphabet der aussagenlogischen Sprache (für }} \alJun
	\text{\emph{)}}
	\idx{Alphabet der logischen Sprache}                    \label{def:alABC}
	\\
	& \glsSym{alJunx} & \subseteq\;\quad & \alJun
	&& \text{, eine Teilmenge der Junktoren für eine Indexvariable }
	x                                                       \label{def:alJunx}
	\\
	& \glsSym{alABCx} & \defeq     \quad & \alVar \cup \alJunx \quad
	&& \text{, Alphabet der aussagenlogischen Sprache \emph{für} } \alJunx
	\idx{Teil-Alphabet der aussagenlogischen Sprache}       \label{def:alABCx}
\end{align}
%
Für Elemente aus $\alVar$ werden hier normalerweise die großen lateinischen Buchstaben \objqt{A}, \objqt{B}, \objqt{C}, \textusw\ verwendet.
Sie werden auch \emph{\Idx{Satzbuchstabe}n} oder kurz \emph{\Idx{Atom}e} genannt.

\subsubsection{Aussagenlogische Formeln}%  - - - - - - - - - - - - - - - - -
\label{subsub:Formeln}

Neben dem Alphabet $\alABC$ \textbzw\ $\alABCx$ werden noch Klammern als Gliederungszeichen verwendet.
Damit können nun rekursiv für jede Teilmenge $\alJunx$ von $\alJun$ zwei Mengen von Formeln definiert werden:

$\glsSym{alForx}$ sei die Menge der auf folgende Weise definierten \emph{aussagenlogischen Formeln mit Klammerung}%
\idx{aussagenlogische Formel mit Klammerung}:
\begin{align}
	&                                 & \alVar              \subset \alForx
	\\
	&                                 & \alJunx \cap \alCon \subset \alForx
	\\
	A    \in \alForx & \quad \metaimp &  (\opbsp A)         \in     \alForx
	& & \text{, für} \quad \opbsp \in \alUna \cap \alJunx
	\\
	A, B \in \alForx & \quad \metaimp & (A \opbsp B)        \in     \alForx
	& & \text{, für} \quad \opbsp \in \alBin \cap \alJunx
	\formulatoleft
\end{align}
Nur die auf diese Weise konstruierten Formeln sind Elemente von $\alForx$.
\\Für $\alJun = \alJunx$ sei noch $\alFor \defeq \glsSym{alForx}$.

$\glsSym{alForxp}$ sei die Menge der auf folgende Weise definierten aussagenlogischen Formeln in \emph{Polnischer Notation}%
\idx{aussagenlogische Formel in Polnischer Notation}:
\begin{align}
	&                                  & \alVar              \subset \alForxp
	\\
	&                                  & \alJunx \cap \alCon \subset \alForxp
	\\
	A    \in \alForxp & \quad \metaimp &  (\opbsp A)         \in     \alForxp
	& & \text{, für}  \quad \opbsp \in \alUna \cap \alJunx
	\\
	A, B \in \alForxp & \quad \metaimp & (A \opbsp B)        \in     \alForxp
	& & \text{, für}  \quad \opbsp \in \alBin \cap \alJunx
	\formulatoleft
\end{align}
Nur die auf diese Weise konstruierten Formeln sind Elemente von $\alForxp$.
Schließlich sei noch $\alForp \defeq \glsSym{alForxp}$ falls $\alJunx = \alJun$.

Wie man leicht sieht, gilt
\begin{equation}
	\alJunx     \: \subset \: \alJuny \: \subseteq \: \alJun \metaimp
	\begin{cases}
		\alABCx \: \subset \: \alABCy \: \subseteq \: \alABC \\
		\alForx \; \subset \: \alFory \; \subseteq \: \alFor \\
		\alForxp\, \subset \: \alForyp\, \subseteq \: \alForp
	\end{cases}
\end{equation}
und weiterhin gibt es eine bijektive Abbildung von $\alFor$ nach $\alForp$. Auf einen \glsIdx{Beweis} verzichten wir.
%
Durch Anwendung der Klammerregeln \vrefvonsubsub{subsub:Bausteine} lassen sich in der Regel noch viele Klammern der \glsIdxPl{Formel} aus $\alForx$ einsparen.
Die \glsIdxPl{Formel} aus $\alForxp$ sind frei von Klammern.
Die Namen der Operatoren finden sich \vrefintab{tab:Symbole}.
Für aussagenlogische \glsIdxPl{Formel}, \textdh\ von Elementen aus $\alFor$ \textbzgl\ $\alForp$, werden hier normalerweise die kleinen griechischen Buchstaben $\alpha$, $\beta$, $\gamma$, \textusw\ verwendet.
Sie können dabei auch als \emph{\glsIdx{atomareFormelV}} bezeichnet werden, \textdh\ \glsIdxPl{Formel}, die sich nicht weiter zerlegen lassen.\footnote{%
	Nur die Elemente von $\alVar$ und $\alCon$ sind unzerlegbar, sofern letztere nicht durch andere \glsIdxPl{Formel} definiert werden.%
}

\subsection{Definition aussagenlogischer Operatoren durch andere}% -------------
\label{sub:ausOperatorDef}

Im folgenden gelte für zwei aussagenlogische \glsIdxPl{Formel} $\alpha$ und $\beta$:
\begin{itemize}
	\item[] $\alpha \eq    \beta \quad \metadefeq$ \quad $\alpha$ und $\beta$
	stimmen als Zeichenkette überein.
	%
	\item[] $\alpha \equiv \beta \quad \metadefeq$ \quad $\alpha$ und $\beta$
	\parbox[t]{11cm}{können mit Hilfe erlaubter \glsIdxPl{Substitution} und geltender \glsIdxPl{Axiom} -- \vrefseesub{sub:ausAxiome} -- ineinander überführt werden.}
\end{itemize}
%
Es werden verschiedene Teilmengen von $\alJun$ eingeführt, die jeweils ausreichen alle anderen Elemente aus $\alJun$ zu definieren:
\begin{align}
	& \alJun_\xBool & \defeq & \quad\{ \lnot, \land, \lor \} \label{def:Jbool}
	\qquad (\text{\glsIdx{BoolscheSignatur}})
	\\
	& \alJun_\xAnd  & \defeq & \quad\{ \lnot, \land       \} \label{def:Jand}
	\\
	& \alJun_\xOr   & \defeq & \quad\{ \lnot, \lor        \} \label{def:Jor}
	\\
	& \alJun_\xImp  & \defeq & \quad\{ \lnot, \limp       \} \label{def:Jimp}
	\\
	& \alJun_\xRep  & \defeq & \quad\{ \lnot, \lrep       \} \label{def:Jrep}
	\\
	& \alJun_\xNand & \defeq & \quad\{ \lnand             \} \label{def:Jnand}
	\\
	& \alJun_\xNor  & \defeq & \quad\{ \lnor              \} \label{def:Jnor}
	\formulatoleft\formulatoleft\formulatoleft
\end{align}
Solche Teilmengen heißen \glsIdx{logischeSignaturV}.
%
Im Folgenden stehen jeweils links die \glsIdxPl{Formel} in üblicher Schreibweise vollständig geklammert und rechts in Polnischer Notation (ohne Klammern).
Ferner seien $\alpha$ und $\beta$ beliebige, nicht notwendig verschiedene \glsIdxPl{Formel} aus der passenden Menge $\alForx$ \textbzgl\ der um die mit Hilfe der Definitionen erweiterten Formelmenge.

Ausgehend von den Operatoren aus der \glsIdxX{BoolscheSignatur} $\alJun_\xBool$ werden die restlichen Operatoren aus $\alJun$ definiert. Die Definitionen sind in zwei Gruppen eingeteilt, und zwar die mit den Operatoren aus $\alJun_\xAnd$:
\begin{align}
	% folgt ------------------------
	(\alpha \limp \beta) &\;\defeq\; (\lnot (\alpha \land  (\lnot \beta))) &
	\limp \alpha \beta   &\;\defeq\;  \lnot    \land \alpha \lnot \beta
	\label{def:imp}
	\\
	% sofern -----------------------
	(\alpha \lrep \beta) &\;\defeq\; (\lnot (\beta \land  (\lnot \alpha))) &
	\lrep \beta  \alpha  &\;\defeq\;  \lnot    \land \beta \lnot \alpha
	\label{def:rep}
	\\
	% genau dann -------------------
	(\alpha\lequiv\beta) &\;\defeq\;((\alpha\limp\beta)\land(\alpha\lrep\beta))&
	\lequiv\alpha \beta  &\;\defeq\;\land \limp \alpha \beta \lrep \alpha \beta
	\label{def:equiv}
	\\
	%falsch ------------------------
	\lfalse              &\;\defeq\; (\alvar_0 \land (\lnot \alvar_0)) &
	\lfalse              &\;\defeq\;  \land \alvar_0  \lnot \alvar_0   \label{def:false}
	\\
	% mal --------------------------
	(\alpha \lmult \beta)&\;\defeq\; (\alpha \land \beta)          &
	\lmult \alpha  \beta &\;\defeq\;  \land \alpha \beta        \label{def:mult}
	\\
	% NAND -------------------------
	(\alpha \lnand \beta)&\;\defeq\; (\lnot (\alpha \land \beta )) &
	\lnand \alpha  \beta &\;\defeq\;  \lnot  \land \alpha \beta \label{def:nand}
\end{align}
und die mit den Operatoren aus $\alJun_\xOr$:
\begin{align}
	% NOR --------------------------
	(\alpha \lnor \beta) &\;\defeq\; (\lnot (\alpha \lor \beta))   &
	\lnor \alpha  \beta  & \;\defeq\;  \lnot  \lor \alpha \beta \label{def:nor}
	\\
	% plus -------------------------
	(\alpha\ladd\beta)&\;\defeq\;((\alpha\lor\beta)\land(\lnot(\alpha\land\beta)))&
	\ladd\alpha \beta &\;\defeq\;  \land \lor\alpha\beta \lnot \land\alpha\beta
	\label{def:add}
	\\
	% wahr -------------------------
	\ltrue & \;\defeq\; (\alvar_0 \lor (\lnot \alvar_0)) &
	\ltrue & \;\defeq\;  \lor \alvar_0  \lnot \alvar_0
	\label{def:true}
\end{align}
%
Ist \symqt{\lor} oder \symqt{\land} nicht vorgegeben, \textdh\ wird von den Elementen aus $\alJun_\xAnd$ \textbzgl\ $\alJun_\xOr$ statt von denen aus $\alJun_\xBool$ ausgegangen, so muss man den fehlenden Operator mittels der passenden der beiden folgenden Definitionen einführen:
\begin{align}
	% oder aus und -----------------
	(\alpha \lor \beta)  & \;\defeq\; (\lnot((\lnot\alpha)\land(\lnot\beta))) &
	\lor \alpha  \beta   & \;\defeq\;  \lnot \land \lnot \alpha \lnot \beta
	\label{def:orand} \\
	% und aus oder -----------------
	(\alpha \land \beta) & \;\defeq\; (\lnot((\lnot\alpha)\lor(\lnot\beta)))  &
	\land \alpha  \beta  & \;\defeq\;  \lnot \lor \lnot \alpha \lnot \beta
	\label{def:andor}
\end{align}
Nun sind wieder alle Operatoren definiert.

Entsprechend wird bei Vorgabe von $\alJun_\xImp$ \textbzgl\ $\alJun_\xRep$ die passende der beiden folgenden Definitionen ausgewählt:
\begin{align}
	% oder aus imp -----------------
	(\alpha \lor  \beta) & \;\defeq\; ((\lnot \alpha) \limp \beta)         &
	\lor \alpha   \beta  & \;\defeq\;   \limp \lnot \alpha \beta
	\label{def:orrep}
	\\
	% und aus rep ------------------
	(\alpha \land \beta) & \;\defeq\; (\lnot ((\lnot \beta) \lrep \alpha)) &
	\land \alpha  \beta  & \;\defeq\;  \lnot \lrep \lnot \beta \alpha
	\label{def:andrep}
\end{align}
woraufhin dann \eqref{def:imp} \textbzgl\ \eqref{def:rep} als Gleichung nachzuweisen ist.
Da aus \eqref{def:rep} durch Vertauschung der Variablen unmittelbar
\begin{align}
	(\alpha \lrep \beta) & \;\equiv\; (\beta \limp \alpha) &
	\lrep \alpha  \beta  & \;\equiv\;  \limp \beta \alpha  \label{eq:repimp}
\end{align}
folgt, vermindert sich der Aufwand dazu erheblich.

Bei Vorgabe von $\alJun_\xNand$ \textbzgl\ $\alJun_\xNor$ schließlich werden die passenden Definition aus
\begin{align}
	% nicht aus nor ----------------
	(\lnot \alpha) & \;\defeq\; (\alpha \lnor \alpha)  &
	\lnot  \alpha  & \;\defeq\;  \lnor \alpha \alpha   \label{def:notnor} \\
	% nicht aus nand ---------------
	(\lnot \alpha) & \;\defeq\; (\alpha \lnand \alpha) &
	\lnot  \alpha  & \;\defeq\;  \lnand \alpha \alpha  \label{def:notnand}
\end{align}
und, da \symqt{\lnot} jetzt definiert ist, aus
\begin{align}
	% oder aus nor -----------------
	(\alpha \lor \beta)  & \;\defeq\; (\lnot(\alpha \lnor \beta))  &
	\lor \alpha  \beta   & \;\defeq\;  \lnot \lnor \alpha \beta
	\label{def:ornor} \\
	% und aus nand -----------------
	(\alpha \land \beta) & \;\defeq\; (\lnot(\alpha \lnand \beta)) &
	\land \alpha  \beta  & \;\defeq\;  \lnot \lnand \alpha \beta
	\label{def:andnand}
\end{align}
ausgewählt und es ist \eqref{def:nand} \textbzgl\ \eqref{def:nor} als Gleichung nachzuweisen.

Abschließend ist noch nachzuweisen, dass mit Hilfe der jeweils passenden der Definitionen \eqref{def:imp} bis \eqref{def:andnand}, ausgehend vom jeweils passenden $\alForx$, genau die gesamte Formelmenge $\alFor$ erzeugt werden kann.

\subsection{Aussagenlogisches Axiomensystem}% ----------------------------------
\label{sub:ausAxiome}

Ausgehend von der \glsIdxBg{logischeSignaturV}{logischen Signatur} $\alJun_\xAnd = \{\lnot, \land\}$ und der \vrefdef{def:imp} von \symqt{\limp} werden die folgenden vier logischen \glsIdxPl{Axiom} definiert:
\begin{align}
	&
	(\alpha\limp\beta\limp\gamma)\limp(\alpha\limp\beta)\limp(\alpha\limp\gamma)
	\formulaspace &
	& \limp\limp\alpha\limp\beta\gamma\limp\limp\alpha\beta\limp\alpha\gamma \\
	%
	& \alpha \limp \beta \limp \alpha \land \beta
	\formulaspace &
	& \limp \alpha \limp \beta \land \alpha \beta \\
	%
	& \alpha \land \beta \limp \alpha \;; \quad \alpha \land \beta \limp \beta
	\formulaspace &
	& \limp \land \alpha \beta \alpha \;; \quad \limp \land \alpha \beta \beta\\
	%
	&(\alpha \limp \lnot \beta) \limp (\beta \limp \lnot \alpha)
	\formulaspace &
	& \limp \limp \alpha \lnot \beta \limp \beta \lnot \alpha
	\formulatoleft
	%
\end{align}
%
\todo{Aussagenlogik weiter bearbeiten.}%%%
%TODO Aussagenlogik weiter bearbeiten. %%%

\section{Prädikatenlogik}% =====================================================
\beginsection{Prädikatenlogik}
\label{sec:Praedikatenlogik}
\hidden{\glsIdx{Praedikatenlogik}}
%TODO Fehler: Bei Prädikatenlogik im Glossar fehlt Verweis auf diese Stelle

\todo{Prädikatenlogik bearbeiten.}%%%
%TODO Prädikatenlogik bearbeiten. %%%

%%%%%%%%%%%%%%%%%%%%%%%%%%%%%%%%%%%%%%%%%%%%%%%%%%%%%%%%%%%%%%%%%%%%%%%%%%%%%%%%
%%%\section{Schlussregeln}% =======================================================
%%%\beginsection{Schlussregeln}
%%%\label{sec:Schlussregeln}
%%%\hidden{\glsIdx{Schlussregel}}
%%%
%%%In diesem \sectionname\ geht es um die \glsIdxPl{zulaessigeTransformationA}, \textdh\ die \glsIdxBg{allgemeingueltigeSchlussregelV}{allgemeingültigen Schlussregeln}.
%%%Dazu gehören zunächst die \glsIdxPl{Basisregel}.
%%%Dann aber auch alle aus den \glsIdxPl{Basisregel} und den bis dahin \glsIdxBg{allgemeingueltigeSchlussregelV}{allgemeingültigen Schlussregeln} korrekt abgeleiteten neuen \glsIdxPl{Schlussregel}.
%%%Die \glsIdxPl{Schlussregel} haben die Form eines Formalen \glsIdx{Satz}es.
%%%
%%%\subsection{Basisregeln}% ------------------------------------------------------
%%%\label{sub:Basisregeln}
%%%\hidden{\glsIdxPl{Basisregel}}
%%%
%%%Gemäß \cite{bib:Rautenberg} Kapitel~1.4 \emph{Ein vollständiger aussagenlogischer Kalkül} werden sechs \glsIdxPl{Basisregel} definiert. Zuvor werden aber noch einige Definition gebraucht. Dazu seien $n$, $m$, $k$ und $l$ natürliche Zahlen (auch 0), $\alpha$, $\alpha_i$, $\beta$ und $\beta_j$ \glsIdxPl{Formel}, $X$, $X_i$, $Y$ und $Y_j$ Mengen von \glsIdxBg{Formel}{formalen Elementen} und
%%%\begin{align}
%%%	%
%%%	&X&&\defeq&&X_1\cup X_2\cup...\cup X_n\cup\{\alpha_1,\alpha_2,...,\alpha_m\}
%%%	\\
%%%	&Y&&\defeq&&Y_1\cup Y_2\cup...\cup Y_k\cup\{\beta_1, \beta_2, ...,\beta_l \}
%%%	\formulatoleft\formulatoleft
%%%\end{align}
%%%%
%%%$X$ und $Y$ können auch die leere Menge sein. Damit wird definiert:
%%%\begin{align}
%%%	& \alpha \glsSym{derive} \beta \quad \metadefeq \quad
%%%	\parbox[t]{10.5cm}{%
%%%	$\beta$ ist mittels schrittweiser Anwendung \emph{\glsIdxBg{zulaessigeTransformationA}{zulässiger Transformationen}} (siehe weiter unten) aus $\alpha$ ableitbar.
%%%	Sprechweise: Aus $\alpha$ ist $\beta$ \emph{ableitbar} oder \emph{beweisbar};
%%%	kurz: \enquote{$\alpha$ \emph{\glsIdx{ableitbar}} $\beta$} \textbzgl\ \enquote{$\alpha$ \emph{\glsIdx{beweisbar}} $\beta$}
%%%	-- Es kann auch \symqt{\alpha} durch \symqt{X} und/oder \symqt{\beta} durch \symqt{Y} ersetzt werden.
%%%	}
%%%	\label{def:ableitbar}
%%%	\\
%%%	& \derive \beta \quad \metadefeq \quad \emptyset \derive \beta \qquad \text{(\symqt{\textderive} kann dann auch ganz entfallen)}
%%%	\\
%%%	&             X_1, X_2, ...,X_n, \alpha_1, \alpha_2, ..., \alpha_m \quad
%%%	\derive \quad Y_1, Y_2, ...,Y_n,  \beta_1,  \beta_2,  ..., \beta_m \quad
%%%	\metadefeq \quad X \derive Y
%%%	\label{def:ableitbarKurz} \formulatoleft
%%%\end{align}
%%%%
%%%Eine \emph{\glsIdx{zulaessigeTransformationA}} ist die Anwendung einer \emph{\glsIdx{Substitution}}{\vrefnotesub{sub:Indentitaetsregeln} (siehe unten), einer \emph{\glsIdx{Basisregel}} (siehe unten) oder einer davon abgeleiteten sonstigen \emph{\glsIdx{Schlussregel}}, \textzB\ aus \vrefsub{sub:Schlussregeln}.
%%%Bei den \glsIdxPl{Schlussregel} und der \glsIdx{Substitution} ($\subst$) soll das Komma stärker binden als \symqt{\derive}, \symqt{\subst} und \symqt{\srand},
%%% wobei \symqt{\srand} für \enquote{und} \textbzgl\ \symqt{\glsIdx{metaand}}\vrefnotesub{sub:Aussagen} steht und schwächer bindet als \symqt{\derive} und \symqt{\subst}.%
%%%\footnote{siehe Fußnote~3 \vrefvontab{tab:Prio-Aussagenlogik}} %%% Kommentar zu \srand prüfen
%%%
%%%Zur der Auswahl der \glsIdxPl{Basisregel}, der Formulierung und der Bezeichnungen wird auf~\cite{bib:Rautenberg,bib:NatuerlichesSchliessen} zurückgegriffen.
%%%Wie in~\cite{bib:NatuerlichesSchliessen} steht \symqt{E} für \enquote{-Einführung} und \symqt{B} für \enquote{-Beseitigung} (oder \enquote{-Elimination}) von Operatoren.%
%%%\footnote{%
%%%	In der \glsIdx{Monotonieregel} wird hier, anders als in~\cite{bib:Rautenberg}, \forqt{X,Y} statt \forqt{Y \text{ , für } Y \supseteq X} genommen. Das ist gleichwertig, vermeidet aber den Zusatz \forqt{\text{ , für } Y \supseteq X}.
%%%	Außerdem werden bei den Bezeichnungen \forqt{($\land$1)} und \forqt{($\land$2)} gemäß~\cite{bib:NatuerlichesSchliessen} durch \forqt{\andE} \textbzgl\ \forqt{\andB} ersetzt.
%%%}
%%%
%%%Im Folgenden seien $\alpha$ und $\beta$ wieder stets \glsIdxPl{Formel} und $X$ und $Y$ Mengen von \glsIdxBg{Formel}{formalen Elementen}.
%%%Für die sechs \glsIdxPl{Basisregel} werden dann nur noch die logischen Operatoren \symqt{\lnot} und \symqt{\land} benötigt.
%%%Bei den weiteren \glsIdxPl{Schlussregel} wird noch \symqt{\limp} gemäß der Definition~\vref{def:imp} verwendet.
%%%%
%%%\begin{align}
%%%	& \frac{}{\alpha\derive\alpha}
%%%	& & (\text{\glsIdx{Anfangsregel}})
%%%	\tag{\tagAR} \sym{\gls{AR}} \label{def:Anfangsregel}
%%%	\\\\
%%%	& \frac{X\derive\alpha}{X,Y\derive\alpha}
%%%	& & (\text{\glsIdx{Monotonieregel}})
%%%	\tag{\tagMR} \sym{\gls{MR}} \label{def:Monotonieregel}
%%%	\\\\
%%%	& \frac{X\derive\alpha,\lnot\alpha}{X\derive\beta}
%%%	& & (\text{Einführung/Beseitigung der Negation Teil 1})
%%%	\tag{\tagnota} \sym{\gls{nota}} \label{def:nota}
%%%	\\\\
%%%	& \frac{X,\alpha\derive\beta \srand X,\lnot\alpha\derive\beta}{X\derive\beta}
%%%	& & (\text{Einführung/Beseitigung der Negation Teil 2})
%%%	\tag{\tagnotb} \sym{\gls{notb}} \label{def:notb}
%%%	\\\\
%%%	& \frac{X\derive\alpha,\beta}{X\derive\alpha\land\beta}
%%%	& & (\text{Einführung der Konjunktion})
%%%	\tag{\tagandE} \sym{\gls{andE}} \label{def:andE}
%%%	\\\\
%%%	& \frac{X\derive\alpha\land\beta}{X\derive\alpha,\beta}
%%%	& & (\text{Beseitigung der Konjunktion})
%%%	\tag{\tagandB} \sym{\gls{andB}} \label{def:andB}
%%%	\formulatoleft
%%%\end{align}
%%%%
%%%In einer \glsIdx{Schlussregel} werden die \glsIdxBg{Formel}{formalen Elemente}%
%%%\footnote{hier: \glsIdxPl{Aussage} in einer formalen Form.}
%%%über dem Querstrich als \emph{\glsIdxPl{Voraussetzung}} und die unter dem Querstrich als \emph{\glsIdx{Folgerung}} der Regel bezeichnet.
%%%Eine \glsIdx{Schlussregel} steht für die \glsIdx{Aussage}, dass mit ihren \glsIdxPl{Voraussetzung} auch auch ihre \glsIdxPl{Folgerung} gelten.
%%%-- Im Gegensatz zu den weiteren \glsIdxPl{Schlussregel} werden die oben aufgelisteten Basisregeln nicht weiter hinterfragt, \textdh\ sie gelten quasi als \glsIdxPl{Axiom}.
%%%
%%%\subsection{Identitätsregeln}% --------------------------------------------------------
%%%\label{sub:Indentitaetsregeln}
%%%
%%%Die zulässigen Transformationen, \textdh\ die Anwendung der \glsIdxPl{Schlussregel}, erfordern zulässige \glsIdxPl{Substitution}.
%%%Damit wird dem Gleichheits- oder Identitätszeichen \symqt{\eq} dann mittels Einführungs- und Beseitigungsregel eine Bedeutung verliehen.\footnote{siehe~\cite{bib:NatuerlichesSchliessen}}
%%%Dazu seien $\alpha$, $\beta$ und $\gamma$ \glsIdxPl{vergleichbar}%
%%%\footnote{siehe Ende \vrefvonsub{sub:Aussagen}}
%%%\glsIdxPl{Formel}.
%%%Zunächst wird definiert:
%%%\begin{align}
%%%	\gamma(\alpha \subst \beta) \quad \defeq \quad
%%%	\parbox[t]{11cm}{%
%%%		Das \glsIdxBg{Formel}{formale Element}, dass man erhält, wenn in $\gamma$ alle oder nur einige Vorkommen von $\alpha$ durch $\beta$ ersetzt werden.
%%%		-- Gegebenenfalls muss noch die Auswahl der Ersetzungen angegeben werden, andernfalls werden alle Vorkommen ersetzt.
%%%		Letzteres heißt dann \emph{vollständige} \glsIdx{Substitution}.
%%%	} \label{def:Substitution}\\
%%%	\gamma(\alpha \swap \beta) \quad \defeq \quad
%%%	\parbox[t]{11cm}{%
%%%		Das \glsIdxBg{Formel}{formale Element}, dass man erhält, wenn in $\gamma$ alle $\alpha$ und $\beta$ miteinander vertauscht werden.
%%%		Dazu ist es nötig, das $\alpha$ und $\beta$ voneinander unabhängig sind, vorzugsweise zwei verschiedene Variable.
%%%	} \label{def:Vertauschung}
%%%\end{align}
%%%\forqt{\alpha \subst \beta} heißt \emph{\glsIdx{Substitution}} und \forqt{\alpha \swap \beta} \emph{\glsIdx{Vertauschung}} oder kurz \emph{Tausch}.
%%%-- Sei noch $S = (s_1, s_2, ...)$ eine endliche Folge von \glsIdxPl{Substitution}, die auch \glsIdxPl{Vertauschung} enthalten und auch leer sein kann. Dann wird definiert:
%%%\begin{align}
%%%	\gamma(S) & \quad \defeq \quad \gamma(s_1)(s_2)... \label{def:Substitutionen}\\
%%%	\gamma(\emptyset) & \quad \; = \quad \gamma & \text{(nur zur Verdeutlichung)}\\
%%%	\gamma(s_1,s_2,...) & \quad \defeq \quad \gamma(S)
%%%\end{align}
%%%%
%%%Die \glsIdx{Vertauschung} ist eine spezielle Form der \glsIdx{Substitution}.
%%%Wenn $x$ und $y$ zwei verschiedene Variable, die in $\alpha$, $\beta$ und $\gamma$ nicht vorkommen, gilt:
%%%\[
%%%	\gamma(\alpha \swap \beta) = \gamma(\alpha\subst x, \beta\subst y,  y\subst\alpha, x \subst\beta)
%%%\]
%%%
%%%Sei zusätzlich noch $s$ eine \glsIdx{Substitution}.
%%%Folgende Sprechweisen werden verwendet:
%%%\begin{itemize}
%%%	\renewcommand*{\itemindent}{1,5cm}
%%%	\renewcommand*{\labelsep}{5pt}
%%%	\item [$\gamma(\alpha \subst \beta)$ :] In $\gamma$ wird $\alpha$ (vollständig) \emph{durch $\beta$ substituiert}.
%%%	\item [$\gamma(\alpha \swap \beta)$ :] In $\gamma$ werden $\alpha$ und $\beta$ \emph{vertauscht}.
%%%	\item [$\gamma(s)$ :] $s$ wird auf $\gamma$ \emph{angewendet}.
%%%	\item [$\gamma(S)$ :] Die \glsIdxPl{Substitution} aus S werden in der angegebenen Reihenfolge auf $\gamma$ angewendet.
%%%	\item [$\gamma(S)$ :] $S$ wird auf $\gamma$ angewendet.
%%%\end{itemize}
%%%%
%%%Bei obiger Definition der \glsIdx{Substitution} bleibt noch offen, unter welchen \glsIdxPl{Voraussetzung} sie angewendet werden darf. Das soll hier nicht untersucht werden. In diesem \sectionname\ genügt es, das nur \glsIdx{Vertauschung} und vollständige \glsIdx{Substitution} verwendet werden.
%%%In diesen Fällen sind beliebige \glsIdxPl{Substitution} von Variablen durch \glsIdxPl{Formel} erlaubt.
%%%
%%%Ist $\gamma$ wie oben und $S$ eine Menge von \glsIdxPl{Substitution}.
%%%
%%%Nun können die beiden \glsIdxPl{Identitaetsregel} definiert werden:
%%%\begin{align}
%%%	& \frac{}{\alpha\eq\alpha}
%%%	& & (\text{Einführung der Identität})
%%%	\tag{\tageqE} \sym{\gls{eqE}} \label{def:eqE}
%%%	\\\\
%%%	& \frac{\alpha\eq\beta \srand \gamma}{\gamma(\alpha\subst\beta)}
%%%	& & (\text{Beseitigung der Identität})
%%%	\tag{\tageqB} \sym{\gls{eqB}} \label{def:eqB}
%%%	\formulatoleft
%%%\end{align}
%%%%
%%%Die \glsIdxPl{Identitaetsregel} werden hier eingeführt, um die \glsIdx{Substitution} zu rechtfertigen.
%%%Wie die \glsIdxPl{Basisregel} gelten sie als \glsIdxPl{Axiom}, würden also eigentlich dazu gehören.
%%%Da sie aber nicht weiter verwendet werden, werden sie hier nicht zu den \glsIdxPl{Basisregel} gezählt.
%%%
%%%\subsection{Weitere Schlussregeln}% --------------------------------------------
%%%\label{sub:Schlussregeln}
%%%
%%%In~\cite{bib:Rautenberg} werden aus den Basisregeln mittels \glsIdxBg{zulaessigeTransformationA}{zulässiger Transformationen} weitere \glsIdxPl{Schlussregel} abgeleitet.%
%%%%TODO Identitätsregeln kommen bei Rautenberg später vor. ???
%%%\footnote{%
%%%	In~\cite{bib:Rautenberg} werden die \glsIdxPl{Identitaetsregel} zwar weder aufgeführt noch angewandt, ohne \glsIdx{Substitution} geht es aber nicht.
%%%}
%%%Man vergleiche auch mit~\cite{bib:NatuerlichesSchliessen}.
%%%%
%%%\begin{align}
%%%	& \frac{X,\lnot\alpha\derive\alpha}{X\derive\alpha}
%%%	& & (\text{Beseitigung der Negation; Indirekter \glsIdx{Beweis}})
%%%	\tag{\tagnotc} \sym{\gls{notc}} \label{def:notc}
%%%	\\\\
%%%	& \frac{X,\lnot\alpha\derive\beta,\lnot\beta}{X\derive\alpha}
%%%	& & (\text{Reductio ad absurdum})
%%%	\tag{\tagnotd} \sym{\gls{notd}} \label{def:notd}
%%%	\\\\
%%%	& \frac{X,\alpha\derive\beta}{X\derive\alpha\limp\beta}
%%%	& & (\text{Einführung der Implikation})
%%%	\tag{\tagimpE} \sym{\gls{impE}} \label{def:impE}
%%%	\\\\
%%%	& \frac{X\derive\alpha\limp\beta}{X,\alpha\derive\beta}
%%%	& & (\text{Beseitigung der Implikation})
%%%	\tag{\tagimpB} \sym{\gls{impB}} \label{def:impB}
%%%	\\\\
%%%	& \frac{X\derive\alpha \srand X,\alpha\derive\beta}{X\derive\beta}
%%%	& & (\text{\glsIdx{Schnittregel}})
%%%	\tag{\tagSR} \sym{\gls{SR}} \label{def:SR}
%%%	\\\\
%%%	& \frac{X\derive\alpha \srand \alpha\limp\beta}{X\derive\beta}
%%%	& & (\text{\glsIdx{Abtrennungsregel}--\emph{Modus ponens}})
%%%	\tag{\tagTR} \sym{\gls{TR}} \label{def:TR}
%%%	\formulatoleft
%%%\end{align}
%%%%
%%%Dabei werden zum \glsIdx{Beweis} der \glsIdxPl{Schlussregel} in~\cite{bib:Rautenberg} folgende Basisregeln verwendet:
%%%\begin{itemize}
%%%	\renewcommand*{\itemindent}{1cm}
%%%	\renewcommand*{\labelsep}{5pt}
%%%	\item[\tagnotc~:] \tagAR, \tagMR,           \tagnotb
%%%	\item[\tagnotd~:] \tagAR, \tagMR, \tagnota, \tagnotb
%%%	\item[\tagimpE~:] \tagAR, \tagMR, \tagnota, \tagnotb, \tagandE
%%%	\item[\tagimpB~:] \tagAR, \tagMR, \tagnota, \tagnotb          , \tagandB
%%%	\item[\tagSR  ~:] \tagAR, \tagMR, \tagnota, \tagnotb
%%%	\item[\tagTR  ~:] \tagAR, \tagMR, \tagnota, \tagnotb, \tagandE
%%%\end{itemize}
%%%%
%%%\subsection{Beispiel einer Ableitung}% -----------------------------------------
%%%\label{sub:BeispielAbleitung}
%%%
%%%Als Beispiel wird hier die \glsIdx{Schnittregel} aus den Basisregeln abgeleitet.%
%%%\footnote{%
%%%	Die Form der Tabelle ist angelehnt an~\cite{bib:NatuerlichesSchliessen} Kapitel~2.2.4 \emph{Eine Beispielableitung}.%
%%%}
%%%Dazu wird verabredet, dass \vrefintab{tab:AbleitungSchnittregel} der Inhalt der Zelle in der Zeile $i$ und der Spalte $(X_n)$ mit $X_i$ bezeichnet wird.
%%%Zur kürzeren Darstellung wird statt auf die vollständigen Spaltenüberschriften nur auf die dort notierten $(X_n)$ verwiesen. Dass in der Spalte $(n)$ stets die Zeilennummer steht, wird im folgenden nicht mehr extra erwähnt.
%%%-- Für die ausgefüllten Felder wird nun definiert:%
%%%\footnote{%
%%%	Eigentlich müsste man für jede \glsIdx{Substitution} aus $S_i$ eine eigene Zeile vorsehen.
%%%	Um die Tabellen für die \glsIdxPl{Beweis} kürzer zu halten, werden aufeinanderfolgende \glsIdxPl{Substitution} zusammengefasst.%
%%%}
%%%\begin{align}
%%%	R_i & \defeq
%%%	\left\{
%%%		\begin{array}{l}
%%%			\text{- \enquote{\glsIdx{Voraussetzung}} = Die \glsIdx{Aussage} $A_i$ ist eine \glsIdx{Voraussetzung}.}\\
%%%			\text{- \enquote{\glsIdl{Folgerung}} = Die \glsIdx{Aussage} $A_i$ ist eine \glsIdl{Folgerung}.}\\
%%%			\text{- \enquote{Annahme} = Die \glsIdx{Aussage} $A_i$ wird vorübergehend als zutreffend angenommen.}\\
%%%			\text{- $j$ = Verweis auf die \glsIdx{Schlussregel} $\overline{R}_j$ für ein $j < i$.}\\
%%%			\text{- Verweis (ohne Klammern) auf eine \glsIdx{allgemeingueltigeSchlussregelV}.}
%%%		\end{array}
%%%	\right.
%%%	\\
%%%	S_i & \defeq \text{Die Reihe der anzuwendenden \glsIdxPl{Substitution}.}
%%%	\\
%%%	\overline{R}_i & \defeq \text{Das Ergebnis der in der angegebenen Reihenfolge angewendeten}\\
%%%	& \quad\;\; \text{\glsIdxPl{Substitution} aus $S_i$ auf die \glsIdx{Schlussregel} $R_i$}
%%%	\\
%%%	Z_i & \defeq \text{Die Indizes $j$ (mit $j < i$) als Verweise auf eine oder mehrere \glsIdxPl{Aussage} $A_j$,}\\
%%%	& \quad\;\; \text{welche zusammen genau die \glsIdxPl{Voraussetzung} der \glsIdx{Schnittregel} } \overline{R}_i \text{ erfüllen.}
%%%	\\
%%%	A_i & \defeq \text{\glsIdx{Folgerung}(en) der \glsIdx{Schlussregel} $\overline{R}_i$ --}\\
%%%	& \quad\;\; \text{auch in Form der Indizes von einem oder mehreren von $Aj$ (mit $j < i$).}\\
%%%	& \quad\;\; \text{In der Ergebniszeile kann hier auch die bewiesene \glsIdx{Aussage} als Schlussregel stehen.}
%%%	\\
%%%	D_i & \defeq \text{die Indizes der $A_j$, von denen $A_i$ abhängig ist.}
%%%\end{align}
%%%Bis zur Zeile $i$ hat man die folgende \glsIdx{Schlussregel} bewiesen:
%%%\[ \frac{A_{i_1} \srand A_{i_2} ...}{A_i} \quad \text{, für alle } i_j \in D_i \]
%%%Sei nun
%%%\[
%%%	\Gamma_i \defeq
%%%	\left\{
%%%		\begin{array}{ll}
%%%			\text{leer}    & \text{ für } R_i = \text{\enquote{\glsIdx{Voraussetzung}}} \\
%%%			\text{leer}    & \text{ für } R_i = \text{\enquote{\glsIdl{Folgerung}}}     \\
%%%			\text{leer}    & \text{ für } R_i = \text{\enquote{Annahme}}       \\
%%%			\overline{R_j} & \text{ für } R_i = j \quad \text{(eine \emph{interne} \glsIdx{Schlussregel})} \\
%%%			\text{die \glsIdx{Schlussregel}} & \text{ für } R_i = \text{Verweis auf eine \emph{externe} \glsIdx{Schlussregel}}
%%%		\end{array}
%%%	\right.
%%%\]
%%%Damit gilt für die Einträge in einer Zeile $i$:
%%%\begin{itemize}
%%%	\item Wenn $\Gamma_i$ nicht leer ist, ist $R_i$ eine \glsIdx{Schlussregel} mit $R_i = \Gamma_i(S_i)$%
%%%	\footnote{%
%%%		siehe Definition~\eqref{def:Substitutionen} \vrefvonsub{sub:Indentitaetsregeln}%
%%%	}.
%%%	\item Wenn $A_i$ nicht leer ist, ist $R_i = \dfrac{A_{z_1} \srand A_{z_2} \srand ...}{A_i}$ (alle $z_j \in Z_i$).
%%%	\item Wenn $A_i$ nicht leer ist, ist bis jetzt die \glsIdx{Schlussregel} $\dfrac{A_{d_1} \srand A_{d_2} \srand ...}{A_i}$ (alle $d_j \in D_i$) schon bewiesen.
%%%\end{itemize}
%%%$S_i$, $Z_i$ und $D_i$ dürfen dabei auch leer sein.
%%%
%%%\begin{table}[!htb]
%%%	\setlength\tabcolsep{1pt}
%%%	\setlength\extrarowheight{7pt}
%%%	\newcommand*{\centerParbox}[2]{\parbox{#1}{\centering #2}}
%%%	\newcommand*{\titleCell}[3]{\centerParbox{#1}{\textbf{#2} (#3)}}
%%%	\newcommand*{\SnCell}[1]{\centerParbox{1.85cm}{#1}}
%%%	\newcommand*{\DnCell}[1]{\centerParbox{1.95cm}{#1}}
%%%	\begin{tabular}{|c||c|c|c|c|c|c|}
%%%		\hline
%%%		\titleCell{0.95cm}{Zeile}                       {$n$} &
%%%		\titleCell{1.05cm}{Regel}                     {$R_n$} &
%%%		\titleCell{1.85cm}{Substitu"=tionen}          {$S_n$} &
%%%		\titleCell{1.80cm}{erzeugte Regel} {$\overline{R}_n$} &
%%%		\titleCell{2.15cm}{angewendet auf ...}        {$Z_n$} &
%%%		\titleCell{1.65cm}{\glsIdx{Aussage}}          {$A_n$} &
%%%		\titleCell{1.95cm}{Abhängig"=keiten}          {$D_n$}
%%%		\\\hline\hline
%%%		1 & \centerParbox{1.35cm}{Voraus"=setzung} & & & & $X \derive \alpha$ & 1
%%%		\\\hline
%%%		2 & \centerParbox{1.35cm}{Voraus"=setzung} & & & & $X,\alpha \derive \beta$ & 2
%%%		\\\hline
%%%		3 & \centerParbox{1.00cm}{Folge"=rung} & & & & $X \derive \beta$ & 3
%%%		\\\hline
%%%		4 & \tagMR & & $\dfrac{X \derive \alpha}{X, Y \derive \alpha}$ & & &
%%%		\\\hline
%%%		5 & 4 & $Y \subst \lnot\alpha$ & $\dfrac{X \derive \alpha}{X, \lnot\alpha \derive \alpha}$ & 1 & $X, \lnot\alpha \derive \alpha$ & 1
%%%		\\\hline
%%%		6 & \tagAR & & $ \dfrac{}{\alpha \derive \alpha} $ & & &
%%%		\\\hline
%%%		7 & 6 & $\alpha \subst \lnot\alpha$ & $\dfrac{}{\lnot\alpha \derive \lnot\alpha}$ & & $\lnot\alpha \derive \lnot\alpha$ &
%%%		\\\hline
%%%		8 & 4 & \SnCell{%
%%%			$\alpha \subst \lnot\alpha$\\
%%%			$X \subst \lnot\alpha$\\
%%%			$Y \subst X$
%%%		} & $\dfrac{\lnot\alpha \derive \lnot\alpha}{X,\lnot\alpha \derive \lnot\alpha}$ & 7 & $X,\lnot\alpha \derive \lnot\alpha$ &
%%%		\\\hline
%%%		9 & \tagnota & & $\dfrac{X \derive \alpha, \lnot\alpha}{X \derive \beta}$ & & &
%%%		\\\hline
%%%		10 & 9 & $X \subst X, \lnot\alpha$ & $\dfrac{X,\lnot\alpha \derive \alpha, \lnot\alpha}{X,\lnot\alpha \derive \beta}$ & 5, 8 & $X,\lnot\alpha \derive \beta$ & 1
%%%		\\\hline
%%%		11 & \tagnotb & & $\dfrac{X,\alpha \derive \beta \srand X,\lnot\alpha \derive \beta}{X \derive \beta}$ & 2, 10 & 3 & 1, 2
%%%		\\\hline\hline
%%%		12 & \centerParbox{1.4cm}{\tagAR, \tagMR, \tagnota, \tagnotb} & & $\dfrac{A_1 \srand A_2}{A_3}$ & & $\dfrac{X \derive \alpha \srand X, \alpha \derive \beta}{X \derive \beta}$ &
%%%		\\\hline
%%%	\end{tabular}
%%%	\caption{Ableitung der \glsIdx{Schnittregel} aus den \glsIdxPl{Basisregel}}
%%%	\label{tab:AbleitungSchnittregel}
%%%\end{table}
%%%
%%%Die Erzeugung einer Tabelle analog zu~\vref{tab:AbleitungSchnittregel} wird im folgenden beschrieben.
%%%Zellen, für die kein Inhalt angegeben wird, bleiben leer.
%%%Rückwärts-Referenzen auf schon ausgefüllte Zellinhalte sind jederzeit möglich.
%%%Das Eintragen der Zeilennummer $i$ wird nicht extra erwähnt.
%%%-- Die Tabelle und die Beschreibung sind so ausführlich, damit man daraus leicht ein Computerprogramm erstellen kann.
%%%%
%%%\begin{enumerate}
%%%	%
%%%	\item Am Anfang der Tabelle werden zuerst \glsIdxPl{Voraussetzung}, dann zu beweisende \glsIdxPl{Folgerung} und schließlich Annahmen aufgeführt.%
%%%	\footnote{%
%%%		Die Angabe ist dann erforderlich, wenn darauf verwiesen wird.
%%%		Durch die Auflistung hat man aber einen vollständigen Überblick über die \glsIdxPl{Voraussetzung} und \glsIdxPl{Folgerung} eines \glsIdx{Beweis}es und die Zwischenannahmen.
%%%		Auf jede nötige \glsIdx{Voraussetzung} und jede verwendete Zwischenannahme wird in der Spalte $(Z_n$) mindestens einmal verwiesen, so dass sie auch aufgeführt werden müssen.
%%%		Die Angabe der \glsIdxPl{Folgerung} erleichtert die Erstellung einer \emph{Ergebniszeile} (siehe Punkt~\ref{item:Ergebniszeile}).
%%%	}
%%%	Jede der drei Gruppen kann auch leer sein und es ist auch möglich, die Zeilen an anderen Stellen der Tabelle anzugeben, spätestens aber, wenn darauf verwiesen wird.
%%%	Für jede \glsIdx{Voraussetzung}, \glsIdl{Folgerung} und Annahme gibt es eine Zeile:
%%%	\begin{enumerate}
%%%		\item $R_i =$ \enquote{\glsIdx{Voraussetzung}}, \enquote{\glsIdl{Folgerung}} oder \enquote{Annahme}.
%%%		\item $A_i =$ Die aktuelle \glsIdx{Voraussetzung}, \glsIdl{Folgerung} oder Annahme.
%%%		\item $D_i =$ $i$ \quad (ein Verweis auf $A_i$).
%%%	\end{enumerate}
%%%	%
%%%	\item In den nächsten Zeilen werden die \glsIdxPl{Beweisschritt} aufgeführt, für jeden Schritt eine Zeile.
%%%
%%%	Zunächst kann $R_i$ kann auf zwei Arten erzeugt werden:
%%%	\begin{enumerate}
%%%		\setcounter{enumii}{\value{Enumii}}% Nummerierung wird fortgesetzt.
%%%		\item
%%%		\begin{enumerate}
%%%			\item $R_i =$ Verweis auf eine \glsIdx{allgemeingueltigeSchlussregelV}.
%%%			\item $\overline{R}_i =$ Die \glsIdx{Schlussregel}, auf die verwiesen wird.
%%%		\end{enumerate}
%%%		\setcounter{Enumii}{\value{enumii}}% Nummerierung wird fortgesetzt.
%%%	\end{enumerate}
%%%	oder
%%%	\begin{enumerate}
%%%		\item
%%%		\begin{enumerate}
%%%			\item $R_i = j$, wenn die schon bewiesene \glsIdx{Schlussregel} $\overline{R}_j$ (mit $j < i$) angewendet werden soll.
%%%			\item $S_i =$ Die auf die \glsIdx{Schlussregel} $R_i$ anzuwendende \glsIdx{Substitution}.
%%%			\item $\overline{R}_i =$ Das Ergebnis der \glsIdx{Substitution} $S_i$ auf die \glsIdx{Schlussregel} $R_i$.
%%%		\end{enumerate}
%%%		\setcounter{Enumii}{\value{enumii}}% Nummerierung wird fortgesetzt.
%%%	\end{enumerate}
%%%	Man beachte, dass die \glsIdx{Schlussregel} $\overline{R}_i$, stets allgemeingültig ist, da sie ausschließlich aus \glsIdxBg{allgemeingueltigeSchlussregelV}{allgemeingültigen Schlussregeln} mittels \glsIdxPl{Substitution} abgeleitet worden ist.
%%%	Daher gibt es auch keine Beschränkung weiterer \glsIdxPl{Substitution} durch irgendwelche Abhängigkeiten.
%%%
%%%	Nun kann die Zeile beendet werden, oder es geht weiter mit:
%%%	\begin{enumerate}
%%%		\setcounter{enumii}{\value{Enumii}}% Nummerierung wird fortgesetzt.
%%%		\item \label{item:Anwendung} $Z_n =$ Die Indizes aller $A_j$ (mit $j < i$), die eine \glsIdx{Voraussetzung} der \glsIdx{Schlussregel} $\overline{R}_i$ sind, möglichst in der verwendeten Reihenfolge.
%%%		-- Für jedes angegebene $j$ werden noch die Abhängigkeiten $D_j$ den Abhängigkeiten $D_i$ hinzugefügt.
%%%		%
%%%		\item $A_i =$ \glsIdl{Folgerung}(en) der \glsIdx{Schlussregel} $\overline{R}_i$.
%%%		-- Wenn diese \glsIdxPl{Folgerung} schon als \glsIdxPl{Aussage} $A_j$ (mit $j < i$) vorhanden sind, können auch einfach deren Indizes eingetragen werden.
%%%		Damit werden die Zusammenhänge und der Abschluss des \glsIdx{Beweis}es besser ersichtlich.
%%%		%
%%%		\item $D_i =$ Die Verweise wurden schon in (\ref{item:Anwendung}) eingetragen.%
%%%		\footnote{Wenn $D_n$ leer ist, dann ist $A_n$ allgemeingültig.}
%%%		%
%%%	\end{enumerate}
%%%	Der \glsIdx{Beweis} muss so lange fortgeführt werden, bis alle \glsIdxPl{Folgerung} als \glsIdxPl{Aussage} in der Spalte $(A_n)$ erschienen und dort jeweils nur von den gegebenen \glsIdxPl{Voraussetzung} abhängig sind.
%%%	%
%%%	\item \label{item:Ergebniszeile} In einer \emph{Ergebniszeile}, die dann die letzte ist, kann noch die bewiesene Behauptung in Form einer \glsIdx{Schlussregel} formuliert und in einer passenden Spalte notiert werden.
%%%	Zusätzlich können dort auch noch alle verwendeten \glsIdxPl{Schlussregel} gesammelt werden.
%%%	Dies kann \textzB\ folgendermaßen geschehen:
%%%	\begin{enumerate}
%%%		%
%%%		\item $(R_n) =$ Verweise auf alle verwendeten externen \glsIdxPl{Schlussregel}.
%%%		%
%%%		\item $(\overline{R}_n) =$ Die bewiesene Behauptung als \glsIdxPl{Schlussregel}, wobei alle $A_i$, die \glsIdxPl{Voraussetzung} sind, als \glsIdx{Voraussetzung} und alle $A_j$, die \glsIdxPl{Folgerung} sind, als \glsIdl{Folgerung} eingesetzt werden, jeweils in der Form \enquote{$A_i$} \textbzgl\ \enquote{$A_j$}.
%%%		Das ergibt dann:
%%%		\[ \frac{A_{i_1} \srand A_{i_2} \srand ...}{A_{j_1} \srand A_{j_2} \srand ...} \]
%%%		%
%%%		\item $(A_n) =$ $\overline{R}_i$, wobei die \glsIdxPl{Voraussetzung} und \glsIdxPl{Folgerung} aufgelöst werden.
%%%		%
%%%		\item $(D_n) =$ Die Vereinigung aller Abhängigkeiten der \glsIdxPl{Folgerung}, vermindert um die \glsIdxPl{Voraussetzung}.
%%%		-- Wenn das Feld dabei nicht leer bleibt, ist der \glsIdx{Beweis} missglückt!
%%%		%
%%%	\end{enumerate}
%%%	%
%%%\end{enumerate}
%%%%
%%%Ein weiteres Beispiel \vrefintab{tab:AbleitungKontraposition} soll verdeutlichen, wie Abhängigkeiten von Zwischenannahmen wieder beseitigt werden können.\footnote{siehe~\cite{bib:NatuerlichesSchliessen}, Kapitel 2.2.4 \emph{Eine Beispielableitung}}
%%%
%%%\begin{table}[!htb]
%%%	\setlength\tabcolsep{1pt}
%%%	\setlength\extrarowheight{7pt}
%%%	\newcommand*{\centerParbox}[2]{\parbox{#1}{\centering #2}}
%%%	\newcommand*{\titleCell}[3]{\centerParbox{#1}{\textbf{#2} (#3)}}
%%%	\newcommand*{\SnCell}[1]{\centerParbox{2.30cm}{#1}}
%%%	\newcommand*{\DnCell}[1]{\centerParbox{1.95cm}{#1}}
%%%	\begin{tabular}{|c||c|c|c|c|c|c|}
%%%		\hline
%%%		\titleCell{0.95cm}{Zeile}                       {$n$} &
%%%		\titleCell{1.05cm}{Regel}                     {$R_n$} &
%%%		\titleCell{1.85cm}{Substitu"=tionen}          {$S_n$} &
%%%		\titleCell{1.80cm}{erzeugte Regel} {$\overline{R}_n$} &
%%%		\titleCell{2.15cm}{angewendet auf ...}        {$Z_n$} &
%%%		\titleCell{1.65cm}{\glsIdx{Aussage}}          {$A_n$} &
%%%		\titleCell{1.95cm}{Abhängig"=keiten}          {$D_n$}
%%%		\\\hline \hline
%%%		1 & \centerParbox{1.00cm}{Folge"=rung} & & & & $(\alpha\limp\beta)\limp(\lnot\beta\limp\lnot\alpha)$ & 1
%%%		\\\hline
%%%		2 & \centerParbox{1.20cm}{An"=nahme} & & & & $\alpha\limp\beta$ & 2
%%%		\\\hline
%%%		3 & \centerParbox{1.20cm}{An"=nahme} & & & & $\lnot\beta$ & 3
%%%		\\\hline
%%%		4 & \centerParbox{1.20cm}{An"=nahme} & & & & $\alpha$ & 4
%%%		\\\hline
%%%		5 & \tagimpB & & $\dfrac{X \derive \alpha\limp\beta}{X,\alpha \derive \beta}$ & & &
%%%		\\\hline
%%%		6 & -1 & $X \subst \emptyset$ & $\dfrac{\alpha\limp\beta}{\alpha \derive \beta}$ & 2 & $\alpha \derive \beta $ & 2
%%%		\\\hline
%%%		7 & \tagSR & & $\dfrac{X \derive \alpha \srand X,\alpha \derive \beta}{X \derive \beta}$ & & &
%%%		\\\hline
%%%		8 & -1 & $X \subst \emptyset$ & $\dfrac{\alpha \srand \alpha \derive \beta}{\beta}$ & 4, 6 & $\beta $ & 4, 6
%%%		\\\hline
%%%		9' & \tagandE & & $\dfrac{X \derive \alpha, \beta}{X \derive \alpha \land \beta}$ & & &
%%%		\\\hline
%%%		10' & -1 & $X \subst \emptyset$ & $\dfrac{\alpha \srand \beta}{\alpha \land \beta}$ & & &
%%%		\\\hline
%%%		11' & -1 &\SnCell{
%%%			$\alpha \swap \beta$\\
%%%			$\alpha \subst \lnot\beta$
%%%		}  & $\dfrac{\beta \srand \lnot\beta}{\beta \land \lnot\beta}$ & 8, 3 & $\beta \land \lnot\beta$ &
%%%		\\\hline
%%%		9 & \tagnota & & $\dfrac{X \derive \alpha, \lnot\alpha}{X \derive \beta}$ & & &
%%%		\\\hline
%%%		10 & -1 & $X \subst \emptyset$ & $\dfrac{\alpha \srand \lnot\alpha}{\beta}$ & & &
%%%		\\\hline
%%%		11 & -1 & \SnCell{
%%%			$\alpha \swap \beta$\\
%%%			$\alpha \subst \lnot\alpha$
%%%		} & $\dfrac{\beta \srand \lnot\beta}{\lnot\alpha}$ & 8, 3 & $\lnot\alpha$ & 2, 3, 4
%%%		\\\hline
%%%		12 & \tagimpE & & $\dfrac{X, \alpha \derive \beta}{X \derive \alpha\limp\beta}$ & & &
%%%		\\\hline
%%%		13 & -1 & $X \subst \emptyset$ & $\dfrac{\alpha \derive \beta}{\alpha\limp\beta}$ & & &
%%%		\\\hline
%%%		14 & -1 & \SnCell{
%%%			$\alpha \swap \beta$\\
%%%			$\alpha \subst \lnot\alpha$\\
%%%			$\beta \subst \lnot\beta$
%%%		} & $\dfrac{\lnot\beta \derive \lnot\alpha}{\lnot\beta\limp\lnot\alpha}$ & 3, 11, ??? & $\lnot\beta\limp\lnot\alpha$ & 2, 3, 4, ???
%%%		\\\hline
%%%		15 & \tagimpE+1 & \SnCell{
%%%			$\alpha \subst \gamma$\\
%%%			$\beta \subst \delta$\\
%%%			$\gamma \subst \alpha\limp\beta$\\
%%%			$\delta \subst \lnot\beta\limp\lnot\alpha$
%%%		} & $\dfrac{\alpha\limp\beta \derive \lnot\beta\limp\lnot\alpha}
%%%		{(\alpha\limp\beta)\limp(\lnot\beta\limp\lnot\alpha)}$ & 2, 14 &
%%%		$(\alpha\limp\beta)\limp(\lnot\beta\limp\lnot\alpha)$ & 2, 3, 4, ???
%%%		\\\hline\hline
%%%		16 & \centerParbox{1.5cm}{\tagimpE, \tagimpB, \tagSR} & & $\dfrac{}{A_1}$ & & $\dfrac{}{(\alpha\limp\beta)\limp(\lnot\beta\limp\lnot\alpha)}$ &
%%%		\\\hline
%%%	\end{tabular}
%%%	\caption{Ableitung der \glsIdx{Kontraposition} aus \glsIdxBg{allgemeingueltigeSchlussregelV}{allgemeingültigen Schlussregeln}}
%%%	\label{tab:AbleitungKontraposition}
%%%\end{table}
%%%
%%%\todo{Beispielableitung der Kontraposition vervollständigen}%%%
%%%%TODO Beispielableitung der Kontraposition vervollständigen %%%

\section{Mengenlehre}% =========================================================
\beginsection{Mengenlehre}
\label{sec:Mengenlehre}

\todo{Mengenlehre bearbeiten.}%%%
%TODO Mengenlehre bearbeiten. %%%

\endgroup%  Bis hier einfache Quotierung von Objekten

\Endchapter
