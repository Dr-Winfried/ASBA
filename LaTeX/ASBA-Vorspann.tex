%%############################################################################%%
%%                                                                            %%
%% Datei:  ASBA-Vorspann.tex                                                  %%
%% Inhalt: Vorspann für ASBA                                                  %%
%%                                                                            %%
%% Copyright (C) 2017  Winfried Teschers                                      %%
%%                                                                            %%
%% This program is free software: you can redistribute it and/or modify       %%
%% it under the terms of the GNU Affero General Public License as published   %%
%% by the Free Software Foundation, either version 3 of the License, or       %%
%% (at your option) any later version.                                        %%
%%                                                                            %%
%% This program is distributed in the hope that it will be useful,            %%
%% but WITHOUT ANY WARRANTY; without even the implied warranty of             %%
%% MERCHANTABILITY or FITNESS FOR A PARTICULAR PURPOSE.  See the              %%
%% GNU Affero General Public License for more details.                        %%
%%                                                                            %%
%% You should have received a copy of the GNU Affero General Public License   %%
%% along with this program.  If not, see <http://www.gnu.org/licenses/>.      %%
%%                                                                            %%
%% Dr. Winfried Teschers                                                      %%
%% Anton-Günther-Straße 26c                                                   %%
%% 91083 Baiersdorf                                                           %%
%% Germany                                                                    %%
%%                                                                            %%
%% e-mail: winfried.teschers@t-online.de                                      %%
%%                                                                            %%
%%############################################################################%%

% !TeX root = ASBA.tex
% !TeX encoding = UTF-8
% !TeX spellcheck = de_DE

% Glossareinträge werden in "ASBA-Vorspann-Glossar" definiert.
% Elemente, die nur in "ASBA-Mathematik.tex" verwendet werden, werden in "ASBA-Vorspann-Mathematik.tex" definiert.

\documentclass[english, ngerman, parskip=half, headsepline, footsepline, fleqn, notitlepage]{scrreprt}

% Pakete #######################################################################

% Pakete aus "LaTeX 2e Befehlsübersicht" - <...> für Parameter -----------------
\usepackage[utf8]{inputenc}% Für direkte Eingabe von Umlauten mit [utf8]
\usepackage[T1]{fontenc}%    Darstellung von Umlauten mit [T1]
\usepackage[english,ngerman]{babel}% Neue deutsche Rechtschreibung mit [ngerman]
%\usepackage{multicol}%       Verwende <n> Spalten: \begin{multicols}{<n>}
\usepackage[fleqn]{amsmath}% Erweiterung für LaTeX-Mathe; fleqn=einrücken
\usepackage{amssymb}%        Zusätzliche Mathesymbole (bsp. IR)
\usepackage{graphicx}%       Bilder einbinden:\includegraphics[width=<x>]{build}
%\usepackage{url}%            URLs einfügen: \url{http://...}
\usepackage{textcomp}%       Zusätzliche Symbole z.B. \textmu
%\usepackage{upgreek}%        Aufrechte griechische Symbole
%\usepackage{wrapfig}%        Textumflossene Abbildungen
%\usepackage{subcaption}%     Teilabbildungen in \begin{figure} mit
%                            \begin{subfigure}[b]{<Weite>} und \hfill
%\usepackage{pdfpages}%       \includepdf[<pages=1-2>]{<Anhang.pdf>}
% weiter empfohlen ---------------------
\usepackage{lmodern}%        Ersetzt "Computer Modern" durch "Latin Modern"
\usepackage{mathtools}%      Mathematical tools to use with asmmath

% allgemein --------------------------------------------------------------------
\usepackage{scrlayer-scrpage}
\usepackage{geometry}%  Flexible and complete interface to document dimensions.
\usepackage{microtype}% Subliminal refinements towards typographical perfection.
\usepackage{pict2e}%    New implementation of picture commands.
\usepackage[autostyle]{csquotes}% Contex sensitive quotation facilities

% mathematische Pakete ---------------------------------------------------------
\usepackage{amsfonts}%       TeX fonts from the American Mathematical Society.
\usepackage{amsopn}%         Provides a command \DeclareMathOperator.
\usepackage{mathabx}%        Three series of mathematical symbols.
\usepackage{mathpazo}%       Fonts to typeset mathematics to match palatino.
%\usepackage{stix}%           OpenType Unicode maths fonts.
%%                            führt zu: To many symbol fonts declared
%\usepackage{cancel}%         Place lines through mathematical formulae.

% Tabellen ---------------------------------------------------------------------
\usepackage{float}%          An Improved Environment for Floats
\usepackage[table]{xcolor}%  Driver-independent color extensions - vor 'color'?
%\usepackage{ctable}%   Flexible typesetting of table and figure using key/value
% Das Paket 'ctable' fasst (mit anderer Syntax) die Eigenschaften der Pakete
\usepackage{array}%
\usepackage{tabularx}%       Erweiterung von tabular*
\usepackage{booktabs}%       Nicer layout of tables
% zusammen und lädt zusätzlich noch die Pakete
\usepackage{rotating}%       Rotating tools, including rotated full page floats
\usepackage{xspace}%         Behandelt Zwischenraum nach Makros
\usepackage{color}%          LaTeX support for color
\usepackage{xkeyval}%        Extension of the keyval package
% Ende der von 'ctable' geladenen Pakete
\usepackage{threeparttable}% Tables with captions and notes all the same width.
\usepackage{multirow}%       Create tabular cell spanning multiple rows.
\usepackage{diagbox}%        Table heads with diagonal lines.
\usepackage{arydshln}%       Draw dash-lines in array/tabular.
\usepackage{caption}%        Customizing captions in floating environments.

%% Nur für Entwicklung und Test ------------------------------------------------
\usepackage{srcltx}%            Klick im Viewer öffnet Editor und umgekehrt
\setlength{\overfullrule}{5pt}% Breite des schwarzen Balkens rechts von Overfull
%\usepackage{showidx}%           Index auf Seitenrand; nicht mit 'splitidx'\marginparwidth
%\setlength{\marginparwidth}{2cm}% Breite des Seitenrandes

% Verzeichnisse ----------------------------------------------------------------
%TODO 'minitoc' anwenden
\usepackage[germanb]{minitoc}%     Unterverzeichnisse
\usepackage[protected]{splitidx}%  mehrere Indizes - Nur statt 'makeidx'
%\usepackage{makeidx}%             Indexing - Entweder 'makeidx' oder 'splitidx'
%\usepackage{hvindex}%             Support for 'makeidx' - after 'babel

\usepackage{varioref}%             \vref...
\usepackage[colorlinks,destlabel,hyperindex,linkcolor=blue,pagebackref,unicode,
	pdftitle   ={ASBA - Axiome, Sätze, Beweise und Auswertungen},
	pdfauthor  ={Dr. Winfried Teschers},
	pdfsubject ={Projektdokument},
	pdfkeywords={Mathematik, automatisches Beweisen, Beweisunterstützung}
]{hyperref}
% Paket ' hyperref': Extensive support for hypertext in LaTeX
% colorlinks  - colors the links instead of using boxes
% destlabel   - verwendet key von \label{key} nach Start von Kapiteln u.a.
% hyperindex  - make items in the index by hyperlinked back to the text
% linkcolor   - Color for normal internal links
% pagebackref - inserts extra ‘back’ links into the bibliography for each entry
% unicode     - Unicode encoded PDF strings
% pdftitle    - Title    im PDF-Dokument
% pdfauthor   - Author   im PDF-Dokument
% pdfsubject  - Subject  im PDF-Dokument
% pdfkeywords - Keywords im PDF-Dokument
%TODO \autoref u.a. von 'hyperref' nutzen
%%%\addto\extrasngerman{% siehe manual for 'hyperref' Seite 17
%%%	\def\subsectionautorefname{\subsectionname}%
%%%	\def\subsubsectionautorefname{\subsubsectionname}%
%%%}

\usepackage[nopostdot,xindy]{glossaries}
%%%\usepackage[toc,index,nohypertypes={index},nopostdot,xindy]{glossaries}
% Paket 'glossaries': Create glossaries and lists of acronyms
% - nach 'hyperref'
% - läd 'glossaries-german'
% Optionen:         ('glossaries' erfordert 1. zusätzlichen Lauf von pdflatex)
% toc          - in den Inhalt (erfordert 2. zusätzlichen Lauf von pdflatex)
% index
% nohypertypes - ={index} -
% nopostdot    - Keinen zusätzlichen Punkt am Ende der Einträge.
% xindy        - Verwendung von 'xindy' für die Sortierung
% lädt {glossaries-german}% German language module for glossaries package
%\GlsSetQuote{+}% nur für 'makeindex' - siehe 'User Manual for glossaries' S. 31

% Einstellung von globalen Werten und Makro-Redefinitionen #####################

\geometry{textwidth=170mm,textheight=256mm,twoside}% optional Option 'showframe'
%TODO ausprobieren
%%%\geometry{showframe}

% Kopfzeilen ===================================================================
\newcommand*{\texthead}[1]{\textnormal{\textsf{\textbf{#1}}}}% Schriftart
\newcommand*{\Lehead}  [1]{\lehead{\texthead{#1}}}
\newcommand*{\Cehead}  [1]{\cehead{\texthead{#1}}}
\newcommand*{\Rehead}  [1]{\rehead{\texthead{#1}}}
\newcommand*{\Lohead}  [1]{\lohead{\texthead{#1}}}
\newcommand*{\Cohead}  [1]{\cohead{\texthead{#1}}}
\newcommand*{\Rohead}  [1]{\rohead{\texthead{#1}}}
\newcommand*{\Ohead}   [1]{\ohead {\texthead{#1}}}
\newcommand*{\Chead}   [1]{\chead {\texthead{#1}}}
\newcommand*{\Ihead}   [1]{\ihead {\texthead{#1}}}
\newcommand*{\Ofoot}   [1]{\ofoot {\textnormal{\textbf{#1}}}}
\newcommand*{\Cfoot}   [1]{\cfoot {\textnormal{#1}}}
\newcommand*{\Ifoot}   [1]{\ifoot {\textnormal{#1}}}
\newcommand*{\Pagestyle}{\pagestyle{scrheadings}}
\newcommand*{\Thispagestyle}{\thispagestyle{scrheadings}}

% Kopfzeilen mit 'scrlayer-scrpage'
%         \Lehead \Cehead \Rehead | \Lohead \Cohead \Rohead
% \Ohead: \Lehead                                   \Rohead
% \Chead:         \Cehead                   \Cohead
% \Ihead:                 \Rehead   \Lohead
% ASBA <Chapter-Überschrift> \Chaptername~\thechapter
%                            \sectionname~\thesection <Section-Überschrift> ASBA
%Initialisierung
\Ohead{ASBA}%          bleibt unverändert
\Chead{\contentsname}% wird laufend verändert
\Ihead{}%              wird laufend verändert

% Kapitel ======================================================================
\newcommand*{\Chaptername}{\chaptername}% wird mit 'Anhang' überschrieben
\newcommand*{\beginchapter}[2][\Chaptername~\thechapter]{% direkt nach \chapter
	\Chead{#2}%                         Kopfzeile Mitte = <Kapitelname>
	\Ihead{#1}%                         Kopfzeile Innen = Kapitel/Anhang <Nr.>
	\Pagestyle%                         veränderte Kopfzeile aktivieren
	\Thispagestyle%                     ... auch für diese Seite, da ...
}%                                      '\chapter' Kopf-/Fußzeilen deaktiviert
\newcommand*{\Endchapter}{% am Ende eines Kapitels
	\Thispagestyle% sicherheitshalber Kopfzeile für diese Seite aktivieren
}

% Abschnitte ===================================================================
\newcommand*{\beginsection}[1]{%        direkt nach \section
	\Cohead{#1}%                        oben rechts mittig = <Abschnittsname>
	\Lohead{\sectionname~\thesection}%  oben rechts innen = <Abschnittsnummer>
	\Pagestyle%                         veränderte Kopfzeile aktivieren
}

% Fußzeilen ====================================================================
\Ofoot{\thepage}
\Cfoot{Winfried Teschers}
\Ifoot{\today}
\Pagestyle%                                       aktiviert Kopf- und Fußzeilen

% Fußnoten ---------------------------------------------------------------------
\deffootnote[10pt]% Markenbreite
{10pt}% Einzug - für Blocksatz: Markenbreite
{0pt}% Absatzeinzug für Folgeabsätze
{\makebox[9pt][r]{\textsuperscript{\thefootnotemark)} }}% Zeichen; < Markenbreite
\deffootnotemark {\textsuperscript{\thefootnotemark)}}
\newcommand*{\Tnote}[1]{\tnote{#1)}}

% Vordefinierte Werte ändern ===================================================
\setcounter{tocdepth}{3}%    Tiefe des Inhaltsverzeichnisses: 2 => subsection
\setcounter{secnumdepth}{3}% Nummerierung:                    3 => subsubsection
\setlength\extrarowheight{1pt}% Tabellenzellenhöhe vergrößern
\captionsetup{labelfont=bf}%    Tabellenbeschriftung in bf = bold font

% Empfehlung aus: Herbert Voß, LaTeX Referenz, 3. Auflage, Berlin 2014; S. 37f
\renewcommand{\floatpagefraction}{0.7}% Empfehlung: 0.5-0.8 Voreinstellung: 0.9
\renewcommand{\textfraction}{0.15}%                 0.1-0.3                 0.05
\renewcommand{\topfraction}{0.8}%                   0.5-0.85                0.9
\renewcommand{\bottomfraction}{0.5}%                0.2-0.5                 0.9
\setcounter{topnumber}{3}%                                                  2
\setcounter{totalnumber}{15}%                                               3

% Neue Elemente ----------------------------------------------------------------
\newcounter{Enumi}% für unterbrochene Listennummerierung
\newcounter{Enumii}% für unterbrochene Listennummerierung
\newcounter{Enumiii}% für unterbrochene Listennummerierung

% Bildelemente #################################################################

\newcommand*{\textbild}[1]{\textbf{\textsf{#1}}}% Textauszeichnungen für Text im Bild
\newcommand*{\Datei}[4][0.5]{% #2 x #3 = (-#2/2,-#3/2),(#2/2,#3/2)
	% [Eck-Radius], Breite, Höhe, Name
	\put(0,0){\oval[#1](#2,#3)}
	\put(0,0){\makebox(0,0){\textbild{#4}}}
}
\newcommand*{\Datenbank}[5]{% 2(#1) x 2(#2+#3) = (-#1,-#2-#3),(+#1,+#2+#3)
	% Halbmesser x, Halbmesser y, halbe Höhe, Name - Ursprung in der Mitte
	\put(0.0,-#3){
		\qbezier(-#1,0.0)(-#1,-#2)(0.0,-#2)
		\qbezier(+#1,0.0)(+#1,-#2)(0.0,-#2)
	}
	\put(0,0){\Line(-#1,-#3)(-#1, #3)}
	\put(0,0){\Line( #1,-#3)( #1, #3)}
	\put(0.0,#3){
		\qbezier(-#1,0.0)(-#1,-#2)(0.0,-#2)
		\qbezier( #1,0.0)( #1,-#2)(0.0,-#2)
		\qbezier(-#1,0.0)(-#1, #2)(0.0, #2)
		\qbezier( #1,0.0)( #1, #2)(0.0, #2)
	}
	\makebox(0,0){\textbild{#4}}
	\makebox(0,-#3){\textbild{#5}}
}
% '\Männchen' (mit 'ä') führt zu Fehler
\newcommand*{\Maennchen}{% 1x2 = (-0.5,-1.7),(+0.5,+0.3) Ursprung im Kopf
	\put(0,0){\circle{0.6}}
	\Line(0.0,-0.3)(0.0,-1.2)
	\polyline(-0.5,-0.3)(0.0,-0.6)(0.5,-0.3)
	\polyline(-0.5,-1.7)(0.0,-1.2)(0.5,-1.7)
}
\newcommand*{\Marker}[2][0.5]{% 2x#1 x 2x#1 - Kreis mit Text
	{
		\linethickness{0.5pt}
		\color{white}
		\put(0,0){\circle*{#1}}
		\color{black}
		\put(0,0){\circle{#1}}
		\put(0,0){\makebox(0,0){\small\textbild{#2}}}
	}
}
\newcommand*{\marker}[2][0.5]{% 2x#1 x 2x#1 - Kreis mit Text - grau
	{
		\linethickness{0.5pt}
		\color{white}
		\put(0,0){\circle*{#1}}
		\color{gray}
		\put(0,0){\circle{#1}}
		\put(0,0){\makebox(0,0){\small\textbild{#2}}}
	}
}
\newcommand*{\Papier}[3]{% 3x#1+#2 = (0.0,-#2),(2.1,#1) Ursprung links unten
	% Länge (Höhe), Länge Abschluss, Name
	\polyline(0.0,-0.01)(+0.0,+#1)(+2.8,+#1)(+2.8,-0.01)
	\qbezier(+2.1,+#2)(+2.6,+#2)(+2.8,0.0)
	\qbezier(+2.1,+#2)(+1.6,+#2)(+1.4,0.0)
	\qbezier(+0.7,-#2)(+1.2,-#2)(+1.4,0.0)
	\qbezier(+0.7,-#2)(+0.2,-#2)(+0.0,0.0)
	\put(0,0){\makebox(+2.8,+#1){\textbild{#3}}}
}
\newcommand*{\Terminal}[1]{% 2x2 =(-1.0,-1.4),(+1.0,+0.6)Ursprung im Monitor
	% Bildschirm
	%		\put(0,0){\polygon(-1.0,-0.6)(+1.0,-0.6)(+1.0,+0.6)(-1.0,+0.6)}
	\put(-1.0,-0.6){\framebox(2,1.2){#1}}
	\put(0,0){\oval[0.1](1.65,0.85)}
	% Hals
	\put(0,0){\Line(-0.2,-0.6)(-0.2,-1.0)}
	\put(0,0){\Line(+0.2,-0.6)(+0.2,-1.0)}
	% Tastatur
	\multiput(-1.0,-1.0)(+0.0,-0.133){4}{\line(1,0){2.0}}
	\multiput(-1.0,-1,0)(+0.2,+0.0){11}{\line(0,-1){0.4}}
}
\newcommand*{\Wolke}[1]{% 3.0x1.5 = (-1.5,-1.0),(+1.5,+0.5)
	% unterer Bogen
	\qbezier(-1.5,+0.0)(-1.5,-1.0)(-0.0,-1.0)
	\qbezier(+1.5,+0.0)(+1.5,-1.0)(+0.0,-1.0)
	% oberer Bogen rechts
	\qbezier(+1.5,+0.0)(+1.5,+0.5)(+0.8,+0.5)
	\qbezier(+0.4,+0.4)(+0.4,+0.5)(+0.8,+0.5)
	% oberer Bogen Mitte
	\qbezier(+0.5,+0.2)(+0.5,+0.5)(+0.0,+0.5)
	\qbezier(-0.4,+0.4)(-0.4,+0.5)(-0.0,+0.5)
	% oberer Bogen links
	\qbezier(-0.3,+0.2)(-0.3,+0.5)(-0.8,+0.5)
	\qbezier(-1.5,+0.0)(-1.5,+0.5)(-0.8,+0.5)
	\put(-1.5,-1.0){\makebox(3.0,1.5){\textbild{#1}}}
}

% sonstige nützliche Kommandos #################################################

% Im Parameter von '\turl' muss vor jedem Zeichen aus "{}#&%$" ein '\' stehen
% und '\' und '~' durch '\textbackslash' bzw. '\textasctilde' ersetzt werden.
\newcommand*{\tourl}[1]{$\rightarrow$~\url{#1}}
\newcommand*{\formulatoleft}{&&&&&&&&&&}%  Um Formeln nach links zu komprimieren
\newcommand*{\formulaspace} {&&&&}      %  Für Platz zwischen den Formeln
\newcommand*{\todo}[1]{\textbf{>~>~>~#1~<~<~<}}% für TODOs
% Quotierung von Zeichen [chr], Zeichenfolgen [seq] und Zeichenketten [str]
% Parameter im Textmodus
%%%\newcommand*{\chrqt}[1]{\ensuremath{\langle}#1\ensuremath{\rangle}}
\newcommand*{\chrqt}[1]{\ensuremath{\langle\text{#1}\rangle}}% Zeichen/Symbol
\newcommand*{\seqqt}[1]{\ensuremath{\langle\!\langle\text{#1}\rangle\!\rangle}}% Zeichenfolge/Formel
\newcommand*{\charf}[1]{\textbf{\texttt{#1}}}%Schriftart für Zeichen
\newcommand*{\strqt}[1]       {``\charf{#1}''}%              Zeichenkette

% Strukturbezeichnungen ergänzen und Verweise auf Strukturen vereinfachen
\newcommand*{\sectionname}       {Abschnitt}
\newcommand*{\sectionnames}      {Abschnitte}
\newcommand*{\subsectionname}    {Unterabschnitt}
\newcommand*{\subsectionnames}   {Unterabschnitte}
\newcommand*{\subsubsectionname} {Paragraph}
\newcommand*{\subsubsectionnames}{Paragraphen}

%%%% Nominativ
%%%\newcommand*{\nomcha}        [1]{dieses       \chaptername~\ref{#1}}
%%%\newcommand*{\nomsec}        [1]{dieser       \sectionname~\ref{#1}}
%%%\newcommand*{\nomsub}        [1]{dieser    \subsectionname~\ref{#1}}
%%%\newcommand*{\nomsubsub}     [1]{dieser \subsubsectionname~\ref{#1}}

% Dativ
\newcommand*{\datcha}        [1]{diesem       \chaptername~\ref{#1}}
\newcommand*{\datsec}        [1]{diesem       \sectionname~\ref{#1}}
\newcommand*{\datsub}        [1]{diesem    \subsectionname~\ref{#1}}
\newcommand*{\datsubsub}     [1]{diesem \subsubsectionname~\ref{#1}}

% Akkusativ
\newcommand*{\akkcha}        [1]{dieses       \chaptername~\ref{#1}}
\newcommand*{\akksec}        [1]{diesen       \sectionname~\ref{#1}}
\newcommand*{\akksub}        [1]{diesen    \subsectionname~\ref{#1}}
\newcommand*{\akksubsub}     [1]{diesen \subsubsectionname~\ref{#1}}

% Referenzen

\newcommand*{\vreffig}       [1]             {\figurename~\vref{#1}}
\newcommand*{\vreftab}       [1]              {\tablename~\vref{#1}}
\newcommand*{\vrefcha}       [1]            {\chaptername~\vref{#1}}
\newcommand*{\vrefsec}       [1]            {\sectionname~\vref{#1}}
\newcommand*{\vrefsub}       [1]         {\subsectionname~\vref{#1}}
\newcommand*{\vrefsubsub}    [1]      {\subsubsectionname~\vref{#1}}
\newcommand*{\vrefdef}       [1]              {Definition~\vref{#1}}
\newcommand*{\vreffor}    [1]{\eqref{#1} auf \pagename~\pageref{#1}}
\newcommand*{\vrefziel}      [1]                    {Ziel~\vref{#1}}

\newcommand*{\vrefausfig}    [1]{aus der   \vreffig            {#1}}
\newcommand*{\vrefaustab}    [1]{aus der   \vreftab            {#1}}
\newcommand*{\vrefauscha}    [1]{aus       \vrefcha            {#1}}
\newcommand*{\vrefaussec}    [1]{aus       \vrefsec            {#1}}
\newcommand*{\vrefaussub}    [1]{aus       \vrefsub            {#1}}
\newcommand*{\vrefaussubsub} [1]{aus       \vrefsubsub         {#1}}

\newcommand*{\vrefinfig}     [1]{in  der   \vreffig            {#1}}
\newcommand*{\vrefintab}     [1]{in  der   \vreftab            {#1}}
\newcommand*{\vrefincha}     [1]{im        \vrefcha            {#1}}
\newcommand*{\vrefinsec}     [1]{im        \vrefsec            {#1}}
\newcommand*{\vrefinsub}     [1]{im        \vrefsub            {#1}}
\newcommand*{\vrefinsubsub}  [1]{im        \vrefsubsub         {#1}}

\newcommand*{\vrefInfig}     [1]{In  der   \vreffig            {#1}}
\newcommand*{\vrefIntab}     [1]{In  der   \vreftab            {#1}}
\newcommand*{\vrefIncha}     [1]{Im        \vrefcha            {#1}}
\newcommand*{\vrefInsec}     [1]{Im        \vrefsec            {#1}}
\newcommand*{\vrefInsub}     [1]{Im        \vrefsub            {#1}}
\newcommand*{\vrefInsubsub}  [1]{Im        \vrefsubsub         {#1}}

\newcommand*{\vrefvonfig}    [1]{von       \vreffig            {#1}}
\newcommand*{\vrefvontab}    [1]{von       \vreftab            {#1}}
\newcommand*{\vrefvoncha}    [1]{von       \vrefcha            {#1}}
\newcommand*{\vrefvonsec}    [1]{von       \vrefsec            {#1}}
\newcommand*{\vrefvonsub}    [1]{von       \vrefsub            {#1}}
\newcommand*{\vrefvonsubsub} [1]{von       \vrefsubsub         {#1}}

%%%\newcommand*{\vrefdfig}      [1]{die       \vreffig            {#1}}
%%%\newcommand*{\vrefdtab}      [1]{die       \vreftab            {#1}}
%%%\newcommand*{\vrefdcha}      [1]{das       \vrefcha            {#1}}
%%%\newcommand*{\vrefdsec}      [1]{der       \vrefsec            {#1}}
%%%\newcommand*{\vrefdsub}      [1]{der       \vrefsub            {#1}}
%%%\newcommand*{\vrefdsubsub}   [1]{der       \vrefsubsub         {#1}}
\newcommand*{\vrefDfig}      [1]{Die       \vreffig            {#1}}
\newcommand*{\vrefDtab}      [1]{Die       \vreftab            {#1}}
\newcommand*{\vrefDcha}      [1]{Das       \vrefcha            {#1}}
\newcommand*{\vrefDsec}      [1]{Der       \vrefsec            {#1}}
\newcommand*{\vrefDsub}      [1]{Der       \vrefsub            {#1}}
\newcommand*{\vrefDsubsub}   [1]{Der       \vrefsubsub         {#1}}

\newcommand*{\vrefseefig}    [1]{siehe     \vreffig            {#1}}
\newcommand*{\vrefseetab}    [1]{siehe     \vreftab            {#1}}
\newcommand*{\vrefseecha}    [1]{siehe     \vrefcha            {#1}}
\newcommand*{\vrefseesec}    [1]{siehe     \vrefsec            {#1}}
\newcommand*{\vrefseesub}    [1]{siehe     \vrefsub            {#1}}
\newcommand*{\vrefseesubsub} [1]{siehe     \vrefsubsub         {#1}}
\newcommand*{\vrefseedef}    [1]{siehe     \vrefdef            {#1}}
\newcommand*{\vrefseefor}    [1]{siehe     \vreffor            {#1}}
\newcommand*{\vrefseeziel}   [1]{siehe     \vrefziel           {#1}}

\newcommand*{\vrefSeefig}    [1]{Siehe     \vreffig            {#1}}
\newcommand*{\vrefSeetab}    [1]{Siehe     \vreftab            {#1}}
\newcommand*{\vrefSeecha}    [1]{Siehe     \vrefcha            {#1}}
\newcommand*{\vrefSeesec}    [1]{Siehe     \vrefsec            {#1}}
\newcommand*{\vrefSeesub}    [1]{Siehe     \vrefsub            {#1}}
\newcommand*{\vrefSeesubsub} [1]{Siehe     \vrefsubsub         {#1}}
\newcommand*{\vrefSeedef}    [1]{Siehe     \vrefdef            {#1}}
\newcommand*{\vrefSeefor}    [1]{Siehe     \vreffor            {#1}}
\newcommand*{\vrefSeeziel}   [1]{Siehe     \vrefziel           {#1}}

\newcommand*{\vrefnotefig}   [1]{\footnote{\vrefseefig         {#1}}}
\newcommand*{\vrefnotetab}   [1]{\footnote{\vrefseetab         {#1}}}
\newcommand*{\vrefnotecha}   [1]{\footnote{\vrefseecha         {#1}}}
\newcommand*{\vrefnotesec}   [1]{\footnote{\vrefseesec         {#1}}}
\newcommand*{\vrefnotesub}   [1]{\footnote{\vrefseesub         {#1}}}
\newcommand*{\vrefnotesubsub}[1]{\footnote{\vrefseesubsub      {#1}}}
\newcommand*{\vrefnotedef}   [1]{\footnote{\vrefseedef         {#1}}}
\newcommand*{\vrefnotefor}   [1]{\footnote{\vrefseefor         {#1}}}
\newcommand*{\vrefnoteziel}  [1]{\footnote{\vrefseeziel        {#1}}}

\newcommand*{\citesee}       [1]{\seename~\cite                {#1}}
\newcommand*{\citenote}      [1]{\footnote{\citesee            {#1}}}
\newcommand*{\alternativ}   [1]{\footnote{alternativ: \defn{#1}}}
\newcommand*{\alternativen} [2]{\footnote{alternativ: \defn{#1} oder \defn{#2}}}

% Abkürzungen mit Punkten; zur Unterscheidung vom Satzende
\newcommand*{\textbzgl}{bzgl.\@}
\newcommand*{\textbzw} {bzw.\@}
\newcommand*{\textdh}  {d.\@\,h.\@}
\newcommand*{\textDh}  {D.\@\,h.\@}
\newcommand*{\textevtl}{evtl.\@}
\newcommand*{\textggf} {ggf.\@}
\newcommand*{\textGgf} {Ggf.\@}
\newcommand*{\textiAlg}{i.\@\,Alg.\@}
\newcommand*{\textIAlg}{I.\@\,Alg.\@}
\newcommand*{\textua}  {u.\@\,a.\@}
\newcommand*{\textUa}  {U.\@\,a.\@}
\newcommand*{\textusw} {usw.\@}
\newcommand*{\textzB}  {z.\@\,B.\@}
\newcommand*{\textZB}  {Z.\@\,B.\@}
% Weitere Abkürzungen
\newcommand*{\textdots}{…}

% Verzeichnisse ################################################################

% Indices und Symbole
\makeindex
\newindex[Symbolverzeichnis]{sym}
\newindex[Index]{idx}

% Glossareinträge
\makeglossaries
\setacronymstyle{long-sc-short}

% Hervorhebung von neu definierten Begriffen
\newcommand*{\defn}      [1]{\textbf{#1}}
\newcommand*{\textdef}   [1]{\textbf{\textit{\hyperpage{#1}}}}% ein 'Font-Kommando'
% Die folgenden #1 müssen mit einem Makro enden, das genau einen optionalen
% Parameter hat, der ein 'Font-Kommando' wie z.B. '\textdef'sein muss.
% Das sind i.Alg. die vor '\...newglossaryentry' definierten Makros.
\newcommand*{\definition}   [1]  {\defn{#1[textdef]}}
\newcommand*{\undefinition} [1]  {\defn{#1[textdef]}\;}%unär: folgender  Abstand
\newcommand*{\bindefinition}[1]{\;\defn{#1[textdef]}\;}%binär:umgebender Abstand
