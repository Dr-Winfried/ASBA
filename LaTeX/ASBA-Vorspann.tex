%%############################################################################%%
%%                                                                            %%
%% Datei:  ASBA-Vorspann.tex                                                  %%
%% Inhalt: Vorspann für die Datei ASBA.txt                                    %%
%%                                                                            %%
%% Copyright (C) 2017  Winfried Teschers                                      %%
%%                                                                            %%
%% This program is free software: you can redistribute it and/or modify       %%
%% it under the terms of the GNU Affero General Public License as published   %%
%% by the Free Software Foundation, either version 3 of the License, or       %%
%% (at your option) any later version.                                        %%
%%                                                                            %%
%% This program is distributed in the hope that it will be useful,            %%
%% but WITHOUT ANY WARRANTY; without even the implied warranty of             %%
%% MERCHANTABILITY or FITNESS FOR A PARTICULAR PURPOSE.  See the              %%
%% GNU Affero General Public License for more details.                        %%
%%                                                                            %%
%% You should have received a copy of the GNU Affero General Public License   %%
%% along with this program.  If not, see <http://www.gnu.org/licenses/>.      %%
%%                                                                            %%
%% Dr. Winfried Teschers                                                      %%
%% Anton-Günther-Straße 26c                                                   %%
%% 91083 Baiersdorf                                                           %%
%% Germany                                                                    %%
%%                                                                            %%
%% e-mail: winfried.teschers@t-online.de                                      %%
%%                                                                            %%
%%############################################################################%%

% !TeX root = ASBA.tex
% !TeX encoding = UTF-8
% !TeX spellcheck = de_DE

\documentclass[english, ngerman, parskip=half, headsepline, footsepline, fleqn, notitlepage]{scrreprt}

% Pakete #######################################################################

% allgemein --------------------------------------------------------------------
\usepackage[utf8]{inputenc}% Input encoding specification
\usepackage[T1]{fontenc}
\usepackage{lmodern}
\usepackage{scrlayer-scrpage}
\usepackage{geometry}% Flexible and complete interface to document dimensions.
\usepackage{microtype}% Subliminal refinements towards typographical perfection.
\usepackage{graphicx}% Alternative interface to graphics functions.
\usepackage{pict2e}% New implementation of picture commands.
\usepackage{multicol}% An environment for multicolumn output
\usepackage{babel}% Multilingual support for plain TeX or LaTeX.
\usepackage[autostyle]{csquotes}% Contex sensitive quotation facilities

% mathematische Pakete ---------------------------------------------------------
\usepackage{amsmath}% Mathematical facilities for LaLeX from ASM
\usepackage{amsfonts}% TeX fonts from the American Mathematical Society.
\usepackage{amssymb}% Symbols from the American Mathematical Society.
\usepackage{mathtools}% Mathematical tools to use with asmmath.
\usepackage{mathabx}% Three series of mathematical symbols.
\usepackage{mathpazo}% Fonts to typeset mathematics to match palatino.
%\usepackage{cancel}% Place lines through mathematical formulae.

% Tabellen ---------------------------------------------------------------------
\usepackage[table]{xcolor}% Driver-independent color extensions - vor 'color'?
%\usepackage{ctable}% Flexible typesetting of table and figure using key/value
%% Das Paket ctable fast die Eigenschaften der Pakete
\usepackage{array}%
\usepackage{tabularx}% Erweiterung von tabular*
\usepackage{booktabs}% Nicer layout of tables
%% zusammen und lädt zusätzlich noch die Pakete
\usepackage{rotating}% Rotating tools, including rotated full page floats
\usepackage{xspace}% Behandelt Zwischenraum nach Makros
\usepackage{color}% LaTeX support for color
\usepackage{xkeyval}% Extension of the keyval package
% Ende der von 'ctable' geladenen Pakete
\usepackage{threeparttable}% Tables with captions and notes all the same width.
\usepackage{multirow}% Create tabular cell spanning multiple rows.
\usepackage{diagbox}% Table heads with diagonal lines.
\usepackage{arydshln}% Draw dash-lines in array/tabular.
\usepackage{caption}% Customizing captions in floating environments.

% Indizes ----------------------------------------------------------------------
%\usepackage{makeidx}% Indexing - Entweder 'makeidx' oder 'splitidx'
\usepackage[protected]{splitidx}% mehrere Indizes - statt 'makeidx'
%\usepackage{hvindex}% Support for indexing - after 'babel
%\usepackage{showidx}% Index auf Seitenrand anzeigen - Zum Testen der Indizes
%TODO Fehler: 'showindex' gibt direkt und nicht auf Rand aus
\usepackage{glossaries}% Create glossaries and lists of acronyms
%TODO Fehler: 'hyperfirst' hat keine Wirkung'.
%\usepackage{glossaries-german}% German language module for glossaries package

% Verweise ---------------------------------------------------------------------
%\usepackage[germanb]{minitoc}\dosectoc% Unterverzeichnisse erstellen
%TODO Fehler: 'minitoc' funktioniert nicht
\usepackage{varioref}
\usepackage[colorlinks,linktoc=all]{hyperref}% Extensive support for hypertext
\usepackage{glossaries}% Create glossaries and lists of acronyms
% lädt     {glossaries-german}% German language module for glossaries package

% Einstellung von globalen Werten und Makro-Redefinitionen #####################

\geometry{textwidth=170mm,textheight=256mm,twoside}% optional Option 'showframe'

% Kopfzeilen ===================================================================
\newcommand*{\texthead}[1]{\textnormal{\textsf{\textbf{#1}}}}% Schriftart
\newcommand*{\Lehead}[1]{\lehead{\texthead{#1}}}
\newcommand*{\Cehead}[1]{\cehead{\texthead{#1}}}
\newcommand*{\Rehead}[1]{\rehead{\texthead{#1}}}
\newcommand*{\Lohead}[1]{\lohead{\texthead{#1}}}
\newcommand*{\Cohead}[1]{\cohead{\texthead{#1}}}
\newcommand*{\Rohead}[1]{\rohead{\texthead{#1}}}
\newcommand*{\Ohead}[1]{\ohead{\texthead{#1}}}
\newcommand*{\Chead}[1]{\chead{\texthead{#1}}}
\newcommand*{\Ihead}[1]{\ihead{\texthead{#1}}}
\newcommand*{\Ofoot}[1]{\ofoot{\textnormal{\textbf{#1}}}}
\newcommand*{\Cfoot}[1]{\cfoot{\textnormal{#1}}}
\newcommand*{\Ifoot}[1]{\ifoot{\textnormal{#1}}}
\newcommand*{\Pagestyle}{\pagestyle{scrheadings}}
\newcommand*{\Thispagestyle}{\thispagestyle{scrheadings}}

% Kopfzeilen mit 'scrlayer-scrpage'
%         \Lehead \Cehead \Rehead | \Lohead \Cohead \Rohead
% \Ohead: \Lehead                                   \Rohead
% \Chead:         \Cehead                   \Cohead
% \Ihead:                 \Rehead   \Lohead
% ASBA <Chapter-Überschrift> \Chaptername~\thechapter
%                            \sectionname~\thesection <Section-Überschrift> ASBA
%Initialisierung
\Ohead{ASBA}%               bleibt unverändert
\Chead{Inhaltsverzeichnis}% wird laufend verändert
\Ihead{}%                   wird laufend verändert

% Kapitel ======================================================================
\newcommand*{\Chaptername}{\chaptername}% wird mit 'Anhang' überschrieben

\newcommand*{\beforechapter}{% direkt vor \chapter (auch im Kommentar)
	\Thispagestyle%        Kopfzeile für diese Seite aktivieren - vor \clearpage
	\clearpage%            neue Seite
}
\newcommand*{\beginchapter}[1]{%        direkt nach \chapter
	\Chead{#1}%                         Kopfzeile Mitte = <Kapitelname>
	\Ihead{\Chaptername~\thechapter}%   Kopfzeile Innen = Kapitel/Anhang <Nr.>
	\Pagestyle%                         veränderte Kopfzeile aktivieren
	\Thispagestyle%                     ... auch für diese Seite, da ...
}%                                      '\chapter' Kopf-/Fußzeilen deaktiviert
\newcommand*{\likechapter}[2][chapter]{%statt \beginchapter für Inhalts-,
	%                                   Tabellen- und Abbildungsverzeichnis
	\Chead{#2}%                         Mitte in der Kopfzeile = <Kapitelname>
	\Ihead{}%                           Innen in der Kopfzeile = <leer>
	\Pagestyle%                         veränderte Kopfzeile aktivieren ...
	\Thispagestyle%                     sicherheitshalber auch für diese Seite.
	\addcontentsline{toc}{#1}{#2}%      Eintrag ins Inhaltsverzeichnis
	%TODO Fehler: 2. Seite Kopf Mitte für Literaturverzeichnis = Abbildungsverzeichnis
}
\newcommand*{\Endchapter}{% am Ende eines Kapitels
	\Thispagestyle% sicherheitshalber Kopfzeile für diese Seite aktivieren
}
% Indizes ======================================================================
\newcommand*{\idxdictionary}[1]{%       nur für Indices
	\Thispagestyle%       Kopfzeile für diese Seite aktivieren - vor \clearpage
	\clearpage%                         neue Seite
	\extendtheindex{}{%                 aktiviert die Kopfzeile für Index-Seiten
		\Chead{#1}%                     Mitte in der Kopfzeile = <Kapitelname>
		\Ihead{}%                       Innen in der Kopfzeile = <leer>
		\Pagestyle%                     veränderte Kopfzeile aktivieren
		\Thispagestyle%                 ... auch für diese Seite
		\addcontentsline{toc}{section}{#1}% Eintrag ins Inhaltsverzeichnis
	}{}{}
}
\newcommand*{\glodictionary}[1]{%       nur für Glossary
	\Chead{#1}%                         Mitte in der Kopfzeile = <Kapitelname>
	\Ihead{}%                           Innen in der Kopfzeile = <leer>
	\Pagestyle%                         veränderte Kopfzeile aktivieren
}

% Abschnitte ===================================================================
\newcommand*{\beginsection}[1]{%        direkt nach \section
	\Cohead{#1}%                        oben rechts mittig = <Abschnittsname>
	\Lohead{\sectionname~\thesection}%  oben rechts innen = <Abschnittsnummer>
	\Pagestyle%                         veränderte Kopfzeile aktivieren
}

% Fußzeilen ====================================================================
\Ofoot{\thepage}
\Cfoot{Winfried Teschers}
\Ifoot{\today}
\Pagestyle%                                       aktiviert Kopf- und Fußzeilen

% Fußnoten ---------------------------------------------------------------------
\deffootnote[10pt]% Markenbreite
{10pt}% Einzug - für Blocksatz: Markenbreite
{0pt}% Absatzeinzug für Folgeabsätze
{\makebox[9pt][r]{\textsuperscript{\thefootnotemark)} }}% Zeichen; < Markenbreite
\deffootnotemark {\textsuperscript{\thefootnotemark)}}

% Vordefinierte Werte ändern ===================================================
\setcounter{tocdepth}{3}%    Tiefe des Inhaltsverzeichnisses: 2 => subsection
\setcounter{secnumdepth}{3}% Nummerierung:                    3 => subsubsection
\setlength\extrarowheight{1pt}% Tabellenzellenhöhe vergrößern
\captionsetup{labelfont=bf}%    Tabellenbeschriftung in bf = bold font

% Empfehlung aus: Herbert Voß, LaTeX Referenz, 3. Auflage, Berlin 2014; S. 37f
\renewcommand{\floatpagefraction}{0.7}% Empfehlung: 0.5-0.8 Voreinstellung: 0.9
\renewcommand{\textfraction}{0.15}%                 0.1-0.3                 0.05
\renewcommand{\topfraction}{0.8}%                   0.5-0.85                0.9
\renewcommand{\bottomfraction}{0.5}%                0.2-0.5                 0.9
\setcounter{topnumber}{3}%                                                  2
\setcounter{totalnumber}{15}%                                               3

% Neue Elemente ----------------------------------------------------------------
\newcounter{Enumi}% für unterbrochene Listennummerierung

% Bildelemente #################################################################

\newcommand*{\textbild}[1]{\textbf{\textsf{#1}}}% Textauszeichnungen für Text im Bild
\newcommand*{\Datei}[4][0.5]{% #2 x #3 = (-#2/2,-#3/2),(#2/2,#3/2)
	% [Eck-Radius], Breite, Höhe, Name
	\put(0,0){\oval[#1](#2,#3)}
	\put(0,0){\makebox(0,0){\textbild{#4}}}
}
\newcommand*{\Datenbank}[5]{% 2(#1) x 2(#2+#3) = (-#1,-#2-#3),(+#1,+#2+#3)
	% Halbmesser x, Halbmesser y, halbe Höhe, Name - Ursprung in der Mitte
	\put(0.0,-#3){
		\qbezier(-#1,0.0)(-#1,-#2)(0.0,-#2)
		\qbezier(+#1,0.0)(+#1,-#2)(0.0,-#2)
	}
	\put(0,0){\Line(-#1,-#3)(-#1, #3)}
	\put(0,0){\Line( #1,-#3)( #1, #3)}
	\put(0.0,#3){
		\qbezier(-#1,0.0)(-#1,-#2)(0.0,-#2)
		\qbezier( #1,0.0)( #1,-#2)(0.0,-#2)
		\qbezier(-#1,0.0)(-#1, #2)(0.0, #2)
		\qbezier( #1,0.0)( #1, #2)(0.0, #2)
	}
	\makebox(0,0){\textbild{#4}}
	\makebox(0,-#3){\textbild{#5}}
}
% '\Männchen' (mit 'ä') führt zu Fehler
\newcommand*{\Maennchen}{% 1x2 = (-0.5,-1.7),(+0.5,+0.3) Ursprung im Kopf
	\put(0,0){\circle{0.6}}
	\Line(0.0,-0.3)(0.0,-1.2)
	\polyline(-0.5,-0.3)(0.0,-0.6)(0.5,-0.3)
	\polyline(-0.5,-1.7)(0.0,-1.2)(0.5,-1.7)
}
\newcommand*{\Marker}[2][0.5]{% 2x#1 x 2x#1 - Kreis mit Text
	{
		\linethickness{0.5pt}
		\color{white}
		\put(0,0){\circle*{#1}}
		\color{black}
		\put(0,0){\circle{#1}}
		\put(0,0){\makebox(0,0){\small\textbild{#2}}}
	}
}
\newcommand*{\marker}[2][0.5]{% 2x#1 x 2x#1 - Kreis mit Text - grau
	{
		\linethickness{0.5pt}
		\color{white}
		\put(0,0){\circle*{#1}}
		\color{gray}
		\put(0,0){\circle{#1}}
		\put(0,0){\makebox(0,0){\small\textbild{#2}}}
	}
}
\newcommand*{\Papier}[3]{% 3x#1+#2 = (0.0,-#2),(2.1,#1) Ursprung links unten
	% Länge (Höhe), Länge Abschluss, Name
	\polyline(0.0,-0.01)(+0.0,+#1)(+2.8,+#1)(+2.8,-0.01)
	\qbezier(+2.1,+#2)(+2.6,+#2)(+2.8,0.0)
	\qbezier(+2.1,+#2)(+1.6,+#2)(+1.4,0.0)
	\qbezier(+0.7,-#2)(+1.2,-#2)(+1.4,0.0)
	\qbezier(+0.7,-#2)(+0.2,-#2)(+0.0,0.0)
	\put(0,0){\makebox(+2.8,+#1){\textbild{#3}}}
}
\newcommand*{\Terminal}[1]{% 2x2 =(-1.0,-1.4),(+1.0,+0.6)Ursprung im Monitor
	% Bildschirm
	%		\put(0,0){\polygon(-1.0,-0.6)(+1.0,-0.6)(+1.0,+0.6)(-1.0,+0.6)}
	\put(-1.0,-0.6){\framebox(2,1.2){#1}}
	\put(0,0){\oval[0.1](1.65,0.85)}
	% Hals
	\put(0,0){\Line(-0.2,-0.6)(-0.2,-1.0)}
	\put(0,0){\Line(+0.2,-0.6)(+0.2,-1.0)}
	% Tastatur
	\multiput(-1.0,-1.0)(+0.0,-0.133){4}{\line(1,0){2.0}}
	\multiput(-1.0,-1,0)(+0.2,+0.0){11}{\line(0,-1){0.4}}
}
\newcommand*{\Wolke}[1]{% 3.0x1.5 = (-1.5,-1.0),(+1.5,+0.5)
	% unterer Bogen
	\qbezier(-1.5,+0.0)(-1.5,-1.0)(-0.0,-1.0)
	\qbezier(+1.5,+0.0)(+1.5,-1.0)(+0.0,-1.0)
	% oberer Bogen rechts
	\qbezier(+1.5,+0.0)(+1.5,+0.5)(+0.8,+0.5)
	\qbezier(+0.4,+0.4)(+0.4,+0.5)(+0.8,+0.5)
	% oberer Bogen Mitte
	\qbezier(+0.5,+0.2)(+0.5,+0.5)(+0.0,+0.5)
	\qbezier(-0.4,+0.4)(-0.4,+0.5)(-0.0,+0.5)
	% oberer Bogen links
	\qbezier(-0.3,+0.2)(-0.3,+0.5)(-0.8,+0.5)
	\qbezier(-1.5,+0.0)(-1.5,+0.5)(-0.8,+0.5)
	\put(-1.5,-1.0){\makebox(3.0,1.5){\textbild{#1}}}
}

% Metasprachliche Symbole ######################################################

\newcommand*{\metaund}{\&\&}%               Und-Symbol  für Texte
\newcommand*{\metaoder}{||}%                Oder-Symbol für Texte
\newcommand*{\metaand}{\;\metaund\;}%       Und-Symbol  für Formeln
\newcommand*{\metaor}{\;\metaoder\;}%       Oder-Symbol für Formeln
% Nur im Mathematikmodus!
\newcommand*{\metaimp}{\Rightarrow}%        aus ... folgt ...
\newcommand*{\metarep}{\Leftarrow}%         ... folgt aus ...
\newcommand*{\metaequiv}{\Leftrightarrow}%  ... genau dann wenn ...
\newcommand*{\metadefeq}{:\Leftrightarrow}% ... definitionsgemäß genau dann wenn
\newcommand*{\defeq}{:=}%                   ... definitionsgemäß gleich ...
\newcommand*{\eq}{=}%                       ... gleich ...

% Mathematische Symbole ########################################################

% Beispieloperatoren ===========================================================
% \*bsp
\newcommand*{\opbsp}{\circ}
\newcommand*{\relbsp}{\sim}
\newcommand*{\releqbsp}{\simeq}
\newcommand*{\lrelbsp}{\lhd}
\newcommand*{\rrelbsp}{\rhd}
\newcommand*{\lreleqbsp}{\unlhd}
\newcommand*{\rreleqbsp}{\unrhd}

% Definitionen für die Tabelle der Junktoren ===================================
% \l*  -           logischer Operator
% \ln* - negierter logischer Operator
% Logische Operatoren als Addition und Multiplikation
\newcommand*{\ladd}{+}
\newcommand*{\lmult}{\cdot}
% Wahrheitswerte ---------------------------------------------------------------
\newcommand*{\texttrue}{W}%  in einem Kommentar stets 'W'
\newcommand*{\textfalse}{F}% in einem Kommentar stets 'F'
% Konstante --------------------------------------------------------------------
\newcommand*{\ltrue}{\top}%      W - wahr
%\newcommand*{\lnfalse}{\notbot}% " - nicht falsch
\newcommand*{\lfalse}{\bot}%     F - falsch
%\newcommand*{\lntrue}{\nottop}%  " - nicht wahr
% unäre Operatoren -------------------------------------------------------------
%                                                 W F - Aussage A
%\newcommand*{\lutrue}{\operatorname{\top}}%      W W - wahr [unär]
%\newcommand*{\lnufalse}{\operatorname{\notbot}}% " " - nicht falsch [unär]
%                                                 W F - A
%             \lnot                               F W - nicht
%\newcommand*{\lufalse}{\operatorname{\bot}}%     F F - falsch [unär]
%\newcommand*{\lnutrue}{\operatorname{\nottop}}%  " " - nicht wahr [unär]
% binäre Operatoren ------------------------------------------------------------
%                                                    W W F F - Aussage A
%                                                    W F W F - Aussage B
%  - - - - - - - - - - - - - - - - - - - - - - - - - - - - - - - - - - - - - - -
%\newcommand*{\lbtrue}{\operatorname{\top}}%         W W W W - wahr [binär]
%\newcommand*{\lnbfalse}{\operatorname{\notbot}}%    " " " " - nicht falsch
%            \lor                                    W W W F - A oder B
\newcommand*{\lrep}{\leftarrow}%                     W W F W - A folgt aus B
\newcommand*{\lrepA}{\Leftarrow}%
\newcommand*{\lrepB}{\subset}%
\newcommand*{\lleft}{\operatorname{\rfloor}}%        W W F F - A
%  - - - - - - - - - - - - - - - - - - - - - - - - - - - - - - - - - - - - - - -
\newcommand*{\limp}{\rightarrow}%                    W F W W - aus A folgt B
\newcommand*{\limpA}{\Rightarrow}%
\newcommand*{\limpB}{\supset}%
\newcommand*{\lright}{\operatorname{\lfloor}}%       W F W F - B
\newcommand*{\lequiv}{\leftrightarrow}%              W F F W - A genau dann,
\newcommand*{\lequivA}{\Leftrightarrow}%                       wenn B
%            \lnxor                                  " " " " - nicht
%                                                         (entweder A oder B)
%            \land                                   W F F F - A und B
\newcommand*{\landA}{\&}
\newcommand*{\landB}{\lmult}
%  - - - - - - - - - - - - - - - - - - - - - - - - - - - - - - - - - - - - -
\newcommand*{\lnand}{\uparrow}%                      F W W W - nicht
\newcommand*{\lnandA}{\barwedge}%                              (A und B)
\newcommand*{\lnandB}{\mid}%
\newcommand*{\lxor}{\ladd}%                          F W W F - entweder A
\newcommand*{\lxorA}{\operatorname{\dot\lor}}%                 oder B
\newcommand*{\lxorB}{\veebar}%
\newcommand*{\lxorC}{\oplus}%
\newcommand*{\lnequiv}{\nleftrightarrow}%            " " " " - nicht
\newcommand*{\lnequivA}{\nLeftrightarrow}%            (A genau dann, wenn B)
\newcommand*{\lnequivB}{\notequiv}%
\newcommand*{\lnright}{\lceil}%                      F W F W - nicht B
\newcommand*{\lnimp}{\nrightarrow}%                  F W F F - nicht
\newcommand*{\lnimpA}{\nRightarrow}%                         (aus A folgt B)
\newcommand*{\lnimpB}{\nsupset}%
%  - - - - - - - - - - - - - - - - - - - - - - - - - - - - - - - - - - - - -
\newcommand*{\lnleft}{\rceil}%                       F F W W - nicht A
\newcommand*{\lnrep}{\nleftarrow}%                   F F W F - nicht
\newcommand*{\lnrepA}{\nLeftarrow}%                          (A folgt aus B)
\newcommand*{\lnrepB}{\nsubset}%
\newcommand*{\lnor}{\downarrow}%                     F F F W - nicht
\newcommand*{\lnorA}{\operatorname{\overline\vee}}%            (A oder B)
%\newcommand*{\lbfalse}{\operatorname{\bot}}%        F F F F - falsch[binär]
%\newcommand*{\lnbtrue}{\operatorname{\nottop}}%     " " " " - nicht wahr

% Verwendete Mengenbezeichnungen ===============================================

% \gs* = globales Symbol
\newcommand*{\gsN}{\mathbb{N}}%    Menge der natürlichen Zahlen ohne 0
\newcommand*{\gsNo}{\mathbb{N}_0}% Menge der natürlichen Zahlen einschließlich 0

% \as* = aussagenlogisches Symbol
\newcommand*{\asA}{\mathcal{A}}%       Alphabet der Sprache
\newcommand*{\asAx}{\mathcal{A}_x}%    ... davon eine Teilmenge bzgl. \asJx
\newcommand*{\asAy}{\mathcal{A}_y}%    entsprechend \asAx
\newcommand*{\asB}{\mathcal{B}}%       Menge der binären Operatoren
\newcommand*{\asC}{\mathcal{C}}%       Menge der Konstanten
\newcommand*{\asF}{\mathcal{F}}%       Menge der Formeln
\newcommand*{\asFp}{\mathcal{F}^p}%    ... in polnischer Notation
\newcommand*{\asFx}{\mathcal{F}_x}%    Teilenge der Formeln
\newcommand*{\asFxp}{\mathcal{F}_x^p}% ... bzgl. \asJx in polnischer Notation
\newcommand*{\asFy}{\mathcal{F}_y}%    entsprechend \asFx
\newcommand*{\asFyp}{\mathcal{F}_y^p}% entsprechend \asFxp
\newcommand*{\asJ}{\mathcal{J}}%       Menge der Junktoren
\newcommand*{\asJx}{\mathcal{J}_x}%    Teilmenge der Junktoren
\newcommand*{\asJy}{\mathcal{J}_y}%    entsprechend \asJx
\newcommand*{\asM}{\mathcal{M}}%       Metaoperatoren und "Gleichheiten"
\newcommand*{\asU}{\mathcal{U}}%       Menge der unären Operatoren
\newcommand*{\asV}{\mathcal{V}}%       Menge der atomaren Formeln
\newcommand*{\asX}{\mathcal{X}}%       Mengenvariable
% verschiedene Indizes für Teilmengen von \asA, \asF, \asFp und \asJ
\newcommand*{\xAnd}{\mathrm{and}}%
\newcommand*{\xBool}{\mathrm{bool}}%
\newcommand*{\xImp}{\mathrm{imp}}%
\newcommand*{\xNand}{\mathrm{nand}}%
\newcommand*{\xNor}{\mathrm{nor}}%
\newcommand*{\xOr}{\mathrm{or}}%
\newcommand*{\xRep}{\mathrm{rep}}%

% sonstige mathematische Zeichen ===============================================

\newcommand*{\abltb}{\vdash}%           ... ableitbar ...
\newcommand*{\subst}{\curvearrowright}% ... substituiert durch ...

% sonstige Kommandos für den Mathematiksatz ####################################

\mathtoolsset{showonlyrefs,showmanualtags}% Nur mit \ref referenzierte Gleichungen, aber alle manuellen Tags

% sonstige nützliche Kommandos #################################################

% Im Parameter von '\turl' muss '\' vor jedem Zeichen aus '{}#&%$' ein '\'
% stehen und '\' / '~' durch '\textbackslash' / '\textasctilde' ersetzt werden.
\newcommand*{\tourl}[1]{$\rightarrow$~\url{#1}}
\newcommand*{\formulatoleft}{&&&&&&&&&&}%  Um Formeln nach links zu komprimieren
\newcommand*{\formulaspace}{&&&&}%         Für Platz zwischen den Formeln
\newcommand*{\todo}[1]{\textbf{>~>~>~#1~<~<~<}}% für TODOs
\newcommand*{\charqt}[1]{'#1'}%          Quotierung von Zeichen    (character) %\newcommand*{\charqt}[1]{\guilsinglleft#1\guilsinglright}%  ... Alternative
\newcommand*{\strqt}[1]{\enquote{#1}}%   Quotierung von Zeichenketten (string)
%\newcommand*{\strqt}[1]{\guillemotleft#1\guillemotright}%     ... Alternative
% Nur mit Parameter im Mathematikmodus!
\newcommand*{\symqt}[1]{\charqt{#1}}%    Quotierung einzelner Symbole (symbol)
\newcommand*{\forqt}[1]{\strqt{#1}}%     Quotierung von Formeln      (formula)

% Strukturbezeichnungen ergänzen
\newcommand*{\sectionname}{Abschnitt}
\newcommand*{\subsectionname}{Unterabschnitt}
\newcommand*{\subsubsectionname}{Paragraph}

% Abkürzungen mit Punkten; zur Unterscheidung vom Satzende
\newcommand*{\textbzgl}{bzgl.\@ }
\newcommand*{\textbzw}{bzw.\@ }
\newcommand*{\textdh}{d.\@\,h.\@ }
\newcommand*{\textggf}{ggf.\@ }
\newcommand*{\textiAlg}{i.\@\,Alg.\@ }
\newcommand*{\textua}{u.\@\,a.\@ }
\newcommand*{\textusw}{usw.\@ }
\newcommand*{\textzB}{z.\@\,B.\@ }
\newcommand*{\textZB}{Z.\@\,B.\@ }

% Ergebnis von Makros verschwinden lassen
\newcommand*{\hidden}[1]{}

% Glossareinträge ##############################################################

% Indices und Symbole ==========================================================
\makeindex
\newindex[Symbolverzeichnis]{sym}
\newindex[Index]{idx}
\newcommand*{\Idx}[1]{#1\idx{#1}}%   normaler Index
\newcommand*{\Sym}[1]{#1\sym{$#1$}}% Symbol - Nur im Mathematikmodus verwenden!

% Glossareinträge ==============================================================
\GlsSetQuote{+}% wegen Gebrauch von ngerman; see glossaries guide for beginners
\makeglossaries
\setacronymstyle{long-sc-short}

% Symbol/Index und Glossareintrag ==============================================

\newcommand*{\idx}[1]{\sindex[idx]{#1}}
\newcommand*{\sym}[1]{\sindex[sym]{#1}}
\newcommand*{\glsSym}[1]{\glspl{#1}\sym{\gls{#1}}}% Symbol - Im Mathematikmodus!
\newcommand*{\glsIdx}[1]{\gls{#1}\idx{\gls{#1}}}%        normal
\newcommand*{\GlsIdx}[1]{\Gls{#1}\idx{\gls{#1}}}%        groß
\newcommand*{\glsIdxBg}[2]{#2\idx{\gls{#1}}}%                     Beugung
\newcommand*{\GlsIdxBg}[2]{#2\idx{\gls{#1}}}%            groß und Beugung
\newcommand*{\glsIdxPl}[1]{\glspl{#1}\idx{\gls{#1}}}%             Plural
\newcommand*{\GlsIdxPl}[1]{\Glspl{#1}\idx{\gls{#1}}}%    groß und Plural
\newcommand*{\glsIdxX}[1]{\glsIdxPl{#1}}% Sonderfall
\newcommand*{\GlsIdxX}[1]{\GlsIdxPl{#1}}% Sonderfall und groß
\newcommand*{\Tag}[1]{\tag{\glsPl{#1}}\sym{\gls{#1}}}% Glossary, Symbol, Tag in Formel

% ==============================================================================

% Symbole für Beispieloperatoren -----------------------------------------------

\newglossaryentry{lrelbsp}{
	name={$ \lrelbsp$},
	plural={\lrelbsp},% im Mathematikmodus
	description={%
		Ein Beispielsymbol für eine Relation mit Umkehrrelation $\rrelbsp$%
	}
}
\newglossaryentry{lreleqbsp}{
	name={$ \lreleqbsp$},
	plural={\lreleqbsp},% im Mathematikmodus
	description={%
		Ein Beispielsymbol für eine Relation mit Gleichheit und Umkehrrelation $\rreleqbsp$%
	}
}
\newglossaryentry{relbsp}{
	name={$ \relbsp$},
	plural={\relbsp},% im Mathematikmodus
	description={%
		Ein Beispielsymbol für eine Relation%
	}
}
\newglossaryentry{releqbsp}{
	name={$ \releqbsp$},
	plural={\releqbsp},% im Mathematikmodus
	description={%
		Ein Beispielsymbol für eine Relation mit Gleichheit%
	}
}
\newglossaryentry{rrelbsp}{
	name={$ \rrelbsp$},
	plural={\rrelbsp},% im Mathematikmodus
	description={%
		Ein Beispielsymbol für eine Relation mit Umkehrrelation $\lrelbsp$%
	}
}
\newglossaryentry{rreleqbsp}{
	name={$ \rreleqbsp$},
	plural={\rreleqbsp},% im Mathematikmodus
	description={%
		Ein Beispielsymbol für eine Relation mit Gleichheit und Umkehrrelation $\lreleqbsp$%
	}
}

% Symbole für Metaoperatoren ---------------------------------------------------

\newglossaryentry{defeq}{
	name={$ \defeq$},
	plural={\defeq},% im Mathematikmodus
	description={%
		Ein \glsIdx{MetaoperatorV}: ... definitionsgemäß gleich ..%
	}
}
\newglossaryentry{eq}{
	name={$ \eq$},
	plural={\eq},% im Mathematikmodus
	description={%
		Ein \glsIdx{MetaoperatorV}: ... gleich (ist dasselbe wie, ist identisch zu) ..%
	}
}
\newglossaryentry{equiv}{
	name={$ \equiv$},
	plural={\equiv},% im Mathematikmodus
	description={%
		Ein (Meta-)Operator: ... äquivalent (ist das gleiche wie, ist so wie) zu ..%
	}
}
\newglossaryentry{metaand}{
	name={$ \metaund$},
	plural={\metaand},% im Mathematikmodus
	description={%
		Ein \glsIdx{MetaoperatorV}: ... oder ... (\seealso~\glsSym{mid})%
	}
}
\newglossaryentry{metadefeq}{
	name={$ \metadefeq$},
	plural={\metadefeq},% im Mathematikmodus
	description={%
		Ein \glsIdx{MetaoperatorV}: ... definitionsgemäß gleich (definitionsgemäß genau dann, wenn) ..%
	}
}
\newglossaryentry{metaequiv}{
	name={$ \metaequiv$},
	plural={\metaequiv},% im Mathematikmodus
	description={%
		Ein \glsIdx{MetaoperatorV}: ... genau dann wenn ..%
	}
}
\newglossaryentry{metaimp}{
	name={$ \metaimp$},
	plural={\metaimp},% im Mathematikmodus
	description={%
		Ein \glsIdx{MetaoperatorV}: ... dann auch ..%
	}
}
\newglossaryentry{metaor}{
	name={$ \metaoder$},
	plural={\metaor},% im Mathematikmodus
	description={%
		Ein \glsIdx{MetaoperatorV}: ... oder ..%
	}
}
\newglossaryentry{metarep}{
	name={$ \metarep$},
	plural={\metarep},% im Mathematikmodus
	description={%
		Ein \glsIdx{MetaoperatorV}: ... sofern ..%
	}
}
\newglossaryentry{mid}{
	name={$ \mid$},
	plural={\mid},% im Mathematikmodus
	description={%
		Ein \glsIdx{MetaoperatorV}: ... und ... (\seealso~\glsSym{metaand})%
	}
}
\newglossaryentry{ne}{
	name={$ \ne$},
	plural={\ne},% im Mathematikmodus
	description={%
		Ein \glsIdx{MetaoperatorV}: ... ungleich (nicht dasselbe wie, nicht identisch zu) ..%
	}
}
\newglossaryentry{notequiv}{
	name={$ \notequiv$},
	plural={\notequiv},% im Mathematikmodus
	description={%
		Ein (Meta-)Operator: ... nicht äquivalent (ist nicht das gleiche wie, ist nicht so wie) ..%
	}
}

% sonstige mathematische Symbole -----------------------------------------------

\newglossaryentry{abltb}{
	name={$ \abltb$},
	plural={\abltb},% im Mathematikmodus
	description={%
		Ableitungsrelation: ... ableitbar ...
		(\seename\ \emph{\glsIdx{ableitbar}})%
	}
}
\newglossaryentry{lfalse}{
	name={$ \lfalse$},
	plural={\lfalse},% im Mathematikmodus
	description={%
		Eine Aussagenlogische Konstante: Falsch%
	}
}
\newglossaryentry{ltrue}{
	name={$ \ltrue$},
	plural={\ltrue},% im Mathematikmodus
	description={%
		Eine Aussagenlogische Konstante: Wahr%
	}
}
\newglossaryentry{subst}{
	name={$ \subst$},
	plural={\subst},% im Mathematikmodus
	description={%
		Substitution: ... substituiert durch ...
		(siehe die Definition in \subsectionname~\vref{sub:Basisregeln})%
	}
}

% Symbole für Mengen -----------------------------------------------------------

\newglossaryentry{gsN}{
	name={$ \gsN$},
	plural={\gsN},% im Mathematikmodus
	description={%
		Die Menge der natürlichen Zahlen ohne 0%
	}
}
\newglossaryentry{gsNo}{
	name={$ \gsNo$},
	plural={\gsNo},% im Mathematikmodus - erfolgt im falschen Zeichensatz!
	description={%
		Die Menge der natürlichen Zahlen einschließlich 0%
	}
}
\newglossaryentry{asA}{
	name={$ \asA$},
	plural={\asA},% im Mathematikmodus
	description={%
		Das Alphabet der aussagenlogischen Sprache%
	}
}
\newglossaryentry{asAx}{
	name={$ \asAx$},
	plural={\asAx},% im Mathematikmodus
	description={%
		Eine Teilmenge des Alphabets $\asA$ der aussagenlogischen Sprache%
	}
}
\newglossaryentry{asB}{
	name={$ \asB$},
	plural={\asB},% im Mathematikmodus
	description={%
		Die Menge der aussagenlogischen, binären Operatoren%
	}
}
\newglossaryentry{asC}{
	name={$ \asC$},
	plural={\asC},% im Mathematikmodus
	description={%
		Die Menge der aussagenlogischen Konstanten%
	}
}
\newglossaryentry{asF}{
	name={$ \asF$},
	plural={\asF},% im Mathematikmodus
	description={%
		Die Menge der aussagenlogischen Formeln mit Klammerung%
	}
}
\newglossaryentry{asFp}{
	name={$ \asFp$},
	plural={\asFp},% im Mathematikmodus
	description={%
		Die Menge der aussagenlogischen Formeln in polnischer Notation%
	}
}
\newglossaryentry{asFx}{
	name={$ \asFx$},
	plural={\asFx},% im Mathematikmodus
	description={%
		Eine Teilmenge der Menge $\asF$ der aussagenlogischen Formeln mit Klammerung%
	}
}
\newglossaryentry{asFxp}{
	name={$ \asFxp$},
	plural={\asFxp},% im Mathematikmodus
	description={%
		Eine Teilmenge der Menge $\asF$ der aussagenlogischen Formeln in polnischer Notation%
	}
}
\newglossaryentry{asJ}{
	name={$ \asJ$},
	plural={\asJ},% im Mathematikmodus
	description={%
		Die Menge der aussagenlogischen Operatoren%
	}
}
\newglossaryentry{asJx}{
	name={$ \asJx$},
	plural={\asJx},% im Mathematikmodus
	description={%
		Eine Teilmenge der Menge $\asJ$ der aussagenlogischen Operatoren
	}
}
\newglossaryentry{asM}{
	name={$ \asM$},
	plural={\asM},% im Mathematikmodus
	description={%
		Die Menge der \glsIdxPl{MetaoperatorV} und der mit Gleichheit verwandten Symbole%
	}
}
\newglossaryentry{asU}{
	name={$ \asU$},
	plural={\asU},% im Mathematikmodus
	description={%
		Die Menge der aussagenlogischen unären Operatoren%
	}
}
\newglossaryentry{asV}{
	name={$ \asV$},
	plural={\asV},% im Mathematikmodus
	description={%
		Die Menge der aussagenlogischen \glsIdxPl{atomareFormelA}%
	}
}

% Schlussregeln ----------------------------------------------------------------

\newcommand*{\tagAR}{AR}% Argument für \tag
\newglossaryentry{AR}{
	name={(AR)},
	plural={(\text{AR})},% im Mathematikmodus
	description={%
		\glsIdx{Anfangsregel}%
	}
}
\newcommand*{\tagMR}{MR}% Argument für \tag
\newglossaryentry{MR}{
	name={(MR)},
	plural={(\text{MR})},% im Mathematikmodus
	description={%
		\glsIdx{Monotonieregel}%
	}
}
\newcommand*{\tagSR}{SR}% Argument für \tag
\newglossaryentry{SR}{
	name={(SR)},
	plural={(\text{SR})},% im Mathematikmodus
	description={%
		\glsIdx{Schnittregel} (Modus ponens)%
	}
}
\newcommand*{\tagTR}{TR}% Argument für \tag
\newglossaryentry{TR}{
	name={(TR)},
	plural={(\text{TR})},% im Mathematikmodus
	description={%
		\glsIdx{Abtrennungsregel}%
	}
}
\newcommand*{\tagandB}{$\land$B}% Argument für \tag
\newglossaryentry{andB}{
	name={($ \land     $B)},
	plural={(\land\text{B})},% im Mathematikmodus
	description={%
		Beseitigung von \symqt{$\land$}%
	}
}
\newcommand*{\tagandE}{$\land$E}% Argument für \tag
\newglossaryentry{andE}{
	name={($ \land     $E)},
	plural={(\land\text{E})},% im Mathematikmodus
	description={%
		Einführung von \symqt{$\land$}%
	}
}
\newcommand*{\tagimpB}{$\limp$B}% Argument für \tag
\newglossaryentry{impB}{
	name={($ \limp     $B)},
	plural={(\limp\text{B})},% im Mathematikmodus
	description={%
		Beseitigung von \symqt{$\limp$}%
	}
}
\newcommand*{\tagimpE}{$\limp$E}% Argument für \tag
\newglossaryentry{impE}{
	name={($ \limp     $E)},
	plural={(\limp\text{E})},% im Mathematikmodus
	description={%
		Einführung von \symqt{$\limp$}%
	}
}
\newcommand*{\tagnota}{$\lnot$1}% Argument für \tag
\newglossaryentry{nota}{
	name={($ \lnot     $1)},
	plural={(\lnot\text{1})},% im Mathematikmodus
	description={%
		Einführung/Beseitigung von \symqt{$\lnot$} Teil 1%
	}
}
\newcommand*{\tagnotb}{$\lnot$2}% Argument für \tag
\newglossaryentry{notb}{
	name={($ \lnot     $2)},
	plural={(\lnot\text{2})},% im Mathematikmodus
	description={%
		Einführung/Beseitigung von \symqt{$\lnot$} Teil 2%
	}
}
\newcommand*{\tagnotc}{$\lnot$3}% Argument für \tag
\newglossaryentry{notc}{
	name={($ \lnot     $3)},
	plural={(\lnot\text{3})},% im Mathematikmodus
	description={%
		Beweistechnik \strqt{Indirekter Beweis}%
	}
}
\newcommand*{\tagnotd}{$\lnot$4}% Argument für \tag
\newglossaryentry{notd}{% statt "notE"
	name={($ \lnot     $4)},
	plural={(\lnot\text{4})},% im Mathematikmodus
	description={%
		Reductio ad absurdum (indirekter Beweis)%
	}
}
%%%\newglossaryentry{orB}{
%%%	name={($ \lor     $B)},
%%%	plural={(\lor\text{B})},% im Mathematikmodus
%%%	description={%
%%%		Beseitigung von \symqt{$\lor$}%
%%%	}
%%%}
%%%\newglossaryentry{orE}{
%%%	name={($ \lor     $E)},
%%%	plural={(\lor\text{E})},% im Mathematikmodus
%%%	description={%
%%%		Einführung von \symqt{$\lor$}%
%%%	}
%%%}

% keine Symbole mehr -----------------------------------------------------------

\newglossaryentry{ableitbar}{
	name={ableitbar},
	plural={ableitbare},
	description={%
		Wenn sich eine Formel $\beta$ aus einer Formel $\alpha$ mittels zulässiger Transaktionen ableiten lässt, heißt $\beta$ ableitbar aus $\alpha$.
		Sprechweise: \forqt{$\alpha$ ableitbar $\beta$}.
		Eine oder beide Formeln $\alpha$ \textbzw $\beta$ dürfen dabei durch Formelmengen ersetzt werden.
		(\seealso\ \symqt{\glsIdx{abltb}} und \glsIdx{Ableitungsrelation})
		-- Synonym: \glsIdx{beweisbar}%
	}
}
\newglossaryentry{Ableitungsrelation}{
	name={Ableitungsrelation},
	plural={Ableitungsrelationen},
	description={%
		Die Relation \symqt{\glsIdx{abltb}}%
	}
}
\newglossaryentry{Abtrennungsregel}{
	name={Abtrennungsregel},
	plural={Abtrennungsregeln},
	description={%
		Eine \emph{\glsIdx{Schlussregel}} - siehe~\glsIdx{TR}%
	}
}
\newglossaryentry{Anfangsregel}{
	name={Anfangsregel},
	plural={Anfangsregeln},
	description={%
		Eine \emph{\glsIdx{Schlussregel}} um beginnen zu können - siehe~\glsIdx{AR}%
	}
}
\newcommand*{\ASBA}{\glsIdx{ASBA}}
\newacronym{ASBA}{ASBA}{
	Programmsystem, das \textbf{A}xiome, \textbf{S}ätze, \textbf{B}eweise und \textbf{A}uswertungen behandeln kann%
}
\newglossaryentry{atomareFormelA}{
	name={atomare Formel},     % eine ...
	plural={atomaren Formeln}, % alle ...
	description={%
		Eine Formel, die sich nicht weiter zerlegen lässt%
	}%
}
\newglossaryentry{Ausgabeschema}{
	name={Ausgabeschema},
	plural={Ausgabeschemata},
	description={%
		Ein Schema, mit dem bestimmte mathematische Objekte ausgegeben werden sollen%
	}
}
\newglossaryentry{Aussage}{
	name={Aussage},
	plural={Aussagen},
	description={%
		Eine Aussage in natürlicher Sprache oder als Formel, die einen Wahrheitswert liefert%
	}
}
\newglossaryentry{Aussagenlogik}{
	name={Aussagenlogik},
	description={%
		\seename\ \sectionname~\vref{sec:Aussagenlogik}%
	}
}
\newglossaryentry{Axiom}{
	name={Axiom},
	plural={Axiome},
	description={%
		Eine Formel, die unbewiesen als wahr angesehen wird%
	}
}
\newglossaryentry{Basisregel}{
	name={Basisregel},
	plural={Basisregeln},
	description={%
		Eine \emph{\glsIdx{Schlussregel}}, die nicht mehr auf andere zurückgeführt wird%
	}
}
\newglossaryentry{Beweis}{
	name={Beweis},
	plural={Beweise},
	description={%
		Eine zulässige Ableitung von Folgerungen aus gegebenen Voraussetzungen%
	}
}
\newglossaryentry{beweisbar}{
	name={beweisbar},
	plural={beweisbare},
	description={%
		Synonym zu \emph{ableitbar}%
	}
}

\newglossaryentry{Beweisschritt}{
	name={Beweisschritt},
	plural={Beweisschritte},
	description={%
		Eine Vorschrift, wie aus vorgegebenen Aussagen eine weitere folgt%
	}
}
\newglossaryentry{BoolscheSignatur}{
	name={Boolsche Signatur},
	plural={Boolschen Signatur},% Dativ
	description={%
		Die \glsIdx{logischeSignaturV} $\{\lnot, \land, \lor\}$%
	}
}
\newglossaryentry{Fachbegriff}{
	name={Fachbegriff},
	plural={Fachbegriffe},
	description={%
		Ein Name für einen mathematischen Begriff%
	}
}
\newglossaryentry{Fachgebiet}{
	name={Fachgebiet},
	plural={Fachgebiete},
	description={%
		Ein Teil der Mathematik mit einer zugehörigen Basis aus Axiomen, Sätzen und spezifischen Fachbegriffen und Darstellungen%
	}
}
\newglossaryentry{FormalelementV}{
	name={formales Element},   % ein   ...
	plural={formale Elemente}, % viele ...
	description={%
		Ein mathematisches Element in formaler Schreibweise.
		Bis auf wenige Aussagen kommen darin \emph{\glsIdxPl{MetaausdruckV}} nicht mehr vor%
	}%
}
\newglossaryentry{intEigenschaftA}{
	name={interessierende Eigenschaft},      % eine ...
	plural={interessierenden Eigenschaften}, % alle ...
	description={%
		Solche Eigenschaften von Ausdrücken, die im aktuellen Zusammenhang von Interesse sind.%
	}%
}
\newglossaryentry{logischeSignaturV}{
	name={logische Signatur},     % eine  ...
	plural={logische Signaturen}, % viele ...
	description={%
		Eine in \emph{\glsIdx{Metasprache}} verfasste Aussage, die auch zusammengesetzt sein kann%
	}%
}
\newglossaryentry{MetaausdruckV}{
	name={metasprachlicher Ausdruck},   % ein   ...
	plural={metasprachliche Ausdrücke}, % viele ...
	description={%
		Eine in normaler Sprache verfasste Aussage, die auch zusammengesetzt sein kann%
	}%
}
\newglossaryentry{MetaaussageV}{
	name={metasprachliche Aussage},    % eine   ...
	plural={metasprachliche Aussagen}, % viele ...
	description={%
		Eine in \emph{\glsIdx{Metasprache}} verfasste Aussage, die auch zusammengesetzt sein kann%
	}%
}
\newglossaryentry{MetaoperatorV}{
	name={metasprachlicher Operator},    % ein   ...
	plural={metasprachliche Operatoren}, % viele ...
	description={%
		Ein Operator, dessen Operanden \emph{\glsIdxPl{MetaausdruckV}} sind%
	}%
}
\newglossaryentry{Metasprache}{
	name={Metasprache},
	plural={Metasprachen},
	description={%
		Eine Sprache, in der Aussagen über Elemente einer anderen Sprache getroffen werden können%
	}%
}
\newglossaryentry{Monotonieregel}{
	name={Monotonieregel},
	plural={Monotonieregeln},
	description={%
		Eine \emph{\glsIdx{Schlussregel}} - siehe~\glsIdx{MR}%
	}
}
\newglossaryentry{Praedikat}{
	name={Prädikat},
	plural={Prädikate},
	description={%
		Ein Element der \emph{\glsIdx{Praedikatenlogik}} (\see \sectionname~\vref{sec:Praedikatenlogik}).
		\textZB kann man eine \forqt{$Gruppe$} als ein zweistelliges Prädikat \forqt{$Gruppe(G,+)$} definieren, in dem $G$ eine Menge und \symqt{$+$} eine Operation, \textdh eine (zweistellige) Funktion \forqt{$+: G \times G \rightarrow G$} ist, so dass die Gruppenaxiome erfüllt sind%
	}
}
\newglossaryentry{Praedikatenlogik}{
	name={Prädikatenlogik},
	description={%
		\seename\ \sectionname~\vref{sec:Praedikatenlogik}%
	}
}
\newglossaryentry{Satz}{
	name={Satz},
	plural={Sätze},
	description={%
		Eine mathematische Aussage, dass eine bestimmte Folgerung aus gegebenen Voraussetzungen abgeleitet werden kann%
	}
}
\newglossaryentry{Schlussregel}{
	name={Schlussregel},
	plural={Schlussregeln},
	description={%
		Eine Regel für eine (zulässige) Umwandlung von Formeln%
	}
}
\newglossaryentry{Schnittregel}{
	name={Schnittregel},
	plural={Schnittregeln},
	description={%
		Eine \emph{\glsIdx{Schlussregel}} - siehe~\glsIdx{SR}%
	}
}
\newglossaryentry{vergleichbar}{
	name={vergleichbar},
	plural={vergleichbare},
	description={
		Zwei \emph{\glsIdxPl{MetaausdruckV}} \textbzw \emph{\glsIdxPl{FormalelementV}} heißen -- auf eine bestimmte Art -- vergleichbar, wenn sie auf diese Art (\textzB als Zeichenketten oder als vergleichbare Ergebnisse von Formeln) verglichen werden können.
		Die Art muss implizit bekannt oder explizit angegeben sein.
		Meistens genügt es zu wissen, was für \glsIdxPl{MetaausdruckV} \textbzw \glsIdxPl{FormalelementV} es sind.
		Sie müssen dann nur von derselben Art sein%
	}%
}
\newglossaryentry{Wahrheitswert}{
	name={Wahrheitswert},
	plural={Wahrheitswerte},
	description={%
		Wahrheitswerte sind die Werte \strqt{wahr} und \strqt{falsch}, oft auch als \strqt{true} und \strqt{false} oder einfach \charqt{1} und \charqt{0} bezeichnet%
	}
}
\newglossaryentry{zulaessigeTransformation}{
	name={zulässige Transformation},
	plural={zulässige Transformation},
	description={%
		Die Anwendung einer \glsIdx{Basisregel} oder einer davon abgeleiteten sonstigen \glsIdx{Schlussregel}%
	}
}
