%%############################################################################%%
%%                                                                            %%
%% Datei:  ASBA-Vorspann-Logik.tex                                            %%
%% Inhalt: Vorspann Logik für ASBA                                            %%
%%                                                                            %%
%% Copyright (C) 2017  Winfried Teschers                                      %%
%%                                                                            %%
%% This program is free software: you can redistribute it and/or modify       %%
%% it under the terms of the GNU Affero General Public License as published   %%
%% by the Free Software Foundation, either version 3 of the License, or       %%
%% (at your option) any later version.                                        %%
%%                                                                            %%
%% This program is distributed in the hope that it will be useful,            %%
%% but WITHOUT ANY WARRANTY; without even the implied warranty of             %%
%% MERCHANTABILITY or FITNESS FOR A PARTICULAR PURPOSE.  See the              %%
%% GNU Affero General Public License for more details.                        %%
%%                                                                            %%
%% You should have received a copy of the GNU Affero General Public License   %%
%% along with this program.  If not, see <http://www.gnu.org/licenses/>.      %%
%%                                                                            %%
%% Dr. Winfried Teschers                                                      %%
%% Anton-Günther-Straße 26c                                                   %%
%% 91083 Baiersdorf                                                           %%
%% Germany                                                                    %%
%%                                                                            %%
%% e-mail: winfried.teschers@t-online.de                                      %%
%%                                                                            %%
%%############################################################################%%

% !TeX root = ASBA.tex
% !TeX encoding = UTF-8
% !TeX spellcheck = de_DE

% Glossareinträge werden in "ASBA-Vorspann-Glossar" definiert.
% Elemente, die in anderen Dateien als "ASBA-Logik.tex" verwendet werden, werden in "ASBA-Vorspann.tex" definiert.
% Namensbestandteile mit besonderer Bedeutung:
%   Bezeichnungen:
%     <Name>Set     = Bereich  von <Name>n (keine bis alle)
%     <Name>Rel     = Relation von <Name>n
%     <Name>Tup     = Tupel    von <Name>n
%     All<Name>     = Bereich  aller <Name>n
%   Ergebnisse von Operationen auf Bereichen:
%     Pot<Bereich>  = Bereich  der Teilmengen                   von <Name>
%     Rel<Bereich>  = Bereich  der binären Relationen           auf <Name>
%     Relf<Bereich> = Bereich  der endlichen binären Relationen auf <Name>
%     Tup<Bereich>  = Bereich  der Tupel                        auf <Name>
%   Sonstiges:
%     <Relation>Bck = Umkehrrelation
%     <Operator>Eq  = Operator oder Gleich
%     <Operator>N   = negierter Operator
%     <Oper.>BckEqN = Kombination in dieser Reihenfolge
%     \Tag<Tag>     =
%
% Folgende Makros sind alles "eigene"
%   \Ltr...    - nur Zeichen für (den Bereich)   ...       (im        Textmodus)
%   \Str...    - nur Text    für (die Operation) ...       (im        Textmodus)
%   \StrMtsIdx...  - Text    für einen Index für ...       (im  Mathematikmodus)
%   \Raw...    - Symbol/Text ohne Link;erfordert bei Symbolen Mathematikmodus
%   \...Bsp... - Beispielsymbol für Formel- oder Metasprache(im Mathematikmodus)
%   \Bsp...    -                          mit Link ins Symbolverzeichnis
%   \...Mts... - Symbol der Metasprache                    (im  Mathematikmodus)
%   \Mts...    -                          mit Link ins Symbolverzeichnis
%   \...Ojk... - Symbol der Formelsprache                  (im  Mathematikmodus)
%   \Ojk...    -                          mit Link ins Symbolverzeichnis
%   \...Txt... - individuelle Bezeichnung                  (im        Textmodus)
%   \Txt...    -                          mit Link in  Glossar und Index
%
% Weitere "eigene" Kombinationen: OpU, OpB, Bck, Idx, Ltr, Txt

% Beispielsymbole für Operationen und Relationen ===============================
% \RawBsp*; OpU=unär, OpB=binär, Rel=Relation, Bck=Umkehr-; N=nicht, Eq=gleich
\newcommand*{\RawBspOpRel}    {\mathbin{\circ}}
\newcommand*{\RawBspOpU}      {\mathbin{\circleddash}}
\newcommand*{\RawBspOpB}      {\mathbin{\circledast}}
\newcommand*{\RawBspRel}      {\mathrel{\prec}}
\newcommand*{\RawBspRelEq}    {\mathrel{\preceq}}
\newcommand*{\RawBspRelBck}   {\mathrel{\succ}}
\newcommand*{\RawBspRelBckEq} {\mathrel{\succeq}}
\newcommand*{\RawBspRelN}     {\mathrel{\nprec}}
\newcommand*{\RawBspRelEqN}   {\mathrel{\npreceq}}
\newcommand*{\RawBspRelBckN}  {\mathrel{\nsucc}}
\newcommand*{\RawBspRelBckEqN}{\mathrel{\nsucceq}}

% Metasymbole ==================================================================
% \RawMts*
% Metaoperationen, -relationen u.a. (für Aussagen) -----------------------------
\newcommand*{\RawMtsNot}     {\mathbin{\thicksim}}%               ... gilt nicht
\newcommand*{\RawMtsAnd}     {\mathbin{\&}}%                        ... und  ...
\newcommand*{\RawMtsOr}      {\mathbin{\parallel}}%                 ... oder ...
\newcommand*{\RawMtsImp}     {\mathrel{\Rightarrow}}% von ... folgt          ...
\newcommand*{\RawMtsRep}     {\mathrel{\Leftarrow}}%      ... folgt von      ...
\newcommand*{\RawMtsEquiv}   {\mathrel{\Leftrightarrow}}% .. genau dann, wenn ..
%%%\newcommand*{\RawMtsEq}      {\mathrel{=\mkern-8mu=}}%           ...  gleich ...
%%%\newcommand*{\RawMtsEqN}     {\mathrel{=\mkern-15mu/\mkern-15mu=}}% . ungleich .
\newcommand*{\RawMtsEq}      {\mathrel{\equiv}}%          ...       gleich   ...
\newcommand*{\RawMtsEqN}     {\mathrel{\nequiv}}%         ...       ungleich ...
%%%\newcommand*{\RawMtsAequiv}  {\mathrel{\equiv}}%     ...       äquivalent zu ...
%%%\newcommand*{\RawMtsAequivN} {\mathrel{\nequiv}}%    ... nicht äquivalent zu ...
\newcommand*{\RawMtsDefEquiv}{\mathrel{:\mkern-2mu\RawMtsEquiv}}% def.gemäß -"-
\newcommand*{\RawMtsDefEq}   {\mathrel{:\mkern-2mu\RawMtsEq}}% def.gemäß gleich
\newcommand*{\RawMtsUnd}     {\mathbin{\mid}}%nur in Schlussregeln: ... und  ...
\newcommand*{\RawMtsDerive}  {\mathrel{\vdash}}%          ... ableitbar      ...
\newcommand*{\RawMtsSwap}    {\mathbin{\leftrightarrows}}% .. vertauscht mit ...
\newcommand*{\RawMtsSubst}   {\mathbin{\leftarrowtail}}%.. substituiert durch ..
% Elementrelationen (für Elemente und Bereiche) --------------------------------
\newcommand*{\RawMtsIn}       {\in}%     ist Element aus (dem Bereich)
\newcommand*{\RawMtsNi}       {\ni}%    (der Bereich) enthält nicht das Element
\newcommand*{\RawMtsInN}      {\notin}%  ist Element aus (dem Bereich)
\newcommand*{\RawMtsNiN}      {\notni}% (der Bereich) enthält nicht das Element
\newcommand*{\RawMtsSetSep}   {\mid}%  Der Trennstrich in einer Bereichsdefinition
% Bereichsrelationen (für Bereiche) --------------------------------------------
\newcommand*{\RawMtsSubset}   {\subset}%    ist        echte  Teilmenge von
\newcommand*{\RawMtsSubsetEq} {\subseteq}%  ist (gleich oder) Teilmenge von
\newcommand*{\RawMtsSubsetN}  {\nsubset}%   ist  keine echte  Teilmenge von
\newcommand*{\RawMtsSubsetEqN}{\nsubseteq}% ist  keine        Teilmenge von
\newcommand*{\RawMtsSupset}   {\supset}%    ist        echte  Obermenge von
\newcommand*{\RawMtsSupsetEq} {\supseteq}%  ist (gleich oder) Obermenge von
\newcommand*{\RawMtsSupsetN}  {\nsupset}%   ist  keine echte  Obermenge von
\newcommand*{\RawMtsSupsetEqN}{\nsupseteq}% ist  keine        Obermenge von
% Bereichsoperationen (für Bereiche) -------------------------------------------
\newcommand*{\RawMtsCap}      {\cap}%       Durchschnitt          von Bereichen
\newcommand*{\RawMtsCup}      {\cup}%       Vereinigung           von Bereichen
\newcommand*{\RawMtsSetminus} {\setminus}%  Differenz             von Bereichen
\newcommand*{\RawMtsTimes}    {\times}%     karthesisches Produkt von Bereichen
\newcommand*{\RawMtsEmptyset} {\emptyset}%  der leere Bereich (die leere Menge)
% Komponentenrelationen (für Komponenten und Folgen) ---------------------------
\newcommand*{\RawMtsSeqIn}    {\sqsubset\mkern-19mu-}%  ist  Komponente (der Folge)
\newcommand*{\RawMtsSeqNi}    {\sqsupset\mkern-19mu-}%  (die Folge) enthält nicht das Symbol
\newcommand*{\RawMtsSeqInN}   {\nsqsubset\mkern-19mu-}% ist  Komponente (der Folge)
\newcommand*{\RawMtsSeqNiN}   {\nsqsupset\mkern-19mu-}% (die Folge) enthält nicht das Symbol
% Folgenoperationen (für Folgen) -----------------------------------------------
\newcommand*{\RawMtsCat}      {\sqcup}%       Verkettung           von Folgen
% Folgenrelationen (für Folgen) ------------------------------------------------
\newcommand*{\RawMtsSubseq}   {\sqsubset}%    ist        echte  Teilmenge von
\newcommand*{\RawMtsSubseqEq} {\sqsubseteq}%  ist (gleich oder) Teilmenge von
\newcommand*{\RawMtsSubseqN}  {\nsqsubset}%   ist  keine echte  Teilmenge von
\newcommand*{\RawMtsSubseqEqN}{\nsqsubseteq}% ist  keine        Teilmenge von
\newcommand*{\RawMtsSupseq}   {\sqsupset}%    ist        echte  Obermenge von
\newcommand*{\RawMtsSupseqEq} {\sqsupseteq}%  ist (gleich oder) Obermenge von
\newcommand*{\RawMtsSupseqN}  {\nsqsupset}%   ist  keine echte  Obermenge von
\newcommand*{\RawMtsSupseqEqN}{\nsqsupseteq}% ist  keine        Obermenge von

% Text-, Meta- und Objekt-Wahrheitswerte
\newcommand*{\RawTxtFalse}    {\emph{\StrTxtFalse}}%                      (Text)
\newcommand*{\RawTxtTrue}     {\emph{\StrTxtTrue}}%                       (Text)
\newcommand*{\RawMtsFalse}    {\mathord{\Conft{\StrMtsFalse}}}%F-falsch (Symbol)
\newcommand*{\RawMtsTrue}     {\mathord{\Conft{\StrMtsTrue}}}% W-wahr   (Symbol)
\newcommand*{\RawOjkFalse}    {\mathord{\bot}}%                         (Symbol)
\newcommand*{\RawOjkTrue}     {\mathord{\top}}%                         (Symbol)

% Definitionen für die Tabelle der Junktoren -----------------------------------
% \RawOjk*
% Wahrheitswert von A                            W W F F
% Wahrheitswert von B                            W F W F
% unäre Operationen ------------------------------------------------------------
\newcommand*{\RawOjkNot}      {\lnot}%           F W - - - nicht A
% binäre Operationen -----------------------------------------------------------
\newcommand*{\RawOjkAnd}      {\land}%           W F F F - A und B
\newcommand*{\RawOjkOr}       {\lor}%            W W W F - A oder B
\newcommand*{\RawOjkImp}      {\rightarrow}%     W F W W - von A folgt B
\newcommand*{\RawOjkRep}      {\leftarrow}%      W W F W - A folgt von B
\newcommand*{\RawOjkEquiv}    {\leftrightarrow}% W F F W - A genau dann wenn B
\newcommand*{\RawOjkNand}     {\uparrow}%        F W W W - nicht   (A und  B)
\newcommand*{\RawOjkNor}      {\downarrow}%      F F F W - weder    A noch B
\newcommand*{\RawOjkXor}      {\dot\lor}%        F W W F - entweder A oder B

% außerhalb der Tabelle --------------------------------------------------------
\newcommand*{\RawOjkEq}       {=}%          Gleichheit   in Formeln
\newcommand*{\RawOjkEqN}      {\ne}%        Ungleichheit in Formeln
% Quantoren --------------------------------------------------------------------
\newcommand*{\RawMtsForAll}   {\forall}%     für alle          <x>         gilt:
\newcommand*{\RawMtsExists}   {\exists}%     es gibt       ein <x> für das gilt:
\newcommand*{\RawMtsExiOne}   {\exists!}%    es gibt genau ein <x> für das gilt:
\newcommand*{\RawOjkForAll}   {\bigwedge}%   für alle          <x>         gilt:
\newcommand*{\RawOjkExists}   {\bigvee}%     es gibt       ein <x> für das gilt:
\newcommand*{\RawOjkExiOne}   {\dot\bigvee}% es gibt genau ein <x> für das gilt:

% weitere Symbole --------------------------------------------------------------
\newcommand*{\RawMtsFktSep}   {:}%                 f \MtsFktSep A \MtsFktArrow B
\newcommand*{\RawMtsFktArrow} {\rightarrow}

% Neue Metaoperationen ---------------------------------------------------------
\DeclareMathOperator*{\RawMtsValue}  {\StrMtsValue}% Wert  einer      Formel
\DeclareMathOperator*{\RawMtsGraph}  {\StrMtsGraph}% Graph einer      Funktion/Relation
\DeclareMathOperator*{\RawMtsTraeger}{\StrMtsTraeger}% Trägermenge einer       Relation
\DeclareMathOperator*{\RawMtsStel}   {\StrMtsStel}% Stelligkeit einer Funktion/Relation
\DeclareMathOperator*{\RawMtsStelF}  {\StrMtsStel_f}% Stelligkeit für [F]unktionen
\DeclareMathOperator*{\RawMtsStelR}  {\StrMtsStel_r}% Stelligkeit für [R]elationen
\DeclareMathOperator*{\RawMtsQb}     {\StrMtsQb}% Quellbereich einer partiellen Funktion
\DeclareMathOperator*{\RawMtsDb}     {\StrMtsDb}% Definitionsbereich einer      Funktion
\DeclareMathOperator*{\RawMtsZb}     {\StrMtsZb}% Zielbereich        einer      Funktion
\DeclareMathOperator*{\RawMtsWb}     {\StrMtsWb}% Wertebereich       einer      Funktion
\DeclareMathOperator*{\RawMtsLen}    {\StrMtsLen}% Länge            eines/r Tupels/Folge
\DeclareMathOperator*{\RawMtsSet}    {\StrMtsSet}% Komponentenmenge eines/r Tupels/Folge

% Platzhalter zur Anzeige des Ortes von Indizes
%%%\newcommand*{\Owner}{\square}
\newcommand*{\Owner}{[]}
% spezielle Indizes (rechts oben)
\newcommand*{\LtrMtsIdxLogisch} {A}%    die Logik betreffend
% spezielle Indizes für Teilmengen (rechts unten)
\newcommand*{\StrMtsIdxBin}     {b}%    binär
\newcommand*{\StrMtsIdxCon}     {c}%    constant; konstant
\newcommand*{\StrMtsIdxUna}     {u}%    unär
\newcommand*{\StrMtsIdxAnd}     {and}%  Signatur not, and
\newcommand*{\StrMtsIdxBool}    {bool}% Signatur not, and, or
\newcommand*{\StrMtsIdxImp}     {imp}%  Signatur not, imp
\newcommand*{\StrMtsIdxNand}    {nand}% Signatur      nand
\newcommand*{\StrMtsIdxNor}     {nor}%  Signatur      nor
\newcommand*{\StrMtsIdxOr}      {or}%   Signatur not, or
\newcommand*{\StrMtsIdxRep}     {rep}%  Signatur not, rep
% spezielle Indizes mit Schriftart
\newcommand*{\RawMtsIdxPolnisch}{\Idxft{\LtrMtsIdxPolnisch}}% ^
\newcommand*{\RawMtsIdxLogisch} {\Idxft{\LtrMtsIdxLogisch}}%  ^
\newcommand*{\RawMtsIdxEndlich} {\Idxft{\LtrMtsIdxEndlich}}%  _
\newcommand*{\RawMtsIdxGraph}   {\Idxft{\LtrMtsIdxGraph}}%    _
\newcommand*{\RawMtsIdxBin}     {\Idxft{\StrMtsIdxBin}}
\newcommand*{\RawMtsIdxCon}     {\Idxft{\StrMtsIdxCon}}
\newcommand*{\RawMtsIdxUna}     {\Idxft{\StrMtsIdxUna}}
\newcommand*{\RawMtsIdxAnd}     {\Idxft{\StrMtsIdxAnd}}
\newcommand*{\RawMtsIdxBool}    {\Idxft{\StrMtsIdxBool}}
\newcommand*{\RawMtsIdxImp}     {\Idxft{\StrMtsIdxImp}}
\newcommand*{\RawMtsIdxNand}    {\Idxft{\StrMtsIdxNand}}
\newcommand*{\RawMtsIdxNor}     {\Idxft{\StrMtsIdxNor}}
\newcommand*{\RawMtsIdxOr}      {\Idxft{\StrMtsIdxOr}}
\newcommand*{\RawMtsIdxRep}     {\Idxft{\StrMtsIdxRep}}

% Indexoperationenen
\newcommand*{\links} [1] {#1^{\scriptscriptstyle <}}% linkes  Element vom Paar
\newcommand*{\rechts}[1] {#1^{\scriptscriptstyle >}}% rechtes Element vom Paar

% abgeleitete Bereiche - ohne Link ins Glossar
\newcommand*{\RawMtsFol} {\Setft{\LtrMtsFol}}%  Bereich der Folgen             auf
\newcommand*{\RawMtsFolf}{\RawMtsFol_{\RawMtsIdxEndlich}}% ... nur die endlichen Folgen
\newcommand*{\RawMtsTup} {\Setft{\LtrMtsTup}}%  Bereich der Tupel              auf
\newcommand*{\RawMtsPot} {\Setft{\LtrMtsPot}}%  Bereich der Teilmengen         von
\newcommand*{\RawMtsPotf}{\RawMtsPot_{\RawMtsIdxEndlich}}%... nur die endlichen Teilmengen
\newcommand*{\RawMtsRel} {\Setft{\LtrMtsRel}}%  Bereich der binären Relationen auf
\newcommand*{\RawMtsRelf}{\RawMtsRel_{\RawMtsIdxEndlich}}%... nur die endlichen Relationen

% natürliche Zahlen u.a. - ohne Link ins Glossar
% alternativ: '{\fam5' statt '\mathbb{'
\newcommand*{\RawMtsIN}   {{\fam5\LtrMtsIN}}%  Menge der natürlichen Zahlen ohne 0
%\newcommand*{\RawMtsIN}{\mathbb{\LtrMtsIN}}%  Menge der natürlichen Zahlen ohne 0
\newcommand*{\RawMtsINo}        {\RawMtsIN_0}% Menge der natürlichen Zahlen mit  0
\newcommand*{\RawMtsMo}               {M^0}
\newcommand*{\RawMtsMn}               {M^n}

% weitere Bereiche
\newcommand*{\RawMtsUniversum}        {\Setft{\LtrMtsUniversum}}% Diskursuniversum
\newcommand*{\RawMtsObjekte}          {\Setft{\LtrMtsObjekte}}%   Bereich aller zulässigen Objekte
\newcommand*{\RawMtsAussagen}         {\Setft{\LtrMtsAussagen}}%  Bereich aller zulässigen Aussagen
\newcommand*{\RawMtsSprache}          {\Setft{\LtrMtsSprache}}%   Formel-Sprache
\newcommand*{\RawMtsRelAllDerive}     {\ensuremath{\RawMtsRel(\RawMtsPot(\RawMtsSprache))}}%   R(P(L))
% ... - mit Link ins Glossar
\newcommand*{\MtsPotSprache}          {\ensuremath{\MtsPot(\MtsSprache)}}%        P(L)
\newcommand*{\MtsPotfSprache}         {\ensuremath{\MtsPotf(\MtsSprache)}}%      Pe(L)
\newcommand*{\MtsAllDerive}           {\ensuremath{\MtsPotSprache^2}}%            P(L)^2
\newcommand*{\MtsPotAllDerive}        {\ensuremath{\MtsPot(\MtsAllDerive)}}%    P(P(L)^2)
\newcommand*{\MtsRelAllDerive}        {\ensuremath{\MtsRel(\MtsPotSprache)}}%   R(P(L))
\newcommand*{\MtsAllSchlussregel}     {\ensuremath{\MtsPotAllDerive^2}}%        P(P(L)^2)^2
\newcommand*{\MtsRelSchlussregel}     {\ensuremath{\MtsRel(\MtsRelAllDerive)}}% R(P(L)^2)
\newcommand*{\MtsPotfAllDerive}       {\ensuremath{\MtsPotf(\MtsAllDerive)}}%  Pe(P(L)^2)
\newcommand*{\MtsRelfAllDerive}       {\ensuremath{\MtsRelf(\MtsPotSprache)}}% Re(P(L))

% Elemente und Mengen für Beweise - ohne Link ins Glossar
\newcommand*{\RawMtsPraemisse}        {\drvft{\LtrMtsPraemisse}}%         eine      Prämisse
\newcommand*{\RawMtsPraemisseSet}     {\Setft{\LtrMtsPraemisseSet}}%      Menge der Prämissen
\newcommand*{\RawMtsPraemisseRel}     {\RawMtsDerive_{\RawMtsPraemisseSet}}%... als Relation
\newcommand*{\RawMtsKonklusion}       {\drvft{\LtrMtsKonklusion}}%        eine      Konklusion
\newcommand*{\RawMtsKonklusionSet}    {\Setft{\LtrMtsKonklusionSet}}%     Menge der Konklusionen
\newcommand*{\RawMtsKonklusionRel}    {\RawMtsDerive_{\RawMtsKonklusionSet}}%.. als Relation
\newcommand*{\RawMtsErgebnis}         {\drvft{\LtrMtsErgebnis}}%          ein       Ergebnis
\newcommand*{\RawMtsErgebnisSet}      {\Setft{\LtrMtsErgebnisSet}}%       Menge von Ergebnissen
\newcommand*{\RawMtsErgebnisRel}      {\RawMtsDerive_{\RawMtsErgebnisSet}}% ... als Relation
\newcommand*{\RawMtsBeweisschritt}    {\Elmft{\LtrMtsBeweisschritt}}%     ein       Beweisschritt
\newcommand*{\RawMtsBeweisschrittTup} {\vec{\RawMtsBeweisschritt}}%       Folge der Beweisschritte
\newcommand*{\RawMtsBeweisschrittSet} {\Setft{\LtrMtsBeweisschrittSet}}%  Menge der Beweisschritte
\newcommand*{\RawMtsTransformation}   {\Elmft{\LtrMtsTransformation}}%    eine      Transformation
\newcommand*{\RawMtsTransformationTup}{\Setft{\LtrMtsTransformation}}%    Folge von Transformationen
\newcommand*{\RawMtsSchlussregel}     {\Elmft{\LtrMtsSchlussregel}}%      eine      Schlussregel
\newcommand*{\RawMtsSchlussregelSet}  {\Setft{\RawMtsSchlussregel}}%      Menge von Schlussregeln
\newcommand*{\RawMtsErsetzung}        {\Elmft{\LtrMtsErsetzung}}%         eine      Ersetzung
\newcommand*{\RawMtsErsetzungSet}     {\Setft{\LtrMtsErsetzung}}%         Menge von Ersetzungen
\newcommand*{\RawMtsAxiom}            {\Elmft{\LtrMtsAxiom}}%             ein       Axiom
\newcommand*{\RawMtsAxiomSet}         {\Setft{\LtrMtsAxiom}}%             Menge von Axiomen

% Mengen der Aussagenlogik - ohne Link ins Glossar
\newcommand*{\RawOjkvar} {\varft{\LtrOjkvar}}% Variablensymbol
\newcommand*{\RawOjkVar} {\Setft{\LtrOjkVar}}% Menge der Variablensymbole
\newcommand*{\RawOjkABC} {\Setft{\LtrOjkABC}}% Menge der Buchstaben (Alphabet) der aussagenlogischen Sprache
\newcommand*{\RawOjkJun} {\Setft{\LtrOjkJun}}% Menge der Junktoren
\newcommand*{\RawOjkCon} {\RawOjkJun_{\RawMtsIdxCon}}% Menge der         Konstantensymbole
\newcommand*{\RawOjkUna} {\RawOjkJun_{\RawMtsIdxUna}}% Menge der unären  Operationssymbole
\newcommand*{\RawOjkBin} {\RawOjkJun_{\RawMtsIdxBin}}% Menge der binären Operationssymbole
\newcommand*{\RawOjkFor} {\Setft{\LtrOjkFor}^{\RawMtsIdxLogisch}}% Menge der aussagenlogischen Formeln
\newcommand*{\RawOjkForp}{\Setft{\LtrOjkFor}^{\RawMtsIdxLogisch\RawMtsIdxPolnisch}}% ...in Polnischer Notation

% Prädikate und praedikatähnliche Makros =======================================
% Mathmode; ohne Link ins Glossar
\newcommand*    {\RawMengeDef}[2]{\{ #1 \RawMtsSetSep #2 \}}
\newcommand*     {\QuantorDef}[3]{ #1 #2 : #3 }
\newcommand*     {\InfRuleDef}[2] {\frac{#1}{#2}}
\newcommand*    {\dInfRuleDef}[2]{\dfrac{#1}{#2}}
\newcommand*    {\tInfRuleDef}[2]{\tfrac{#1}{#2}}
\newcommand*{\RawMtsForAllDef}[2]{\QuantorDef{\RawMtsForAll}{#1}{#2}}
\newcommand*{\RawMtsExiOneDef}[2]{\QuantorDef{\RawMtsExiOne}{#1}{#2}}
\newcommand*{\RawMtsExistsDef}[2]{\QuantorDef{\RawMtsExists}{#1}{#2}}
\newcommand*{\RawOjkForAllDef}[2]{\QuantorDef{\RawOjkForAll}{#1}{#2}}
\newcommand*{\RawOjkExiOneDef}[2]{\QuantorDef{\RawOjkExiOne}{#1}{#2}}
\newcommand*{\RawOjkExistsDef}[2]{\QuantorDef{\RawOjkExists}{#1}{#2}}
% Mathmode; mit  Link ins Glossar
\newcommand*    {\FunktionDef}[3]   {\ensuremath{ #1 \MtsFktSep #2 \MtsFktArrow #3 }}% f:A->B
\newcommand*       {\MengeDef}[2]   {\ensuremath{\{ #1 \MtsSetSep #2 \}}}% {x|A(x)}
\newcommand*   {\MtsForAllDef}[2]   {\QuantorDef{\MtsForAll}{#1}{#2}}
\newcommand*   {\MtsExiOneDef}[2]   {\QuantorDef{\MtsExiOne}{#1}{#2}}
\newcommand*   {\MtsExistsDef}[2]   {\QuantorDef{\MtsExists}{#1}{#2}}
\newcommand*   {\OjkForAllDef}[2]   {\QuantorDef{\OjkForAll}{#1}{#2}}
\newcommand*   {\OjkExiOneDef}[2]   {\QuantorDef{\OjkExiOne}{#1}{#2}}
\newcommand*   {\OjkExistsDef}[2]   {\QuantorDef{\OjkExists}{#1}{#2}}


% sonstige Makro für den Mathematiksatz ########################################

\newcommand*{\card}[1]{|#1|}% Die Mächtigkeit einer Menge

\mathtoolsset{showonlyrefs,showmanualtags}% Nur mit \ref referenzierte Gleichungen, aber alle manuellen Tags

% Gleichung im Rahmen - siehe Rautenberg Seite 389
%%% #1=Rahmenfarbe, #2=Hintergrundfarbe, #3=mathematische Formel, #4=Label
%%%\makeatletter
%%%\def\myMathBox    {@ifnextchar[{\my@MBoxi}     {\my@MBoxii[black]}}
%%%\def\my@MBoxi [#1]{@ifnextchar[{\my@MBoxii[#1]}{\my@MBoxii[white]}}
%%%\def\my@MBoxii[#1][#2]#3#4{%
%%%	\par
%%%	\noindent\fcolorbox{#1}{#2}{%
%%%		\parbox{\linewidth-1.5\labelwidth-2\fboxrule-2\fboxsep{#3}}%
%%%	}%
%%%	\parbox{1.5\labelwidth}{%
%%%		\begin{eqnarray}\label{#4}\end{eqnarray}%
%%%	}
%%%	\par
%%%}
%%%\makeatother
