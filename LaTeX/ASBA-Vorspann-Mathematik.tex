%%############################################################################%%
%%                                                                            %%
%% Datei:  ASBA-Vorspann.tex                                                  %%
%% Inhalt: Vorspann für die Datei ASBA.txt                                    %%
%%                                                                            %%
%% Copyright (C) 2017  Winfried Teschers                                      %%
%%                                                                            %%
%% This program is free software: you can redistribute it and/or modify       %%
%% it under the terms of the GNU Affero General Public License as published   %%
%% by the Free Software Foundation, either version 3 of the License, or       %%
%% (at your option) any later version.                                        %%
%%                                                                            %%
%% This program is distributed in the hope that it will be useful,            %%
%% but WITHOUT ANY WARRANTY; without even the implied warranty of             %%
%% MERCHANTABILITY or FITNESS FOR A PARTICULAR PURPOSE.  See the              %%
%% GNU Affero General Public License for more details.                        %%
%%                                                                            %%
%% You should have received a copy of the GNU Affero General Public License   %%
%% along with this program.  If not, see <http://www.gnu.org/licenses/>.      %%
%%                                                                            %%
%% Dr. Winfried Teschers                                                      %%
%% Anton-Günther-Straße 26c                                                   %%
%% 91083 Baiersdorf                                                           %%
%% Germany                                                                    %%
%%                                                                            %%
%% e-mail: winfried.teschers@t-online.de                                      %%
%%                                                                            %%
%%############################################################################%%

% !TeX root = ASBA.tex
% !TeX encoding = UTF-8
% !TeX spellcheck = de_DE

% Elemente, die in anderen Dateien als "ASBA-Mathematik" verwendet werden, werden in "ASBA-Vorspann" definiert.

% Metasprachliche Symbole ######################################################

\newcommand*{\srand}{\mid}% in formalen Sätzen und Schlussregeln:   ... und ...
% Nur im Mathematikmodus!
\DeclareMathOperator{\metaand}{\&}%   ... und  ...
\DeclareMathOperator{\metaor}{||}%    ... oder ...
\DeclareMathOperator{\metaorr}{||\,}% ... oder ... (besserer Abstand)
\DeclareMathOperator{\metaimp}{\Rightarrow}%   aus ... folgt              ...
\DeclareMathOperator{\metarep}{\Leftarrow}%        ... folgt aus          ...
\DeclareMathOperator{\metaequiv}{\Leftrightarrow}% ... genau dann wenn    ...
\DeclareMathOperator{\metadefeq}{:\!\metaequiv}%   ... definitionsgemäß " ...
\DeclareMathOperator{\eq}{=}%                      ... gleich             ...
\DeclareMathOperator{\defeq}{:\!\eq}%              ... definitionsgemäß " ...

% Mathematische Symbole ########################################################

% Beispieloperatoren ===========================================================
% \*bsp
\DeclareMathOperator{\opbsp}{\bullet}
\DeclareMathOperator{\relbsp}{\sim}
\DeclareMathOperator{\relBsp}{\relbsp_1}
\DeclareMathOperator{\relnbsp}{\nsim}
\DeclareMathOperator{\releqbsp}{\simeq}
\DeclareMathOperator{\lrelbsp}{\lhd}
\DeclareMathOperator{\rrelbsp}{\rhd}
\DeclareMathOperator{\lreleqbsp}{\unlhd}
\DeclareMathOperator{\rreleqbsp}{\unrhd}

% Definitionen für die Tabelle der Junktoren ===================================
% \l*  -           logischer Operator
% \ln* - negierter logischer Operator
% Logische Operatoren als Addition und Multiplikation
\DeclareMathOperator{\ladd}{+}
\DeclareMathOperator{\lmult}{\cdot}
% Wahrheitswerte ---------------------------------------------------------------
\newcommand*{\texttrue}{W}%  in einem Kommentar stets 'W'
\newcommand*{\textfalse}{F}% in einem Kommentar stets 'F'
% Konstante --------------------------------------------------------------------
\newcommand*{\ltrue}{\top}%      W - wahr
%\newcommand*{\lnfalse}{\notbot}% " - nicht falsch
\newcommand*{\lfalse}{\bot}%     F - falsch
%\newcommand*{\lntrue}{\nottop}%  " - nicht wahr
% unäre Operatoren -------------------------------------------------------------
%                                                        W F - Aussage A
%\DeclareMathOperator{\lutrue}{\operatorname{\top}}%     W W - wahr
%\DeclareMathOperator{\lnufalse}{\operatorname{\notbot}}%" " - nicht falsch
%                                                        W F - A
%             \lnot                                      F W - nicht
%\DeclareMathOperator{\lufalse}{\operatorname{\bot}}%    F F - falsch
%\DeclareMathOperator{\lnutrue}{\operatorname{\nottop}}% " " - nicht wahr
% binäre Operatoren ------------------------------------------------------------
%                                                        W W F F - Aussage A
%                                                        W F W F - Aussage B
%  - - - - - - - - - - - - - - - - - - - - - - - - - - - - - - - - - - - - - - -
%\DeclareMathOperator{\lbtrue}{\operatorname{\top}}%     W W W W - wahr [binär]
%\DeclareMathOperator{\lnbfalse}{\operatorname{\notbot}}%" " " " - nicht falsch
%            \lor                                        W W W F - A oder B
\DeclareMathOperator{\lrep}{\leftarrow}%                 W W F W - A folgt aus B
\DeclareMathOperator{\lrepA}{\Leftarrow}%
\DeclareMathOperator{\lrepB}{\subset}%
\DeclareMathOperator{\lleft}{\operatorname{\rfloor}}%    W W F F - A
%  - - - - - - - - - - - - - - - - - - - - - - - - - - - - - - - - - - - - - - -
\DeclareMathOperator{\limp}{\rightarrow}%                W F W W - aus A folgt B
\DeclareMathOperator{\limpA}{\Rightarrow}%
\DeclareMathOperator{\limpB}{\supset}%
\DeclareMathOperator{\lright}{\operatorname{\lfloor}}%   W F W F - B
\DeclareMathOperator{\lequiv}{\leftrightarrow}%          W F F W - A genau dann,
\DeclareMathOperator{\lequivA}{\Leftrightarrow}%                    wenn B
%            \lnxor                                      " " " " - nicht
%                                                            (entweder A oder B)
%            \land                                       W F F F - A und B
\DeclareMathOperator{\landA}{\&}
\DeclareMathOperator{\landB}{\lmult}
%  - - - - - - - - - - - - - - - - - - - - - - - - - - - - - - - - - - - - - - -
\DeclareMathOperator{\lnand}{\uparrow}%                  F W W W - nicht
\DeclareMathOperator{\lnandA}{\barwedge}%                          (A und B)
\DeclareMathOperator{\lnandB}{\mid}%
\DeclareMathOperator{\lxor}{\ladd}%                      F W W F - entweder A
\DeclareMathOperator{\lxorA}{\operatorname{\dot\lor}}%                 oder B
\DeclareMathOperator{\lxorB}{\veebar}%
\DeclareMathOperator{\lxorC}{\oplus}%
\DeclareMathOperator{\lnequiv}{\nleftrightarrow}%        " " " " - nicht
\DeclareMathOperator{\lnequivA}{\nLeftrightarrow}%        (A genau dann, wenn B)
\DeclareMathOperator{\lnequivB}{\nequiv}%
\DeclareMathOperator{\lnright}{\lceil}%                  F W F W - nicht B
\DeclareMathOperator{\lnimp}{\nrightarrow}%              F W F F - nicht
\DeclareMathOperator{\lnimpA}{\nRightarrow}%                     (aus A folgt B)
\DeclareMathOperator{\lnimpB}{\nsupset}%
%  - - - - - - - - - - - - - - - - - - - - - - - - - - - - - - - - - - - - - - -
\DeclareMathOperator{\lnleft}{\rceil}%                   F F W W - nicht A
\DeclareMathOperator{\lnrep}{\nleftarrow}%               F F W F - nicht
\DeclareMathOperator{\lnrepA}{\nLeftarrow}%                      (A folgt aus B)
\DeclareMathOperator{\lnrepB}{\nsubset}%
\DeclareMathOperator{\lnor}{\downarrow}%                 F F F W - nicht
\DeclareMathOperator{\lnorA}{\operatorname{\overline\vee}}%        (A oder B)
%\DeclareMathOperator{\lbfalse}{\operatorname{\bot}}%    F F F F - falsch[binär]
%\DeclareMathOperator{\lnbtrue}{\operatorname{\nottop}}% " " " " - nicht wahr

% Verwendete Mengenbezeichnungen ===============================================

% \gs* = globales Symbol
\newcommand*{\gsN}{\mathbb{N}}%    Menge der natürlichen Zahlen ohne 0
\newcommand*{\gsNo}{\mathbb{N}_0}% Menge der natürlichen Zahlen einschließlich 0

% Elemente und Mengen für Beweise
\newcommand*{\Voraussetzung}{R}
\newcommand*{\Beweisschritt}{S}
\newcommand*{\Folgerung}    {T}
\newcommand*{\Voraussetzungen}{\mathcal{\Voraussetzung}}
\newcommand*{\Beweisschritte} {\mathcal{\Beweisschritt}}
\newcommand*{\Folgerungen}    {\mathcal{\Folgerung}}

% \as* = aussagenlogisches Symbol
\newcommand*{\asA}{\mathcal{A}}%       Alphabet der Sprache
\newcommand*{\asAx}{\mathcal{A}_x}%    ... davon eine Teilmenge bzgl. \asJx
\newcommand*{\asAy}{\mathcal{A}_y}%    entsprechend \asAx
\newcommand*{\asB}{\mathcal{B}}%       Menge der binären Operatoren
\newcommand*{\asC}{\mathcal{C}}%       Menge der Konstanten
\newcommand*{\asF}{\mathcal{F}}%       Menge der Formeln
\newcommand*{\asFp}{\mathcal{F}^p}%    ... in polnischer Notation
\newcommand*{\asFx}{\mathcal{F}_x}%    Teilenge der Formeln
\newcommand*{\asFxp}{\mathcal{F}_x^p}% ... bzgl. \asJx in polnischer Notation
\newcommand*{\asFy}{\mathcal{F}_y}%    entsprechend \asFx
\newcommand*{\asFyp}{\mathcal{F}_y^p}% entsprechend \asFxp
\newcommand*{\asJ}{\mathcal{J}}%       Menge der Junktoren
\newcommand*{\asJx}{\mathcal{J}_x}%    Teilmenge der Junktoren
\newcommand*{\asJy}{\mathcal{J}_y}%    entsprechend \asJx
\newcommand*{\asM}{\mathcal{M}}%       Metaoperatoren und "Gleichheiten"
\newcommand*{\asU}{\mathcal{U}}%       Menge der unären Operatoren
\newcommand*{\asV}{\mathcal{V}}%       Menge der atomaren Formeln
\newcommand*{\asX}{\mathcal{X}}%       Mengenvariable
% verschiedene Indizes für Teilmengen von \asA, \asF, \asFp und \asJ
\newcommand*{\xAnd}{\mathrm{and}}%
\newcommand*{\xBool}{\mathrm{bool}}%
\newcommand*{\xImp}{\mathrm{imp}}%
\newcommand*{\xNand}{\mathrm{nand}}%
\newcommand*{\xNor}{\mathrm{nor}}%
\newcommand*{\xOr}{\mathrm{or}}%
\newcommand*{\xRep}{\mathrm{rep}}%

% sonstige mathematische Zeichen ===============================================

\newcommand*{\textderive}{$\vdash$}%               im Textmodus
\DeclareMathOperator{\derive}{\;\vdash\;}%         ... ableitbar ...
\DeclareMathOperator{\derivegls}{\vdash}%          -"- für Glossar
\DeclareMathOperator{\subst}{\leftarrowtail}%      ... substituiert durch ...
\DeclareMathOperator{\swap}{\leftrightarrows}%     ... vertauscht mit ...
%%%\DeclareMathOperator{\subst}{\curvearrowright}% ... substituiert durch ...

% sonstige Kommandos für den Mathematiksatz ####################################

\mathtoolsset{showonlyrefs,showmanualtags}% Nur mit \ref referenzierte Gleichungen, aber alle manuellen Tags

% Glossareinträge ##############################################################

% Symbole für Beispieloperatoren -----------------------------------------------

\newglossaryentry{lrelbsp}{
	name={$ \lrelbsp$},
	plural={\lrelbsp},% im Mathematikmodus
	description={%
		Ein Beispielsymbol für eine Relation mit Umkehrrelation $\rrelbsp$%
	}
}
\newglossaryentry{lreleqbsp}{
	name={$ \lreleqbsp$},
	plural={\lreleqbsp},% im Mathematikmodus
	description={%
		Ein Beispielsymbol für eine Relation mit Gleichheit und Umkehrrelation $\rreleqbsp$%
	}
}
\newglossaryentry{relbsp}{
	name={$ \relbsp$},
	plural={\relbsp},% im Mathematikmodus
	description={%
		Ein Beispielsymbol für eine Relation%
	}
}
\newglossaryentry{releqbsp}{
	name={$ \releqbsp$},
	plural={\releqbsp},% im Mathematikmodus
	description={%
		Ein Beispielsymbol für eine Relation mit Gleichheit%
	}
}
\newglossaryentry{relnbsp}{
	name={$ \relnbsp$},
	plural={\relnbsp},% im Mathematikmodus
	description={%
		Verneinung von $\relbsp$%
	}
}
\newglossaryentry{rrelbsp}{
	name={$ \rrelbsp$},
	plural={\rrelbsp},% im Mathematikmodus
	description={%
		Ein Beispielsymbol für eine Relation mit Umkehrrelation $\lrelbsp$%
	}
}
\newglossaryentry{rreleqbsp}{
	name={$ \rreleqbsp$},
	plural={\rreleqbsp},% im Mathematikmodus
	description={%
		Ein Beispielsymbol für eine Relation mit Gleichheit und Umkehrrelation $\lreleqbsp$%
	}
}

% Symbole für Metaoperatoren ---------------------------------------------------

\newglossaryentry{defeq}{
	name={$ \defeq$},
	plural={\defeq},% im Mathematikmodus
	description={%
		Ein \emph{Metaoperator}: ... definitionsgemäß gleich ..%
	}
}
\newglossaryentry{eq}{
	name={$ \eq$},
	plural={\eq},% im Mathematikmodus
	description={%
		Ein \emph{Metaoperator}: ... gleich (ist dasselbe wie, ist identisch zu) ..%
	}
}
\newglossaryentry{equiv}{
	name={$ \equiv$},
	plural={\equiv},% im Mathematikmodus
	description={%
		Ein (Meta-)Operator: ... äquivalent (ist das gleiche wie, ist so wie) zu ..%
	}
}
\newglossaryentry{metaand}{
	name={$ \metaand$},
	plural={\metaand},% im Mathematikmodus
	description={%
		Ein \emph{Metaoperator}: ... und ...
		Die Priorität ist höher als die von \symqt{$\srand$}%
	}
}
\newglossaryentry{metadefeq}{
	name={$ \metadefeq$},
	plural={\metadefeq},% im Mathematikmodus
	description={%
		Ein \emph{Metaoperator}: ... definitionsgemäß gleich (definitionsgemäß genau dann, wenn) ..%
	}
}
\newglossaryentry{metaequiv}{
	name={$ \metaequiv$},
	plural={\metaequiv},% im Mathematikmodus
	description={%
		Ein \emph{Metaoperator}: ... genau dann wenn ..%
	}
}
\newglossaryentry{metaimp}{
	name={$ \metaimp$},
	plural={\metaimp},% im Mathematikmodus
	description={%
		Ein \emph{Metaoperator}: ... dann auch ..%
	}
}
\newglossaryentry{metaor}{
	name={$ \metaor$},
	plural={\metaor},% im Mathematikmodus
	description={%
		Ein \emph{Metaoperator}: ... oder ..%
	}
}
\newglossaryentry{metarep}{
	name={$ \metarep$},
	plural={\metarep},% im Mathematikmodus
	description={%
		Ein \emph{Metaoperator}: ... sofern ..%
	}
}
\newglossaryentry{ne}{
	name={$ \ne$},
	plural={\ne},% im Mathematikmodus
	description={%
		Ein \emph{Metaoperator}: ... ungleich (nicht dasselbe wie, nicht identisch zu) ..%
	}
}
\newglossaryentry{nequiv}{
	name={$ \nequiv$},
	plural={\nequiv},% im Mathematikmodus
	description={%
		Ein (Meta-)Operator: ... nicht äquivalent (ist nicht das gleiche wie, ist nicht so wie) ..%
	}
}
\newglossaryentry{subst}{
	name={$ \subst$},
	plural={\subst},% im Mathematikmodus
	description={%
		\emph{Substitution}: ... substituiert durch ...
		-- siehe die Definition \vrefinsub{sub:Indentitaetsregeln}%
	}
}
\newglossaryentry{swap}{
	name={$ \swap$},
	plural={\swap},% im Mathematikmodus
	description={%
		\emph{Vertauschung}: ... vertauscht mit ...
		-- siehe die Definition \vrefinsub{sub:Indentitaetsregeln}%
	}
}
\newglossaryentry{srand}{
	name={$ \srand$},
	plural={\srand},% im Mathematikmodus
	description={%
		Ein \emph{Metaoperator}: ... und ...
		Die Priorität ist niedriger als die von \symqt{$\metaand$}%
	}
}

% sonstige mathematische Symbole -----------------------------------------------

\newglossaryentry{derive}{
	name={$ \derivegls$},
	plural={\derivegls},% im Mathematikmodus
	description={%
		Ableitungsrelation: ... ableitbar ...
		-- siehe \emph{ableitbar}%
	}
}
\newglossaryentry{lfalse}{
	name={$ \lfalse$},
	plural={\lfalse},% im Mathematikmodus
	description={%
		Eine Aussagenlogische Konstante: Falsch%
	}
}
\newglossaryentry{ltrue}{
	name={$ \ltrue$},
	plural={\ltrue},% im Mathematikmodus
	description={%
		Eine Aussagenlogische Konstante: Wahr%
	}
}

% Symbole für Mengen -----------------------------------------------------------

\newglossaryentry{gsN}{
	name={$ \gsN$},
	plural={\gsN},% im Mathematikmodus
	description={%
		Die Menge der natürlichen Zahlen ohne 0%
	}
}
\newglossaryentry{gsNo}{
	name={$ \gsNo$},
	plural={\gsNo},% im Mathematikmodus - erfolgt im falschen Zeichensatz! ???
	description={%
		Die Menge der natürlichen Zahlen einschließlich 0%
	}
}
\newglossaryentry{asA}{
	name={$ \asA$},
	plural={\asA},% im Mathematikmodus
	description={%
		Das Alphabet der aussagenlogischen Sprache%
	}
}
\newglossaryentry{asAx}{
	name={$ \asAx$},
	plural={\asAx},% im Mathematikmodus
	description={%
		Eine Teilmenge des Alphabets $\asA$ der aussagenlogischen Sprache%
	}
}
\newglossaryentry{asB}{
	name={$ \asB$},
	plural={\asB},% im Mathematikmodus
	description={%
		Die Menge der aussagenlogischen, binären Operatoren%
	}
}
\newglossaryentry{asC}{
	name={$ \asC$},
	plural={\asC},% im Mathematikmodus
	description={%
		Die Menge der aussagenlogischen Konstanten%
	}
}
\newglossaryentry{asF}{
	name={$ \asF$},
	plural={\asF},% im Mathematikmodus
	description={%
		Die Menge der aussagenlogischen Formeln mit Klammerung%
	}
}
\newglossaryentry{asFp}{
	name={$ \asFp$},
	plural={\asFp},% im Mathematikmodus
	description={%
		Die Menge der aussagenlogischen Formeln in polnischer Notation%
	}
}
\newglossaryentry{asFx}{
	name={$ \asFx$},
	plural={\asFx},% im Mathematikmodus
	description={%
		Eine Teilmenge der Menge $\asF$ der aussagenlogischen Formeln mit Klammerung%
	}
}
\newglossaryentry{asFxp}{
	name={$ \asFxp$},
	plural={\asFxp},% im Mathematikmodus
	description={%
		Eine Teilmenge der Menge $\asF$ der aussagenlogischen Formeln in polnischer Notation%
	}
}
\newglossaryentry{asJ}{
	name={$ \asJ$},
	plural={\asJ},% im Mathematikmodus
	description={%
		Die Menge der aussagenlogischen Operatoren%
	}
}
\newglossaryentry{asJx}{
	name={$ \asJx$},
	plural={\asJx},% im Mathematikmodus
	description={%
		Eine Teilmenge der Menge $\asJ$ der aussagenlogischen Operatoren
	}
}
\newglossaryentry{asM}{
	name={$ \asM$},
	plural={\asM},% im Mathematikmodus
	description={%
		Die Menge der \emph{Metaoperatoren} und der mit Gleichheit verwandten Symbole%
	}
}
\newglossaryentry{asU}{
	name={$ \asU$},
	plural={\asU},% im Mathematikmodus
	description={%
		Die Menge der aussagenlogischen unären Operatoren%
	}
}
\newglossaryentry{asV}{
	name={$ \asV$},
	plural={\asV},% im Mathematikmodus
	description={%
		Die Menge der aussagenlogischen \emph{atomaren Formeln}%
	}
}

% Schlussregeln ----------------------------------------------------------------

\newcommand*{\tagAR}{AR}% Argument für \tag - im Textmodus
\newcommand*{\AR}{(\text{AR})}% im Mathematikmodus
\newglossaryentry{AR}{
	name={$ \AR$},
	plural={\AR},% im Mathematikmodus
	description={%
		\emph{Anfangsregel}%
	}
}
\newcommand*{\tagFS}{FS}% Argument für \tag - im Textmodus
\newcommand*{\FS}{(\text{FS})}% im Mathematikmodus
\newglossaryentry{FS}{
	name={$ \FS$},
	plural={\FS},% im Mathematikmodus
	description={%
		\emph{formaler Satz}%
	}
}
\newcommand*{\tagMR}{MR}% Argument für \tag - im Textmodus
\newcommand*{\MR}{(\text{MR})}% im Mathematikmodus
\newglossaryentry{MR}{
	name={$ \MR$},
	plural={\MR},% im Mathematikmodus
	description={%
		\emph{Monotonieregel}%
	}
}
\newcommand*{\tagSR}{SR}% Argument für \tag - im Textmodus
\newcommand*{\SR}{(\text{SR})}% im Mathematikmodus
\newglossaryentry{SR}{
	name={$ \SR$},
	plural={\SR},% im Mathematikmodus
	description={%
		\emph{Schnittregel} (Modus ponens)%
	}
}
\newcommand*{\tagTR}{TR}% Argument für \tag - im Textmodus
\newcommand*{\TR}{(\text{TR})}% im Mathematikmodus
\newglossaryentry{TR}{
	name={$ \TR$},
	plural={\TR},% im Mathematikmodus
	description={%
		\emph{Abtrennungsregel}%
	}
}
\newcommand*{\tageqB}{$\eq$B}% Argument für \tag - im Textmodus
\newcommand*{\eqB}{(\eq\text{B})}% im Mathematikmodus
\newglossaryentry{eqB}{
	name={($\eqB$)},
	plural={\eqB},% im Mathematikmodus
	description={%
		Beseitigung von \symqt{$\eq$}%
	}
}
\newcommand*{\tageqE}{$\eq$E}% Argument für \tag - im Textmodus
\newcommand*{\eqE}{(\eq\text{E})}% im Mathematikmodus
\newglossaryentry{eqE}{
	name={($\eqE$)},
	plural={\eqE},% im Mathematikmodus
	description={%
		Einführung von \symqt{$\eq$}%
	}
}
\newcommand*{\tagandB}{$\land$B}% Argument für \tag - im Textmodus
\newcommand*{\andB}{(\land\text{B})}% im Mathematikmodus
\newglossaryentry{andB}{
	name={($\andB$)},
	plural={\andB},% im Mathematikmodus
	description={%
		Beseitigung von \symqt{$\andB$}%
	}
}
\newcommand*{\tagandE}{$\land$E}% Argument für \tag - im Textmodus
\newcommand*{\andE}{(\land\text{E})}% im Mathematikmodus
\newglossaryentry{andE}{
	name={($\andE$)},
	plural={\andE},% im Mathematikmodus
	description={%
		Beseitigung von \symqt{$\andE$}%
	}
}
\newcommand*{\tagimpB}{$\limp$B}% Argument für \tag - im Textmodus
\newcommand*{\impB}{(\limp\text{B})}% im Mathematikmodus
\newglossaryentry{impB}{
	name={($\impB$)},
	plural={\impB},% im Mathematikmodus
	description={%
		Beseitigung von \symqt{$\impB$}%
	}
}
\newcommand*{\tagimpE}{$\limp$E}% Argument für \tag - im Textmodus
\newcommand*{\impE}{(\limp\text{E})}% im Mathematikmodus
\newglossaryentry{impE}{
	name={($\impE$)},
	plural={\impE},% im Mathematikmodus
	description={%
		Beseitigung von \symqt{$\impE$}%
	}
}
\newcommand*{\tagnota}{$\lnot$1}% Argument für \tag - im Textmodus
\newcommand*{\nota}{(\lnot\text{1})}% im Mathematikmodus
\newglossaryentry{nota}{
	name={($\nota$)},
	plural={\nota},% im Mathematikmodus
	description={%
		Einführung/Beseitigung von \symqt{$\lnot$} Teil 1%
	}
}
\newcommand*{\tagnotb}{$\lnot$2}% Argument für \tag - im Textmodus
\newcommand*{\notb}{(\lnot\text{2})}% im Mathematikmodus
\newglossaryentry{notb}{
	name={($\notb$)},
	plural={\notb},% im Mathematikmodus
	description={%
		Einführung/Beseitigung von \symqt{$\lnot$} Teil 1%
	}
}
\newcommand*{\tagnotc}{$\lnot$3}% Argument für \tag - im Textmodus
\newcommand*{\notc}{(\lnot\text{3})}% im Mathematikmodus
\newglossaryentry{notc}{
	name={($\notc$)},
	plural={\notc},% im Mathematikmodus
	description={%
		Beweistechnik \strqt{Indirekter Beweis}%
	}
}
\newcommand*{\tagnotd}{$\lnot$4}% Argument für \tag - im Textmodus
\newcommand*{\notd}{(\lnot\text{4})}% im Mathematikmodus
\newglossaryentry{notd}{% statt "notE"
	name={($\notd$)},
	plural={\notd},% im Mathematikmodus
	description={%
		Reductio ad absurdum (indirekter Beweis)%
	}
}
%%%\newcommand*{\tagorB}{$\lor$B}% Argument für \tag - im Textmodus
%%%\newcommand*{\orB}{(\lor\text{B})}% im Mathematikmodus
%%%\newglossaryentry{orB}{
%%%	name={($\obB$)},
%%%	plural={\orB},% im Mathematikmodus
%%%	description={%
%%%		Beseitigung von \symqt{$\lor$}%
%%%	}
%%%}
%%%\newcommand*{\tagorE}{$\lor$E}% Argument für \tag - im Textmodus
%%%\newcommand*{\orE}{(\lor\text{E})}% im Mathematikmodus
%%%\newglossaryentry{orE}{
%%%	name={($\obE$)},
%%%	plural={\orE},% im Mathematikmodus
%%%	description={%
%%%		Beseitigung von \symqt{$\lor$}%
%%%	}
%%%}

% Fachbegriffe -----------------------------------------------------------------

\newglossaryentry{ableitbar}{
	name={ableitbar},
	plural={ableitbare},
	description={%
		Wenn sich eine Formel $\beta$ aus einer Formel $\alpha$ mittels zulässiger Transaktionen ableiten lässt, heißt $\beta$ ableitbar aus $\alpha$.
		Sprechweise: \forqt{$\alpha$ ableitbar $\beta$}.
		Eine oder beide Formeln $\alpha$ \textbzw\ $\beta$ dürfen dabei durch Formelmengen ersetzt werden.
		-- siehe \emph{Ableitungsrelation} und \symqt{$\derive$}
		\newline
		Synonym: \emph{beweisbar}%
	}
}
\newglossaryentry{Ableitungsrelation}{
	name={Ableitungsrelation},
	plural={Ableitungsrelationen},
	description={%
		Die Relation \symqt{$\derive$}%
	}
}
\newglossaryentry{Abtrennungsregel}{
	name={Abtrennungsregel},
	plural={Abtrennungsregeln},
	description={%
		Eine \emph{Schlussregel} -- siehe~\emph{TR}%
	}
}
\newglossaryentry{allgemeingueltigeSchlussregelV}{
	name={allgemeingültige Schlussregel},    % eine  ...
	plural={allgemeingültige Schlussregeln}, % viele ..
	description={%
		Eine \emph{Schlussregel} die aus den \emph{Basisregeln} und den schon bekannten allgemeingültigen Schlussregeln abgeleitet werden kann%
	}
}
\newglossaryentry{Anfangsregel}{
	name={Anfangsregel},
	plural={Anfangsregeln},
	description={%
		Eine \emph{Schlussregel} um beginnen zu können -- siehe~\emph{AR}%
	}
}
\newglossaryentry{atomareFormelA}{
	name={atomare Formel},     % eine ...
	plural={atomaren Formeln}, % alle ...
	description={%
		Eine Formel, die sich nicht weiter zerlegen lässt%
	}%
}
\newglossaryentry{Aussage}{
	name={Aussage},
	plural={Aussagen},
	description={%
		Eine Aussage in natürlicher Sprache oder als Formel, die einen \emph{Wahrheitswert} liefert%
	}
}
\newglossaryentry{Aussagenlogik}{
	name={Aussagenlogik},
	description={%
		\vrefseesec{sec:Aussagenlogik}%
	}
}
\newglossaryentry{Basisregel}{
	name={Basisregel},
	plural={Basisregeln},
	description={%
		Eine \emph{Schlussregel}, die nicht mehr auf andere zurückgeführt wird.
		Obwohl das auch auf die \emph{Identitätsregeln} zutrifft, werden diese hier aber nicht dazu gezählt%
	}
}
\newglossaryentry{beweisbar}{
	name={beweisbar},
	plural={beweisbare},
	description={%
		Synonym zu \emph{ableitbar}%
	}
}
\newglossaryentry{Beweisschritt}{
	name={Beweisschritt},
	plural={Beweisschritte},
	description={%
		Eine Vorschrift, wie aus vorgegebenen Aussagen eine weitere folgt%
	}
}
\newglossaryentry{BoolscheSignatur}{
	name={Boolsche Signatur},
	plural={Boolschen Signatur},% Dativ
	description={%
		Die \emph{logische Signatur} $\{\lnot, \land, \lor\}$%
	}
}
\newglossaryentry{Folgerung}{
	name={Folgerung},
	plural={Folgerungen},
	description={%
		Die Folgerungen einer \emph{Schlussregel} sind die \emph{Aussagen} über ihrem Querstrich.
	}
}
\newglossaryentry{formalerSatzV}{
	name={formaler Satz},   % ein   ...
	plural={formale Sätze}, % viele ...
	description={%
		Formale Darstellung eines mathematischen Satzes -- siehe~\emph{FS}%
	}
}
\newglossaryentry{formalesElementV}{
	name={formales Element},   % ein   ...
	plural={formale Elemente}, % viele ...
	description={%
		Ein mathematisches Element in formaler Schreibweise.
		Bis auf wenige Aussagen kommen darin \emph{Metaausdrücke} nicht mehr vor%
	}%
}
\newglossaryentry{Identitaetsregel}{
	name={Identitätsregel},
	plural={Identitätsregeln},
	description={%
		Eigentlich eine \emph{Basisregel} zur Identität.
		Da die Identitätsregeln nur zur Rechtfertigung der \emph{Substitution} verwendet werden, werden sie hier nicht zu den \emph{Basisregeln} gezählt%
	}
}
\newglossaryentry{intEigenschaftA}{
	name={interessierende Eigenschaft},      % eine ...
	plural={interessierenden Eigenschaften}, % alle ...
	description={%
		Solche Eigenschaften von Ausdrücken, die im aktuellen Zusammenhang von Interesse sind%
	}%
}
\newglossaryentry{Kontraposition}{
	name={Kontraposition},
	plural={Kontraposition},
	description={%
		Die allgemeingültige Aussage $(\alpha \limp \beta) \limp (\lnot\beta \limp \lnot\alpha)$%
	}%
}
\newglossaryentry{logischeSignaturV}{
	name={logische Signatur},     % eine  ...
	plural={logische Signaturen}, % viele ...
	description={%
		Eine in \emph{Metasprache} verfasste Aussage, die auch zusammengesetzt sein kann%
	}%
}
\newglossaryentry{MetaausdruckV}{
	name={metasprachlicher Ausdruck},   % ein   ...
	plural={metasprachliche Ausdrücke}, % viele ...
	description={%
		Eine in normaler Sprache verfasste Aussage, die auch zusammengesetzt sein kann%
	}%
}
\newglossaryentry{MetaaussageV}{
	name={metasprachliche Aussage},    % eine   ...
	plural={metasprachliche Aussagen}, % viele ...
	description={%
		Eine in \emph{Metasprache} verfasste Aussage, die auch zusammengesetzt sein kann%
	}%
}
\newglossaryentry{MetaoperatorV}{
	name={metasprachlicher Operator},    % ein   ...
	plural={metasprachliche Operatoren}, % viele ...
	description={%
		Ein Operator, dessen Operanden \emph{Metaausdrücke} sind%
	}%
}
\newglossaryentry{Metasprache}{
	name={Metasprache},
	plural={Metasprachen},
	description={%
		Eine Sprache, in der Aussagen über Elemente einer anderen Sprache getroffen werden können%
	}%
}
\newglossaryentry{Monotonieregel}{
	name={Monotonieregel},
	plural={Monotonieregeln},
	description={%
		Eine \emph{Schlussregel} -- siehe~\emph{MR}%
	}
}
\newglossaryentry{Praedikat}{
	name={Prädikat},
	plural={Prädikate},
	description={%
		Ein Element der \emph{Prädikatenlogik} (\vrefseesec{sec:Praedikatenlogik}).
		\textZB\ kann man eine \forqt{$Gruppe$} als ein zweistelliges Prädikat \forqt{$Gruppe(G,+)$} definieren, in dem $G$ eine Menge und \symqt{$+$} eine Operation, \textdh\ eine (zweistellige) Funktion \forqt{$+: G \times G \rightarrow G$} ist, so dass die Gruppenaxiome erfüllt sind%
	}
}
\newglossaryentry{Praedikatenlogik}{
	name={Prädikatenlogik},
	description={%
		\vrefseesec{sec:Praedikatenlogik}%
	}
}
\newglossaryentry{Schlussregel}{
	name={Schlussregel},
	plural={Schlussregeln},
	description={%
		Eine Regel für eine (zulässige) Umwandlung von Formeln%
	}
}
\newglossaryentry{Schnittregel}{
	name={Schnittregel},
	plural={Schnittregeln},
	description={%
		Eine \emph{allgemeingültige Schlussregel} -- siehe~\emph{SR}%
	}
}
\newglossaryentry{Substitution}{
	name={Substitution},
	plural={Substitutionen},
	description={%
		Die Ersetzung von einem, mehreren oder allen \emph{formalen Elementen} ($\alpha$) in einem anderen \emph{formalen Element} ($\gamma$) durch ein drittes \emph{formales Element} ($\beta$)
		-- formal: $\gamma(\alpha\subst\beta)$.
		Wenn alle $\alpha$ in $\gamma$ durch $\beta$ ersetzt werden, ist die Substitution \emph{vollständig}.
		(\vrefseesub{sub:Indentitaetsregeln})%
	}
}
\newglossaryentry{vergleichbar}{
	name={vergleichbar},
	plural={vergleichbare},
	description={
		Zwei \emph{Metaausdrücke} \textbzw\ \emph{formale Elemente} heißen -- auf eine bestimmte Art -- \emph{vergleichbar}, wenn sie auf diese Art (\textzB\ als Zeichenketten oder als vergleichbare Ergebnisse von Formeln) verglichen werden können.
		Die Art muss implizit bekannt oder explizit angegeben sein.
		Meistens genügt es zu wissen, was für \emph{Metaausdrücke} \textbzw\ \emph{formale Elemente} es sind.
		Sie müssen dann nur von derselben Art sein%
	}
}
\newglossaryentry{Vertauschung}{
	name={Vertauschung},
	plural={Vertauschungen},
	description={%
		Die Vertauschung von zwei unabhängigen \emph{formalen Elementen} ($\alpha$ und $\beta$) in einem anderen \emph{formalen Element} ($\gamma$)
		-- formal: $\gamma(\alpha\swap\beta)$.
		Die Vertauschung ist eine spezielle Form der \emph{Substitution}.
		-- siehe die Definition~\eqref{def:Vertauschung} \vrefinsub{sub:Indentitaetsregeln}%
	}
}
\newglossaryentry{Voraussetzung}{
	name={Voraussetzung},
	plural={Voraussetzungen},
	description={%
		Die Voraussetzungen einer \emph{Schlussregel} sind die \emph{Aussagen} über ihrem Querstrich.
	}
}
\newglossaryentry{Wahrheitswert}{
	name={Wahrheitswert},
	plural={Wahrheitswerte},
	description={%
		Wahrheitswerte sind die Werte \strqt{wahr} und \strqt{falsch}, oft auch als \strqt{true} und \strqt{false} oder einfach \charqt{1} und \charqt{0} bezeichnet%
	}
}
\newglossaryentry{zulaessigeTransformationA}{
	name={zulässige Transformation},      % eine ...
	plural={zulässigen Transformationen}, % alle ...
	description={%
		Eine zulässige Umformung oder Erzeugung einer Formel aus einer vorgegebenen Menge von Formeln,
		\textdh\ die Anwendung einer \emph{allgemeingültigen Schlussregel}%
	}
}
