%%############################################################################%%
%%                                                                            %%
%% Datei:  ASBA-Vorspann.tex                                                  %%
%% Inhalt: Vorspann für die Datei ASBA.txt                                    %%
%%                                                                            %%
%% Copyright (C) 2017  Winfried Teschers                                      %%
%%                                                                            %%
%% This program is free software: you can redistribute it and/or modify       %%
%% it under the terms of the GNU Affero General Public License as published   %%
%% by the Free Software Foundation, either version 3 of the License, or       %%
%% (at your option) any later version.                                        %%
%%                                                                            %%
%% This program is distributed in the hope that it will be useful,            %%
%% but WITHOUT ANY WARRANTY; without even the implied warranty of             %%
%% MERCHANTABILITY or FITNESS FOR A PARTICULAR PURPOSE.  See the              %%
%% GNU Affero General Public License for more details.                        %%
%%                                                                            %%
%% You should have received a copy of the GNU Affero General Public License   %%
%% along with this program.  If not, see <http://www.gnu.org/licenses/>.      %%
%%                                                                            %%
%% Dr. Winfried Teschers                                                      %%
%% Anton-Günther-Straße 26c                                                   %%
%% 91083 Baiersdorf                                                           %%
%% Germany                                                                    %%
%%                                                                            %%
%% e-mail: winfried.teschers@t-online.de                                      %%
%%                                                                            %%
%%############################################################################%%

% !TeX root = ASBA.tex
% !TeX encoding = UTF-8
% !TeX spellcheck = de_DE

% Elemente, die in anderen Dateien als "ASBA-Mathematik.tex" verwendet werden, werden in "ASBA-Vorspann.tex" definiert.

% Metasprachliche Symbole ######################################################

\newcommand*{\srand}{\mid}% in formalen Sätzen und Schlussregeln:   \textdots\ und \textdots\
% Nur im Mathematikmodus!
\newcommand*{\metaandsym}{\&}%              \textdots\ und       \textdots\
\newcommand*{\metaand}{\;\metaandsym\;}%    \textdots\ und       \textdots\ (besserer Abstand)
\newcommand*{\metaorsym}{||}%               \textdots\ oder      \textdots\
\newcommand*{\metaor}{\;\metaorsym\;}%      \textdots\ oder      \textdots\ (besserer Abstand)
\newcommand*{\derivesym}{\vdash}%           \textdots\ ableitbar \textdots\
\newcommand*{\derive}{\;\derivesym\;}%      \textdots\ ableitbar \textdots\ (besserer Abstand)
\newcommand*{\metaimp}{\Rightarrow}%    aus \textdots\ folgt              \textdots\
\newcommand*{\metarep}{\Leftarrow}%         \textdots\ folgt aus          \textdots\
\newcommand*{\metaequiv}{\Leftrightarrow}%  \textdots\ genau dann wenn    \textdots\
\newcommand*{\metadefeq}{:\metaequiv}%      \textdots\ definitionsgemäß " \textdots\
\newcommand*{\eq}{=}%                       \textdots\ gleich             \textdots\
\newcommand*{\defeq}{\coloneqq}%            \textdots\ definitionsgemäß " \textdots\
\newcommand*{\swap}     {\leftrightarrows}% \textdots\ vertauscht mit     \textdots\
\newcommand*{\subst}    {\leftarrowtail}%   \textdots\ substituiert durch \textdots\

% Beispieloperatoren ===========================================================
% \*bsp
\newcommand*{\opbsp}{\circledast}
\newcommand*{\opubsp}{\circleddash}
\newcommand*{\relbsp}{\sim}
\newcommand*{\relnbsp}{\nsim}
\newcommand*{\releqbsp}{\simeq}
\newcommand*{\lrelbsp}{\lhd}
\newcommand*{\rrelbsp}{\rhd}
\newcommand*{\lreleqbsp}{\unlhd}
\newcommand*{\rreleqbsp}{\unrhd}

% Definitionen für die Tabelle der Junktoren ===================================
% \l*  -           logischer Operator
% \ln* - negierter logischer Operator
% Wahrheitswerte ---------------------------------------------------------------
% Konstante --------------------------------------------------------------------
\newcommand*{\ltrue} {\top}%            W       - wahr   (Wahrheitswert)
\newcommand*{\lfalse}{\bot}%            F       - falsch (Wahrheitswert)
% unäre Operatoren -------------------------------------------------------------
%            \lnot                      F W     - nicht A
% binäre Operatoren ------------------------------------------------------------
%            \lor                       W W W F - A oder B
\newcommand*{\lrep}  {\leftarrow}%      W W F W - A folgt aus B
\newcommand*{\limp}  {\rightarrow}%     W F W W - aus A folgt B
\newcommand*{\lequiv}{\leftrightarrow}% W F F W - A genau dann wenn B
%            \land                      W F F F - A und B
\newcommand*{\lnand} {\uparrow}%        F W W W - nicht   (A und  B)
\newcommand*{\lxor}  {+}%               F W W F - entweder A oder B
\newcommand*{\lnor}  {\downarrow}%      F F F W - weder    A noch B

% Verwendete Konstanten- und Mengenbezeichnungen ===============================

% \gs* = globales Symbol
\newcommand*{\gsN} {\mathbb{N}}%   Menge der natürlichen Zahlen ohne           0
\newcommand*{\gsNo}{\mathbb{N}_0}% Menge der natürlichen Zahlen einschließlich 0

% Elemente und Mengen für Beweise
\newcommand*{\Formelmenge}         {\mathcal{L}}             % [l]anguage
\newcommand*{\Voraussetzung}       {V}                       % [V]oraussetzung
\newcommand*{\Voraussetzungsmenge} {\mathcal{\Voraussetzung}}
\newcommand*{\Folgerung}           {F}                       % [F]olgerung
\newcommand*{\Folgerungsmenge}     {\mathcal{\Folgerung}}
\newcommand*{\Beweisschritt}       {B}                       % [B]eweisschritt
\newcommand*{\Beweisschrittfolge}  {\mathcal{S}}             % [s]equenz
\newcommand*{\Beweisschrittmenge}  {\mathcal{\Beweisschritt}}
\newcommand*{\Transformation}      {T}                       % [T]ransformation
\newcommand*{\Transformationsmenge}{\mathcal{\Transformation}}
\newcommand*{\Schlussregel}        {C}                       % [c]onclusion
\newcommand*{\Schlussregelmenge}   {\mathcal{\Schlussregel}}
\newcommand*{\Substitution}        {E}                       % [E]rsetzung
\newcommand*{\Substitutionsmenge}  {\mathcal{\Substitution}}

% verwendete Mengensymbole: AB C EF JKL O Q STUV
% \al\textdots\ = aussagenlogisch
\newcommand*{\alvar} {q}%               Name einer Variablen
% Mengen
\newcommand*{\alVar} {\mathcal{Q}}%     Menge der Variablen
\newcommand*{\alCon} {\mathcal{K}}%     Menge der [K]onstantensymbole
\newcommand*{\alUna} {\mathcal{U}}%     Menge der [u]nären Operatorsymbole
\newcommand*{\alBin} {\mathcal{O}}%     Menge der binären [O]peratorsymbole
\newcommand*{\alJun} {\mathcal{J}}%     Menge der [J]unktoren
\newcommand*{\alABC} {\mathcal{A}}%     [A]lphabet der aussagenlogischen Sprache
\newcommand*{\alFor} {\mathcal{L}}%     Menge der Formeln (Worte) ([l]anguage)
\newcommand*{\alForp}{\alFor^\mathrm{p}}%   \textdots\ in polnischer Notation
% Indizes für Teilmengen von \alJun, \alABC, \alFor und \alForp
\newcommand*{\iAnd} {\mathrm{and}}%
\newcommand*{\iBool}{\mathrm{bool}}%
\newcommand*{\iImp} {\mathrm{imp}}%
\newcommand*{\iNand}{\mathrm{nand}}%
\newcommand*{\iNor} {\mathrm{nor}}%
\newcommand*{\iOr}  {\mathrm{or}}%
\newcommand*{\iRep} {\mathrm{rep}}%
% Konstanten und Symbole
\newcommand*{\true}  {\mathrm{true}}
\newcommand*{\false} {\mathrm{false}}

% sonstige Kommandos für den Mathematiksatz ####################################

\mathtoolsset{showonlyrefs,showmanualtags}% Nur mit \ref referenzierte Gleichungen, aber alle manuellen Tags

% Glossareinträge ##############################################################
%TODO Symbole in  Glossar und Index eintragen

% Symbole für Mengen -----------------------------------------------------------

\newglossaryentry{gsN}{
	name ={$\gsN$},
	plural={\gsN},% im Mathematikmodus
	description={Die Menge der natürlichen Zahlen ohne 0.}
}
\newglossaryentry{gsNo}{
	name ={$\gsNo$},
	%TODO falscher Zeichensatz?
	plural={\gsNo},% im Mathematikmodus
	description={Die Menge der natürlichen Zahlen einschließlich 0.}
}
\newglossaryentry{alABC}{
	name ={$\alABC$},
	plural={\alABC},% im Mathematikmodus
	description={Das Alphabet der aussagenlogischen Sprache.}
}
\newglossaryentry{alABCx}{
	name ={$\alABC_x$},
	plural={\alABC_x},% im Mathematikmodus
	description={
		Eine Teilmenge des Alphabets $\alABC$ der aussagenlogischen Sprache.
	}
}
\newglossaryentry{alBin}{
	name ={$\alBin$},
	plural={\alBin},% im Mathematikmodus
	description={Die Menge der aussagenlogischen, binären Operatoren.}
}
\newglossaryentry{alCon}{
	name ={$\alCon$},
	plural={\alCon},% im Mathematikmodus
	description={Die Menge der aussagenlogischen Konstanten.}
}
\newglossaryentry{alFor}{
	name ={$\alFor$},
	plural={\alFor},% im Mathematikmodus
	description={Die Menge der aussagenlogischen \emph{Formeln} mit Klammerung.}
}
\newglossaryentry{alForp}{
	name ={$\alForp$},
	plural={\alForp},% im Mathematikmodus
	description={
		Die Menge der aussagenlogischen \emph{Formeln} in polnischer Notation.
	}
}
\newglossaryentry{alForx}{
	name ={$\alFor_x$},
	plural={\alFor_x},% im Mathematikmodus
	description={
		Eine Teilmenge der Menge $\alFor$ der aussagenlogischen \emph{Formeln} mit Klammerung.
	}
}
\newglossaryentry{alForxp}{
	name ={$\alForp_x$},
	plural={\alForp_x},% im Mathematikmodus
	description={
		Eine Teilmenge der Menge $\alFor$ der aussagenlogischen \emph{Formeln} in polnischer Notation.
	}
}
\newglossaryentry{alJun}{
	name ={$\alJun$},
	plural={\alJun},% im Mathematikmodus
	description={Die Menge der aussagenlogischen \emph{Junktoren}.}
}
\newglossaryentry{alJunx}{
	name ={$\alJun_x$},
	plural={\alJun_x},% im Mathematikmodus
	description={
		Eine Teilmenge der Menge $\alJun$ der aussagenlogischen Operatoren.
	}
}
\newglossaryentry{alUna}{
	name ={$\alUna$},
	plural={\alUna},% im Mathematikmodus
	description={Die Menge der aussagenlogischen unären Operatoren.}
}
\newglossaryentry{alVar}{
	name ={$\alVar$},
	plural={\alVar},% im Mathematikmodus
	description={Die Menge der aussagenlogischen Variablen.}
}
%=====================================
\newglossaryentry{Formelmenge}{
	name ={$\Formelmenge$},
	plural={\Formelmenge},% im Mathematikmodus
	description={
		\textiAlg\ eine Menge von Formeln \objqt{\Formel} \textbzw\ Worten.
		Man nennt $\Formelmenge$ auch \emph{Sprache}.
	}
}
%%%\newcommand*{\Voraussetzungsmenge} {\mathcal{\Voraussetzung}}
%%%\newcommand*{\Folgerungsmenge}     {\mathcal{\Folgerung}}
%%%\newcommand*{\Beweisschrittfolge}  {\mathcal{S}}             % [s]equenz
%%%\newcommand*{\Beweisschrittmenge}  {\mathcal{\Beweisschritt}}
%%%\newcommand*{\Transformationsmenge}{\mathcal{\Transformation}}
%%%\newcommand*{\Schlussregelmenge}   {\mathcal{\Schlussregel}}
%%%\newcommand*{\Substitutionsmenge}  {\mathcal{\Substitution}}

% Symbole für Beispieloperatoren -----------------------------------------------

\newglossaryentry{lrelbsp}{
	name ={$\lrelbsp$},
	plural={\lrelbsp},% im Mathematikmodus
	description={
		Ein Beispielsymbol für eine Relation mit Umkehrrelation $\rrelbsp$.
	}
}
\newglossaryentry{lreleqbsp}{
	name ={$\lreleqbsp$},
	plural={\lreleqbsp},% im Mathematikmodus
	description={
		Ein Beispielsymbol für eine Relation mit Gleichheit und Umkehrrelation $\rreleqbsp$.
	}
}
\newglossaryentry{relbsp}{
	name ={$\relbsp$},
	plural={\relbsp},% im Mathematikmodus
	description={Ein Beispielsymbol für eine Relation.}
}
\newglossaryentry{releqbsp}{
	name ={$\releqbsp$},
	plural={\releqbsp},% im Mathematikmodus
	description={Ein Beispielsymbol für eine Relation mit Gleichheit.}
}
\newglossaryentry{relnbsp}{
	name ={$\relnbsp$},
	plural={\relnbsp},% im Mathematikmodus
	description={Verneinung von $\relbsp$.}
}
\newglossaryentry{rrelbsp}{
	name ={$\rrelbsp$},
	plural={\rrelbsp},% im Mathematikmodus
	description={
		Ein Beispielsymbol für eine Relation mit Umkehrrelation $\lrelbsp$.
	}
}
\newglossaryentry{rreleqbsp}{
	name ={$\rreleqbsp$},
	plural={\rreleqbsp},% im Mathematikmodus
	description={
		Ein Beispielsymbol für eine Relation mit Gleichheit und Umkehrrelation $\lreleqbsp$.
	}
}

% Meta-Symbole -----------------------------------------------------------------

\newglossaryentry{defeq}{
	name ={$:=$},
	plural={:=},% im Mathematikmodus
	description={Definition: \textdots\ definitionsgemäß gleich \textdots}
}
\newglossaryentry{derive}{
	name ={$\derivesym$},
	plural={\derive},% im Mathematikmodus
	description={
		\emph{Ableitungsrelation}: \textdots\ ableitbar \textdots
		-- siehe \emph{ableitbar}%
	}
}
\newglossaryentry{eq}{
	name ={$\eq$},
	plural={\eq},% im Mathematikmodus
	description={
		Eine \emph{(Meta-)Relation}: \textdots\ gleich (ist dasselbe wie, ist identisch zu) \textdots
	}
}
\newglossaryentry{equiv}{
	name ={$\equiv$},
	plural={\equiv},% im Mathematikmodus
	description={
		Eine \emph{(Meta-)Relation}: \textdots\ äquivalent zu (ist das gleiche wie, ist so wie) \textdots
	}
}
\newglossaryentry{metaand}{
	name ={$\metaandsym$},
	plural={\metaand},% im Mathematikmodus
	description={Ein \emph{Metaoperator}: \textdots\ und \textdots}
}
\newglossaryentry{metadefeq}{
	name ={$\metadefeq$},
	plural={\metadefeq},% im Mathematikmodus
	description={
		\emph{Metadefinition}: \textdots\ definitionsgemäß gleich (definitionsgemäß genau dann, wenn) \textdots
	}
}
\newglossaryentry{metaequiv}{
	name ={$\metaequiv$},
	plural={\metaequiv},% im Mathematikmodus
	description={Eine \emph{Metarelation}: \textdots\ genau dann wenn \textdots}
}
\newglossaryentry{metaimp}{
	name ={$\metaimp$},
	plural={\metaimp},% im Mathematikmodus
	description={Eine \emph{Metarelation}: \textdots\ dann auch \textdots}
}
\newglossaryentry{metaor}{
	name ={$\metaorsym$},
	plural={\metaor},% im Mathematikmodus
	description={Ein \emph{Metaoperator}: \textdots\ oder \textdots}
}
\newglossaryentry{metarep}{
	name ={$\metarep$},
	plural={\metarep},% im Mathematikmodus
	description={Eine \emph{Metarelation}: \textdots\ sofern \textdots}
}
\newglossaryentry{ne}{
	name ={$\ne$},
	plural={\ne},% im Mathematikmodus
	description={
		Eine (Meta-)Operator: \textdots\ ungleich (nicht dasselbe wie, nicht identisch zu) \textdots
	}
}
\newglossaryentry{nequiv}{
	name ={$\nequiv$},
	plural={\nequiv},% im Mathematikmodus
	description={
		Eine \emph{(Meta-)Relation}: \textdots\ nicht äquivalent (ist nicht das gleiche wie, ist nicht so wie) \textdots
	}
}
\newglossaryentry{subst}{
	name ={$\subst$},
	plural={\subst},% im Mathematikmodus
	description={
		\emph{Substitution}: \textdots\ substituiert durch \textdots\
		-- siehe die Definition \vrefinsub{sub:Identitätsregeln}
	}
}
\newglossaryentry{swap}{
	name ={$\swap$},
	plural={\swap},% im Mathematikmodus
	description={
		\emph{Vertauschung}: \textdots\ vertauscht mit \textdots\
		-- siehe die Definition \vrefinsub{sub:Identitätsregeln}
	}
}
\newglossaryentry{srand}{
	name ={$\srand$},
	plural={\srand},% im Mathematikmodus
	description={
		Ein \emph{Metaoperator}: \textdots\ und \textdots\
		-- wird nur bei den \emph{Schlussregeln} verwendet
	}
}

% sonstige mathematische Symbole -----------------------------------------------

\newglossaryentry{lfalse}{
	name ={$\lfalse$},
	plural={\lfalse},% im Mathematikmodus
	description={Eine aussagenlogische Konstante (\emph{Wahrheitswert}): Falsch}
}
\newglossaryentry{ltrue}{
	name ={$\ltrue$},
	plural={\ltrue},% im Mathematikmodus
	description={Eine aussagenlogische Konstante (\emph{Wahrheitswert}): Wahr}
}

% Schlussregeln ----------------------------------------------------------------

\newcommand*{\tagAR}{AR}% Argument für \tag - im Textmodus
\newcommand*{\AR}{(\text{AR})}% im Mathematikmodus
\newglossaryentry{AR}{
	name ={$\AR$},
	plural={\AR},% im Mathematikmodus
	description={\emph{Anfangsregel}}
}
\newcommand*{\tagFS}{FS}% Argument für \tag - im Textmodus
\newcommand*{\FS}{(\text{FS})}% im Mathematikmodus
\newglossaryentry{FS}{
	name ={$\FS$},
	plural={\FS},% im Mathematikmodus
	description={\emph{formaler Satz}}
}
\newcommand*{\tagMR}{MR}% Argument für \tag - im Textmodus
\newcommand*{\MR}{(\text{MR})}% im Mathematikmodus
\newglossaryentry{MR}{
	name ={$\MR$},
	plural={\MR},% im Mathematikmodus
	description={\emph{Monotonieregel}}
}
\newcommand*{\tagSR}{SR}% Argument für \tag - im Textmodus
\newcommand*{\SR}{(\text{SR})}% im Mathematikmodus
\newglossaryentry{SR}{
	name ={$\SR$},
	plural={\SR},% im Mathematikmodus
	description={\emph{Schnittregel} (Modus ponens)}
}
\newcommand*{\tagTR}{TR}% Argument für \tag - im Textmodus
\newcommand*{\TR}{(\text{TR})}% im Mathematikmodus
\newglossaryentry{TR}{
	name ={$\TR$},
	plural={\TR},% im Mathematikmodus
	description={\emph{Abtrennungsregel}}
}
\newcommand*{\tageqB}{$\eq$B}% Argument für \tag - im Textmodus
\newcommand*{\eqB}{(\eq\text{B})}% im Mathematikmodus
\newglossaryentry{eqB}{
	name ={$\eqB$},
	plural={\eqB},% im Mathematikmodus
	description={Beseitigung von \symqt{\eq}}
}
\newcommand*{\tageqE}{$\eq$E}% Argument für \tag - im Textmodus
\newcommand*{\eqE}{(\eq\text{E})}% im Mathematikmodus
\newglossaryentry{eqE}{
	name ={$\eqE$},
	plural={\eqE},% im Mathematikmodus
	description={Einführung von \symqt{\eq}}
}
\newcommand*{\tagandB}{$\land$B}% Argument für \tag - im Textmodus
\newcommand*{\andB}{(\land\text{B})}% im Mathematikmodus
\newglossaryentry{andB}{
	name ={$\andB$},
	plural={\andB},% im Mathematikmodus
	description={Beseitigung von \symqt{\andB}}
}
\newcommand*{\tagandE}{$\land$E}% Argument für \tag - im Textmodus
\newcommand*{\andE}{(\land\text{E})}% im Mathematikmodus
\newglossaryentry{andE}{
	name ={$\andE$},
	plural={\andE},% im Mathematikmodus
	description={Beseitigung von \symqt{\andE}}
}
\newcommand*{\tagimpB}{$\limp$B}% Argument für \tag - im Textmodus
\newcommand*{\impB}{(\limp\text{B})}% im Mathematikmodus
\newglossaryentry{impB}{
	name ={$\impB$},
	plural={\impB},% im Mathematikmodus
	description={Beseitigung von \symqt{\impB}}
}
\newcommand*{\tagimpE}{$\limp$E}% Argument für \tag - im Textmodus
\newcommand*{\impE}{(\limp\text{E})}% im Mathematikmodus
\newglossaryentry{impE}{
	name ={$\impE$},
	plural={\impE},% im Mathematikmodus
	description={Beseitigung von \symqt{\impE}}
}
\newcommand*{\tagnota}{$\lnot$1}% Argument für \tag - im Textmodus
\newcommand*{\nota}{(\lnot\text{1})}% im Mathematikmodus
\newglossaryentry{nota}{
	name={ $\nota$},
	plural={\nota},% im Mathematikmodus
	description={Einführung/Beseitigung von \symqt{\lnot} Teil 1}
}
\newcommand*{\tagnotb}{$\lnot$2}% Argument für \tag - im Textmodus
\newcommand*{\notb}{(\lnot\text{2})}% im Mathematikmodus
\newglossaryentry{notb}{
	name ={$\notb$},
	plural={\notb},% im Mathematikmodus
	description={Einführung/Beseitigung von \symqt{\lnot} Teil 2}
}
\newcommand*{\tagnotc}{$\lnot$3}% Argument für \tag - im Textmodus
\newcommand*{\notc}{(\lnot\text{3})}% im Mathematikmodus
\newglossaryentry{notc}{
	name ={$\notc$},
	plural={\notc},% im Mathematikmodus
	description={Beweistechnik \enquote{Indirekter \emph{Beweis}}}
}
\newcommand*{\tagnotd}{$\lnot$4}% Argument für \tag - im Textmodus
\newcommand*{\notd}{(\lnot\text{4})}% im Mathematikmodus
\newglossaryentry{notd}{% statt "notE"
	name ={$\notd$},
	plural={\notd},% im Mathematikmodus
	description={Reductio ad absurdum (Indirekter \emph{Beweis})}
}
%%%\newcommand*{\tagorB}{$\lor$B}% Argument für \tag - im Textmodus
%%%\newcommand*{\orB}{(\lor\text{B})}% im Mathematikmodus
%%%\newglossaryentry{orB}{
%%%	name ={$\obB$},
%%%	plural={\orB},% im Mathematikmodus
%%%	description={Beseitigung von \symqt{\lor}}
%%%}
%%%\newcommand*{\tagorE}{$\lor$E}% Argument für \tag - im Textmodus
%%%\newcommand*{\orE}{(\lor\text{E})}% im Mathematikmodus
%%%\newglossaryentry{orE}{
%%%	name ={$\obE$},
%%%	plural={\orE},% im Mathematikmodus
%%%	description={Beseitigung von \symqt{\lor}}
%%%}

% Fachbegriffe -----------------------------------------------------------------
%TODO Alle Fachbegriffe in Glossar und Index eintragen

\newglossaryentry{ableitbar}{
	name  ={ableitbar},
	plural={ableitbare},
	description={
		Wenn sich eine \emph{Formel} $\beta$ aus einer anderen \emph{Formel} $\alpha$ mittels zulässiger Transaktionen ableiten lässt, heißt $\beta$ \emph{ableitbar} aus $\alpha$.
		Sprechweise: \seqqt{$ \alpha \text{ableitbar} \beta $}.
		Eine oder beide \emph{Formeln} \objqt{\alpha} \textbzw\ \objqt{\beta} dürfen dabei durch \emph{Formelmengen} ersetzt werden.
		-- siehe \emph{Ableitungsrelation} und \symqt{\derive}.
		\newline
		Synonym: \emph{beweisbar}%
	}
}
\newglossaryentry{Ableitungsrelation}{
	name  ={Ableitungsrelation},
	plural={Ableitungsrelationen},
	description={Die Relation \symqt{\derive}}
}
\newglossaryentry{Abtrennungsregel}{
	name  ={Abtrennungsregel},
	plural={Abtrennungsregeln},
	description={Eine \emph{Schlussregel} -- siehe~\emph{TR}}
}
\newglossaryentry{allgemeingueltige-Schlussregel}{
	name  ={allgemeingültige Schlussregel},
	plural={allgemeingültige Schlussregeln},
	description={
		Eine \emph{Schlussregel} die aus den \emph{Basisregeln} und den schon bekannten \emph{allgemeingültigen Schlussregeln} abgeleitet werden kann.
	}
}
\newglossaryentry{Anfangsregel}{
	name  ={Anfangsregel},
	plural={Anfangsregeln},
	description={
		Eine \emph{Schlussregel} um beginnen zu können -- siehe~\emph{AR}}
}
\newglossaryentry{atomare-Formel}{
	name  ={atomare Formel},
	plural={atomare Formeln},
	description={Eine \emph{Formel}, die sich nicht weiter zerlegen lässt.}
}
\newglossaryentry{Aussage}{
	name  ={Aussage},
	plural={Aussagen},
	description={
		Eine \emph{Aussage} in natürlicher Sprache oder als \emph{Formel}, die einen \emph{Wahrheitswert} liefert.
	}
}
\newglossaryentry{Aussagenlogik}{
	name  ={Aussagenlogik},
	description={\vrefseesec{sec:Aussagenlogik}}
}
\newglossaryentry{Axiomensystem}{
	name  ={Axiomensystem},
	plural={Axiomensysteme},
	description={Eine Menge von \emph{Axiomen}}
}
\newglossaryentry{Basisregel}{
	name  ={Basisregel},
	plural={Basisregeln},
	description={
		Eine \emph{Schlussregel}, die nicht mehr auf andere zurückgeführt wird.
		Obwohl das auch auf die \emph{Identitätsregeln} zutrifft, werden diese hier aber nicht dazu gezählt.
	}
}
\newglossaryentry{beweisbar}{
	name  ={beweisbar},
	plural={beweisbare},
	description={Synonym zu \emph{ableitbar}.}
}
\newglossaryentry{Beweisschritt}{
	name  ={Beweisschritt},
	plural={Beweisschritte},
	description={
		Eine Vorschrift, wie aus vorgegebenen \emph{Aussagen} (den \emph{Voraussetzungen}) eine weitere (die \emph{Folgerung}) folgt.
	}
}
\newglossaryentry{Beweisschrittfolge}{
	name  ={Beweisschrittfolge},
	plural={Beweisschrittfolgen},
	description={Eine Folge von \emph{Beweisschritten}.}
}
\newglossaryentry{Beweisschrittmenge}{
	name  ={Beweisschrittmenge},
	plural={Beweisschrittmengen},
	description={
		Die Menge der \emph{Beweisschritte}, \textdh\ der Glieder der \emph{Beweisschrittfolge} eines \emph{Beweises}.
	}
}
\newglossaryentry{Boolsche-Signatur}{
	name  ={Boolsche Signatur},
	plural={Boolsche Signaturen},
	description={Die \emph{logische Signatur} $\{\lnot, \land, \lor\}$.}
}
\newglossaryentry{Folgerung}{
	name  ={Folgerung},
	plural={Folgerungen},
	description={
		Die \emph{Folgerungen} einer \emph{Schlussregel} sind die \emph{Aussagen} über ihrem Querstrich.
	}
}
\newglossaryentry{formaler-Satz}{
	name  ={formaler Satz},
	plural={formale  Sätze},
	description={
		Formale Darstellung eines mathematischen Satzes -- siehe~\emph{FS}
	}
}
\newglossaryentry{Formel}{
	name  ={Formel},
	plural={Formeln},
	description={
		Unter einer \emph{Formel} verstehen wir in diesem Dokument stets eine mathematische \emph{Formel}.
		Diese kann auch mehrdimensional sein, lässt sich aber mittels geeigneter Definitionen immer eindeutig als eine \emph{Zeichenfolge} schreiben.
		\emph{Schlussregeln} betrachten wir \emph{nicht} als \emph{Formeln}.
	}%
}
\newglossaryentry{Gleichheitsrelation}{
	name  ={Gleichheitsrelation},
	plural={Gleichheitsrelationen},
	description={
		Eine mit der Gleichheit verwandte Relation: \objqt{=}, \objqt{\ne}, \objqt{\equiv} und \objqt{\nequiv}
	}%
}
%TODO Identitätsregel nötig?
\newglossaryentry{Identitaetsregel}{
	name  ={Identitätsregel},
	plural={Identitätsregeln},
	description={
		Eigentlich eine \emph{Basisregel} zur Identität.
		Da die \emph{Identitätsregeln} nur zur Rechtfertigung der \emph{Substitution} verwendet werden, werden sie hier nicht zu den \emph{Basisregeln} gezählt.
	}
}
\newglossaryentry{interessierende-Eigenschaft}{
	name  ={interessierende Eigenschaft},
	plural={interessierende Eigenschaften},
	description={
		Solche Eigenschaften von \emph{Objekten}, die im aktuellen Zusammenhang von Interesse sind.
	}
}
\newglossaryentry{Junktor}{
	name  ={Junktor},
	plural={Junktoren},
	description={Ein Operatorsymbol, \textdh\ ein Symbol für einen Operator.}
}
\newglossaryentry{Kontraposition}{
	name  ={Kontraposition},
	plural={Kontraposition},
	description={
		Die allgemeingültige \emph{Aussage}: $ (\alpha \limp \beta) \limp (\lnot\beta \limp \lnot\alpha) $
	}
}
\newglossaryentry{logische-Signatur}{
	name  ={logische Signatur},
	plural={logische Signaturen},
	description={
		Eine Teilmenge von $\alJun$, die ausreicht, alle anderen Elemente aus $\alJun$ zu definieren.
	}
}
\newglossaryentry{Mengenlehre}{
	name={Mengenlehre},
	description={\vrefseesec{sec:Mengenlehre}.}
}
\newglossaryentry{Metaoperator}{
	name  ={Metaoperator},
	plural={Metaoperatoren},
	description={
		Ein Operator der \emph{Metasprache}: \objqt{\metaandsym}, \objqt{\metaorsym} und \objqt{\srand}.
	}
}
\newglossaryentry{Metarelation}{
	name  ={Metarelation},
	plural={Metarelationen},
	description={
		Eine Relation der \emph{Metasprache}: \objqt{\metaimp}, \objqt{\metarep} und \objqt{\metaequiv}
	}
}
\newglossaryentry{Metasprache}{
	name  ={Metasprache},
	plural={Metasprachen},
	description={
		Eine Sprache, in der \emph{Aussagen} über Elemente einer anderen Sprache getroffen werden können.
		In diesem Dokument ist dies immer die normale Sprache.
		\vrefseesec{sec:Metasprache}
	}
}
\newglossaryentry{Monotonieregel}{
	name  ={Monotonieregel},
	plural={Monotonieregeln},
	description={
		Eine \emph{Schlussregel} -- siehe~\emph{MR}
	}
}
\newglossaryentry{Objekt}{
	name  ={Objekt},
	plural={Objekte},
	description={
		Symbole, \emph{Formeln} und \emph{Aussagen} sowie Mengen, \emph{Zeichenfolgen}, Zahlen, ganz allgemein reale oder gedachte Dinge an sich.
	}
}
\newglossaryentry{Praedikat}{
	name  ={Prädikat},
	plural={Prädikate},
	description={
		Ein Element der \emph{Prädikatenlogik} (\vrefseesec{sec:Prädikatenlogik}).
		\textZB\ kann man eine Gruppe als ein zweistelliges Prädikat $\mathrm{Gruppe}(G,+)$ definieren, in dem $G$ eine Menge und $+$ eine Operation, \textdh\ eine (zweistellige) Funktion $ +: G \times G \rightarrow G $ ist, so dass die Gruppenaxiome erfüllt sind.
	}
}
\newglossaryentry{Praedikatenlogik}{
	name={Prädikatenlogik},
	description={\vrefseesec{sec:Prädikatenlogik}}
}
\newglossaryentry{Schlussregel}{
	name  ={Schlussregel},
	plural={Schlussregeln},
	description={
		Eine \emph{Schlussregel} $\frac{\Voraussetzungsmenge}{\Folgerungsmenge}$ entspricht der \emph{Aussage}:
		Wenn alle \emph{Voraussetzungen} \objqt{\Voraussetzung} aus \objqt{\Voraussetzungsmenge} zutreffen, dann auch alle \emph{Folgerungen} \objqt{\Folgerung} aus \objqt{\Folgerungsmenge}.
		Wenn diese \emph{Aussage} zutrifft, kann die Schlussregel zur zulässigen Umwandlung von \emph{Formeln} dienen.
	}
}
\newglossaryentry{Schlussregelmenge}{
	name  ={Schlussregelmenge},
	plural={Schlussregelmengen},
	description={
		Eine Menge von \emph{Schlussregeln}, meistens mit \objqt{\Schlussregelmenge} bezeichnet.
	}
}
\newglossaryentry{Schnittregel}{
	name  ={Schnittregel},
	plural={Schnittregeln},
	description={Eine \emph{allgemeingültige Schlussregel} -- siehe~\emph{SR}}
}
\newglossaryentry{Sprache}{
	name  ={Sprache},
	plural={Sprachen},
	description={siehe \emph{Formelmenge}}
}
\newglossaryentry{Substitution}{ %TODO ggf. überarbeiten
	name  ={Substitution},
	plural={Substitutionen},
	description={
		Die Ersetzung von einem, mehreren oder allen \emph{formalen Elementen} ($\alpha$) in einem anderen \emph{formalen Element} ($\gamma$) durch ein drittes \emph{formales Element} ($\beta$)
		-- formal: $\gamma(\alpha\subst\beta)$.
		Wenn alle $\alpha$ in $\gamma$ durch $\beta$ ersetzt werden, ist die \emph{Substitution vollständig}.
		-- \vrefseesub{sub:Identitätsregeln}
	}
}
\newglossaryentry{Transformation}{
	name  ={Transformation},
	plural={Transformationen},
	description={
		Eine Umformung oder Erzeugung einer \emph{Formel} aus einer vorgegebenen Menge von \emph{Formeln},
		\textdh\ die Anwendung einer \emph{Schlussregel}.
	}
}
\newglossaryentry{Transformationsmenge}{
	name  ={Transformationsmenge},
	plural={Transformationsmenge},
	description={
		Eine Menge von \emph{Transformationen}.
	}
}
\newglossaryentry{vergleichbar}{
	name  ={vergleichbar},
	plural={vergleichbare},
	description={
		Zwei \emph{Objekte} \objqt{A} und \objqt{B} sind vergleichbar, wenn beide von derselben Art sind, \textdh\ wenn \textzB\ jeweils beide Mengen, \emph{Zeichenfolgen}, Zahlen, \textusw\ sind.
		Dabei muss bei \emph{Formeln} zwischen der Formel an sich und dem Ergebnis der Formel unterschieden werden.
		-- \vrefseesec{subsub:vergleichbar}%
	}
}
\newglossaryentry{Vertauschung}{ %TODO ggf. überarbeiten
	name  ={Vertauschung},
	plural={Vertauschungen},
	description={
		Die \emph{Vertauschung} von zwei unabhängigen \emph{formalen Elementen} ($\alpha$ und $\beta$) in einem anderen formalen Element ($\gamma$)
		-- formal: $\gamma(\alpha\swap\beta)$.
		Die Vertauschung ist eine spezielle Form der \emph{Substitution}.
		-- siehe die Definition~\eqref{def:Vertauschung} \vrefinsub{sub:Identitätsregeln}
	}
}
\newglossaryentry{Voraussetzung}{
	name  ={Voraussetzung},
	plural={Voraussetzungen},
	description={
		Die Voraussetzungen einer \emph{Schlussregel} $\frac{\Voraussetzungsmenge}{Nenner\Folgerungsmenge}$ sind die \emph{Aussagen} aus $\Voraussetzungsmenge$.
	}
}
\newglossaryentry{Wahrheitswert}{
	name  ={Wahrheitswert},
	plural={Wahrheitswerte},
	description={
		Wahrheitswerte sind die Werte \symqt{\ltrue} und \symqt{\lfalse}, oft auch mit \symqt{\mathrm{wahr}} und \symqt{\mathrm{falsch}}, \symqt{\mathrm{true}} und \symqt{\mathrm{false}} oder einfach \symqt{1} und \symqt{0} bezeichnet.
	}
}
\newglossaryentry{Zeichenfolge}{
	name  ={Zeichenfolge},
	plural={Zeichenfolgen},
	description={
		Folgen von unzerlegbaren Zeichen und Symbolen, wobei Leerstellen und sonstiger Zwischenraum nicht zählen und nur zur besseren Darstellung dienen.
		Dabei sind als spezielle Symbole auch \emph{Zeichenketten} erlaubt, solange die Zerlegung eindeutig bleibt.
		\textZB\ kann \symqt{sin} als ein einzelnes Symbol -- für die Sinusfunktion -- aufgefasst werden, aber auch als Folge der Buchstaben \chrqt{s}, \chrqt{i} und \chrqt{b}.
		\emph{Formeln} werden immer als Zeichenfolgen aufgefasst.
	}
}
\newglossaryentry{Zeichenkette}{
	name  ={Zeichenkette},
	plural={Zeichenketten},
	description={
		Folgen von unzerlegbaren Zeichen, auch Leerstellen und sonstigem Zwischenraum.
		-- siehe auch \emph{Zeichenfolge}
	}
}
\newglossaryentry{zulaessige-Transformation}{%TODO ggf. überarbeiten
	name  ={zulässige Transformation},
	plural={zulässige Transformationen},
	description={
		Eine \emph{Transformation} aus einer vorgegebenen Menge von Transformationen oder eine daraus zulässiger weise abgeleitete Transformation.
	}
}
