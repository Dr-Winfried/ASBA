%%############################################################################%%
%%                                                                            %%
%% Datei:  ASBA-Vorspann-Mathematik.tex                                       %%
%% Inhalt: Vorspann Mathematik für ASBA                                       %%
%%                                                                            %%
%% Copyright (C) 2017  Winfried Teschers                                      %%
%%                                                                            %%
%% This program is free software: you can redistribute it and/or modify       %%
%% it under the terms of the GNU Affero General Public License as published   %%
%% by the Free Software Foundation, either version 3 of the License, or       %%
%% (at your option) any later version.                                        %%
%%                                                                            %%
%% This program is distributed in the hope that it will be useful,            %%
%% but WITHOUT ANY WARRANTY; without even the implied warranty of             %%
%% MERCHANTABILITY or FITNESS FOR A PARTICULAR PURPOSE.  See the              %%
%% GNU Affero General Public License for more details.                        %%
%%                                                                            %%
%% You should have received a copy of the GNU Affero General Public License   %%
%% along with this program.  If not, see <http://www.gnu.org/licenses/>.      %%
%%                                                                            %%
%% Dr. Winfried Teschers                                                      %%
%% Anton-Günther-Straße 26c                                                   %%
%% 91083 Baiersdorf                                                           %%
%% Germany                                                                    %%
%%                                                                            %%
%% e-mail: winfried.teschers@t-online.de                                      %%
%%                                                                            %%
%%############################################################################%%

% !TeX root = ASBA.tex
% !TeX encoding = UTF-8
% !TeX spellcheck = de_DE

% Glossareinträge werden in "ASBA-Vorspann-Glossar" definiert.
% Elemente, die in anderen Dateien als "ASBA-Mathematik.tex" verwendet werden, werden in "ASBA-Vorspann.tex" definiert.
% Namensbestandteile mit besonderer Bedeutung:
%   Bezeichnungen:
%     <Name>Ltr     = Bezeichnung (Buchstabe)      für (die Menge)     <Name>
%     <Name>Idx     = Bezeichnung (Buchstabe/Text) für einen Index zu  <Name>
%     <Name>Txt     = Bezeichnung           (Text) für (die Operation) <Name>
%     <Name>Symbol  = Bezeichnung    (Symbol/Text) für (die Operation) <Name>
%     <Name>Set     = Menge    von <Name>n (keine bis alle)
%     <Name>Rel     = Relation von <Name>n
%     <Name>Tup     = Tupel    von <Name>n
%     All<Name>     = Menge  aller <Name>n
%   Ergebnisse von Operationen auf Mengen:
%     Pot<Menge>    = Menge    der Teilmengen                   von <Name>
%     Rel<Menge>    = Menge    der binären Relationen           auf <Name>
%     Relf<Menge>   = Menge    der endlichen binären Relationen auf <Name>
%     Tup<Menge>    = Menge    der Tupel                        auf <Name>
%   Sonstiges:
%     <Relation>Bck = Umkehrrelation
%     <Operator>Eq  = Operator oder Gleich
%     <Operator>N   = negierter Operator
%     <Oper.>BckEqN = Kombination in dieser Reihenfolge
%
% Folgende Makros sind alles "eigene"
%   \Raw...    - Symbol/Text ohne Verweis;erfordert bei Symbolen Mathematikmodus
%   \...Bsp... - Beispielsymbol für Formel- oder Metasprache (f.Mathematikmodus)
%   \Bsp...    -                    mit Verweis ins Symbolverzeichnis
%   \...Mts... - Symbol der Metasprache                    (für Mathematikmodus)
%   \Mts...    -                    mit Verweis ins Symbolverzeichnis
%   \...Frm... - Symbol der Formelsprache                  (für Mathematikmodus)
%   \Frm...    -                    mit Verweis ins Symbolverzeichnis
%   \...Txt... - individuelle Bezeichnung                  (für       Textmodus)
%   \Txt...    -                    mit Verweis in  Glossar und Index
%
% Weitere "eigene" Kombinationen: OpU, OpB, Bck

% Metasprachliche Symbole ######################################################
% \RawMts*
\newcommand*{\RawMtsNot}     {\mathbin{\thicksim}}%               ... gilt nicht
\newcommand*{\RawMtsAnd}     {\mathbin{\&}}%                        ... und  ...
\newcommand*{\RawMtsUnd}     {\mathbin{\mid}}%nur in Schlussregeln: ... und  ...
\newcommand*{\RawMtsOr}      {\mathbin{\parallel}}%                 ... oder ...
\newcommand*{\RawMtsDerive}  {\mathrel{\vdash}}%          ... ableitbar      ...
\newcommand*{\RawMtsImp}     {\mathrel{\Rightarrow}}% von ... folgt          ...
\newcommand*{\RawMtsRep}     {\mathrel{\Leftarrow}}%      ... folgt von      ...
\newcommand*{\RawMtsEquiv}   {\mathrel{\Leftrightarrow}}% .. genau dann, wenn ..
\newcommand*{\RawMtsDefEquiv}{\mathrel{:\mkern-2mu\RawMtsEquiv}}% def.gemäß -"-
\newcommand*{\RawMtsEq}      {\mathrel{=\mkern-10mu=}}%          ...  gleich ...
\newcommand*{\RawMtsEqN}     {\mathrel{=\mkern-16mu/\mkern-16mu=}}% . ungleich .
\newcommand*{\RawMtsDefEq}   {\mathrel{:\mkern-2mu\RawMtsEq}}% def.gemäß gleich
\newcommand*{\RawMtsAequiv}  {\mathrel{\equiv}}%     ...       äquivalent zu ...
\newcommand*{\RawMtsNAequiv} {\mathrel{\nequiv}}%    ... nicht äquivalent zu ...
\newcommand*{\RawMtsSwap}    {\mathbin{\leftrightarrows}}% .. vertauscht mit ...
\newcommand*{\RawMtsSubst}   {\mathbin{\leftarrowtail}}%.. substituiert durch ..

% Beispieloperationen ==========================================================
% \RawBsp*; OpU=unär, OpB=binär, Rel=Relation, Bck=Umkehr-; N=nicht, Eq=gleich
\newcommand*{\RawBspOpU}      {\mathbin{\circleddash}}
\newcommand*{\RawBspOpB}      {\mathbin{\circledast}}
\newcommand*{\RawBspRel}      {\mathrel{\prec}}
\newcommand*{\RawBspRelEq}    {\mathrel{\preceq}}
\newcommand*{\RawBspRelBck}   {\mathrel{\succ}}
\newcommand*{\RawBspRelBckEq} {\mathrel{\succeq}}
\newcommand*{\RawBspRelN}     {\mathrel{\nprec}}
\newcommand*{\RawBspRelEqN}   {\mathrel{\npreceq}}
\newcommand*{\RawBspRelBckN}  {\mathrel{\nsucc}}
\newcommand*{\RawBspRelBckEqN}{\mathrel{\nsucceq}}

% Definitionen für die Tabelle der Junktoren ===================================
% Konstante (Wahrheitswete) ----------------------------------------------------
\newcommand*{\RawMtsFalse}    {\mathord{\mathrm{\MtsFalseTxt}}}%F - falsch (Sym)
\newcommand*{\RawMtsTrue}     {\mathord{\mathrm{\MtsTrueTxt}}}% W - wahr   (Sym)
\newcommand*{\RawFrmFalse}    {\mathord{\bot}}%                         (Symbol)
\newcommand*{\RawFrmTrue}     {\mathord{\top}}%                         (Symbol)
\newcommand*{\RawTxtFalse}    {\emph{\TxtFalseTxt}}%                      (Text)
\newcommand*{\RawTxtTrue}     {\emph{\TxtTrueTxt}}%                       (Text)
% unäre Operationen ------------------------------------------------------------
% Wahrheitswert von A                            W F
\newcommand*{\RawFrmNot}      {\lnot}%           F W     - nicht A
% binäre Operationen -----------------------------------------------------------
% Wahrheitswert von A                            W W F F
% Wahrheitswert von B                            W F W F
\newcommand*{\RawFrmOr}       {\lor}%            W W W F - A oder B
\newcommand*{\RawFrmRep}      {\leftarrow}%      W W F W - A folgt von B
\newcommand*{\RawFrmImp}      {\rightarrow}%     W F W W - von A folgt B
\newcommand*{\RawFrmEquiv}    {\leftrightarrow}% W F F W - A genau dann wenn B
\newcommand*{\RawFrmAnd}      {\land}%           W F F F - A und B
\newcommand*{\RawFrmNand}     {\uparrow}%        F W W W - nicht   (A und  B)
\newcommand*{\RawFrmXor}      {+}%               F W W F - entweder A oder B
\newcommand*{\RawFrmNor}      {\downarrow}%      F F F W - weder    A noch B
% außerhalb der Tabelle --------------------------------------------------------
\newcommand*{\RawFrmEq}       {=}%          Gleichheit   in Formeln
\newcommand*{\RawFrmEqN}      {\ne}%        Ungleichheit in Formeln
\newcommand*{\RawMtsIn}       {\in}%        ist  Element aus (der Menge)
\newcommand*{\RawMtsSubset}   {\subset}%    ist        echte  Teilmenge von
\newcommand*{\RawMtsSubsetN}  {\nsubset}%   ist  keine echte  Teilmenge von
\newcommand*{\RawMtsSubsetEq} {\subseteq}%  ist (gleich oder) Teilmenge von
\newcommand*{\RawMtsNi}       {\ni}%        (die Menge) enthält das Element
\newcommand*{\RawMtsSupset}   {\supset}%    ist        echte  Obermenge von
\newcommand*{\RawMtsSupsetN}  {\nsupset}%   ist  keine echte  Obermenge von
\newcommand*{\RawMtsSupsetEq} {\supseteq}%  ist (gleich oder) Obermenge von
\newcommand*{\RawMtsTimes}    {\times}%     karthesisches Produkt

% Neue Operatoren
\DeclareMathOperator{\RawMtsGraph}  {\MtsGraphTxt}% Graph von        Funktion/Relation
\DeclareMathOperator{\RawMtsTraeger}{\MtsTraegerTxt}% Trägermenge einer       Relation
\DeclareMathOperator{\RawMtsStel}   {\MtsStelTxt}% Stelligkeit einer Funktion/Relation
\DeclareMathOperator{\RawMtsStelF}  {\MtsStelTxt_f}% Stelligkeit für [F]unktionen
\DeclareMathOperator{\RawMtsStelR}  {\MtsStelTxt_r}% Stelligkeit für [R]elationen
\DeclareMathOperator{\RawMtsQb}     {\MtsQbTxt}% Quellbereich einer partiellen Funktion
\DeclareMathOperator{\RawMtsDb}     {\MtsDbTxt}% Definitionsbereich einer      Funktion
\DeclareMathOperator{\RawMtsZb}     {\MtsZbTxt}% Zielbereich        einer      Funktion
\DeclareMathOperator{\RawMtsWb}     {\MtsWbTxt}% Wertebereich       einer      Funktion
\DeclareMathOperator{\RawMtsLen}    {\MtsLenTxt}% Länge            eines/r Tupels/Folge
\DeclareMathOperator{\RawMtsSet}    {\MtsSetTxt}% Komponentenmenge eines/r Tupels/Folge

% Schriftarten
\newcommand*{\varFt}[1]  {\mathbf{#1}}% Variable aus Alphabet   (Kleinbuchstabe)
\newcommand*{\DrvFt}[1]  {\mathbf{#1}}% Mengen von Ableitungen   (Großbuchstabe)
\newcommand*{\drvFt}[1]  {\mathbf{#1}}% ein Element davon (Klein-/Großbuchstabe)
\newcommand*{\IdxFt}[1]  {\mathrm{#1}}% fester Index      (Klein-/Großbuchstabe)
\newcommand*{\SetFt}[1] {\mathcal{#1}}% vorgegebene Menge        (Großbuchstabe)
\newcommand*{\ElmFt}[1]          {#1}%  ein Element davon        (Großbuchstabe)
\newcommand*{\SetOp}[1]{\mathfrak{#1}}% Mengenoperation                   (Text)

% Indizes für Teilmengen von \RawFrmJun, \RawFrmABC, \RawFrmFor und \RawFrmForp} (rechts unten)
\newcommand*{\iAnd}    {\IdxFt{and}}%
\newcommand*{\iBool}   {\IdxFt{bool}}%
\newcommand*{\iImp}    {\IdxFt{imp}}%
\newcommand*{\iNand}   {\IdxFt{nand}}%
\newcommand*{\iNor}    {\IdxFt{nor}}%
\newcommand*{\iOr}     {\IdxFt{or}}%
\newcommand*{\iRep}    {\IdxFt{rep}}%
% spezielle Indizes (rechts oben)
\newcommand*{\RawMtsFinite}  {\IdxFt{\MtsFiniteLtr}}% nur die endlichen Elemente
\newcommand*{\RawFrmPolnisch}{\IdxFt{\FrmPolnischLtr}}% in Polnischer Notation
\newcommand*{\RawFrmLogisch} {\IdxFt{\FrmLogischLtr}}% die Aussagenlogik betreffend
\newcommand*{\links} [1]{#1^{\scriptscriptstyle <}}% linkes  Element vom Paar
\newcommand*{\rechts}[1]{#1^{\scriptscriptstyle >}}% rechtes Element vom Paar

% abgeleitete Mengen - ohne Verweis ins Glossar
\newcommand*{\RawMtsFol} {\SetOp{\MtsFolLtr}}%  Menge der Folgen             auf
\newcommand*{\RawMtsFolf}{\RawMtsFol_{\RawMtsFinite}}% ... nur die endlichen Folgen
%TODO \MtsTup durch \MtsFolf ersetzen
\newcommand*{\RawMtsTup} {\SetOp{\MtsTupLtr}}%  Menge der Tupel              auf
\newcommand*{\RawMtsPot} {\SetOp{\MtsPotLtr}}%  Menge der Teilmengen         von
\newcommand*{\RawMtsPotf}{\RawMtsPot_{\RawMtsFinite}}%... nur die endlichen Teilmengen
\newcommand*{\RawMtsRel} {\SetOp{\MtsRelLtr}}%  Menge der binären Relationen auf
\newcommand*{\RawMtsRelf}{\RawMtsRel_{\RawMtsFinite}}%... nur die endlichen Relationen

% natürliche Zahlen u.a. - ohne Verweis ins Glossar
% alternativ: '{\fam5' statt '\mathbb{'
\newcommand*{\RawMtsIN}   {{\fam5\MtsINLtr}}% Menge der natürlichen Zahlen ohne 0
%\newcommand*{\RawMtsIN}{\mathbb{\MtsINLtr}}% Menge der natürlichen Zahlen ohne 0
\newcommand*{\RawMtsINo}     {\RawMtsIN_0}%   Menge der natürlichen Zahlen mit  0
\newcommand*{\RawMtsMo}  {M^0}
\newcommand*{\RawMtsMn}  {M^n}

% weitere Mengen
\newcommand*{\RawMtsSprache}         {\SetFt{\MtsSpracheLtr}}% Formel-Sprache
% ... - mit Verweis ins Glossar
\newcommand*{\MtsPotSprache}         {\ensuremath{\MtsPot(\MtsSprache)}}%        P(L)
\newcommand*{\MtsPotfSprache}        {\ensuremath{\MtsPotf(\MtsSprache)}}%      Pe(L)
\newcommand*{\MtsAllDerive}          {\ensuremath{\MtsPotSprache^2}}%            P(L)^2
\newcommand*{\MtsPotAllDerive}       {\ensuremath{\MtsPot(\MtsAllDerive)}}%    P(P(L)^2)
\newcommand*{\MtsRelAllDerive}       {\ensuremath{\MtsRel(\MtsPotSprache)}}%   R(P(L))
\newcommand*{\MtsAllSchlussregel}    {\ensuremath{\MtsPotAllDerive^2}}%        P(P(L)^2)^2
\newcommand*{\MtsRelSchlussregel}    {\ensuremath{\MtsRel(\MtsRelAllDerive)}}% R(P(L)^2)
\newcommand*{\MtsPotfAllDerive}      {\ensuremath{\MtsPotf(\MtsAllDerive)}}%  Pe(P(L)^2)
\newcommand*{\MtsRelfAllDerive}      {\ensuremath{\MtsRelf(\MtsPotSprache)}}% Re(P(L))

% Elemente und Mengen für Beweise - ohne Verweis ins Glossar
\newcommand*{\RawMtsVoraussetzung}   {\drvFt{\MtsVoraussetzungLtr}}%    eine      Voraussetzung
\newcommand*{\RawMtsVoraussetzungSet}{\SetFt{\MtsVoraussetzungSetLtr}}% Menge der Voraussetzungen
\newcommand*{\RawMtsVoraussetzungRel}{\RawMtsDerive_{\RawMtsVoraussetzungSet}}%... als Relation
\newcommand*{\RawMtsFolgerung}       {\drvFt{\MtsFolgerungLtr}}%        eine      Folgerung
\newcommand*{\RawMtsFolgerungSet}    {\SetFt{\MtsFolgerungSetLtr}}%     Menge der Folgerungen
\newcommand*{\RawMtsFolgerungRel}    {\RawMtsDerive_{\RawMtsFolgerungSet}}% ... als Relation
\newcommand*{\RawMtsErgebnis}        {\drvFt{\MtsErgebnisLtr}}%         ein       Ergebnis
\newcommand*{\RawMtsErgebnisSet}     {\SetFt{\MtsErgebnisSetLtr}}%      Menge von Ergebnissen
\newcommand*{\RawMtsErgebnisRel}     {\RawMtsDerive_{\RawMtsErgebnisSet}}% ... als Relation
\newcommand*{\RawMtsBeweisschritt}   {\ElmFt{\MtsBeweisschrittLtr}}%    ein       Beweisschritt
\newcommand*{\RawMtsBeweisschrittTup}{\vec{\RawMtsBeweisschritt}}%      Folge der Beweisschritte
\newcommand*{\RawMtsBeweisschrittSet}{\SetFt{\MtsBeweisschrittSetLtr}}% Menge der Beweisschritte
\newcommand*{\RawMtsUmwandlung}      {\ElmFt{\MtsUmwandlungLtr}}%       eine      Umwandlung
\newcommand*{\RawMtsUmwandlungTup}   {\SetFt{\MtsUmwandlungLtr}}%       Folge von Umwandlungen
\newcommand*{\RawMtsSchlussregel}    {\ElmFt{\MtsSchlussregelLtr}}%     eine      Schlussregel
\newcommand*{\RawMtsSchlussregelSet} {\SetFt{\RawMtsSchlussregel}}%     Menge von Schlussregeln
\newcommand*{\RawMtsErsetzung}       {\ElmFt{\MtsErsetzungLtr}}%        eine      Ersetzung
\newcommand*{\RawMtsErsetzungSet}    {\SetFt{\MtsErsetzungLtr}}%        Menge von Ersetzungen
\newcommand*{\RawMtsAxiom}           {\ElmFt{\MtsAxiomLtr}}%            ein       Axiom
\newcommand*{\RawMtsAxiomSet}        {\SetFt{\MtsAxiomLtr}}%            Menge von Axiomen

% Mengen der Aussagenlogik - ohne Verweis ins Glossar
\newcommand*{\RawFrmvar} {\varFt{\FrmvarLtr}}% Variablensymbol
\newcommand*{\RawFrmVar} {\SetFt{\FrmVarLtr}}% Menge der Variablensymbole
\newcommand*{\RawFrmABC} {\SetFt{\FrmABCLtr}}% Menge der Buchstaben (Alphabet) der aussagenlogischen Sprache
\newcommand*{\RawFrmJun} {\SetFt{\FrmJunLtr}}% Menge der Junktoren
\newcommand*{\RawFrmCon} {\RawFrmJun_{\IdxFt{\FrmConIdx}}}% Menge der         Konstantensymbole
\newcommand*{\RawFrmUna} {\RawFrmJun_{\IdxFt{\FrmUnaIdx}}}% Menge der unären  Operationssymbole
\newcommand*{\RawFrmBin} {\RawFrmJun_{\IdxFt{\FrmBinIdx}}}% Menge der binären Operationssymbole
\newcommand*{\RawFrmFor} {\SetFt{\FrmForLtr}^{\RawFrmLogisch}}% Menge der aussagenlogischen Formeln
\newcommand*{\RawFrmForp}{\SetFt{\FrmForLtr}^{\RawFrmLogisch\RawFrmPolnisch}}% ...in Polnischer Notation

% sonstige Makro für den Mathematiksatz ########################################

\mathtoolsset{showonlyrefs,showmanualtags}% Nur mit \ref referenzierte Gleichungen, aber alle manuellen Tags

% Gleichung im Rahmen - siehe Rautenberg Seite 389
% #1=Rahmenfarbe, #2=Hintergrundfarbe, #3=mathematische Formel, #4=Marke
\makeatletter
\def\myMathBox    {@ifnextchar[{\my@MBoxi}     {\my@MBoxii[black]}}
\def\my@MBoxi [#1]{@ifnextchar[{\my@MBoxii[#1]}{\my@MBoxii[white]}}
\def\my@MBoxii[#1][#2]#3#4{%
	\par
	\noindent\fcolorbox{#1}{#2}{%
		\parbox{\linewidth-1.5\labelwidth-2\fboxrule-2\fboxsep{#3}}%
	}%
	\parbox{1.5\labelwidth}{%
		\begin{eqnarray}\label{#4}\end{eqnarray}%
	}
	\par
}
\makeatother