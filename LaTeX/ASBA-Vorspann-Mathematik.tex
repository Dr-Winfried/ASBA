%%############################################################################%%
%%                                                                            %%
%% Datei:  ASBA-Vorspann-Mathematik.tex                                       %%
%% Inhalt: Vorspann Mathematik für ASBA                                       %%
%%                                                                            %%
%% Copyright (C) 2017  Winfried Teschers                                      %%
%%                                                                            %%
%% This program is free software: you can redistribute it and/or modify       %%
%% it under the terms of the GNU Affero General Public License as published   %%
%% by the Free Software Foundation, either version 3 of the License, or       %%
%% (at your option) any later version.                                        %%
%%                                                                            %%
%% This program is distributed in the hope that it will be useful,            %%
%% but WITHOUT ANY WARRANTY; without even the implied warranty of             %%
%% MERCHANTABILITY or FITNESS FOR A PARTICULAR PURPOSE.  See the              %%
%% GNU Affero General Public License for more details.                        %%
%%                                                                            %%
%% You should have received a copy of the GNU Affero General Public License   %%
%% along with this program.  If not, see <http://www.gnu.org/licenses/>.      %%
%%                                                                            %%
%% Dr. Winfried Teschers                                                      %%
%% Anton-Günther-Straße 26c                                                   %%
%% 91083 Baiersdorf                                                           %%
%% Germany                                                                    %%
%%                                                                            %%
%% e-mail: winfried.teschers@t-online.de                                      %%
%%                                                                            %%
%%############################################################################%%

% !TeX root = ASBA.tex
% !TeX encoding = UTF-8
% !TeX spellcheck = de_DE

% Glossareinträge werden in "ASBA-Vorspann-Glossar" definiert.
% Elemente, die in anderen Dateien als "ASBA-Mathematik.tex" verwendet werden, werden in "ASBA-Vorspann.tex" definiert.

% Metasprachliche Symbole ######################################################

\newcommand*{\srand}{\mid}% in formalen Sätzen und Schlussregeln:   ... und ...
% Nur im Mathematikmodus!
\newcommand*{\metanot}{\thicksim}%          ... gilt nicht
\newcommand*{\metaandsym}{\&}%              ... und       ...
\newcommand*{\metaand}{\mathbin\metaandsym}%... und       ... (besserer Abstand)
\newcommand*{\metaorsym}{||}%               ... oder      ...
\newcommand*{\metaor}{\mathbin\metaorsym}%  ... oder      ... (besserer Abstand)
\newcommand*{\derivesym}{\vdash}%           ... ableitbar ...
\newcommand*{\derive}{\mathbin\derivesym}%  ... ableitbar ... (besserer Abstand)
\newcommand*{\metaimp}{\Rightarrow}%    aus ... folgt              ...
\newcommand*{\metarep}{\Leftarrow}%         ... folgt aus          ...
\newcommand*{\metaequiv}{\Leftrightarrow}%  ... genau dann wenn    ...
\newcommand*{\metadefeq}{:\metaequiv}%      ... definitionsgemäß " ...
\newcommand*{\eq}{=}%                       ... gleich             ...
\newcommand*{\defeq}{\coloneqq}%            ... definitionsgemäß " ...
\newcommand*{\swap}     {\leftrightarrows}% ... vertauscht mit     ...
\newcommand*{\subst}    {\leftarrowtail}%   ... substituiert durch ...

% Mathematische Operatoren =====================================================
\DeclareMathOperator*{\stelrel} {stel_r}%      - Funktion, Relation
\DeclareMathOperator*{\stelfunc}{stel_f}%      - Funktion, Relation
\DeclareMathOperator*{\graph}{graph}%  graph   - Funktion, Relation
\DeclareMathOperator*{\traeger}{car}%  carrier - Relation
\DeclareMathOperator*{\Db}{dom}%       domain  - Funktion
\DeclareMathOperator*{\Zb}{tar}%       target  - Funktion
%%%\DeclareMathOperator*{\Wb}{ran}%       range   - Funktion
%%%\DeclareMathOperator*{\Qb}{src}%       source  - partielle Funktion
%%%\DeclareMathOperator*{command}{definition}

% Beispieloperationen ==========================================================
% \*bsp
\newcommand*{\opbsp}{\mathbin\circledast}
\newcommand*{\opubsp}{\mathbin\circleddash}
\newcommand*{\relbsp}{\mathrel\sim}
\newcommand*{\relnbsp}{\mathrel\nsim}
\newcommand*{\releqbsp}{\mathrel\simeq}
\newcommand*{\relnebsp}{\mathrel\nsimeq}
\newcommand*{\relbackbsp}{\mathrel\backsim}
\newcommand*{\relbacknbsp}{\mathrel\backnsim}
\newcommand*{\relbackeqbsp}{\mathrel\backsimeq}
\newcommand*{\relbacknebsp}{\mathrel\backnsimeq}

%%%\newcommand*{\lrelbsp}{\triangleleft}
%%%\newcommand*{\rrelbsp}{\triangleright}
%%%\newcommand*{\lreleqbsp}{\trianglelefteq}
%%%\newcommand*{\rreleqbsp}{\trianglerighteq}
%%%\newcommand*{\lrelnbsp}{\ntriangleleft}
%%%\newcommand*{\rrelnbsp}{\ntriangleright}
%%%\newcommand*{\lrelnebsp}{\ntrianglelefteq}
%%%\newcommand*{\rrelnebsp}{\ntrianglerighteq}

% Definitionen für die Tabelle der Junktoren ===================================
% \l*  -           logische Operation
% \ln* - negierter logische Operation
% Wahrheitswerte ---------------------------------------------------------------
% Konstante --------------------------------------------------------------------
\newcommand*{\ltrue} {\top}%            W       - wahr   (Wahrheitswert, Symbol)
\newcommand*{\wahr}  {\emph{wahr}}%                                       (Text)
\newcommand*{\lfalse}{\bot}%            F       - falsch (Wahrheitswert, Symbol)
\newcommand*{\falsch}{\emph{falsch}}%                                     (Text)
% unäre Operationen ------------------------------------------------------------
%            \lnot                      F W     - nicht A
% binäre Operationen -----------------------------------------------------------
%            \lor                       W W W F - A oder B
\newcommand*{\lrep}  {\leftarrow}%      W W F W - A folgt aus B
\newcommand*{\limp}  {\rightarrow}%     W F W W - aus A folgt B
\newcommand*{\lequiv}{\leftrightarrow}% W F F W - A genau dann wenn B
%            \land                      W F F F - A und B
\newcommand*{\lnand} {\uparrow}%        F W W W - nicht   (A und  B)
\newcommand*{\lxor}  {+}%               F W W F - entweder A oder B
\newcommand*{\lnor}  {\downarrow}%      F F F W - weder    A noch B

% Verwendete Konstanten- und Mengenbezeichnungen ===============================

% \gs* = globales Symbol
\newcommand*{\gsN} {\mathbb{N}}%   Menge der natürlichen Zahlen ohne           0
\newcommand*{\gsNo}{\mathbb{N}_0}% Menge der natürlichen Zahlen einschließlich 0

% Elemente und Mengen für Beweise
\newcommand*{\formulaset}       {\mathcal{L}}             % [l]anguage
\newcommand*{\prerequisite}     {V}                       % [V]oraussetzung
\newcommand*{\prerequisiteset}  {\mathcal{\prerequisite}}
\newcommand*{\conclusion}       {F}                       % [F]olgerung
\newcommand*{\conclusionset}    {\mathcal{\conclusion}}
\newcommand*{\proofstep}        {B}                       % [B]eweisschritt
\newcommand*{\proofstepsequenz} {\mathcal{S}}             % [s]equenz
\newcommand*{\proofstepset}     {\mathcal{\proofstep}}
\newcommand*{\transformation}   {T}                       % [T]ransformation
\newcommand*{\transformationset}{\mathcal{\transformation}}
\newcommand*{\conclusionrule}   {C}                       % [c]onclusion
\newcommand*{\conclusionruleset}{\mathcal{\conclusionrule}}
\newcommand*{\substitution}     {E}                       % [E]rsetzung
\newcommand*{\substitutionset}  {\mathcal{\substitution}}

% Potenzmenge:                       P
% neue Mengensymbole: ABC EF   JKL  O Q  TUV
% frei:                  D  GHI   MN   R    WXYZ

% \al... = aussagenlogisch
\newcommand*{\alvar} {q}%               Name einer Variablen
% Mengen der Aussagenlogik
\newcommand*{\alVar} {\mathcal{Q}}%     Menge der Variablensymbole
\newcommand*{\alCon} {\mathcal{K}}%     Menge der [K]onstantensymbole
\newcommand*{\alUna} {\mathcal{U}}%     Menge der [u]nären Operationssymbole
\newcommand*{\alBin} {\mathcal{O}}%     Menge der binären [O]perationssymbole
\newcommand*{\alJun} {\mathcal{J}}%     Menge der Operationssymbole ([J]unktor)
\newcommand*{\alABC} {\mathcal{A}}%     [A]lphabet der aussagenlogischen Sprache
\newcommand*{\alFor} {\mathcal{L}}%     Menge der Formeln (Worte) ([l]anguage)
\newcommand*{\alForp}{\alFor^\mathrm{p}}%   ... in polnischer Notation
% Indizes für Teilmengen von \alJun, \alABC, \alFor und \alForp
\newcommand*{\iAnd} {\mathrm{and}}%
\newcommand*{\iBool}{\mathrm{bool}}%
\newcommand*{\iImp} {\mathrm{imp}}%
\newcommand*{\iNand}{\mathrm{nand}}%
\newcommand*{\iNor} {\mathrm{nor}}%
\newcommand*{\iOr}  {\mathrm{or}}%
\newcommand*{\iRep} {\mathrm{rep}}%
% Konstanten und Symbole
\newcommand*{\true}  {\mathrm{true}}
\newcommand*{\false} {\mathrm{false}}

% sonstige Kommandos für den Mathematiksatz ####################################

\mathtoolsset{showonlyrefs,showmanualtags}% Nur mit \ref referenzierte Gleichungen, aber alle manuellen Tags
