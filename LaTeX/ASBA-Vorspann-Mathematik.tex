%%############################################################################%%
%%                                                                            %%
%% Datei:  ASBA-Vorspann-Mathematik.tex                                       %%
%% Inhalt: Vorspann Mathematik für ASBA                                       %%
%%                                                                            %%
%% Copyright (C) 2017  Winfried Teschers                                      %%
%%                                                                            %%
%% This program is free software: you can redistribute it and/or modify       %%
%% it under the terms of the GNU Affero General Public License as published   %%
%% by the Free Software Foundation, either version 3 of the License, or       %%
%% (at your option) any later version.                                        %%
%%                                                                            %%
%% This program is distributed in the hope that it will be useful,            %%
%% but WITHOUT ANY WARRANTY; without even the implied warranty of             %%
%% MERCHANTABILITY or FITNESS FOR A PARTICULAR PURPOSE.  See the              %%
%% GNU Affero General Public License for more details.                        %%
%%                                                                            %%
%% You should have received a copy of the GNU Affero General Public License   %%
%% along with this program.  If not, see <http://www.gnu.org/licenses/>.      %%
%%                                                                            %%
%% Dr. Winfried Teschers                                                      %%
%% Anton-Günther-Straße 26c                                                   %%
%% 91083 Baiersdorf                                                           %%
%% Germany                                                                    %%
%%                                                                            %%
%% e-mail: winfried.teschers@t-online.de                                      %%
%%                                                                            %%
%%############################################################################%%

% !TeX root = ASBA.tex
% !TeX encoding = UTF-8
% !TeX spellcheck = de_DE

% Glossareinträge werden in "ASBA-Vorspann-Glossar" definiert.
% Elemente, die in anderen Dateien als "ASBA-Mathematik.tex" verwendet werden, werden in "ASBA-Vorspann.tex" definiert.
% Namensbestandteile mit besonderer Bedeutung:
%   Bezeichnungen:
%     <Name>Set     = Menge    von <Name>n (keine bis alle)
%     <Name>Rel     = Relation von <Name>n
%     <Name>Tup     = Tupel    von <Name>n
%     All<Name>     = Menge  aller <Name>n
%   Ergebnisse von Operationen auf Mengen:
%     Pot<Menge>    = Menge    der Teilmengen                   von <Name>
%     Rel<Menge>    = Menge    der binären Relationen           auf <Name>
%     Relf<Menge>   = Menge    der endlichen binären Relationen auf <Name>
%     Tup<Menge>    = Menge    der Tupel                        auf <Name>
%   Sonstiges:
%     <Relation>Bck = Umkehrrelation
%     <Operator>Eq  = Operator oder Gleich
%     <Operator>N   = negierter Operator
%     <Oper.>BckEqN = Kombination in dieser Reihenfolge
%     \Tag<Tag>     =
%
% Folgende Makros sind alles "eigene"
%   \Ltr...    - nur Zeichen für (die Menge)     ...       (für       Textmodus)
%   \Str...    - nur Text    für (die Operation) ...       (für       Textmodus)
%   \StrMtsIdx...    - Text für einen Index für        ...       (für Mathematikmodus)
%   \Raw...    - Symbol/Text ohne Verweis;erfordert bei Symbolen Mathematikmodus
%   \...Bsp... - Beispielsymbol für Formel- oder Metasprache (f.Mathematikmodus)
%   \Bsp...    -                          mit Verweis ins Symbolverzeichnis
%   \...Mts... - Symbol der Metasprache                    (für Mathematikmodus)
%   \Mts...    -                          mit Verweis ins Symbolverzeichnis
%   \...Ojk... - Symbol der Formelsprache                  (für Mathematikmodus)
%   \Ojk...    -                          mit Verweis ins Symbolverzeichnis
%   \...Txt... - individuelle Bezeichnung                  (für       Textmodus)
%   \Txt...    -                          mit Verweis in  Glossar und Index
%
% Weitere "eigene" Kombinationen: OpU, OpB, Bck, Idx, Ltr, Txt

% Beispielsymbole für Operationen und Relationen ===============================
% \RawBsp*; OpU=unär, OpB=binär, Rel=Relation, Bck=Umkehr-; N=nicht, Eq=gleich
\newcommand*{\RawBspOpU}      {\mathbin{\circleddash}}
\newcommand*{\RawBspOpB}      {\mathbin{\circledast}}
\newcommand*{\RawBspRel}      {\mathrel{\prec}}
\newcommand*{\RawBspRelEq}    {\mathrel{\preceq}}
\newcommand*{\RawBspRelBck}   {\mathrel{\succ}}
\newcommand*{\RawBspRelBckEq} {\mathrel{\succeq}}
\newcommand*{\RawBspRelN}     {\mathrel{\nprec}}
\newcommand*{\RawBspRelEqN}   {\mathrel{\npreceq}}
\newcommand*{\RawBspRelBckN}  {\mathrel{\nsucc}}
\newcommand*{\RawBspRelBckEqN}{\mathrel{\nsucceq}}

% Metasymbole ==================================================================
% \RawMts*
% Metaoperationen, -relationen u.a. (für Aussagen) -----------------------------
\newcommand*{\RawMtsNot}     {\mathbin{\thicksim}}%               ... gilt nicht
\newcommand*{\RawMtsAnd}     {\mathbin{\&}}%                        ... und  ...
\newcommand*{\RawMtsOr}      {\mathbin{\parallel}}%                 ... oder ...
\newcommand*{\RawMtsImp}     {\mathrel{\Rightarrow}}% von ... folgt          ...
\newcommand*{\RawMtsRep}     {\mathrel{\Leftarrow}}%      ... folgt von      ...
\newcommand*{\RawMtsEquiv}   {\mathrel{\Leftrightarrow}}% .. genau dann, wenn ..
\newcommand*{\RawMtsEq}      {\mathrel{=\mkern-10mu=}}%          ...  gleich ...
\newcommand*{\RawMtsEqN}     {\mathrel{=\mkern-16mu/\mkern-16mu=}}% . ungleich .
\newcommand*{\RawMtsAequiv}  {\mathrel{\equiv}}%     ...       äquivalent zu ...
\newcommand*{\RawMtsNAequiv} {\mathrel{\nequiv}}%    ... nicht äquivalent zu ...
\newcommand*{\RawMtsDefEquiv}{\mathrel{:\mkern-2mu\RawMtsEquiv}}% def.gemäß -"-
\newcommand*{\RawMtsDefEq}   {\mathrel{:\mkern-2mu\RawMtsEq}}% def.gemäß gleich
\newcommand*{\RawMtsUnd}     {\mathbin{\mid}}%nur in Schlussregeln: ... und  ...
\newcommand*{\RawMtsDerive}  {\mathrel{\vdash}}%          ... ableitbar      ...
\newcommand*{\RawMtsSwap}    {\mathbin{\leftrightarrows}}% .. vertauscht mit ...
\newcommand*{\RawMtsSubst}   {\mathbin{\leftarrowtail}}%.. substituiert durch ..
% Elementrelationen (für Elemente und Mengen) ----------------------------------
\newcommand*{\RawMtsIn}       {\in}%        ist  Element aus (der Menge)
\newcommand*{\RawMtsNi}       {\ni}%       (die Menge) enthält nicht das Element
\newcommand*{\RawMtsInN}      {\notin}%     ist  Element aus (der Menge)
\newcommand*{\RawMtsNiN}      {\notni}%    (die Menge) enthält nicht das Element
% Mengenrelationen (für Mengen) ------------------------------------------------
\newcommand*{\RawMtsSubset}   {\subset}%    ist        echte  Teilmenge von
\newcommand*{\RawMtsSubsetEq} {\subseteq}%  ist (gleich oder) Teilmenge von
\newcommand*{\RawMtsSubsetN}  {\nsubset}%   ist  keine echte  Teilmenge von
\newcommand*{\RawMtsSubsetEqN}{\nsubseteq}% ist  keine        Teilmenge von
\newcommand*{\RawMtsSupset}   {\supset}%    ist        echte  Obermenge von
\newcommand*{\RawMtsSupsetEq} {\supseteq}%  ist (gleich oder) Obermenge von
\newcommand*{\RawMtsSupsetN}  {\nsupset}%   ist  keine echte  Obermenge von
\newcommand*{\RawMtsSupsetEqN}{\nsupseteq}% ist  keine        Obermenge von
% Mengenoperationen (für Mengen) -----------------------------------------------
\newcommand*{\RawMtsCap}      {\cap}%       Durchschnitt          von Mengen
\newcommand*{\RawMtsCup}      {\cup}%       Vereinigung           von Mengen
\newcommand*{\RawMtsSetminus} {\setminus}%  Differenz             von Mengen
\newcommand*{\RawMtsTimes}    {\times}%     karthesisches Produkt von Mengen
\newcommand*{\RawMtsEmptyset} {\emptyset}%  die leere Menge
% Komponentenrelationen (für Komponenten und Folgen) ---------------------------
\newcommand*{\RawMtsSeqIn}    {\sqsubset\mkern-19mu-}%  ist  Komponente (der Folge)
\newcommand*{\RawMtsSeqNi}    {\sqsupset\mkern-19mu-}%  (die Folge) enthält nicht das Symbol
\newcommand*{\RawMtsSeqInN}   {\nsqsubset\mkern-19mu-}% ist  Komponente (der Folge)
\newcommand*{\RawMtsSeqNiN}   {\nsqsupset\mkern-19mu-}% (die Folge) enthält nicht das Symbol
% Folgenrelationen (für Folgen) ------------------------------------------------
\newcommand*{\RawMtsSubseq}   {\sqsubset}%    ist        echte  Teilmenge von
\newcommand*{\RawMtsSubseqEq} {\sqsubseteq}%  ist (gleich oder) Teilmenge von
\newcommand*{\RawMtsSubseqN}  {\nsqsubset}%   ist  keine echte  Teilmenge von
\newcommand*{\RawMtsSubseqEqN}{\nsqsubseteq}% ist  keine        Teilmenge von
\newcommand*{\RawMtsSupseq}   {\sqsupset}%    ist        echte  Obermenge von
\newcommand*{\RawMtsSupseqEq} {\sqsupseteq}%  ist (gleich oder) Obermenge von
\newcommand*{\RawMtsSupseqN}  {\nsqsupset}%   ist  keine echte  Obermenge von
\newcommand*{\RawMtsSupseqEqN}{\nsqsupseteq}% ist  keine        Obermenge von

% Text-, Meta- und Objekt-Wahrheitswerte
\newcommand*{\RawTxtFalse}    {\emph{\StrTxtFalse}}%                      (Text)
\newcommand*{\RawTxtTrue}     {\emph{\StrTxtTrue}}%                       (Text)
\newcommand*{\RawMtsFalse}    {\mathord{\mathrm{\StrMtsFalse}}}%F - falsch (Sym)
\newcommand*{\RawMtsTrue}     {\mathord{\mathrm{\StrMtsTrue}}}% W - wahr   (Sym)
\newcommand*{\RawOjkFalse}    {\mathord{\bot}}%                         (Symbol)
\newcommand*{\RawOjkTrue}     {\mathord{\top}}%                         (Symbol)

% Definitionen für die Tabelle der Junktoren -----------------------------------
% \RawOjk*
% Wahrheitswert von A                            W W F F
% Wahrheitswert von B                            W F W F
% unäre Operationen ------------------------------------------------------------
\newcommand*{\RawOjkNot}      {\lnot}%           F W - - - nicht A
% binäre Operationen -----------------------------------------------------------
\newcommand*{\RawOjkAnd}      {\land}%           W F F F - A und B
\newcommand*{\RawOjkOr}       {\lor}%            W W W F - A oder B
\newcommand*{\RawOjkImp}      {\rightarrow}%     W F W W - von A folgt B
\newcommand*{\RawOjkRep}      {\leftarrow}%      W W F W - A folgt von B
\newcommand*{\RawOjkEquiv}    {\leftrightarrow}% W F F W - A genau dann wenn B
\newcommand*{\RawOjkNand}     {\uparrow}%        F W W W - nicht   (A und  B)
\newcommand*{\RawOjkNor}      {\downarrow}%      F F F W - weder    A noch B
\newcommand*{\RawOjkXor}      {\dot\lor}%        F W W F - entweder A oder B

% außerhalb der Tabelle --------------------------------------------------------
\newcommand*{\RawOjkEq}       {=}%          Gleichheit   in Formeln
\newcommand*{\RawOjkEqN}      {\ne}%        Ungleichheit in Formeln
% Quantoren --------------------------------------------------------------------
\newcommand*{\RawMtsForall}   {\forall}%   für alle          <x>         gilt:
\newcommand*{\RawMtsExists}   {\exists}%     es gibt       ein <x> für das gilt:
\newcommand*{\RawMtsExione}   {\dot\exists}% es gibt genau ein <x> für das gilt:
\newcommand*{\RawOjkForall}   {\bigwedge}%   für alle          <x>         gilt:
\newcommand*{\RawOjkExists}   {\bigvee}%     es gibt       ein <x> für das gilt:
\newcommand*{\RawOjkExione}   {\dot\bigvee}% es gibt genau ein <x> für das gilt:

% weitere Symbole --------------------------------------------------------------
\newcommand*{\RawMtsFktSep}   {:}%                 f \MtsFktSep A \MtsFktArrow B
\newcommand*{\RawMtsFktArrow} {\rightarrow}

% Neue Metaoperationen ---------------------------------------------------------
\DeclareMathOperator{\RawMtsGraph}  {\StrMtsGraph}% Graph von        Funktion/Relation
\DeclareMathOperator{\RawMtsTraeger}{\StrMtsTraeger}% Trägermenge einer       Relation
\DeclareMathOperator{\RawMtsStel}   {\StrMtsStel}% Stelligkeit einer Funktion/Relation
\DeclareMathOperator{\RawMtsStelF}  {\StrMtsStel_f}% Stelligkeit für [F]unktionen
\DeclareMathOperator{\RawMtsStelR}  {\StrMtsStel_r}% Stelligkeit für [R]elationen
\DeclareMathOperator{\RawMtsQb}     {\StrMtsQb}% Quellbereich einer partiellen Funktion
\DeclareMathOperator{\RawMtsDb}     {\StrMtsDb}% Definitionsbereich einer      Funktion
\DeclareMathOperator{\RawMtsZb}     {\StrMtsZb}% Zielbereich        einer      Funktion
\DeclareMathOperator{\RawMtsWb}     {\StrMtsWb}% Wertebereich       einer      Funktion
\DeclareMathOperator{\RawMtsLen}    {\StrMtsLen}% Länge            eines/r Tupels/Folge
\DeclareMathOperator{\RawMtsSet}    {\StrMtsSet}% Komponentenmenge eines/r Tupels/Folge

% Schriftarten
\newcommand*{\varFt}[1]  {\mathbf{#1}}% Variable aus Alphabet   (Kleinbuchstabe)
\newcommand*{\DrvFt}[1]  {\mathbf{#1}}% Mengen von Ableitungen   (Großbuchstabe)
\newcommand*{\drvFt}[1]  {\mathbf{#1}}% ein Element davon (Klein-/Großbuchstabe)
\newcommand*{\IdxFt}[1]  {\mathrm{#1}}% fester Index      (Klein-/Großbuchstabe)
\newcommand*{\SetFt}[1] {\mathcal{#1}}% vorgegebene Menge        (Großbuchstabe)
\newcommand*{\ElmFt}[1]          {#1}%  ein Element davon        (Großbuchstabe)
\newcommand*{\SetOp}[1]{\mathfrak{#1}}% Mengenoperation                   (Text)

% spezielle Indizes (rechts oben)
\newcommand*{\LtrMtsIdxLogisch} {A}%    die Logik betreffend
% spezielle Indizes für Teilmengen (rechts unten)
\newcommand*{\StrMtsIdxBin}     {b}%    binär
\newcommand*{\StrMtsIdxCon}     {c}%    constant; konstant
\newcommand*{\StrMtsIdxUna}     {u}%    unär
\newcommand*{\StrMtsIdxAnd}     {and}%  Signatur not, and
\newcommand*{\StrMtsIdxBool}    {bool}% Signatur not, and, or
\newcommand*{\StrMtsIdxImp}     {imp}%  Signatur not, imp
\newcommand*{\StrMtsIdxNand}    {nand}% Signatur      nand
\newcommand*{\StrMtsIdxNor}     {nor}%  Signatur      nor
\newcommand*{\StrMtsIdxOr}      {or}%   Signatur not, or
\newcommand*{\StrMtsIdxRep}     {rep}%  Signatur not, rep
% spezielle Indizes mit Schriftart
\newcommand*{\RawMtsIdxPolnisch}{\IdxFt{\LtrMtsIdxPolnisch}}% ^
\newcommand*{\RawMtsIdxLogisch} {\IdxFt{\LtrMtsIdxLogisch}}%  ^
\newcommand*{\RawMtsIdxEndlich} {\IdxFt{\LtrMtsIdxEndlich}}%  _
\newcommand*{\RawMtsIdxGraph}   {\IdxFt{\LtrMtsIdxGraph}}%    _
\newcommand*{\RawMtsIdxBin}     {\IdxFt{\StrMtsIdxBin}}
\newcommand*{\RawMtsIdxCon}     {\IdxFt{\StrMtsIdxCon}}
\newcommand*{\RawMtsIdxUna}     {\IdxFt{\StrMtsIdxUna}}
\newcommand*{\RawMtsIdxAnd}     {\IdxFt{\StrMtsIdxAnd}}
\newcommand*{\RawMtsIdxBool}    {\IdxFt{\StrMtsIdxBool}}
\newcommand*{\RawMtsIdxImp}     {\IdxFt{\StrMtsIdxImp}}
\newcommand*{\RawMtsIdxNand}    {\IdxFt{\StrMtsIdxNand}}
\newcommand*{\RawMtsIdxNor}     {\IdxFt{\StrMtsIdxNor}}
\newcommand*{\RawMtsIdxOr}      {\IdxFt{\StrMtsIdxOr}}
\newcommand*{\RawMtsIdxRep}     {\IdxFt{\StrMtsIdxRep}}

% Indexoperationenen
\newcommand*{\links} [1] {#1^{\scriptscriptstyle <}}% linkes  Element vom Paar
\newcommand*{\rechts}[1] {#1^{\scriptscriptstyle >}}% rechtes Element vom Paar

% abgeleitete Mengen - ohne Verweis ins Glossar
\newcommand*{\RawMtsFol} {\SetOp{\LtrMtsFol}}%  Menge der Folgen             auf
\newcommand*{\RawMtsFolf}{\RawMtsFol_{\RawMtsIdxEndlich}}% ... nur die endlichen Folgen
\newcommand*{\RawMtsTup} {\SetOp{\LtrMtsTup}}%  Menge der Tupel              auf
\newcommand*{\RawMtsPot} {\SetOp{\LtrMtsPot}}%  Menge der Teilmengen         von
\newcommand*{\RawMtsPotf}{\RawMtsPot_{\RawMtsIdxEndlich}}%... nur die endlichen Teilmengen
\newcommand*{\RawMtsRel} {\SetOp{\LtrMtsRel}}%  Menge der binären Relationen auf
\newcommand*{\RawMtsRelf}{\RawMtsRel_{\RawMtsIdxEndlich}}%... nur die endlichen Relationen

% natürliche Zahlen u.a. - ohne Verweis ins Glossar
% alternativ: '{\fam5' statt '\mathbb{'
\newcommand*{\RawMtsIN}   {{\fam5\LtrMtsIN}}%  Menge der natürlichen Zahlen ohne 0
%\newcommand*{\RawMtsIN}{\mathbb{\LtrMtsIN}}%  Menge der natürlichen Zahlen ohne 0
\newcommand*{\RawMtsINo}        {\RawMtsIN_0}% Menge der natürlichen Zahlen mit  0
\newcommand*{\RawMtsMo}               {M^0}
\newcommand*{\RawMtsMn}               {M^n}

% weitere Mengen
\newcommand*{\RawMtsSprache}          {\SetFt{\LtrMtsSprache}}% Formel-Sprache
% ... - mit Verweis ins Glossar
\newcommand*{\MtsPotSprache}          {\ensuremath{\MtsPot(\MtsSprache)}}%        P(L)
\newcommand*{\MtsPotfSprache}         {\ensuremath{\MtsPotf(\MtsSprache)}}%      Pe(L)
\newcommand*{\MtsAllDerive}           {\ensuremath{\MtsPotSprache^2}}%            P(L)^2
\newcommand*{\MtsPotAllDerive}        {\ensuremath{\MtsPot(\MtsAllDerive)}}%    P(P(L)^2)
\newcommand*{\MtsRelAllDerive}        {\ensuremath{\MtsRel(\MtsPotSprache)}}%   R(P(L))
\newcommand*{\MtsAllSchlussregel}     {\ensuremath{\MtsPotAllDerive^2}}%        P(P(L)^2)^2
\newcommand*{\MtsRelSchlussregel}     {\ensuremath{\MtsRel(\MtsRelAllDerive)}}% R(P(L)^2)
\newcommand*{\MtsPotfAllDerive}       {\ensuremath{\MtsPotf(\MtsAllDerive)}}%  Pe(P(L)^2)
\newcommand*{\MtsRelfAllDerive}       {\ensuremath{\MtsRelf(\MtsPotSprache)}}% Re(P(L))

% Elemente und Mengen für Beweise - ohne Verweis ins Glossar
\newcommand*{\RawMtsPraemisse}        {\drvFt{\LtrMtsPraemisse}}%         eine      Prämisse
\newcommand*{\RawMtsPraemisseSet}     {\SetFt{\LtrMtsPraemisseSet}}%      Menge der Prämissen
\newcommand*{\RawMtsPraemisseRel}     {\RawMtsDerive_{\RawMtsPraemisseSet}}%... als Relation
\newcommand*{\RawMtsKonklusion}       {\drvFt{\LtrMtsKonklusion}}%        eine      Konklusion
\newcommand*{\RawMtsKonklusionSet}    {\SetFt{\LtrMtsKonklusionSet}}%     Menge der Konklusionen
\newcommand*{\RawMtsKonklusionRel}    {\RawMtsDerive_{\RawMtsKonklusionSet}}%.. als Relation
\newcommand*{\RawMtsErgebnis}         {\drvFt{\LtrMtsErgebnis}}%          ein       Ergebnis
\newcommand*{\RawMtsErgebnisSet}      {\SetFt{\LtrMtsErgebnisSet}}%       Menge von Ergebnissen
\newcommand*{\RawMtsErgebnisRel}      {\RawMtsDerive_{\RawMtsErgebnisSet}}% ... als Relation
\newcommand*{\RawMtsBeweisschritt}    {\ElmFt{\LtrMtsBeweisschritt}}%     ein       Beweisschritt
\newcommand*{\RawMtsBeweisschrittTup} {\vec{\RawMtsBeweisschritt}}%       Folge der Beweisschritte
\newcommand*{\RawMtsBeweisschrittSet} {\SetFt{\LtrMtsBeweisschrittSet}}%  Menge der Beweisschritte
\newcommand*{\RawMtsTransformation}   {\ElmFt{\LtrMtsTransformation}}%    eine      Transformation
\newcommand*{\RawMtsTransformationTup}{\SetFt{\LtrMtsTransformation}}%    Folge von Transformationen
\newcommand*{\RawMtsSchlussregel}     {\ElmFt{\LtrMtsSchlussregel}}%      eine      Schlussregel
\newcommand*{\RawMtsSchlussregelSet}  {\SetFt{\RawMtsSchlussregel}}%      Menge von Schlussregeln
\newcommand*{\RawMtsErsetzung}        {\ElmFt{\LtrMtsErsetzung}}%         eine      Ersetzung
\newcommand*{\RawMtsErsetzungSet}     {\SetFt{\LtrMtsErsetzung}}%         Menge von Ersetzungen
\newcommand*{\RawMtsAxiom}            {\ElmFt{\LtrMtsAxiom}}%             ein       Axiom
\newcommand*{\RawMtsAxiomSet}         {\SetFt{\LtrMtsAxiom}}%             Menge von Axiomen

% Mengen der Aussagenlogik - ohne Verweis ins Glossar
\newcommand*{\RawOjkvar} {\varFt{\LtrOjkvar}}% Variablensymbol
\newcommand*{\RawOjkVar} {\SetFt{\LtrOjkVar}}% Menge der Variablensymbole
\newcommand*{\RawOjkABC} {\SetFt{\LtrOjkABC}}% Menge der Buchstaben (Alphabet) der aussagenlogischen Sprache
\newcommand*{\RawOjkJun} {\SetFt{\LtrOjkJun}}% Menge der Junktoren
\newcommand*{\RawOjkCon} {\RawOjkJun_{\RawMtsIdxCon}}% Menge der         Konstantensymbole
\newcommand*{\RawOjkUna} {\RawOjkJun_{\RawMtsIdxUna}}% Menge der unären  Operationssymbole
\newcommand*{\RawOjkBin} {\RawOjkJun_{\RawMtsIdxBin}}% Menge der binären Operationssymbole
\newcommand*{\RawOjkFor} {\SetFt{\LtrOjkFor}^{\RawMtsIdxLogisch}}% Menge der aussagenlogischen Formeln
\newcommand*{\RawOjkForp}{\SetFt{\LtrOjkFor}^{\RawMtsIdxLogisch\RawMtsIdxPolnisch}}% ...in Polnischer Notation

% Prädikate und praedikatähnliche Makros =======================================

\newcommand*{\FunktionDef}[3]{\ensuremath{   #1 \MtsFktSep #2 \MtsFktArrow #3 }}
\newcommand*   {\MengeDef}[2]{\ensuremath{\{ #1 \MtsSetSep #2               \}}}

% sonstige Makro für den Mathematiksatz ########################################

\mathtoolsset{showonlyrefs,showmanualtags}% Nur mit \ref referenzierte Gleichungen, aber alle manuellen Tags

% Gleichung im Rahmen - siehe Rautenberg Seite 389
%%% #1=Rahmenfarbe, #2=Hintergrundfarbe, #3=mathematische Formel, #4=Marke
%%%\makeatletter
%%%\def\myMathBox    {@ifnextchar[{\my@MBoxi}     {\my@MBoxii[black]}}
%%%\def\my@MBoxi [#1]{@ifnextchar[{\my@MBoxii[#1]}{\my@MBoxii[white]}}
%%%\def\my@MBoxii[#1][#2]#3#4{%
%%%	\par
%%%	\noindent\fcolorbox{#1}{#2}{%
%%%		\parbox{\linewidth-1.5\labelwidth-2\fboxrule-2\fboxsep{#3}}%
%%%	}%
%%%	\parbox{1.5\labelwidth}{%
%%%		\begin{eqnarray}\label{#4}\end{eqnarray}%
%%%	}
%%%	\par
%%%}
%%%\makeatother
