%%############################################################################%%
%%                                                                            %%
%% Datei:  ASBA-Vorspann-Mathematik.tex                                       %%
%% Inhalt: Vorspann Mathematik für ASBA                                       %%
%%                                                                            %%
%% Copyright (C) 2017  Winfried Teschers                                      %%
%%                                                                            %%
%% This program is free software: you can redistribute it and/or modify       %%
%% it under the terms of the GNU Affero General Public License as published   %%
%% by the Free Software Foundation, either version 3 of the License, or       %%
%% (at your option) any later version.                                        %%
%%                                                                            %%
%% This program is distributed in the hope that it will be useful,            %%
%% but WITHOUT ANY WARRANTY; without even the implied warranty of             %%
%% MERCHANTABILITY or FITNESS FOR A PARTICULAR PURPOSE.  See the              %%
%% GNU Affero General Public License for more details.                        %%
%%                                                                            %%
%% You should have received a copy of the GNU Affero General Public License   %%
%% along with this program.  If not, see <http://www.gnu.org/licenses/>.      %%
%%                                                                            %%
%% Dr. Winfried Teschers                                                      %%
%% Anton-Günther-Straße 26c                                                   %%
%% 91083 Baiersdorf                                                           %%
%% Germany                                                                    %%
%%                                                                            %%
%% e-mail: winfried.teschers@t-online.de                                      %%
%%                                                                            %%
%%############################################################################%%

% !TeX root = ASBA.tex
% !TeX encoding = UTF-8
% !TeX spellcheck = de_DE

% Glossareinträge werden in "ASBA-Vorspann-Glossar" definiert.
% Elemente, die in anderen Dateien als "ASBA-Mathematik.tex" verwendet werden, werden in "ASBA-Vorspann.tex" definiert.

% Metasprachliche Symbole ######################################################

% Nur im Mathematikmodus!
\newcommand*{\metanot}   {\thicksim}%             ... gilt nicht
\newcommand*{\metaand}   {\mathbin{\&}}%          ... und       ...
\newcommand*{\srand}     {\mathrel{\mid}}%        ... und ... (in Schlussregeln)
\newcommand*{\metaor}    {\mathbin{||}}%          ... oder
\newcommand*{\derive}    {\mathbin{\vdash}}%      ... ableitbar
\newcommand*{\metaimp}   {\Rightarrow}%       aus ... folgt              ...
\newcommand*{\metarep}   {\Leftarrow}%            ... folgt aus          ...
\newcommand*{\metaequiv} {\Leftrightarrow}%       ... genau dann wenn    ...
\newcommand*{\metadefeq} {:\metaequiv}%           ... definitionsgemäß " ...
\newcommand*{\eq}        {=}%                     ... gleich             ...
\newcommand*{\defeq}     {\coloneqq}%             ... definitionsgemäß " ...
\newcommand*{\swap}      {\leftrightarrows}%      ... vertauscht mit     ...
\newcommand*{\subst}     {\leftarrowtail}%        ... substituiert durch ...
%%%\newcommand*{\wie}       {\stackrel{\wedge}{=}}%  ... entspricht         ...

% Mathematische Operatoren =====================================================
\DeclareMathOperator*{\stelrel} {stel_r}%        - Funktion, Relation
\DeclareMathOperator*{\stelfunc}{stel_f}%        - Funktion, Relation
\DeclareMathOperator*{\graph}   {graph}% graph   - Funktion, Relation
\DeclareMathOperator*{\traeger} {car}%   carrier - Relation
\DeclareMathOperator*{\Db}{dom}%         domain  - Funktion
\DeclareMathOperator*{\Zb}{tar}%         target  - Funktion
%%%\DeclareMathOperator*{\Wb}{ran}%         range   - Funktion
%%%\DeclareMathOperator*{\Qb}{src}%         source  - partielle Funktion
%%%\DeclareMathOperator*{command}{definition}%

% Beispieloperationen ==========================================================
% \*bsp
\newcommand*{\opbsp}       {\mathbin{\circledast}}
\newcommand*{\opubsp}      {\mathbin{\circleddash}}
\newcommand*{\relbsp}      {\mathrel{\sim}}
\newcommand*{\relnbsp}     {\mathrel{\nsim}}
\newcommand*{\releqbsp}    {\mathrel{\simeq}}
\newcommand*{\relnebsp}    {\mathrel{\nsimeq}}
\newcommand*{\relbackbsp}  {\mathrel{\backsim}}
\newcommand*{\relbacknbsp} {\mathrel{\backnsim}}
\newcommand*{\relbackeqbsp}{\mathrel{\backsimeq}}
\newcommand*{\relbacknebsp}{\mathrel{\backnsimeq}}

% Definitionen für die Tabelle der Junktoren ===================================
% \l*  -           logische Operation
% \ln* - negierter logische Operation
% Wahrheitswerte ---------------------------------------------------------------
% Konstante --------------------------------------------------------------------
\newcommand*{\ltrue} {\top}%            W       - wahr   (Wahrheitswert, Symbol)
\newcommand*{\wahr}  {\emph{wahr}}%                                       (Text)
\newcommand*{\lfalse}{\bot}%            F       - falsch (Wahrheitswert, Symbol)
\newcommand*{\falsch}{\emph{falsch}}%                                     (Text)
% unäre Operationen ------------------------------------------------------------
%            \lnot                      F W     - nicht A
% binäre Operationen -----------------------------------------------------------
%            \lor                       W W W F - A oder B
\newcommand*{\lrep}  {\leftarrow}%      W W F W - A folgt aus B
\newcommand*{\limp}  {\rightarrow}%     W F W W - aus A folgt B
\newcommand*{\lequiv}{\leftrightarrow}% W F F W - A genau dann wenn B
%            \land                      W F F F - A und B
\newcommand*{\lnand} {\uparrow}%        F W W W - nicht   (A und  B)
\newcommand*{\lxor}  {+}%               F W W F - entweder A oder B
\newcommand*{\lnor}  {\downarrow}%      F F F W - weder    A noch B

% Verwendete Konstanten- und Mengenbezeichnungen ===============================

% spezielle Indizes
\newcommand*{\links} [1]{#1^{\scriptscriptstyle <}}
\newcommand*{\rechts}[1]{#1^{\scriptscriptstyle >}}

% natürliche Zahlen
\newcommand*{\IN}{{\fam5N}}% Menge der natürlichen Zahlen ohne           0
\newcommand*{\INo}{\IN_0}%   Menge der natürlichen Zahlen einschließlich 0

% Schriftarten
\newcommand*{\abl}[1]{\mathbf{#1}}% Ableitungen (Kleinbuchstaben)
\newcommand*{\Abl}[1]{\mathbf{#1}}% Mengen von Ableitungen (Großbuchstaben)

% Elemente und Mengen für Beweise
\newcommand*{\formulaSet}       {\mathcal{L}}           % [l]anguage
\newcommand*{\formulaSetSet}    {\Pot(\formulaSet)}     %     P(L)
\newcommand*{\deriveSet}        {\formulaSetSet^2}      %     P(L)^2
\newcommand*{\deriveSetSet}     {\Pot(\deriveSet)}      %   P(P(L)^2)
\newcommand*{\deriveSetRel}     {\Rel(\formulaSetSet)}  %   R(P(L))
\newcommand*{\conclusionrulerel}{\Rel(\deriveSetRel)}   % R(R(P(L)))

%%%\newcommand*{\formulaSetSetf}{\Potf(\formulaSet)}       %    Pe(L)
\newcommand*{\deriveSetSetf}    {\Potf(\deriveSet)}     %  Pe(P(L)^2)
\newcommand*{\deriveSetRelf}    {\Relf(\formulaSetSet)} %  Re(P(L))


\newcommand*{\prerequisite}     {\mathbf{v}}              % [V]oraussetzung
\newcommand*{\prerequisiteSet}  {\mathbf{V}}
\newcommand*{\prerequisiterel}  {\derive_\prerequisiteSet}
\newcommand*{\conclusion}       {\mathbf{f}}              % [F]olgerung
\newcommand*{\conclusionSet}    {\mathbf{F}}
\newcommand*{\conclusionrel}    {\derive_\conclusionSet}
\newcommand*{\outcome}          {O}                       % [o]utcome
\newcommand*{\outcomeSet}       {\mathcal{\outcome}}
\newcommand*{\outcomerel}       {\derive_\outcomeSet}
\newcommand*{\proofstep}        {B}                       % [B]eweisschritt
\newcommand*{\proofstepsequenz} {\mathcal{S}}             % [s]equenz
\newcommand*{\proofstepSet}     {\mathcal{\proofstep}}
\newcommand*{\transformation}   {T}                       % [T]ransformation
\newcommand*{\transformsequenz} {\mathcal{\transformation}}
\newcommand*{\conclusionrule}   {C}                       % [c]onclusion
\newcommand*{\conclusionruleSet}{\mathcal{\conclusionrule}}
\newcommand*{\substitution}     {E}                       % [E]rsetzung
\newcommand*{\substitutionSet}  {\mathcal{\substitution}}
\newcommand*{\tupelSet}         {\mathcal{T}}             % [T]upelmenge
\newcommand*{\tupelSetf}        {\tupelSet_\finite}
\newcommand*{\finite}           {\mathrm{e}}
\newcommand*{\Pot}              {\mathcal{P}}             % [P]otenzmenge
\newcommand*{\Potf}             {\Pot_\finite}
\newcommand*{\Rel}              {\mathcal{R}}             % [R]elationsmenge
\newcommand*{\Relf}             {\Rel_\finite}
\newcommand*{\axiom}            {X}                       % A[x]iom
\newcommand*{\axiomSet}         {\mathcal{\axiom}}

% neue Mengensymbole: ABC EF   JKL  OPQ  TUV X
% frei:                  D  GHI   MN   R    W YZ
% doppelt:                          O    T

% \al... = aussagenlogisch
\newcommand*{\alvar} {q}%               Name einer Variablen
% Mengen der Aussagenlogik
\newcommand*{\alVar} {\mathcal{Q}}%     Menge der Variablensymbole
\newcommand*{\alCon} {\mathcal{K}}%     Menge der [K]onstantensymbole
\newcommand*{\alUna} {\mathcal{U}}%     Menge der [u]nären Operationssymbole
\newcommand*{\alBin} {\mathcal{O}}%     Menge der binären [O]perationssymbole
\newcommand*{\alJun} {\mathcal{J}}%     Menge der Operationssymbole ([J]unktor)
\newcommand*{\alABC} {\mathcal{A}}%     [A]lphabet der aussagenlogischen Sprache
\newcommand*{\alFor} {\mathcal{L}}%     Menge der Formeln (Worte) ([l]anguage)
\newcommand*{\alForp}{\alFor^\mathrm{p}}%   ... in polnischer Notation
% Indizes für Teilmengen von \alJun, \alABC, \alFor und \alForp
\newcommand*{\iAnd} {\mathrm{and}}%
\newcommand*{\iBool}{\mathrm{bool}}%
\newcommand*{\iImp} {\mathrm{imp}}%
\newcommand*{\iNand}{\mathrm{nand}}%
\newcommand*{\iNor} {\mathrm{nor}}%
\newcommand*{\iOr}  {\mathrm{or}}%
\newcommand*{\iRep} {\mathrm{rep}}%
% Konstanten und Symbole
\newcommand*{\true}  {\mathrm{true}}
\newcommand*{\false} {\mathrm{false}}

% sonstige Kommandos für den Mathematiksatz ####################################

\mathtoolsset{showonlyrefs,showmanualtags}% Nur mit \ref referenzierte Gleichungen, aber alle manuellen Tags
