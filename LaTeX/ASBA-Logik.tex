%%############################################################################%%
%%                                                                            %%
%% Datei:  ASBA-Logik.tex                                                     %%
%% Inhalt: Kapitel "Logische Grundlagen"                                      %%
%%                                                                            %%
%% Copyright (C) 2017  Winfried Teschers                                      %%
%%                                                                            %%
%% This program is free software: you can redistribute it and/or modify       %%
%% it under the terms of the GNU Affero General Public License as published   %%
%% by the Free Software Foundation, either version 3 of the License, or       %%
%% (at your option) any later version.                                        %%
%%                                                                            %%
%% This program is distributed in the hope that it will be useful,            %%
%% but WITHOUT ANY WARRANTY; without even the implied warranty of             %%
%% MERCHANTABILITY or FITNESS FOR A PARTICULAR PURPOSE.  See the              %%
%% GNU Affero General Public License for more details.                        %%
%%                                                                            %%
%% You should have received a copy of the GNU Affero General Public License   %%
%% along with this program.  If not, see <http://www.gnu.org/licenses/>.      %%
%%                                                                            %%
%% Dr. Winfried Teschers                                                      %%
%% Anton-Günther-Straße 26c                                                   %%
%% 91083 Baiersdorf                                                           %%
%% Germany                                                                    %%
%%                                                                            %%
%% e-mail: winfried.teschers@t-online.de                                      %%
%%                                                                            %%
%%############################################################################%%

% !TeX root = ASBA.tex
% !TeX encoding = UTF-8
% !TeX spellcheck = de_DE

\chapter     {Logische Grundlagen}% #######################################
\beginchapter{Logische Grundlagen}
\label            {cha:Grundlagen}

\footnoteForNotDefinedItem

\begin{offen}%%%
	Die logischen Grundlagen werden einerseits gebraucht, um die erlaubten \Beweisschritte\vrefnotesec{sub:Beweisschritte} zu definieren, andererseits dienen sie auch zum Testen von \ASBA.
	Daher werden sie in \datcha{cha:Grundlagen} ausführlicher behandelt, als für die Erstellung von \ASBA\ erforderlich ist.
	Alle \hier\ aufgeführten \Axiome, \Saetze\ und \Beweise\ sollen dazu kodiert und die \Beweise\ dann von \ASBA\ verifiziert werden.
\end{offen}%%%

Speziell in \datcha{cha:Grundlagen} wollen wir mit möglichst exakt definierten \defGlo{\Begriffen} und den zugehörigen einheitlichen, systematischen \defGlo{\Bezeichnungen} (\textdh\ \defGlo{\Benennungen} und \Symbolen) arbeiten.
Wenn sie \likeHyperTxt{in dieser} Schriftart erscheinen, gibt es eine Definition im Symbolverzeichnis (ab \Pageref{dic:Symbolverzeichnis}) oder Glossar (ab \Pageref{dic:Glossar})%
\footnote{%
	Möglicherweise steht dort statt einer Definition auch nur eine Referenz zur Definition im laufenden Text.
},
und diese Bedeutung ist dann gemeint.
Gleichzeitig ist damit im PDF-Dokument ein Link dorthin verbunden.
An Stellen, wo eine \Benennung\footnote{%
	Für Symbole gilt kann leider nur die Farbe, nicht die Schriftart geändert werden.%
} definiert wird, wird sie \likeHyperDef{in dieser} Schriftart ausgegeben.
Wenn die \Benennung\ mit der Fußnote "`\footnotemark[0]"' versehen ist, steht die vollständige Definition nur im Glossar und nicht im laufenden Text.
Eine vertiefende Beschreibung im Glossar oder Symbolverzeichnis ist unabhängig davon immer möglich.

\begin{center}
	\begin{tabular}{|l|c|c|}
		\hline
		Die Sache an sich:  & \multicolumn{2}{c|}{\Begriff}     \\
		\hline
		Darstellung:        & \multicolumn{2}{c|}{\Bezeichnung} \\
		Darstellungsmittel: & \Benennung      & \Symbol         \\
		\hline
	\end{tabular}
\end{center}

Wenn im Text "`wir"' verwendet wird, geht es um Definitionen, die von allgemein bekannten möglicherweise abweichen.\footnote{%
	"`Wir"' und nicht "`ich"', da der Leser eingeschlossen werden soll und in Zukunft möglicherweise auch andere Autoren an diesem Dokument beteiligt sein werden.
}
Die Verwendung von "`wir"' ist allerdings möglicherweise nicht konsistent und soll nur als Hinweis dienen.

\section     {Metasprache}% ====================================================
\beginsection{Metasprache}
\label   {sec:Metasprache}

Wenn man über eine Sprache, die sogenannte \Objektsprache, spricht, braucht man eine zweite Sprache, die sogenannte \Metasprache, in der \Aussagen\ über erstere getroffen werden können.%
\footnote{%
	Die beiden Sprachen können auch übereinstimmen, \textzB\ wenn man über die natürliche Sprache spricht.
}
Wenn die \Objektsprache\ die der Mathematik ist, wählt man üblicherweise die natürliche Sprache als \Metasprache.
Leider ist diese oft ungenau, nicht immer eindeutig und abhängig vom Zusammenhang, in dem sie gesprochen wird.%
\footnote{%
	Man betrachte die beiden formal verschiedenen \Aussagen\ \statement{Studenten und Rentner zahlen die Hälfte.} und \statement{Studenten oder Rentner zahlen die Hälfte.}, die beide das gleiche meinen.
	--- Entnommen aus \cite{bib:Rautenberg} \sectionname~1.2 Bemerkung 1.
}
Um diese Probleme in den Griff zu bekommen, kann die \Metasprache\ auch formalisiert werden.
Durch diese Formalisierung erinnert sie dann schon an mathematische \Formeln.
Die \defGlo{\Sprachebenen} sollten aber sorgfältig unterschieden werden.

\subsection[Sprachebenen]{\Sprachebenen}% --------------------------------------
\label {sub:Sprachebenen}

\begin{description}
	\item[\defTxt{\Metasprache}] \Hier\ die obere \Sprachebene:
	\glsBeschreibung{Metasprache}
	Ihre \Syntax\ und \Semantik\ wird \hier\ nicht behandelt.

	\item[\defTxt{\formaleMetasprache}] \Hier\ die mittlere \Sprachebene:
	\glsBeschreibung{formaleMetasprache}
	Ihre \Syntax\ und \Semantik\ werden im Folgenden noch entwickelt.

	\item[\defTxt{\Objektsprache}] \Hier\ die untere \Sprachebene:
	\glsBeschreibung{Objektsprache}

	Die entsprechende \Syntax\ wird im Folgenden noch entwickelt.
	Die \Semantik\ kann bis zu einem gewissen Grad offen bleiben, um so auch Raum für alternative \Logiken\ zu lassen.
\end{description}

\subsection[Aussagen]{\Aussagen}% ----------------------------------------------
\label {sub:Aussagen}

Wir definieren zunächst noch einige \Begriffe.
\begin{description}
	\item[\defTxt{\Wahrheitswert}] \glsBeschreibung{Wahrheitswert}
	\item[\defTxt{\Aussage}]       \glsBeschreibungMitWiki{Aussage}
\end{description}
Beispiele für \Aussagen\ in \Metasprache\ sind
\begin{enumerate}
	\item[(a)] \label{Bsp:a} \statement{Morgen scheint die Sonne.}
	\item[(b)] \label{Bsp:b} \statement{Ich bin 1,83\,m groß.}
	\item[(c)] \label{Bsp:c} \statement{Ich habe ein rotes Auto und das kann 200\,km/h schnell fahren.}
	\item[(d)] \label{Bsp:d} \statement{Alle Iren haben rote Haare.}
\end{enumerate}
Wie (c) zeigt, kann eine \Aussage\ auch aus anderen \Aussagen\ zusammengesetzt sein.
Wir definieren daher:
\begin{description}
	\item[\defTxt{\Teilaussage}]       \glsBeschreibung{Teilaussage}
	\item[\defTxt{\echteTeilaussage}]  \glsBeschreibung{echteTeilaussage}
%%%	\item[\defTxt{\Oberaussage}]       \glsBeschreibung{Oberaussage}
%%%	\item[\defTxt{\echteOberaussage}]  \glsBeschreibung{echteOberaussage}
	\item[\defTxt{\atomareAussage}]    \glsBeschreibung{atomareAussage}
	\item[\defTxt{\zerlegbareAussage}] \glsBeschreibung{zerlegbareAussage}
\end{description}
Während (a) und (b) \atomare\ \Aussagen\ sind, ist (c) \zerlegbarA.
Für alle vier \Aussagen\ ist es sinnvoll zu fragen, ob sie gelten oder nicht;
für (a) allerdings nur im Nachhinein und für den zweiten Teil von (c) nur weil klar ist, worauf sich "`das"' bezieht.
Offensichtlich muss manchmal der Zusammenhang, in dem eine \Aussage\ formuliert wird, bekannt sein.
\textZB\ ist die Bedeutung von "`Ich"' nur dann bekannt, wenn man weiss, von wem die \Aussage\ ist.

In (b) kann man "`Ich"' durch irgend eine andere Person $X$ ersetzten\footnote{Dass dann auch "`bin"' durch "`ist"' ersetzt werden muss, ist von untergeordneter Bedeutung.} und (d) kann umformuliert werden.
Es ergeben sich dann die \Aussagen
\begin{enumerate}
	\item[(e)] \label{Bsp:e} \statement{$X$ ist 1,83\,m groß}
	\item[(f)] \label{Bsp:f} \statement{Für alle $X$ gilt: Wenn $X$ ein Ire ist, dann hat $X$ rote Haare.}
\end{enumerate}
Den \Wahrheitswert\ von (e) kann man erst dann bestimmen, wenn der \Wert\ der \Variablen\ $X$ bekannt ist, während bei (f) alle zulässigen Werte für $X$ im Prinzip schon bekannt sind.
Man sagt, dass die \Variable $X$ in (e) \freiV\ und in (f) \gebundenV\ vorkommt.
Die \freienVariablen\ einer \Aussage\ nennt man auch ihre \Parameter\ und eine \Aussage\ mit mindestens einem \Parameter\ eine \parametrisierteAussage.

Die \Parameter\ einer \Aussage\ dürfen, soweit nicht anderweitig eingeschränkt, durch jedes zulässige \DefFt{\Objekt}\footnote{\glsBeschreibung{Objekt}} ersetzt werden.
Denkbar sind \Symbole, \Formeln\ und \Aussagen\ sowie \Mengen, \Symbolfolgen\ und Zahlen; ganz allgemein jedes reale oder gedachte Ding an sich.

Wir definieren noch für \Aussagen\ \textbzw\ \Objekte\ $A$ und $B$:%
\footnote{%
	Üblicherweise werden mit den Definitionen neue, \textggf\ parametrisierte, \Begriffe\ und \Symbole\ eingeführt.
	Die Anforderungen an $A$ und $B$ sind intuitiv klar.
	Insbesondere darf $B$ nicht von $A$ abhängig sein.
	Rekursive Definitionen sind allerdings zulässig.
	Man betrachtet dann die gegebenen Definitionen mit \Parametern\ als eine \Menge\ von Definitionen, in denen für bestimmte \Parameter\ alle möglichen \Ersetzungen\ durchgeführt wurden.
	Dann muss diese \Menge\ nur noch in die richtige Reihenfolge gebracht werden können.
}
\begin{description}
	\item[\defTxt{\Eigenschaft}] \glsBeschreibung{Eigenschaft}
	%
	\item[$\defSym{\MtsDefEq}$] \defTxt{\Objektdefinition} \glsBeschreibung{Objektdefinition}%
	\footnote{%
		Nach den Definitionen von \MtsDefEquiv\ und \MtsDefEq\ sind zwei Ausdrücke $P$ und $Q$ schon dann gleich, wenn nach der Ersetzung aller Vorkommen von $A$ durch $B$ sowohl in $P$ als auch in $Q$ die resultierenden Ausdrücke $\overline{P}$ und $\overline{Q}$ gleich sind.
	}
	%
	\item[$\defSym{\MtsDefEquiv}$] \defTxt{\Aussagedefinition} \glsBeschreibung{Aussagedefinition}%
	\footnote{%
		Wenn \Aussagen\ auch \Objekte\ sind, kann \MtsDefEquiv durch \MtsDefEq ersetzt werden.
	}
\end{description}

\subsection[Bereiche]{\Bereiche}% ----------------------------------------------
\label {sub:Bereiche}

Wir definieren:
\begin{description}
	\item[\defTxt{\Bereich}%
	,     \defTxt{\Element}]                        \glsBeschreibung{Bereich}
\end{description}
und für \Aussagen\ und \Objekte\ $a$ und \Bereiche\ $A$:
\begin{align}
	a \defSymBin{\MtsIn} A \quad \MtsDefEquiv \quad
	\text{$a$ ist ein \DefFt{\Element\ aus} $A$} \label{def:MtsIn}\formulatoleft
\end{align}
\MtsIn\ ist eine \Relation.
Gemäß~\eqref{def:relback} \Pageref{def:relback} ist \defSymBin{\MtsNi} die \Umkehrrelationen\ zu \MtsIn\ (Sprechweise: \emph{\textdots\ enthält als \Element\ \textdots}.
Schließlich sind \defSymBin{\MtsInN} und \defSymBin{\MtsNiN} gemäß~\eqref{def:relnot} \Pageref{def:relnot} noch die zugehörigen \Negationen.
Diese vier \Relationen\ bezeichnen wir als \defTxt{\Elementrelationen}.

Wir definieren noch für \Bereiche\ $A$ und $B$%
\footnote{%
	In der Literatur wird \chrqt{\MtsSubset} oft in der Bedeutung von \chrqt{\MtsSubsetEq} verwendet.
	Wir verwenden \chrqt{\MtsSubset} jedoch nur, wenn wir explizit \Ungleichheit\ verlangen.
}
\begin{align}
	& A \defSymBin{\MtsSubsetEq} B & \MtsDefEquiv \quad &
	\parbox[t]{10cm}{alle \Elemente\ aus $A$ sind auch \Elemente\ aus $B$
		\newline Sprechweise: $A$ ist ein \defTxt{\Teilbereich} von $B$}
	\label{def:MtsSubeq} \\
	& A \defSymBin{\MtsEq}       B & \MtsDefEquiv \quad &
	A \MtsSubsetEq B \text{ und } B \MtsSubsetEq A
	\label{def:MtsEq}    \\
	& A \defSymBin{\MtsSubset}   B & \MtsDefEquiv \quad &
	\parbox[t]{10cm}{$A$ \MtsSubsetEq $B$ und nicht $A$ \MtsEq $B$
		\newline Sprechweise: $A$ ist \defTxt{\echterTeilbereich} von $B$}
	\label{def:MtsSub}   \formulatoleft
\end{align}

Gemäß~\eqref{def:relback} \Pageref{def:relback} sind \defSymBin{\MtsSupset} und \defSymBin{\MtsSupsetEq} die \Umkehrrelationen\ zu \MtsSubset\ und \MtsSubsetEq\ (Sprechweisen: \emph{\textdots\ ist [\defTxt{\echteOM}] \defTxt{\Obermenge} von \textdots}).
Es gelten entsprechende Gleichungen wie~\eqref{def:releq} und~\eqref{def:relbsp} \Pageref{def:releq}.
Schließlich sind \defSymBin{\MtsSubsetN}, \defSymBin{\MtsSubsetEqN}, \defSymBin{\MtsSupsetN} und \defSymBin{\MtsSupsetEqN} gemäß~\eqref{def:relnot} \Pageref{def:relnot} noch die zugehörigen \Negationen.
Diese acht \Relationen\ bezeichnen wir als \defTxt{\Bereichsrelationen}.

Wir definieren:
\begin{description}
	\item[\defTxt{\Diskursuniversum} \MtsUniversum] \glsBeschreibung{Diskursuniversum}
	\item[\defTxt{\Aussagenbereich}  \MtsAussagen]  \glsBeschreibung{Aussagenbereich}
	\item[\defTxt{\Objektbereich}    \MtsObjekte]   \glsBeschreibung{Objektbereich}
	\item[\defSym{\MtsIN}~] ist der \Bereich\ der \defGlo{\natuerlichenZahlen}  ohne           $0$
	\item[\defSym{\MtsINo}] ist der \Bereich\ der \defGlo{\natuerlichenZahlen} (einschließlich $0$)
\end{description}
Wenn wir von einer \natuerlichenZahl\ sprechen, meinen wir immer ein \Element\ aus \MtsINo.

%TODO Bis hier Korrektur gelesen

\subsection[Metaoperationen]{\Metaoperationen}% --------------------------------
\label  {sub:Metaoperationen}

\Zerlegbare\ \Aussagen\ wie (c) können zum Teil formalisiert werden.
Dies wird mit den folgenden Definitionen erreicht:%
\footnote{%
	Damit es nicht zu Verwechslungen führt, verwenden wir für die metasprachliche Negation nicht das logische Symbol \chrqt{\OjkNot}.
	Wegen \eqref{def:relback} \Pageref{def:relback} ist die Definition von \chrqt{\MtsRep} überflüssig, wird wegen der angegebenen Sprechweise aber dennoch angegeben.
}
\begin{align}
	%
	&    \defSymUna{\MtsNot}   A & \MtsDefEquiv \qquad &
	\text{$A$ \DefFt{gilt nicht}.}
	\\
	%
	& A \defSymBin{\MtsImp}   B & \MtsDefEquiv \qquad &
	\text{\DefFt{Wenn} $A$ gilt \DefFt{dann} gilt auch $B$.}
	\\
	& A \defSymBin{\MtsRep}   B & \MtsDefEquiv \qquad &
	\text{$A$ gilt \DefFt{sofern} $B$ gilt.}
	\\
	& A \defSymBin{\MtsEquiv} B & \MtsDefEquiv \qquad &
	\text{$A$ gilt \DefFt{genau dann wenn} $B$ gilt.}
	\\
	& A \defSymBin{\MtsAnd}   B & \MtsDefEquiv \qquad &
	\text{$A$ \DefFt{und}  $B$.}
	\\
	& A \defSymBin{\MtsOr}    B & \MtsDefEquiv \qquad &
	\text{$A$ \DefFt{oder} $B$.}
	\formulatoleft
\end{align}

Offensichtlich sind das alles ebenfalls \Aussagen, jetzt aber teilweise formalisiert.
(c) lässt sich dann ausdrücken als \statement{\statement{Ich habe ein rotes Auto} \MtsAnd\ \statement{das kann 200\,km/h schnell fahren.}}.
\seqqt{$A \defSymBin{\MtsRep} B$} ist nur eine andere Schreibweise für \seqqt{$B \MtsImp A$}.
-- Ein Symbol für "`nicht"' wird \hier\ nicht gebraucht.

Wir nennen \MtsAnd\ und \MtsOr\ \defTxt{\Metaoperationen} und \MtsImp, \MtsRep\ und \MtsEquiv\ \defTxt{\Metarelationen}%
\footnote{%
	Man könnte \Metaoperationen\ und \Metarelationen\ auch als \DefFt{Metajunktoren} bezeichnen. Zur Unterscheidung von \Operationen\ und \Relationen\ vergleiche aber auch die Fußnote~\ref{def:Junktor} auf \Pageref{def:Junktor}.
}.
Die damit gebildeten \Aussagen\ können natürlich auch geklammert werden, um die Reihenfolge der Auswertung eindeutig zu machen.
Für den Fall fehlender Klammern sind ihre Prioritäten \vrefintab{tab:Prioritaeten} angegeben.

Um Verwechslungen mit den \Junktoren\ zu vermeiden, verwenden wir für die metasprachlichen \Operationen\ "`und"' und "`oder"' die Symbole \chrqt{\MtsAnd} und \chrqt{\MtsOr}.
$A$ und $B$ können als Operanden von \chrqt{\MtsEquiv}, \chrqt{\MtsAnd} und \chrqt{\MtsOr} vertauscht werden, ohne das Ergebnis zu ändern.%
\footnote{%
	\textDh\ die \Operationen\ \chrqt{\MtsEquiv}, \chrqt{\MtsAnd} und \chrqt{\MtsOr} sind \emph{kommutativ}.
}
Wird in einer (Teil"~)\Aussage\ nur eine der \Operationen\ \MtsAnd\ oder \MtsOr\ verwendet, können die Klammern dort weggelassen und die Operationen in beliebiger Reihenfolge ausgewertet werden, wiederum ohne das Ergebnis zu ändern.%
\footnote{%
	\textDh\ die \Operationen\ \MtsAnd\ und \MtsOr\ sind auch \emph{assoziativ}.
	Bei den den logischen \Operationen\ \OjkAnd\ und \OjkOr\ müssen Kommutativität und Assoziativität durch \Axiome\ gefordert werden.
	Die Kommutativität von \MtsEquiv\ kann abgeleitet werden.
}
Zusammengefasst ist die Reihenfolge der \Operationen\ und der Auswertung dort beliebig.

\subsection[Mit Gleichheit verwandte Relationen]{Mit \Gleichheit\ verwandte \Relationen}
\label     {sub:Gleichheit}

\subsubsection[Vergleichbar]{\Vergleichbar}% - - - - - - - - - - - - - - - - - -
\label {subsub:Vergleichbar}

Zwei \Objekte\ $A$ und $B$ sind \defTxt{\vergleichbar}, wenn beide von derselben \Objektart\ sind, \textdh\ wenn \textzB\ jeweils beide Mengen, \Symbolfolgen, Zahlen, \textusw\ sind.
Dabei muss bei \Formeln\ zwischen der \Formel\ an sich und dem Ergebnis der \Formel\ unterschieden werden. Siehe Beispiel (a).

Intuitiv scheint klar zu sein, was damit  gemeint ist.
Wenn aber entschieden werden muss, ob \textzB\ (a) "`1+1"' gleich "`2"' oder (b) "`1+1"' gleich "`1 + 1"' ist, muss man erst entscheiden, von welcher \Objektart\ die beiden zu vergleichenden Ausdrücke sind, \textdh\ \emph{wie} verglichen wird.
Wenn sie als jeweiliges Ergebnis der beiden \Formeln, verglichen werden, dann ist (a) richtig.
Wenn sie als \Formeln, \textdh\ als \Symbolfolgen, verglichen werden, ist (a) falsch.
Wenn die Ausdrücke in (b) als \Symbolfolgen\ verglichen werden, dann ist (b) richtig.
Wenn sie als \Zeichenketten\ verglichen werden, ist (b) falsch.

Die folgende Tabelle fasst dass zusammen:

\begin{center}
	\begin{tabular}{|c|c|c|c|}
		\hline
		$        A $  &        $B$        & \Objektart\    & $A$ gleich $B$ \\
		\hline
		$       1+1$  &        $2$        & \Objekt       & richtig \\
		\seqqt{$1+1$} & \seqqt{$2$}       & \Formel       & falsch  \\
		\seqqt{$1+1$} & \seqqt{$1\;+\;1$} & \Symbolfolge & richtig \\
		\strqt{1+1}   & \strqt{1 + 1}     & \Zeichenkette & falsch  \\
		\hline
	\end{tabular}
\end{center}

\subsubsection{Vergleiche}%- - - - - - - - - - - - - - - - - - - - - - - - - - -
\label {subsub:Vergleiche}

$A$ und $B$ seien \Objekte.
Dann definieren wir:

\begin{description}
	%
	\item[$\defSym{\MtsEq}$] \defTxt{\Gleichheit} \label{def:Gleichheit}
	\seqqt{$A \MtsEq B$} heißt, dass $A$ und $B$ in den \interessierendenEigenschaften\ für \MtsEq\ übereinstimmen.%
	\footnote{%
		\textZB\ sind zwei \Junktoren\ üblicherweise dann gleich, wenn sie stets denselben \emph{\Wahrheitswert} liefern.
		Ihre \Bezeichnungen\ können dabei durchaus verschieden sein, interessieren bei der Feststellung der \Gleichheit\ aber nicht.
		\textZB\ bezeichnen \chrqt{\MtsAnd} und \chrqt{\MtsUnd} dieselbe \Operation, haben aber verschiedene Priorität. --- \vrefseetab{tab:Prioritaeten}
	}
	Sprechweisen: \standsfor{$A$ ist \emph{dasselbe} wie $B$} oder \standsfor{$A$ ist \emph{identisch} zu $B$}
	--- Inwieweit die \Begriffe\ \emph{Gleichheit} und \emph{Identität} korrelieren, wird \hier\ nicht erörtert.\citenote{bib:Identitaet}
	%
	\item[$\defSym{\MtsEqN}$] \defTxt{\Ungleichheit} \label{def:Ungleichheit}
	\seqqt{$A \MtsEqN B$} heißt, dass $A$ und $B$ in mindestens einer \interessierendenEigenschaft\ für \MtsEq\ nicht übereinstimmen.
	Sprechweisen: \standsfor{$A$ ist \emph{nicht dasselbe} wie $B$} (aber vielleicht das gleiche; siehe \MtsEquiv) oder \standsfor{$A$ ist \emph{nicht identisch} zu $B$}.
	%
%%%	\item[$\defSym{\MtsAequiv}$] \defTxt{\Aequivalenz} \label{def:Aequivalenz}
%%%	\seqqt{$A \MtsAequiv B$} heißt, dass $A$ und $B$ in den \interessierendenEigenschaften\ für \MtsAequiv\ übereinstimmen.
%%%	Sprechweisen: \standsfor{$A$ ist \emph{das gleiche} wie $B$} (aber nicht unbedingt dasselbe; siehe \MtsEq) oder \standsfor{$A$ ist \emph{so wie} $B$}.
%%%	--- Es kann auch verschiedene Äquivalenzen geben, für die dann verschiedene \Bezeichnungen\ verwendet werden.
%%%	%
%%%	\item[$\defSym{\MtsAequivN}$] \defTxt{\Kontravalenz} \label{def:Kontravalenz}
%%%	\seqqt{$A \MtsAequivN B$} heißt, dass $A$ und $B$ in mindestens einer \interessierendenEigenschaft\ für \MtsAequivN\ nicht übereinstimmen.
%%%	Sprechweisen: \standsfor{$A$ ist \emph{nicht das gleiche} wie $B$} oder \standsfor{$A$ ist \emph{nicht so wie} $B$}.
%%%	%
\end{description}

%%%\MtsEq, \MtsEqN, \MtsAequiv\ und \MtsAequivN\ bezeichnen wir als  \defTxt{\Gleichheitsrelationen}.
%%%\Gleichheit\ und \Aequivalenz\ sind \defTxt{\Aequivalenzrelationen}, \textdh\ sie sind \emph{reflexiv} ($a \sim a$), \emph{transitiv} ($((a \sim b) \MtsAnd (b \sim c)) \MtsImp (a \sim c)$) und \emph{symmetrisch} ($(a \sim b) \MtsImp (b \sim a)$)
\MtsEq und \MtsEqN\ bezeichnen wir als  \defTxt{\Gleichheitsrelationen}.
\Gleichheit\ ist eine \defTxt{\Aequivalenzrelation}, \textdh\ sie ist \emph{reflexiv} ($a \sim a$), \emph{transitiv} ($((a \sim b) \MtsAnd (b \sim c)) \MtsImp (a \sim c)$) und \emph{symmetrisch} ($(a \sim b) \MtsImp (b \sim a)$)
-- jeweils für alle zulässigen Objekte $a$, $b$ und $c$.

%%%Jede \interessierendeEigenschaft\ für \MtsAequiv\ oder eine andere \Aequivalenz\ muss auch eine für \MtsEq\ sein.
%%%Daraus folgt insbesondere, dass mit $(A \MtsEq B)$ auch $(A \MtsAequiv B)$ und mit $(A \MtsAequivN B)$ auch $(A \MtsEqN B)$ gilt.

\subsection[Bezeichnungen]{\Bezeichnungen}% ------------------------------------
\label {sub:Bezeichnungen}

\begin{description}

	% ----- Symbol -------------------------------------------------------------
	%TODO Unterschied einfach und atomar(unzerlegbar), zusammengesetzt und zerlegbar
	\item [\Symbole] umfassen neben speziellen \Symbolen\ auch Buchstaben, Ziffern und Sonderzeichen.
	\Symbole, für die es kein eigenes typographisches Zeichen gibt, können auch durch Aufeinanderfolge mehrerer typographischer Zeichen, \textiAlg\ lateinische Buchstaben, dargestellt werden.
	Wir nennen sie dann \DefFt{zusammengesetzte Symbole}, im Gegensatz zu den \DefFt{einfachen Symbolen}.
	Charakteristisch für ein Symbol ist, dass es ohne Bedeutungsverlust nicht zerlegt werden kann.
	Ein \zusammengesetztesSymbol\ ist \textiAlg\ \zerlegbar, kann aber auch als \atomar, \textdh\ \unzerlegbar, definiert werden, wie \textzB\ $\sin$ als \Symbol\ für die Sinusfunktion.
	\Symbole\ werden \chrqt{so} quotiert; \zerlegbare\ können aber auch wie \Symbolfolgen\ quotiert werden.
	--- Die Quotierung ist kein Bestandteil des \Symbols!

	Wird für bestimmte \Objekte\ ein \Symbol\ verwendet, so nennen wir dies ein \defTxt{\Objektsymbol}.
	Ist das Objekt eine Funktion, Operation, Relation \textusw, so nennen wir das Symbol ein \DefFt{Funktionssymbol}, \DefFt{Operationssymbol}, \DefFt{Relationssymbol} usw.

	% ----- Zeichenkette -------------------------------------------------------
	\item [\Zeichenketten] sind Folgen von einfachen \Symbolen, in denen im Prinzip auch Leerstellen und andere nicht druckbare Zeichen zulässig sind.%
	\footnote{%
		Da beim Ausdruck optisch nicht entschieden werden kann, ob ein Zwischenraum (white space) aus einem Tabulator oder \textevtl\ mehreren Leerzeichen besteht, verwenden wir nur einzelne Leerzeichen als Zwischenraumzeichen und vermeiden Zeilenumbrüche.
	}
	Damit Leerstellen in \Zeichenketten\ leicht bestimmt und sogar gezählt werden können,
	werden \Zeichenketten\ stets \strqt{in dieser} Schriftart und Quotierung dargestellt.
	--- Die Quotierung ist kein Bestandteil der \Zeichenkette!

	% ----- Symbolfolge -------------------------------------------------------
	\item [\Symbolfolgen] sind ähnlich wie \Zeichenketten, außer das sie als Bausteine neben einfachen auch zusammengesetzte, aber \atomare\ \Symbole\ enthalten können und Leerzeichen und andere Zwischenraumzeichen nicht zählen.
	Letztere dienen nur der optischen Trennung der \Symbole\ und der besseren Lesbarkeit.
	\Symbolfolgen\ werden stets \seqqt{in dieser} Quotierung dargestellt.
	--- Die Quotierung ist kein Bestandteil der \Symbolfolge!

	% ----- Formel -------------------------------------------------------------
	\item [\Formeln] \label{def:Formel} sind \hier\ immer nach vorgegebenen Regeln aufgebaute \Symbolfolgen%
	\footnote{%
		Es kann verschiedene Arten von \Formeln\ geben, \textzB\ \aussagenlogischeF, prädikatenlogische und solche, die ein Taschenrechner auswerten kann.
	}.
	Daher werden sie wie \Symbolfolgen\ quotiert.
	--- Die Quotierung ist kein Bestandteil der \Symbolfolge!

	Man kann eine \Formel\ auch dadurch charakterisieren, dass sie ein \Element\ aus einer vorgegebenen \Menge\ \MtsSprache\ von \Symbolfolgen\ ist.%
	\footnote{%
		Die \Formel\ wird dann auch \defTxt{\Wort} der \defTxt{\Sprache} \MtsSprache\ genannt - besonders, wenn die \Elemente\ aus \MtsSprache\ \Zeichenketten\ statt \Symbolfolgen\ sind.
		Wir bleiben der Klarheit willen bei "`\Formel"'.
	}
	Das ist dann so ziemlich die einfachste Regel.

	Wenn eine \Symbolfolge\ nicht korrekt nach den vorgegebenen Regeln aufgebaut ist \textbzw\ kein \Element\ aus der vorgegebenen \Menge\ \MtsSprache\ ist, werden wir sie \emph{nicht} als \Formel\ bezeichnen, auch nicht als "`fehlerhafte Formel"' oder ähnlich.
	Sie ist dann einfach keine \Formel.

	% ----- Objekt -------------------------------------------------------------
	\item [\Objekte] sind \textzB\ \Symbole, \Zeichenketten, \Symbolfolgen\ und \Formeln, oder auch \Aussagen, Mengen, Zahlen, \textusw\ --- ganz allgemein reale oder gedachte Dinge an sich.
	Eine \Formel, die nicht quotiert ist, steht für den Wert dieser \Formel, der dann wieder ein \Objekt\ ist.
	Entsprechend steht ein \Symbol, das nicht quotiert ist, für das dadurch bezeichnete \Objekt.
	\textZB\ bezeichnet das \Symbol\ \chrqt{\MtsIN} die \Menge\ \MtsIN der natürlichen Zahlen ohne 0.

\end{description}

\subsection{Quotierung}% -------------------------------------------------------
\label {sub:Quotierung}

Zur Verdeutlichung der soeben definierten Quotierungen ein Beispiel:\footnote{%
	Was \atomare\ und was \zerlegbare\ \Symbole\ sind, muss jeweils definiert werden, \textbzw\ ergibt sich aus dem Zusammenhang.
}

\begin{tabular}{llll}
	&        $\sin$  & \Objekt
	& die Sinusfunktion
	\\
	& \chrqt{$\sin$} & \Bezeichnung
	& für das \Objekt
	\\
	& \seqqt{$\sin$} & \Symbolfolge\ (\Formel)
	& aus dem zusammengesetzten, \atomaren\ \Symbol\ \chrqt{$\sin$}
	\\
	& \seqqt {$sin$} & \Symbolfolge\ (\Formel)
	& aus den einfachen \Symbolen\ \chrqt{$s$}, \chrqt{$i$} und \chrqt{$n$}
	\\
	& \strqt  {sin}  & \Zeichenkette
	& aus den einfachen \Symbolen\ \chrqt{\CharFt{s}}, \chrqt{\CharFt{i}} und \chrqt{\CharFt{n}}
\end{tabular}

Die \Bezeichnung\ eines \Objekts\ kann auch aus mehreren Symbolen bestehen, \textdh\ einer \Symbolfolge\ oder sogar einer ganzen \Formel; \textzB\ ist die Bezeichnung für das indizierte \Objekt\ $a_i$ gleich \seqqt{$a_i$}.

\subsection[Weitere Bezeichnungen]{Weitere \Bezeichnungen}% --------------------
\label  {sub:weitereBezeichnungen}

\begin{description}

%TODO überarbeiten, Dopplungen meiden, mit Glossar abgleichen; eindeutige Bezeichnungen
	% ----- Folge --------------------------------------------------------------
	\item[\Folge] %TODO Folge beschreiben

	% ----- Tupel --------------------------------------------------------------
	\item [\Tupel] Ein \DefFt{$n$-\Tupel} ist eine endliche Folge $\vec{a} = (a_1, \dots, a_n)$ mit folgenden Eigenschaften:
	\begin{itemize}
		\item $n$, die \DefFt{Länge}, \textdh\ die Anzahl der \DefFt{Komponenten} aus $\vec{a}$, ist eine natürliche Zahl.

		$\defSymUna{\MtsLen} \vec{a} \MtsDefEq \defSym{\MtsLen}(\vec{a}) \MtsDefEq n$
		%
		\item Die $a_i$ für $1 \le i \le n$ sind \Elemente\ meist vorgegebener \Mengen.
		%
		\item $\defSymUna{\MtsSet} \vec{a} \MtsDefEq \defSym{\MtsSet}(\vec{a}) \MtsDefEq$ die \Menge\ aller Komponenten $a_i$ aus $\vec{a}$.
	\end{itemize}
	Für $n=0$ ist $\vec{a} = ()$, das \DefFt{leere \Tupel} oder \DefFt{$0$-\Tupel}.

	Wo immer $\vec{a}$ und $a_i$ mit $i \in \MtsINo$ gemeinsam vorkommen, ist $a_i$ die $i$-te Komponente aus $\vec{a}$.

	% ----- Relation -----------------------------------------------------------
	\item [\Relation] Eine \DefFt{$n$-stellige \Relation}\citenote{bib:RelationMehrstellig} $R$ ist ein (1+$n$)-\Tupel\ $(G,A_1,\dots,A_n$) mit folgenden Eigenschaften:
	\begin{itemize}
		\item $n$, die \DefFt{relationale \Stelligkeit}, ist eine natürliche Zahl.

		$\MtsStelR R \MtsDefEq \MtsStelR(R) \MtsDefEq n$
		%
		\item Die $A_i$ für $1 \le i \le n$ sind Mengen, die \defTxt{\Traegermengen} (carrier) von $R$.

		$\MtsTraeger_i R \MtsDefEq \MtsTraeger_i(R) \MtsDefEq A_i$
		%
		\item $G$, der \defTxt{\Graph} von $R$, ist eine \Teilmenge\ des kartesischen Produkts $A_1 \MtsTimes \dots \MtsTimes A_n$.

		$\MtsGraph R \MtsDefEq \MtsGraph(R) \MtsDefEq G \quad$ (oft einfach mit $R$ bezeichnet)
		%
		\item $R(a_1,\dots,a_n) \MtsDefEquiv (a_1,\dots,a_n) \in G$
	\end{itemize}
	Für $n=0$ ist $G \MtsSubsetEq \{()\}$%
	\footnote{%
		Das kartesische Produkt enthält nur noch das $0$-\Tupel\ $()$.
	},
	\textdh\ $R()$ ist entweder \TxtTrue\ (\MtsTrue) oder \TxtFalse\ (\MtsFalse).
	\\Für $n=1$ ist $G \MtsSubsetEq A_1$, \textdh\ $R$ kann als \Teilmenge\ von $A_1$ aufgefasst werden.
	\\Für $n=2$ heißt die Relation \defTxt{\binaer} und man schreibt \seqqt{$x R y$} statt \seqqt{$R(x,y)$} \textbzw\ \seqqt{$(x,y) \in R$}.

	Ist $R=(G,M,\dots,M)$, so heißt $R$ eine $n$-stellige Relation \DefFt{auf}\alternativi{in} $M$.

	Ist $|G|$ endlich, so nennen wir auch $R$ \DefFt{endlich}.

	% ----- Umkehrrelation -----------------------------------------------------
	\item [\Umkehrrelation] Die \defTxt{\Umkehrrelation} \DefFt{von}\alternativi{für} einer \binaeren\ Relation $(G,A,B)$ ist die Relation $(G',B,A)$ mit $G' \MtsDefEq \MengeDef{(b,a)}{(a,b) \in G}$.
	Üblicherweise wird das zugehörige Relationssymbol gespiegelt.

	% ----- Funktion -----------------------------------------------------------
	\item [\Funktion] Eine \defTxt{$n$-stellige \Funktion}\citenote{bib:FunktionMengentheoretisch} ist ein (1+$n$+1)-\Tupel\ $f = (G,A_1,\dots,A_n,B)$ mit folgenden Eigenschaften:
	\begin{itemize}
		\item $n$, die \defTxt{\Stelligkeit}%
		\footnote{%
			Die Werte der Stelligkeit als Relation und als Funktion sind verschieden, \textdh\ es gilt stets: $\MtsStelR(f) = \MtsStelF(f) + 1$.
		},
		ist eine natürliche Zahl.

		$\MtsStelF f \MtsDefEq \MtsStelF(f) \MtsDefEq n$

		\item $f$ ist eine ($n$+1)-stellige Relation.

		\item Zu jedem $n$-\Tupel\ $\vec{a} = (a_1,\dots,a_n)$ mit $a_i \in A_i$ für $1 \le i \le n$ gibt es genau ein $b \in B$ mit $(a_1,\dots,a_n,b) \in G$, den \defTxt{\Funktionswert} von $\vec{a}$.

		$f\vec{a} \MtsDefEq f a_1 \dots a_n \MtsDefEq f(\vec{a}) \MtsDefEq f(a_1,\dots,a_n) \MtsDefEq b$
		\footnote{%
			$f(a_1,\dots,a_n)$ und $f(a_1,\dots,a_n,b)$ sind wohl zu unterscheiden.
			Ersteres ist ein Funktionsaufruf mit einem Funktionswert, letzteres eine Relation mit einem Wahrheitswert.
		}

		\item $A = A_1 \MtsTimes \dots \MtsTimes A_n$ ist der \defTxt{\Definitionsbereich} (domain) von $f$.

		$\MtsDb f \MtsDefEq \MtsDb(f) \MtsDefEq A_1 \MtsTimes \dots \MtsTimes A_n$

		\item $B$ ist der \defTxt{\Zielbereich} (target) von $f$

		$\MtsZb f \MtsDefEq \MtsZb(f)$
	\end{itemize}
	Für $n = 0$ ist $G = ((),b)$ für ein $b \in B$ und somit $f() = b$. $f$ kann damit auch als Konstante $b$ aufgefasst werden.%
	\footnote{%
		Bei der Schreibweise ohne Klammern steht da statt \seqqt{$f()$} nur noch \seqqt{$f$} und statt \seqqt{$f()=b$}, insgesamt also nur noch \seqqt{$f=b$}.
	}

	Man sagt: $f$ ist eine $n$-stellige \Funktion\ von $A_1 \MtsTimes \dots \MtsTimes A_n$ \DefFt{nach}\alternativi{in} $B$ (Schreibweise: $\FunktionDef{f}{A_1 \MtsTimes \dots \MtsTimes A_n}{B}$) oder, im Fall $n=1$, $f$ ist eine Funktion von $A$ nach $B$ (Schreibweise: \FunktionDef{f}{A}{B}).
	Mit $A \MtsDefEq A_1 \MtsTimes \dots \MtsTimes A_n$ kann für $n > 0$ jede Funktion als $1$-stellig aufgefasst werden.

	% ----- Operation ----------------------------------------------------------
	\item [\Operationen] in oder auf einer \Menge\ $M$ sind $n$-stellige Funktionen $\MtsMn \MtsFktArrow M$.
	Für eine \defTxt{\binaere}, \textdh\ 2-stellige \Operation\ \BspOpB\ schreibt man \textiAlg\ \seqqt{$x \BspOpB y$} statt \seqqt{$\BspOpB(x,y)$}.
	Wenn nicht anders angegeben, sind \Operationen\ stets \binaer.
	0-stellige \Operationen\ können wieder als Konstante aufgefasst werden.

	Um Missverständnisse zu vermeiden, werden wir die \Bezeichnung\ "`Operator"' nicht verwenden.

	% ----- Junktor ------------------------------------------------------------
	\item [\Junktoren] sind \aussagenlogischeRelationen\ und \aOperationen.%
	\footnote{\label{def:Junktor}%
		Ein $n$-stelliger \Junktor\ $J$ sei eine \Operation\ und somit eine \Funktion.
		Wegen $M = \{\MtsTrue,\MtsFalse\}$ kann er auch als eine $n$-stellige \Relation\ $J'$ aufgefasst werden:
		$J' \MtsDefEq \MengeDef{\vec{a} \in \MtsMn}{J(\vec{a}) = \MtsTrue}$.

		~~Umgekehrt kann eine $n$-stellige \aussagenlogischeRelation\ $J'$ mittels:
		$J''(\vec{a}) \MtsDefEq \MtsTrue \text{ für } \vec{a} \in J', \MtsFalse \text{ sonst}$, für $\vec{a} \in \MtsMn$, auch als $n$-stellige Operation aufgefasst werden.

		~~Falls $J(\vec{a})=\MtsTrue$ ist $\vec{a} \in J'$ und somit $J''(\vec{a})=\MtsTrue$.
		Für $J(\vec{a})=\MtsFalse$ ist $\vec{a} \notin J'$ und somit $J''(\vec{a})=\MtsFalse$.
		Also ist $J=J''$ und so können die $n$-stelligen \aussagenlogischenRelationen\ und \Operationen\ einander eineindeutig zugeordnet werden.

		~~Daher sind in der Aussagenlogik \Relationen\ und \Operationen\ nicht von vornherein unterscheidbar.
		Wegen der Verabredungen in \vrefsub{sub:Beispielsymbole} muss für die verwendeten \Junktoren\ daher jeweils wohl definiert sein, ob sie als \Relation\ und \Operation\ zu verstehen sind.
	}
\end{description}

\subsection[Relationen und Operationen]{\Relationen\ und \Operationen}% --------
\label{sub:Beispielsymbole}

Als Beispielsymbol für \unaere\ \Operationen\ wird \chrqt{\defSym{\BspOpU}} und für \binaere\ \Operationen\ \chrqt{\defSym{\BspOpB}} verwendet.
Beispielsymbole für \binaere\ Relationen sind \chrqt{\defSym{\BspRel}} und \chrqt{\defSym{\BspRelEq}}, für ihre \Umkehrrelationen\ \chrqt{\defSym{\BspRelBck}} \textbzw\ \chrqt{\defSym{\BspRelBckEq}} sowie für ihre \DefFt{Negationen} \chrqt{\defSym{\BspRelN}} \textbzw\ \chrqt{\defSym{\BspRelEqN}}.%
\footnote{%
	Die Relationen brauchen keine Ordnungsrelationen sein, auch wenn die angegebenen Symbole dies nahe legen.
	Wenn eine der Relationen \BspRel, \BspRelEq, \BspRelBck\ oder \BspRelBckEq\ definiert ist,
	sind wegen \eqref{def:relback}, \eqref{def:releq} und \eqref{def:relbsp} auch die anderen drei Relationen definiert sowie wegen \eqref{def:relnot} auch \BspRelN, \BspRelEqN, \defSym{\BspRelBckN} und \defSym{\BspRelBckEqN}.
	Der senkrechte Strich bei den Negationen kann auch schräg sein, wie \textzB\ bei \MtsEqN.
}
Wenn nichts anderes gesagt wird, gelte mit diesen Symbolen bei gegebenem \chrqt{\BspRel}%
\footnote{%
	entsprechend mit \chrqt{\BspRelBck}, \chrqt{\BspRelEq}, \chrqt{\BspRelBckEq} und anderen, nicht horizontal symmetrischen \Symbolen.
} stets:
\begin{align}
	& (A \defSymBin{\BspRelBck} B) & \MtsDefEquiv \quad &  (B \BspRel A)
	& \quad \text{, die \defTxt{\Umkehrrelation} von } \BspRel
	\label{def:relback} \\
	& (A \defSymBin{\BspRelN}    B) & \MtsDefEquiv \quad & \MtsNot (A \BspRel B)
	& \quad \text{, die \defTxt{\Negation}       von } \BspRel
	\label{def:relnot}  \formulatoleft
\end{align}
Dabei ist \chrqt{\BspRelBck} ist die waagerechte Spiegelung von \chrqt{\BspRel} und statt des senkrechten kann auch ein schräger Strich genommen werden.

Sei nun \BspRel\ gegeben und  \BspRelBckN\ die \Umkehrrelation\ der \Negation\ von \BspRel.
Dann gilt wegen \vref{def:relback} und \vref{def:relnot}
\[(A \BspRelBckN B) \MtsEquiv (B \BspRelN A) \MtsEquiv \MtsNot (B \BspRel A)\]
Sei nun umgekehrt \BspRelBckN\ die \Negation\ der \Umkehrrelation\ von \BspRel.
Dann gilt wegen \vref{def:relnot} und \vref{def:relback}
\[(A \BspRelBckN B) \MtsEquiv \MtsNot (A \BspRelBck B) \MtsEquiv \MtsNot (B \BspRel A)\]
Also stimmt die \Umkehrrelation\ der \Negation\ mit der \Negation\ der \Umkehrrelation\ überein und wir brauchen keine verschiedenen Symbole dafür.

Je nachdem ob \BspRel\ oder \BspRelEq\ gegeben ist%
\footnote{%
	entsprechend mit \BspRelBck\ oder \BspRelBckEq oder anderen nicht horizontal symmetrischen Paaren von \Symbolen.
}
gelte ferner:
\begin{align}
	& (A \defSymBin{\BspRelEq}   B) & \MtsDefEquiv \quad & ((A \BspRel   B) \MtsOr  (A \MtsEq B))
	\label{def:releq} \\
	& (A \defSymBin{\BspRel}     B) & \MtsDefEquiv \quad & ((A \BspRelEq B) \MtsAnd (A \MtsEqN B))
	\label{def:relbsp}   \formulatoleft\formulatoleft\formulatoleft
\end{align}

Man beachte, dass, wenn man \chrqt{\MtsDefEquiv} durch \chrqt{\MtsEquiv} ersetzt, weder \eqref{def:releq} aus \eqref{def:relbsp} folgt noch umgekehrt.
\eqref{def:releq} und \eqref{def:relbsp} folgen aber dann auseinander, wenn aus \chrqt{\BspRel} die Ungleichheit \textbzw\ aus der Gleichheit \chrqt{\BspRelEq} folgt.
Beispiele dazu sind \vrefintab{tab:Gegenbeispiel} angegeben.
%
\begin{table}[H]
	\centering
	\setlength\extrarowheight{1.5pt}
	\begin{tabularx}{9.7cm}{|@{~\extracolsep{\fill}}c|cccc|l|}
		\hline
		~          & $A,\;       A$ & $A,\;       B$ & $B,\;A$& $B,\;       B$ &
		\\
		\hline
		~\MtsEq    & $A=         A$ &                &        & $B=         B$ &
		\\
		\hline
		~\BspRel   &                & $A\BspRel   B$ &        &                &
		\text{Es gilt \eqref{def:releq}}
		\\
		~\BspRelEq & $A\BspRelEq A$ & $A\BspRelEq B$ &        & $B\BspRelEq B$ &
		\text{und \eqref{def:relbsp}}
		\\
		\hline
		~\BspRel   &                & $A\BspRel   B$ &        & $B\BspRel   B$ &
		\text{Es gilt \eqref{def:releq}}
		\\
		~\BspRelEq & $A\BspRelEq A$ & $A\BspRelEq B$ &        & $B\BspRelEq B$ &
		\text{aber nicht \eqref{def:relbsp}}
		\\
		\hline
		~\BspRel   &                & $A\BspRel   B$ &        &                &
		\text{Es gilt \eqref{def:relbsp}}
		\\
		~\BspRelEq & $A\BspRelEq A$ & $A\BspRelEq B$ &        &                &
		\text{aber nicht \eqref{def:releq}}
		\\
		\hline
	\end{tabularx}
	\caption{Beispiele für \BspRel\ und \BspRelEq}
	\label{tab:Gegenbeispiel}% Erst nach '\caption'!
\end{table}
%
Seien $\RawBspOpRel_1$ und $\RawBspOpRel_2$ \binaere\ \Operationen\ oder \Relationen\ (auch gemischt) und mindestens eins von beiden eine \Relation.
Dann treffen wir folgende Vereinbarung:%
\footnote{%
	wird auch in der Literatur verwendet, \textzB\ \textzB~\cite{bib:Rautenberg}, Notationen Seite~xxi
}
\[ \label{def:OpRel}
	A \RawBspOpRel_1  B \RawBspOpRel_2 C \text{ steht für }
	A \RawBspOpRel_1  B \quad \MtsAnd \quad B \RawBspOpRel_2 C
\]
Ist diese Interpretation nicht gewünscht, so müssen Klammern verwendet werden.

Für den Fall fehlender Klammern sind die Prioritäten \vrefintab{tab:Prioritaeten} angegeben.
Damit wären dann alle Klammern in \datsub{sub:Beispielsymbole} überflüssig.

\subsection{Prioritäten}% ------------------------------------------------------
\label {sub:Prioritaeten}

\vrefDtab{tab:Prioritaeten} listet zur Vermeidung von Klammern die Prioritäten der \hier\ verwendeten \Operationen, \Relationen, \Junktoren\ und \Metadefinitionen\ in absteigender Folge von höherer zu niedrigerer Priorität, \textdh\ von starker zu schwacher Bindung auf.%
\footnote{Priorität 1 ist höher und bindet damit stärker als Priorität 2, usw.}
Das Weglassen redundanter Klammern wird in \datcha{cha:Grundlagen} nicht weiter thematisiert.%
\footnote{%
	Gesetzt den Fall, dass \ASBA\ die \Praemissen\ und \Konklusionen\ eines mathematischen \Satzes\ richtig und die \Beweisschritte, \textzB\ durch fehlerhafte Interpretation einer \Formel, falsch einliest, ansonsten aber richtig arbeitet.
	Dann kann man folgende Fälle unterscheiden:\\
	--- Ein falscher \Satz\ kann dadurch nicht als richtig bewertet werden.\\
	--- Ein richtiger \Satz\ wird wahrscheinlich auch bei eigentlich richtigem \Beweis\ als nicht bewiesen gelten, was natürlich unbefriedigend ist.\\
	--- In äußerst unwahrscheinlichen Fällen kann dabei auch ein eigentlich falscher \Beweis\ in einen richtigen verwandelt werden, was zwar schön ist, aber leider steht in der Dokumentation dann ein falscher \Beweis.\\
	In keinem Fall wird durch diesen Fehler die \Menge\ der richtigen \Saetze\ durch einen falschen \Satz\ "`verunreinigt"'.
}
Zur besseren Verständlichkeit werden aber gelegentlich auch redundante Klammern verwendet, insbesondere wenn Prioritäten unklar oder in der Literatur auch anders definiert sind.
Die Prioritäten der \Junktoren\ wurden aus~\cite{bib:Rautenberg} Kapitel~1.1 Seite~5 entnommen und ergänzt und die der \Metaoperationen\ daran angeglichen.

\begin{table}[p]
	\centering
	\begin{threeparttable}
		\setlength\extrarowheight{3pt}
		\begin{tabularx}{12.5cm}{|@{~~}l|@{\extracolsep{\fill}}l|}
			\hline
			Klammern & $(\quad)$ \quad $\quad$ \chrqt{$\quad$} \quad \seqqt{$\quad$} \quad \strqt{$\quad$} \\
			\hline\hline
			\multicolumn{2}{|c|}{\Operationen\ haben unterschiedliche Priorität.} \\
			\hline
			Unäre \Operationen\ \Tnote{1} \Tnote{2} & $\BspOpU \quad \OjkNot \quad \MtsNot$ \\
			\hline
			Binäre \Bereichsoperationen &
			\begin{tabular}{@{\extracolsep{\fill}}l}
				$ \MtsTimes $ \\
				\hline
				$ \MtsCup $   \\
				\hline
				$ \MtsCap $   \\
			\end{tabular}  \\
			\hline
			Binäre \Operationen\ \Tnote{1} & $ \BspOpB $ \\
			\hline
			Binäre \Junktoren\ \Tnote{2} &
			\begin{tabular}{@{\extracolsep{\fill}}l}
				$ \OjkAnd \quad \OjkNand               $ \\
				\hline
				$ \OjkOr  \quad \OjkXor \quad \OjkNor  $ \\
				\hline
				$ \OjkRep \quad \OjkImp                $ \\
				\hline
				$ \OjkEquiv                            $ \\
			\end{tabular}                                \\
			\hline\hline
			\multicolumn{2}{|c|}{Binäre Relationen haben gleiche Priorität.} \\
			\hline
			Binäre \Elementrelationen \Tnote{3}
			& $ \MtsIn \quad \MtsInN \quad \MtsNi \quad \MtsNiN $ \\
			\hdashline
			Binäre \Bereichsrelationen \Tnote{3}
			& $ \MtsSubset \quad \MtsSubsetN \quad \MtsSubsetEq \quad \MtsSubsetEqN \quad \MtsSupset \quad \MtsSupsetN \quad \MtsSupsetEq \quad \MtsSupsetEqN $ \\
			\hdashline
			Binäre \Relationen\ \Tnote{1}
			& $ \BspRel \quad \BspRelN \quad \BspRelEq \quad \BspRelEqN \quad \BspRelBck \quad \BspRelBckN \quad \BspRelBckEq \quad \BspRelBckEqN $ \\
			\hdashline
			\Gleichheitsrelation\ \Tnote{4}
%%%			& $ \MtsEq \quad \MtsEqN \quad \MtsAequiv \quad \MtsAequivN $ \\
			& $ \MtsEq \quad \MtsEqN $ \\
			\hdashline
			\Ableitungsrelation\  \Tnote{5}
			& $ \MtsDerive $ \\
			\hdashline
			\Ersetzung\ \Tnote{5}
			& $ \MtsSwap \quad \MtsSubst $  \\
			\hline\hline
			\multicolumn{2}{|c|}{Sonstige \binaere\ Verknüpfungen haben unterschiedliche Priorität.} \\
			\hline
			\Objektdefinition\ \Tnote{6} & $ \MtsDefEq $ \\
			\hline
			Binäre \Metaoperationen\ \Tnote{7} \Tnote{8} &
			\begin{tabular}{@{\extracolsep{\fill}}l}
				$ \MtsAnd$ \\
				\hline
				$ \MtsOr $ \\
				\hline
				$ \MtsUnd  $ \\
				\hline
				$ \MtsRep \quad \MtsEquiv \quad \MtsImp $
			\end{tabular}     \\
			\hline
			\Aussagedefinition\ \Tnote{6} & $ \MtsDefEquiv $ \\
			\hline\hline
			\multicolumn{2}{|c|}{Natürliche Sprache} \\
			\hline
			\parbox[][1.1cm][c]{6.3cm}{%
				Innerhalb natürlicher Sprache deren Strukturelemente, \textzB\ Satzzeichen \Tnote{9}%
			}
			& . \quad , \quad ; \quad usw. \\
			\hline
		\end{tabularx}
		\begin{tablenotes}
			\footnotesize
			\item[1] \vrefseesub{sub:Beispielsymbole}
			\item[2] \vrefseetab{tab:Symbole}
			\item[3] \vrefseesub{sub:Bezeichnungen}
			\item[4] \vrefseesubsub{subsub:Vergleiche}
			\item[5] \vrefseesub{sub:Basisregeln}
			\item[6] \vrefseesubsub{bereich}
			\item[7] \vrefseesub{sub:Metaoperationen}
			\item[8] \chrqt{\MtsUnd} wird nur bei den \Schlussregeln\ (\vrefseesub{sub:Schlussregeln}) verwendet.
			\chrqt{\MtsAnd} und \chrqt{\MtsUnd} bezeichnen die gleiche \Operation, haben aber unterschiedliche Priorität.
			\item[9] Innerhalb von \Formeln\ können Satzzeichen eine andere Bedeutung und Priorität haben.
		\end{tablenotes}
	\end{threeparttable}
	\caption{Prioritäten in abnehmender Reihenfolge}
	\label{tab:Prioritaeten}% Erst nach '\caption'!
\end{table}

Für \Operationen\ derselben Priorität wählen wir \hier\ Rechtsklammerung%
\footnote{%
	Die Symbole \unaerer\ \Operationen\ stehen \hier\ stets links \emph{vor} dem Operanden, so dass es für sie nur Rechtsklammerung geben kann.
	Zur Rechtsklammerung bei \binaeren\ Operationen ein Zitat aus~\cite{bib:Rautenberg} Kapitel~1.1 Seite~5:
	"`Diese hat gegenüber Linksklammerung Vorteile bei der Niederschrift von Tautologien in \OjkImp, [...]"'.
	Die meisten Autoren bevorzugen Linksklammerung, was natürlicher erscheint.
	Dann sollte man aber für die Potenz doch noch Rechtsklammerung wählen, sonst ist \seqqt{$ a^{x^y} = (a^x)^y = a^{(x*y)} $} und nicht wie wahrscheinlich erwünscht \seqqt{$a^{(x^y)}$}.
}.

\section[Beweise in ASBA]{\Beweise\ in \ASBA}% ================================0
\beginsection            {\Beweise\ in \ASBA}
\label                {sec:BeweiseASBA}

Die Regeln zur Formulierung und Prüfung der \Beweise\ müssen in \ASBA\ fest codiert werden.
Sie sind quasi die \Axiome\ von \ASBA\ und sollten daher möglichst wenig voraussetzen.
In \ASBA\ wird dazu ein \emph{Genzen-Kalkül}%
\footnote{%
	\citesee{bib:Rautenberg} Kapitel~1.4 und~\cite{bib:Schlussregel,bib:NatuerlichesSchliessen}
} verwendet.
Die Definition von \emph{\Schlussregel} und \emph{\Beweis} ist \hier\ \ASBA-spezifisch, um später eine leichtere Programmierung zu erreichen.
Insbesondere müssen alle abzuspeichernden Mengen endlich sein.
Dies berücksichtigen wir in den Beispielen, fordern zunächst aber nicht notwendig Beschränktheit.
Zuerst brauchen wir aber noch ein paar Definitionen.

\subsection{Definitionen und Verabredungen}% -----------------------------------
\label                  {sub:Verabredungen}

Zu \chrqt{\MtsLen} und \chrqt{\MtsSet} Vergleiche die Definition von \emph{$n$-\Tupel} \vrefinsub{sub:weitereBezeichnungen}.

\begin{align}
	& |M|                          & \MtsDefEq \quad & \text{Kardinalität von } M
	&&\text{, die \DefFt{Anzahl der \Elemente} aus } M
	\label{def:Anzahl}
	\\
	& \defSym{\MtsMn}     & \MtsDefEq \quad & M \MtsTimes \dots \MtsTimes M \quad \text{ , für } n \in \MtsINo
	&&\text{, das \DefFt{kartesische Produkt} aus $n$ Mengen } M
	\label{def:kartesischesProdukt}
	\\
	& \MtsMo                  &    \MtsEq \quad & \{()\}
	&&\text{, wobei $()$ das \DefFt{0-\Tupel} ist}
	\label{def:Mo}
	\\
	& \defSym{\MtsTup}(M) & \MtsDefEq \quad & \MengeDef{\vec{a} \in \MtsMn}{n \in \MtsINo}
	&&\text{, die \Menge\ der \defTxt{\Tupel} \DefFt{über} $M$ (\defTxt{\Tupelmenge})}
	\label{def:Tupelmenge}
	\\
	& \links{(A,B)}                & \MtsDefEq \quad & A
	&& \text{, die \DefFt{linke Seite} eines geordneten Paares.}
	\label{def:links}
	\\
	& \rechts{(A,B)}               & \MtsDefEq \quad & B
	&& \text{, die \DefFt{rechte Seite} eines geordneten Paares.}
	\label{def:rechts}
	\\
	& \defSym{\MtsPot}(M)      & \MtsDefEq \quad & \MengeDef{A}{A \MtsSubsetEq M}
	&&\text{, die \defTxt{\Potenzmenge} der \Menge\ } M
	\label{def:Potenzmenge}
	\\
	& \defSym{\MtsPotf}(M)     & \MtsDefEq \quad & \MengeDef{A \MtsSubsetEq M}{|A| \in \MtsINo}
	&& \text{, die \DefFt{endlichen \Teilmengen} von } M
	\label{def:endlichePotenzmenge}
	\\
	& \defSym{\MtsRel}(M)      & \MtsDefEq \quad & \MengeDef{R}{R \MtsSubsetEq M \MtsTimes M}
	&& \text{, die \Menge\ der \DefFt{\binaeren\ \Relationen\ in} } M
	\label{def:Relationsmenge}
	\\
	& \defSym{\MtsRelf}(M)     & \MtsDefEq \quad & \MengeDef{R \MtsSubsetEq M \MtsTimes M}{|R| \in \MtsINo}
	&& \text{, die \DefFt{endlichen \binaeren\ \Relationen\ in} } M
	\label{def:endlicheRelationsmenge}
	\\
	& \defSym{\MtsDeriveR}     & \MtsDefEq \quad & R
	&& \text{, für Relationen } R \in \MtsRelAllDerive
	\label{def:Ableitung}
\end{align}
Offensichtlich gilt für Mengen $M$ und $N$:
\begin{align}
	& \MtsPotf(M) \MtsSubsetEq \MtsPot          (M)
	& ,          \qquad
	& \MtsRelf(M) \MtsSubsetEq \MtsRel          (M)
	\label{eq:Setf} \\
	& \MtsRel (M) =            \MtsPot (M \MtsTimes M)=\MtsPot (M^2)
	& ,          \qquad
	& \MtsRelf(M) =            \MtsPotf(M \MtsTimes M)=\MtsPotf(M^2)
	\label{eq:relPot} \\
	& \MtsPot (M) \MtsSubset   \MtsPot          (N)
	& \MtsEquiv \qquad
	& \MtsPotf(M) \MtsSubset   \MtsPotf         (N)
	& \MtsEquiv \qquad
	&               M  \MtsSubset                          N
	\label{eq:potPot} \\
	& \MtsRel (M) \MtsSubset   \MtsRel          (N)
	& \MtsEquiv \qquad
	& \MtsRelf(M) \MtsSubset   \MtsRelf         (N)
	& \MtsEquiv \qquad
	&               M  \MtsSubset                          N
	\label{eq:relRel} \\
	&                                 \vec{a}  \in \MtsTup(M^2)
	& \MtsEquiv \qquad  & \MtsSet(\vec{a}) \in \MtsRelf    (M)
	\label{eq:vecrel}
\end{align}

\subsection[Formeln und Ableitungen]{\Formeln\ und \Ableitungen}% --------------
\label             {sub:Ableitungen}

Im Folgenden sei \MtsSprache\ stets eine gegebene \Menge\ von \Formeln, \textzB\ alle korrekten \Formeln\ der \Aussagenlogik\ oder der \Praedikatenlogik.
Für die folgenden Betrachtungen ist aber nur nötig, dass die \Elemente\ aus \MtsSprache\ \Symbolfolgen\ sind.
Die \Teilmengen\ von \MtsSprache\ nennen wir \defTxt{\Formelmengen}.
Es sind genau die \Elemente\ aus \MtsPotSprache.

Bei einem \Beweis\ werden aus einer \Formelmenge\ $\Gamma$ von \Axiomen\ und schon bewiesenen \Formeln\ mittels zulässiger
\footnote{%
	Was \emph{zulässig} heißt, muss im entsprechenden Kontext jeweils definiert sein.
	Üblicherweise sind das bestimmte Ableitungsregeln und Ersetzungen.
}
\Ableitungen\ die \Formeln\ einer \Formelmenge\ $\Delta$ abgeleitet; Schreibweise: \seqqt{$\Gamma \MtsDerive \Delta$}.

Für \Teilmengen\ $\Gamma$ und $\Delta$ von \MtsSprache\ sei also:
\begin{itemize}
	\item $\Gamma \defSymBin{\MtsDerive} \Delta \MtsDefEquiv$ $\Gamma$ \defTxt{\ableitbar} $\Delta$; oder auch $\Gamma$ \defTxt{\beweisbar} $\Delta$.
	%
	\item $\Gamma \defSymBin{\MtsDerive} \Delta$ nennen wir auch eine \defTxt{\Ableitung} \DefFt{in} \MtsSprache.
	Damit ist $(\Gamma,\Delta)$ ein \Element\ aus einer \binaeren\ Relation \MtsDerive\ in \MtsPotSprache, einer sogenannten \defTxt{\Ableitungsrelation}.
	%
	\item Wenn wir von einer Ableitung $\drvft{a}$ sprechen, meinen wir immer ein \Element\ aus einer \Ableitungsrelation, \textdh\ ein geordnetes Paar, \textzB\ $(\Gamma, \Delta) \in \MtsPotSprache \MtsTimes \MtsPotSprache$, dargestellt als $\Gamma \MtsDerive \Delta$.
	%
	\item Um möglicherweise verschiedene \Ableitungsrelationen\ unterscheiden zu können, indizieren wir \chrqt{$\defSym{\MtsDerive}$} \textggf\ mit der zugrundeliegenden \Relation\ R, \textdh\ wir schreiben \chrqt{$\defSym{\MtsDeriveR}$} und sprechen dann von \defTxt{$R$-\ableitbar}, \DefFt{$R$-\beweisbar} und \DefFt{$R$-\Ableitung}.
\end{itemize}
%
Zur Vereinfachung der Darstellung und besseren Lesbarkeit treffen wir noch folgende Vereinbarungen für die beiden Seiten von \seqqt{$\Gamma \MtsDerive \Delta$} (natürlich nur, wenn dies nicht zu Verwechslungen führt):
\begin{itemize}
	\item Eine Aufzählung von \Formelmengen\ und einzelnen \Formeln\ steht für die Vereinigung der \Formelmengen\ mit der \Menge\ der einzeln angegebenen \Formeln.
	\textZB\ steht \seqqt{$\Gamma, \alpha \MtsDerive \beta$} für \seqqt{$(\Gamma \MtsCup \{\alpha\}) \MtsDerive \{\beta\}$}.
	%
	\item Diese Aufzählungen können auch leer sein und stehen dann für die \leereMenge.
	\\\textZB\ steht \seqqt{$\MtsDerive\; \alpha \OjkImp (\beta \OjkImp \alpha)$} für \seqqt{$\MtsEmptyset \MtsDerive \{\alpha \OjkImp (\beta \OjkImp \alpha)\}$}.
	%
	\item Ist die Aufzählung links vom Relationssymbol \chrqt{\MtsDerive} leer, kann auch das Relationssymbol wegfallen.
	Im letzten Beispiel also einfach \seqqt{$\{\alpha \OjkImp (\beta \OjkImp \alpha)\}$}.
	Das entspricht dann einem \defTxt{\Axiom}.
\end{itemize}
%
Im Folgenden halten wir uns bei der Verwendung von Buchstaben so weit wie möglich an folgende Vereinbarungen:%
\footnote{Die letzte Gleichung ergibt sich aus \vreffor{eq:relPot}.}
\begin{align}
	&  \text{griechisch, klein:}       && \alpha, \beta, \gamma, \dots
	&& \text{\Formel}                  && \in \qquad \; \; \MtsSprache
	\\
	&  \text{griechisch, groß:}        && \Gamma, \Delta, \Theta, \dots
	&& \text{\Formelmenge}             && \in \quad \; \MtsPotSprache
	\\
	&  \text{lateinisch, fett, klein:} && \drvft{a}, \drvft{b}, \drvft{c}, \dots
	&& \text{\Ableitung}               && \in \quad \; \MtsAllDerive
	\\
	&  \text{lateinisch, fett, groß:}  && \Drvft{A}, \Drvft{B}, \Drvft{C}, \dots
	&& \text{\Ableitungsrelation}      && \in \MtsPotAllDerive = \MtsRelAllDerive
\end{align}
Damit definieren wir folgende Aussagen:
\begin{align}
	\frac{\; \Drvft{A}  \;}{\; \Drvft{B} \;}
	& \quad \MtsDefEquiv \quad
	\text{ Mit den \Ableitungen\ aus $\Drvft{A}$ lassen sich die aus $\Drvft{B}$ ableiten.}
	\label{def:AB}
	\\
	\frac{\; \vec{\drvft{a}} \;}{\; \vec{\drvft{b}} \;} \qquad
	& \quad \MtsDefEquiv \quad
	\text{ Mit den Komponenten aus $\vec{\drvft{a}}$ lassen sich die aus $\vec{\drvft{b}}$ ableiten.}
	\label{def:ab}
	\\
	\frac{\drvft{a}_1 \MtsUnd \dots \MtsUnd \drvft{a}_n}{\drvft{b}_1 \MtsUnd \dots \MtsUnd \drvft{b}_m}
	& \quad \MtsDefEquiv \quad
	\text{ Mit den \Ableitungen\ $\drvft{a}_i$ lassen sich die $\drvft{b}_j$ ableiten.}
	\label{def:aabb}
\end{align}
wobei in der letzten Definition $1 \le i \le n$ und $1 \le j \le m$ sei und die $\drvft{a}_i$ und die $\drvft{b}_j$ dabei jeweils beliebig permutiert werden können.
\chrqt{\defSym{\MtsUnd}} und Bruchstrich stehen für die \Metaoperationen\ \chrqt{\MtsAnd} und \chrqt{\MtsImp}.%
\footnote{%
	Der Bruchstrich hat die übliche Priorität, \MtsUnd\ die schwächste.
	Man beachte, dass Zähler und Nenner auch leer sein können, \textdh\ $n$ und $m$ gleich $0$ sein dürfen.
	In der Praxis liegen sie bei kleinen Werten, typischerweise 0, 1 oder 2.
}
Wir nennen alle drei Formen \defTxt{\Schlussregeln}%
\footnote{%
	Genau genommen nur um die \Darstellung\ einer Schlussregel.
	Die Exakte Definition erfolgt \vrefinsub{sub:Schlussregeln}.
}.
Die \Elemente\ aus $A$ \textbzw\ die Komponenten $a_i$ nennen wir die \defTxt{\Praemissen} und die \Elemente\ aus $B$ \textbzw\ die Komponenten $b_j$ die \defTxt{\Konklusionen}\synonym{\defTxt{\Folgerungen}} der \Schlussregel.
Offensichtlich gilt:
\begin{align}
	& \frac{a_1 \MtsUnd \dots \MtsUnd a_n}{b_1 \MtsUnd \dots \MtsUnd b_m} \; \MtsEquiv \; \frac{\; \vec{a} \;}{\; \vec{b} \;} \; \MtsEquiv \; \frac{\MtsSet(\vec{a})}{\MtsSet(\vec{b})} \label{eq:AB}
\end{align}
Wir nennen eine \Schlussregel\ auch einen \defTxt{\formalenSatz} und nennen sie \defTxt{\beschraenkt}, wenn sie nur endlich viele \Praemissen\ und \Konklusionen\ hat.
Die \Schlussregeln\ nach \eqref{def:ab} und \eqref{def:aabb} sind per se beschränkt.
Die nach \eqref{def:AB} genau dann, wenn $\Drvft{A}$ und $\Drvft{B}$ endliche Mengen sind, \textdh\ wenn sie \Elemente\ aus ...%TODO Text fehlt

Die Mengen der \Praemissen\ und \Konklusionen\ dürfen auch leer sein.
Dies führt zu den folgenden Spezialfällen:
\begin{itemize}
	\item[] Eine \Schlussregel\ $\frac{A}{\MtsEmptyset}$ ohne \Konklusionen\ ist immer gültig.
	%
	\item[] Ein \Menge\ $B$ von Ableitungen, die als \Axiome\ dienen sollen, kann als \Schlussregel\ $\frac{\MtsEmptyset}{B}$ ohne \Praemissen\ repräsentiert werden.
\end{itemize}

\subsection[Schlussregeln]{\Schlussregeln}% ------------------------------------
\label {sub:Schlussregeln}

Wir betrachten zuerst noch die \Menge\ der \binaeren\ Relationen\vrefnotesub{sub:weitereBezeichnungen} in \MtsPotSprache.
Sei also $R$ eine solche \binaere\ Relation und $A \in R$.
Dann gilt wegen~\eqref{def:links}, \eqref{def:rechts}, \eqref{def:Potenzmenge}, \eqref{def:Relationsmenge} und~\vreffor{def:Ableitung}:
\begin{align}
	&  A \in R \in \MtsRelAllDerive   \\
	&  A = (\links{A},\rechts{A})
	&& \text{und es gilt}
	&& \links{A}, \rechts{A} \MtsSubsetEq \MtsSprache \\
	&  \links{A} \MtsDeriveR \rechts{A}
	&& \text{oder einfach}
	&& \links{A} \MtsDerive  \rechts{A}
	&& \text{ist eine $R$-\Ableitung}                  \\
	&  \links{A} \; \text{$R$-\ableitbar} \; \rechts{A}
	&& \text{oder einfach} \qquad
	&& \links{A} \;\; \text{\ableitbar} \;\; \rechts{A}
	\formulatoleft
\end{align}

Nach diesen Vorbereitungen fassen wir noch mal zusammen:\\
Ein geordnetes Paar $(\MtsPraemisseSet, \MtsKonklusionSet) \in \MtsPotAllDerive^2 = \MtsRelAllDerive^2$ heißt eine
\defTxt{\Schlussregel} \DefFt{für} \MtsSprache, geschrieben $\frac{\MtsPraemisseSet}{\MtsKonklusionSet}$; und es gilt:
\begin{align}
	& \MtsPraemisseSet \in \MtsRelAllDerive
	&& \text{, die \defTxt{\Praemissen}}
	&& \text{, eine \Menge\ von \DefFt{\MtsPraemisseSet-\Ableitungen}.}
	\label{def:ruleRelationPraemissen}
	\\
	& \MtsKonklusionSet   \in \MtsRelAllDerive
	&& \text{, die \defTxt{\Konklusionen}}
	&& \text{, eine \Menge\ von   \DefFt{\MtsKonklusionSet-\Ableitungen}.}
	\label{def:ruleRelationKonklusionen}
	\\
	& \drvft{a} \in \MtsPraemisseSet \quad \MtsImp
	&& \drvft{a} = (\Gamma, \Delta) \; \MtsAnd \; \Gamma, \Delta \in \MtsPotSprache
	&& \text{, Schreibweise: } \Gamma \MtsDerive_{\MtsPraemisseSet} \Delta
	\\
	& \drvft{a} \in \MtsKonklusionSet \quad \MtsImp
	&& \drvft{a} = (\Gamma, \Delta) \; \MtsAnd \; \Gamma, \Delta \in \MtsPotSprache
	&& \text{, Schreibweise: } \Gamma \MtsDerive_{\MtsKonklusionSet} \Delta
	\formulatoleft
\end{align}
mit $\Gamma$ und $\Delta$ jeweils passend.

***** Fehlende Verweise: \Ableitungsmenge, \OjkEqN, \MtsTrue, \MtsDerive, \MtsDeriveR. *****

Die \Schlussregel\ entspricht der \Aussage:
\begin{itemize}
	\item[] \emph{Mit den \Praemissen\ aus \MtsPraemisseSet\ lassen sich alle \Konklusionen\ aus \MtsKonklusionSet\ ableiten}%
	\footnote{mittels noch zu definierender \emph{\zulaessigerTransformationen}}.
\end{itemize}
Die \Schlussregel\ heißt \DefFt{allgemeingueltig}, wenn aus den \Praemissen\ alle \Konklusionen\ abgleitet werden können.
In diesem Fall kann sie zur \zulaessigenTransformation\ von weiteren \Formeln\ dienen.

Die Mengen der \Praemissen\ und \Konklusionen\ sowie die beiden Seiten einer \Ableitung\ dürfen auch leer sein.
Dies führt zu den folgenden semantischen Spezialfällen:
\begin{itemize}
	\item Eine \Ableitung\ $(A,\MtsEmptyset)$ ist trivial allgemeingültig.
	Daher können solche Prämissen und Konklusionen ohne Probleme weggelassen werden.
	%
	\item Ein \Menge\ $B$ von \Formeln, die \Axiome\ sein sollen, kann durch eine \Praemisse\ $(\MtsEmptyset,B)$ repräsentiert werden.
	%
	\item Ein \Menge\ $B$ von \Formeln, die als allgemeingültig zu beweisen sind, kann durch eine \Konklusion\ $(\MtsEmptyset,B)$ repräsentiert werden.
\end{itemize}
%
Wenn eine Schlussregel $\frac{\MtsPraemisseSet}{\MtsKonklusionSet}$ beschränkt ist, sind \MtsPraemisseSet\ und \MtsKonklusionSet\ endliche Mengen und es gibt wegen~\vreffor{eq:vecrel} zwei \Tupel\ $\vec{\MtsPraemisse}, \vec{\MtsKonklusion} \in \MtsTup(\MtsAllDerive)$, so dass gilt:
\footnote{%
	Statt $\ge$ könnte in \eqref{eq:SRTb} auch \MtsEq\ genommen werden.
	Dann müssten die $\MtsPraemisse_n$ und die $\MtsKonklusion_m$ jeweils paarweise verschieden sein, was wir nicht voraussetzen wollen.
}
\begin{align}
	&     \MtsPraemisseSet    & \MtsEq \quad & \MtsSet(\vec{\MtsPraemisse})
	&,\;& \MtsKonklusionSet        & \MtsEq \quad & \MtsSet(\vec{\MtsKonklusion})
	\label{eq:SRTa}          \\
	&     N                       &    \ge \quad & |\MtsPraemisseSet|
	&,\;& M                       &    \ge \quad & |\MtsKonklusionSet|
	&,\;& \text{mit } N, M \in \MtsINo
	\label{eq:SRTb}          \\
	& \vec{\MtsPraemisse}     & \MtsEq \quad & \{\MtsPraemisse_1,\dots,\MtsPraemisse_N \}
	&,\;& \vec{\MtsKonklusion}     & \MtsEq \quad & \{\MtsKonklusion_1,\dots,\MtsKonklusion_M\}
	\label{eq:SRTc}          \\
	&       \MtsPraemisse_n   & \MtsEq \quad & ( \links{\MtsPraemisse}_n, \rechts{\MtsPraemisse}_n )
	&,\;& \MtsKonklusion_m         & \MtsEq \quad & ( \links{\MtsKonklusion}_m, \rechts{\MtsKonklusion}_m )
	&,\;& \text{für } 1 \le n \le N \text{ , } 1 \le m \le M
	\label{eq:SRTd}          \\
	& \links{\MtsPraemisse}_n & \MtsDerive_{\MtsPraemisseSet} \quad & \rechts{\MtsPraemisse}_n
	&,\;& \links{\MtsKonklusion}_m & \MtsDerive_{\MtsKonklusionSet}     \quad & \rechts{\MtsKonklusion}_m
	&,\;& \text{für } 1 \le n \le N \text{ , } 1 \le m \le M
	\label{eq:SRTe}          \formulatoleft
\end{align}
also
\begin{align}
	&  \vec{\MtsPraemisse}  & = \quad & \MengeDef{(\links{\MtsPraemisse}_n,
	\rechts{\MtsPraemisse}_n)}{1 \le n \le N}
	\label {def:Praemissen}
	\\
	&  \vec{\MtsKonklusion} & = \quad & \MengeDef{(\links{\MtsKonklusion}_m,
	\rechts{\MtsKonklusion}_m)}{1 \le m \le M }
	\label {def:Konklusionen} \formulatoleft\formulatoleft
\end{align}
und wir nennen auch das Paar $(\vec{\MtsPraemisse}, \vec{\MtsKonklusion})$ \Schlussregel.
Diese ist per se \beschraenkt\ und ein \Element\ aus $\MtsTup(\MtsAllDerive)^2$.
Nun haben wir alternative Schreibweisen für \beschraenkte\ \Schlussregeln:%
\footnote{%
	Nach \eqref{def:AB}, \eqref{def:ab} und \vreffor{def:aabb} sind die "`Brüche"' \Aussagen, und keine Paare mehr.
	Die Äquivalenz der Aussagen steht schon in \vreffor{eq:AB}
}
\[
	\frac{             \MtsPraemisseSet}{             \MtsKonklusionSet} \; \MtsEquiv \;
	\frac{\MtsSet(\vec{\MtsPraemisse}) }{\MtsSet(\vec{\MtsKonklusion}) } \; \MtsEquiv \;
	\frac{        \vec{\MtsPraemisse}  }{        \vec{\MtsKonklusion}  } \; \MtsEquiv \;
	\frac{
		\links{\MtsPraemisse}_1 \MtsDerive_{\MtsPraemisseSet} \rechts{\MtsPraemisse}_1 \MtsUnd
		\dots \MtsUnd
		\links{\MtsPraemisse}_N \MtsDerive_{\MtsPraemisseSet} \rechts{\MtsPraemisse}_N }{
		\links{\MtsKonklusion}_1     \MtsDerive_{\MtsKonklusionSet}     \rechts{\MtsKonklusion}_1     \MtsUnd
		\dots \MtsUnd
		\links{\MtsKonklusion}_M     \MtsDerive_{\MtsKonklusionSet}     \rechts{\MtsKonklusion}_M
	}
	\quad \text{\DefFt{, \defTxt{\Schlussregel}} oder \defTxt{\formalerSatz}}
	\tagFS \label{def:FS}
\]

\subsection[Beweise]{\Beweise}% ------------------------------------------------
\label {sub:Beweise}

Für einen \defTxt{\Beweis} in \ASBA\ ist stets gegeben:%
\footnote{%
	\ASBA\ selbst kann nur endliche Mengen aBspeichern.
	Für \ASBA muss daher einschränkend $\MtsSchlussregelSet \in \MtsRelf(\MtsRelf(\MtsPotf(\MtsSprache)))$ und $\MtsErgebnisSet \in \MtsRelf(\MtsPotf(\MtsSprache))$ sein.
}
\begin{align}
	& \MtsSprache     &           \quad &
	&& \text{, eine \Menge\ von \Formeln, die zugrundeliegende \defTxt{\Sprache}.}
	\label{def:Sprache}      \\
	& \MtsErsetzungSet   & \MtsSubsetEq \quad & \MengeDef{\MtsErsetzung}{\FunktionDef{\MtsErsetzung}{\MtsSprache}{\MtsSprache}}
	&& \text{, eine \Menge\ von \Funktionen, die \defTxt{\Ersetzungen}.}
	\label{def:Ersetzung} \\
	& \MtsSchlussregelSet & \in       \quad & \MtsRelSchlussregel
	&& \text{, eine \Menge\ von \defTxt{\Schlussregeln}.}
	\label{def:Schlussregel} \\
	& \MtsErgebnisSet        & \in       \quad & \MtsRelAllDerive
	&& \text{, eine \Menge\ von \Ableitungen, die \defTxt{\Ergebnisse}.}
	\label{def:Konklusion} &&
\end{align}
%
Die \emph{\Ersetzungen} sorgen \textzB\ dafür, dass aus einer \allgemeingueltigenFormel\ wie  \seqqt{$\alpha \OjkImp (\beta \OjkImp \alpha)$} \textzB\ die \allgemeingueltigeFormel\ \seqqt{$\gamma \OjkImp (\delta \OjkImp \gamma)$} abgeleitet werden kann.
%
Die \emph{\Schlussregeln} geben erlaubte Schlussfolgerungen aus gegebenen \Elementen\ an und umfassen auch die Prämissen eines \Satzes.
Die \emph{\Ergebnisse} schließlich sind das, was mittels eines \Beweises\ aus den gegebenen Prämissen \MtsSprache, \MtsErsetzungSet\ und \MtsSchlussregelSet\ gefolgert werden soll.

Im Fall von \beschraenkten\ \Schlussregeln\ können statt \MtsSchlussregelSet\ und \MtsErgebnisSet\ auch
\begin{align}
	& \vec{\MtsSchlussregel} & \in \quad & \MtsTup(\MtsTup(\MtsAllDerive)^2)
	&& \text{, ein \Tupel\ aus \defTxt{\Schlussregeln}.}
	\label{def:Schlussregelvector} \\
	& \vec{\MtsErgebnis}        & \in \quad & \quad \; \MtsTup(\MtsAllDerive)
	&& \text{, ein \Tupel\ aus \defTxt{\Ableitungen}, die \defTxt{\Ergebnisse}.}
	\label{def:Konklusionsvector}    \formulatoleft
\end{align}
gegeben sein. Mit
\begin{align}
	& \MtsSchlussregelSet \MtsDefEq \MengeDef{(\MtsSet(\vec{\MtsPraemisse}), \MtsSet(\vec{\MtsKonklusion}))}{(\vec{\MtsPraemisse}, \vec{\MtsKonklusion}) \in \MtsSet(\vec{\MtsSchlussregel})}
	\\
	& \MtsErgebnisSet \MtsDefEq \MtsSet(\vec{\MtsErgebnis})
\end{align}
ergibt sich wegen \eqref{eq:Setf} und \vreffor{eq:vecrel} wieder die erste Form.

\subsection[Beispiel für einen Beweis]{Beispiel für einen \Beweis}% ------------
\label {sub:Beispielbeweis}

\todo{Nacharbeiten}     %TODO *** Nacharbeiten ***

\todo{Hier weitermachen}%TODO *** hier weitermachen ***

Zur Veranschaulichung ein Beispiel:\citenote{bib:HilbertKalkuelModusPonens}
\begin{align}
	& \MtsErsetzung_{\alpha,\beta}(\delta) & \MtsDefEq \quad & \text{das }\delta \text{, bei dem alle Vorkommen von $\alpha$ durch $\beta$ ersetzt wurden} \\
	& \MtsSprache & \MtsDefEq \quad & \text{die \Menge\ aller \Formeln\ der \aussagenlogischenSprache} \\
	& \MtsPraemisse_1    & \MtsDefEq \quad & (A, \{\alpha\}) \\
	& \MtsPraemisse_2    & \MtsDefEq \quad & (B, \{\alpha \OjkImp \beta\}) \\
	& \MtsPraemisse_3    & \MtsDefEq \quad & (A \MtsCup B, \{\beta\}) \\
	& \MtsErsetzungSet   & \MtsDefEq \quad & \{\MtsErsetzung_{\alpha,\delta}, \MtsErsetzung_{\beta,B}, \MtsErsetzung_{\beta,B\OjkImp \delta}, \MtsErsetzung_{\gamma,\delta} \} \\
	& \MtsSchlussregelSet & \MtsDefEq \quad & \dots \\
	&          & \chi_1 \; \MtsDefEq \quad & \alpha \OjkImp (\beta \OjkImp \alpha) \\
	&          & \chi_2 \; \MtsDefEq \quad & (\alpha \OjkImp (\beta \OjkImp \gamma)) \OjkImp ((\alpha \OjkImp \beta) \OjkImp (\alpha \OjkImp \gamma)) \\
	& \MtsAxiomSet          & \MtsDefEq \quad & \{\chi_1, \chi_2\} \\
	& \MtsKonklusionRel     & \MtsDefEq \quad & \dots
	\formulatoleft
\end{align}
%TODO Beispiel vervollständigen

\subsection[Beweisschritte]{\Beweisschritte}% ----------------------------------
\label {sub:Beweisschritte}

%TODO Elimination von Prämissen behandeln!
Ein \Beweis%
\footnote{\citesee{bib:Rautenberg} Kapitel~1.6 und~3.6}
in \ASBA\ besteht aus
\begin{align}
	& \text{einer \Schlussregel} && \frac{\MtsPraemisseSet}{\MtsKonklusionSet}
	\\
	& \text{einer Folge} && \MtsBeweisschrittTup = (\MtsBeweisschritt_1, \MtsBeweisschritt_2, ..., \MtsBeweisschritt_K)
	&& \text{von \emph{\Beweisschritten} } \MtsBeweisschritt_k
	&& \text{, die \defTxt{\Beweisschrittfolge}}
	\label{def:Beweisschrittfolge}
	\\
	& \text{einer Folge} && \MtsTransformationTup = (\MtsTransformation_1, \MtsTransformation_2, ..., \MtsTransformation_K)
	&& \text{von \emph{\Transformationen} } \MtsTransformation_k
	&& \text{, die \defTxt{\Transformationsfolge}}
	\label{def:Transformationsfolge}
\end{align}
Dabei ist $K$ ein \Element\ aus \MtsINo, $0 \le k \le K$, die \defTxt{\Beweisschritte} $\MtsBeweisschritt_k$ sind \Schlussregeln\ und die \Transformationen\ $\MtsTransformation_k$ werden später definiert.
%TODO Verweis auf Definition der Transformationen fehlt
Wir definieren noch:
\begin{align}
	& \MtsBeweisschrittSet_k & \MtsDefEq \quad & \{\MtsBeweisschritt_1, \dots, \MtsBeweisschritt_k\} & \quad \text{, für~~} 0 \le k \le K
	\label{def:Beweisschrittebis} \\
	& \MtsBeweisschrittSet   & \MtsDefEq \quad & \MtsBeweisschrittSet_K \label{def:Beweisschrittmenge}
	\formulatoleft\formulatoleft\formulatoleft
\end{align}
und nennen \MtsBeweisschrittSet\ die \defTxt{\Beweisschrittmenge} der \Beweisschrittfolge\ \MtsBeweisschrittTup.
Dann ist $\MtsBeweisschrittSet_0=\MtsEmptyset$ und $\MtsBeweisschrittSet_i\MtsSubsetEq\MtsBeweisschrittSet_j\MtsSubsetEq\MtsBeweisschrittSet$ für $0\le i\le j\le K$.
-- Wir nennen die \Beweisschrittfolge\ auch eine \defTxt{\Ableitung} aus \MtsKonklusionSet\ aus \MtsPraemisseSet.

%TODO Rolle der Transformationen erläutern
Jeder \Beweisschritt\ $ \MtsBeweisschritt_k \text{ für } 1 \le k \le K $ muss entweder eine \Praemisse\ aus \MtsPraemisseSet\ oder durch Anwendung einer \allgemeingueltigenSchlussregel\ auf eine \Teilmenge\ von $\MtsBeweisschrittSet_{k-1}$ eine wahre \Formel\ oder eine weitere \allgemeingueltigeSchlussregel\ sein.
Schließlich muss noch
\[ \MtsKonklusionSet \MtsSubsetEq \MtsBeweisschrittSet \]
sein, da jede \Konklusion\ aus \MtsKonklusionSet\ in der Folge \MtsBeweisschrittTup\ vorkommen und somit \Element\ aus der \Menge\ \MtsBeweisschrittSet\ sein muss.

========================================================================
%TODO ==================================================================

\Endchapter
