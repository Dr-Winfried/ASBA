%%############################################################################%%
%%                                                                            %%
%% Datei:  ASBA-Literaturverzeichnis.tex                                      %%
%% Inhalt: Literaturverzeichnis                                               %%
%%                                                                            %%
%% Copyright (C) 2017  Winfried Teschers                                      %%
%%                                                                            %%
%% This program is free software: you can redistribute it and/or modify       %%
%% it under the terms of the GNU Affero General Public License as published   %%
%% by the Free Software Foundation, either version 3 of the License, or       %%
%% (at your option) any later version.                                        %%
%%                                                                            %%
%% This program is distributed in the hope that it will be useful,            %%
%% but WITHOUT ANY WARRANTY; without even the implied warranty of             %%
%% MERCHANTABILITY or FITNESS FOR A PARTICULAR PURPOSE.  See the              %%
%% GNU Affero General Public License for more details.                        %%
%%                                                                            %%
%% You should have received a copy of the GNU Affero General Public License   %%
%% along with this program.  If not, see <http://www.gnu.org/licenses/>.      %%
%%                                                                            %%
%% Dr. Winfried Teschers                                                      %%
%% Anton-Günther-Straße 26c                                                   %%
%% 91083 Baiersdorf                                                           %%
%% Germany                                                                    %%
%%                                                                            %%
%% e-mail: winfried.teschers@t-online.de                                      %%
%%                                                                            %%
%%############################################################################%%

% !TeX root = ASBA.tex
% !TeX encoding = UTF-8
% !TeX spellcheck = de_DE

%chapter{Literaturverzeichnis}% ################################################

\begin{flushleft}
	\begin{thebibliography}{12}
		\label{dic-Literaturverzeichnis}
		\likesection{\bibname}

		\bibitem{bib:Rautenberg} Wolfgang Rautenberg,
		\emph{Einführung in die Mathematische Logik}:
		Ein Lehrbuch, 3.\@ Auflage, Vieweg+Teubner 2008

		\bibitem{bib:Schwarz} Norbert Schwarz,
		"`unveränderte"' PDF-Fassung der 3. Auflage von 1991
		$\rightarrow$%
		\emph{Einführung in \TeX}:
		\footnote{%
			Das Datum hinter dem Link gibt --- je nachdem welches bekannt ist --- das Datum der letzten Änderung, den Stand der Seite oder das Datum, an dem die Seite angeschaut wurde an.
			Sind mehrere Daten vorhanden, wird das erste vorhandene in der angegebenen Reihenfolge genommen.
			--- Dies gilt für alle hier aufgelisteten Seiten im Internet.
		}
		\url{http://www.ruhr-uni-bochum.de/www-rz/schwanbs/TeX/}
		--- 06.02.2002

		\bibitem{bib:Apacheii} \emph{Apache License}, Version 2.0
		\footnote{%
			Der Pfeil~($\rightarrow$) verweist stets auf einen Link zu einer Seite im Internet.
		}
		\tourl{http://www.apache.org/licenses/LICENSE-2.0}
		--- 01.2004%

		\bibitem{bib:BSLi} \emph{Boost Software License} 1.0
		\tourl{http://www.boost.org/users/license.html}
		--- 17.08.2003

		\bibitem{bib:EPL} \emph{Eclipse Public License} Version 1.0
		\tourl{http://www.eclipse.org/org/documents/epl-v10.php}
		--- 09.03.2017

		\bibitem{bib:AGPL} \emph{GNU Affero General Public License}
		\tourl{http://www.gnu.org/licenses/agpl}
		--- 19.11.2007

		\bibitem{bib:GPLi} \emph{GNU General Public License}
		\tourl{http://www.gnu.org/licenses/old-licenses/gpl-1.0}
		--- 02.1989

		\bibitem{bib:GPLii} \emph{GNU General Public License}, Version 2
		\tourl{http://www.gnu.org/licenses/old-licenses/gpl-2.0}
		--- 06.1991

		\bibitem{bib:LGPLii} \emph{GNU Lesser General Public License},
		Version 2.1
		\tourl{http://www.gnu.org/licenses/old-licenses/lgpl-2.1}
		--- 02.1999

		\bibitem{bib:Clover} Lizenz für \emph{Clover}
		\tourl{https://www.atlassian.com/software/clover}
		--- 2017

		\bibitem{bib:EULA} Lizenz
		für \emph{Microsoft Visual Studio Express 2015}
		\tourl{https://www.visualstudio.com/de/license-terms/mt171551/}
		--- 2017

		\bibitem{bib:MiKTeX} Lizenz für \emph{MikTeX}
		\tourl{https://miktex.org/kb/copying}
		--- 13.04.2017

		\bibitem{bib:SAX} Lizenz für \emph{SAX}
		\tourl{http://www.saxproject.org/copying.html}
		--- 05.05.2000

		\bibitem{bib:MIT} \emph{MIT License}
		\tourl{https://opensource.org/licenses/MIT/}
		--- 09.03.2017

		\bibitem{bib:JavaSE} \emph{Oracle Binary Code License Agreement}
		\tourl{http://java.com/license}
		--- 02.04.2013

		\bibitem{bib:OSI} \emph{OSI Certified Open Source Software}
		\tourl{https://opensource.org/pressreleases/certified-open-source.php}
		--- 16.06.1999

		\bibitem{bib:WDCDL} \emph{W3C Document License}
		\tourl{http://www.w3.org/Consortium/Legal/2015/doc-license}
		--- 01.02.2015

		\bibitem{bib:WDCSNL} \emph{W3C Software Notice and License}
		\tourl{http://www.w3.org/Consortium/Legal/2002/copyright-software-20021231.html}
		--- 13.05.2015

		\bibitem{bib:HilbertII} \emph{Hilbert II --- Introduction}
		\tourl{http://www.qedeq.org/}
		--- 20.01.2014

		\bibitem{bib:qedeq} \emph{Formal Correct Mathematical Knowledge}:
		GitHub Repository vom Projekt Hilbert II
		\tourl{https://github.com/m-31/qedeq/}
		--- 18.03.2017

		\bibitem{bib:ASBA} \emph{ASBA --- Axiome, Sätze, Beweise und Auswertungen}.
		Projekt zur maschinellen Überprüfung von mathematischen Beweisen
		und deren Ausgabe in lesbarer Form:
		GitHub Repository vom Projekt ASBA
		--- in Bearbeitung
		\tourl{https://github.com/Dr-Winfried/ASBA}

		\bibitem{bib:LogikDe} Meyling, Michael:
		\emph{Anfangsgründe der mathematischen Logik}
		\tourl{http://www.qedeq.org/current/doc/math/qedeq\_logic\_v1\_de.pdf}
		--- 24.~Mai~2013 (in Bearbeitung)

		\bibitem{bib:PraedikatenlogikDe} Meyling, Michael:
		\emph{Formale Prädikatenlogik}
		\tourl{http://www.qedeq.org/current/doc/math/qedeq\_formal\_logic\_v1\_de.pdf}
		--- 24.~Mai~2013 (in Bearbeitung)

		\bibitem{bib:MengenlehreDe} Meyling, Michael:
		\emph{Axiomatische Mengenlehre}
		\tourl{http://www.qedeq.org/current/doc/math/qedeq\_set\_theory\_v1\_de.pdf}
		--- 24.~Mai~2013 (in Bearbeitung)

		\bibitem{bib:LogikEn} Meyling, Michael:
		\emph{Elements of Mathematical Logic}
		\tourl{http://www.qedeq.org/current/doc/math/qedeq\_logic\_v1\_en.pdf}
		--- 24.~Mai~2013 (in Bearbeitung)

		\bibitem{bib:PraedikatenlogikEn} Meyling, Michael:
		\emph{Formal Predicate Calculus}
		\tourl{http://www.qedeq.org/current/doc/math/qedeq\_formal\_logic\_v1\_en.pdf}
		--- 24.~Mai~2013 (in Bearbeitung)

		\bibitem{bib:MengenlehreEn} Meyling, Michael:
		\emph{Axiomatic Set Theory}
		\tourl{http://www.qedeq.org/current/doc/math/qedeq\_set\_theory\_v1\_en.pdf}
		--- 24.~Mai~2013 (in Bearbeitung)

		\bibitem{bib:Wikipedia} Wikipedia Hauptseite
		\tourl{https://de.wikipedia.org/wiki/Wikipedia:Hauptseite}
		--- 07.11.2017

		\bibitem                        {bib:Aussagenlogik} Wikipedia:
		\emph                               {Aussagenlogik}
		\tourl{https://de.wikipedia.org/wiki/Aussagenlogik}
		--- 18.01.2018

		\bibitem                        {bib:AussagenlogikFormalerZugang} Wikipedia:
		\emph                               {Aussagenlogik} \chaptername~4 \emph{Formaler Zugang}
		\tourl{https://de.wikipedia.org/wiki/Aussagenlogik\#Formaler\_Zugang}
		--- 18.01.2018

		\bibitem                        {bib:Funktion} Wikipedia:
		\emph                               {Funktion (Mathematik)} \chaptername~2.1 \emph{Mengentheoretische Definition}
		\tourl{https://de.wikipedia.org/wiki/Funktion\_(Mathematik)\#Mengentheoretische\_Definition}
		--- 27.01.2018

		\bibitem                        {bib:HilbertKalkuelModusPonens} Wikipedia:
		\emph                               {Hilbert-Kalkül} \chaptername~1.4 \emph{Modus (ponendo) ponens}
		\tourl{https://de.wikipedia.org/wiki/Hilbert-Kalk\%C3\%BCl\#Modus\_(ponendo)\_ponens}
		--- 18.06.16

		\bibitem                        {bib:Ableitung} Wikipedia:
		\emph                               {Ableitung (Logik)}
		\tourl{https://de.wikipedia.org/wiki/Ableitung_(Logik)}
		--- 20.02.2018

		\bibitem                        {bib:Aussage} Wikipedia:
		\emph                               {Aussage (Logik)}
		\tourl{https://de.wikipedia.org/wiki/Aussage_(Logik)}
		--- 11.03.2018

		\bibitem                        {bib:Element} Wikipedia:
		\emph                               {Element (Mathematik)}
		\tourl{https://de.wikipedia.org/wiki/Element_(Mathematik)}
		--- 09.01.2016

		\bibitem                        {bib:Identitaet} Wikipedia:
		\emph                               {Identität (Logik)} \chaptername~2.3 \emph{Identität in der Informatik}
		\tourl{https://de.wikipedia.org/wiki/Identit\%C3\%A4t\_(Logik)\#Identit.C3.A4t\_in\_der\_Informatik}
		--- 18.05.2017

		\bibitem                        {bib:JunktorMoeglich} Wikipedia:
		\emph                               {Junktor} \chaptername~2.2 \emph{Mögliche Junktoren}
		\tourl{https://de.wikipedia.org/wiki/Junktor\#M.C3.B6gliche\_Junktoren}
		--- 21.10.2017

		\bibitem                        {bib:Kalkuel} Wikipedia:
		\emph                               {Kalkül}
		\tourl{https://de.wikipedia.org/wiki/Kalk\%C3\%BCl}
		--- 26.02.2017

		\bibitem                        {bib:Konstante} Wikipedia:
		\emph                               {Konstante (Logik)}
		\tourl{https://de.wikipedia.org/wiki/Konstante_(Logik)}
		--- 20.01.2016

		\bibitem                        {bib:Logik} Wikipedia:
		\emph                               {Logik}
		\tourl{https://de.wikipedia.org/wiki/Logik}
		--- 28.01.2018

		\bibitem                        {bib:Menge} Wikipedia:
		\emph                               {Menge}
		\tourl{https://de.wikipedia.org/wiki/Menge_(Mathematik)}
		--- 07.03.2018

		\bibitem                        {bib:Mengenlehre} Wikipedia:
		\emph                               {Mengenlehre}
		\tourl{https://de.wikipedia.org/wiki/Mengenlehre}
		--- 17.01.2018

		\bibitem                        {bib:Praedikatenlogik} Wikipedia:
		\emph                               {Prädikatenlogik}
		\tourl{https://de.wikipedia.org/wiki/Pr\%C3\%A4dikatenlogik}
		--- 01.03.2018

		\bibitem                        {bib:PraedikatenlogikErsterStufe} Wikipedia:
		\emph                               {Prädikatenlogik erster Stufe}
		\tourl{https://de.wikipedia.org/wiki/Pr\%C3\%A4dikatenlogik\_erster\_Stufe}
		--- 26.11.2017

		\bibitem                        {bib:RelationMehrstellig} Wikipedia:
		\emph                               {Relation (Mathematik)} \chaptername~1.1 \emph                               {Mehrstellige Relation}
		\tourl{https://de.wikipedia.org/wiki/Relation\_(Mathematik)\#Mehrstellige\_Relation}
		--- 27.01.2018

		\bibitem                        {bib:Schlussregel} Wikipedia:
		\emph                               {Schlussregel}
		\tourl{https://de.wikipedia.org/wiki/Schlussregel}
		--- 29.03.2015

		\bibitem                        {bib:Signatur} Wikipedia:
		\emph                               {Signatur (Modelltheorie)}
		\tourl{https://de.wikipedia.org/wiki/Signatur_(Modelltheorie)}
		--- 04.03.2018

		\bibitem                                {bib:NatuerlichesSchliessen} Wikipedia:
		\emph                               {Systeme natürlichen Schließens}
		\tourl{https://de.wikipedia.org/wiki/Systeme\_nat\%C3\%BCrlichen\_Schlie\%C3\%9Fens}
		--- 25.10.2017

		\bibitem                        {bib:Tupel} Wikipedia:
		\emph                               {Tupel}
		\tourl{https://de.wikipedia.org/wiki/Tupel}
		--- 17.12.2017

		\bibitem                        {bib:Variable} Wikipedia:
		\emph                               {Variable (Mathematik)}
		\tourl{https://de.wikipedia.org/wiki/Variable_(Mathematik)}
		--- 08.03.2018

		\bibitem                        {bib:Wahrheitswert} Wikipedia:
		\emph                               {Wahrheitswert}
		\tourl{https://de.wikipedia.org/wiki/Wahrheitswert}
		--- 03.07.2017

		\clearpage % schon hier!
	\end{thebibliography}
\end{flushleft}
