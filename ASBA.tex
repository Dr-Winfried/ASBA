%% Datei ASBA.tex zur Erzeugung der Projektbeschreibung von ASBA.
%%
%% Copyright (C) 2017  Winfried Teschers
%%
%% This program is free software: you can redistribute it and/or modify
%% it under the terms of the GNU Affero General Public License as published
%% by the Free Software Foundation, either version 3 of the License, or
%% (at your option) any later version.
%%
%% This program is distributed in the hope that it will be useful,
%% but WITHOUT ANY WARRANTY; without even the implied warranty of
%% MERCHANTABILITY or FITNESS FOR A PARTICULAR PURPOSE.  See the
%% GNU Affero General Public License for more details.
%%
%% You should have received a copy of the GNU Affero General Public License
%% along with this program.  If not, see <http://www.gnu.org/licenses/>.
%%
%% Dr. Winfried Teschers
%% Anton-Günther-Str. 26c
%% 91083 Baiersdorf
%% Germany
%%
%% e-mail: winfried.teschers@t-online.de

\documentclass[english,ngerman,parskip=half,headsepline,footsepline]{scrreprt}
\usepackage[utf8]{inputenc} % Input encoding specification
\usepackage[T1]{fontenc}
\usepackage{lmodern}
\usepackage{scrlayer-scrpage}
\usepackage{graphicx}
\usepackage{varioref}
\usepackage{amsmath} % ASM mathematical facilities for LaLeX.
\usepackage{amsfonts} % TeX fonts from the American Mathematical Society.
\usepackage{amssymb} % Symbols from the American Mathematical Society.
\usepackage{babel} % Multilingual support for plain TeX or LaTeX.
\usepackage{microtype} % Subliminal refinements towards typographical perfection.
\usepackage{rotating} % Rotating tools, including rotated full page floats.
\usepackage{geometry} % Flexible and complete interface to document dimensions.
%% \usepackage{cleveref} % Intelligent cross-referencing - funktioniert nicht!
\usepackage{booktabs-de} % Nicer layout of tables.
\usepackage{threeparttable} % Tables with captions and notes all the same width.
\usepackage{caption} % Customizing captions in floating environments.
\usepackage{diagbox} % Table heads with diagonal lines.
\usepackage{arydshln} % Draw dash-lines in array/tabular.
%\usepackage{url} % Verbatim with URL-sensitive linebreaks.
\usepackage[colorlinks,linktoc=all]{hyperref} % Extensive support for hypertext in LaTeX - muss als letztes Package angegeben werden.

\geometry{textwidth=170mm,textheight=256mm,twoside}
\ihead{}
\chead{}
\ohead{ASBA}
\ifoot{\today}
\cfoot{Winfried Teschers}
\ofoot{\thepage}
	
\newcounter{Enumi}
\captionsetup{labelfont=bf}

\title{ASBA\\Axiome, Sätze, Beweise und Ausgaben}
\subtitle{Projekt zur maschinellen Überprüfung von mathematischen Beweisen und deren Ausgabe in lesbarer Form}
\author{Winfried Teschers}
\date{\today}
\publishers{Es wird ein Projekt beschrieben, das zu eingegebenen Axiomen, Sätzen, und Beweisen letztere prüft, Auswertungen generiert und zu gegebenen Ausgabeschemata eine Ausgabe der Elemente in üblicher Formelschreibweise im \LaTeX-Format erstellt.}

\begin{document}
	\maketitle
	\tableofcontents	
	\vfill
	Copyright (C) 2017  Winfried Teschers
	
	\vspace{12pt}
	\selectlanguage{english}
	Permission is granted to copy, distribute and/or modify this document under the terms of the GNU Free Documentation License, Version 1.3 or any later version published by the Free Software Foundation; with no Invariant Sections, no Front-Cover Texts, and no Back-Cover Texts. You should have received a copy of the GNU Free Documentation License along with this document.  If not, see \url{http://www.gnu.org/licenses/}.
	\selectlanguage{ngerman}
	\vspace{12pt}
	
	Dr. Winfried Teschers\\
	Anton-Günther-Straße 26c\\
	91083 Baiersdorf\\
	Germany
	
	e-mail: winfried.teschers@t-online.de\par
	
	\thispagestyle{scrheadings}
	
	
	%===============================================================================
	
	\chapter{Analyse}
	\thispagestyle{scrheadings}
	
	In der Mathematik gibt es eine unüberschaubare Menge an Axiomen, Sätzen, Beweisen, Fachbegriffen\footnote{ Fachbegriffe sind Namen für Axiome, Sätze, Beweise und Disziplinen. Symbole können als spezielle Fachbegriffe aufgefasst werden.} und Disziplinen. Dabei verstehen wir unter einer \emph{Disziplin} einen Teil der Mathematik  mit einer zugehörigen Basis von Axiomen, Sätzen und spezifischen Fachbegriffen, zum Beispiel Logik, Mengenlehre und Gruppentheorie\footnote{ Eine Disziplin kann hier sehr klein sein und im Extremfall einen einzigen Satz enthalten. \emph{Umgebung} wäre in diesem Projekt eine bessere Bezeichnung, könnte aber zu Verwechslungen führen, da dies schon ein verbreiteter Fachbegriff ist.}. Zu den meisten Disziplinen gibt es auch noch ungelöste Probleme.
	
	Es fehlt ein Überblick und die Möglichkeit, Beweise automatisch zu überprüfen. Außerdem muss all dies in üblicher mathematischer Schreibweise ausgegeben werden können.
	
	\section{Fragen}
	\label{sec:Fragen}
	Einige der Fragen, die in diesem Zusammenhang auftauchen, werden hier formuliert:
	
	\begin{enumerate}
		
		\item \label{Frage:Grundlagen} \emph{Grundlagen}: Was sind die Grundlagen? Zum Beispiel welche Logik und Mengenlehre.
		
		\item \label{Frage:Basis} \emph{Basis}: Welche wichtigen Axiome, Sätze, Beweise, Fachbegriffe und Diszipline gibt es?
		
		\item \label{Frage:Axiome} \emph{Axiome}: Welche Axiome werden bei einem Satz oder Beweis vorausgesetzt? Allgemein anerkannte oder auch strittige, wie zum Beispiel den \emph{Satz vom ausgeschlossenen Dritten} (\emph{tertium non datur}) oder das \emph{Auswahlaxiom}.
		
		\item \label{Frage:Beweis} \emph{Beweis}: Ist ein Beweis fehlerfrei?
		
		\item \label{Frage:Konstruktion} \emph{Konstruktion}: Gibt es einen konstruktiven Beweis?
		
		\item \label{Frage:Vergleiche} \emph{Vergleiche}: Welcher Beweis ist besser? Nach welchem Kriterium? Zum Beispiel elegant, kurz, einsichtig oder wenige Axiome. Was heißt eigentlich \emph{elegant}?
		
		\item \label{Frage:Definitionen} \emph{Definitionen}: Was ist mit einem Fachbegriff oder einer Disziplin jeweils genau gemeint? Zum Beispiel \emph{Stetigkeit}, \emph{Integral} und \emph{Analysis}.
		
		\item \label{Frage:Abhängigkeiten} \emph{Abhängigkeiten}: Wie heißt ein Fachbegriff oder eine Disziplin in einer anderen Sprache? Ist wirklich dasselbe gemeint? Was ist mit Fachbegriffen in verschiedenen Disziplinen?
		
		\item \label{Frage:Bekanntheit} \emph{Bekanntheit}: Ist ein Axiom, Satz, Beweis, Fachbegriff oder eine Disziplin schon einmal \textendash\ ggf. abweichend \textendash\ definiert, formuliert oder bewiesen worden?
		
		\item \label{Frage:Darstellung} \emph{Darstellung}: Wie kann man einen Satz und den zugehörigen Beweis \textendash\ ggf. auch spezifisch für eine Disziplin \textendash\ darstellen?
		
		\item \label{Frage:Forschung} \emph{Forschung}: Welche Probleme gibt es noch zu erforschen.
		
	\end{enumerate}
	
	\section{Mission}
	\label{sec:Mission}
	Um zur Lösung obiger Fragen beizutragen, soll ein System entwickelt werden, das die folgenden Eigenschaften hat:
	
	\begin{enumerate}
		\item \label{Mission:Daten} \emph{Daten}: Axiome, Sätze, Beweise, Fachbegriffe und Diszipline können in formaler Form gespeichert werden \textendash\ auch nicht oder unvollständig bewiesene Sätze.
		
		\item \label{Mission:Definitionen} \emph{Definitionen}: Es können Fachbegriffe für Axiome, Sätze, Beweise und Diszipline \textendash\ letztere mit eigenen Axiomen, Sätzen, Beweisen, Fachbegriffen und übergeordneten Disziplinen \textendash\ definiert werden. Die Definitionen dürfen wiederum schon bekannte Fachbegriffe und Diszipline verwenden.
		
		\item \label{Mission:Prüfung} \emph{Prüfung}: Vorhandene Beweise können automatisch geprüft werden.
		
		\item \label{Mission:Ausgaben} \emph{Ausgaben}: Die Axiome, Sätze und Beweise können in üblicher Schreibweise \textendash\ abhängig von Sprache und Disziplin \textendash\ ausgegeben werden.
		
		\item \label{Mission:Auswertungen} \emph{Auswertungen}: Zusätzlich zur Ausgabe der gespeicherten Daten sind verschiedene Auswertungen möglich, unter anderem für die meisten der unter Abschnitt~\vref{sec:Fragen} behandelten Fragen.
		
		\setcounter{Enumi}{\value{enumi}}
	\end{enumerate}

	Damit das System nicht umsonst erstellt wird und möglichst breite Verwendung findet, werden noch zwei Punkte angefügt:
	
	\begin{enumerate}
		\setcounter{enumi}{\value{Enumi}}
		
		\item \label{Mission:Lizenz} \emph{Lizenz}: Die Software ist \emph{Open Source}.
		
		\item \label{Mission:Akzeptanz} \emph{Akzeptanz}: Das System wird von den Fachleuten akzeptiert und verwendet.
	\end{enumerate}
	
	\section{Ziele}
	\label{sec:Ziele}
	Um die Mission zu erfüllen, soll ein System entwickelt werden, das die folgenden Anforderungen erfüllt:
	
	\begin{enumerate}
		\item \label{Ziel:Daten} \emph{Daten}: Das System enthält möglichst viele wichtige Axiome, Sätze, Beweise, Fachbegriffe, Diszipline und Ausgabeschemata\footnote{ Um den Punkt~\vref{Mission:Ausgaben} von Abschnitt~\vref{sec:Mission} erfüllen zu können, brauchen wir noch für Disziplinen spezifische Ausgabeschemata, welche die Art der Ausgabe von Daten beschreiben.}.
		
		\item \label{Ziel:Form} \emph{Form}: Die Daten liegt in formaler, geprüfter Form vor.
		
		\item \label{Ziel:Eingaben} \emph{Eingaben}: Die Eingabe von Daten erfolgt in einer formalen Syntax. Folgende Daten können eingegeben werden:
		\begin{enumerate}
			\item Axiome
			\item Sätze
			\item Beweise
			\item Fachbegriffe
			\item Diszipline
			\item Ausgabeschemata
		\end{enumerate}
		Dabei sind alle Daten nur innerhalb einer (übergeordneten) Disziplin gültig. Die oberste Disziplin erhält keinen Namen, es ist quasi die ganze Mathematik.
		
		\item \label{Ziel:Prüfung} \emph{Prüfung}: Vorhandene Beweise können automatisch geprüft werden.
		
		\item \label{Ziel:Ausgaben} \emph{Ausgaben}: Die Ausgabe kann in einer eindeutigen, formalen Syntax oder gemäß vorhandener Ausgabeschemata erfolgen.
		
		\item \label{Ziel:Auswertungen} \emph{Auswertungen}: Zusätzlich zur Ausgabe der Daten sind verschiedene Auswertungen möglich. Insbesondere kann zu jedem Beweis eines Satzes angegeben werden, wie viele Beweisschritte und welche Axiome und Sätze\footnote{ Sätze, die quasi als Axiome verwendet werden.} er benötigt.
		
		\item \label{Ziel:Anpassbarkeit} \emph{Anpassbarkeit}: Fachbegriffe und die Darstellung bei der Ausgabe können mit Hilfe von \textendash\ gegebenenfalls unbenannten \textendash\ Disziplinen angepasst werden.
		
		\item \label{Ziel:Individualität} \emph{Individualität}: Axiome und Sätze können für jeden Beweis individuell vorausgesetzt werden. Dabei sind fachspezifische Fachbegriffe erlaubt.
		
		\item \label{Ziel:Internet} \emph{Internet}: Die Daten können auf mehrere Dateien verteilt sein. Ein Teil davon \textendash\ oder sogar alle \textendash\ können im Internet liegen.
		
		\item \label{Ziel:Kommunikation} \emph{Kommunikation}: Die Kommunikation mit dem System kann mit den Fachbegriffen der einzelnen Diszipline erfolgen.
		
		\item \label{Ziel:Zugriff} \emph{Zugriff}: Der Zugriff auf das System kann lokal und über das Internet erfolgen.
		
		\item \label{Ziel:Unabhängigkeit} \emph{Unabhängigkeit}: Das System kann offline und online arbeiten.
		
		\item \label{Ziel:Rekursion} \emph{Rekursion}: Es kann rekursiv über alle verwendeten Dateien \textendash\ auch solchen, die im Internet liegen \textendash\ ausgewertet werden.
		
		\item \label{Ziel:Bedienbarkeit} \emph{Bedienbarkeit}: Das System ist einfach zu bedienen.
		
		\item \label{Ziel:Lizenz} \emph{Lizenz}: Die Software ist Open Source sein.
	\end{enumerate}
	
	\section{Zusammenhänge}
	\label{sec:Zusammenhänge}
	Ausgehend von einer Liste der Fragen haben wir über die Zwischenstufe Mission Anforderungen an das zu realisierende System gestellt. Mit einem großen X werden Spalten markiert, deren Punkte für die Erfüllung der Anforderungen in den Zeilen nötig sind. Idealerweise soll die Erfüllung der Anforderungen die Fragen beantworten bzw. zur Beantwortung beitragen.
	
	\begin{threeparttable}
		\begin{tabular}{@{}r@{ }l|*{7}{c}|}
			\multicolumn{2}{l|}{\diagbox{\textbf{Fragen}}{\textbf{Mission}}}
			& \rotatebox{90}{\mbox{\ref{Mission:Daten}        Daten        }}
			& \rotatebox{90}{\mbox{\ref{Mission:Definitionen} Definitionen }}
			& \rotatebox{90}{\mbox{\ref{Mission:Prüfung}      Prüfung      }}
			& \rotatebox{90}{\mbox{\ref{Mission:Ausgaben}     Ausgaben     }}
			& \rotatebox{90}{\mbox{\ref{Mission:Auswertungen} Auswertungen }}
			& \rotatebox{90}{\mbox{\ref{Mission:Lizenz}       Lizenz       }}
			& \rotatebox{90}{\mbox{\ref{Mission:Akzeptanz}    Akzeptanz    }}
			\\\hline
			\ref{Frage:Grundlagen}     & Grundlagen     &X&X&-&X&X&-&-\\
			\ref{Frage:Basis}          & Basis          &X&X&-&X&X&-&-\\
			\ref{Frage:Axiome}         & Axiome         &X&X&-&X&X&-&-\\
			\cdashline{1-9}[2pt/2pt]
			\ref{Frage:Beweis}         & Beweis         &X&-&X&X&-&-&-\\
			\ref{Frage:Konstruktion}   & Konstruktion   &X&-&-&X&-&-&-\\
			\ref{Frage:Vergleiche}     & Vergleiche     &X&-&-&-&X&-&-\\
			\cdashline{1-9}[2pt/2pt]
			\ref{Frage:Definitionen}   & Definitionen   &X&X&-&X&-&-&-\\
			\ref{Frage:Abhängigkeiten} & Abhängigkeiten &X&-&-&X&-&-&-\\
			\ref{Frage:Bekanntheit}    & Bekanntheit    &X&-&-&-&X&-&-\\
			\cdashline{1-9}[2pt/2pt]
			\ref{Frage:Darstellung}    & Darstellung    &-&X&-&X&-&-&-\\
			\ref{Frage:Forschung}      & Forschung      &X&-&-&-&X&-&-\\
			\hline
		\end{tabular}
		\caption{Fragen $\to$ Mission}
		\label{tbl:FragenMission}
	\end{threeparttable}\par~\par
	
	\begin{threeparttable}
		\begin{tabular}{@{}r@{ }l|*{15}{c}|}
			\multicolumn{2}{l|}{\diagbox{\textbf{Mission}}{\textbf{Ziele}}}
			& \rotatebox{90}{\mbox{\ref{Ziel:Daten}          Daten          }}
			& \rotatebox{90}{\mbox{\ref{Ziel:Form}           Form           }}
			& \rotatebox{90}{\mbox{\ref{Ziel:Eingaben}       Eingaben       }}
			& \rotatebox{90}{\mbox{\ref{Ziel:Prüfung}        Prüfung        }}
			& \rotatebox{90}{\mbox{\ref{Ziel:Ausgaben}       Ausgaben       }}
			& \rotatebox{90}{\mbox{\ref{Ziel:Auswertungen}   Auswertungen   }}
			& \rotatebox{90}{\mbox{\ref{Ziel:Anpassbarkeit}  Anpassbarkeit  }}
			& \rotatebox{90}{\mbox{\ref{Ziel:Individualität} Individualität }}
			& \rotatebox{90}{\mbox{\ref{Ziel:Internet}       Internet       }}
			& \rotatebox{90}{\mbox{\ref{Ziel:Kommunikation}  Kommunikation  }}
			& \rotatebox{90}{\mbox{\ref{Ziel:Zugriff}        Zugriff        }}
			& \rotatebox{90}{\mbox{\ref{Ziel:Unabhängigkeit} Unabhängigkeit }}
			& \rotatebox{90}{\mbox{\ref{Ziel:Rekursion}      Rekursion      }}
			& \rotatebox{90}{\mbox{\ref{Ziel:Bedienbarkeit}  Bedienbarkeit  }}
			& \rotatebox{90}{\mbox{\ref{Ziel:Lizenz}         Lizenz         }}
			\\\hline
			\ref{Mission:Daten}        & Daten        &X&X&X&-&-&-&-&-&-&-&-&-&-&-&-\\
			\ref{Mission:Definitionen} & Definitionen &X&-&X&-&-&-&-&-&-&-&-&-&-&-&-\\
			\ref{Mission:Prüfung}      & Prüfung      &-&-&-&X&-&-&-&-&-&-&-&-&-&-&-\\
			\cdashline{1-17}[2pt/2pt]
			\ref{Mission:Ausgaben}     & Ausgaben     &-&-&-&-&X&-&-&-&-&-&-&-&-&-&-\\
			\ref{Mission:Auswertungen} & Auswertungen &-&-&-&-&-&X&-&-&-&-&-&-&-&-&-\\
			\ref{Mission:Lizenz}       & Lizenz       &-&-&-&-&-&-&-&-&-&-&-&-&-&-&X\\
			\cdashline{1-17}[2pt/2pt]
			\ref{Mission:Akzeptanz}    & Akzeptanz    &X&X&X&X&X&X&X&X&X&X&X&X&X&X&X\\
			\hline
		\end{tabular}
		\caption{Mission $\to$ Ziele (Anforderungen)}
		\label{tbl:MissionZiele}
	\end{threeparttable}\par~\par
	
	Die nächste Tabelle ist eine Kombination aus den Tabellen~\vref{tbl:FragenMission} und~\vref{tbl:MissionZiele}. Die Fragen \emph{Akzeptanz} und \emph{Lizenz} kommen aus Abschnitt~ref{sec:Mission} \emph{Mission} dazu. Mit einem kleinen x werden Spalten markiert, deren Punkte für die Erfüllung der Anforderungen in den Zeilen nicht nötig, aber von Interesse sind.
	
	\begin{threeparttable}
		\begin{tabular}{@{}r@{ }l|*{15}{c}|}
			\multicolumn{2}{l|}{\diagbox{\textbf{Fragen}}{\textbf{Ziele}}}
			& \rotatebox{90}{\mbox{\ref{Ziel:Daten}          Daten          }}
			& \rotatebox{90}{\mbox{\ref{Ziel:Form}           Form           }}
			& \rotatebox{90}{\mbox{\ref{Ziel:Eingaben}       Eingaben       }}
			& \rotatebox{90}{\mbox{\ref{Ziel:Prüfung}        Prüfung        }}
			& \rotatebox{90}{\mbox{\ref{Ziel:Ausgaben}       Ausgaben       }}
			& \rotatebox{90}{\mbox{\ref{Ziel:Auswertungen}   Auswertungen   }}
			& \rotatebox{90}{\mbox{\ref{Ziel:Anpassbarkeit}  Anpassbarkeit  }}
			& \rotatebox{90}{\mbox{\ref{Ziel:Individualität} Individualität }}
			& \rotatebox{90}{\mbox{\ref{Ziel:Internet}       Internet       }}
			& \rotatebox{90}{\mbox{\ref{Ziel:Kommunikation}  Kommunikation  }}
			& \rotatebox{90}{\mbox{\ref{Ziel:Zugriff}        Zugriff        }}
			& \rotatebox{90}{\mbox{\ref{Ziel:Unabhängigkeit} Unabhängigkeit }}
			& \rotatebox{90}{\mbox{\ref{Ziel:Rekursion}      Rekursion      }}
			& \rotatebox{90}{\mbox{\ref{Ziel:Bedienbarkeit}  Bedienbarkeit  }}
			& \rotatebox{90}{\mbox{\ref{Ziel:Lizenz}         Lizenz         }}
			\\\hline
			\ref{Frage:Grundlagen}     & Grundlagen     &X&X&X&-&X&X&x&-&-&-&-&-&-&-&-\\
			\ref{Frage:Basis}          & Basis          &X&X&X&-&X&X&x&x&-&-&-&-&-&-&-\\
			\ref{Frage:Axiome}         & Axiome         &X&X&X&-&X&X&x&-&-&-&-&-&-&-&-\\
			\cdashline{1-17}[2pt/2pt]
			\ref{Frage:Beweis}         & Beweis         &X&X&X&X&X&-&-&x&-&-&-&-&-&-&-\\
			\ref{Frage:Konstruktion}   & Konstruktion   &X&X&X&-&X&-&-&x&-&-&-&-&-&-&-\\
			\ref{Frage:Vergleiche}     & Vergleiche     &X&X&X&-&-&X&-&x&-&-&-&-&-&-&-\\
			\cdashline{1-17}[2pt/2pt]
			\ref{Frage:Definitionen}   & Definitionen   &X&X&X&-&X&-&x&-&-&-&-&-&-&-&-\\
			\ref{Frage:Abhängigkeiten} & Abhängigkeiten &X&X&X&-&X&-&x&-&-&-&-&-&-&-&-\\
			\ref{Frage:Bekanntheit}    & Bekanntheit    &X&X&X&-&-&X&x&-&-&-&-&-&-&-&-\\
			\cdashline{1-17}[2pt/2pt]
			\ref{Frage:Darstellung}    & Darstellung    &X&-&X&-&X&-&x&-&-&-&-&-&-&-&-\\
			\ref{Frage:Forschung}      & Forschung      &X&X&X&-&-&X&x&-&-&-&-&-&-&-&-\\
			\cdashline{1-17}[2pt/2pt]
			~                          & Lizenz         &-&-&-&-&-&-&-&-&-&-&-&-&-&-&X\\
			~                          & Akzeptanz      &X&X&X&X&X&X&X&X&X&X&X&X&X&X&X\\
			\hline
		\end{tabular}
		\caption{Fragen $\to$ Ziele (Anforderungen)}
		\label{tbl:FragenZiele}
	\end{threeparttable}\par~\par


	%===============================================================================
	
	\chapter{Design}
	\thispagestyle{scrheadings}
	
	Diese Projekt soll Open Source sein. Daher gilt für die Dokumente die \emph{GNU Free Documentation License} und für die Software die \emph{GNU Affero General Public License}. Die \emph{GNU General Public License} reicht für die Software nicht, da das Programm auch als Server betrieben werden kann und soll.
	
	Wenn die Lizenzen nicht mitgeliefert wurden, können sie unter \url{http://www.gnu.org/licenses/} gefunden werden.
	
	\section{Anforderungen}
	\begin{enumerate}
		\item \label{Anforderung:key} \emph{key}: ???
	\end{enumerate}
	
	\section{Datenstruktur}
	\begin{enumerate}
		\item \label{Datenstruktur:key} \emph{key}: ???
	\end{enumerate}
	
	\section{Bausteine}
	\begin{enumerate}
		\item \label{Baustein:key} \emph{key}: ???
	\end{enumerate}
	
	\section{Werkzeuge}
	Da dies ein Open Source Projekt sein soll, müssen alle Werkzeuge, die zum Ablauf der Software erforderlich sind, ebenfalls Open Source sein.
	
	Angedachte Werkzeuge:
	\begin{itemize}
		\item \label{Werkzeug:VSC} Entwicklungsumgebung: \emph{Visual Studio Community 2015} (\textbf{VS})
		\item \label{Werkzeug:VSCDB} Datenbank: Integriert in VS
		\item \label{Werkzeug:RapidXml} Ein- und Ausgabe in XML: \emph{RapidXml}
		\item \label{Werkzeug:MiKTeX} Lesbare Ausgabe in \LaTeX: \emph{MiK\TeX}
		\item \label{Werkzeug:TeXstudio} \LaTeX Editor: \emph{\TeX studio}
		\item \label{Werkzeug:Subversion} Versionsverwaltungssystem: \emph{Subversion} (\textbf{SVN})
		\item \label{Werkzeug:VisualSVN} Integration von SVN in VS: \emph{VisualSVN}
		\item \label{Werkzeug:TortoiseSVN} Shell-Integration von SVN: \emph{TortoiseSVN}
	\end{itemize}
	
	
	%===============================================================================
	\appendix
	%===============================================================================
	
	\chapter{Anhang}
	\thispagestyle{scrheadings}
	
	\section{Offene Aufgaben}
	\begin{enumerate}
		\item Fragen und Ziele auf Homepage auflisten.
		\item Homepage und Lizenzen herunterladen.
		\item Verwendung von Github begründen.
		\item Mails vom 12. und 28.12.2016 von Michael verarbeiten
		\item Formale Syntax definieren
		\item Datenstruktur definieren
		\item Formale Prüfung definieren
		\item Axiome für das System bestimmen
		\item Visual Studio mit Subversion und Github verbinden
		\item Eingabeprogramm erstellen (liest XML)
		\item Prüfprogramm erstellen
		\item Ausgabeprogramm erstellen (schreibt XML)
		\item Formelausgabe erstellen (erzeugt \LaTeX aus XML)
		\item Axiome sammeln und eingeben
		\item Sätze sammeln und eingeben
		\item Beweise sammeln und eingeben
		\item Fachbegriffe und Symbole sammeln und eingeben
		\item Diszipline sammeln und eingeben
		\item Ausgabeschemata sammeln und eingeben
	\end{enumerate}
		
	%===============================================================================


	\listoftables \addcontentsline{toc}{chapter}{Tabellenverzeichnis}
	\listoffigures\addcontentsline{toc}{chapter}{Abbildungsverzeichnis}
	\thispagestyle{scrheadings}
	
\end{document}